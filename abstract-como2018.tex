\documentclass{article}
\begin{document}

\subsection*{How topos theory can help commutative algebra}

\noindent
{\scshape Ingo Blechschmidt}\index{Blechschmidt, Ingo}\\
{\scshape University of Augsburg / Max Planck Institute for
Mathematics in the Sciences, Leipzig}
\medskip

Topos theory and commutative algebra are closely linked: In view of the mantra
``toposes are rings'', commutative algebra has been informing topos theory, and
by employing the internal language of toposes, topos theory has been used to
transfer results from commutative algebra to subjects such as algebraic
geometry and differential geometry. In this talk, we explore a third link: How
topos theory can help commutative algebra and neighboring disciplines.

One such relation is given by new reduction techniques. For instance, there is
a way how we can assume, without loss of generality, that any reduced ring is a
field. This technique allows to give a short and simple proof of Grothendieck's
generic freeness lemma, a basic theorem used in the setup of the theory of
moduli spaces, which substantially improves on the previous somewhat
convoluted proofs.

These topos-theoretic reduction techniques cannot generally be mimicked by traditional
commutative algebra, and in the special cases where they can,
they improve on the traditional methods by yielding fully constructive proofs.
The precise sense in which all this is true will be carefully explained in the
talk.

A further relation is given by synthetic approaches to algebraic geometry,
allowing to treat schemes with all their complex algebro-geometrical structure
as plain sets and morphisms between schemes as maps between these sets.
Fundamental to this account is the notion of ``synthetic quasicoherence'',
which doesn't have a counterpart in synthetic differential geometry and which
endows the relevant internal universes with a distinctive algebraic flavor.

Somewhat surprisingly, the work on synthetic algebraic geometry is related to
an age-old question in the study of classifying toposes. We hope to report on
recent results by Matthias Hutzler in this regard, and close with an invitation
to the many open problems of the field. The talk will begin with an
introduction to the internal language of toposes, so as to be accessible to
audience members who are not familiar with it and to provide value
outside of commutative algebra.

\end{document}
