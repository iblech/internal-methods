\documentclass[a4paper,english,11pt]{scrartcl}

\usepackage[utf8]{inputenc}

\usepackage[english]{babel}

\usepackage[protrusion=true,expansion=true]{microtype}

\usepackage[T1]{fontenc}
\usepackage{libertine}

\begin{document}

% Exploring mathematical objects from custom-tailored mathematical universes

Mathematicians study mathematical objects from the point of view of an ambient
mathematical universe. Feelings vary on the ontological status of the ambient
universe and on foundational details; but in any case it remains fixed for all
objects of study, providing a common home for relations with other objects and
setting the allowed laws of reasoning.

It is, however, also possible to study an object from the point of view of
alternate mathematical universes -- toposes -- which can be custom-tailored to
the object in question. These universes typically paint a different picture of
the studied object, which may from the alternate point of view lack properties
which it had when embedded in the usual universe, and which conversely may enjoy
unique new properties.

Studying objects from the point of view of nonstandard toposes is not done out
of abstract curiosity. It's rather done because there are certain precise
relationships between the properties from the point of view of nonstandard
toposes and properties from the point of view of the usual ambient universe.
In this way, we can gain insights we otherwise couldn't.

For instance, this approach has yielded a memorable and constructive proof of
Grothendieck's generic freeness lemma, a certain statement in commutative
algebra about modules and their localizations, for which previously only
somewhat convoluted proofs were known. The technique originated in algebraic
geometry and has yielded applications there, in commutative algebra, and in
constructive mathematics.

A basic fact of this approach is that relevant alternate universes typically
don't validate full classical logic, but only intuitionistic logic. The law of
excluded middle and the law of double negation elimination are therefore not
generally applicable in these universes; this phenomenon is irrespective of
philosophical convictions about the metalevel.

The proposed talk will explain this approach in an expository manner, without
requiring prerequisites in topos theory. The talk will center around an
easy-to-grasp example, namely that provided by the ``Zariski topos of a ring''
This topos is an alternate universe, custom-tailored to a given ring, such that
from its point of view the ring is always a field -- even if, from the usual
point of view, it is not. This observation allows to apply any intuitionistic
theorem about fields in the alternate universe. The results obtained in this
way translate to results about the ring as judged by the usual ambient
universe. The result about Grothendieck's generic freeness lemma alluded to
above is obtained in this fashion: The freeness lemma is trivial to prove
for fields and therefore holds in the Zariski topos; translated, it yields the
usual freeness lemma for rings.

The talk doesn't explicitly feature content of philosophy of mathematics. It's
rather an invitation to philosophers of mathematics to explore this technique
from a philosophical point of view and to take nonstandard toposes seriously. I
believe that this will be very worthwile, since this area has as of yet only
been explored mathematically, not philosophically.

\end{document}
