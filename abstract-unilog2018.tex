\documentclass{article}
\begin{document}

\subsection*{Exploring the internal language of toposes}
\addcontentsline{toc}{subsection}{Exploring the internal language of toposes, by Ingo
Blechschmidt}

\noindent
{\scshape Ingo Blechschmidt}\index{Blechschmidt, Ingo}\\
{\scshape Institute for Mathematics,
University of Augsburg, Augsburg, Germany}\\
{\scshape ingo.blechschmidt@math.uni-augsburg.de}\\

Since the work of the early pioneers in the 1970s, it's known that any topos
supports an \emph{internal language}, which allows to speak and reason about
its objects and morphisms in a naive element-based language: From the internal
perspective, objects of the topos look like sets, morphisms look like maps
between sets, epimorphisms look like surjective maps, group objects look like
plain groups and so on; and any theorem which has an intuitionistic proof also
holds in the internal universe of a topos.

With recent discoveries of new applications of the internal language in
algebra, geometry, homotopy theory, mathematical physics and measure theory,
the study of the internal language of toposes is currently experiencing a
resurgence. Our goal is give an introduction to this topic and illustrate the
usefulness of the internal language with two concrete examples.

Firstly, the internal language of the ``little Zariski topos'' allows us to
assume without loss of generality that any reduced ring is Noetherian and in
fact a field, as long as we restrict to intuitionistic reasoning. This
technique yields for instance a simple one-paragraph proof of
Grothendieck's generic freeness lemma, because it is trivial for fields. We
thereby improve on the substantially longer and somewhat convoluted previously known proofs.

Secondly, the internal language of the ``big Zariski topos'' can be used to
develop a synthetic account of algebraic geometry, in which schemes appear as
plain sets and morphisms of schemes appear as maps between these sets.
Fundamental to this account is the notion of ``synthetic quasicoherence'',
which doesn't have a counterpart in synthetic differential geometry and which
endows the internal universe with a distinctive algebraic flavor.

Somewhat surprisingly, the work on synthetic algebraic geometry is related to
an age-old question in the study of classifying toposes. The talk closes with
an invitation to the many open problems of the field.

\end{document}
