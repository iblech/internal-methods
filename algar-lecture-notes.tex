\documentclass[10pt,reqno,a4paper,openany]{amsbook}
\usepackage{etex}
\usepackage[utf8]{inputenc}
\usepackage[english]{babel}
\usepackage{etoolbox,chngcntr}
\usepackage{amsmath,amsthm,amssymb,array,stmaryrd,color,graphicx,mathtools,multirow,setspace}
\usepackage{soul}\setul{0.3ex}{}
\usepackage{bussproofs}
\usepackage{manfnt}
\usepackage{xspace}
\usepackage{longtable}
\usepackage{booktabs}
\usepackage[protrusion=true,expansion=true]{microtype}
\usepackage[bookmarksdepth=2,pdfencoding=auto]{hyperref}

\graphicspath{{images/}}
\usepackage{libertine}

% Hack to load extpfeil from https://tex.stackexchange.com/a/297109/32372
\expandafter\def\csname opt@stmaryrd.sty\endcsname
{only,shortleftarrow,shortrightarrow}
\usepackage{extpfeil}
\newextarrow{\xbigtoto}{{20}{20}{20}{20}}
   {\bigRelbar\bigRelbar{\bigtwoarrowsleft\rightarrow\rightarrow}}

\usepackage[all]{xy}
\usepackage{tikz}
\usetikzlibrary{calc,shapes.callouts,shapes.arrows,matrix,patterns}
\newcommand{\hcancel}[5]{%
    \tikz[baseline=(tocancel.base)]{
        \node[inner sep=0pt,outer sep=0pt] (tocancel) {#1};
        \draw[red, line width=0.3mm] ($(tocancel.south west)+(#2,#3)$) -- ($(tocancel.north east)+(#4,#5)$);
    }%
}

\usepackage[natbib=true,style=numeric,maxnames=10]{biblatex}
\usepackage[babel]{csquotes}
\bibliography{bibliography}

\theoremstyle{definition}
\newtheorem{defn}{Definition}[chapter]
\newtheorem{ex}[defn]{Example}

\theoremstyle{plain}
\newtheorem{prop}[defn]{Proposition}
\newtheorem{cor}[defn]{Corollary}
\newtheorem{lemma}[defn]{Lemma}
\newtheorem{thm}[defn]{Theorem}
\newtheorem{scholium}[defn]{Scholium}

\theoremstyle{remark}
\newtheorem{rem}[defn]{Remark}
\newtheorem{question}[defn]{Question}
\newtheorem{speculation}[defn]{Speculation}
\newtheorem{caveat}[defn]{Caveat}
\newtheorem{conjecture}[defn]{Conjecture}

\newcommand{\ZZ}{\mathbb{Z}}
\newcommand{\FF}{\mathbb{F}}
\renewcommand{\AA}{\mathbb{A}}
\newcommand{\A}{\mathcal{A}}
\newcommand{\B}{\mathcal{B}}
\newcommand{\C}{\mathcal{C}}
\newcommand{\D}{\mathcal{D}}
\newcommand{\E}{\mathcal{E}}
\newcommand{\F}{\mathcal{F}}
\newcommand{\G}{\mathcal{G}}
\let\acuteH\H
\newcommand{\konig}{K\acuteH onig}
\renewcommand{\H}{\mathcal{H}}
\renewcommand{\O}{\mathcal{O}}
\newcommand{\K}{\mathcal{K}}
\newcommand{\N}{\mathcal{N}}
\newcommand{\M}{\mathcal{M}}
\renewcommand{\L}{\mathcal{L}}
\renewcommand{\P}{\mathcal{P}}
\newcommand{\R}{\mathcal{R}}
\newcommand{\T}{\mathcal{T}}
\newcommand{\I}{\mathcal{I}}
\newcommand{\J}{\mathcal{J}}
\renewcommand{\S}{\mathcal{S}}
\newcommand{\U}{\mathcal{U}}
\newcommand{\V}{\mathcal{V}}
\newcommand{\NN}{\mathbb{N}}
\newcommand{\PP}{\mathbb{P}}
\newcommand{\RR}{\mathbb{R}}
\newcommand{\CC}{\mathbb{C}}
\newcommand{\QQ}{\mathbb{Q}}
\newcommand{\GG}{\mathbb{G}}
\newcommand{\TT}{\mathbb{T}}
\newcommand{\aaa}{\mathfrak{a}}
\newcommand{\bbb}{\mathfrak{b}}
\newcommand{\ccc}{\mathfrak{c}}
\newcommand{\ppp}{\mathfrak{p}}
\newcommand{\qqq}{\mathfrak{q}}
\newcommand{\mmm}{\mathfrak{m}}
\newcommand{\nnn}{\mathfrak{n}}
\newcommand{\?}{\,{:}\,}
\renewcommand{\_}{\mathpunct{.}\,}

\newenvironment{indentblock}{%
  \list{}{\leftmargin\leftmargin}%
  \item\relax
}{%
  \endlist
}

\newcommand{\defeq}{\vcentcolon=}
\newcommand{\defequiv}{\vcentcolon\equiv}
\newcommand{\seq}[1]{\mathrel{\vdash\!\!\!_{#1}}}

\definecolor{gray}{rgb}{0.7,0.7,0.7}

\title{The curious world of constructive mathematics}
\author{Ingo Blechschmidt}
%\email{iblech@speicherleck.de}

\makeatletter
\counterwithout{section}{chapter}
\counterwithout{footnote}{chapter}
\counterwithout{table}{chapter}
\counterwithout{figure}{chapter}
\patchcmd{\@thm}{\let\thm@indent\indent}{\let\thm@indent\noindent}{}{}
\patchcmd{\@thm}{\thm@headfont{\scshape}}{\thm@headfont{\bfseries}}{}{}
\patchcmd{\@makechapterhead}{\chaptername}{Lecture}{}{}
\patchcmd{\@chapter}{\chaptername}{Part}{}{}
\patchcmd{\@schapter}{\chaptername}{Part}{}{}
\addto\captionsenglish{\renewcommand\chaptername{Lecture}}
\def\l@section{\@tocline{1}{0pt}{1pc}{}{}} % \bfseries}}
\def\l@chapter{\@tocline{-1}{12pt}{0pt}{}{\bfseries}}
\renewcommand\thechapter{\Roman{chapter}}
\newcommand{\nocontentsline}[3]{}
\newcommand{\tocless}[1]{\let\addcontentsline=\nocontentsline}
\normalparindent=12pt
\parindent=\normalparindent
\renewenvironment{proof}[1][\proofname]{\par
  \pushQED{\qed}%
  \normalfont \topsep6\p@\@plus6\p@\relax
  \trivlist
  \item[\hskip\labelsep
        \itshape
    #1\@addpunct{.}]\ignorespaces
}{%
  \popQED\endtrivlist\@endpefalse
}
\let\@afterindenttrue\@afterindentfalse
\def\subsection{\@startsection{subsection}{2}%
  {0pt}{.5\linespacing\@plus.7\linespacing}{-.5em}%
  {\normalfont\bfseries}}
\makeatother

\newenvironment{intro}{\begin{quote}}{\end{quote}\bigskip}

\newtheorem{exercise}[defn]{Exercise}
\renewcommand{\theenumi}{\alph{enumi}}

\begin{document}

\begin{abstract}
Constructive mathematics is a flavor of mathematics in which we use the
axiom of choice and the technique of proof by contradiction only in
certain special cases. The square root of two is constructively still
irrational, but there might be vector spaces without a basis.

As a result, proofs are more informative (for instance regarding
bounds), finer distinctions can be made (for instance between positive
existence and mere impossibility of non-existence) and results apply
more generally: Every constructive result also has a geometric
interpretation, where it applies to continuous families, and an
algorithmic interpretation, yielding computational witnesses such as
procedures for computing the objects whose existence has been shown.

Relinquishing the axiom of choice and the law of excluded middle also
allows us to explore axioms and notions which are incompatible with
these classical laws, such as mathematical settings in which all
functions are continuous or in which the intuitive idea of a ``generic
ring'' can be put on a firm basis.
\end{abstract}

{
\renewcommand{\newpage}{\ }
\renewcommand{\vfill}{\ }
\renewcommand{\vfil}{\ }
\maketitle
}

\setcounter{tocdepth}{1}
{
\renewcommand{\newpage}{\ }
\tableofcontents
}


\chapter{A first glimpse of constructive mathematics}

\begin{intro}
\it
This lecture provides a first glimpse of constructive mathematics with a
focus on applications of constructive mathematics and on providing
intuition for quickly discerning which techniques and results hold
constructively.
\end{intro}

\begin{prop}There are irrational numbers~$x$ and~$y$ such that~$x^y$ is
rational.
\end{prop}
\begin{proof}[First proof] The number~$\sqrt{2}^{\sqrt{2}}$ is rational or
irrational. In the first case, set~$x \defeq \sqrt{2}$, $y \defeq \sqrt{2}$.
In the second case, set~$x \defeq \sqrt{2}^{\sqrt{2}}$, $y \defeq \sqrt{2}$.
\end{proof}
\begin{proof}[Second proof] Set~$x \defeq \sqrt{2}$ and~$y \defeq \log_{\sqrt{2}} 3$.
Then~$x^y = 3$ is rational. The verification that~$y$ is irrational is even
easier than that of~$\sqrt{2}$.\footnote{Let~$y = a/b$ with~$a, b \in \ZZ$
and~$b \neq 0$. Since~$y > 0$, we may assume~$a, b \in \NN$. Then~$3 =
(\sqrt{2})^{a/b}$, hence~$3^{2b} = 2^a$. This is in contradiction to the
uniqueness of the prime factor decomposition, since the factor~$3$ occurs on
the left but not on the right.}
\end{proof}

The first proof is \emph{unconstructive}: It does not actually give us an
example for a pair~$(x,y)$ as desired. In contrast, the second proof is
constructive -- the existential claim is verified by an explicit construction
of a suitable example.

Of the many axioms and inference rules of classical logic, exactly one is
responsible for enabling unconstructive arguments, namely the \emph{principle
of excluded middle}:
\[ \varphi \ \vee\ \neg\varphi. \]
The first proof above used this principle in its very first step. In
constructive mathematics, we abstain from this principle; we build
constructive mathematics on \emph{intuitionistic logic}, which contains neither
this principle nor the (equivalent) \emph{principle of double negation
elimination} stating~$\neg\neg\varphi \Rightarrow \varphi$, and insofar as we
layer a set theory on top of our logical foundation, we abstain from the axiom
of choice (which in presence of other common set-theoretical axioms implies the
principle of excluded middle, see Exercise~\ref{ex:diaconescu}).

\marginpar{\dbend}
In constructive mathematics, we do \emph{not} claim that the principle of
excluded middle is false. Indeed, intuitionistic logic is downwardly compatible
with classical logic (every intuitionistic proof is a fortiori also a classical
proof), and some special instances of the principle of excluded middle are
intuitionistically verifiable (an example is given in
Proposition~\ref{prop:discreteness}). Instead, in constructive mathematics we
merely do not use the principle of excluded middle.

\begin{exercise}[Basics on negation]
Recalling that negation is defined as implying absurdity,~$\neg\varphi
\defequiv (\varphi \Rightarrow \bot)$, verify intuitionistically without
recourse to truth tables:
\begin{enumerate}
\item $\varphi \Rightarrow \neg\neg\varphi$
\item $\neg\neg\neg\varphi \Leftrightarrow \neg\varphi$
\item $\neg\neg(\varphi \vee \neg\varphi)$
\item $(\forall\psi\_ (\psi \vee \neg\psi))
\Longleftrightarrow (\forall\psi\_ ((\neg\neg\psi) \Rightarrow
\psi))$

{\noindent\scriptsize\emph{Hint.}
Don't try to verify that double negation elimination for a specific
statement~$\psi$ implies the principle of excluded middle for that same
statement~$\psi$ -- this cannot be shown. There are subtleties regarding the
quantification over $\psi$ (this is not expressible in pure first-order logic),
however the exercise is still instructive if we gloss over this issue.\par}
\end{enumerate}
\end{exercise}

\begin{exercise}[Constructive status of classical tautologies]
Which of the following classical tautologies can reasonably be expected to
admit constructive proofs?
\begin{enumerate}
\item $\neg(\alpha \vee \beta) \ \Longrightarrow\ \neg\alpha \wedge \neg\beta$
\item $\neg(\alpha \wedge \beta) \ \Longrightarrow\ \neg\alpha \vee \neg\beta$
\item $(\alpha \vee \beta) \wedge \neg\alpha \ \Longrightarrow\ \beta$
\item $\forall M \in P(X)\_ (\exists x\?X\_ x \in M) \vee M = \emptyset$
(already interesting for~$X = \{\star\}$)
\item $\forall n\?\NN\_ (n = 0 \vee n \neq 0)$
\item $\forall x\?\RR\_ (x = 0 \vee x \neq 0)$
\item $\forall x\?\RR\_ (\neg(\exists y\?\RR\_ xy=1) \Rightarrow x = 0)$
\item $\forall z\?\overline{\QQ}\_ (z = 0 \vee z \neq 0)$
\item $\forall f \? \NN \to \{0,1\}\_ (\neg\neg\exists n \? \NN\_ f(n) = 0)
\Rightarrow (\exists n \? \NN\_ f(n) = 0)$
\item $\forall f \? \NN \to \{0,1\}\_ \exists n \? \NN\_ (f(n) = 1
\Leftrightarrow (\forall m \? \NN\_ f(m) = 1))$
\item $\forall f \? \NN \to \{0,1\}\_ (\exists n \? \NN\_ f(n) = 0) \vee
(\forall n \? \NN\_ f(n) = 1)$
\item $\forall f \? \NN_\infty \to \{0,1\}\_ (\exists n \? \NN_\infty\_ f(n) = 0) \vee
(\forall n \? \NN_\infty\_ f(n) = 1)$

{\noindent\scriptsize\emph{Remark.} The set~$\NN_\infty$ is the \emph{one-point
compactification} of~$\NN$. A sensible definition in constructive mathematics
is as the set of decreasing binary sequences~$(x_0,x_1,x_2,\ldots)$. The
naturals embed into~$\NN_\infty$ by mapping~$n$ to the sequence~$1^n 0^\omega =
(1,\ldots,1,0,\ldots,0)$, and an element not in the image of this embedding
is~$\infty \defeq 1^\omega = (1,1,\ldots)$. Assuming the law of excluded
middle (or already weaker principles), every element of~$\NN_\infty$ is of one
of these two forms. Martín Escardó has worked extensively on unexpected
instances of the principle of omniscience for searchable sets
like~$\NN_\infty$~\cite{escardo:omniscient1,escardo:omniscient2}.\par}
\end{enumerate}
\end{exercise}

\begin{exercise}[An epistemic riddle on transcendental numbers]
Verify, using a proof by contradiction, that at least one of the numbers~$e +
\pi$ and~$e - \pi$ is transcendental; and that at least one of the
numbers~$e + \pi$ and~$e \cdot \pi$ is transcendental.
\end{exercise}

\begin{exercise}[Diaconescu's theorem]
\label{ex:diaconescu}
The axiom of choice can be put as: ``Every surjective map has a
section.'' (A \emph{section}~$s$ to a surjective map~$f$ is a map in the other direction such that~$f \circ s =
\mathrm{id}$.) It is a theorem due to Diaconescu that
the axiom of choice implies the law of excluded middle. To this end, let~$\varphi$ be a
statement and consider the subsets
\begin{align*}
  U &= \{ x \in X \,|\, (x = 0) \vee \varphi \} \\
  V &= \{ x \in X \,|\, (x = 1) \vee \varphi \}
\end{align*}
of the discrete set~$X \defeq \{ 0,1 \}$.
\begin{enumerate}
\item Verify that~$U = V$ if and only if~$\varphi$.
\item Using that~$x = y \vee x \neq y$ for all elements~$x,y \in X$, show that
the existence of a section of the surjective map
\[ \begin{array}{@{}rcl@{}}
  X &\longrightarrow& \{U,V\} \\
  0 &\longmapsto& U \\
  1 &\longmapsto& V
\end{array} \]
implies~$\varphi \vee \neg\varphi$.
\end{enumerate}
\end{exercise}

\begin{exercise}[A stronger form of the irrationality of~$\sqrt{2}$]
\label{ex:sqrt2}
Mine the proof of Proposition~\ref{prop:sqrt2} to give an intuitionistic proof
that for every rational number~$x$, the distance~$|\sqrt{2}-x|$ is positive.

{\noindent\scriptsize\emph{Note.} Strictly speaking, this exercise presupposes
familiarity with an intuitionistic account of the basics of undergraduate real
analysis. Without it, one cannot really be expected to precisely think about
these matters. One such account (though assuming the axiom of dependent choice)
is~\cite{bridges}. However, this exercise is insightful even when carried out
slightly informally. Keep in mind that, to show that a real number is positive,
constructively it is not enough to merely verify that it cannot be zero or
negative. A safe way to verify that a real number~$a$ is positive is to exhibit
a rational number~$b$ such that~$a \geq b > 0$.\par}
\end{exercise}

\vfill
{\small
\begin{verbatim}
- What constructive mathematics is about: finer distinctions, algorithmic
  interpretation, geometric interpretation (including teaser on why we
  might care)
- Proof by and proof of a contradiction: powers of irrational numbers,
  discreteness of N, Q, Qbar, failure of discreteness of R
- Positive information from negative? de Morgan, minimums of sets of
  natural numbers
- Relation to classical mathematics
- Finite sets and the axiom of choice
- Axiomatic freedom: all functions continuous, all functions
  computable, generic prime ideal
\end{verbatim}
}


\chapter{On the constructive content of classical proofs}

\begin{intro}
Decades of experience in constructive mathematics show: \emph{Most results
in classical mathematics, even those whose proof rests on
non-constructive principles like the axiom of choice or the law of
excluded middle, have a hidden constructive core.} With a mix of
experience, seasoned tools and general metatheorems, this constructive
content can be extracted from classical proofs. In this way we obtain
constructive reformulations of classical results, especially
if they are of a sufficiently concrete nature.

For instance, while the existence of maximal ideals in arbitrary rings
is equivalent to the axiom of choice, every first-order consequence of
their existence for linear algebra over rings also holds constructively.

This lecture illustrates the latent constructive nature of classical
proofs with examples and presents two general metatheorems which
elucidate proof mining for constructive content.
\end{intro}


\section{The double-negation translation}

\section{Barr's theorem}

\section{Examples from quadratic form theory}


\chapter{Generic models and their applications}

\begin{intro}
Commutative algebra progressed when the intuitive but informal notion of
``the generic element of a given field~$k$'' was reified in the form of the
specific element~$X$ of the polyomial ring~$k[X]$. The caveat is, of course,
that the expanded ring~$k[X]$ is no longer a field.

A similar story unfolds one level up. \emph{Topos theory provides us with ``the
generic ring'', the ring we are implicitly picturing when someone utters
the phrase ``Let~$R$ be a ring''.} The generic ring has exactly those
properties (of the large class of ``geometric implications'') which all rings
have. To have the generic ring in our ontology,
we need to broaden our notion of existence---the generic ring is not a
ring in the usual sense of the word, but a ring object in an enlarged
topos, and still close enough to the familiar rings in that all
constructive theorems about rings apply to it.

Bizarrely, the generic ring has the property (not formalizable as a
geometric implication) that it is even a field. Hence, if we are to
prove a geometric implication for all rings, we can just as well assume
the field property.

The lecture introduces the notion of topos-theoretic generic models in
general, focussing on the generic ring, the generic prime ideal and
concrete applications.
\end{intro}

\vfill{\small%
\begin{verbatim}
- Existence of generic models
- Generic ring is a field
- Generic prime ideal
- Kripke–Joyal semantics
\end{verbatim}
}


\end{document}

omniscient1
https://www.cs.bham.ac.uk/~mhe/.talks/dagstuhl2011/omniscient.pdf

omniscient2
https://www.ioc.ee/~tarmo/tsem16/escardo2605-slides.pdf
