\documentclass[10pt,reqno,a4paper,openany]{amsbook}
\usepackage{etex}
\usepackage[utf8]{inputenc}
\usepackage[english]{babel}
\usepackage{etoolbox,chngcntr}
\usepackage{amsmath,amsthm,amssymb,array,stmaryrd,color,graphicx,mathtools,multirow,setspace}
\usepackage{soul}\setul{0.3ex}{}
\usepackage{bussproofs}
\usepackage{xspace}
\usepackage{longtable}
\usepackage{booktabs}
\usepackage[protrusion=true,expansion=true]{microtype}
\usepackage[bookmarksdepth=2,pdfencoding=auto]{hyperref}

\usepackage{libertine}

% Hack to load extpfeil from https://tex.stackexchange.com/a/297109/32372
\expandafter\def\csname opt@stmaryrd.sty\endcsname
{only,shortleftarrow,shortrightarrow}
\usepackage{extpfeil}
\newextarrow{\xbigtoto}{{20}{20}{20}{20}}
   {\bigRelbar\bigRelbar{\bigtwoarrowsleft\rightarrow\rightarrow}}

\usepackage[all]{xy}
\usepackage{tikz}
\usetikzlibrary{calc,shapes.callouts,shapes.arrows,matrix,patterns}
\newcommand{\hcancel}[5]{%
    \tikz[baseline=(tocancel.base)]{
        \node[inner sep=0pt,outer sep=0pt] (tocancel) {#1};
        \draw[red, line width=0.3mm] ($(tocancel.south west)+(#2,#3)$) -- ($(tocancel.north east)+(#4,#5)$);
    }%
}

\usepackage[natbib=true,style=numeric,maxnames=10]{biblatex}
\usepackage[babel]{csquotes}
\bibliography{bibliography}

\theoremstyle{definition}
\newtheorem{defn}{Definition}[section]
\newtheorem{ex}[defn]{Example}

\theoremstyle{plain}
\newtheorem{prop}[defn]{Proposition}
\newtheorem{cor}[defn]{Corollary}
\newtheorem{lemma}[defn]{Lemma}
\newtheorem{thm}[defn]{Theorem}
\newtheorem{scholium}[defn]{Scholium}

\theoremstyle{remark}
\newtheorem{rem}[defn]{Remark}
\newtheorem{question}[defn]{Question}
\newtheorem{speculation}[defn]{Speculation}
\newtheorem{caveat}[defn]{Caveat}
\newtheorem{conjecture}[defn]{Conjecture}

\newcommand{\ZZ}{\mathbb{Z}}
\newcommand{\FF}{\mathbb{F}}
\renewcommand{\AA}{\mathbb{A}}
\newcommand{\A}{\mathcal{A}}
\newcommand{\B}{\mathcal{B}}
\newcommand{\C}{\mathcal{C}}
\newcommand{\D}{\mathcal{D}}
\newcommand{\E}{\mathcal{E}}
\newcommand{\F}{\mathcal{F}}
\newcommand{\G}{\mathcal{G}}
\let\acuteH\H
\newcommand{\konig}{K\acuteH onig}
\renewcommand{\H}{\mathcal{H}}
\renewcommand{\O}{\mathcal{O}}
\newcommand{\K}{\mathcal{K}}
\newcommand{\N}{\mathcal{N}}
\newcommand{\M}{\mathcal{M}}
\renewcommand{\L}{\mathcal{L}}
\renewcommand{\P}{\mathcal{P}}
\newcommand{\R}{\mathcal{R}}
\newcommand{\T}{\mathcal{T}}
\newcommand{\I}{\mathcal{I}}
\newcommand{\J}{\mathcal{J}}
\renewcommand{\S}{\mathcal{S}}
\newcommand{\U}{\mathcal{U}}
\newcommand{\V}{\mathcal{V}}
\newcommand{\NN}{\mathbb{N}}
\newcommand{\PP}{\mathbb{P}}
\newcommand{\RR}{\mathbb{R}}
\newcommand{\CC}{\mathbb{C}}
\newcommand{\QQ}{\mathbb{Q}}
\newcommand{\GG}{\mathbb{G}}
\newcommand{\TT}{\mathbb{T}}
\newcommand{\aaa}{\mathfrak{a}}
\newcommand{\bbb}{\mathfrak{b}}
\newcommand{\ccc}{\mathfrak{c}}
\newcommand{\ppp}{\mathfrak{p}}
\newcommand{\qqq}{\mathfrak{q}}
\newcommand{\mmm}{\mathfrak{m}}
\newcommand{\nnn}{\mathfrak{n}}
\newcommand{\?}{\,{:}\,}

\newenvironment{indentblock}{%
  \list{}{\leftmargin\leftmargin}%
  \item\relax
}{%
  \endlist
}

\newcommand{\defeq}{\vcentcolon=}
\newcommand{\defequiv}{\vcentcolon\equiv}
\newcommand{\seq}[1]{\mathrel{\vdash\!\!\!_{#1}}}

\definecolor{gray}{rgb}{0.7,0.7,0.7}

\title{The curious world of constructive mathematics}
\author{Ingo Blechschmidt}
%\email{iblech@speicherleck.de}

\makeatletter
\counterwithout{section}{chapter}
\counterwithout{footnote}{chapter}
\counterwithout{table}{chapter}
\counterwithout{figure}{chapter}
\patchcmd{\@thm}{\let\thm@indent\indent}{\let\thm@indent\noindent}{}{}
\patchcmd{\@thm}{\thm@headfont{\scshape}}{\thm@headfont{\bfseries}}{}{}
\patchcmd{\@makechapterhead}{\chaptername}{Lecture}{}{}
\patchcmd{\@chapter}{\chaptername}{Part}{}{}
\patchcmd{\@schapter}{\chaptername}{Part}{}{}
\addto\captionsenglish{\renewcommand\chaptername{Lecture}}
\def\l@section{\@tocline{1}{0pt}{1pc}{}{}} % \bfseries}}
\def\l@chapter{\@tocline{-1}{12pt}{0pt}{}{\bfseries}}
\renewcommand\thechapter{\Roman{chapter}}
\newcommand{\nocontentsline}[3]{}
\newcommand{\tocless}[1]{\let\addcontentsline=\nocontentsline}
\normalparindent=12pt
\parindent=\normalparindent
\renewenvironment{proof}[1][\proofname]{\par
  \pushQED{\qed}%
  \normalfont \topsep6\p@\@plus6\p@\relax
  \trivlist
  \item[\hskip\labelsep
        \itshape
    #1\@addpunct{.}]\ignorespaces
}{%
  \popQED\endtrivlist\@endpefalse
}
\let\@afterindenttrue\@afterindentfalse
\def\subsection{\@startsection{subsection}{2}%
  {0pt}{.5\linespacing\@plus.7\linespacing}{-.5em}%
  {\normalfont\bfseries}}
\makeatother

\newenvironment{intro}{\begin{quote}}{\end{quote}}

\begin{document}

\begin{abstract}
Constructive mathematics is a flavor of mathematics in which we use the
axiom of choice and the technique of proof by contradiction only in
certain special cases. The square root of two is constructively still
irrational, but there might be vector spaces without a basis.

As a result, proofs are more informative (for instance regarding
bounds), finer distinctions can be made (for instance between positive
existence and mere impossibility of non-existence) and results apply
more generally: Every constructive result also has a geometric
interpretation, where it applies to continuous families, and an
algorithmic interpretation, yielding computational witnesses such as
procedures for computing the objects whose existence has been shown.

Relinquishing the axiom of choice and the law of excluded middle also
allows us to explore axioms and notions which are incompatible with
these classical laws, such as mathematical settings in which all
functions are continuous or in which the intuitive idea of a ``generic
ring'' can be put on a firm basis.
\end{abstract}

{
\renewcommand{\newpage}{\ }
\renewcommand{\vfill}{\ }
\renewcommand{\vfil}{\ }
\maketitle
}

\setcounter{tocdepth}{1}
{
\renewcommand{\newpage}{\ }
\tableofcontents
}


\chapter{A first glimpse of constructive mathematics}

\begin{intro}
\it
This lecture provides a first glimpse of constructive mathematics with a
focus on applications of constructive mathematics and on providing
intuition for quickly discerning which techniques and results hold
constructively.
\end{intro}

\vfill
{\small
\begin{verbatim}
- What constructive mathematics is about: finer distinctions, algorithmic
  interpretation, geometric interpretation (including teaser on why we
  might care)
- Proof by and proof of a contradiction: powers of irrational numbers,
  discreteness of N, Q, Qbar, failure of discreteness of R
- Positive information from negative? de Morgan, minimums of sets of
  natural numbers
- Relation to classical mathematics
- Finite sets and the axiom of choice
- Axiomatic freedom: all functions continuous, all functions
  computable, generic prime ideal
\end{verbatim}
}


\chapter{On the constructive content of classical proofs}

\begin{intro}
Decades of experience in constructive mathematics show: \emph{Most results
in classical mathematics, even those whose proof rests on
non-constructive principles like the axiom of choice or the law of
excluded middle, have a hidden constructive core.} With a mix of
experience, seasoned tools and general metatheorems, this constructive
content can be extracted from classical proofs. In this way we obtain
constructive reformulations of classical results, especially
if they are of a sufficiently concrete nature.

For instance, while the existence of maximal ideals in arbitrary rings
is equivalent to the axiom of choice, every first-order consequence of
their existence for linear algebra over rings also holds constructively.

This lecture illustrates the latent constructive nature of classical
proofs with examples and presents two general metatheorems which
elucidate proof mining for constructive content.
\end{intro}


\section{The double-negation translation}

\section{Barr's theorem}

\section{Examples from quadratic form theory}


\chapter{Generic models and their applications}

\begin{intro}
Commutative algebra progressed when the intuitive but informal notion of
``the generic element of a given field~$k$'' was reified in the form of the
specific element~$X$ of the polyomial ring~$k[X]$. The caveat is, of course,
that the expanded ring~$k[X]$ is no longer a field.

A similar story unfolds one level up. \emph{Topos theory provides us with ``the
generic ring'', the ring we are implicitly picturing when someone utters
the phrase ``Let~$R$ be a ring''.} The generic ring has exactly those
properties (of the large class of ``geometric implications'') which all rings
have. To have the generic ring in our ontology,
we need to broaden our notion of existence---the generic ring is not a
ring in the usual sense of the word, but a ring object in an enlarged
topos, and still close enough to the familiar rings in that all
constructive theorems about rings apply to it.

Bizarrely, the generic ring has the property (not formalizable as a
geometric implication) that it is even a field. Hence, if we are to
prove a geometric implication for all rings, we can just as well assume
the field property.

The lecture introduces the notion of topos-theoretic generic models in
general, focussing on the generic ring, the generic prime ideal and
concrete applications.
\end{intro}

\vfill{\small%
\begin{verbatim}
- Existence of generic models
- Generic ring is a field
- Generic prime ideal
- Kripke–Joyal semantics
\end{verbatim}
}


\end{document}
