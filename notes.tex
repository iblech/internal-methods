\documentclass[10pt]{amsart}
\usepackage[utf8]{inputenc}
\usepackage[english]{babel}
\usepackage{amsmath,amsthm,amssymb,stmaryrd,color,graphicx,multirow}
\usepackage{setspace}
\usepackage{bussproofs}
\usepackage{xspace}
\usepackage{array}
\usepackage[protrusion=true,expansion=true]{microtype}
\usepackage[bookmarksdepth=2,pdfencoding=auto]{hyperref}
\usepackage[all]{xy}

\usepackage{tikz}
\usetikzlibrary{calc,shapes.callouts,shapes.arrows}
\newcommand{\hcancel}[5]{%
    \tikz[baseline=(tocancel.base)]{
        \node[inner sep=0pt,outer sep=0pt] (tocancel) {#1};
        \draw[red, line width=0.3mm] ($(tocancel.south west)+(#2,#3)$) -- ($(tocancel.north east)+(#4,#5)$);
    }%
}

\usepackage[natbib=true,style=numeric]{biblatex}
\usepackage[babel]{csquotes}
\bibliography{bibliography}

\theoremstyle{definition}
\newtheorem{defn}{Definition}[section]
\newtheorem{ex}[defn]{Example}

\theoremstyle{plain}

\newtheorem{prop}[defn]{Proposition}
\newtheorem{cor}[defn]{Corollary}
\newtheorem{lemma}[defn]{Lemma}
\newtheorem{thm}[defn]{Theorem}

\theoremstyle{remark}
\newtheorem{rem}[defn]{Remark}

\newcommand{\ZZ}{\mathbb{Z}}
\newcommand{\FF}{\mathbb{F}}
\renewcommand{\AA}{\mathbb{A}}
\newcommand{\A}{\mathcal{A}}
\renewcommand{\C}{\mathcal{C}}
\newcommand{\D}{\mathcal{D}}
\newcommand{\E}{\mathcal{E}}
\newcommand{\F}{\mathcal{F}}
\renewcommand{\G}{\mathcal{G}}
\renewcommand{\H}{\mathcal{H}}
\renewcommand{\O}{\mathcal{O}}
\newcommand{\K}{\mathcal{K}}
\newcommand{\N}{\mathcal{N}}
\renewcommand{\L}{\mathcal{L}}
\renewcommand{\P}{\mathcal{P}}
\newcommand{\R}{\mathcal{R}}
\newcommand{\I}{\mathcal{I}}
\renewcommand{\S}{\mathcal{S}}
\newcommand{\NN}{\mathbb{N}}
\newcommand{\RR}{\mathbb{R}}
\newcommand{\QQ}{\mathbb{Q}}
\newcommand{\GG}{\mathbb{G}}
\newcommand{\ppp}{\mathfrak{p}}
\newcommand{\Hom}{\mathrm{Hom}}
\newcommand{\HOM}{\mathcal{H}\mathrm{om}}
\newcommand{\id}{\mathrm{id}}
\newcommand{\GL}{\mathrm{GL}}
\newcommand{\placeholder}{\underline{\quad}}
\newcommand{\ul}[1]{\underline{#1}}
\newcommand{\Set}{\mathrm{Set}}
\newcommand{\Grp}{\mathrm{Grp}}
\newcommand{\Vect}{\mathrm{Vect}}
\newcommand{\Sh}{\mathrm{Sh}}
\newcommand{\PSh}{\mathrm{PSh}}
\newcommand{\Zar}{\mathrm{Zar}}
\newcommand{\Sch}{\mathrm{Sch}}
\newcommand{\Mod}{\mathrm{Mod}}
\DeclareMathOperator{\Spec}{Spec}
\DeclareMathOperator{\colim}{colim}
\DeclareMathOperator{\rank}{rank}
\DeclareMathOperator{\Ann}{Ann}
\DeclareMathOperator{\Int}{int}
\DeclareMathOperator{\Clos}{cl}
\DeclareMathOperator{\Kernel}{ker}
\DeclareMathOperator{\supp}{supp}
\newcommand{\Open}{\mathrm{Op}}
\newcommand{\?}{\,{:}\,}
\renewcommand{\_}{\mathpunct{.}\,}
\newcommand{\speak}[1]{\ulcorner\text{\textnormal{#1}}\urcorner}
\newcommand{\Ll}{:\Longleftrightarrow}
\newcommand{\notat}[1]{{!#1}}
\newcommand{\lra}{\longrightarrow}
\newcommand{\lhra}{\ensuremath{\lhook\joinrel\relbar\joinrel\rightarrow}}
\newcommand{\hra}{\hookrightarrow}
\newcommand{\brak}[1]{{\llbracket{#1}\rrbracket}}
\newcommand{\ie}{i.\,e.\@\xspace}
\newcommand{\eg}{e.\,g.\@\xspace}
\newcommand{\notnot}{\emph{not not}\xspace}

\definecolor{gray}{rgb}{0.7,0.7,0.7}

\title{Using the internal language of toposes in algebraic geometry}
\author{Ingo Blechschmidt}
\email{iblech@web.de}

\begin{document}

\begin{abstract}
  There are several important topoi associated to a scheme, for instance the
  petit and gros Zariski topoi. These support an internal mathematical language
  which closely resembles the usual formal language of mathematics, but is ``local
  on the base scheme'':

  For example, from the internal perspective, the structure sheaf looks like an
  ordinary local ring (instead of a sheaf of rings with local stalks) and vector
  bundles look like ordinary free modules (instead of sheaves of modules
  satisfying a certain condition). The translation of internal statements and
  proofs is facilitated by an easy mechanical procedure.

  These expository notes give an introduction to this topic and show how the internal
  point of view can be exploited to give simpler definitions and more conceptual
  proofs of the basic notions and observations in algebraic geometry. No prior
  knowledge about topos theory and formal logic is assumed.
\end{abstract}

\maketitle

\setcounter{tocdepth}{1}
\tableofcontents

\section{Introduction}

\subsection*{Internal language of toposes}
A \emph{topos} is a category which shares certain categorical properties with
the category of sets; the archetypical example is the category of sets, and
the most important example for the purposes of these notes is the category of
set-valued sheaves on a topological space.

Any topos~$\E$ supports an \emph{internal language}. This is a device which
allows one to \emph{pretend} that the objects of~$\E$ are plain sets and that
the morphisms are plain maps between sets, even if in fact they are not. For
instance, consider a morphism~$\alpha : X \to Y$ in~$\E$. From the \emph{internal
point of view}, this looks like a map between sets, and we can formulate the
condition that this map is surjective; we write this as
\[ \E \models \forall y\?Y\_ \exists x\?X\_ \alpha(x) = y. \]
The appearance of the colons instead of the usual element signs reminds us that
this expression is not to be taken literally --~$X$ and~$Y$ are objects of~$\E$
and thus not necessarily sets. The definition of the internal language is made
in such a way so that the meaning of this internal statement is that~$\alpha$
is an epimorphism. Similarly, the translation of the internal statement
that~$\alpha$ is injective is that~$\alpha$ is a monomorphism.

Furthermore, we can \emph{reason} with the internal language. There is a
metatheorem to the effect that if some statement~$\varphi$ holds from the
internal point of view of a topos~$\E$ and if~$\varphi$ logically implies some
further statement~$\psi$, then~$\psi$ holds in~$\E$ as well. As a simple
example, consider the elementary fact that the composition of surjective maps
is surjective. Interpreting this statement in the internal language of~$\E$, we
obtain the more abstract result that the composition of epimorphisms in~$\E$ is
epic.

There is, however, a slight caveat to this metatheorem. Namely, the internal
language of a topos is in general only \emph{intuitionistic}, not
\emph{classical}. This means that internally, one can not use the law of
excluded middle~($\varphi \vee \neg\varphi$), nor the law of double negation
elimination~($\neg\neg\varphi \Rightarrow \varphi$), nor the axiom of choice.
For instance, one rendition of the axiom of choice is that any surjection
splits. But it need not be the case that an epimorphism in a topos
splits.
% XXX: the translation of splitting would be that locally (!), there is a right
% inverse.


\subsection*{Algebraic geometry}
We apply this internal language to algebraic geometry as follows. If~$X$ is a
scheme, the structure sheaf~$\O_X$ is a sheaf of rings, \ie the sets of
local sections carry ring structures and these ring structures are compatible
with restriction. From the internal point of view of the topos of set-valued
sheaves on~$X$, denoted~``$\Sh(X)$'' in the following, the structure
sheaf~$\O_X$ looks much simpler: It looks just like a plain ring (and
not a sheaf of rings). Similarly, a sheaf of~$\O_X$-modules looks just like a
plain module over that ring.

This allows to import notions and facts from basic linear and commutative
algebra into the sheaf setting. For instance, it turns out that a sheaf
of~$\O_X$-modules is of finite type if and only if, from the internal
perspective, it is finitely generated as an~$\O_X$-module. Now consider the
following fact of linear algebra: If in a short exact sequence of modules the two
outer ones are finitely generated, then the middle one is too. The usual proof of
this fact is intuitionistically acceptable and can thus be interpreted in the
internal language. It then \emph{automatically} yields the following more advanced
proposition: If in a short exact sequence of sheaves of~$\O_X$-modules the
two outer ones are of finite type, then the middle one is too.

The internal language machinery thus allows us to understand the basic notions
and statements of scheme theory as notions and statements of linear and
commutative algebra, interpreted in a suitable sheaf topos. This brings
conceptual clarity and reduces technical overhead.

In these notes, we explain how the internal language works and then develop a
\emph{dictionary} between common notions of scheme theory and corresponding
notions of algebra. Once built, this dictionary can be used arbitrarily often.

Two highlights of our approach are the following. Let~$X$ be a reduced scheme
and~$\F$ a sheaf of~$\O_X$-modules of finite type. Then it is well-known
that~$\F$ is locally free on some dense open subset of~$X$; for instance, this
is stated in Vakil's lecture notes as an ``important hard
exercise''~\cite[exercise~13.7.K]{vakil:foag}. In fact, this fact is just the
interpretation of the following statement of intuitionistic linear algebra in
the sheaf topos: Any finitely generated vector space is \emph{not not} free.
The proof of this statement is entirely straightforward.\footnote{Intuitionistically,
the statement that any finitely generated vector space is free is stronger than
the doubly negated version and can not be shown. It would imply that any sheaf
of finite type is not only locally free on some dense open subset, but locally
free on the whole space.}

The second highlight is that we can shed light on the phenomenon that
sometimes, truth of a property at a point~$x$ spreads to some open
neighbourhood of~$x$; and in particular that sometimes, truth of a property at
the generic point spreads to some dense open subset. For instance, if the stalk
of a sheaf of finite type is zero at some point, the sheaf is even zero on some
open neighbourhood; but this spreading does not occur for general sheaves which
may fail to be of finite type.

We formalize this by introducing a \emph{modal operator}~$\Box$ into the
internal language, such that the internal statement~$\Box\varphi$ means
that~$\varphi$ holds on some open neighbourhood of~$x$. Furthermore, we
introduce a simple operation on formulas, the~\emph{$\Box$-translation}
$\varphi \mapsto \varphi^\Box$, such that~$\varphi^\Box$ means that~$\varphi$
holds at the point~$x$. The question whether truth at~$x$ spreads to truth on a
neighbourhood can thus be formulated in the following way: Does~$\varphi^\Box$
intuitionistically imply~$\Box\varphi$?

This allows to deal with the question in a simpler, more logical way, with the
technicalities of sheaves blinded out. We can also give a metatheorem which
covers a wide range of cases. Namely, spreading occurs for all those properties
which can be formulated in the internal language without
using~``$\Rightarrow$'',~``$\forall$'', and~``$\neg$''.

To illustrate the example above, consider the property of a module~$\F$ being
the zero module. In the internal language, it can be formulated as~$(\forall x\?\F\_ x = 0)$.
Because of the appearance of~``$\forall$'', the metatheorem is not
applicable to this statement. But if~$\F$ is of finite type, there are
generators~$x_1,\ldots,x_n\?\F$ from the internal point of view, and the
condition can be reformulated as~$x_1 = 0 \wedge \cdots \wedge x_n = 0$; the
metatheorem is applicable to this statement.


\subsection*{Limitations} The internal language is \emph{local}, in the sense
that if~$X = \bigcup_i U_i$ is an open covering and an internal statement
holds in the sheaf toposes~$\Sh(U_i)$, it holds in~$\Sh(X)$ as well. On the one
hand, this property is very useful. But on the other hand, it gives an inherent
limitation of the internal language:
Global properties of sheaves of modules like ``generated by global
sections'' or ``being ample'' and global properties of schemes like ``being
quasicompact'' or ``the cohomology of some sheaf vanishes'' can \emph{not} be
expressed in the internal language.

Thus for global considerations, the internal language of~$\Sh(X)$ is only
useful in that local subparts can be simplified. Also, some global features
reflect themselves in certain meta properties of the internal language (for
instance, a scheme is quasicompact if and only if the internal language
fulfills a weak version of the so-called disjunction property of mathematical
logic).


\subsection*{Introductory literature and related work} These notes are intended
to be self-contained, supposing only basic knowledge of scheme theory. In
particular, we assume no prior familiarity with topos theory or formal logic.
But if the interested reader is so inclined, she will find a gentle
introduction to topos theory in an article by Tom
Leinster~\cite{leinster:introduction}. Standard references for the internal
language of a topos include the book of Saunders Mac~Lane and
Ieke Moerdijk~\cite[chapter~VI]{moerdijk-maclane:sheaves-logic} and part~D of
Peter Johnstone's Elephant~\cite{johnstone:elephant}. In the 1970s, there was a
flurry of activity on applications of the internal language. An article by
Christopher Mulvey~\cite{mulvey:repr} of this time gives a very accessible
introduction to the topic, culminating in an internal proof of the Serre--Swan
theorem (with just one external ingredient needed).

The internal language of toposes was applied to algebraic geometry before. For
instance, Gavin Wraith used it to construct (and verify the universal property
of) the big étale topos of a scheme by internally developing the theory of
strict henselization~\cite{wraith:generic-galois-theory}. However, to the best
of my knowledge, a systematic creation of a dictionary between external and
% XXX: word "creation"
internal notions has not been attempted before, and the use of modal operators
to study the spreading of properties from points to neighbourhoods seems to be
new as well.

In other branches of mathematics, the internal language is used as well. For
instance, there is an ongoing effort in mathematical physics to understand
quantum mechanical systems from an internal point of view: To any quantum
mechanical system, one can associate a so-called Bohr topos containing an
internal mirror image of the system. This mirror image looks like a
system of classical mechanics from the internal perspective, and therefore
tools like Gelfand duality can be used to construct an internal
phase space for the system~\cite{bohr1,bohr2}.

% mention and explain: Mulvey/Burden, Vickers, Awodey, Coquand, ...
% uses of the big Zariski topos, further work

%Of particular importance is the
%case~$\Box\varphi :\equiv ((\varphi \Rightarrow \notat{x}) \Rightarrow
%\notat{x})$ where~$x \in X$ is a point, since for this modal operator, the
%proposition specializes to
%\[ \Sh(X) \models \varphi^\Box \quad\text{iff}\quad
%  \text{$\varphi$ holds at~$x$}. \]
%Thus the question whether truth of a proposition~$\varphi$ at a point~$x$ spreads
%to a neighbourhood of~$x$ can be formulated in the following way:
%\emph{Does~$\varphi^\Box$ imply~$\Box\varphi$?} We can give a general
%answer to this question.

\begin{itemize}
\item dictionary; intuitionistic logic; microscope/telescope into another
universe; types instead of sets; (dependent types to encompass almost all
mathematics)
\item explain that with the internal language business, it becomes more
transparent where scheme condition enters
\item note that in-depth knowledge of formal logic or topos theory is not necessary for
applications
\item give pointers to introductory literature
\end{itemize}


\section{The internal language of a sheaf topos}

\subsection{Internal statements}
Let~$X$ be a topological space. Later, $X$ will be the underlying space of a
scheme. The meaning of internal statements is given by a set of rules, the
\emph{Kripke--Joyal semantics} of the topos of sheaves on~$X$.

\begin{defn}The meaning of 
\[ U \models \varphi \quad\text{(``$\varphi$ holds on $U$'')} \]
for open subsets~$U \subseteq X$ and formulas~$\varphi$ over~$U$ is given by
the rules listed in table~\ref{table:kripke-joyal}, recursively in the structure of~$\varphi$.
In a \emph{formula over~$U$} there may appear sheaves defined on~$U$ as domains
of quantifications,~$U$-sections of sheaves as terms and morphisms of sheaves
on~$U$ as function symbols. The symbols~``$\top$'' and~``$\bot$'' denote truth
and falsehold, respectively. The universal and existential quantifiers come in
two flavors: for bounded and unbounded quantification.
The translation of~$U \models \neg\varphi$ does not have to be defined, since
negation can be expressed using other symbols: $\neg\varphi :\equiv (\varphi
\Rightarrow \bot)$. If we want to emphasize the particular topos, we write
\[ \Sh(X) \models \varphi \quad\Ll\quad X \models \varphi. \]
\end{defn}

\begin{table}
  \centering
  \[ \renewcommand{\arraystretch}{1.3}\begin{array}{@{}lcl@{}}
    U \models s = t \? \F &\Ll& s|_U = t|_U \in \Gamma(U, \F) \\
    U \models s \in \G &\Ll& s|_U \in \Gamma(U,\G) \quad\quad\text{($\G$ a
    subsheaf of~$\F$, $s$ a section of~$\F$)} \\
    U \models \top &\Ll& U = U \text{ (always fulfilled)} \\
    U \models \bot &\Ll& U = \emptyset \\
    U \models \varphi \wedge \psi &\Ll&
      \text{$U \models \varphi$ and $U \models \psi$} \\
    U \models \bigwedge_{j \in J} \varphi_j &\Ll&
      \text{for all~$j \in J$: $U \models \varphi_j$} \quad\quad\text{($J$ an
      index set)} \\
    U \models \varphi \vee \psi &\Ll&
      \hcancel{\text{$U \models \varphi$ or $U \models \psi$}}{0pt}{3pt}{0pt}{-2pt} \\
    && \text{there exists a covering $U = \bigcup_i U_i$ such that for all~$i$:} \\
    && \quad\quad \text{$U_i \models \varphi$ or $U_i \models \psi$} \\
    U \models \bigvee_{j \in J} \varphi_j &\Ll&
      \hcancel{\text{$U \models \varphi_j$ for some~$j \in J$}}{0pt}{3pt}{0pt}{-2pt}
      \quad\quad\text{($J$ an index set)} \\
    && \text{there exists a covering $U = \bigcup_i U_i$ such that for all~$i$:} \\
    && \quad\quad \text{$U_i \models \varphi_j$ for some~$j \in J$} \\
    U \models \varphi \Rightarrow \psi &\Ll&
      \text{for all open~$V \subseteq U$:
      $V \models \varphi$ implies $V \models \psi$} \\
    U \models \forall s \? \F\_ \varphi(s) &\Ll&
      \text{for all sections~$s \in \Gamma(V, \F)$, open $V \subseteq U$: $V \models
      \varphi(s)$} \\
    U \models \exists s \? \F\_ \varphi(s) &\Ll&
      \hcancel{\text{there exists a section~$s \in \Gamma(U,\F)$ such that $U
      \models \varphi(s)$}}{0pt}{3pt}{0pt}{-2pt} \\
    &&
      \text{there exists an open covering $U = \bigcup_i U_i$ such that for all~$i$:} \\
    && \quad\quad \text{there exists~$s_i \in \Gamma(U_i, \F)$ such that
    $U_i \models \varphi(s_i)$} \\
    U \models \forall \F\_ \varphi(\F) &\Ll&
      \text{for all sheaves $\F$ on $V$, open $V \subseteq U$: $V \models \varphi(\F)$} \\
    U \models \exists \F\_ \varphi(\F) &\Ll&
      \text{there exists an open covering $U = \bigcup_i U_i$ such that for all~$i$:} \\
    && \quad\quad \text{there exists a sheaf~$\F_i$ on~$U_i$ such that
    $U_i \models \varphi(\F_i)$}
  \end{array} \]
  \caption{\label{table:kripke-joyal}The Kripke--Joyal semantics of a sheaf
  topos.}
\end{table}

\begin{rem}The last two rules in table~\ref{table:kripke-joyal}, concerning
\emph{unbounded quantification}, and are not part of the classical Kripke--Joyal
semantics, but instead of Mike Shulman's stack semantics~\cite{shulman:stack},
a slight extension. They are needed so that we can formulate universal
properties in the internal language.
\end{rem}

\begin{ex}Let~$\alpha : \F \to \G$ be a morphism of sheaves on~$X$. Then
$\alpha$ is a monomorphism of sheaves if and only if, from the internal
perspective,~$\alpha$ is simply an injective map:
\allowdisplaybreaks
\begin{align*}
  & X \models \speak{$\alpha$ is injective} \\[0.5em]
  \Longleftrightarrow\
  & X \models \forall s\?\F\_ \forall t\?\F\_ \alpha(s) = \alpha(t) \Rightarrow s = t \\[0.5em]
  \Longleftrightarrow\ &
    \text{for all open~$U \subseteq X$, sections $s \in \Gamma(U, \F)$:} \\
  & \text{for all open~$V \subseteq U$, sections $t \in \Gamma(V, \F)$:} \\
  &\qquad\qquad
      V \models \alpha(s) = \alpha(t) \Rightarrow s = t \\[0.5em]
  \Longleftrightarrow\ &
    \text{for all open~$U \subseteq X$, sections $s, t \in \Gamma(U, \F)$:} \\
  &\qquad\qquad
      U \models \alpha(s) = \alpha(t) \Rightarrow s = t \\[0.5em]
  \Longleftrightarrow\ &
    \text{for all open~$U \subseteq X$, sections $s, t \in \Gamma(U, \F)$:} \\
  &\qquad\qquad
      \text{for all open~$W \subseteq U$:} \\
  &\qquad\qquad\qquad\qquad
        \text{$\alpha_W(s|_W) = \alpha_W(t|_W)$ implies $s|_W = t|_W$} \\[0.5em]
  \Longleftrightarrow\ &
    \text{for all open~$U \subseteq X$, sections $s, t \in \Gamma(U, \F)$:} \\
  &\qquad\qquad
        \text{$\alpha_U(s|_U) = \alpha_U(t|_U)$ implies $s|_U = t|_U$} \\[0.5em]
  \Longleftrightarrow\ &
    \text{$\alpha$ is a monomorphism of sheaves}
\end{align*}
The corner quotes ``$\speak{\ldots}$'' indicate that translation into formal
language is left to the reader. Similarly,~$\alpha$ is an epimorphism of
sheaves if and only if, from the internal perspective,~$\alpha$ is a
surjective map. Notice that injectivity and surjectivity are
notions of a simple element-based language, and the Kripke--Joyal semantics
takes care to properly handle \emph{all} sections, not only global ones.
\end{ex}

The rules are not all arbitrary. They are finely concerted to make the
following propositions true, which are crucial for a proper appreciation of the
internal language.

\begin{prop}[Locality of the internal language]
Let~$U = \bigcup_i U_i$ be covered by open subsets. Let~$\varphi$
be a formula over~$U$. Then
\[ U \models \varphi \qquad\text{iff}\qquad
  \text{$U_i \models \varphi$ for each $i$}. \]
\end{prop}
\begin{proof}Induction on the structure of~$\varphi$. Note that the canceled
rules would make this proposition false.\end{proof}

As a corollary, one may restrict the open coverings and universal
quantifications in the the definition of the Kripke--Joyal semantics
(table~\ref{table:kripke-joyal}) to open subsets of some basis of the topology.
For instance, if~$X$ is a scheme, one may restrict to affine open subsets.

Furthermore, the proposition shows that the internal language is monotone in
the following sense: If~$U \models \varphi$, and~$V$ is an open subset of~$U$,
then~$V \models \varphi$. (This follows by applying the proposition to the
trivial covering~$U = V \cup U$.)

\begin{prop}[Soundness of the internal language]
If a formula~$\varphi$ implies a further formula~$\psi$ in intuitionistic logic, then
$U \models \varphi$ implies $U \models \psi$.
\end{prop}
\begin{proof}
Proof by induction on the structure of formal intuitionistic proofs; we are to
show that any inference rule of intuitionistic logic is satisfied by the
Kripke--Joyal semantics. For instance, there is the following rule governing
disjunction:
\begin{quote}
If~$\varphi \vee \psi$ holds, and both $\varphi$ and $\psi$ imply a further
formula~$\chi$, then~$\chi$ holds.
\end{quote}
So we are to prove that if~$U \models \varphi \vee \psi$, $U \models (\varphi
\Rightarrow \chi)$, and $U \models (\psi \Rightarrow \chi)$, then $U \models \chi$.
This is done as follows: By assumption, there exists a covering~$U = \bigcup_i
U_i$ such that on each~$U_i$, $U_i \models \varphi$ or $U_i \models \psi$.
Again by assumption, we may conclude that~$U_i \models \chi$ for each~$i$. The statement
follows because of the locality of the internal language.

A complete list of which rules are to prove is
in~\cite[D1.3.1]{johnstone:elephant}.
\end{proof}
% XXX: Put rules into an appendix and give some explanation regarding contexts
% etc. Don't forget the rules for \in, \bigwedge, \bigvee.

% XXX: Is it clear that any constructively valid statement yields a valid sheaf
% statement?

Because of the multitude of quantifiers, literal translations of internal statements
can sometimes get slightly unwieldy. There are simplification rules for certain
often-occuring special cases:
\begin{prop}\label{prop:simplification}
    \[ \renewcommand{\arraystretch}{1.3}\begin{array}{@{}lcl@{}}
      U \models \forall s\?\F\_ \forall t\?\G\_ \varphi(s,t)
      &\Longleftrightarrow&
      \text{for all open~$V \subseteq U$,} \\
      && \text{sections~$s \in \Gamma(V,\F)$, $t \in \Gamma(V,\G)$:
      $V \models \varphi(s,t)$} \\[0.3em]
      U \models \forall s\?\F\_ \varphi(s) \Rightarrow \psi(s)
      &\Longleftrightarrow&
      \text{for all open~$V \subseteq U$, sections~$s \in \Gamma(V,\F)$:} \\
      &&\qquad\qquad \text{$V \models \varphi(s)$ implies $V \models \psi(s)$}
      \\[0.3em]
      U \models \exists!s\?\F\_ \varphi(s)
      &\Longleftrightarrow&
      \text{for all open~$V \subseteq U$,} \\
      &&
      \text{there is exactly one section~$s \in \Gamma(V,\F)$ with:} \\
      &&\qquad\qquad V \models \varphi(s)
    \end{array} \]
\end{prop}
\begin{proof}Straightforward. By way of example, we prove the existence claim
in the ``only if'' direction of the last rule. (Note that this rule formalizes
the saying ``unique existence implies global existence''.) By definition of~$\exists!$, it
holds that
\[ U \models \exists s\?\F\_ \varphi(s) \]
and
\[ U \models \forall s,t\?\F\_ \varphi(s) \wedge \varphi(t) \Rightarrow s = t.  \]
Let~$V \subseteq U$ be an arbitrary open subset. Then there exist local
section~$s_i \in \Gamma(V_i,\F)$ such that~$V_i \models \varphi(s_i)$, where~$V
= \bigcup_i V_i$ is an open covering. By the locality of the internal language,
on intersections it holds that~$V_i \cap V_j \models \varphi(s_i)$, so by the
uniqueness assumption, it follows that the local sections agree on intersections.
They therefore glue to a section~$s \in \Gamma(V,\F)$. Since~$V_i \models
\varphi(s)$ for any~$i$, the locality of the internal language allows us to
conclude that~$V \models \varphi(s)$.
\end{proof}

\begin{rem}Note that~$\Sh(X) \models \neg\varphi$ is in general a much stronger
statement that merely supposing that~$\Sh(X) \models \varphi$ does not hold:
The former always implies the latter (unless~$X = \emptyset$, in which case
\emph{any} internal statement is true), but the converse does not hold: The
former statement means that~$U = \emptyset$ is the \emph{only} open subset on
which~$\varphi$ holds.\end{rem}
% XXX: Find appropriate place for this remark.


\subsection{Internal constructions} The Kripke--Joyal semantics defines the
interpretation of internal statements. The interpretation of internal
constructions is given by the following definition.

\begin{defn}The interpretation of an internal construction~$T$
is denoted by~$\brak{T} \in \Sh(X)$ and given by the following rules.
\begin{itemize}\item If~$\F$ and~$\G$ are sheaves, $\brak{\F \times \G}$ is the
categorical product of~$\F$ and~$\G$ (\ie their product as presheaves).
\item If~$\F$ and~$\G$ are sheaves, $\brak{\F \amalg \G}$ is the categorical
coproduct of~$\F$ and~$\G$, \ie the sheafification of the presheaf
$U \mapsto \Gamma(U,\F) \amalg \Gamma(U,\G)$.
\item If~$\F$ is a sheaf, the interpretation~$\brak{\P(\F)}$ of the power set
construction is the sheaf given by
\[ \text{$U \subseteq X$ open} \longmapsto \{ \G \hookrightarrow \F|_U \}, \]
\ie sections on an open set~$U$ are subsheaves of~$\F|_U$ (either literally
or isomorphism classes of general monomorphisms into~$\F|_U$).
\item If~$\F$ is a sheaf and~$\varphi(s)$ is a formula containing a free
variable~$s\?\F$, the interpretation~$\brak{\{s\?\F\,|\,\varphi(s)\}}$ is given
by the subpresheaf of~$\F$ defined by
\[ \text{$U \subseteq X$ open} \longmapsto \{ s \in \Gamma(U,\F) \ |\ 
  U \models \varphi(s) \}. \]
Note that by the locality of the internal language, this presheaf is in fact a
sheaf.
\end{itemize}
\end{defn}

The definition is made in such a way that, from the internal perspective, the
constructions enjoy their expected properties. For instance, it holds that
\[ \Sh(X) \models
  \bigl[\forall x\?\brak{\{s\?\F \,|\, \varphi(s)\}}\_ \psi(x)\bigr]
  \Longleftrightarrow
  \bigl[\forall x\?\F\_ \varphi(x) \Rightarrow \psi(x)\bigr]. \]
We gloss over several details here. See~\cite[???]{johnstone:elephant} for
a proper treatment.

To be able to fully express all constructions of ``usual mathematics'' in the
internal language (\ie not those specifically designed to test the limitations
of the ambient logical framework), % XXX: weaken this phrase
we need \emph{dependent types}.
% XXX: short explanation, for instance by an example
% "A common example are coproducts indexed by elements of some sheaf"
In these notes, we do not describe how to deal with those. However, everything
carries over to the more general setting, and we refer to an article by Awodey and
Bauer~\cite{awodey-bauer:bracket} for a review of dependent types and their
categorical semantics.

% XXX: Construction of morphisms


\subsection{Geometric formulas and constructions} In categorical logic,
so-called geometric formulas play a special role, because their meaning is
preserved under pullback with geometric morphisms. % XXX gibberish!
\begin{defn}A formula is \emph{geometric} if and only if it consists only of
\[ {=} \quad {\in} \quad {\top} \quad {\bot} \quad {\wedge} \quad {\vee} \quad
{\bigvee} \quad {\exists}, \]
but not~``$\bigwedge$'' nor ``$\Rightarrow$'' nor~``$\forall$'' (and thus
not~``$\neg$'' either, since this is defined using~``$\Rightarrow$'').
A \emph{geometric implication} is a formula of the form
\[ \forall \cdots \forall\_ (\cdots) \Rightarrow (\cdots) \]
with the bracketed subformulas being geometric.
\end{defn}
We say that a formula~$\varphi$ holds \emph{at a point~$x \in X$} if and only
if the formula obtained by substituting all parameters in~$\varphi$ (sheaves
being quantified over, sections of sheaves appearing as terms and morphisms of
sheaves appearing as function symbols) with their stalks at~$x$ holds in the usual
mathematical sense.

\begin{lemma}\label{lemma:geometric-stalk-neighbourhood}
Let~$x \in X$ be a point. Let~$\varphi$ be a geometric formula (over some open
neighbourhood of~$x$).
Then~$\varphi$ holds at~$x$ if and only if there exists an open neighbourhood~$U
\subseteq X$ of~$x$ such that~$\varphi$ holds on~$U$.
\end{lemma}
\begin{proof}This is a very general instance of the phenomenom that sometimes,
truth at a point spreads to truth on a neighbourhood. It can be proven by
induction on the structure of~$\varphi$, but we will give a more conceptual
proof later (corollary~\ref{cor:geometric-spreading}).
\end{proof}

This lemma is in fact a very useful metatheorem. We will properly discuss its
significance in section~\ref{sect:spreading}. For now, we just use it to prove a
simple criterion for the internal truth of a geometric implication; we will
apply this criterion many times.

\begin{cor}A geometric implication holds on~$X$ if and only if it holds at
every point of~$X$.\end{cor}
\begin{proof}For notational simplicity, we consider a geometric implication of
the form
\[ \forall s\?\F\_ \varphi(s) \Rightarrow \psi(s). \]
For the ``only if'' direction, assume that this formula holds on~$X$ and let~$x
\in X$ be an arbitrary point. Let~$s_x \in \F_x$ be the germ of an arbitrary
local section~$s$ of~$\F$ and assume that~$\varphi(s)$ holds at~$x$. Then by
the lemma, it follows that~$\varphi(s)$ holds on some open neighbourhood of~$x$. By
assumption,~$\psi(s)$ holds on this neighbourhood as well. Again by the
lemma,~$\psi(s)$ holds at~$x$.

For the ``if'' direction, assume that the geometric implication holds at every
point. Let~$U \subseteq X$ be an arbitrary open subset and let~$s \in
\Gamma(U,\F)$ be a local section such that~$\varphi(s)$ holds on~$U$. By the
lemma and the locality of the internal language, to show that~$\psi(s)$ holds
on~$U$, it suffices to show that~$\psi(s)$
holds at every point of~$U$. This is clear, since again by the
lemma,~$\varphi(s)$ holds at every point of~$U$.
\end{proof}

\begin{ex}Injectivity and surjectivity are geometric implications (surjectivity
can be spelled~$\forall y\?\G\_ \top \Rightarrow \exists x\?\F\_ \alpha(x) =
y$). Thus the corollary gives a deeper reason for the well-known fact that a
morphism of sheaves is a monomorphism resp. an epimorphism if and only if it is
stalkwise injective resp. surjective.\end{ex}

A construction is \emph{geometric} if and only if it commutes with pullback
under arbitrary geometric morphisms. We do not want to discuss the notion of
geometric morphisms here; suffice it to say that calculating the stalk at a
point~$x \in X$ is an instance of such a pullback. Among others, the following
constructions are geometric:
\begin{itemize}
\item finite product: $(\F \times \G)_x \cong \F_x \times \G_x$
\item finite coproduct: $(\F \amalg \G)_x \cong \F_x \amalg \G_x$
\item arbitrary coproduct: $(\coprod_i \F_i)_x \cong \coprod_i (\F_i)_x$
\item set comprehension with respect to a \emph{geometric} formula~$\varphi$:
\[ \brak{\{ s\?\F \,|\, \varphi(s) \}}_x \cong \{ [s]\in\F_x \,|\,
\text{$\varphi(s)$ holds at $x$} \} \]
\item free module: $(\R\langle \F \rangle)_x \cong \R_x\langle \F_x
\rangle$ ($\R$ a sheaf of rings, $\F$ a sheaf of sets)
\item localization of a module: $\F[\S^{-1}]_x \cong \F_x[\S_x^{-1}]$
\end{itemize}
% XXX: compatibility with stalks is not sufficient for geometricity.
% This list makes one think it is.

The following constructions are not in general geometric:
\begin{itemize}
\item arbitrary product
\item set comprehension with respect to a non-geometric formula
\item powerset
\item internal Hom: $\HOM(\F,\G)_x \not\cong \Hom(\F_x,\G_x)$
\end{itemize}

\begin{itemize}
\item crash course on intuitionistic logic
\end{itemize}


\section{Sheaves of rings}

Recall that a \emph{sheaf of rings} can be categorically described as a
sheaf of sets~$\R$ together with maps of sheaves $+, \cdot : \R \times \R \to
\R$ and global elements~$0, 1$ such that certain axioms hold. For instance, the
axiom on the commutativity of addition is rendered in diagrammatic form as
follows:
\[ \xymatrix{
  \R \times \R \ar[rr]^{\mathrm{swap}} \ar[rd]_{+} && \R \times \R \ar[ld]^{+} \\
  & \R
} \]

From the internal perspective, a sheaf of rings looks just like a plain ring.
This is the content of the following proposition:

\begin{prop}\label{prop:rings-internally}
Let~$X$ be a topological space. Let~$\R$ be a sheaf of sets on~$X$.
Let~$+, \cdot : \R \times \R \to \R$ be maps of sheaves and let~$0, 1$ be
global elements of~$\R$. Then these data define a sheaf of rings if and only
if, from the internal perspective, these data fulfill the usual equational ring
axioms.\end{prop}
\begin{proof}We only discuss the commutativity axiom. The internal statement
\[ \Sh(X) \models \forall x,y\?\R\_ x + y = y + x \]
means that for any open subset~$U \subseteq X$ and any local sections~$x,y \in
\Gamma(U,\R)$, it holds that~$x + y = y + x \in \Gamma(U,\R)$. This is
precisely the external commutativity condition.
\end{proof}

\begin{lemma}Let~~$X$ be a topological space. Let~$\R$ be a sheaf of rings
on~$X$. Let~$f$ be a global section of~$\R$. Then the following statements are
equivalent:
\begin{enumerate}
\item $f$ is invertible from the internal point of view, \ie $\Sh(X) \models
\exists g\?\R\_ fg = 1$.
\item $f$ is is invertible in all stalks~$\R_x$.
\item $f$ is in invertible in~$\Gamma(X,\R)$.
\end{enumerate}
\end{lemma}
\begin{proof}Since invertibility is a geometric implication, the equivalence of
the first two statements is clear. Also, it's obvious that the third statement
implies the other two. For the remaining direction, note that the
uniqueness of inverses in rings can be proven intuitionistically. Therefore, if~$f$ is invertible
from the internal point of view, it actually holds that
\[ \Sh(X) \models \exists! g\?\R\_ fg = 1. \]
Since unique internal existence implies global existence
(proposition~\ref{prop:simplification}), this shows that the first statement
implies the third.
\end{proof}


\subsection{Reducedness} Recall that a scheme~$X$ is \emph{reduced} if and only
if all stalks~$\O_{X,x}$ are reduced rings. Since the condition on a ring~$R$
to be reduced is a geometric implication,
\[ \forall s\?R\_ \Bigl(\bigvee_{n \geq 0} s^n = 0\Bigr) \Longrightarrow s = 0, \]
we immediately obtain the following characterization of reducedness in the
internal language:
\begin{prop}A scheme~$X$ is reduced iff, from the internal point of view, the
ring~$\O_X$ is reduced.\end{prop}


\subsection{Locality} Recall the usual definition of a local ring: a ring
possessing exactly one maximal ideal. This is a higher-order condition and in
particular not of a geometric form. Therefore, for our purposes, it's better to
adopt the following elementary definition of a local ring.
\begin{defn}A \emph{local ring} is a ring~$R$ such that~$1 \neq 0$ in~$R$ and
for all~$x,y \in R$
\[ \text{$x+y$ invertible} \quad\Longrightarrow\quad
  \text{$x$ invertible}\ \vee\ \text{$y$ invertible}. \]
\end{defn}
In classical logic, it's an easy exercise to show that this definition is
equivalent to the usual one. In intuitionistic logic, we would need to be
more precise in order to even state the question of equivalence, since
intuitionistically, the notion of a maximal ideal bifurcates into several
non-equivalent notions.

\begin{prop}In the internal language of a scheme~$X$ (or a locally ringed
space), the ring~$\O_X$ is a local ring.\end{prop}
\begin{proof}The stated locality condition is a conjunction of two geometric
implications (the first one being~$1 = 0 \Rightarrow \bot$, the second being
the displayed one) and holds on each stalk.\end{proof}


\subsection{Field properties} From the internal point of view, the structure
sheaf~$\O_X$ of a scheme~$X$ is \emph{almost} a field, in the sense that any
element which is not invertible is nilpotent. This is a genuine property of
schemes, not shared with general locally ringed spaces.

\begin{prop}Let~$X$ be a scheme. Then
\[ \Sh(X) \models \forall s\?\O_X\_ \neg(\speak{$s$ invertible}) \Rightarrow
\speak{$s$ nilpotent}. \]
\end{prop}
\begin{proof}By the locality of the internal language and since~$X$ can be
covered by open affine subsets, it's enough to show that for any affine
scheme~$X = \Spec A$ and global function~$s \in \Gamma(X,\O_X) = A$ it holds
that
\[ X \models \neg(\speak{$s$ invertible}) \quad\text{implies}\quad
  X \models \speak{$s$ nilpotent}. \]
The meaning of the antecedent is that any open subset on which~$s$ is
invertible is empty. So in particular, the standard open subset~$D(s)$ is
% XXX: missing: "Let's assume the antecedent to be true"
empty. Therefore~$s$ is an element of any prime ideal of~$A$ and thus
nilpotent. This implies the a priori weaker statement~$X \models \speak{$s$
nilpotent}$ (which would allow~$s$ to have different indices of nilpotency on
an open covering).
\end{proof}

\begin{cor}\label{cor:field-reduced}
Let~$X$ be a scheme. If~$X$ is reduced, the ring~$\O_X$ is a field
from the internal point of view, in the sense that
\[ \Sh(X) \models \forall s\?\O_X\_ \neg(\speak{$s$ invertible}) \Rightarrow
s=0. \]
The converse holds as well.\end{cor}
\begin{proof}We can prove this purely in the internal language: It suffices to
give an intuitionistic proof of the fact that a local ring which satisfies the
condition of the previous proposition fulfills the stated field condition if
and only if it is reduced. This is straightforward.
\end{proof}

This field property is very useful. We will put it to good use when giving a
simple proof of the fact that~$\O_X$-modules of finite type on a reduced scheme
are locally free on a dense open subset (lemma~\ref{lemma:locally-free-dense}).

The following proposition says that one can deduce a certain unconditional
statement from the premise that an element~$s\?\O_X$ is zero under the
assumption that some further element~$f\?\O_X$ is invertible.

\begin{prop}\label{prop:cond-zero}
Let~$X$ be a scheme. Then
\[ \Sh(X) \models
  \forall f\?\O_X\_
  \forall s\?\O_X\_
  (\speak{$f$ inv.} \Rightarrow s = 0) \Longrightarrow
  \textstyle
  \bigvee_{n \geq 0} f^n s = 0. \]
\end{prop}
\begin{proof}It's enough to show that for any affine scheme~$X = \Spec A$ and
global functions~$f, s \in A$ such that
\[ X \models (\speak{$f$ inv.} \Rightarrow s = 0), \]
it holds that $X \models \textstyle \bigvee_{n \geq 0} f^n s = 0$. This is
obvious, since by assumption such a function~$s$ is zero on~$D(f)$, \ie $s$
is zero as an element of~$A[f^{-1}]$.
\end{proof}

\begin{itemize}
\item Remark that intuitionistically, the notion of a field bifurcates into
several inequivalent notions
\item normal rings, principal ideal domains, \ldots
\item discreteness
\end{itemize}


\section{Sheaves of modules}

From the internal perspective, a sheaf of~$\R$-modules, where~$\R$ is a sheaf
of rings, looks just like a plain module over the plain ring~$\R$. This is
proven just as the correspondence between sheaf of rings and internal rings
(proposition~\ref{prop:rings-internally}).


\subsection{Local finite freeness}
% XXX: grammar?

Recall that an~$\O_X$-module~$\F$ is \emph{locally finitely free} if and only
if there exists a covering of~$X$ by open subsets~$U$ such that on each
such~$U$, the restricted module~$\F|_U$ is isomorphic as an~$\O_X|_U$-module
to~$(\O_X|_U)^n$ for some natural number~$n$ (which may depend on~$U$).

\begin{prop}Let~$X$ be a scheme (or ringed space). Let~$\F$ be
an~$\O_X$-module. Then~$\F$ is locally finitely free if and only if, from the
internal perspective,~$\F$ is a finitely free module, \ie
\[ \Sh(X) \models \bigvee_{n \geq 0} \speak{$\F \cong (\O_X)^n$} \]
or more elementary
\[ \Sh(X) \models \bigvee_{n \geq 0}
  \exists x_1,\ldots,x_n\?\F\_
  \forall x\?\F\_
  \exists! a_1,\ldots,a_n\?\O_X\_
  x = \textstyle\sum\limits_i a_i x_i. \]
\end{prop}
\begin{proof}By the expression~``$(\O_X)^n$'' in the internal language we mean
the internally constructed object~$\O_X \times \cdots \times \O_X$ with its
pointwise~$\O_X$-module structure. This coincides with the sheaf~$(\O_X)^n$ as
usually understood.

It is clear that the two stated internal conditions are equivalent, since the
corresponding proof in linear algebra is intuitionistic. The equivalence with
the external notion of locally finite freeness is obvious, since the
interpretation of the first condition with the Kripke--Joyal semantics is the
following: There exists a covering of~$X$ by open subsets~$U$ such that for
each such~$U$, there exists a natural number~$n$ and a morphism of
sheaves~$\varphi : \F|_U \to (\O_X|_U)^n$ such that
\[ U \models \speak{$\varphi$ is~$\O_X$-linear} \quad\text{and}\quad
  U \models \speak{$\varphi$ is bijective}. \]
The first subcondition means that~$\varphi$ is a morphism of sheaves
of~$\O_X$-modules and the second one means that~$\varphi$ is an isomorphism of
sheaves.
\end{proof}


\subsection{Finite type, finite presentation, coherence}
Recall the conditions of an~$\O_X$-module~$\F$ on a scheme~$X$ (or ringed
space) to be of finite type, of finite presentation and to be coherent:
\begin{itemize}
\item $\F$ is \emph{of finite type} if and only if there exists a covering of~$X$ by
open subsets~$U$ such that on each such~$U$, there exists an exact sequence
\[ (\O_X|_U)^n \longrightarrow \F|_U \longrightarrow 0 \]
of~$\O_X|_U$-modules.
\item $\F$ is \emph{of finite presentation} if and only if there exists a covering of~$X$ by
open subsets~$U$ such that on each such~$U$, there exists an exact sequence
\[ (\O_X|_U)^m \longrightarrow (\O_X|_U)^n \longrightarrow \F|_U \longrightarrow 0. \]
\item $\F$ is \emph{coherent} if and only if~$\F$ is of finite type and the
kernel of any~$\O_X|_U$-linear morphism~$(\O_X|_U)^n \to \F|_U$, $U \subseteq
X$ any open subset, is of finite type.
\end{itemize}

The following proposition gives translations of these definitions into the
internal language.
\begin{prop}Let~$X$ be a scheme (or ringed space). Let~$\F$ be
an~$\O_X$-module. Then:
\begin{itemize}
\item $\F$ is of finite type if and only if~$\F$, considered as an ordinary
module from the internal perspective, is finitely generated, \ie if
\[ {\qquad\qquad} \Sh(X) \models
  \bigvee_{n \geq 0}
  \exists x_1,\ldots,x_n\?\F\_
  \forall x\?\F\_
  \exists a_1,\ldots,a_n\?\F\_
  x = \textstyle\sum\limits_i a_i x_i. \]
\item $\F$ is of finite presentation if and only if~$\F$ is a finitely
presented module from the internal perspective, \ie if
\[ {\qquad\qquad} \Sh(X) \models \bigvee_{n,m \geq 0}
  \speak{there is a short exact sequence $\O_X^m \to \O_X^n \to \F \to 0$}.
  \]
\item $\F$ is coherent if and only if~$\F$ is a coherent module from the
internal perspective, \ie if
\begin{multline*}
{\qquad\qquad\qquad}
  \Sh(X) \models \speak{$\F$ is finitely generated} \mathop{\wedge} \\
  \bigwedge_{n \geq 0} \forall \varphi \? \HOM_{\O_X}(\O_X^n,\F)\_
  \speak{$\Kernel \varphi$ is finitely generated}.
\end{multline*}
\end{itemize}
\end{prop}
\begin{proof}Straightforward: The translations of the internal statements using
the Kripke--Joyal semantics are precisely the corresponding external
statements.
\end{proof}

Recall that an~$\O_X$-module~$\F$ is \emph{generated by global sections} if and
only if there exist global sections~$s_i \in \Gamma(X,\F)$ such that for any~$x
\in X$, the stalk~$\F_x$ is generated by the germs of the~$s_i$.
This condition is of course not local on the base space. Therefore there cannot
exist a formula~$\varphi$ such that for any space~$X$ and
any~$\O_X$-module~$\F$ it holds that~$\F$ is generated by global sections if
and only if~$\Sh(X) \models \varphi(\F)$. But still, global generation can be
characterized by a mixed internal/external statement:

\begin{prop}Let~$X$ be a scheme (or ringed space). Let~$\F$ be
an~$\O_X$-module. Then~$\F$ is generated by global sections if and only if
there exist global sections~$s_i \in \Gamma(X,\F)$, $i \in I$ such that
\[ \Sh(X) \models \forall x\?\F\_ \bigvee\nolimits_{\textnormal{$J=\{i_1,\ldots,i_n\} \subseteq I$ finite}}
  \exists a_1,\ldots,a_n\?\O_X\_
  x = \sum_j a_j x_{i_j}. \]
Furthermore,~$\F$ is generated by finitely many global sections if and only if
there exist global sections~$s_1,\ldots,s_n \in \Gamma(X,\F)$ such that
\[ \Sh(X) \models \forall x\?\F\_ \exists a_1,\ldots,a_n\?\O_X\_ x = \sum_j a_j
x_j. \]
\end{prop}
\begin{proof}The given internal statements are geometric implications, their
validity can thus be checked stalkwise.\end{proof}


\subsection{Tensor product} Recall that the tensor product
of~$\O_X$-modules~$\F$ and~$\G$ on a scheme~$X$ (or ringed space) is usually
constructed as the sheafification of the presheaf
\[ \text{$U \subseteq X$ open} \longmapsto \Gamma(U,\F) \otimes_{\Gamma(U,\O_X)}
\Gamma(U,\G). \]
From the internal point of view,~$\F$ and~$\G$ look like ordinary modules, so
that we can consider their tensor product as usually constructed in
commutative algebra, as a certain quotient of a free module on the elements
of~$\F \times \G$:
\[ \O_X\langle x \otimes y \,|\, x\?\F, y\?\G \rangle / R, \]
where~$R$ is the submodule generated by
\begin{gather*}
  (x+x') \otimes y - x \otimes y - x' \otimes y, \\
  x \otimes (y+y') - x \otimes y - x \otimes y', \\
  (sx) \otimes y - s(x \otimes y), \\
  x \otimes (sy) - s(x \otimes y)
\end{gather*}
with~$x,x'\?\F$, $y,y'\?\G$, $s\?\O_X$.
This internal construction will give rise to the same sheaf
of modules as the externally defined tensor product:
% XXX: grammar

\begin{prop}Let~$X$ be scheme (or a ringed space). Let~$\F$ and~$\G$
be~$\O_X$-modules. Then the internally constructed tensor product~$\F
\otimes_{\O_X} \G$ coincides with the external one.
\end{prop}
\begin{proof}
Since the proof of the corresponding fact of commutative algebra is
intuitionistic, the internally defined tensor product~$\F \otimes_{\O_X} \G$
fulfills the following universal property: For any~$\O_X$-module~$\H$,
any~$\O_X$-bilinear map~$\F \times \G \to \H$ uniquely factors over the
canonical map~$\F \times \G \to \F \otimes_{\O_X} \G$.

Interpreting this property with the Kripke--Joyal semantics, we see that the
internally constructed tensor product fulfills the following external property:
For any open subset~$U \subseteq X$ and any~$\O_X|_U$-module~$\H$ on~$U$,
any~$\O_X|_U$-bilinear morphism~$\F|_U \times \G|_U \to \H$ uniquely factors over the
canonical morphism~$\F \times \G \to (\F \otimes_{\O_X} \G)|_U$.

In particular, for~$U = X$, this property is well-known to be the universal
property satisfied by the externally constructed tensor product. Therefore the
claim follows.
\end{proof}

By the internal construction, a description of the stalks of the tensor product
follows purely by considering the logical form of the construction:
\begin{cor}Let~$X$ be scheme (or a ringed space). Let~$\F$ and~$\G$
be~$\O_X$-modules. Then the stalks of the tensor product coincide with the
tensor products of the stalks: $(\F \otimes_{\O_X} \G)_x \cong \F_x
\otimes_{\O_{X,x}} \G_x$.\end{cor}
\begin{proof}
We constructed the tensor product using the following operations: product of
two sets, free module on a set, quotient module with respect to a submodule;
submodule generated by a set of elements given by a geometric formula.
All of these operations are geometric, so the tensor product construction is
geometric as well. Hence taking stalks commutes with performing the
construction.
\end{proof}

Recall that an~$\O_X$-module~$\F$ is \emph{flat} if and only if all
stalks~$\F_x$ are flat~$\O_{X,x}$-modules. We can characterize flatness in the
internal language.
\begin{prop}Let~$X$ be a scheme (or ringed space). Let~$\F$ be
an~$\O_X$-module. Then~$\F$ is flat if and only if, from the internal
perspective,~$\F$ is a flat~$\O_X$-module.
\end{prop}
\begin{proof}
Recall that flatness of an~$A$-module~$M$ can be characterized without
reference to tensor products by the following condition (using
suggestive vector notation): For any natural number~$p$,
any $p$-tuple~$m \? M^p$ of elements of~$M$ and
any~$p$-tuple $a \? A^p$ of elements of~$A$,
\[
  a^T m = 0 \ \Longrightarrow\ 
  \bigvee\limits_{q \geq 0} \exists n\?M^q, B\?A^{p \times q}\_
  Bn = m \wedge a^T B = 0. \]
The equivalence of this condition with tensoring being exact holds
intuitionistically as
well~\cite[theorem~III.5.3]{mines-richman-ruitenburg:constructive-algebra}.
This formulation of flatness has the advantage that it is the conjunction of
geometric implications (one for each~$p \geq 0$); therefore it holds internally
if and only if it holds at any stalk.
\end{proof}


\subsection{Support} Recall that the \emph{support} of an~$\O_X$-module~$\F$ is
the subset~$\supp\F := \{ x \in X \,|\, \F_x \neq 0 \} \subseteq X$. If~$\F$ is
of finite type, this set is closed, since its complement is then open by a
standard lemma. (We will give an internal proof of this lemma in
lemma~\ref{lemma:module-zero-point-neighbourhood}.)

\begin{prop}\label{prop:characterization-support}
Let~$X$ be a scheme (or ringed space). Let~$\F$ be
an~$\O_X$-module. Then the interior of the complement of the support of~$\F$
can be characterized as the largest open subset of~$X$ on which the internal
statement~$\F = 0$ holds.
\end{prop}
\begin{proof}
For any open subset~$U \subseteq X$, it holds that:
\begin{align*}
  &\ U \subseteq \Int(X \setminus \supp \F) \\
  \Longleftrightarrow&\ U \subseteq X \setminus \supp \F \\
  \Longleftrightarrow&\ U \subseteq \{ x \in X \,|\, \forall s \in \F_x\_ s = 0 \} \\
  \Longleftrightarrow&\ U \models \forall s\?\F\_ s = 0 \\
  \Longleftrightarrow&\ U \models \speak{$\F = 0$}
\end{align*}
The second to last equivalence is because~``$\forall s\?\F\_ s = 0$'' is a
geometric implication and can thus be checked stalkwise.
\end{proof}

\begin{rem}\label{rem:support-sheaf-of-sets}
The support of a sheaf of \emph{sets}~$\F$ is defined as the subset~$\{ x \in X \,|\,
\text{$\F_x$ is not a singleton} \}$. A similar proof shows that the interior
of its complement can be characterized as the largest open subset of~$X$ where
the internal statement~$\speak{$\F$ is a singleton}$ holds.\end{rem}


\subsection{Torsion} Omitted from this first draft.
% XXX


\subsection{Internal proofs of common lemmas}

\begin{lemma}Let~$X$ be a scheme (or ringed space). Let
\[ 0 \lra \F \lra \G \lra \H \lra 0 \]
be a short exact sequence of~$\O_X$-modules. If~$\F$ and~$\H$ are of finite
type, so is~$\G$; similarly, if~$\F$ and~$\H$ are locally finitely free, so
ist~$\G$.
\end{lemma}
\begin{proof}From the internal perspective, we are given a short exact sequence
of modules with the outer ones being finitely generated (resp. finitely free)
and we have to show that the middle one is finitely generated (resp. finitely
free) as well. It is well-known that this follows; and since the usual proof of
this fact is intuitionistic, we are done.
\end{proof}

\begin{lemma}\label{lemma:coherent-stuff}
Let~$X$ be a scheme (or ringed space). Then:
\begin{itemize}
\item Let~$0 \to \F \to \G \to \H \to 0$ be an exact sequence
of~$\O_X$-modules. If two of the three modules are coherent, so is the third.
\item Let~$\F \to \G$ be a morphism of~$\O_X$-modules such that~$\F$ is
of finite type and~$\G$ is coherent. Then its kernel is of finite type as well.
\item If~$\F$ is a finitely presented~$\O_X$-module and~$\G$ is a
coherent~$\O_X$-module, the~$\O_X$-modules~$\HOM(\F,\G)$ and~$\F \otimes \G$
are coherent as well.
\end{itemize}
\end{lemma}
\begin{proof}These statements follow directly from interpreting the
corresponding standard proofs of commutative algebra in the internal language.
For those standard proofs, see for instance the lecture notes of Ravi
Vakil~\cite[section~13.8]{vakil:foag}, where they are given as a series of
exercises.
\end{proof}

\begin{lemma}\label{lemma:kernel-of-epi-fingen}
Let~$X$ be a scheme (or locally ringed space). Let~$\alpha : \G
\to \H$ be an epimorphism of locally finitely free~$\O_X$-modules. Then the
kernel of~$\alpha$ is locally finitely free as well.\end{lemma}
\begin{proof}It suffices to give an intuitionistic proof of the following
statement: The kernel of a matrix over a local ring, which as a linear map is
surjective, is finitely free.

Let~$M \in R^{n \times m}$ be such a matrix. Since by the surjectivity
assumption some linear combination of the columns is~$e_1$ (the first canonical
basis vector), some linear combination of the entries of the first row of~$M$
is~$1$. By locality of~$R$, at least one entry of the first row is invertible.
By applying appropriate column and row transformations, we may assume that~$M$
is of the form
\[ \left(
  \begin{array}{c|ccc}
    1 & 0 & \cdots & 0 \\ \hline
    0 & \multicolumn{3}{c}{\multirow{3}{*}{\raisebox{-5mm}{\scalebox{1.5}{$\widetilde M$}}}} \\
    \raisebox{2pt}{\vdots} & & & \\
    0 & & &
  \end{array}
\right) \]
with the submatrix~$\widetilde M$ fulfilling the same condition as~$M$.
Continuing in this way, it follows that~$m \geq n$ and that we may assume
that~$M$ is of the form
\[ \left(
  \begin{array}{ccc|c}
    1 & & & \multirow{3}{*}{\raisebox{-3mm}{\scalebox{1.5}{\ 0}}} \\
    & \ddots && \\
    && 1
  \end{array}
\right)\!. \]
The kernel of such a matrix is obviously freely generated by the canonical
basis vectors corresponding to the zero columns. In particular, the rank of the
kernel is~$m-n$.
\end{proof}

\begin{rem}The internal language machinery gives no reason to believe that the
dual statement is true, \ie that the cokernel of a monomorphism of locally
finitely free~$\O_X$-modules is locally finitely free: This would follow from
an intuitionistic proof of the statement that the cokernel of an injective map
between finitely free modules over a local ring is finitely free. But this
statement is false, as the following example shows.
\[ 0 \lra \ZZ_{(2)} \stackrel{\cdot 2}{\lra} \ZZ_{(2)} \lra \FF_2 \lra 0 \]
\end{rem}

\begin{lemma}Let~$X$ be a scheme (or locally ringed space). Let~$\alpha : \G
\to \H$ be an epimorphism of locally finitely free~$\O_X$-modules of the same
rank. Then~$\alpha$ is an isomorphism.\end{lemma}
\begin{proof}It suffices to give an intuitionistic proof of the following
statement: A square matrix over a local ring, which as a linear map is
surjective, is invertible.

This follows from the proof of the previous lemma, since it shows that the
kernel of such a matrix is finitely free of rank zero.
\end{proof}

\begin{lemma}Let~$X$ be a scheme (or ringed space). Let
$0 \to \F \to \G \to \H \to 0$ be a short exact sequence of~$\O_X$-modules.
Then
$\Clos \supp \G = \Clos \supp \F \cup \Clos \supp \H$.
\end{lemma}
\begin{proof}Switching to complements, we have to prove that
\[ \Int(X \setminus \supp\G) = \Int(X \setminus \supp\F) \cap \Int(X \setminus
\supp\H). \]
By proposition~\ref{prop:characterization-support}, it suffices to prove
\[ \Sh(X) \models (\G = 0 \Longleftrightarrow \F = 0 \wedge \H = 0); \]
this is a basic observation in linear algebra, valid intuitionistically.
\end{proof}
% XXX: This is kind of a lame example.

\begin{lemma}Let~$X$ be a scheme (or locally ringed space). Let~$\L$ be a line
bundle on~$X$, \ie an~$\O_X$-module locally free of rank~1.
Let~$s_1,\ldots,s_n \in \Gamma(X,\L)$ be global sections. Then these sections
globally generate~$\F$ if and only if
\[ \Sh(X) \models \bigvee_i \speak{$\alpha(s_i)$ is invertible for some
isomorphism~$\alpha : \L \to \O_X$}. \]
\end{lemma}
\begin{proof}It suffices to give an intuitionistic proof of the following fact:
Let~$R$ be a local ring. Let~$L$ be a free~$R$-module of rank~1.
Let~$s_1,\ldots,s_n\?L$ be given elements. Then~$L$ is generated as
an~$R$-module by these elements if and only if for some~$i$, the image of~$s_i$
under some isomorphism~$L \to R$ is invertible.

Note that the choice of such an isomorphism does not matter, since any two such
isomorphisms~$\alpha, \beta : L \to R$ differ by a unit of~$R$: $\alpha(x) =
\alpha(\beta^{-1}(1)) \cdot \beta(x)$ for any~$x\?L$,
and~$\alpha(\beta^{-1}(1)) \cdot \beta(\alpha^{-1}(1)) = 1$ in~$R$.

For the ``if'' direction, we have that some~$\alpha(s_i)$ is a generator
of~$R$. Since~$\alpha$ is an isomorphism, it follows that~$s_i$ generates~$L$,
and thus in particular, the family~$s_1,\ldots,s_n$ generates~$L$.

For the ``only if'' direction, we have that the unit of~$R$ can be expressed as
a linear combination of the~$\alpha(s_i)$, where~$\alpha : L \to R$ is some
isomorphism (whose existence is assured by the assumption on the rank of~$L$).
Since~$R$ is a local ring, it follows that one of the summands and thus one of
the~$\alpha(s_i)$ is invertible.
\end{proof}

\begin{rem}Note that the canonical ring homomorphism~$\O_{X,x}
\twoheadrightarrow k(x)$ is local. Therefore a germ in~$\O_{X,x}$ is invertible
if and only if its image in~$k(x)$ is not zero. From this one can follow that
global sections~$s_1,\ldots,s_n \in \Gamma(X,\F)$ generate~$\F$ if and only if,
for any point~$x \in X$, the images~$s_i \in \F|_x$ in the fibers do not vanish
simultaneously.
\end{rem}

\begin{lemma}Let~$X$ be a scheme (or ringed space). Let~$\L$ be a locally finitely
free~$\O_X$-module. Then~$\L^\vee \otimes_{\O_X} \L \cong \O_X$.\end{lemma}
\begin{proof}Recall that the dual is defined by~$\L^\vee :=
\HOM_{\O_X}(\L,\O_X)$. Since~``$\HOM$'' looks like~``$\Hom$'' from the internal
point of view, the dual sheaf~$\L^\vee$ looks just like the ordinary dual
module. However, to prove the claim, it does \emph{not} suffice to give an
intuitionistic proof of the following fact of linear algebra: Let~$L$ be a
free~$R$-module of rank~1. Then there exists an isomorphism~$L^\vee \otimes_R L
\to R$. Since the interpretation of ``$\exists$'' using the Kripke--Joyal
semantics is local existence, this would only show that~$\L^\vee \otimes_{\O_X}
\L$ is \emph{locally} isomorphic to~$\O_X$.

Instead, we have to actually \emph{write down} (\ie explicitly give) an
isomorphism in the internal language -- not using the assumption that~$L$ is
free of rank~1, as this would introduce an existential quantifier again (see XXX).
So we have to prove the following fact: Let~$L$ be an~$R$-module. Then there
explicitly exists a linear map~$L^\vee \otimes_R L \to R$ such that this map is
an isomorphism if~$L$ is free of rank~1.

This is done as usual: Define~$\alpha : L^\vee \otimes_R L \to R$ by~$\lambda
\otimes x \mapsto \lambda(x)$. Since~$L$ is free of rank~1, there is an
isomorphism~$L \cong R$. Precomposing~$\alpha$
with the induced isomorphism~$R^\vee \otimes_R R \to L^\vee \otimes_R L$,
we obtain the linear map~$R^\vee \otimes_R R \to R$ given by the same
term: $\lambda \otimes x \mapsto \lambda(x)$. One can check that an inverse is given
by~$x \mapsto \id_R \otimes x$.
\end{proof}


\begin{itemize}
\item basic lemmas: filtered colimits, flatness, \ldots
\end{itemize}


\section{Upper semicontinuous functions}

\subsection{Interlude on natural numbers}
In classical logic, the natural numbers are complete in the sense that any
inhabited set of natural numbers possesses a minimal element. This statement
can not be proven intuitionistically -- intuitively, this is because one cannot
explicitly pinpoint the (classically existing) minimal element of an arbitrary
inhabited set. In intuitionistic logic, this principle can be salvaged in two
essentially different ways: either be strengthening the premise, or by
weakening the conclusion.
% XXX: Give sheaf-theoretic interpretation of the failure.

\begin{lemma}\label{lemma:minimum-subset-naturals}
Let~$U \subseteq \NN$ be an inhabited subset of the natural
numbers.
\begin{enumerate}
\item Assume~$U$ to be \emph{detachable}, \ie assume that for any natural
number~$n$, either~$n \in U$ or~$n \not\in U$. Then~$U$ possesses a minimal
element.
% XXX: Give sheaf-theoretic interpretation of detachability. With this
% interpretation, it should be totally clear that $U$ possesses a minimal
% element.
\item In any case,~$U$ does \emph{not not} possess a minimal element.
\end{enumerate}
\end{lemma}
\begin{proof}
\begin{enumerate}
\item By induction on the witness of inhabitation, \ie the given number~$n$ such
that~$n \in U$. Details omitted, since we will not need this statement.
\item We give a careful proof since logical subtleties matter. To simplify the
exposition, we assume that~$U$ is upward-closed, \ie that any number
larger than some element of~$U$ lies in~$U$ as well. Any subset can be closed
in this way (by considering~$\{ n \in \NN \,|\, \exists m \in U\_ n \geq m \}$)
and a minimal element of the closure will be a minimal element for~$U$ as well.

We induct on the number~$n \in U$ given by the assumption that~$U$ is
inhabited. In the case~$n = 0$ we are done since~$0$ is a minimal element
of~$U$. For the induction step~$n \to n+1$, the weak law of excluded middle
gives
\[ \neg\neg(n \in U \vee n \not\in U). \]
If we can show that~$n \in U \vee n \not\in U$ implies the conclusion, we're
done by XXX. So assume~$n \in U \vee n \not\in U$.
If~$n \in U$, then~$U$ does not not possess a minimal element by the induction
hypothesis. If~$n \not\in U$, then~$n+1$ is a minimal element (and so, in
particular,~$U$ does not not possess a minimal element): For if~$m$ is
any element of~$U$, we have~$m \geq n+1$ or~$m \leq n$. In the first case,
we're done. In the second case, it follows that~$n \in U$ because~$U$ is
upward-closed and so we obtain a contradiction. From this contradiction we can
deduce~$m \geq n+1$. \qedhere
\end{enumerate}
\end{proof}

If we want to work with a complete set of natural numbers in intuitionistic
logic, we have to construction a completion.
\begin{defn}The partially ordered set of \emph{completed natural numbers} is
the set~$\widehat{\NN}$ of all inhabited upward-closed subsets of~$\NN$, ordered by
reverse inclusion.\end{defn}
\begin{lemma}The poset of completed natural numbers is the least partially
ordered set containing~$\NN$ and possessing minima
of arbitrary inhabited subsets.\end{lemma}
\begin{proof}
The embedding $\NN \hookrightarrow \widehat\NN$ is given by
\[ n \in \NN \longmapsto {\uparrow}(n) := \{ m \in \NN \,|\, m \geq n \}. \]
If~$M \subseteq \widehat\NN$ is an inhabited subset, its minimum is
\[ \min M = \bigcup M \in \widehat\NN. \]
The proof of the universal property is left to the reader.
\end{proof}

\begin{rem}\label{rem:surjectivity-embedding}
In classical logic, the map~$\widehat\NN \to \NN,\ U \mapsto \min U$
is a well-defined isomorphism of partially ordered sets. In fact, it is the
inverse of the canonical embedding~$\NN \hookrightarrow \widehat\NN$. In
intuitionistic logic, this embedding is still injective, but it can not be
shown to be surjective: It's only the case that any element of~$\widehat\NN$
possesses \notnot a preimage (by lemma~\ref{lemma:minimum-subset-naturals}).
\end{rem}


\subsection{A geometric interpretation}
We are interested in the completed natural numbers for the following reason: A
completed natural number of the topos of sheaves on a topological space~$X$ is
the same as an upper semicontinuous function~$X \to \NN$.

\begin{lemma}Let~$X$ be a topological space. The sheaf~$\widehat\NN$ of
completed natural numbers on~$X$ is canonically isomorphic to the sheaf of upper
semicontinuous~$\NN$-valued functions on~$X$.\end{lemma}
\begin{proof}
When referring to the natural numbers in the internal language, we actually
refer to the constant sheaf~$\ul{\NN}$ on~$X$. (This is because the
sheaf~$\ul{\NN}$ fulfills the axioms of a natural numbers object,
cf.~\cite[section~VI.1]{moerdijk-maclane:sheaves-logic}.) Recall that its sections on an
open subset~$U \subseteq X$ are continuous functions~$U \to \NN$, where~$\NN$
is equipped with the discrete topology.

Therefore, a section of~$\widehat\NN$ on an open subset~$U \subseteq X$ is
given by a subsheaf~$\A \hookrightarrow \ul{\NN}|_U$ such that
\[ U \models \exists n\?\ul{\NN}\_ n \in \A
  \quad\text{and}\quad
  U \models \forall n,m\?\ul{\NN}\_ n \geq m \wedge n \in \A \Rightarrow m \in
  \A. \]
Since these conditions are geometric, they are satisfied if and only if any
stalk~$\A_x$ is an inhabited upward-closed subset of~$\ul{\NN}_x \cong \NN$.
The association
\[ x \in X \longmapsto \min\{ n \in \NN \,|\, n \in \A_x \} \]
thus defines a map~$X \to \NN$. This map is indeed upper semicontinuous, since
if~$n \in \A_x$, there exists an open neighbourhood~$V$ of~$x$ such that the constant
function with value~$n$ is an element of~$\Gamma(V,\A)$ and therefore~$n \in
\A_y$ for all~$y \in V$.

Conversely, let~$\alpha : U \to \NN$ be a upper semi-continous function. Then
\[ \text{$V \subseteq U$ open} \longmapsto \{ f : V \to \NN \,|\, \text{$f$
continuous,\ $f \geq \alpha$ on~$V$} \} \]
is a subobject of~$\ul{\NN}|_U$ which internally is inhabited and upward-closed.
Further details are left to the reader.
\end{proof}

Under the correspondence given by the lemma, locally \emph{constant}
functions map exactly to the (image of the) \emph{ordinary} internal natural numbers
(in the completed natural numbers).

\begin{rem}In a similar vein, the sheaf given by the internal construction of
the set of \emph{all} upward-closed subsets of the natural numbers (not
only the inhabited ones) is canonically isomorphic to the sheaf of
upper semicontinuous functions with values in~$\NN \cup \{ +\infty
\}$.\end{rem}


\subsection{The upper semicontinuous rank function}
Recall that the rank of an~$\O_X$-module~$\F$ on a scheme~$X$ (or
locally ringed space) at a point~$x \in X$ is defined as the~$k(x)$-dimension
of the vector space~$\F_x \otimes_{\O_{X,x}} k(x)$. If we assume that~$\F$ is
of finite type around~$x$, this dimension is finite and equals the minimal
number of elements needed to generate~$\F_x$ as an~$\O_{X,x}$-module (by
Nakayama's lemma).

In the internal language, we can define an element of~$\widehat\NN$ by
\[ \rank\F := \min\{ n \in \NN \,|\, \speak{there is a gen. family
for~$\F$ consisting of~$n$ elements} \} \in \widehat\NN. \]
If~$\F$ is locally finitely free, it will be a finitely free module from the
internal point of view and the rank defined in this way will be an
actual natural number (see below); but in general, the rank is really an element of the
completion.

\begin{prop}
Let~$\F$ be an~$\O_X$-module of finite type on a scheme~$X$ (or locally ringed
space). Under the correspondence given by the previous lemma, the internally
defined rank maps to the rank function of~$\F$.
\end{prop}
\begin{proof}
We have to show that for any point~$x \in X$ and natural number~$n$, there
exists a generating family for~$\F_x$ consisting of~$n$
elements if and only if there exists an open neighbourhood~$U$ of~$x$ such that
\[ U \models \speak{there exists a generating family
for~$\F$ consisting of~$n$ elements}. \]
The ``if'' direction is obvious. For the ``only if'' direction, consider
(liftings to local sections of a)
generating family~$s_1,\ldots,s_n$ of~$\F_x$. Since~$\F$ is of finite type,
there also exist sections~$t_1,\ldots,t_m$ on some neighbourhood~$V$ of~$x$ which
generate any stalk~$\F_y$, $y \in V$. Since the~$t_i$ can be expressed as a
linear combination of the~$s_j$ in~$\F_x$, the same is true on some open
neighbourhood~$U \subseteq V$ of~$x$. On this neighbourhood, the~$s_j$ generate
any stalk~$\F_y$, $y \in U$, so by geometricity we have
\[ U \models \speak{$s_1,\ldots,s_n$ generate~$\F$}. \qedhere \]
\end{proof}
\begin{rem}Once we understand when properties holding at a stalk spread to a
neighbourhood, we will be able to give a simpler proof of the proposition (see
lemma~\ref{lemma:gen-family-n}).\end{rem}

\begin{lemma}Let~$X$ be a locally ringed space. Let~$\F$ be an~$\O_X$-module of
finite type. If~$\F$ is locally finitely free, its rank function is locally
constant. The converse holds if~$X$ is a reduced scheme.
\end{lemma}
\begin{proof}The rank function is locally constant if and only if internally,
the rank of~$\F$ is an actual natural number. Furthermore, if~$X$ is a reduced
scheme, the structure sheaf fulfills an appropriate field condition
(corollary~\ref{cor:field-reduced}). Therefore it suffices to give a
proof of the following fact of intuitionistic linear algebra: Let~$R$ be a
local ring. Let~$M$ be a finitely generated~$R$-module. If~$M$ is finitely
free, its rank is an actual natural number. The converse holds if~$R$ fulfills
the field condition that any element which is not invertible is zero.

So assume that such a module~$M$ is finitely free. Then it is isomorphic
to~$R^n$ for some actual natural number~$n$; by the internal proof in
lemma~\ref{lemma:kernel-of-epi-fingen}, the rank of~$M$ is therefore this
number~$n$ (for any surjection~$R^m \twoheadrightarrow R^n$ it holds that~$m
\geq n$).

Conversely, assume that the rank of~$M$ is an actual natural number. Then
there exists a minimal generating family~$x_1,\ldots,x_n\?M$. This family is
linearly independent (and thus a basis, demonstrating that~$M$ is finitely
free): Let~$\sum_i a_i x_i = 0$ with~$a_i\?\R$. If any~$a_i$ were
invertible, the family~$x_1,\ldots,x_{i-1},x_{i+1},\ldots,x_n$ would too
generate~$M$, contradicting the minimality. So each~$a_i$ is not invertible.
Since~$R$ fulfills the appropriate field condition, each~$a_i$ is zero.
\end{proof}

\begin{lemma}\label{lemma:locally-free-dense}
Let~$X$ be a reduced scheme. Let~$\F$ be an~$\O_X$-module of
finite type. Then~$\F$ is locally free on a dense open subset.\end{lemma}
\begin{proof}Since ``dense open'' translates to ``not not'' in the internal
language (proposition~\ref{prop:modops-kripke}), it suffices to give an
intuitionistic proof of the following fact: Let~$R$ be a local ring which fulfills an
appropriate field condition. Let~$M$ be a finitely generated~$R$-module.
Then~$R$ is \notnot finitely free.

By remark~\ref{rem:surjectivity-embedding}, the rank of such a module~$M$ is
\notnot an actual natural number. By the last part of the
previous proof, it thus follows that~$M$ is \notnot finitely free.
\end{proof}

\begin{rem}Note that besides basics on natural numbers in an intuitionistic
setting and some dictionary terms (``reduced'', ``locally finitely free'',
``finite type``, ``dense open''), this proof does not depend on any further
tools. In particular, Nakayama's lemma and facts about semicontinuous functions
do not enter. For the (more complex) standard proof of this fact, see for
instance~\cite{vakil:foag}, where the claim is dubbed an ``important hard
exercise'' (exercise~13.7.K).\end{rem}


\section{Modalities}

\subsection{Basics on truth values and modal operators}

\begin{defn}The \emph{set of truth values~$\Omega$} is the powerset of the
singleton set~$1 := \{\star\}$, where~$\star$ is a formal symbol.\end{defn}

In classical logic, any subset of~$\{\star\}$ is either empty or inhabited, so
that~$\Omega$ contains exactly two elements, the empty set (``false'')
and~$\{\star\}$ (``true''). But
in intuitionistic logic, this can not be shown; indeed, if we interpret the
definition in the topos of sheaves on a space~$X$, we obtain a sheaf~$\Omega$
with
\[ \text{$U \subseteq X$ open} \longmapsto \Gamma(U,\Omega) = \{ V \subseteq U \,|\, \text{$V$
open} \}. \]
(This is because by definition of~$\Omega$ as the power object of the terminal
sheaf~$1$, sections of~$\Omega$ on an open subset~$U$ correspond to
subsheaves~$\F \hookrightarrow 1|_U$, and those are given by the greatest open
subset~$V \subseteq U$ such that~$\Gamma(V,\F)$ is inhabited.)

The \emph{truth value} of a formula~$\varphi$ is by definition the subset
$\{ x \in 1 \,|\, \varphi \} \in \Omega$, where~``$x$'' is a fresh variable not
appearing in~$\varphi$. This subset is inhabited if and only
if~$\varphi$ holds and is empty if and only if~$\neg\varphi$ holds.
Conversely, we can associate to a subset~$F \subseteq 1$ the
formula~$\speak{$F$ is inhabited}$.
% XXX: "formula" is not the correct term here.

Under this correspondence of formulas with truth values, logical operations
like~$\wedge$ and~$\vee$ map to set-theoretic operations like~$\cap$ and~$\cup$
-- for instance, we have
\[ \{ x \in 1 \,|\, \varphi \} \cap \{ x \in 1 \,|\, \psi \} =
  \{ x \in 1 \,|\, \varphi \wedge \psi \}. \]
This justifies a certain abuse of notation: We will sometimes treat elements
of~$\Omega$ as propositions and use logical instead of set-theoretic
connectives. In particular, if~$\varphi$ and~$\psi$ are elements of~$\Omega$,
we will write~``$\varphi \Rightarrow \psi$'' to mean~$\varphi \subseteq \psi$;
``$\bot$'' to mean~$\emptyset$; and~``$\top$'' to mean~$1$.

\begin{defn}A \emph{modal operator} is a map~$\Box : \Omega \to \Omega$ such
that for all~$\varphi, \psi \in \Omega$,
\begin{enumerate}
\item $\varphi \Longrightarrow \Box\varphi$,
\item $\Box\Box\varphi \Longrightarrow \Box\varphi$,
\item $\Box(\varphi \wedge \psi) \Longleftrightarrow \Box\varphi \wedge \Box\psi$.
\end{enumerate}
\end{defn}

The intuition is that~$\Box\varphi$ is a certain weakening of~$\varphi$, where
the precise meaning of ``weaker'' depends on the modal operator. By the second
axiom, weakening twice is the same as weakening once.

In classical logic, where~$\Omega = \{ \bot, \top \}$, there are only two modal
operators: the identity function and the constant function with value~$\top$.
Both of these are not very interesting: The identity operator does not weaken
propositions at all, while the constant operator weakens every proposition to
the trivial statement~$\top$.

In intuitionistic logic, there can potentially exist further modal operators.
For applications to algebraic geometry, the following four operators will have
a clear geometric meaning and be of particular importance:
\begin{enumerate}
\item $\Box\varphi :\equiv (\alpha \Rightarrow \varphi)$, where~$\alpha$ is a
fixed proposition.
\item $\Box\varphi :\equiv (\varphi \vee \alpha)$, where~$\alpha$ is a
fixed proposition.
\item $\Box\varphi :\equiv \neg\neg\varphi$ (the \emph{double negation
modality}).
\item $\Box\varphi :\equiv ((\varphi \Rightarrow \alpha) \Rightarrow \alpha)$,
where~$\alpha$ is a fixed proposition.
\end{enumerate}

\begin{lemma}Any modal operator~$\Box$ is monotonic, \ie if~$\varphi
\Rightarrow \psi$, then~$\Box\varphi \Rightarrow \Box\psi$. Furthermore, there
holds a modus ponens rule: If~$\Box\varphi$ holds, and~$\varphi$
implies~$\Box\psi$, then~$\Box\psi$ holds as well.\end{lemma}
\begin{proof}Assume~$\varphi \Rightarrow \psi$. This is equivalent to
supposing~$\varphi \wedge \psi \Leftrightarrow \varphi$. We are to show
that~$\Box\varphi \Rightarrow \Box\psi$, \ie that~$\Box\varphi \wedge
\Box\psi \Leftrightarrow \Box\varphi$. The statement follows since by the third
axiom on a modal operator, we have~$\Box\varphi \wedge \Box\psi \Leftrightarrow
\Box(\varphi \wedge \psi)$.

For the second statement, consider that if~$\varphi \Rightarrow \Box\psi$, by
monotonicity and the second axiom on a modal operator it follows
that~$\Box\varphi \Rightarrow \Box\Box\psi \Rightarrow \Box\psi$.
\end{proof}

The modus ponens rule justifies the following proof scheme: When showing
that a boxed statement~$\Box\psi$ holds given that a further boxed
statement~$\Box\varphi$ holds, we may assume that indeed~$\varphi$ holds.


\subsection{Geometric meaning} Let~$X$ be a topological space. As discussed
above, an open subset~$U \subseteq X$ defines an internal truth value (a global
section of the sheaf~$\Omega$) also
denoted by~``$U$'' such that
\[ V \models U \quad\Longleftrightarrow\quad V \subseteq U \]
for any open subset~$V \subseteq X$. (Shortcutting the various intermediate
steps, this can also be taken as a definition of~``$V \models U$''.)
If~$A \subseteq X$ is a closed subset, there is thus an internal truth
value~$A^c$ corresponding to the open subset~$A^c = X \setminus A$. If~$x \in
X$ is a point, we define~``$\notat{x}$'' to denote the truth value
corresponding to~$\Int(X \setminus \{x\})$, such that
\[ V \models \notat{x} \quad\Longleftrightarrow\quad V \subseteq \Int(X
\setminus \{ x \}) \quad\Longleftrightarrow\quad x \not\in V. \]

\begin{prop}\label{prop:modops-kripke}
Let~$U \subseteq X$ be a fixed open and~$A \subseteq X$ be a fixed
closed subset. Let~$x \in X$. Then, for any open subset~$V \subseteq X$, it
holds that:
\[ \renewcommand{\arraystretch}{1.3}\begin{array}{@{}lcl@{}}
  V \models (U \Rightarrow \varphi) &\Longleftrightarrow&
    V \cap U \models \varphi. \\[0.3em]
  V \models (\varphi \vee A^c) &\Longleftrightarrow&
    \textnormal{there is an open subset~$W \subseteq V$} \\
  && \quad\quad \textnormal{containing~$A \cap V$ such that $W \models \varphi$.} \\[0.3em]
  V \models \neg\neg\varphi &\Longleftrightarrow&
    \textnormal{there is a dense open subset~$W \subseteq V$ s.\,th.\@ $W \models
    \varphi$.} \\[0.3em]
  V \models ((\varphi \Rightarrow \notat{x}) \Rightarrow \notat{x}) &\Longleftrightarrow&
    \textnormal{$x \not\in V$ or there is an open neighbourhood~$W \subseteq V$} \\
  && \quad\quad \textnormal{of~$x$ such that $W \models \varphi$.}
\end{array} \]
\end{prop}
\begin{proof}
\begin{enumerate}
\item Omitted.

\item Let~$V \models \varphi \vee A^c$. Then there exists an open covering~$V =
\bigcup_i V_i$ such that for each~$i$, $V_i \models \varphi$ or $V_i \subseteq
A^c$. Let~$W \subseteq V$ be the union of those~$V_i$ such that~$V_i \models \varphi$.
Then~$W \models \varphi$ by the locality of the internal language and~$A \cap V
\subseteq W$.

Conversely, let~$W \subseteq V$ be an open subset containing~$A \cap V$ such
that~$W \models \varphi$. Then~$V = W \cup (V \cap A^c)$ is an open covering
attesting~$V \models \varphi \vee A^c$.

\item For the ``only if'' direction, let~$W \subseteq V$ be the largest
open subset on which~$\varphi$ holds, \ie the union of all open subsets
of~$V$ on which~$\varphi$ holds. For the ``if'' direction, we may assume that
the given~$W$ is also the largest open subset on which~$\varphi$ holds (by
enlarging~$W$ if necessary). The claim then follows by the following chain of
equivalences:
\begin{align*}
  &\ V \models \neg\neg\varphi \\
  \Longleftrightarrow&\ \forall \text{$Y \subseteq V$ open}\_
    \left[\forall \text{$Z \subseteq Y$ open}\_ Z \models \varphi \Rightarrow Z
    = \emptyset\right] \Longrightarrow Y = \emptyset \\
  \Longleftrightarrow&\ \forall \text{$Y \subseteq V$ open}\_
    \left[\forall \text{$Z \subseteq Y$ open}\_ Z \subseteq W \Rightarrow Z
    = \emptyset\right] \Longrightarrow Y = \emptyset \\
  \Longleftrightarrow&\ \forall \text{$Y \subseteq V$ open}\_
    Y \cap W = \emptyset \Longrightarrow Y = \emptyset \\
  \Longleftrightarrow&\ \text{$W$ is dense in~$V$.}
\end{align*}

\item Straightforward, since the interpretation of the internal statement with
the Kripke--Joyal semantics is
\[ \forall \text{$Y \subseteq V$ open}\_
  \left[\forall \text{$Z \subseteq Y$ open}\_
    Z \models \varphi \Rightarrow x \not\in Z\right] \Longrightarrow x \not\in
    Y. \qedhere \]
\end{enumerate}
\end{proof}


\subsection{The subspace associated to a modal operator}
Any modal operator~$\Box : \Omega \to \Omega$ in the sheaf topos of~$X$ induces
on global sections a map
\[ j : \Open(X) \to \Open(X), \]
where~$\Open(X) = \Gamma(X,\Omega)$ is the set of open subsets of~$X$. By the
axioms on a modal operator, the map~$j$ fulfills similar axioms: For any open
subsets~$U, V \subseteq X$,
\begin{enumerate}
\item $U \subseteq j(U)$,
\item $j(j(U)) \subseteq j(U)$,
\item $j(U \cap V) = j(U) \cap j(V)$.
\end{enumerate}
Such a map is called a \emph{nucleus} on~$\Open(X)$. Table~\ref{table:nuclei}
lists the nuclei associated to the four modal operators
of proposition~\ref{prop:modops-kripke}.

\begin{table}
  \centering
  \setlength{\extrarowheight}{0.4em}
  \begin{tabular}{llll}
    Modal operator & associated nucleus &
      $j(V) = X$ iff \ldots &
      subspace \\\hline
    $\Box\varphi :\equiv (U \Rightarrow \varphi)$ &
      $j(V) = \Int(U^c \cup V)$ & $U \subseteq V$ & $U$ \\
    $\Box\varphi :\equiv (\varphi \vee A^c)$ &
      $j(V) = V \cup A^c$ & $A \subseteq V$ & $A$ \\
    $\Box\varphi :\equiv \neg\neg\varphi$ &
      $j(V) = \Int(\Clos(V))$ & $V$ is dense in $X$ & (see text) \\
    $\Box\varphi :\equiv ((\varphi \Rightarrow \notat{x}) \Rightarrow \notat{x})$ &
%      $\Int(\Clos(V \cap \Clos\{x\}) \cup (X \setminus \Clos\{x\}))$ &
      $j(V) = \begin{cases}
        X \setminus \Clos\{x\}, & \text{if $x \not\in V$} \\
        X, & \text{if $x \in V$}
      \end{cases}$ &
      $x \in V$ & $\{x\}$
  \end{tabular}

  \caption{\label{table:nuclei}List of important modal operators and their
  associated nuclei (notation as in proposition~\ref{prop:modops-kripke}).}
\end{table}

Any nucleus~$j$ defines a subspace~$X_j$ of~$X$, with a small caveat: In
general, the subspace~$X_j$ can not be realized as a topological subspace, but
only as a so-called \emph{sublocale}; the notion of a locale is a slight
generalization of the notion of a topological space, in which an underlying set
of points is not part of the definition. Instead, a locale is simply given by a
lattice of general \emph{opens} -- these may, but do not necessarily have to,
be sets of points. Sheaf theory carries over to locales essentially unchanged,
since the notions of presheaves and sheaves only need opens and coverings.
% XXX: give introductory reference

\begin{defn}\label{defn:subspace-by-nucleus}Let~$j$ be a nucleus on~$\Open(X)$.
Then the sublocale~$X_j$ of~$X$ is given by the lattice of opens
$\Open(X_j) := \{ U \in \Open(X) \,|\, j(U) = U \}$.
\end{defn}
If~$j$ is induced by a modal operator~$\Box$, we also write~``$X_\Box$''
for~$X_j$. In three of the four cases listed in table~\ref{table:nuclei}, the
sublocale~$X_\Box$ can indeed be realized as a topological subspace. The only
exception is the sublocale~$X_{\neg\neg}$ associated to the double negation
modality. It can be also be described as the \emph{smallest dense sublocale}
of~$X$; this is obviously a true locale-theoretic notion, since a topological
space does not have (in general) a smallest dense topological subspace
(consider~$\RR$ and its dense subsets~$\QQ$ and~$\RR \setminus \QQ$).
% XXX: word "true"
% XXX: give introductory references, i.e. Johnstone

The inclusion~$i : X_j \hookrightarrow X$ can not in general be described on the
level of points, since~$X_j$ might not be realizable as a topological subspace.
But for sheaf-theoretic purposes, it suffices to describe~$i$ on the level of
opens. This is done as follows:
\[ i^{-1} : \Open(X) \lra \Open(X_j), \quad U \longmapsto j(U). \]
Thus we can relate the toposes of sheaves on~$X_j$ and~$X$ by the usual
pullback and pushforward functors.
\begin{align*}
  i^{-1} \F &= \text{sheafification of $(U \mapsto \colim_{U \preceq i^{-1}V} \Gamma(V,\F))$} \\
  i_* \G &= (U \mapsto \Gamma(i^{-1}U, \G) = \Gamma(j(U), \G))
\end{align*}
As familiar from honest topological subspace inclusions, the pushforward
functor~$i_* : \Sh(X_j) \to \Sh(X)$ is fully faithful and the composition~$i^{-1}
\circ i_* : \Sh(X_j) \to \Sh(X_j)$ is (canonically isomorphic to) the identity.


\subsection{Internal sheaves and sheafification}\label{sect:internal-sheaves}
It turns out that the image of
the pushforward functor~$i_* : \Sh(X_\Box) \to \Sh(X)$, where~$\Box$ is a modal
operator in~$\Sh(X)$, can be explicitly described: Namely, it consists exactly
of those sheaves which from the internal point of view
are so-called~\emph{$\Box$-sheaves}, a notion explained below.

Furthermore, if we identify~$\Sh(X_\Box)$ with its image in~$\Sh(X)$, the
pullback functor is given by an internal sheafification process with respect to
the modality~$\Box$. Thus the external situation of pushforward/pullback
translates to forget/sheafify. This broadens the scope of the internal
language: It can not only be used to talk about sheaves on~$X$ in a simple,
element-based language, but also to talk about sheaves on arbitrary subspaces
of~$X$.

To describe the notion of~$\Box$-sheaves and related ones, we switch to the internal
perspective and thus forget~$X$; we are simply given a model operator~$\Box :
\Omega \to \Omega$ and have to take care that our proofs are intuitionistic. A
reference for the material in this subsection is a preprint by Fer-Jan de
Vries~\cite{vries:sheafification}\footnote{Note that on page~5 of that
preprint, there is a slight typing error: Fact~2.1(i) gives the
characterization of~$j$-closedness, not~$j$-denseness. The correct
characterization of~$j$-denseness in that context is~$\forall b \in B\_ j(b \in
A)$.}.

\begin{defn}\label{defn:box-sheaves}
A set~$F$ is \emph{$\Box$-separated} if and only if
\[ \forall x,y\?F\_ \Box(x = y) \Longrightarrow x = y. \]
A set~$F$ is a \emph{$\Box$-sheaf} if and only if it is~$\Box$-separated and
\[ \forall S \subseteq F\_
  \Box(\speak{$S$ is a singleton}) \Longrightarrow
  \exists x\?F\_ \Box(x \in S). \]
\end{defn}

The two conditions can be combined: A set~$F$ is a~$\Box$-sheaf if and only if
\[ \forall S \subseteq F\_
  \Box(\speak{$S$ is a singleton}) \Longrightarrow
  \exists! x\?F\_ \Box(x \in S). \]

\begin{defn}The \emph{plus construction} of a set~$F$ with respect to~$\Box$ is the set
\[ F^+ := \{ S \subseteq F \,|\, \Box(\speak{$S$ is a singleton}) \}/{\sim},
\]
where the equivalence relation is defined by~$S \sim T :\Leftrightarrow
\Box(S = T)$. There is a canonical map~$F \to F^+$ given by~$x \mapsto
[\{x\}]$. The \emph{$\Box$-sheafi\-fi\-ca\-tion} of a set~$F$ is the
set~$F^{++}$.
\end{defn}

If~$F$ is~$\Box$-separated, then for any subset~$S \subseteq F$ it holds
that
\[ \Box(\speak{$S$ is a singleton}) \quad\Longleftrightarrow\quad
  \speak{$S$ is a subsingleton} \wedge \Box(\speak{$S$ is inhabited}). \]
% XXX: introduce "singleton", "subsingleton"

\begin{rem}The topos of \emph{pre}sheaves on a topological space~$X$ admits an
internal language as well~\cite[???]{moerdijk-maclane:sheaves-logic}. In it, there
exists a modal operator~$\Box$ reflecting the topology of~$X$. A presheaf is separated
in the usual sense if, from the internal perspective of~$\PSh(X)$, it
is~$\Box$-separated; and it is a sheaf if, from the internal perspective, it
is a~$\Box$-sheaf. Furthermore, the~$\Box$-sheafification of a presheaf
(considered as a set from the internal perspective) coincides with the usual
sheafification.\end{rem}

\begin{ex}\label{ex:special-sets-sheaves}
Any singleton set is a~$\Box$-sheaf. The empty set is
always~$\Box$-separated (trivially) and is a~$\Box$-sheaf if and only
if~$\Box\bot \Rightarrow \bot$.\end{ex}

\begin{lemma}For any set~$F$, it holds that: \begin{enumerate}
\item $F^+$ is~$\Box$-separated.
\item The canonical map~$F \to F^+$ is injective if and only if~$F$
is~$\Box$-separated.
\item If~$F$ is~$\Box$-separated, $F^+$ is a~$\Box$-sheaf.
\item If~$F$ is a~$\Box$-sheaf, the canonical map~$F \to F^+$ is bijective.
\end{enumerate}
Let``$\Sh_\Box(\Set)$'' denote the full subcategory of~$\Set$ consisting of
the~$\Box$-sheaves. Then it holds that:
\begin{enumerate}
\addtocounter{enumi}{4}
\item The functor~$(\placeholder)^+ : \Set \to \Set$ is left exact.
\item The functor~$(\placeholder)^{++} : \Set \to \Sh_\Box(\Set)$ is left exact and left
adjoint to the forgetful functor~$\Sh_\Box(\Set) \to \Set,\ F \mapsto F$.
\end{enumerate}\end{lemma}
\begin{proof}These are all straightforward, and it fact simpler than their
classical counterparts, since there are no colimit constructions which would have to
be dealt with.
\end{proof}

\begin{rem}As is to be expected from the familiar inclusion of sheaves in
presheaves on topological spaces, the forgetful functor~$\Sh_\Box(\Set) \to \Set$
does not in general preserve colimits. It is instructive to see why
epimorphisms in~$\Sh_\Box(\Set)$ need not be epimorphisms in~$\Set$: A map~$f:A
\to B$ between~$\Box$-sheaves is an epimorphism in~$\Sh_\Box(\Set)$ if and only
if
\[ \forall y\?B\_ \Box(\exists x\?X\_ f(x) = y), \]
\ie preimages do not need to exist, but merely need to~``$\Box$-exist''.
(Using results about the~$\Box$-translation, to be introduced below, this
characterization will be obvious.) This condition is intuitionistically weaker
that the condition that~$f$ is an epimorphism in~$\Set$, \ie that~$f$ is
surjective.\end{rem}


\subsection{Sheaves for the double negation modality}
% XXX: introductory words

\begin{prop}Let~$X$ be a topological space. Let~$\F$ be a sheaf on~$X$. Then:
\begin{enumerate}
\item $\F$ is~$\neg\neg$-separated if and only if it is sufficient for local
sections to be equal to agree on a dense open subset of their common domain.
\item $\F$ is a~$\neg\neg$-sheaf if and only if it is~$\neg\neg$-separated and
for any open subset~$U \subseteq X$ and any open subset~$V \subseteq U$ dense
in~$U$, any~$V$-section of~$\F$ extends to a~$U$-section of~$\F$.
\item If~$\F$ is~$\neg\neg$-separated, the sections of $\F^+$ on an open
subset~$U \subseteq X$ can be described as pairs~$(V,s)$, where~$V$ is a dense
open subset of~$U$ and~$s$ is a section of~$\F$ on~$V$. Two such pairs~$(V,s),
(V',s')$ give the same element in~$\Gamma(U,\F^+)$ if and only if~$s$ and~$s'$
agree on~$V \cap V'$.
\end{enumerate}
\end{prop}
\begin{proof}
The first statement is obvious from the definition of~$\neg\neg$-separatedness
(definition~\ref{defn:box-sheaves} for~$\Box = \neg\neg$) and the geometric
interpretation of double negation (proposition~\ref{prop:modops-kripke}).

For the second statement, it suffices to show that if~$\F$
is~$\neg\neg$-separated,~$\F$ has the extension property if and only if
\begin{multline*}
  \Sh(X) \models \forall \S \? \P(\F)\_
  \speak{$\S$ is a subsingleton} \wedge
  \neg\neg(\speak{$\S$ is inhabited}) \Longrightarrow \\
  \exists x\?\F\_ \neg\neg(x \in \S).
\end{multline*}
Note that a section~$\S \in \Gamma(U,\P(\F))$ which internally is a
subsingleton and not not inhabited is precisely a subsheaf~$\S \hookrightarrow
\F$ such that all stalks~$\S_x$, $x \in U$ are subsingletons and such that for
some dense open subset~$V \subseteq U$, the stalks~$\S_x$, $x \in V$ are
inhabited. This is precisely the datum of a section of~$\F$ defined on some
dense open subset of~$U$: Consider the gluing of the unique germs in~$\S_x$ for
those points~$x$ such that~$\S_x$ is inhabited. (Conversely, a section~$s \in
\Gamma(V,\F)$ defines a subsheaf~$\S$ by setting~$\Gamma(W,\S) := \{ s|_W \,|\,
W \subseteq V \}$.)

In view of this explicit description and the observation that the existence in
question~(``$\exists x\?\F\_ \neg\neg(x \in \S)$'') is actually a question of
unique existence, the second statement follows.
% this is hard to understand

For the third statement, one can check that the presheaf on~$X$ defined by
\[ \text{$U \subseteq X$ open} \longmapsto
  \{ (V,s) \,|\, \text{$V \subseteq U$ dense open},\ s \in \Gamma(V,\F)
  \}/{\sim} \]
is in fact a sheaf (with respect to the topology of~$X$), a $\neg\neg$-sheaf
and that it fulfills the universal property of the~$\neg\neg$-sheafification
of~$\F$.
\end{proof}

% XXX: introduce notion of Box-stability


\subsection{\texorpdfstring{The~$\Box$-translation}{The □-translation}}
There is certain well-known transformation~$\varphi
\mapsto \varphi^{\neg\neg}$ on formulas, the \emph{double negation
translation}, with the following curious property: A formula~$\varphi$ is
derivable in classical logic if and only if its
translation~$\varphi^{\neg\neg}$ is derivable in intuitionistic logic. The
translation~$\varphi^{\neg\neg}$ is obtained from~$\varphi$ by putting
``$\neg\neg$'' before any subformula, \ie before any~``$\exists$''
and~``$\forall$'', around any logical connective and around any atomic
statement (``$x=y$'', ``$x \in A$'').

We will describe a slight generalization of the double negation translation,
the~$\Box$-translation for any modal operator~$\Box$.
% check http://arxiv.org/pdf/1101.5442.pdf for references to cite

\begin{defn}The~\emph{$\Box$-translation} is recursively defined as follows.
\newcommand{\optBox}{\textcolor{gray}{\Box}}
\begin{align*}
  (f = g)^\Box &:\equiv \Box(f = g) \\
  (x \in A)^\Box &:\equiv \Box(x \in A) \\
  \top^\Box &:\equiv \Box\top \quad \text{($\Leftrightarrow \top$)} \\
  \bot^\Box &:\equiv \Box\bot \\
  (\varphi \wedge \psi)^\Box &:\equiv \optBox(\varphi^\Box \wedge \psi^\Box) &
  \textstyle (\bigwedge_i \varphi_i)^\Box &:\equiv \textstyle \optBox(\bigwedge_i \varphi_i^\Box) \\
  (\varphi \vee \psi)^\Box &:\equiv \Box(\varphi^\Box \vee \psi^\Box) &
  \textstyle (\bigvee_i \varphi_i)^\Box &:\equiv \textstyle \Box(\bigvee_i \varphi_i^\Box) \\
  (\varphi \Rightarrow \psi)^\Box &:\equiv \optBox(\varphi^\Box \Rightarrow \psi^\Box) \\
  (\forall x\?X\_ \varphi)^\Box &:\equiv \optBox(\forall x\?X\_ \varphi^\Box) &
  (\forall X\_ \varphi)^\Box &:\equiv \optBox(\forall X\_ \varphi^\Box) \\
  (\exists x\?X\_ \varphi)^\Box &:\equiv \Box(\exists x\?X\_ \varphi^\Box) &
  (\exists X\_ \varphi)^\Box &:\equiv \Box(\exists X\_ \varphi^\Box)
\end{align*}
\end{defn}

\begin{lemma}\begin{enumerate}
\item Formulas in the image of the $\Box$-translation are~$\Box$-stable,
\ie for any formula~$\varphi$ it holds that
$\Box(\varphi^\Box) \Longrightarrow \varphi^\Box$.
\item In the definition of the~$\Box$-translation, one may omit the boxes
printed in gray.
\end{enumerate}\end{lemma}
\begin{proof}The first statement is obvious, since one of the axioms on a modal
operator demands that~$\Box\Box\varphi \Rightarrow \Box\varphi$ for any
formula~$\varphi$. The second statement follows by an induction on the
formula structure. By way of example, we prove the case for~``$\Rightarrow$'':
\newcommand{\withgray}{\text{$\Box$ with the gray parts}}
\newcommand{\withoutgray}{\text{$\Box$ without the gray parts}}
\begin{align*}
  &\ (\varphi \Rightarrow \psi)^\withgray \\
  \Longleftrightarrow &\ \Box(\varphi^\withgray \Rightarrow \psi^\withgray) \\
  \Longleftrightarrow &\ (\varphi^\withgray \Rightarrow \psi^\withgray) \\
  \Longleftrightarrow &\ (\varphi^\withoutgray \Rightarrow \psi^\withoutgray) \\
  \Longleftrightarrow &\ (\varphi \Rightarrow \psi)^\withoutgray
\end{align*}
The first step is by definition; the second by~$\Box$-stability
of~$\psi^\withgray$; the third by the induction hypothesis; the fourth by
definition.
\end{proof}

\begin{lemma}\label{lemma:open-stalk}
Let~$\varphi$ be a formula such that for any subformulas~$\psi$
appearing as antecedents of implications, it holds that~$\psi^\Box \Rightarrow
\Box\psi$. (In particular, this condition is satisfied if there are
no~``$\Rightarrow$'' signs in~$\varphi$.) Then $\Box\varphi \Rightarrow
\varphi^\Box$.\end{lemma}
\begin{proof}We prove this by an induction on the formula structure. All cases
except for~``$\Rightarrow$'' are obvious. For this case, assume~$\Box(\psi
\Rightarrow \chi)$; we are to show that~$(\psi^\Box \Rightarrow \chi^\Box)$.
Since this is a~$\Box$-stable statement, we can in fact assume that~$(\psi
\Rightarrow \chi)$. We then have
\[ \psi^\Box \Longrightarrow \Box\psi \Longrightarrow \Box\chi
\Longrightarrow \chi^\Box, \]
with the first step being by the requirement on antecedents, the second by the
monotonicity of~$\Box$, and the third by the induction hypothesis.
\end{proof}

\begin{lemma}\label{lemma:stalk-open}
Let~$\varphi$ be a geometric formula.
Then $\varphi^\Box \Rightarrow \Box\varphi$.\end{lemma}
\begin{proof}By induction on the formula structure. By way of example, we prove
the case for~``$\bigvee$''. So assume~$\Box(\bigvee_i \varphi_i^\Box)$; we are
to show that~$\Box(\bigvee_i \varphi_i)$. Since this is a boxed statement, we
may in fact assume~$\bigvee_i \varphi_i^\Box$, so for some index~$j$, it holds
that~$\varphi_j^\Box$. By the induction hypothesis, it follows
that~$\Box\varphi_j$. By~$\varphi_j \Rightarrow \bigvee_i \varphi_i$ and the
monotonicity of~$\Box$, it follows that that~$\Box(\bigvee_i \varphi_i)$.
\end{proof}

%\begin{ex}The failure of~$(\bigwedge_i \varphi_i)^\Box \equiv \bigwedge_i
%\varphi_i^\Box \Rightarrow \Box(\bigwedge_i \varphi_i)$ can be nicely
%illustrated with the internal language of a sheaf topos~$\Sh(X)$.
%\end{ex}

\begin{rem}In the special case that~$\Box$ is the double negation modality, the
lemma holds with slightly weaker hypotheses: Namely, implications may occur
in~$\varphi$, provided that for their antecedents~$\psi$ it holds that~$\psi
\Rightarrow \psi^\Box$.\end{rem}
% is this remark useful?

\begin{lemma}\label{lemma:stalk-open-with-hypothesis}
Let~$\varphi, \varphi', \psi$ be formulas. Assume that:
\begin{itemize}
\item The formula $\varphi'$ is geometric. [It suffices for~$(\varphi')^\Box$
to imply~$\Box\varphi'$.]
\item There is an intuitionistic proof that~$\varphi$
and~$\varphi'$ are equivalent under the (only) hypothesis~$\psi$.
\item Both~$\Box\psi$ and~$\psi^\Box$ hold.
\end{itemize}
Then $\varphi^\Box \Rightarrow \Box\varphi$.
\end{lemma}
\begin{proof}
Assume~$\varphi^\Box$. Since~$\psi^\Box$, $(\varphi \wedge \psi)^\Box$. Because
the~$\Box$-translation is sound with respect to intuitionistic logic (see~XXX),
it follows that~$(\varphi')^\Box$. As~$\varphi'$ is geometric, it follows
that~$\Box\varphi'$. Since~$\Box\psi$ holds, it follows that~$\Box\varphi$.
\end{proof}

\begin{rem}The requirement that there exists an intuitionistic proof is
stronger than merely \emph{knowing} that the equivalence holds.
\end{rem}
% XXX: move this remark to the place where soundness of the Box-translation is
% discussed


\begin{ex}\label{ex:module-zero-geometric}
Let~$M$ be an~$R$-module. Then the statement that~$M$ is zero is not
geometric: $\varphi :\equiv (\forall x\?M\_ x = 0)$. But if~$M$ is generated by some finite
family~$x_1,\ldots,x_n\?M$, then~$\varphi$ is equivalent to the
statement~$\varphi' :\equiv (x_1 = 0
\wedge \cdots \wedge x_n = 0)$ which is geometric; and there is an
intuitionistic proof of this equivalence. Since no implication signs occur
in~$\psi :\equiv \speak{$M$ is generated by~$x_1,\ldots,x_n$}$, the lemma is
applicable and shows that~$\varphi^\Box$ implies~$\Box\varphi$.\end{ex}

\begin{lemma}For the modality~$\Box$ defined by~$\Box\varphi :\equiv ((\varphi
\Rightarrow \alpha) \Rightarrow \alpha)$, where~$\alpha$ is a fixed
proposition, the~$\Box$-translation of the law of excluded middle holds.
In particular, this applies to the double negation modality~$\Box = \neg\neg$, where~$\alpha =
\bot$.\end{lemma}
\begin{proof}We are to show that~$(\varphi \vee \neg\varphi)^\Box$, \ie that
\[ (((\varphi^\Box \vee (\varphi^\Box \Rightarrow \alpha)) \Longrightarrow
\alpha) \Longrightarrow \alpha. \]
So assume that the antecedent holds. If~$\varphi^\Box$ would hold, then in
particular~$\varphi^\Box \vee (\varphi^\Box \Rightarrow \alpha)$ and thus~$\alpha$
would hold. Therefore it follows that~$(\varphi^\Box \Rightarrow \alpha)$. This
implies~$\varphi^\Box \vee (\varphi^\Box \Rightarrow \alpha)$ and
thus~$\alpha$.
\end{proof}


\subsection{Truth at stalks vs. truth on neighbourhoods}\label{sect:spreading}
We now state the crucial property of the~$\Box$-translation. Recall
that~``$X_\Box$'' denotes the sublocale of~$X$ induced by~$\Box$
(definition~\ref{defn:subspace-by-nucleus}).
\begin{thm}\label{thm:box-translation-semantically}
Let~$X$ be a topological space. Let~$\Box$ be a modal operator
in~$\Sh(X)$. Let~$\varphi$ be a formula over~$X$. Then
\[ \Sh(X) \models \varphi^\Box \quad\text{iff}\quad
  \Sh(X_\Box) \models \varphi, \]
where on the right hand side, all parameters occuring in~$\varphi$ were pulled
back to~$X_\Box$ by the inclusion~$X_\Box \hookrightarrow X$.
\end{thm}

We have not explicitly stated the Kripke--Joyal semantics for a sheaf topos
over a locale, which~$X_\Box$ is in general. The definition is exactly the same
as in the case for sheaf toposes over a topological space, only that any
mention of ``open sets'' has to be substituted by the more general ``opens''
and any mention of the union operator~``$\bigcup$'' has to be interpreted by
the supremum operator in the lattice of opens of the locale. For~$X_\Box$, this
is~$\sup U_i = j(\bigcup_i U_i)$. Before giving a proof of the theorem, we want
to discuss some of its consequences.

\begin{cor}\label{cor:spreading}
Let~$X$ be a topological space.
\begin{enumerate}
\item Let~$U \subseteq X$ be an open subset and let~$\Box\varphi :\equiv (U
\Rightarrow \varphi)$. Then
\[ \Sh(X) \models \varphi^\Box \quad\text{iff}\quad \Sh(U) \models \varphi. \]
\item Let~$A \subseteq X$ be a closed subset and let~$\Box\varphi :\equiv
(\varphi \vee A^c)$. Then
\[ \Sh(X) \models \varphi^\Box \quad\text{iff}\quad \Sh(A) \models \varphi. \]
\item Let~$x \in X$ be a point and let~$\Box\varphi :\equiv ((\varphi
\Rightarrow \notat{x}) \Rightarrow \notat{x})$. Then
\[ \Sh(X) \models \varphi^\Box \quad\text{iff}\quad \text{$\varphi$ holds
at~$x$}. \]
% change text below ("discuss the third case") if numbering of the items changes
\end{enumerate}
\end{cor}
\begin{proof}Combine theorem~\ref{thm:box-translation-semantically} and
table~\ref{table:nuclei}.\end{proof}

We want to discuss the third case of the corollary in more detail. Let~$x$ be a
point of a topological space~$X$ and let~$\varphi$ be a formula. Let~$\Box$ be
the modal operator given in the corollary. Then~$\varphi$ \emph{holds at~$x$}
if and only if, from the internal perspective of~$\Sh(X)$, the translated
formula~$\varphi^\Box$ holds; and~$\varphi$ \emph{holds on some open
neighbourhood of~$x$} if and only if, from the internal perspective, the
formula~$\Box\varphi$ holds.

Thus the question whether the truth of~$\varphi$ at the point~$x$ spreads to
some open neighbourhood can be formulated in the following way:
\begin{quote}
\emph{Does~$\varphi^\Box$ imply~$\Box\varphi$ in the internal language
of~$\Sh(X)$?}
\end{quote}

\begin{cor}\label{cor:geometric-spreading}
Let~$X$ be a topological space. Let~$\varphi$ be a formula.
If~$\varphi$ is geometric, truth of~$\varphi$ at a point~$x \in X$ implies
truth of~$\varphi$ on some open neighbourhood of~$x$, and vice versa.\end{cor}
\begin{proof}By the purely logical lemmas of the previous section, it holds
that~$\varphi^\Box \Leftrightarrow \Box\varphi$.
\end{proof}

% discuss the case x = generic point, and discuss that for us, absence of a
% generic point is no problem

\begin{ex}Let~$X$ be a scheme (or a ringed space). Since the condition for a
function~$f\?\O_X$ to be nilpotent is geometric (it is~$\bigvee_{n \geq 0} f^n
= 0$), nilpotency of~$f$ at a point is equivalent to nilpotency on some open
neighbourhood.\end{ex}

\begin{proof}[Proof of theorem~\ref{thm:box-translation-semantically}]
A fancy proof goes as follows. First, one shows intuitionistically that for a
modal operator~$\Box$ in~$\Set$, it holds that
\[ \Set \models \varphi^\Box \quad\Longleftrightarrow\quad
  \Sh_\Box(\Set) \models \varphi. \]
This can be done by an easy and nontechnical induction on the structure of
formulas~$\varphi$. Then one interprets this result in the sheaf topos~$\Sh(X)$:
\begin{align*}
  &\ \Sh(X) \models \varphi^\Box \\
  \Longleftrightarrow&\ 
  \Sh(X) \models \speak{$\Set \models \varphi^\Box$} &&\text{by idempotency}\\
  \Longleftrightarrow&\ 
  \Sh(X) \models \speak{$\Sh_\Box(\Set) \models \varphi$} &&\text{by the first step} \\
  \Longleftrightarrow&\ 
  \Sh_\Box(\Sh(X)) \models \varphi &&\text{by idempotency} \\
  \Longleftrightarrow&\ 
  \Sh(X_\Box) \models \varphi &&\text{since~$\Sh_\Box(\Sh(X)) \simeq
  \Sh(X_\Box)$}
\end{align*}
By \emph{idempotency}, we mean that internally employing the Kripke--Joyal
semantics to interpret doubly-internal statements is the same as using the
Kripke--Joyal semantics once. However, we do not want to discuss this here any further;
some details can be found in the original article on the stack
semantics~\cite[lemma~7.20]{shulman:stack}, but the lemma given there is not
general enough to justify the second use of idempotency above. For this, one
needs to extend the stack semantics to support internal statements about
locally internal categories like~$\Sh(X_\Box) \hookrightarrow \Sh(X)$ (which
then look like locally small categories from the internal point of view). This
is worthwhile for other reasons too, but shall not be pursued in these notes.

Therefore, we give a more explicit proof. By induction, we are going to prove
that for any open subset~$U \subseteq X$ and any formula~$\varphi$ over~$U$, it
holds that
\[ U \models_X \varphi^\Box \quad\Longleftrightarrow\quad j(U) \models_{X_\Box}
\varphi, \]
where the internal statements are to be interpreted by the Kripke--Joyal
semantics of~$X$ and~$X_\Box$ respectively and~$j$ is the nucleus associated
to~$\Box$. We may assume that any sheaves occuring in~$\varphi$ as domains of
quantifications are in fact~$\Box$-sheaves; we justify this in a separate lemma
below.

The cases~$\varphi \equiv \top$,~$\varphi \equiv (\psi \wedge \chi)$,
and~$\varphi \equiv \bigwedge_i \psi_i$ are trivial. For~$\varphi \equiv \bot$,
the claim is that~$U \models_X \Box\varphi$ if and only if~$j(U)
\models_{X_\Box} \bot$. The former means~$U \subseteq j(\emptyset)$ and the
latter means~$j(U) = \sup \emptyset = j(\emptyset)$, so the claim follows from
the first two axioms on a nucleus.
\end{proof}

\begin{lemma}Let~$\Box$ be a modal operator. Let~$\varphi$ be a formula.
Let~$\psi :\equiv \varphi^\Box$ be the~$\Box$-translation of~$\varphi$.
Let~$\psi'$ be the formula obtained from~$\psi$ by substituting any occuring
domain of quantification by its~$\Box$-sheafification. Then~$\varphi$
and~$\varphi'$ are equivalent.
\end{lemma}
\begin{proof}
For any formula~$\varphi$, we denote by~``$\varphi^\boxplus$'' the result of
first applying the~$\Box$-translation to~$\varphi$ and then substituting any
set~$F$ occuring in~$\varphi$ as a domain of quantification by the plus
construction~$F^+$. Recall that for any such~$F$ there is a canonical map~$F
\to F^+, x \mapsto [\{x\}]$. We are going to show by induction that for any
formula~$\varphi(x_1,\ldots,x_n)$ in which elements~$x_i\?F_i$ may occur as
terms, it holds that~$\varphi^\Box(x_1,\ldots,x_n)$ is equivalent
to~$\varphi^\boxplus([\{x_1\}],\ldots,[\{x_n\}])$. This suffices to prove the
lemma.

The cases for
\[ \top \quad \bot \quad \wedge \quad \bigwedge \quad \vee \quad \bigvee \quad \implies \]
are trivial. The cases for unbounded~``$\forall$'' and~``$\exists$'' are
trivial as well. The case for~``$=$'' is slightly more interesting; let~$\varphi(x,y)
\equiv (x = y)$. Then we are to show that~$\varphi^\Box(x,y) \equiv \Box(x=y)$
(equality in some set~$F$) is equivalent to~$\varphi^\boxplus([\{x\}],[\{y\}])
\equiv \Box([\{x\}] = [\{y\}])$ (equality in~$F^+$). This is obvious. The case
for~``$\in$'' is similar.

Let~$\varphi \equiv (\exists x\?F\_ \psi(x))$, where we have dropped further
variables occuring in~$\psi$ for simplicity. Then we are to show
that~$\varphi^\Box \equiv \Box(\exists x\?F\_ \psi^\Box(x))$ is equivalent
to~$\varphi^\boxplus \equiv \Box(\exists \bar x\?F^+\_ \psi^\boxplus(\bar x))$.
The ``only if'' direction is trivial (set~$\bar x := [\{x\}]$). For the ``if''
direction, we may assume that there exists~$\bar x\?F^+$ such
that~$\psi^\boxplus(\bar x)$ since we want to prove a boxed statement. By
definition of the plus construction, it holds that~$\Box(\speak{$\bar x$ is a
singleton})$. So, again since we want to prove a boxed statement, we may assume
that~$\bar x$ is actually a singleton. Therefore there exists~$x\?F$ such
that~$\bar x = [\{x\}]$ and that~$\psi^\boxplus([\{x\})$ holds. By the induction
hypothesis, it follows that~$\psi^\Box(x)$. From this the claim follows.

The case for~``$\forall$'' is similar.
\end{proof}


\subsection{Internal proofs of common lemmas}

\begin{lemma}\label{lemma:gen-family-n}
Let~$X$ be a scheme (or ringed space). Let~$\F$ be an~$\O_X$-module
of finite type. Let~$x \in X$ be a point. Let~$n$ be a natural number. Then the
following statements are equivalent:
\begin{enumerate}
\item There exists a generating family for~$\F_x$ consisting of~$n$ elements.
\item There exists an open neighbourhood~$U$ of~$x$ such that
\[ U \models \speak{there exists a generating family for~$\F$ consisting of~$n$
elements}. \]
\end{enumerate}
\end{lemma}
\begin{proof}Using the modal operator~$\Box$ defined by~$\Box\varphi :\equiv
((\varphi \Rightarrow \notat{x}) \Rightarrow \notat{x})$, we have to show that
the following statements in the internal language are equivalent:
\begin{enumerate}
\item $\speak{there exists a generating family
for~$\F$ consisting of~$n$ elements}^\Box$.
\item $\Box(\speak{there exists a generating family
for~$\F$ consisting of~$n$ elements})$.
\end{enumerate}
By lemma~\ref{lemma:open-stalk}, the second statement implies the first -- note
that in a formal spelling of the statement in quotes,
\begin{equation}
  \tag{$\star$}
  \exists x_1,\ldots,x_n\?\F\_
  \forall x\?\F\_
  \exists a_1,\ldots,a_n\?\O_X\_
  x = \textstyle\sum_i a_i x_i,
\end{equation}
no implication signs occur. To show the converse direction,
we may assume that there is a generating family~$y_1,\ldots,y_m\?\F$ for~$\F$
(since~$\F$ is, externally speaking, of finite type). Then
the~$\Box$-translation of the statement that the~$y_i$ generate~$\F$ holds as
well (again by lemma~\ref{lemma:open-stalk}). Since there is an intuitionistic
proof of
\begin{multline*}
  \speak{$y_1,\ldots,y_m$ generate~$\F$} \Longrightarrow \\
  \bigl(\speak{there exist $x_1,\ldots,x_n\?\F$ which generate~$\F$}
    \Longleftrightarrow \\
    \exists x_1,\ldots,x_n\?\F\_
    \exists A\?\O^{m \times n}\_ \speak{$\vec y = A \vec x$}\bigr),
\end{multline*}
we can substitute the non-geometric formula~$(\star)$ by the geometric
formula
\[ \exists x_1,\ldots,x_n\?\F\_ \exists A\?\O^{m \times n}\_ \speak{$\vec
y = A \vec x$} \]
(lemma~\ref{lemma:stalk-open-with-hypothesis}). Thus the claim follows.
\end{proof}
% XXX: give backreference

\begin{lemma}\label{lemma:module-zero-point-neighbourhood}
Let~$X$ be a scheme (or ringed space). Let~$\F$ be an~$\O_X$-module
of finite type.
\begin{itemize}
\item Let~$x \in X$ be a point. Then the stalk~$\F_x$ is zero if and
only if~$\F$ is zero on some open neighbourhood of~$x$.
\item Let~$A \subseteq X$ be a closed subset. Then the restriction~$\F|_A$ (\ie
the pullback of~$\F$ to~$A$) is zero if and only if~$\F$ is zero on some open
subset of~$X$ containing~$A$.
\end{itemize}
\end{lemma}
\begin{proof}\emph{Both} statements are simply internalizations of
example~\ref{ex:module-zero-geometric}, using the modal operators~$\Box =
(\placeholder \vee A^c)$ and~$\Box = ((\placeholder \Rightarrow
\notat{x}) \Rightarrow \notat{x})$.
\end{proof}

\begin{rem}Note that the proposition fails if one drops the hypothesis
that~$\F$ is of finite type. Indeed, in this case one cannot reformulate the
condition that~$\F$ is zero in a geometric way.\end{rem}

\begin{lemma}Let~$X$ be a scheme (or ringed space). Let~$\alpha : \F \to \G$ be
a morphism of~$\O_X$-modules. Let~$\G$ be of finite type and assume
that~$\alpha_x : \F_x \to \G_x$ is surjective for some point~$x \in X$.
Then~$\alpha$ is an epimorphism on some open neighbourhood of~$x$.\end{lemma}
\begin{proof}In the presence of generators~$y_1,\ldots,y_n\?\G$, the
non-geometric surjectivity condition ($\forall y\?\G\_ \exists x\?\F\_
\alpha(x) = y$) can be reformulated in a geometric way: $\bigwedge_{i=1}^n
\exists x\?\F\_ \alpha(x) = y_i$. Thus the claim follows by
lemma~\ref{lemma:stalk-open-with-hypothesis}.\end{proof}

\begin{lemma}Let~$X$ be a scheme (or ringed space). Let~$\alpha : \F \to \G$ be
a morphism of~$\O_X$-modules. Let~$\F$ be of finite type and~$\G$ be coherent.
Suppose that~$\alpha_x$ is injective at some point~$x \in X$. Then~$\alpha$ is
a monomorphism on some open neighbourhood of~$x$.
\end{lemma}
\begin{proof}The kernel of~$\alpha$ is of finite type (by
lemma~\ref{lemma:coherent-stuff}) and zero at~$x$. By the previous lemma, it is
therefore zero on some open neighbourhood of~$x$.
\end{proof}

\begin{lemma}Let~$i : A \hookrightarrow X$ be a closed immersion of schemes (or
ringed spaces). Let~$\F$ be an~$\O_A$-module. Then~$i_*\F$ is of finite type if
and only if~$\F$ is of finite type.\end{lemma}
\begin{proof}
Let~$\Box$ be the modal operator defined by~$\Box\varphi :\equiv (\varphi \vee
A^c)$. From the internal perspective, we have a surjective ring homomorphism~$i^\sharp
: \O_X \to \O_A$, where we omit the forgetful functor~$i_*$ from~$\Box$-sheaves
to arbitrary sets in the notation, and an~$\O_A$-module~$\F$. Furthermore, we
may assume that~$\F$ is a~$\Box$-sheaf. We can regard~$\F$ as an~$\O_X$-module
by~$i^\sharp$.

Note that~$A^c \Rightarrow \F = 0$, by~$\Box$-separatedness of~$\F$.

We are to show that~$\F$ is a finitely generated~$\O_X$-module if and only if
the~$\Box$-translation of ``$\F$ is a finitely generated~$\O_A$-module'' holds.
In explicit terms, we have to show the equivalence of the following statements:
\begin{enumerate}
\item $\bigvee_{n \geq 0} \exists x_1,\ldots,x_n\?\F\_
  \forall x\?\F\_ \exists a_1,\ldots,a_n\?\O_X\_ x = \sum_i i^\sharp(a_i) x_i$.
\item $\Box(\bigvee_{n \geq 0} \Box(\exists x_1,\ldots,x_n\?\F\_
  \forall x\?\F\_ \Box(\exists b_1,\ldots,b_n\?\O_A\_ \Box(
    x = \sum_i b_i x_i))))$.
\end{enumerate}
It is clear that the first statement implies the second. For the converse
direction, we just have to repeatedly use the observation that~$\Box\varphi$
implies~$\varphi \vee (\F = 0)$ (once for each occurence of~$\Box$). So in each
step, we either obtain the statement we want~($\varphi$) or may assume
that~$\F$ is the trivial module, in which case any subclaim trivially follows. By
surjectivity of~$i^\sharp$, we may write any~$b\?\O_A$ as~$b =
i^\sharp(a)$ for some~$a\?\O_X$.
\end{proof}

\begin{lemma}Let~$X$ be a scheme. Let~$\F$ and~$\G$ be~$\O_X$-modules. Let~$x
\in X$. Then $\HOM_{\O_X}(\F,\G)_x \cong \Hom_{\O_{X,x}}(\F_x,\G_x)$ if~$\F$ is
of finite presentation around~$x$.\end{lemma}
\begin{proof}It suffices to give an intuitionistic proof of the following fact:
The construction~$\Hom_R(M,\placeholder)$ is geometric if~$M$ is a finitely
presented~$R$-module. So assume that~$M$ is the cokernel of a presentation
matrix~$(a_{ij}) \in R^{n \times m}$. Then we can calculate the Hom with
any~$R$-module~$N$ as
\[ \Hom_R(M,N) \cong \Bigl\{ x \? N^n \ \Big|\ \bigwedge_{j=1}^m \sum_{i=1}^n a_{ij}
x_i = 0 \? N \Bigr\}, \]
and this construction is patently geometric (set comprehension with respect to
a geometric formula).
\end{proof}

\begin{lemma}Let~$X$ be a scheme. Let~$\F$ be an~$\O_X$-module of finite
presentation. Let~$x \in X$. Then the stalk~$\F_x$ is a finitely
free~$\O_{X,x}$-module if and only if~$\F$ is locally free on some open
neighbourhood of~$x$.\end{lemma}
\begin{proof}The internal statement that~$\F$ is a free module is not geometric:
\[ \bigvee_{n \geq 0}
  \exists x_1,\ldots,x_n\?\F\_
  \forall x\?\F\_
  \exists! a_1,\ldots,a_n\?\O_X\_
  x = \textstyle\sum_i a_i x_i. \]
But it can equivalently be reformulated as
\[ \bigvee_{n \geq 0}
  \exists \alpha\?\HOM_{\O_X}(\F,\O_X^n)\_
  \exists \beta\?\HOM_{\O_X}(\O_X^n,\F)\_
  \alpha \circ \beta = \id \wedge \beta \circ \alpha = \id. \]
This reformulation is geometric, therefore it holds at~$x$ if and only if it
holds on some open neighbourhood of~$x$. The claim follows since, by the
previous proposition, taking stalks commutes with
calculating~$\HOM_{\O_X}(\F,\placeholder)$ resp.~$\HOM_{\O_X}(\O_X^n,\placeholder)$.
\end{proof}

\begin{itemize}
\item general explanation of modalities (as for instance in philosophy)
\item explain that for some modal operators, the~$\Box$-translation of the law
of excluded middle is valid; explain consequences
\item spreading of properties from stalk to neighbourhood: give many examples
\item give proof of the expressions for the nuclei listed in the table
\item baby version of Barr's theorem
\end{itemize}


\section{Rational functions and Cartier divisors}

\subsection{The sheaf of rational functions} Recall that the sheaf~$\K_X$ of rational
functions on a scheme~$X$ (or ringed space) can be defined as the sheaf
associated to the presheaf
\[ \text{$U \subseteq X$ open} \longmapsto \Gamma(U,\O_X)[\Gamma(U,\S)^{-1}], \]
where~$\Gamma(U,\S)$ is the multiplicative set of those sections of~$\O_X$ on~$U$,
which are regular in each stalk~$\O_{X,x}$, $x \in U$. Recall also there are
some wrong definitions in the literature~\cite{kleiman:misconceptions}.

Using the internal language, we can give a simpler definition of~$\K_X$.
Recall that we can associate to any ring~$R$ its total quotient ring, \ie
its localization at the multiplicative subset of regular elements. Since from
the internal perspective~$\O_X$ is an ordinary ring, we can associate to it its
total quotient ring $\O_X[\S^{-1}]$,
where~$\S$ is internally defined by the formula
\[ \S := \{ s\?\O_X \,|\, \speak{$s$ is regular} \} \subseteq \O_X. \]
Externally, this ring is the sheaf~$\K_X$.
\begin{prop}Let~$X$ be a scheme (or a ringed space). The sheaf of rings defined
in the internal language by localizing~$\O_X$ at its set of regular elements is
(canonically isomorphic to) the sheaf~$\K_X$ of rational functions.
\end{prop}
\begin{proof}Internally, the ring~$\O_X[\S^{-1}]$ fulfills the following
universal property: For any ring~$R$ and any homomorphism~$\O_X \to R$ which
maps the elements of~$\S$ to units, there exists exactly one
homomorphism~$\O_X[\S^{-1}] \to R$ which makes the evident diagram commute.
\[ \xymatrix{
  \O_X \ar[rr] \ar[dr] && R \\
  & \O_X[\S^{-1}] \ar@{-->}[ru]
} \]
The translation using the Kripke--Joyal semantics gives the following universal
property: For any open subset~$U \subseteq X$, any sheaf of rings~$\R$ on~$U$ and any
homomorphism~$\O_X|_U \to \R$ which maps all elements of~$\Gamma(V,\S)$, $V
\subseteq U$ to units, there exists exactly one homomorphism~$\O_X[\S^{-1}]|_U \to
\R$ which makes the evident diagram commute.
It is well-known~\cite{???} that the sheaf~$\K_X$ as usually defined satisfies
this universal property as well.
\end{proof}

\begin{prop}\label{prop:stalks-kx}
Let~$X$ be a scheme (or ringed space). Then the stalks of~$\K_X$
are given by
\[ \K_{X,x} = \O_{X,x}[\S_x^{-1}]. \]
The elements of~$\S_x$ are exactly the germs of those local sections which are
regular not only in~$\O_{X,x}$, but in all rings~$\O_{X,y}$ where~$y$
ranges over some neighbourhood of~$x$ (depending on the section).\end{prop}
\begin{proof}
Since localization is a geometric construction, the first statement is entirely
trivial. The second statement follows since
\[ \Gamma(U,\S) = \{ s\in\Gamma(U,\O_X) \,|\, U \models \speak{$s$ is regular}
\} \]
and regularity is a geometric implication, so that
$U \models \speak{$s$ is regular}$ if and only if the germ~$s_y$ is regular
in~$\O_{X,y}$ for all~$y \in U$.
\end{proof}

\begin{rem}Speaking internally, the multiplicative set~$\S$ is saturated.
Therefore an element~$s/t \? \K_X$ is invertible in~$\K_X$ if and only if the
numerator~$s$ belongs to~$\S$, \ie is an regular element of~$\O_X$.\end{rem}


\subsection{Regularity of local functions}
It is well known that on a locally Noetherian scheme, regularity spreads from
stalks to neighbourhoods, \ie a section of~$\O_X$ is regular
in~$\O_{X,x}$ if and only if it is regular on some neighbourhood on~$x$.
This fact has a simple proof in the internal language.
\begin{prop}\label{prop:regularity-spreading}
Let~$X$ be a locally Noetherian scheme. Let~$s \in \Gamma(U,\O_X)$
be a local function on~$X$. Let~$x \in U$. Then the following statements are
equivalent:
\begin{enumerate}
\item The section~$s$ is regular in~$\O_{X,x}$.
\item The section~$s$ is regular in all local rings~$\O_{X,y}$ where~$y$ ranges
over some neighbourhood of~$x$.
\end{enumerate}
\end{prop}
\begin{proof}
Let~$\Box$ be the modal operator defined by~$\Box(\varphi) :\equiv ((\varphi
\Rightarrow {!x}) \Rightarrow {!x})$. By corollary~\ref{cor:spreading}, we are
to show that the following statements of the internal language are equivalent:
\begin{enumerate}
\item $(\speak{$s$ is regular})^\Box$, \ie
$\forall t\?\O_X\_ st = 0 \Rightarrow \Box(t = 0)$.
\item $\Box(\speak{$s$ is regular})$, \ie
$\Box(\forall t\?\O_X\_ st = 0 \Rightarrow t = 0)$.
\end{enumerate}
It is clear that the second statement implies the first -- in fact, this is true
without any assumptions on~$X$: Let~$t\?\O_X$ be such that~$st = 0$. Since we want to
prove the boxed statement~$\Box(t=0)$, we may assume that~$s$ is regular and
prove~$t = 0$. This follows by definition.

For the converse direction, consider the annihilator of~$s$, \ie the ideal
\[ I := \Ann_{\O_X}(s) = \{ t\?\O_X \,|\, st = 0 \} \subseteq \O_X. \]
This ideal satisfies the quasicoherence condition (example~\ref{ex:annihilator-qcoh}),
thus~$I$ is a quasicoherent submodule of a finitely generated module. Since~$X$ is
locally Noetherian, it follows that~$I$ is finitely generated as well. By
assumption, each generator~$x_i \? I$ fulfills~$\Box(x_i = 0)$. Since we want
to prove a boxed statement, we may in fact assume~$x_i = 0$. Thus~$I = (0)$ and
the assertion that~$s$ is regular follows.
\end{proof}

\begin{cor}Let~$X$ be a locally Noetherian scheme. Then the stalks~$\K_{X,x}$
of the sheaf of rational functions are given by the total quotient rings of the
local rings~$\O_{X,x}$.\end{cor}
\begin{proof}Combine proposition~\ref{prop:stalks-kx} and
proposition~\ref{prop:regularity-spreading}.\end{proof}


\subsection{Geometric interpretation of rational functions}

\begin{prop}\label{prop:kx-is-negneg-sheafification}
Let~$X$ be a reduced scheme. Then~$\K_X$ is
the~$\neg\neg$-sheafification of~$\O_X$.\end{prop}
\begin{proof}Recall from corollary~\ref{cor:field-reduced} that
\[ \Sh(X) \models \forall s\?\O_X\_ \neg(\speak{$s$ invertible}) \Leftrightarrow
s=0. \]
From this we can deduce that~$\O_X$ is~$\neg\neg$-separated:
Assume~$\neg\neg(s=0)$ for~$s\?\O_X$. If~$s$ were invertible, we would
have~$\neg\neg(1=0)$ and thus~$\bot$. Therefore~$s$ is not invertible and thus
zero.

Furthermore, we can can deduce that internally, an element~$s\?\O_X$ is regular
if and only if it is \notnot invertible: For the ``only if'' direction,
note that a regular element is not zero (if it were, then the true statement~$0
\cdot 0 = 0 \cdot 1$ would imply the false statement~$0 = 1$). For the ``if''
direction, let~$st = 0$ in~$\O_X$. Since~$s$ is \notnot invertible, it follows
that~$t$ is \notnot zero. Since~$\O_X$ is~$\neg\neg$-separated, this implies
that~$t$ really is zero.

With these observations, we can proceed to that~$\K_X$ is~$\neg\neg$-separated.
So assume~$\neg\neg(a/s = 0)$ for~$a/s \? \K_X$. Since~$\K_X$ is obtained
from~$\O_X$ by localizing at regular elements, it holds that~$a/s = 0$
in~$\K_X$ if and only if~$a = 0$ in~$\O_X$. Thus it follows that~$\neg\neg(a =
0)$ in~$\O_X$ and thus~$a = 0$ in~$\O_X$; in particular, $a/s = 0$ in~$\K_X$.

We defer the proof that~$\K_X$ is a~$\neg\neg$-sheaf to the end and first
verify the universal property of~$\neg\neg$-sheafification.
% grammar?
So let~$G$ be a~$\neg\neg$-sheaf and let~$\alpha : \O_X \to G$ be a map. We
can define an extension~$\bar\alpha : \K_X \to G$ in the following way:
Let~$f \? \K_X$. Define the subsingleton~$S := \{ x \? G \,|\, \exists
b\?\O_X\_ f = b/1 \wedge x = \alpha(b) \} \subseteq G$. Since~$f$ can be
written in the form~$a/s$ with~$s$ \notnot invertible, it follows that~$S$
is \notnot inhabited. Since~$G$ is a~$\neg\neg$-sheaf, there exists a
unique~$x\?G$ such that~$\neg\neg(x \in S)$. We declare~$\bar\alpha(f)$ to be
this~$x$. It is straightforward to check that the composition~$\O_X \to \K_X
\to G$ equals~$\alpha$ and that~$\bar\alpha$ is unique with this property.

Up to this point, the proof did not need that~$X$ is a scheme -- it was enough
for~$X$ to be a ringed space such that the display equivalence above holds and
such that~$\neg(0 = 1)$ in~$\O_X$. Only now, to show that~$\K_X$ is
a~$\neg\neg$-sheaf, the scheme condition enters.
% XXX: finish proof
\end{proof}


\subsection{Cartier divisors} Let~$X$ be a scheme (or ringed space). Recall
that a \emph{Cartier divisor} on~$x$ is a global section of the sheaf of
groups~$\K_X^* / \O_X^*$. This sheaf can be constructed internally, with the
same notation: It is the quotient of the group of invertible elements of the
ring~$\K_X$ by the subgroup of invertible elements of the ring~$\O_X$. So an
arbitrary section of~$\K_X^*/\O_X^*$ is internally of the form~$[s/t]$
with~$s,t\?\O_X$ being regular elements; this is a simpler description than the
usual external one (as a family~$(f_i)_i$ of functions~$f_i \in
\Gamma(U_i,\K_X^*)$ such that~$f_i^{-1}|_{U_i \cap U_j} \cdot f_j|_{U_i \cap
U_j} \in \Gamma(U_i \cap U_j, \O_X^*)$ for all~$i,j$).

We can sketch the basic theory of Cartier divisors completely from the internal
perspective. In accordance with common practice, we will write the group
operation of~$\K_X^*/\O_X^*$ (which is induced by multiplication of elements
in~$\K_X^*$) additively.

\begin{defn}A Cartier divisor is \emph{effective} if and only if, from the
internal perspective, it can be written in the form~$[s/1]$ with~$s\?\O_X$
being a regular element.\end{defn}

Thus a Cartier divisor~$[s/t]$ is effective if and only if~$s$ is
an~$\O_X$-multiple of~$t$.

\begin{defn}A Cartier divisor~$D$ is \emph{principal} if and only if there
exists a global section~$f \in \Gamma(X,\K_X^*)$ such that internally,~$D = [f]$.
Two Cartier divisors are \emph{linearly equivalent} if and only if their
difference is a principal divisor.
\end{defn}

Note that decidedly, principality is a global notion: For any divisor~$D$ it is
true that locally there exists sections~$f$ of~$\K_X^*$ such that~$D = [f]$.

\begin{defn}The \emph{line bundle associated to a Cartier divisor}~$D$
is the~$\O_X$-submodule
\[ \O_X(D) := \{ g \in \K_X \,|\, g D \in \O_X \} = D^{-1} \O_X \subseteq \K_X
\]
of~$\K_X$. Here we are abusing language for~``$gD \in \O_X$'' to mean that~$gf
\in \O_X$ if~$D = [f]$ with~$f\?\K_X$; and for~``$D^{-1} \O_X$'' to
mean~$f^{-1}\O_X$. This condition resp. submodule does not depend on the
representative~$f$, since~$f$ is well-defined up to multiplication by an element
of~$\O_X^*$.\end{defn}

The submodule~$\O_X(D)$ is indeed locally free of rank~$1$, since
internally~$f^{-1}$ gives an one-element basis. Note that~$D$ is effective if
and only if~$\O_X(-D)$ is a subset of~$\O_X$ from the internal perspective. In
this case, we can define the \emph{support} of~$D$ to be the closed subscheme
of~$X$ associated to the sheaf of ideals~$\O_X(-D) \subseteq \O_X$.

\begin{defn}The \emph{Cartier divisor associated to a free submodule~$\L \subseteq
\K_X$ of rank~1} is~$D := [f^{-1}]$, where~$f\?\K_X$ is the unique element of
some one-element-basis of~$\L$.\end{defn}

The basis element~$f\?\K_X$ does indeed lie in~$\K_X^*$: Write~$f
= s/t$ with~$s,t \? \O_X$. It suffices to show that~$s$ is a regular element
of~$\O_X$. So let~$h\?\O_X$ such that~$sh = 0$ in~$\O_X$. Then in
particular~$hf = 0$ in~$\K_X$. By linear independence, it follows that~$h = 0$
in~$\K_X$ and thus~$h = 0$ in~$\O_X$.

Furthermore, the associated divisor does not depend on the choice of~$f$,
since~$f$ is well-defined up to multiplication by an element of~$\O_X^*$: If~$f
\O_X = g \O_X \subseteq \K_X$, then there exist~$u,v\?\O_X$ such that~$fu = g$
and~$gv = f$ in~$\K_X$. It follows that~$uv = fuvf^{-1} = gvf^{-1} = ff^{-1} =
1$ in~$\K_X$ and thus in~$\O_X$, by injectivity of the canonical map~$\O_X \to
\K_X$. Therefore~$u$ and~$v$ are elements of~$\O_X^*$.

\begin{lemma}Let~$D$ and~$D'$ be divisors on~$X$. Then~$\O_X(D) \otimes_{\O_X}
\O_X(D') \cong \O_X(D + D')$.\end{lemma}
\begin{proof}The wanted morphism of sheaves~$\O_X(D) \otimes \O_X(D') \to
\O_X(D + D')$ is given by multiplication. That this is well-defined and an
isomorphism can be checked from the internal point of view, where the claims
are obvious.\end{proof}

\begin{prop}The association~$D \mapsto \O_X(D)$ defines an one-to-one
correspondence between Cartier divisors on~$X$ and rank-one submodules
of~$\K_X$. This correpondence descends to an one-to-one correspondence between
Cartier divisiors up to linear equivalance and rank-one submodules of~$\K_X$ up
to isomorphism.\end{prop}
\begin{proof}The first statement is obvious from the definitions. For the
second statement, it suffices to show that~$\O_X(D)$ is isomorphic to~$\O_X$ if
and only if~$D$ is principal. A given isomorphism~$\O_X \to \O_X(D)$ gives a
global section~$f \in \K_X^*$ (by considering the image of the unit element)
such that internally,~$D = [f^{-1}]$; this shows that~$D$ is principal. The
converse is similar.
\end{proof}

\begin{rem}Locally principal subschemes (closed subschemes which are locally
the vanishing subscheme of a regular section of~$\O_X$) up to isomorphisms of
subschemes are in one-to-one correspondence with rank-1 submodules of~$\O_X$
(see~XXX). Thus locally principal subschemes (up to isomorphisms of abstract
schemes) are in one-to-one correspondence with effective Cartier divisors (up
to linear equivalence).\end{rem}
% XXX: check this.

\begin{prop}Assume that~$X$ is an integral scheme. Then any line bundle on~$X$
is (uncanonically) a submodule of~$\K_X$.\end{prop}
% XXX: weaker hypothesis possible?
\begin{proof}Let~$\xi$ be the generic point of~$X$ and let~$\Box := \neg\neg$
denote the modal operator such that internal sheafification with respect
to~$\Box$ is the same as pulling back to~$\{\xi\}$ and then pushing forward
to~$X$ again (see~XXX). Let~$\L$ be a line bundle on~$X$. Since~$\L_\xi \cong
\O_{X,\xi}$ (uncanonically), there is some injection~$\L_\xi \to \K_{X,\xi}$;
this corresponds internally to an injection~$\L^{++} \to \K_X^{++}$.
Since~$\K_X$ is already a~$\Box$-sheaf (see
proposition~\ref{prop:kx-is-negneg-sheafification}) and~$\L$ is~$\Box$-separated
(being isomorphic to~$\O_X$), we have the global injection
\[ \L \lhra \L^{++} \lhra \K_X^{++} \stackrel{({\cong})^{-1}}{\longrightarrow} \K_X. \qedhere \]
\end{proof}

\begin{itemize}
\item on reduced schemes, $\K_X$ is the sheaf of meromorphic functions
\item show~$\K_X = j_*(\O_X)$?
\item divisor associated to rational sections
\end{itemize}


\section{Compactness principles}

As stated in the introduction, quasicompactness of a space can not be detected
by the internal language: There cannot exist a formula~$\varphi$ such that a
topological space is quasicompact if and only if~$\Sh(X) \models \varphi$,
since the latter is always a local property on~$X$ while quasicompactness is not.
However, quasicompactness can be characterized by a \emph{metaproperty} of the
internal language.

This result is best stated in a way which does not explicitly refer to a notion
of finiteness. So recall that quasicompactness of a topological space~$X$ can
be phrased in the following way: For any directed set~$I$ and any monotone
family~$(U_i)_{i \in I}$ of open subsets, if~$X = \bigcup_i U_i$ then~$X = U_i$
for some~$i \in I$. As usual, a \emph{directed set} is an inhabited partially
ordered set such that for any two elements, there exists a common upper bound.
A family~$(U_i)_{i \in I}$ is \emph{monotone} if and only if~$i \preceq j$
implies~$U_i \subseteq U_j$.

\begin{prop}Let~$X$ be a topological space. Then~$X$ is quasicompact if and
only if the internal language of~$\Sh(X)$ fulfills the following metaproperty:
For any directed set~$I$ and any monotone family~$(\varphi_i)_{i \in I}$ of
formulas over~$X$, it holds that
\[ \Sh(X) \models \bigvee_{i \in I} \varphi_i
  \quad\text{implies}\quad
  \text{for some~$i \in I$, $\Sh(X) \models \varphi_i$}. \]
The monotonicity condition means that~$\Sh(X) \models (\varphi_i \Rightarrow
\varphi_j)$ for any~$i \preceq j$ in~$I$.
\end{prop}

Stated more succintly, a topological space~$X$ is quasicompact if and only
if~``$\Sh(X) \models$'' commutes with directed~``$\bigvee_{i \in I}$'''s.

\begin{proof}For the ``only if'' direction, let such a family of formulas be
given. Declare~$U_i$ to be the largest open subset of~$X$ where~$\varphi_i$
holds. Then by assumption, the~$U_i$ form a monotone family and cover~$X$. By
quasicompactness of~$X$, some single~$U_i$ covers~$X$ as well, such that the
corresponding formula~$\varphi_i$ holds on~$X$.

For the ``if'' direction, note that a monotone family~$(U_i)$ of open subsets
induces a monotone family of formulas by defining~$\varphi_i :\equiv U_i$. This
correspondence is such that~$\Sh(X) \models \bigvee_i \varphi_i$ holds if and
only if~$X = \bigcup_i U_i$ and such that~$\Sh(X) \models \varphi_i$ if and
only if~$X = U_i$. With these observations the claim is obvious.
\end{proof}

% XXX: formulate for locally constant index sheaves I as well.

\begin{ex}Let~$X$ be a quasicompact scheme (or quasicompact ringed space).
Let~$f \in \Gamma(X,\O_X)$ be a global function. Endow the set of natural
numbers with the usual ordering. Then the family of formulas given by~$(f^n =
0)_{n \in \NN}$ is monotone. Thus, if it internally holds that~$f$ is
nilpotent, then~$f$ is nilpotent as an element of~$\Gamma(X,\O_X)$ as
well.\end{ex}

\begin{prop}Let~$X$ be a topological space. Let~$K \subseteq X$ be an open
subset which is \emph{locally quasicompact} in the sense that there exists an open
covering~$X = \bigcup_j U_j$ such that each~$K \cap U_j$ is quasicompact. Then the
internal language of~$\Sh(X)$ fulfills the following metaproperty: For any
directed set~$I$ and monotone family~$(\varphi_i)_{i \in I}$ of formulas
over~$X$ it holds that
\[ \Sh(X) \models \bigl(K \Rightarrow \bigvee_i \varphi_i\bigr)
  \quad\text{implies}\quad
  \Sh(X) \models \bigvee_i (K \Rightarrow \varphi_i). \]
If additionally for any open subset~$V \subseteq X$ the set~$K \cap V$ is
locally quasicompact in~$V$, the following stronger and purely internal
statement holds:
\[ \Sh(X) \models \bigl(K \Rightarrow \bigvee_i \varphi_i\bigr)
  \Longrightarrow
  \bigvee_i (K \Rightarrow \varphi_i). \]
\end{prop}
\begin{proof}Assume that~$\Sh(X) \models (K \Rightarrow \bigvee_i \varphi_i)$.
This is equivalent to~$K \models \bigvee_i \varphi_i$. By the locality of the
internal language, it follows that~$K \cap U_j \models \bigvee_i \varphi_i$.
Since~$K \cap U_j$ is quasicompact, it follows by the previous proposition that
there exists an index~$i_j \in I$ such that~$K \cap U_j \models \varphi_{i_j}$.
This is equivalent to~$U_j \models (K \Rightarrow \varphi_{i_j})$. In
particular, it holds that~$U_j \models \bigvee_i (K \Rightarrow \varphi_i)$.
Since this is true for any~$j$, it follows that~$X \models \bigvee_i (K
\Rightarrow \varphi_i)$, again by the locality of the internal language.

The second statement is a corollary of the first one.
\end{proof}

\begin{ex}\label{ex:df-locally-compact}
Let~$X$ be a scheme and~$f \in \Gamma(X,\O_X)$ be a global function.
Then the open set~$D(f) = \{ x \in X \,|\, \text{$f_x$ is invertible in~$\O_{X,x}$}
\}$ is locally quasicompact in the sense of the proposition, even in the
stronger sense: Let~$V \subseteq X$ be any open set and consider a covering~$V = \bigcup_i
U_i$ by open affine subsets~$U_i = \Spec A_i$. Then~$D(f) \cap U_i \cong \Spec
A_i[f^{-1}]$ is quasicompact.\end{ex}

From this example, it will trivially follow that the nilradical~$\sqrt{(0)}
\subseteq \O_X$ of a scheme and indeed the radical of any quasicoherent ideal
sheaf is quasicoherent (lemma~\ref{ex:radical-qcoh}).

\begin{rem}In applications, the open set~$K$ of the proposition will be given
as the largest open subset on which some formula~$\psi$ holds. Then the
conclusion of the proposition is that \emph{assuming that~$\psi$ holds commutes
with directed disjunctions}.\end{rem}

A stronger condition on a topological space~$X$ than quasicompactness is
locality: A topological space is \emph{local} if and only if for any open
covering~$X = \bigcup_i U_i$ (not necessarily directed) a certain single~$U_i$
covers~$X$ as well.  Locality has a similar characterization as a metaproperty
of~$\Sh(X)$:

\begin{prop}Let~$X$ be a topological space. Then~$X$ is local if and
only if the internal language of~$\Sh(X)$ fulfills the following metaproperty:
For any set~$I$ and any family~$(\varphi_i)_{i \in I}$ of
formulas over~$X$, it holds that
\[ \Sh(X) \models \bigvee_{i \in I} \varphi_i
  \quad\text{implies}\quad
  \text{for some~$i \in I$, $\Sh(X) \models \varphi_i$}. \]
Furthermore, this metaproperty is equivalent to the following one: For any
sheaf~$\F$ on~$X$ and any formula~$\varphi(s)$ containing a variable~$s\?\F$,
it holds that
\[ \Sh(X) \models \exists s\?\F\_ \varphi(s)
  \quad\text{implies}\quad
  \text{for some~$s \in \Gamma(X,\F)$, $\Sh(X) \models \varphi(s)$}. \]
\end{prop}
\begin{proof}The proof of the first part is very similar to the proof of the
previous proposition. For the ``only if'' direction of the second part, note
that the antecedent implies that there exist local section~$s_i \in
\Gamma(U_i,\F)$ such that~$U_i \models \varphi(s_i)$ for some open covering~$X
= \bigcup_i U_i$. By locality of~$X$, one such~$U_i$ suffices to cover~$X$; so
the corresponding section~$s_i$ is actually a global section and verifies~$X
\models \varphi(s_i)$.

For the converse direction, note that given a family~$(U_i)_{i \in I}$ of open
subsets, we can define a sheaf on~$X$ by the internal definition
\[ \F := \{ M \in \Omega \,|\, \bigvee_{i \in I} (M = U_i) \}. \]
Recall that~$\Omega$ is the object of truth values, internally defined as the
power set of~$1 := \{\star\}$ and that~``$M = U_i$'' is abuse of notation
and means~$M = \{ x \in 1 \,|\, U_i \}$.
% XXX: the global sections of \F are not those which I expected! Is the claim
% true?
\end{proof}


\section{Quasicoherent sheaves of modules}

Recall that an~$\O_X$-module~$\F$ on a ringed space~$X$ is \emph{quasicoherent}
if and only if there exists a covering of~$X$ by open subsets~$U$ such that on
each such~$U$, there exists an exact sequence
\[ (\O_X|_U)^J \longrightarrow (\O_X|_U)^I \longrightarrow \F|_U \longrightarrow 0 \]
of~$\O_X|_U$-modules, where~$I$ and~$J$ are arbitrary sets (which may depend
on~$U$).

If~$X$ is indeed a scheme, quasicoherence can also be characterized in
terms of inclusions of distinguished open subsets of affines:
An~$\O_X$-module~$\F$ is quasicoherent if and only if for any open affine
subscheme~$U = \Spec A$ of~$X$ and any function~$f \in A$, the canonical map
\[ \Gamma(U,\F)[f^{-1}] \longrightarrow \Gamma(D(f),\F),\ 
  \tfrac{s}{f^n} \longmapsto f^{-n} s|_{D(f)} \]
is an isomorphism of~$A[f^{-1}]$-modules. Here~$D(f) \subseteq U$ denotes the
standard open subset~$\{ \ppp \in \Spec A \,|\, f \not\in \ppp \}$. Both
conditions can be internalized.

\begin{prop}Let~$X$ be a ringed space. Let~$\F$ be an~$\O_X$-module. Then~$\F$
is quasicoherent if and only if
\[ \Sh(X) \models \exists I,J\ \mathrm{lc}\_ \speak{there exists an
  exact sequence~$\O_X^J \to \O_X^I \to \F \to 0$}. \]
The ``\textnormal{lc}'' indicates that when interpreting this internal statement with the
Kripke--Joyal semantics,~$I$ and~$J$ should only be instantiated with
\emph{locally constant} sheaves.
\end{prop}
\begin{proof} We only sketch the proof.
The translation of the internal statement is that there exists a covering
of~$X$ by open subsets~$U$ such that for each such~$U$, there exist sets~$I,J$
and an exact sequence
\[ (\O_X|_U)^{\ul{J}} \longrightarrow (\O_X|_U)^{\ul{I}} \longrightarrow \F|_U
\longrightarrow 0 \]
where~$\ul{I}$ and~$\ul{J}$ are the constant sheaves associated to~$I$
respectively~$J$. The term~``$(\O_X|_U)^{\ul{I}}$'' refers to the internally
defined free~$\O_X$-module with basis the elements of~$\ul{I}$. By exploiting
that~$\ul{I}$ is a discrete set from the internal point of view (\ie any two
elements are either equal or not), one can show that this is the same
as~$(\O_X|_U)^I$; similarly for~$J$. With this observation, the statement
follows.
\end{proof}

In practice, the internal condition given by the proposition is not very
useful, since at the moment, we do not know of any internal characterization of
locally constant sheaves. The internal condition given by the following
proposition does not have this defect.

\begin{prop}\label{qcoh:sheafchar}
Let~$X$ be scheme. Let~$\F$ be an~$\O_X$-module. Then~$\F$ is
quasicoherent if and only if, from the internal perspective, the localized
module~$\F[f^{-1}]$ is a sheaf for the modal operator~$(\speak{$f$ inv.}
\Rightarrow \placeholder)$ for any~$f\?\O_X$.
\end{prop}

In detail, the internal condition is that for any~$f\?\O_X$, it holds that
\[ \forall s\?\F[f^{-1}]\_
  (\speak{$f$ inv.} \Rightarrow s = 0) \Longrightarrow s = 0 \]
and for any subsingleton~$S \subseteq \F[f^{-1}]$ it holds that
\[ (\speak{$f$ inv.} \Rightarrow \speak{$S$ inhabited}) \Longrightarrow
  \exists s\?\F[f^{-1}]\_
  (\speak{$f$ inv.} \Rightarrow s \in S). \]
Unlike with the internalizations of finite type, finite presentation and
coherence, this condition is \emph{not} a standard condition of commutative
algebra. In fact, in classical logic, this condition is always satisfied --
for trivial logical reasons if~$f$ is invertible and because~$\F[f^{-1}]$ is
the zero module if~$f$ is not invertible (since then, it's nilpotent). This is
to be expected: \emph{Any} module~$M$ in commutative algebra is quasicoherent in
the sense that the associated sheaf of modules~$M^\sim$ is quasicoherent.
% XXX: More to the point, commutative algebra does not deal with
% quasicoherence, since quasicoherence is an interesting condition only on on
% arbitrary schemes, not on affine schemes.

The proof will explain the origin of this condition.

\begin{ex}The zero~$\O_X$-module is quasicoherent, since (it and) all
localizations of it are singleton sets from the internal perspective and
thus~$\Box$-sheaves for any modal operator~$\Box$
(example~\ref{ex:special-sets-sheaves}).\end{ex}

\begin{proof}[Proof of proposition~\ref{qcoh:sheafchar}]\ldots\end{proof}

\begin{cor}Let~$X$ be a scheme. Let~$\F$ be a quasicoherent~$\O_X$-module.
Let~$\G \subseteq \F$ be a submodule. Then~$\G$ is quasicoherent if and only
if
\[ \Sh(X) \models \forall f\?\O_X\_
  \forall s\?\F\_
  (\speak{$f$ inv.} \Rightarrow s \in \G) \Longrightarrow
  \bigvee_{n \geq 0} f^n s \in \G. \]
\end{cor}
\begin{proof}We can give a purely internal proof. Let~$f\?\O_X$.
Since subpresheaves of separated sheaves are separated, the module~$\G[f^{-1}]$
is in any case separated with respect to the modal operator~$(\speak{$f$ inv.}
\Rightarrow \placeholder)$.

Now suppose that~$\G$ is quasicoherent. Let~$f\?\O_X$. Let $s\?\F$ and assume that
if~$f$ were invertible,~$s$ would be an element of~$\G$. Define the
subsingleton~$S := \{ t\?\G[f^{-1}] \,|\, \speak{$f$ inv.} \wedge t=s/1 \}$.
Then~$S$ would be inhabited by~$s/1$ if~$f$ were invertible. Since~$\G[f^{-1}]$
is a sheaf, it follows that there exists an element~$u/f^n$ of~$\G[f^{-1}]$
such that, if~$f$ were invertible, it would be the case that~$u/f^n = s/1 \in
\G[f^{-1}] \subseteq \F[f^{-1}]$.
Since~$\F[f^{-1}]$ is separated, it follows that it actually holds that~$u/f^n
= s/1 \in \F[f^{-1}]$. Therefore there exists~$m\?\NN$ such that $f^m f^n s =
f^m u \in \F$. Thus~$f^{m+n} s$ is an element of~$\G$.

For the converse direction, assume that~$\G$ fulfills the stated condition.
Let$f\?\O_X$. Let~$S \subseteq \G[f^{-1}]$ be a subsingleton which would be
inhabited if~$f$ were invertible. By regarding~$S$ as a subset of~$\F[f^{-1}]$,
it follows that there exists an element~$u/f^n \in \F[f^{-1}]$ such that,
if~$f$ were invertible, $u/f^n$ would be an element of~$S$. In particular,~$u$
would be an element of~$\G$. By assumption
it follows that there exists~$m\?\NN$ such that~$f^m u \in G$. Thus~$(f^m u) /
(f^m f^n)$ is an element of~$\G[f^{-1}]$ such that, if~$f$ were invertible, it
would be an element of~$S$.
\end{proof}

\begin{ex}\label{ex:annihilator-qcoh}
Let~$X$ be a scheme and~$s$ be a global section of~$\O_X$. Then the
annihilator of~$s$, \ie the sheaf of ideals internally defined by the
formula
\[ I := \Ann_{\O_X}(s) = \{ t\?\O_X \,|\, st = 0 \} \subseteq \O_X \]
is quasicoherent. To prove this in the internal language, it suffices to
verify the condition of the proposition.
So let~$f\?\O_X$ and~$t\?\O_X$ be arbitrary and assume~$\speak{$f$ inv.} \Rightarrow t \in I$,
\ie assume that if~$f$ were invertible,~$st$ would be zero. By
proposition~\ref{prop:cond-zero} it follows that~$f^n st = 0$ for
some~$n\?\NN$, \ie that~$f^n t \in I$.
\end{ex}

\begin{ex}\label{ex:radical-qcoh} Let~$X$ be a scheme and~$\I \subseteq \O_X$
be a quasicoherent ideal sheaf.  Then the radical of~$\I$, internally definable
as \[ \sqrt{\I} := \Bigl\{ s\?\O_X \,\Big|\, \bigvee_{n \geq 0} s^n \in \I \Bigr\}, \] is
quasicoherent as well: Let~$f\?\O_X$ and~$s\?\O_X$ be arbitrary and
assume~$\speak{$f$ inv.} \Rightarrow s \in \sqrt{\I}$, \ie assume that if~$f$
were invertible, some power~$s^n$ would be an element of~$\I$. Since
\emph{assuming that~$f$ is invertible commutes with directed disjunctions}
(example~\ref{ex:df-locally-compact}), it follows that for some natural
number~$n$, it holds that~$\speak{$f$ inv.} \Rightarrow s^n \in \I$. By
quasicoherence of~$\I$, we may deduce that for some natural number~$m$, it
holds that~$f^m s^n \in \I$. Thus~$fs \in \sqrt{\I}$.\end{ex}


\begin{itemize}
\item is the condition good enough to show that modules of finite type are
quasicoherent? To show that cokernels are quasicoherent?
\item discussion meaning of the sheaf condition in external language
\item give more examples: $(h)$, \ldots
\item Noetherian hypotheses: for example, that any quasicoherent submodule of a
module of finite type is of finite type as well
\end{itemize}


\section{Subschemes}

\subsection{Sheaves on open and closed subspaces}

% XXX: Remind the reader of abuse of notation like "U" for formulas vs. open sets.

\begin{lemma}\label{lemma:extension-by-empty-set}
Let~$X$ be a topological space. Let~$j : U \hookrightarrow X$ be the inclusion
of an open subspace. Then there is a canonical functor~$j_! : \Sh(U) \to
\Sh(X)$ called \emph{extension by the empty set} with the following properties:
\begin{enumerate}
\item The functor~$j_!$ is left adjoint to the restriction functor~$j^{-1} : \Sh(X) \to
\Sh(U)$.
\item The composition~$j^{-1} \circ j_! : \Sh(U) \to \Sh(U)$ is (canonically
isomorphic to) the identity.
\item The essential image of~$j_!$ consists of exactly those sheaves~$\F$ on~$X$
whose stalks are empty at all points of~$U^c$. In this case, it holds
that~$j_!j^{-1}\F \cong \F$ (canonically).
\end{enumerate}
\end{lemma}
\begin{proof}Internally, for a set~$\F$, we can define~$j_!(\F)$ simply be the
set comprehension
\[ j_!(\F) := \{ x\?\F \,|\, U \}. \]
Externally, the sections of the thusly defined sheaf on an open subset~$V
\subseteq X$ are given by~$\{ x \in \Gamma(V,\F) \,|\, V \subseteq U \}$,
\ie the whole of~$\Gamma(V,\F)$ if~$V \subseteq U$ and the empty set otherwise.
With this short internal description, all of the stated properties can be
easily verified in the internal language.

For instance, recall that internally the functor~$j^{-1}$ is given by
sheafifying with respect to the modal operator~$\Box :\equiv (U \Rightarrow
\placeholder)$. Thus, to show the second statement, we have to give a
bijection~$(j_!(\F))^{++} \to \F$ for any~$\Box$-sheaf~$\F$. (This map has to
be given explicitly, to not only show a weaker statement about a local
isomorphism, see discussion at XXX). To this end, we can use the composition
\[ (j_!(\F))^{++} \lhra \F^{++} \stackrel{({\cong})^{-1}}{\lra} \F, \]
where the first map is injective since sheafifying is exact. It is also
surjective, since the~$\Box$-translation of the statement~$\speak{$j_!(F) \to
\F$ is surjective}$ holds: For any element~$x\?\F$, it holds
that~$\Box(\speak{$x$ possesses a preimage})$.

For the third property, note that a sheaf~$\F$ on~$X$ fulfills the stated
condition on stalks if and only if, from the internal perspective, it holds
that~$U \Rightarrow \speak{$\F$ is inhabited}$. We omit further details.
\end{proof}

\begin{lemma}\label{lemma:extension-by-zero}
Let~$X$ be a ringed space. Let~$j : U \hookrightarrow X$ be the inclusion
of an open subspace. Then there is a canonical functor~$j_! : \Mod_U(\O_U) \to
\Mod_X(\O_X)$ called \emph{extension by zero} with the following properties:
\begin{enumerate}
\item The functor~$j_!$ is left adjoint to the restriction functor~$j^{-1} :
\Mod_X(\O_X) \to \Mod_U(\O_U)$.
\item The composition~$j^{-1} \circ j_! : \Mod_U(\O_U) \to \Mod_U(\O_U)$ is (canonically
isomorphic to) the identity.
\item The essential image of~$j_!$ consists of exactly those~$\O_X$-modules
whose stalks are zero at all points of~$U^c$. In this case, it holds
that~$j_!j^{-1}\F \cong \F$ (canonically).
\end{enumerate}
\end{lemma}
\begin{proof}Internally, a sheaf of modules on~$\O_U$ is simply a module
on~$\O_X^{++}$ which is a~$\Box$-sheaf, where~$\Box :\equiv (U \Rightarrow
\placeholder)$. The suitable internal definition for the extension by zero of
such a module~$\F$ is
\[ j_!(\F) := \{ x\?\F \,|\, (x = 0) \vee U \}. \]
With this description, all necessary verifications are easy. Note that
an~$\O_X$-module~$\F$ fulfills the stated condition on stalks if and only if
internally, it holds that~$\forall x\?\F\_ (x = 0) \vee U$.
\end{proof}

\begin{lemma}\label{lemma:essim-closed-immersion}
Let~$X$ be a topological space. Let~$i : A \hookrightarrow X$ be the inclusion
of a closed subspace. The essential image of the
inclusion~$i_* : \Sh(A) \to \Sh(X)$ consists of exactly those sheaves~$\F$ whose support
is a subset of~$A$. In this case, it holds that~$i_* i^{-1} \F \cong \F$
(canonically).\end{lemma}
\begin{proof}Recall that the modal operator associated to~$A$ is~$\Box\varphi
:\equiv (\varphi \vee A^c)$, and that by section~\ref{sect:internal-sheaves} the
essential image of~$i_*$ consists of exactly those sheaves which
are~$\Box$-sheaves from the internal perspective. Let~$\F$ be a sheaf on~$X$.
Then it holds that
\[ \supp\F \subseteq A \quad\Longleftrightarrow\quad
  A^c \subseteq X \setminus \supp\F \quad\Longleftrightarrow\quad
  A^c \subseteq \Int(X \setminus \supp\F). \]
Since the interior of the complement of~$\supp\F$ can be characterized as the
largest open subset of~$X$ on which the internal statement~``$\F$ is a
singleton'' holds (remark~\ref{rem:support-sheaf-of-sets}), the condition on
the support is fulfilled if and only if
\[ Sh(X) \models (A^c \Rightarrow \speak{$\F$ is a singleton}). \]
We thus have to show that this internal condition is equivalent to~$\F$ being
a~$\Box$-sheaf. For the ``if'' direction, assume~$A^c$. Then the empty subset~$S
\subseteq \F$ trivially verifies the condition that~$\Box(\speak{$S$ is a
singleton})$. There thus exists an element~$x\?\F$ (such that~$\Box(x \in S)$).
If we're given a further element~$y\?\F$, it trivially holds that~$\Box(x =
y)$. By~$\Box$-separatedness, it thus follows that~$x = y$. Thus~$\F$ is the
singleton~$\{x\}$. The proof of the ``only if'' direction is similar.

The second statement says that internally, sheafifying a~$\Box$-sheaf with
respect to the modal operator~$\Box$ and then forgetting that the result is
a~$\Box$-sheaf amounts to doing nothing. This is obvious.
\end{proof}

\subsection{Closed subschemes} Let~$X$ be a ringed space. Recall
that an ideal sheaf~$\I \subseteq \O_X$ defines a closed subset~$V(\I) = \{ x
\in X \,|\, \I_x \neq (1) \subseteq \O_{X,x} \}$, a sheaf of
rings~$\O_X/\I$, and a ringed space~$(V(\I), \O_{V(\I)})$ where~$\O_{V(\I)}$ is
the pullback of~$\O_X/\I$ to~$V(\I)$. In the internal universe, we can
reify~$V(\I)$ by giving a modal operator~$\Box$ such that externally, the
subspace~$X_\Box$ coincides with~$V(\I)$.

\begin{prop}\label{prop:basics-closed-subspace}
Let~$X$ be a ringed space. Let~$\I \subseteq \O_X$ be an ideal
sheaf. Then:
\begin{enumerate}
\item The subspace of~$X$ associated to the modal operator~$\Box$ defined
by~$\Box\varphi :\equiv (\varphi \vee (1 \in \I))$ is~$V(\I)$.
\item The support of~$\O_X/\I$ is exactly~$V(\I)$.
\item The canonical morphism~$i : V(\I) \to X$ is a closed immersion
of ringed spaces.
\end{enumerate}\end{prop}
\begin{proof}For any open subset~$U \subseteq X$, it holds that~$U \models 1
\in \I$ if and only if~$U \subseteq D(\I) = X \setminus V(\I)$. Thus~$D(\I)$
can be characterized as the largest open subset on which~``$1 \in \I$'' holds.
According to table~\ref{table:nuclei} on page~\pageref{table:nuclei}, the
stated modal operator thus defines the subspace~$D(\I)^c$, \ie~$V(\I)$.

For the second statement, note that since~$\O_X/\I$ is a sheaf of rings, its
support is closed.  Therefore the largest open subset of~$X$ where the internal
statement~``$\O_X/\I = 0$'' holds is the complement of the support
(proposition~\ref{prop:characterization-support}). Since~$D(\I)$ is the largest
open subset where the internal statement~``$\I = (1)$'' holds, it suffices to
show that internally,~$\O_X/\I = 0$ if and only if~$\I = (1)$. This is obvious.

The topological part of the third statement is clear. For the ring-theoretic
part, we have to show that the canonical ring homomorphism~$\O_X \to i_*
\O_{V(\I)}$, that is the canonical projection~$\O_X \to \O_X/(\I)$, is an
epimorphism of sheaves. This is obvious.
\end{proof}

By lemma~\ref{lemma:essim-closed-immersion}, the sheaf~$\O_X/\I$ is
thus a~$\Box$-sheaf from the internal perspective.

\begin{prop}Let~$X$ be a locally ringed space. Let~$\I \subseteq \O_X$ be an
ideal sheaf. Then the ringed space~$(V(\I), \O_{V(\I)})$ is too locally
ringed.\end{prop}
\begin{proof}We have to show that
\[ \Sh(V(\I)) \models \speak{$\O_{V(\I)}$ is a local ring}. \]
By theorem~\ref{thm:box-translation-semantically}, this is equivalent to
\[ \Sh(X) \models (\speak{$\O_X/\I$ is a local ring})^\Box, \]
where~$\Box$ is the modal operator given by~$\Box\varphi :\equiv (\varphi \vee
(1 \in \I))$. We therefore have to give an intuitionistic proof of the fact
\[ \forall x,y\?\O_X/\I\_ \speak{$x+y$ inv.} \Longrightarrow
  \Box(\speak{$x$ inv.} \vee \speak{$y$ inv.}). \]
So let~$x = [s], y = [t] \? \O_X/\I$ such that~$x + y$ is invertible
in~$\O_X/\I$. This means that there exists~$u\?\O_X$ and~$v\?\I$ such that~$us
+ ut + v = 1$ in~$\O_X$. Since~$\O_X$ is a local ring, it follows
that~$us$,~$ut$, or~$v$ is invertible. In the first two cases, it follows
that~$x$ respectively~$y$ are invertible in~$\O_X/\I$. In the third case, it
follows that~$1 \in \I$ and thus any boxed statement is trivially true.
\end{proof}

If~$X$ is a scheme and~$\I \subseteq \O_X$ is an ideal sheaf, it is well-known
that the locally ringed space~$V(\I)$ is a scheme if and only if~$\I$ is
quasicoherent. We cannot give an internal proof of this fact since we lack an
internal characterization of being a scheme.

\begin{lemma}Let~$X$ be a scheme (or ringed space). Let~$\I \subseteq \O_X$ be
an ideal sheaf. The ringed space~$V(\I)$ is reduced if and only if, from the
internal perspective of~$\Sh(X)$, the ideal~$\I$ is a radical ideal.\end{lemma}
\begin{proof}The following chain of equivalences holds:
\begin{align*}
  &\ \Sh(V(\I)) \models \speak{$\O_{V(\I)}$ is a reduced ring} \\
  \Longleftrightarrow&\ 
    \Sh(V(\I)) \models \bigwedge_{n \geq 0} \forall s\?\O_{V(\I)}\_
      s^n = 0 \Longrightarrow s = 0 \\
  \Longleftrightarrow&\ 
    \Sh(X) \models \bigl(\bigwedge_{n \geq 0} \forall s\?\O_X/\I\_ s^n = 0
    \Rightarrow s = 0\bigr)^\Box \\
  \Longleftrightarrow&\ 
    \Sh(X) \models \bigwedge_{n \geq 0} \forall s\?\O_X/\I\_ s^n = 0 \Rightarrow \Box(s = 0) \\
  \Longleftrightarrow&\ 
    \Sh(X) \models \bigwedge_{n \geq 0} \forall s\?\O_X\_ s^n \in \I
    \Rightarrow \Box(s \in \I) \\
  \Longleftrightarrow&\ 
    \Sh(X) \models \bigwedge_{n \geq 0} \forall s\?\O_X\_ s^n \in \I
    \Rightarrow s \in \I \\
  \Longleftrightarrow&\ 
    \Sh(X) \models \speak{$\I$ is a radical ideal}
\end{align*}
In the second-to-last step, we used that~$\Box(s \in \I) \equiv ((s \in \I) \vee
(1 \in \I))$ implies~$s \in \I$. This is trivial in both cases of the
disjunction.
\end{proof}

\begin{lemma}Let~$X$ be a scheme (or ringed space).
\begin{enumerate}
\item There exists a reduced closed sub-ringed space~$X_\mathrm{red}
\hookrightarrow X$ having the same underlying topological space as~$X$ such
that the following universal property is fulfilled: Any morphism~$Y \to X$
of (ringed or locally ringed) spaces such that~$Y$ is reduced factors uniquely
over the closed immersion~$X_\mathrm{red} \hookrightarrow X$.
\item Let~$A \subseteq X$ be a closed subset. Then there exists a structure of
a reduced closed ringed subspace on~$A$ fulfilling a similar universal
property.
\end{enumerate}
\end{lemma}
\begin{proof}For the first statement, let~$\N \subseteq \O_X$ be the nilradical
of~$\O_X$. This can internally be simply defined by~$\N := \sqrt{(0)} = \{
s\?\O_X \,|\, \bigvee_{n \geq 0} s^n = 0 \}$. Define~$X_\mathrm{red}$ as the closed
subspace associated to this ideal sheaf. This ringed space is reduced by the
previous lemma. The proof of the universal property can also be done in the
internal language, by using that the well-known fact of locale theory that the
category of locales over~$X$ is equivalent to internal locales in~$\Sh(X)$; but
we do not want to discuss this further.
% XXX: mention result on N(quasicompact)

For the second statement, internally define the ideal~$\I := \sqrt{\{ s\?\O_X \,|\, s = 0 \vee
A^c \}} \subseteq \O_X$. Then~$1 \in \I$ if and only if~$A^c$, thus by
proposition~\ref{prop:basics-closed-subspace} the closed ringed subspace defined
by~$\I$ has~$A$ as underlying topological space. It is reduced since~$\I$ is a
radical ideal.\end{proof}

\begin{itemize}
\item open subschemes
\item Koszul resolution
\end{itemize}


\section{Unsorted}
\begin{itemize}
\item ``functoriality''
\item Kähler differentials
\item closed and open subschemes
\item $j_! \O_U$ flat over~$\O_X$, \ldots
\item Koszul resolution
\item meta properties, uses (e.g. nilpotent on stalks iff globally nilpotent,
some lemmas about limits of modules)
\item compactness principle for ``$f$ inv.''
\item locally small categories
\item big Zariski topos
\item open/closed immersions
\item morphisms of schemes...
\item proper maps...
\item limits and colimits...
\item relative spectrum...
\end{itemize}


\nocite{*}
\printbibliography

\end{document}

XXX: ``completed natural number'' is a misnomer.

XXX: standardize level of generality: ringed locales, where possible?

XXX: remark that for simplicity, we work in a classical metatheory

XXX: replace []'s with ()'s where appropriate

XXX: word "intuitionistic"

XXX: check whether the negneg translation ``finitely generated ==> free''
implies the hard but important exercise

Snippet that may be useful later:
By appealing to the axiom of unique choice (see~XXX), we can
define a morphism of sheaves~$\O_X(D) \otimes \O_X(D') \to \O_X(D + D')$ by
internally describing a suitable map using representatives~$D = [f]$,~$D' =
[f']$, as long as the resulting map does not depend on the choice of representatives.

For the third statement, note that it is equivalent to show that
\[ \Gamma(U,\F^+) = \{ (V,s) \,|\,
  \text{$V \subseteq U$ dense open},\ 
  s \in \Gamma(V,\F),\ 
  \text{$(V,s)$ maximal} \}, \]
where~``$(V,s)$ maximal'' means that for any other such pair~$(W,t)$ such
that~$V \subseteq W$ and~$t|_V = s$, it holds that~$V = W$. This follows from
the fact that the plus construction can also be defined as
\[ \F^+ := \{ S \subseteq F \,|\,
  \speak{$S$ subsingleton},\ 
  \neg\neg(\speak{$S$ inhabited}),\ 
  \speak{$S$ $\neg\neg$-stable} \}. \]

XXX: simplification rule "box(phi) => box(psi) iff phi => box(psi)".
