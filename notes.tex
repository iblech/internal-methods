\documentclass[10pt]{amsart}
\usepackage[utf8]{inputenc}
\usepackage[english]{babel}
\usepackage{etex}
\usepackage{amsmath,amsthm,amssymb,stmaryrd,color,graphicx,multirow}
\usepackage{mathtools}
\usepackage{setspace}
\usepackage{bussproofs}
\usepackage{xspace}
\usepackage{longtable}
\usepackage{booktabs}
\usepackage{array}
\usepackage[protrusion=true,expansion=true]{microtype}
\usepackage[bookmarksdepth=2,pdfencoding=auto]{hyperref}
\usepackage[all]{xy}

\usepackage{tikz}
\usetikzlibrary{calc,shapes.callouts,shapes.arrows}
\newcommand{\hcancel}[5]{%
    \tikz[baseline=(tocancel.base)]{
        \node[inner sep=0pt,outer sep=0pt] (tocancel) {#1};
        \draw[red, line width=0.3mm] ($(tocancel.south west)+(#2,#3)$) -- ($(tocancel.north east)+(#4,#5)$);
    }%
}

\usepackage[natbib=true,style=numeric]{biblatex}
\usepackage[babel]{csquotes}
\bibliography{bibliography}

\theoremstyle{definition}
\newtheorem{defn}{Definition}[section]
\newtheorem{ex}[defn]{Example}

\theoremstyle{plain}

\newtheorem{prop}[defn]{Proposition}
\newtheorem{cor}[defn]{Corollary}
\newtheorem{lemma}[defn]{Lemma}
\newtheorem{thm}[defn]{Theorem}

\theoremstyle{remark}
\newtheorem{rem}[defn]{Remark}
\newtheorem{question}[defn]{Question}

\newcommand{\ZZ}{\mathbb{Z}}
\newcommand{\FF}{\mathbb{F}}
\renewcommand{\AA}{\mathbb{A}}
\newcommand{\A}{\mathcal{A}}
\renewcommand{\C}{\mathcal{C}}
\newcommand{\D}{\mathcal{D}}
\newcommand{\E}{\mathcal{E}}
\newcommand{\F}{\mathcal{F}}
\renewcommand{\G}{\mathcal{G}}
\renewcommand{\H}{\mathcal{H}}
\renewcommand{\O}{\mathcal{O}}
\newcommand{\K}{\mathcal{K}}
\newcommand{\N}{\mathcal{N}}
\newcommand{\M}{\mathcal{M}}
\renewcommand{\L}{\mathcal{L}}
\renewcommand{\P}{\mathcal{P}}
\newcommand{\R}{\mathcal{R}}
\newcommand{\T}{\mathcal{T}}
\newcommand{\I}{\mathcal{I}}
\newcommand{\J}{\mathcal{J}}
\renewcommand{\S}{\mathcal{S}}
\newcommand{\NN}{\mathbb{N}}
\newcommand{\PP}{\mathbb{P}}
\newcommand{\RR}{\mathbb{R}}
\newcommand{\QQ}{\mathbb{Q}}
\newcommand{\GG}{\mathbb{G}}
\newcommand{\aaa}{\mathfrak{a}}
\newcommand{\ppp}{\mathfrak{p}}
\newcommand{\qqq}{\mathfrak{q}}
\newcommand{\mmm}{\mathfrak{m}}
\newcommand{\nnn}{\mathfrak{n}}
\newcommand{\Hom}{\mathrm{Hom}}
\newcommand{\HOM}{\mathcal{H}\mathrm{om}}
\newcommand{\id}{\mathrm{id}}
\newcommand{\GL}{\mathrm{GL}}
\newcommand{\placeholder}{\underline{\quad}}
\newcommand{\ul}[1]{\underline{#1}}
\newcommand{\Set}{\mathrm{Set}}
\newcommand{\Grp}{\mathrm{Grp}}
\newcommand{\Vect}{\mathrm{Vect}}
\newcommand{\Sh}{\mathrm{Sh}}
\newcommand{\PSh}{\mathrm{PSh}}
\newcommand{\Zar}{\mathrm{Zar}}
\newcommand{\Et}{\mathrm{\acute{E}t}}
\newcommand{\Sch}{\mathrm{Sch}}
\newcommand{\Mod}{\mathrm{Mod}}
\newcommand{\Alg}{\mathrm{Alg}}
\newcommand{\Ring}{\mathrm{Ring}}
\newcommand{\LRL}{\mathrm{LRL}}
\newcommand{\pt}{\mathrm{pt}}
\newcommand{\tors}{\mathrm{tors}}
\DeclareMathOperator{\Spec}{Spec}
\newcommand{\QcohSpec}[2]{\mathrm{Spec}^{\mathrm{qcoh}}_{#1}{#2}}
\newcommand{\RelSpec}[2]{\mathrm{RelSpec}_{#1}{#2}}
\newcommand{\op}{\mathrm{op}}
\DeclareMathOperator{\colim}{colim}
\DeclareMathOperator{\rank}{rank}
\DeclareMathOperator{\Ann}{Ann}
\DeclareMathOperator{\Int}{int}
\DeclareMathOperator{\Clos}{cl}
\DeclareMathOperator{\Kernel}{ker}
\DeclareMathOperator{\supp}{supp}
\newcommand{\Ass}{\mathrm{Ass}}
\newcommand{\Open}{\T}
\newcommand{\?}{\,{:}\,}
\renewcommand{\_}{\mathpunct{.}\,}
\newcommand{\speak}[1]{\ulcorner\text{\textnormal{#1}}\urcorner}
\newcommand{\Ll}{\vcentcolon\Longleftrightarrow}
\newcommand{\notat}[1]{{!#1}}
\newcommand{\lra}{\longrightarrow}
\newcommand{\lhra}{\ensuremath{\lhook\joinrel\relbar\joinrel\rightarrow}}
\newcommand{\hra}{\hookrightarrow}
\newcommand{\brak}[1]{{\llbracket{#1}\rrbracket}}
\newcommand{\sdense}{{\widehat\Box}}
\newcommand{\sdenseother}{\Box}
\newcommand{\ie}{i.\,e.\@\xspace}
\newcommand{\eg}{e.\,g.\@\xspace}
\newcommand{\vs}{vs.\@\xspace}
\newcommand{\resp}{resp.\@\xspace}
\newcommand{\inv}{inv.\@}
\newcommand{\notnot}{\emph{not not}\xspace}
\newcommand{\affl}{\ensuremath{{\ul{\AA}^1_S}}\xspace}
\newcommand{\afflx}{\ensuremath{{\ul{\AA}^1_X}}\xspace}
\newcommand{\affla}{\ensuremath{{\ul{\AA}^1_{\Spec A}}}\xspace}
\newcommand{\xra}{\xrightarrow}

\newcommand{\XXX}[1]{\textbf{XXX: #1}}

\newcommand{\defeq}{\vcentcolon=}
\newcommand{\defequiv}{\vcentcolon\equiv}
\newcommand{\seq}[1]{\mathrel{\vdash\!\!\!_{#1}}}

\definecolor{gray}{rgb}{0.7,0.7,0.7}

\title{Using the internal language of toposes in algebraic geometry}
\author{Ingo Blechschmidt}
\email{iblech@web.de}

\begin{document}

\begin{abstract}
  There are several important toposes associated to a scheme, for instance the
  little and big Zariski toposes. These support an internal mathematical language
  which closely resembles the usual formal language of mathematics, but is ``local
  on the base scheme'':
  For example, from the internal perspective, the structure sheaf looks like an
  ordinary local ring (instead of a sheaf of rings with local stalks) and vector
  bundles look like ordinary free modules (instead of sheaves of modules
  satisfying a certain condition). The translation of internal statements and
  proofs is facilitated by an easy mechanical procedure.

  These expository notes give an introduction to this topic and show how the internal
  point of view can be exploited to give simpler definitions and more conceptual
  proofs of the basic notions and observations in algebraic geometry.
  % For instance, any theorem about modules yields a corresponding theorem about
  % sheaves of modules (with a small caveat).
  We also employ this framework to study the phenomenon that some properties
  spread from points to open neighbourhoods. We give a general sufficient
  condition for this spreading to occur, depending only on the logical
  form of the property in question.

  No prior knowledge about topos theory or formal logic is assumed.
\end{abstract}
% XXX: update

\maketitle

\begin{center}Rough draft in progress, missing explanations, references,
proofs, and a proper copyediting. I am happy about comments of any
kind; please direct them to \texttt{iblech@web.de}.\end{center}

%\setcounter{tocdepth}{1}
\tableofcontents

\section{Introduction}

\subsection*{Internal language of toposes}
A \emph{topos} is a category which shares certain categorical properties with
the category of sets; the archetypical example is the category of sets, and
the most important example for the purposes of these notes is the category of
set-valued sheaves on a topological space.

Any topos~$\E$ supports an \emph{internal language}. This is a device which
allows one to \emph{pretend} that the objects of~$\E$ are plain sets and that
the morphisms are plain maps between sets, even if in fact they are not. For
instance, consider a morphism~$\alpha : X \to Y$ in~$\E$. From the \emph{internal
point of view}, this looks like a map between sets, and we can formulate the
condition that this map is surjective; we write this as
\[ \E \models \forall y\?Y\_ \exists x\?X\_ \alpha(x) = y. \]
The appearance of the colons instead of the usual element signs reminds us that
this expression is not to be taken literally --~$X$ and~$Y$ are objects of~$\E$
and thus not necessarily sets. The definition of the internal language is made
in such a way so that the meaning of this internal statement is that~$\alpha$
is an epimorphism. Similarly, the translation of the internal statement
that~$\alpha$ is injective is that~$\alpha$ is a monomorphism.

Furthermore, we can \emph{reason} with the internal language. There is a
metatheorem to the effect that if some statement~$\varphi$ holds from the
internal point of view of a topos~$\E$ and if~$\varphi$ logically implies some
further statement~$\psi$, then~$\psi$ holds in~$\E$ as well. As a simple
example, consider the elementary fact that the composition of surjective maps
is surjective. Interpreting this statement in the internal language of~$\E$, we
obtain the more abstract result that the composition of epimorphisms in~$\E$ is
epic.

There is, however, a slight caveat to this metatheorem. Namely, the internal
language of a topos is in general only \emph{intuitionistic}, not
\emph{classical}. This means that internally, one can not use the law of
excluded middle~($\varphi \vee \neg\varphi$), the law of double negation
elimination~($\neg\neg\varphi \Rightarrow \varphi$), or the axiom of choice.
For instance, one rendition of the axiom of choice is that any vector space is
free. But it need not be the case that a vector space internal to a topos
is free as seen from the internal perspective: By the technique explained in
these notes, this would imply the absurd statement that any sheaf of modules on
a reduced scheme is locally free.

The restriction to intuitionistic reasoning is not as confining as it might first
appear. We will discuss its practical consequences below (on
page~\pageref{sect:appreciating-intuitionistic-logic}).


\subsection*{Algebraic geometry}
We apply the internal language of toposes to algebraic geometry as follows. If~$X$ is a
scheme, the structure sheaf~$\O_X$ is a sheaf of rings, \ie the sets of
local sections carry ring structures and these ring structures are compatible
with restriction. From the internal point of view of the topos of set-valued
sheaves on~$X$, denoted~``$\Sh(X)$'' in the following, the structure
sheaf~$\O_X$ looks much simpler: It looks just like a plain ring (and
not a sheaf of rings). Similarly, a sheaf of~$\O_X$-modules looks just like a
plain module over that ring.

This allows to import notions and facts from basic linear and commutative
algebra into the sheaf setting. For instance, it turns out that a sheaf
of~$\O_X$-modules is of finite type if and only if, from the internal
perspective, it is finitely generated as an~$\O_X$-module. Now consider the
following fact of linear algebra: If in a short exact sequence of modules the two
outer ones are finitely generated, then the middle one is too. The usual proof of
this fact is intuitionistically acceptable and can thus be interpreted in the
internal language. It then \emph{automatically} yields the following more advanced
proposition: If in a short exact sequence of sheaves of~$\O_X$-modules the
two outer ones are of finite type, then the middle one is too.

This example was not special: \emph{Any (intuitionistically valid) theorem
about modules yields a corresponding theorem about sheaves of modules.}

The internal language machinery thus allows us to understand the basic notions
and statements of scheme theory as notions and statements of linear and
commutative algebra, interpreted in a suitable sheaf topos. This brings
conceptual clarity and reduces technical overhead.

In these notes, we explain how the internal language machinery works and then develop a
\emph{dictionary} between common notions of scheme theory and corresponding
notions of algebra. Once built, this dictionary can be used arbitrarily often.
We stress that no in-depth knowledge of topos theory or categorical logic is
necessary to apply this apparatus.

Two highlights of our approach are the following. Let~$X$ be a reduced scheme
and~$\F$ be a sheaf of~$\O_X$-modules of finite type. Then it is well-known
that~$\F$ is locally free on some dense open subset of~$X$; for instance, this
is stated in Vakil's lecture notes as an ``important hard
exercise''~\cite[Exercise~13.7.K]{vakil:foag}. In fact, this proposition is just the
interpretation of the following statement of intuitionistic linear algebra in
the sheaf topos: Any finitely generated vector space is \emph{not not} free.
The proof of this statement is entirely straightforward.\footnote{Intuitionistically,
the statement that any finitely generated vector space is \emph{free} is stronger than
the doubly negated version and can not be shown. It would imply that any sheaf
of finite type is not only locally free on some dense open subset, but locally
free on the whole space. We discuss this example in more detail in
Section~\ref{sect:upper-semicontinuous-functions} and in particular in
Lemma~\ref{lemma:locally-free-dense}.}

The second highlight is that we can shed light on the phenomenon that
sometimes, truth of a property at a point~$x$ spreads to some open
neighbourhood of~$x$; and in particular that sometimes, truth of a property at
the generic point spreads to some dense open subset. For instance, if the stalk
of a sheaf of finite type is zero at some point, the sheaf is even zero on some
open neighbourhood; but this spreading does not occur for general sheaves which
may fail to be of finite type.

We formalize this by introducing a \emph{modal operator}~$\Box$ into the
internal language, such that the internal statement~$\Box\varphi$ means
that~$\varphi$ holds on some open neighbourhood of~$x$. Furthermore, we
introduce a simple operation on formulas, the~\emph{$\Box$-translation}
$\varphi \mapsto \varphi^\Box$, such that~$\varphi^\Box$ means that~$\varphi$
holds at the point~$x$. This translation is defined on a purely syntactical
level. The question whether truth at~$x$ spreads to truth on a
neighbourhood can then be formulated in the following way: Does~$\varphi^\Box$
intuitionistically imply~$\Box\varphi$?

This allows to deal with the question in a simpler, more logical way, with the
technicalities of sheaves blinded out. We also give a metatheorem which
covers a wide range of cases. Namely, spreading occurs for all those properties
which can be formulated in the internal language without
using~``$\Rightarrow$'',~``$\forall$'', and~``$\neg$''.

To illustrate the example above, consider the property of a module~$\F$ being
the zero module. In the internal language, it can be formulated as~$(\forall x\?\F\_ x = 0)$.
Because of the appearance of~``$\forall$'', the metatheorem is not
applicable to this statement. But if~$\F$ is of finite type, there are
generators~$x_1,\ldots,x_n\?\F$ from the internal point of view, and the
condition can be reformulated as~$x_1 = 0 \wedge \cdots \wedge x_n = 0$; the
metatheorem is applicable to this statement.

All of these applications employ the \emph{little Zariski topos} of the base
scheme~$X$, the topos of sheaves on the underlying topological space of~$X$.
For treating \emph{schemes over~$X$} as opposed to ``local objects on~$X$''
such as sheaves of rings and modules from an internal point of view, this topos
is not particularly well-adapted. For such applications the \emph{big
Zariski topos} has to be used. We discuss this topos and related toposes, such
as the big étale topos, in Section~\ref{sect:big-zariski}. The key point is
that they can be used to internalize Grothendieck's ``functor of points''
approach to algebraic geometry. From the internal point of view of the big
Zariski topos of a scheme~$X$, schemes over~$X$ look just like plain sets and
morphisms between~$X$-schemes look like plain maps between those sets. Elements
of those sets correspond to~$T$-points of such an~$X$-scheme, with the internal
language machinery keeping track of the point's domain~$T$ of definition.

Therefore basic constructions of relative scheme theory, like building the
affine or projective space over a base scheme or associating to a quasicoherent
sheaf of algebras its relative spectrum, can be described and reasoned about in
a simple, element-based language. For instance, projective~$n$-space over~$X$
is described by the internal expression
\[ \PP^n \defeq \{ (x_0,\ldots,x_n) \? (\afflx)^{n+1} \,|\,
  x_0 \neq 0 \vee \cdots \vee x_n \neq 0 \}/(\text{rescaling}) \]
in the big Zariski topos of~$X$, with the scheme structure automatically taken
care of. Also, the (not necessarily locally trivial) tangent bundle of
an~$X$-scheme~$T$ can be thought of as the set of maps from the somewhat
curious set~$\Delta \defeq \{ \varepsilon\?\afflx \,|\, \varepsilon^2 = 0 \}$
to~$T$, thus formalizing the intuitive picture of tangent vectors as
infinitesimal curves.

Modal operators are important in the big topos setting as well. For instance,
there is a modal operator~$\Box_\text{ét}$ in the big Zariski topos such that
the internal statement~$\Box_\text{ét} \varphi$ roughly means that~$\varphi$
holds on an étale covering and such that the translated
formula~$\varphi^{\Box_\text{ét}}$ means that~$\varphi$ holds in the big étale
topos.


\subsection*{Limitations} The internal language is \emph{local}, in the sense
that if~$X = \bigcup_i U_i$ is an open covering and an internal statement
holds in the sheaf toposes~$\Sh(U_i)$, it holds in~$\Sh(X)$ as well. On the one
hand, this property is very useful. But on the other hand, it gives an inherent
limitation of the internal language:
Global properties of sheaves of modules like ``being generated by global
sections'', ``being ample'', or ``having vanishing sheaf cohomology'' and global properties of schemes like ``being
quasicompact'' can \emph{not} be
expressed in the internal language.

Thus for global considerations, the internal language of~$\Sh(X)$ is only
useful in that local subparts can be simplified. Also, some global features
reflect themselves in certain metaproperties of the internal language. For
instance, a scheme is quasicompact if and only if the internal language
has a weak version of the so-called disjunction property of mathematical
logic (Section~\ref{sect:compactness}).


\subsection*{Introductory literature} These notes are intended
to be self-contained, requiring only basic knowledge of scheme theory. In
particular, we assume no prior familiarity with topos theory or formal logic.
But if the interested reader is so inclined, she will find a gentle
introduction to topos theory in an article by
Leinster~\cite{leinster:introduction}. Standard references for the internal
language of a topos include the book of Mac~Lane and
Moerdijk~\cite[Chapter~VI]{moerdijk-maclane:sheaves-logic}, the book of
Borceux~\cite[Chapter~6]{borceux:handbook3}, and part~D of
Johnstone's Elephant~\cite{johnstone:elephant}. In the 1970s, there was a
flurry of activity on applications of the internal language. An article by
Mulvey~\cite{mulvey:repr} of this time gives a very accessible
introduction to the topic, culminating in an internal proof of the Serre--Swan
theorem (with just one external ingredient needed).


\subsection*{Related work} The internal language of toposes was applied to algebraic geometry before. For
instance, Wraith used it to construct (and verify the universal property
of) the little étale topos of a scheme by internally developing the theory of
strict henselization~\cite{wraith:generic-galois-theory}. However, to the best
of my knowledge, systematically building a dictionary between external and
internal notions has not been attempted before, and the use of modal operators
to study the spreading of properties from points to neighbourhoods seems to be
new as well.

Brandenburg put forward a related program of internalization in his PhD
thesis~\cite{brandenburg:tensor-foundations}. However, he internalizes
constructions of algebraic geometry not in toposes, but in tensor categories.
There is some overlap in working out precise universal properties, particularly
when dealing with the big Zariski topos, which we deal with in
Section~\ref{sect:big-zariski}.

In other branches of mathematics, the internal language of toposes is used as well. For
instance, there is an ongoing effort in mathematical physics to understand
quantum mechanical systems from an internal point of view: To any quantum
mechanical system, one can associate a so-called Bohr topos containing an
internal mirror image of the system. This mirror image looks like a
system of classical mechanics from the internal perspective, and therefore
tools like Gelfand duality can be used to construct an internal
phase space for the system~\cite{bohr1,bohr2}.

In stochastics, the usefulness of an internal language was recently stressed by
Tao~\cite{tao:analysis-rel-measure-space}. Such a language makes the
common notational practice of dropping the explicit dependence of the
value~$X(\omega)$ of a random variable on the sample~$\omega$ completely
rigorous and simplifies the basic theory. Tao also highlighted how a suitable
language can be used to simplify ``$\varepsilon$/$\delta$ management'' in
analysis~\cite{tao:cheap-nsa}. Furthermore, there is a topos-theoretic approach to
measure theory, in which the sheaf of measurable real functions on
a~$\sigma$-algebra looks like the ordinary set of real numbers from an internal point
of view~\cite{jackson:sheaf-theoretic-measure-theory}; this has applications in
noncommutative geometry~\cite{henry:measure-theory-boolean-toposes}.

Intuitionistic methods have found many applications in computer science.
Recently, the internal language of a topos of trees and a suitable modal
operator was used to study guarded recursion, encompassing, for instance, an
internal Banach fixed-point theorem~\cite{birkedal:al:sgdt}.

In constructive mathematics, the internal language of toposes is routinely used
to obtain models of intuitionistic theories fulfilling certain anti-classical
axioms. For instance, there are toposes in which the axiom ``any map~$\RR \to
\RR$ is continuous'' (appropriately formulated) holds~\cite{kock:sdg,moerdijk:reyes:models}
and toposes in which the Church--Turing thesis holds (realizability toposes).
The internal language can also be used to extract computational content
out of classical constructions. To cite just one recent example, Mannaa and
Coquand used it to implement algorithms for working with the algebraic closure
of an arbitrary field of characteristic zero~\cite{mannaa:coquand:alg-closure}.

More specifically, our contribution is related to the program of constructive
mathematics in that intuitionistic mathematics gains new areas of application.
For instance, the constructive account of the theory of Krull dimension was
originally developed to remove Noetherian hypotheses, extract computational meaning, and
simplify proofs~\cite{dyn:krull-integral,dyn:char-krull}. It can now also be used to
reason about the dimension of schemes, since the topological dimension of a
scheme~$X$ coincides with the Krull dimension of the structure sheaf~$\O_X$
(regarded as an ordinary ring from the internal perspective of~$\Sh(X)$,
Section~\ref{sect:krull-dimension}).

We obtained a second contribution to constructive mathematics as a byproduct of
deducing transfer principles which relate a module over a ring~$A$ with its
induced quasicoherent sheaf on~$\Spec A$: Using the internal language of the
little Zariski topos we can algorithmically turn certain non-constructive
arguments concerning prime ideals into constructive ones. We discuss this in
Section~\ref{sect:eliminating-prime-ideals}; it is related to the
\emph{dynamical methods in algebra} explored by Coquand, Coste, Lombardi, Roy,
and others~\cite{clr:dynamicalmethod,cl:logical}.

Caramello uses topos theory to build bridges between different mathematical
subjects, in a certain precise sense~\cite{caramello:1,caramello:2}. She
exploits that toposes can admit presentations by different sites. Since we
focus only on a specific presentation of a few specific toposes associated to a
scheme, our contribution is only indirectly related to her's.

\XXX{mention and explain: Mulvey/Burden, Vickers, Awodey, Coquand, ...}

\XXX{further work, ...}


\subsection*{Notational convention} Occasionally, when quantifying, we use
colons instead of element signs not only in internal statements, but also in
external ones; particularly if we want to stress that a discussion takes place
in an intuitionistic context.


%Of particular importance is the
%case~$\Box\varphi \defequiv ((\varphi \Rightarrow \notat{x}) \Rightarrow
%\notat{x})$ where~$x \in X$ is a point, since for this modal operator, the
%proposition specializes to
%\[ \Sh(X) \models \varphi^\Box \quad\text{iff}\quad
%  \text{$\varphi$ holds at~$x$}. \]
%Thus the question whether truth of a proposition~$\varphi$ at a point~$x$ spreads
%to a neighbourhood of~$x$ can be formulated in the following way:
%\emph{Does~$\varphi^\Box$ imply~$\Box\varphi$?} We can give a general
%answer to this question.

\begin{itemize}
\item dictionary; microscope/telescope into another
universe; types instead of sets; (dependent types to encompass almost all
mathematics)
\item explain that with the internal language business, it becomes more
transparent where scheme condition enters
\end{itemize}


\section{The internal language of a sheaf topos}

\subsection{Internal statements}
Let~$X$ be a topological space. Later, $X$ will be the underlying space of a
scheme. The meaning of internal statements is given by a set of rules, the
\emph{Kripke--Joyal semantics} of the topos of sheaves on~$X$.

\begin{defn}The meaning of 
\[ U \models \varphi \quad\text{(``$\varphi$ holds on $U$'')} \]
for open subsets~$U \subseteq X$ and formulas~$\varphi$ over~$U$ is given by
the rules listed in Table~\ref{table:kripke-joyal}, recursively in the structure of~$\varphi$.
In a \emph{formula over~$U$} there may appear sheaves defined on~$U$ as domains
of quantifications,~$U$-sections of sheaves as terms, and morphisms of sheaves
on~$U$ as function symbols. The symbols~``$\top$'' and~``$\bot$'' denote truth
and falsehold, respectively. The universal and existential quantifiers come in
two flavors: for bounded and unbounded quantification.
The translation of~$U \models \neg\varphi$ does not have to be separately defined, since
negation can be expressed using other symbols: $\neg\varphi \defequiv (\varphi
\Rightarrow \bot)$. If we want to emphasize the particular topos, we write
\[ \Sh(X) \models \varphi \quad\Ll\quad X \models \varphi. \]
\end{defn}

\begin{table}
  \centering
  \[ \renewcommand{\arraystretch}{1.3}\begin{array}{@{}lcl@{}}
    U \models s = t \? \F &\Ll& s|_U = t|_U \in \Gamma(U, \F) \\
    U \models s \in \G &\Ll& s|_U \in \Gamma(U,\G) \quad\quad\text{($\G$ a
    subsheaf of~$\F$, $s$ a section of~$\F$)} \\
    U \models \top &\Ll& U = U \text{ (always fulfilled)} \\
    U \models \bot &\Ll& U = \emptyset \\
    U \models \varphi \wedge \psi &\Ll&
      \text{$U \models \varphi$ and $U \models \psi$} \\
    U \models \bigwedge_{j \in J} \varphi_j &\Ll&
      \text{for all~$j \in J$: $U \models \varphi_j$} \quad\quad\text{($J$ an
      index set)} \\
    U \models \varphi \vee \psi &\Ll&
      \hcancel{\text{$U \models \varphi$ or $U \models \psi$}}{0pt}{3pt}{0pt}{-2pt} \\
    && \text{there exists a covering $U = \bigcup_i U_i$ such that for all~$i$:} \\
    && \quad\quad \text{$U_i \models \varphi$ or $U_i \models \psi$} \\
    U \models \bigvee_{j \in J} \varphi_j &\Ll&
      \hcancel{\text{$U \models \varphi_j$ for some~$j \in J$}}{0pt}{3pt}{0pt}{-2pt}
      \quad\quad\text{($J$ an index set)} \\
    && \text{there exists a covering $U = \bigcup_i U_i$ such that for all~$i$:} \\
    && \quad\quad \text{$U_i \models \varphi_j$ for some~$j \in J$} \\
    U \models \varphi \Rightarrow \psi &\Ll&
      \text{for all open~$V \subseteq U$:
      $V \models \varphi$ implies $V \models \psi$} \\
    U \models \forall s \? \F\_ \varphi(s) &\Ll&
      \text{for all sections~$s \in \Gamma(V, \F)$, open $V \subseteq U$: $V \models
      \varphi(s)$} \\
    U \models \exists s \? \F\_ \varphi(s) &\Ll&
      \hcancel{\text{there exists a section~$s \in \Gamma(U,\F)$ such that $U
      \models \varphi(s)$}}{0pt}{3pt}{0pt}{-2pt} \\
    &&
      \text{there exists an open covering $U = \bigcup_i U_i$ such that for all~$i$:} \\
    && \quad\quad \text{there exists~$s_i \in \Gamma(U_i, \F)$ such that
    $U_i \models \varphi(s_i)$} \\
    U \models \forall \F\_ \varphi(\F) &\Ll&
      \text{for all sheaves $\F$ on $V$, open $V \subseteq U$: $V \models \varphi(\F)$} \\
    U \models \exists \F\_ \varphi(\F) &\Ll&
      \text{there exists an open covering $U = \bigcup_i U_i$ such that for all~$i$:} \\
    && \quad\quad \text{there exists a sheaf~$\F_i$ on~$U_i$ such that
    $U_i \models \varphi(\F_i)$}
  \end{array} \]
  \caption{\label{table:kripke-joyal}The Kripke--Joyal semantics of a sheaf
  topos.}
\end{table}

\begin{rem}The last two rules in Table~\ref{table:kripke-joyal}, concerning
\emph{unbounded quantification}, and are not part of the classical Kripke--Joyal
semantics. They are part of Mike Shulman's stack semantics~\cite{shulman:stack},
a slight extension. They are needed so that we can formulate universal
properties in the internal language.
\end{rem}

\begin{ex}\label{ex:injective-surjective}
Let~$\alpha : \F \to \G$ be a morphism of sheaves on~$X$. Then
$\alpha$ is a monomorphism of sheaves if and only if, from the internal
perspective,~$\alpha$ is simply an injective map:
\allowdisplaybreaks
\begin{align*}
  & X \models \speak{$\alpha$ is injective} \\[0.5em]
  \Longleftrightarrow\
  & X \models \forall s\?\F\_ \forall t\?\F\_ \alpha(s) = \alpha(t) \Rightarrow s = t \\[0.5em]
  \Longleftrightarrow\ &
    \text{for all open~$U \subseteq X$, sections $s \in \Gamma(U, \F)$:} \\
  & \text{for all open~$V \subseteq U$, sections $t \in \Gamma(V, \F)$:} \\
  &\qquad\qquad
      V \models \alpha(s) = \alpha(t) \Rightarrow s = t \\[0.5em]
  \Longleftrightarrow\ &
    \text{for all open~$U \subseteq X$, sections $s \in \Gamma(U, \F)$:} \\
  & \text{for all open~$V \subseteq U$, sections $t \in \Gamma(V, \F)$:} \\
  &\qquad\qquad
      \text{for all open~$W \subseteq V$:} \\
  &\qquad\qquad\qquad\qquad
        \text{$\alpha_W(s|_W) = \alpha_W(t|_W)$ implies $s|_W = t|_W$} \\[0.5em]
  \Longleftrightarrow\ &
    \text{for all open~$U \subseteq X$, sections $s, t \in \Gamma(U, \F)$:} \\
  &\qquad\qquad
        \text{$\alpha_U(s|_U) = \alpha_U(t|_U)$ implies $s|_U = t|_U$} \\[0.5em]
  \Longleftrightarrow\ &
    \text{$\alpha$ is a monomorphism of sheaves}
\end{align*}
The corner quotes ``$\speak{\ldots}$'' indicate that translation into formal
language is left to the reader. Similarly,~$\alpha$ is an epimorphism of
sheaves if and only if, from the internal perspective,~$\alpha$ is a
surjective map. Notice that injectivity and surjectivity are
notions of a simple element-based language. The Kripke--Joyal semantics
takes care to properly handle \emph{all} sections, not only global ones.
\end{ex}

The rules are not all arbitrary. They are finely concerted to make the
following propositions true, which are crucial for a proper appreciation of the
internal language.

\begin{prop}[Locality of the internal language]
Let~$U = \bigcup_i U_i$ be covered by open subsets. Let~$\varphi$
be a formula over~$U$. Then
\[ U \models \varphi \qquad\text{iff}\qquad
  \text{$U_i \models \varphi$ for each $i$}. \]
\end{prop}
\begin{proof}Induction on the structure of~$\varphi$. Note that the canceled
rules would make this proposition false.\end{proof}

As a corollary, one may restrict the open coverings and universal
quantifications in the the definition of the Kripke--Joyal semantics
(Table~\ref{table:kripke-joyal}) to open subsets of some basis of the topology.
For instance, if~$X$ is a scheme, one may restrict to affine open subsets.

Furthermore, the proposition shows that the internal language is monotone in
the following sense: If~$U \models \varphi$, and~$V$ is an open subset of~$U$,
then~$V \models \varphi$. (This follows by applying the proposition to the
trivial covering~$U = V \cup U$.)

\begin{prop}[Soundness of the internal language]
If a formula~$\varphi$ implies a further formula~$\psi$ in intuitionistic logic, then
$U \models \varphi$ implies $U \models \psi$.
\end{prop}
\begin{proof}
Proof by induction on the structure of formal intuitionistic proofs; we are to
show that any inference rule of intuitionistic logic is satisfied by the
Kripke--Joyal semantics. For instance, there is the following rule governing
disjunction:
\begin{quote}
If~$\varphi \vee \psi$ holds, and both $\varphi$ and $\psi$ imply a further
formula~$\chi$, then~$\chi$ holds.
\end{quote}
So we are to prove that if~$U \models \varphi \vee \psi$, $U \models (\varphi
\Rightarrow \chi)$, and $U \models (\psi \Rightarrow \chi)$, then $U \models \chi$.
This is done as follows: By assumption, there exists a covering~$U = \bigcup_i
U_i$ such that on each~$U_i$, $U_i \models \varphi$ or $U_i \models \psi$.
Again by assumption, we may conclude that~$U_i \models \chi$ for each~$i$. The statement
follows because of the locality of the internal language.

A complete list of which rules are to prove is
in Appendix~\ref{appendix:inference-rules}.
\end{proof}

In particular, if a formula~$\psi$ has an unconditional intuitionistic proof,
then~$U \models \psi$.

The restriction to intuitionistic logic is really necessary at this point. We
will encounter many examples of classically equivalent internal statements whose
translation using the Kripke--Joyal semantics are wildly different. In
particular, our treatment of modal operators to understand spreading of
properties from points to neighbourhoods depends on having the ability to make
finer distinctions -- distinctions which are not visible in classical logic.
At the end of this section there are some notes on what the restriction to
intuitionistic logic amounts to in practice.

\XXX{Put rules into an appendix and give some explanation regarding contexts
etc. Don't forget the rules for $\in$, $\bigwedge$, $\bigvee$.}

Because of the multitude of quantifiers, literal translations of internal statements
can sometimes get slightly unwieldy. There are simplification rules for certain
often-occuring special cases:
\begin{prop}\label{prop:simplification}
    \[ \renewcommand{\arraystretch}{1.3}\begin{array}{@{}lcl@{}}
      U \models \forall s\?\F\_ \forall t\?\G\_ \varphi(s,t)
      &\Longleftrightarrow&
      \text{for all open~$V \subseteq U$,} \\
      && \text{sections~$s \in \Gamma(V,\F)$, $t \in \Gamma(V,\G)$:
      $V \models \varphi(s,t)$} \\[0.3em]
      U \models \forall s\?\F\_ \varphi(s) \Rightarrow \psi(s)
      &\Longleftrightarrow&
      \text{for all open~$V \subseteq U$, sections~$s \in \Gamma(V,\F)$:} \\
      &&\qquad\qquad \text{$V \models \varphi(s)$ implies $V \models \psi(s)$}
      \\[0.3em]
      U \models \exists!s\?\F\_ \varphi(s)
      &\Longleftrightarrow&
      \text{for all open~$V \subseteq U$,} \\
      &&
      \text{there is exactly one section~$s \in \Gamma(V,\F)$ with:} \\
      &&\qquad\qquad V \models \varphi(s)
    \end{array} \]
\end{prop}
\begin{proof}Straightforward. By way of example, we prove the existence claim
in the ``only if'' direction of the last rule. (Note that this rule formalizes
the saying ``unique existence implies global existence''.) By definition of~$\exists!$, it
holds that
\[ U \models \exists s\?\F\_ \varphi(s) \]
and
\[ U \models \forall s,t\?\F\_ \varphi(s) \wedge \varphi(t) \Rightarrow s = t.  \]
Let~$V \subseteq U$ be an arbitrary open subset. Then there exist local
sections~$s_i \in \Gamma(V_i,\F)$ such that~$V_i \models \varphi(s_i)$, where~$V
= \bigcup_i V_i$ is an open covering. By the locality of the internal language,
on intersections it holds that~$V_i \cap V_j \models \varphi(s_i)$, so by the
uniqueness assumption, it follows that the local sections agree on intersections.
They therefore glue to a section~$s \in \Gamma(V,\F)$. Since~$V_i \models
\varphi(s)$ for all~$i$, the locality of the internal language allows us to
conclude that~$V \models \varphi(s)$.
\end{proof}

\begin{rem}Note that~$\Sh(X) \models \neg\varphi$ is in general a much stronger
statement than merely supposing that~$\Sh(X) \models \varphi$ does not hold:
The former always implies the latter (unless~$X = \emptyset$, in which case
\emph{any} internal statement is true), but the converse does not hold: The
former statement means that~$U = \emptyset$ is the \emph{only} open subset on
which~$\varphi$ holds.\end{rem}


\subsection{Internal constructions}
\label{sect:internal-constructions}
The Kripke--Joyal semantics defines the
interpretation of internal \emph{statements}. The interpretation of internal
\emph{constructions} is given by the following definition.

\begin{defn}The interpretation of an internal construction~$T$
is denoted by~$\brak{T} \in \Sh(X)$ and given by the following rules.
\begin{itemize}\item If~$\F$ and~$\G$ are sheaves, $\brak{\F \times \G}$ is the
categorical product of~$\F$ and~$\G$ (\ie their product as presheaves).
\item If~$\F$ and~$\G$ are sheaves, $\brak{\F \amalg \G}$ is the categorical
coproduct of~$\F$ and~$\G$, \ie the sheafification of the presheaf
$U \mapsto \Gamma(U,\F) \amalg \Gamma(U,\G)$.
\item If~$\F$ is a sheaf, the interpretation~$\brak{\P(\F)}$ of the power set
construction is the sheaf given by
\[ \text{$U \subseteq X$ open} \quad\longmapsto\quad \{ \G \hookrightarrow \F|_U \}, \]
\ie sections on an open set~$U$ are subsheaves of~$\F|_U$ (either literally
or isomorphism classes of arbitrary monomorphisms into~$\F|_U$).
\item If~$\F$ is a sheaf and~$\varphi(s)$ is a formula containing a free
variable~$s\?\F$, the interpretation~$\brak{\{s\?\F\,|\,\varphi(s)\}}$ is given
by the subpresheaf of~$\F$ defined by
\[ \text{$U \subseteq X$ open} \quad\longmapsto\quad \{ s \in \Gamma(U,\F) \ |\
  U \models \varphi(s) \}. \]
Note that by the locality of the internal language, this presheaf is in fact a
sheaf.
\end{itemize}
\end{defn}

The definition is made in such a way that, from the internal perspective, the
constructions enjoy their expected properties. For instance, it holds that
\[ \Sh(X) \models
  \bigl(\forall x\?\brak{\{s\?\F \,|\, \varphi(s)\}}\_ \psi(x)\bigr)
  \Longleftrightarrow
  \bigl(\forall x\?\F\_ \varphi(x) \Rightarrow \psi(x)\bigr). \]
We gloss over several details here. See~\cite[Section~D4.1]{johnstone:elephant} for
a proper treatment.

Morphisms can internally be constructed by appealing to the \emph{principle of
unique choice}: Let~$\varphi(s,t)$ be a formula with free variables of
type~$s\?\F$, $t\?\G$. Assume
\[ \Sh(X) \models \forall s\?\F\_ \exists!t\?\G\_ \varphi(s,t). \]
Then there is one and only one morphism~$\alpha : \F \to \G$ of sheaves such
that for any local section~$s \in \Gamma(U,\F)$, $\Sh(X) \models
\varphi(s,\alpha(s))$. This follows from the meaning of unique existence with
the Kripke--Joyal semantics (Proposition~\ref{prop:simplification}).

An important application is showing that two sheaves~$\F$ and~$\G$ are
isomorphic (usually as objects with more structure, for instance sheaves of
modules). To this end, it suffices to give a formula~$\varphi(s,t)$ satisfying,
in addition to the condition above, the condition
$\Sh(X) \models \forall t\?\G\_ \exists! s\?\F\_ \varphi(s,t)$,
expressing that the induced morphism~$\alpha$ is a bijective map from the
internal perspective. Note that this implies the statement
\[ \Sh(X) \models \exists \alpha\?\HOM(\F,\G)\_ \speak{$\alpha$ is bijective},
\]
but this statement is strictly weaker: Its interpretation with the
Kripke--Joyal semantics is that the sheaves~$\F$ and~$\G$ are \emph{locally}
isomorphic.


\subsection{Geometric formulas and
constructions}\label{sect:geometric-formulas-and-constructions}
In formal and categorical logic so-called geometric formulas play a
special role. They are named that way because, in a sense which can be made
precise, their meaning is preserved under pullback with geometric morphisms.
\begin{defn}A formula is \emph{geometric} if and only if it consists only of
\[ {=} \quad {\in} \quad {\top} \quad {\bot} \quad {\wedge} \quad {\vee} \quad
{\bigvee} \quad {\exists}, \]
but not~``$\bigwedge$'' nor ``$\Rightarrow$'' nor~``$\forall$'' (and thus
not~``$\neg$'' either, since this is defined using~``$\Rightarrow$'').
A \emph{geometric implication} is a formula of the form
\[ \forall \cdots \forall\_ (\cdots) \Rightarrow (\cdots) \]
with the bracketed subformulas being geometric.
\end{defn}
The \emph{parameters} of a formula~$\varphi$ are the sheaves
being quantified over, sections of sheaves appearing as terms, and morphisms of
sheaves appearing as function symbols in~$\varphi$.
We say that a formula~$\varphi$ holds \emph{at a point~$x \in X$} if and only
if the formula obtained by substituting all parameters in~$\varphi$ with their
stalks at~$x$ holds in the usual mathematical sense.

\begin{lemma}\label{lemma:geometric-stalk-neighbourhood}
Let~$x \in X$ be a point. Let~$\varphi$ be a geometric formula (over some open
neighbourhood~V of~$x$).
Then~$\varphi$ holds at~$x$ if and only if there exists an open neighbourhood~$U
\subseteq X$ of~$x$ (contained in~V) such that~$\varphi$ holds on~$U$.
\end{lemma}
\begin{proof}This is a very general instance of the phenomenon that sometimes,
truth at a point spreads to truth on a neighbourhood. It can be proven by
induction on the structure of~$\varphi$, but we will give a more conceptual
proof later (Corollary~\ref{cor:geometric-spreading}).
\end{proof}

This lemma is in fact a very useful metatheorem. We will properly discuss its
significance in Section~\ref{sect:spreading}. For now, we just use it to prove a
simple criterion for the internal truth of a geometric implication; we will
apply this criterion many times.

\begin{cor}\label{cor:geometric-implication}
A geometric implication holds on~$X$ if and only if it holds at
every point of~$X$.\end{cor}
\begin{proof}For notational simplicity, we consider a geometric implication of
the form
\[ \forall s\?\F\_ \varphi(s) \Rightarrow \psi(s). \]
For the ``only if'' direction, assume that this formula holds on~$X$ and let~$x
\in X$ be an arbitrary point. Let~$s_x \in \F_x$ be the germ of an arbitrary
local section~$s$ of~$\F$ and assume that~$\varphi(s)$ holds at~$x$. Then by
the lemma, it follows that~$\varphi(s)$ holds on some open neighbourhood of~$x$. By
assumption,~$\psi(s)$ holds on this neighbourhood as well. Again by the
lemma,~$\psi(s)$ holds at~$x$.

For the ``if'' direction, assume that the geometric implication holds at every
point. Let~$U \subseteq X$ be an arbitrary open subset and let~$s \in
\Gamma(U,\F)$ be a local section such that~$\varphi(s)$ holds on~$U$. By the
lemma and the locality of the internal language, to show that~$\psi(s)$ holds
on~$U$, it suffices to show that~$\psi(s)$
holds at every point of~$U$. This is clear, since again by the
lemma,~$\varphi(s)$ holds at every point of~$U$.
\end{proof}

\begin{ex}Injectivity and surjectivity are geometric implications (surjectivity
can be spelled~$\forall y\?\G\_ (\top \Rightarrow \exists x\?\F\_ \alpha(x) =
y)$). Thus the corollary gives a deeper reason for the well-known fact that a
morphism of sheaves is a monomorphism \resp an epimorphism if and only if it is
stalkwise injective \resp surjective.\end{ex}

A construction is \emph{geometric} if and only if it commutes with pullback
under arbitrary geometric morphisms. We do not want to discuss the notion of
geometric morphisms here; suffice it to say that calculating the stalk at a
point~$x \in X$ is an instance of such a pullback. Among others, the following
constructions are geometric:
\begin{itemize}
\item finite product: $(\F \times \G)_x \cong \F_x \times \G_x$
\item finite coproduct: $(\F \amalg \G)_x \cong \F_x \amalg \G_x$
\item arbitrary coproduct: $(\coprod_i \F_i)_x \cong \coprod_i (\F_i)_x$
\item set comprehension with respect to a \emph{geometric} formula~$\varphi$:
\[ \brak{\{ s\?\F \,|\, \varphi(s) \}}_x \cong \{ [s]\in\F_x \,|\,
\text{$\varphi(s)$ holds at $x$} \} \]
\item free module: $(\R\langle \F \rangle)_x \cong \R_x\langle \F_x
\rangle$ ($\R$ a sheaf of rings, $\F$ a sheaf of sets)
\item localization of a module: $\F[\S^{-1}]_x \cong \F_x[\S_x^{-1}]$
\end{itemize}
Note that compatibility with taking stalks is not sufficient for geometricity.
It is just the most easily visualized requirement.
The following constructions are not in general geometric:
\begin{itemize}
\item arbitrary product
\item set comprehension with respect to a non-geometric formula
\item powerset
\item internal Hom: $\HOM(\F,\G)_x \not\cong \Hom(\F_x,\G_x)$
\end{itemize}


\subsection{Appreciating intuitionistic logic}
\label{sect:appreciating-intuitionistic-logic}
The principal (and only) difference between classical and intuitionistic logic
is that in classical logic, the axioms schemes of \emph{excluded middle} and
\emph{double negation elimination} are added.
\[ \varphi \vee \neg\varphi \qquad\qquad \neg\neg\varphi \Rightarrow \varphi \]
A classically trained mathematician might legitimately wonder why one should
drop these axioms: Are they not obviously true? The pragmatic answer to this
question is that the translations of these axioms with the Kripke--Joyal
semantics are, except for uninteresting special cases of the base space~$X$,
plainly false -- irrespective of one's philosophical convictions. Therefore the
internal language is in general only sound with respect to intuitionistic logic and
not with respect to classical logic. Concretely, there is the following
proposition.
\begin{prop}The internal language of
a~$\mathrm{T}_1$-space~$X$ is \emph{Boolean}, \ie it verifies the classical
axiom schemes displayed above, if and only if~$X$ is discrete.
The internal language of an irreducible or locally Noetherian scheme~$X$ is Boolean if and only if~$X$ has
dimension~$\leq 0$.
\end{prop}
\begin{proof}\label{prop:lang-boolean}
The internal language of~$\Sh(X)$ is Boolean if and only if for
any open subset~$U \subseteq X$ it holds that~$U$ is the only dense open subset
of~$U$. This can be checked manually, by using the definition of the
Kripke--Joyal semantics, but we'll be able to give a more conceptual proof
later (Lemma~\ref{lemma:boolean-dense}). The first claim is then an exercise in
point-set topology, while the second is more difficult
(Corollary~\ref{cor:boolean-dim0}).
\end{proof}

However, there is also a more satisfying answer, which furthermore
illuminates how to intuitively picture intuitionistic mathematics.
Namely, when doing intuitionistic mathematics, we use the same formal symbols as classically, but with
\emph{a different intended meaning}. For instance, the classical reading of an
existential statement like~$\exists x\?A\_ \varphi(x)$ is that there exists
some element~$x \? A$ with the property~$\varphi(x)$. In contrast, its
intuitionistic reading is that such an element can actually be
\emph{constructed}, \ie explicitly given in some form. This is a much stronger
statement. Classically, a proof that it is \emph{not} the case that such an
element does \emph{not} exist -- formally $\neg\neg \exists x\?A\_ \varphi(x)$
(or, equivalently even in intuitionistic mathematics, $\neg\forall x\? A\_
\neg\varphi(x)$) suffices to demonstrate the existential statement; this is not
so in intuitionistic mathematics.

Similarly, the intuitionistic meaning of a disjunction~$\varphi \vee \psi$ is
not only that one of the disjuncts is true, but that one can explicitly state
which case holds. It is in general not enough to show that it is impossible
that both~$\varphi$ and~$\psi$ fail.

In this picture, it is obvious that one should not adopt the law of excluded
middle or the principle of double negation elimination as axioms. Note that we
do not \emph{reject} those axioms in the sense of postulating their
converses either, we simply don't use them. Therefore any intuitionistically
true result is also true classically. In fact, for some special instances,
these two classical axioms do hold intuitionistically. For example, any natural
number is zero or is not zero -- this is not a triviality, but can be proven by
induction.\footnote{The analogous statement about real numbers cannot be
shown. Intuitively, for a number given by a decimal expansion starting
with~$0.0000\ldots$ one cannot decide whether the string of zeros will continue
indefinitely or whether eventually a non-zero digit will occur. This argument
can be made rigorous. The analogous statement about algebraic numbers
\emph{can} be proven; the information contained in a witness of algebraicity (a
normed polynomial which the given number is a zero of) suffices to make the
case distiction~\cite[Chapter~VI.1, p.~140]{mines-richman-ruitenburg:constructive-algebra}.}

A consequence of not adopting these axioms is that proofs by contradiction are
not generally justified; they are intuitionistically acceptable only for those
statements which can be proven to be true or false. Note that a proof of a
\emph{negated formula} is not the same as a proof by contradiction. For
instance, the usual proof that~$\sqrt{2}$ is not rational is
intuitionistically perfectly fine: From the assumption that~$\sqrt{2}$ is
rational one deduces a contradiction~($\bot$). This is exactly the definition
of~$\neg(\speak{$\sqrt{2}$ is rational})$.

A more positive consequence of not adopting the law of excluded middle and the
principle of double negation elimination is that intuitionistically, we can
make \emph{finer distinctions}. For instance, for a formula~$\varphi$, the doubly
negated formula~$\neg\neg\varphi$ (``\notnot~$\varphi$'') is a certain kind of weakening of~$\varphi$:
If~$\varphi$ holds, then~$\neg\neg\varphi$ does as well, while the converse can
not be shown in general.\footnote{A detailed proof of the correct implication
goes as follows: Assume~$\varphi$. We are to show~$\neg\neg\varphi$, \ie
$(\neg\varphi \Rightarrow \bot)$. So assume~$\neg\varphi$, we are to
show~$\bot$. Since~$\varphi$ and~$\varphi \Rightarrow \bot$,~$\bot$ indeed
follows.} An example from everday life runs as follows: If in the morning you
can't find the key for your appartment, but you know that it must hide
somewhere since you used it to open the door in the evening before, you
intuitionistically know~$(\neg\neg\exists x\_ \speak{the key is at position
$x$})$, but you cannot claim the unnegated proposition. One cannot model this
distinction with pure classical logic.

Double negation also has a concrete geometric meaning with the
Kripke--Joyal semantics. Namely,~$X \models \neg\neg\varphi$ holds if and
only if there is a dense open subset~$U$ of~$X$ such that~$U \models \varphi$.
This is of course a weaker statement that~$X \models \varphi$.
In Section~\ref{sect:modalities}, we will discuss this fact and other
\emph{modal operators} in more detail. For instance, there is a similarly defined modal
operator~$\Box$ such that~$X \models \Box\varphi$ if and only if there is an
open neighbourhood~$U$ of a given point~$x$ such that~$U \models \varphi$. Also
there is a operator~$\Box$ such that~$X \models \Box\varphi$ if and only
if~$\varphi$ holds on a scheme-theoretically dense open subset.

For future reference, note that if~$\varphi \Rightarrow \psi$,
then also~$\neg\neg\varphi \Rightarrow \neg\neg\psi$; and note that weakening
twice has no further effect, \ie~$\neg\neg\neg\neg\varphi \Leftrightarrow
\neg\neg\varphi$.\footnote{In fact, negating thrice is the same as negating
once: Assume~$\neg\neg\neg\varphi$. We are to show~$\neg\varphi$. So
assume~$\varphi$, we are to show~$\bot$. Since~$\varphi$,~$\neg\neg\varphi$.
By~$\neg\neg\neg\varphi$,~$\bot$ follows.}

A classical mathematician might then ask which classical results are valid
intuitionistically. The answer is that in linear and commutative algebra, most
of the basic theorems stay valid, provided one exercises some caution in
formulating them (for instance, one should not arbitrarily weaken assumptions
by introducing double negations). This is because the proofs of these
statements are usually direct; if intuitionistically unacceptable case
distictions do occur, they can often be eliminated by streamlining the proof.

Consider as a simple example the proposition that the kernel of a linear map is
a linear subspace. The case distiction ``either the kernel consists just of the
zero vector, in which case the claim is trivial, or otherwise \ldots'' is not
intuitionistically acceptable, but it can be entirely dispensed with: The proof
for the general case works in the special case just as well.

Finally, we should clarify the status of the axiom of choice. This axiom, which
is strictly speaking not part of classical logic, but of a classical set
theory, is not accepted in an intuitionistic context: By \emph{Diaconescu's
theorem}, it implies the law of excluded middle in presence of the other axioms
of set theory.

The standard reference for intuitionistic algebra is a textbook by Ray Mines,
Fred Richman and Wim
Ruitenburg~\cite{mines-richman-ruitenburg:constructive-algebra}, the standard
reference for intuitionistic analysis is a book by Erret Bishop and Douglas
Bridges~\cite{bishop-bridges:constructive-analysis}. Further explanations and
pointers to relevant literature can be found in an expository article and a
recorded lecture by Andrej Bauer~\cite{bauer:int-mathematics,bauer:video}. A
recent survey of intuitionistic logic from a historical and logical point of
view is~\cite{melikhov:intuitionistic-logic}.

\begin{rem}For ease of exposition, we work in a classical metatheory. This
means that we allow ourselves to occasionally use the law of excluded middle
and the axiom of choice when reasoning \emph{about} the internal language. In
particular, we have the theory of schemes as commonly presented at our
disposal. But we should note that this concession is really a cop out, and that
it would be better to develop an intuitionistic theory of schemes. If this were
done, one could extend our approach to understand morphisms of schemes from an
internal point of view -- a morphism~$Y \to X$ would internally look like a
morphism~$Y \to \pt$. See Section~\ref{sect:relative-spectrum} for details.\end{rem}


\section{Sheaves of rings}

Recall that a \emph{sheaf of rings} can be categorically described as a
sheaf of sets~$\R$ together with maps of sheaves $+, \cdot : \R \times \R \to
\R$ and global elements~$0, 1$ such that certain axioms hold. For instance, the
axiom on the commutativity of addition is rendered in diagrammatic form as
follows:
\[ \xymatrix{
  \R \times \R \ar[rr]^{\mathrm{swap}} \ar[rd]_{+} && \R \times \R \ar[ld]^{+} \\
  & \R
} \]

From the internal perspective, a sheaf of rings looks just like a plain ring.
This is the content of the following proposition.

\begin{prop}\label{prop:rings-internally}
Let~$X$ be a topological space. Let~$\R$ be a sheaf of sets on~$X$.
Let~$+, \cdot : \R \times \R \to \R$ be maps of sheaves and let~$0, 1$ be
global elements of~$\R$. Then these data define a sheaf of rings if and only
if, from the internal perspective, these data fulfill the usual equational ring
axioms.\end{prop}
\begin{proof}We only discuss the commutativity axiom. The internal statement
\[ \Sh(X) \models \forall x,y\?\R\_ x + y = y + x \]
means that for any open subset~$U \subseteq X$ and any local sections~$x,y \in
\Gamma(U,\R)$, it holds that~$x + y = y + x \in \Gamma(U,\R)$. This is
precisely the external commutativity condition.
\end{proof}

\begin{lemma}Let~$X$ be a topological space. Let~$\R$ be a sheaf of rings
on~$X$. Let~$f$ be a global section of~$\R$. Then the following statements are
equivalent:
\begin{enumerate}
\item $f$ is invertible from the internal point of view, \ie $\Sh(X) \models
\exists g\?\R\_ fg = 1$.
\item $f$ is invertible in all stalks~$\R_x$.
\item $f$ is invertible in~$\Gamma(X,\R)$.
\end{enumerate}
\end{lemma}
\begin{proof}Since invertibility is a geometric implication, the equivalence of
the first two statements is clear. Also, it is obvious that the third statement
implies the other two. For the remaining direction, note that the
uniqueness of inverses in rings can be proven intuitionistically. Therefore, if~$f$ is invertible
from the internal point of view, it actually holds that
\[ \Sh(X) \models \exists! g\?\R\_ fg = 1. \]
Since unique internal existence implies global existence
(Proposition~\ref{prop:simplification}), this shows that the first statement
implies the third.
\end{proof}


\subsection{Reducedness}\label{sect:reducedness} Recall that a scheme~$X$ is \emph{reduced} if and only
if all stalks~$\O_{X,x}$ are reduced rings. Since the condition on a ring~$R$
to be reduced is a geometric implication,
\[ \forall s\?R\_ \Bigl(\bigvee_{n \geq 0} s^n = 0\Bigr) \Longrightarrow s = 0, \]
we immediately obtain the following characterization of reducedness in the
internal language:
\begin{prop}\label{prop:reduced-ring}
A scheme~$X$ is reduced iff, from the internal point of view, the
ring~$\O_X$ is reduced.\end{prop}


\subsection{Locality} Recall the usual definition of a local ring: a ring
possessing exactly one maximal ideal. This is a so-called \emph{higher-order
condition} since it involves quantification over subsets. It is also not of a
geometric form. Therefore, for our purposes, it is better to
adopt the following elementary definition of a local ring.
\begin{defn}A \emph{local ring} is a ring~$R$ such that~$1 \neq 0$ in~$R$ and
for all~$x,y \? R$
\[ \text{$x+y$ invertible} \quad\Longrightarrow\quad
  \text{$x$ invertible}\ \vee\ \text{$y$ invertible}. \]
\end{defn}
In classical logic, it is an easy exercise to show that this definition is
equivalent to the usual one. In intuitionistic logic, we would need to be
more precise in order to even state the question of equivalence, since
intuitionistically, the notion of a maximal ideal bifurcates into several
non-equivalent notions.\footnote{For instance, should a maximal ideal~$\mmm$ be
such that if~$\nnn$ is any ideal with~$\mmm \subseteq \nnn \subsetneq (1)$,
then~$\mmm = \nnn$? Or should the condition be that if~$\nnn$ is any ideal
with~$\mmm \subseteq \nnn$, then~$\mmm = \nnn$ or~$\nnn = (1)$?
Intuitionistically, the latter condition is stronger than the former.}

\begin{prop}\label{prop:local-ring}
In the internal language of a scheme~$X$ (or a locally ringed
space), the ring~$\O_X$ is a local ring.\end{prop}
\begin{proof}The stated locality condition is a conjunction of two geometric
implications (the first one being~$1 = 0 \Rightarrow \bot$, the second being
the displayed one) and holds on each stalk.\end{proof}

\begin{rem}When first exposed to locally ringed spaces, one might ask why the
requirement is that the \emph{stalks}~$\O_{X,x}$ are local rings, instead of the
easier-to-define sets of sections~$\O_X(U)$. This question has of course a good
geometric answer. Using the internal language, it also has a purely formal
answer: The requirement that the stalks are local rings is precisely the
requirement that the ring~$\O_X$ is a local ring from the perspective of the
internal language of~$X$.
\end{rem}


\subsection{Field properties} From the internal point of view, the structure
sheaf~$\O_X$ of a scheme~$X$ is \emph{almost} a field, in the sense that any
element which is not invertible is nilpotent. This is a genuine property of
schemes, not shared with arbitrary locally ringed spaces. It is also a specific
feature of the internal universe: Neither the local rings~$\O_{X,x}$ nor the
rings of local sections~$\Gamma(U,\O_X)$ have this property in general.

\begin{prop}\label{prop:neginvnilpotent}Let~$X$ be a scheme. Then
\[ \Sh(X) \models \forall s\?\O_X\_ \neg(\speak{$s$ invertible}) \Rightarrow
\speak{$s$ nilpotent}. \]
\end{prop}
\begin{proof}By the locality of the internal language and since~$X$ can be
covered by open affine subsets, it is enough to show that for any affine
scheme~$X = \Spec A$ and any global function~$s \in \Gamma(X,\O_X) = A$ it holds
that
\[ X \models \neg(\speak{$s$ invertible}) \quad\text{implies}\quad
  X \models \speak{$s$ nilpotent}. \]
The meaning of the antecedent is that any open subset on which~$s$ is
invertible is empty. This implies in particular that the standard open subset~$D(s)$ is
empty. This means that~$s$ is an element of any prime ideal of~$A$, thus
nilpotent, and therefore implies the a priori weaker statement~$X \models \speak{$s$
nilpotent}$ (which would allow~$s$ to have different indices of nilpotency on
an open covering).
\end{proof}

\begin{rem}In classical logic, the statement ``not invertible implies
nilpotent'' is equivalent to ``any element is invertible or nilpotent''.
However, in intuitionistic logic, the latter is strictly stronger than the
former. We will see in the next section
(Corollary~\ref{cor:scheme-dimension-zero}) that the structure sheaf of a
scheme fulfills the latter condition if and only if the scheme is
zero-dimensional (or empty).\end{rem}

\begin{cor}\label{cor:field-reduced}
Let~$X$ be a scheme. If~$X$ is reduced, the ring~$\O_X$ is a field
from the internal point of view, in the sense that
\[ \Sh(X) \models \forall s\?\O_X\_ \neg(\speak{$s$ invertible}) \Rightarrow
s=0. \]
Conversely, if~$\O_X$ is a field in this internal sense, then~$X$ is reduced.\end{cor}
\begin{proof}We can prove this purely in the internal language: It suffices to
give an intuitionistic proof of the fact that a local ring which satisfies the
condition of the previous proposition fulfills the stated field condition if
and only if it is reduced. This is straightforward.
\end{proof}

This field property is very useful. We will put it to good use when giving a
simple proof of the fact that~$\O_X$-modules of finite type on a reduced scheme
are locally free on a dense open subset (Lemma~\ref{lemma:locally-free-dense}).
Note that the field property only holds precisely as stated in the corollary;
the classically equivalent condition that any element is invertible or zero is
intuitionistically stronger. This is a common phenomenon in intuitionistic
mathematics: Classically equivalent notions may bifurcate into related but
inequivalent notions intuitionistically.

The observation that the structure sheaf is (almost) a field is attributed by
Tierney to Mulvey~\cite[p.~209]{tierney:spectrum}.
Tierney also states that ``its precise significance is still somewhat
obscure'' (ibid). We think that it's significant as a special case of the
following more general proposition,
which says that we can deduce a certain unconditional
statement from the premise that, under the assumption that some element~$f\?\O_X$ is invertible, an element~$s\?\O_X$ is zero. This is
interesting on its own, but will be of particular importance in understanding
quasicoherence from the internal point of view (Section~\ref{sect:qcoh}) and
interpreting the relative spectrum as an internal spectrum
(Section~\ref{sect:relative-spectrum}).

\begin{prop}\label{prop:cond-zero}
Let~$X$ be a scheme. Then
\[ \Sh(X) \models
  \forall f\?\O_X\_
  \forall s\?\O_X\_
  (\speak{$f$ \inv} \Rightarrow s = 0) \Longrightarrow
  \textstyle
  \bigvee_{n \geq 0} f^n s = 0. \]
\end{prop}
\begin{proof}It is enough to show that for any affine scheme~$X = \Spec A$ and
any global functions~$f, s \in A$ such that
\[ X \models (\speak{$f$ \inv} \Rightarrow s = 0), \]
it holds that $X \models \textstyle \bigvee_{n \geq 0} f^n s = 0$. This
indeed follows, since by assumption such a function~$s$ is zero on~$D(f)$, \ie $s$
is zero as an element of~$A[f^{-1}]$.
\end{proof}

Proposition~\ref{prop:neginvnilpotent} follows from this proposition by
setting~$s \defeq 1$.


\subsection{Krull dimension}\label{sect:krull-dimension}
Recall that the \emph{Krull dimension} of a
ring is usually defined as the supremum of the lengths of strictly
ascending chains of prime ideals. As with the classical definition of a local ring,
this definition does not lead to a well-behaved notion in an intuitionistic
context. Furthermore, it is a higher-order condition, so interpreting it
with the Kripke--Joyal semantics is a bit unwieldy.

Luckily, there is an elementary definition of the Krull dimension which works
intuitionistically and which is classically equivalent to the usual notion. It
was found by Thierry Coquand and Henri Lombardi, building upon work by André
Joyal and Luis Español~\cite{dyn:krull-integral,dyn:char-krull}, and can be
used to give a short proof that~$\dim k[X_1,\ldots,X_n] = n$, where~$k$ is a
field~\cite{dyn:krull-dim-polynomial-ring}.

\begin{defn}Let~$R$ be a ring. A \emph{complementary sequence} for a
sequence~$(a_0,\ldots,a_n)$ of elements of~$R$ is a sequence~$(b_0,\ldots,b_n)$
such that the following inclusions of radical ideals hold:
\[ \renewcommand{\arraystretch}{1.3}
\left\{\begin{array}{@{}rcl@{}}
  \sqrt{(1)} &\subseteq& \sqrt{(a_0,b_0)} \\
  \sqrt{(a_0 b_0)} &\subseteq& \sqrt{(a_1,b_1)} \\
  \sqrt{(a_1 b_1)} &\subseteq& \sqrt{(a_2,b_2)} \\
  &\vdots \\
  \sqrt{(a_{n-1} b_{n-1})} &\subseteq& \sqrt{(a_n,b_n)} \\
  \sqrt{(a_n b_n)} &\subseteq& \sqrt{(0)}
\end{array}\right. \]
The ring~$R$ is \emph{of Krull dimension~$\leq n$} if
and only if for any sequence~$(a_0,\ldots,a_n)$ there exists a
complementary sequence. (The ring~$R$ is trivial if and only if it is
of Krull dimension~$\leq -1$.)
\end{defn}
Note that unlike the usual definition, this definition posits only a condition
on elements and not on ideals. It is thus of a simpler logical form.
(The radical ideals appear only for convenience. We will dispose of them in the
proof of Proposition~\ref{prop:dimension-scheme-ox}.)
Also note that we do not define the Krull dimension of a ring as some natural
number (this is intuitionistically not possible for general rings). Instead, we
only define what it means for the Krull dimension to be less than or equal to
a given natural number.

For the following, no intuition about the definition is needed; however, we
feel that some motivation might be of use. Recall that we can picture inclusions of
radical ideals geometrically by considering standard open subsets~$D(f) = \{
\ppp \in \Spec R \,|\, f \not\in \ppp \}$: The inclusion~$\sqrt{(f)} \subseteq
\sqrt{(g,h)}$ holds if and only if~$D(f) \subseteq D(g) \cup D(h)$, and
intersections are calculated by products, \ie~$D(f) \cap D(g) = D(fg)$.

The condition that~$(b_0,\ldots,b_n)$ is complementary to~$(a_0,\ldots,a_n)$
thus means that~$D(a_0)$ and~$D(b_0)$ cover the whole of~$\Spec R$; that their
intersection is covered by~$D(a_1)$ and~$D(b_1)$; that in turn their
intersection is covered by~$D(a_2)$ and~$D(b_2)$; \ldots; and that finally, the
intersection of~$D(a_n)$ and~$D(b_n)$ is empty.

For the special case~$n = 0$, the condition that~$R$ is of Krull
dimension~$\leq 0$ means that for any element~$a_0$ there exists an
element~$b_0$ such that~$D(a_0)$ and~$D(b_0)$ cover~$\Spec R$ and are disjoint.

The definition of the Krull dimension can be written in such a way as to mimic the
definition of the inductive Menger--Urysohn dimension of topological
spaces~\cite[Section~1]{dyn:krull-integral}.

\begin{thm}Let~$R$ be a ring.
\begin{enumerate}
\item In classical logic, the ring~$R$ is
of Krull dimension~$\leq n$ if and only if its Krull dimension
as usually defined using chains of prime ideals is less than or equal to~$n$.
\item If the ring~$R$ is
of Krull dimension~$\leq n$, the radical of any finitely generated ideal is
equal to the radical of some ideal which can be generated by~$n+1$ elements.
This holds intuitionistically, and there is an explicit algorithm for computing
the reduced set of generators from the given ones. (Kronecker's theorem)
\end{enumerate}
\end{thm}
\begin{proof}See~\cite[Theorem~1.2]{dyn:krull-integral} for the first
statement. The proof relies on the observation that~$\dim R \leq n$ if and only
if~$\dim R[S_x^{-1}] \leq n-1$ for all~$x \in R$, where~$S_x = x^\NN (1+xR)
\subseteq R$. We put the second statement only to demonstrate that the
definition of the Krull dimension is constructively sensible. It follows from
the identity~$\sqrt{(x,a_0,\ldots,a_n)} =
\sqrt{(a_0-xb_0,\ldots,a_n-xb_n)}$, where~$(b_0,\ldots,b_n)$ is a complementary
sequence for~$(a_0,\ldots,a_n)$.
\end{proof}

We can apply the constructive theory of Krull dimension to the structure
sheaf~$\O_X$ of a scheme~$X$ as follows. Note that the condition that a
scheme~$X$ has dimension exactly~$n$ (in the usual sense using ascending chains
of closed irreducible subsets) is not local -- the dimension may vary on
an open cover; therefore it is not possible to characterize this condition in
the internal language. However, the condition that the dimension of~$X$ is less
than or equal to~$n$ \emph{is} local, thus there is hope that it can be
internalized. And indeed, this is the case.

\begin{prop}\label{prop:dimension-scheme-ox}
Let~$X$ be a scheme. Then:
\[ \dim X \leq n \quad\Longleftrightarrow\quad
  \Sh(X) \models \speak{$\O_X$ is of Krull dimension~$\leq n$}
  \]
\end{prop}
\begin{proof}
% Recall that the topological dimension of~$X$ is defined as the
% supremum of the lengths of strictly ascending chains of irreducible closed
% subsets. It can be calculated as the supremum of the local dimensions~$\dim_x X
% \defeq \inf\{\dim U \,|\, \text{$U$ open neighbourhood of~$x$} \}$, where~$x$
% ranges over all points of~$X$. The local dimension can be characterized
% algebraically: $\dim_x X = \dim \O_{X,x}$.

A condition of the form~``$\sqrt{(f)} \subseteq \sqrt{(g,h)}$''
like in the constructive definition of the Krull dimension is not a geometric
formula when taken on face value. However, it is equivalent to a geometric
condition, namely to
\begin{multline*}
  \exists a,b\?\O_X\_ \bigvee_{m \geq 0} f^m = ag + bh \quad\wedge\quad \\
  \exists u\?\O_X\_ \bigvee_{p \geq 0} g^p = uf \quad\wedge\quad
  \exists v\?\O_X\_ \bigvee_{q \geq 0} h^q = vf.\end{multline*}
Therefore the condition~$\speak{$\O_X$ is of Krull
dimension~$\leq n$}$ is (equivalent to) a geometric implication and thus holds
internally if and only if it holds at every point~$x \in X$. This in turn means that the
Krull dimension of any stalk~$\O_{X,x}$ is less than or equal to~$n$. This is
equivalent to the (Krull) dimension of~$X$ being less than or equal to~$n$.
\end{proof}

We will state and prove a generalization of this lemma about the dimension of closed
subschemes later, as Lemma~\ref{lemma:dim-closed-subscheme}.

If~$X$ is a reduced scheme, we have seen in Corollary~\ref{cor:field-reduced}
that~$\O_X$ is a field from the internal perspective, in the sense that
non-invertible elements are zero. But fields are well-known to be of Krull
dimension zero. Why is this not a contradiction to the proposition just proven?
Intuitionistically, one can only show that fields in the stronger sense that
any element is zero or invertible have Krull dimension zero. Fields in the
weaker sense can have higher Krull dimension, as exhibited by the structure
sheaf of reduced schemes with positive dimension.

For the following corollary, note that if a scheme~$X$ is not empty, $\dim X
\leq 0$ is equivalent to~$\dim X = 0$.
\begin{cor}\label{cor:scheme-dimension-zero}
Let~$X$ be a scheme. Then:
\[ \dim X \leq 0 \quad\Longleftrightarrow\quad
  \Sh(X) \models \forall s\?\O_X\_ \speak{$s$ \inv} \vee \speak{$s$ nilpotent}.
  \]
If furthermore~$X$ is reduced, this is further equivalent to~$\O_X$ being a
field in the strong sense that any element of~$\O_X$ is invertible or zero.
\end{cor}
\begin{proof}By the proposition and the fact that~$\O_X$ is a local ring from
the internal perspective, this is an immediate consequence of
interpreting the following standard fact of ring theory in the internal
language of~$\Sh(X)$: A local ring~$R$ is of Krull
dimension~$\leq 0$ if and only if any element of~$R$ is invertible or
nilpotent.

It is well-known that this holds classically; to make sure that it
holds intuitionistically as well (so that it can be used in the internal
universe), we give a proof of the ``only if'' direction. Let~$a \? R$ be
arbitrary. By assumption on the Krull dimension, there exists an element~$b \?
R$ such that~$\sqrt{(1)} \subseteq \sqrt{(a,b)}$ and~$\sqrt{(ab)} =
\sqrt{(0)}$. The latter means that~$ab$ is nilpotent. Since~$R$ is local, the
former implies that~$a$ is invertible or that~$b$ is invertible. In the first
case, we are done. In the second case, it follows that~$a$ is nilpotent, so we
are done as well.
\end{proof}

As a further corollary note the curious fact that the classicality of the
internal language of~$\Sh(X)$, where~$X$ is a scheme, is tightly coupled with
the properties of the ring~$\O_X$: Internally, the law of excluded middle and
the principle of double negation elimination are ``almost equivalent'' to the
Krull dimension of~$\O_X$ being~$\leq 0$.
\begin{cor}\label{cor:boolean-dim0}
Let~$X$ be a scheme. If the internal language of~$\Sh(X)$ is Boolean, then
$\dim X \leq 0$. The converse holds if~$X$ is irreducible or locally Noetherian.
\end{cor}
\begin{proof}
We show that any element of~$\O_X$ is invertible or nilpotent, therefore
verifying the hypothesis of the previous corollary.
Let~$s\?\O_X$ be given. By assumption, either~$s$ is invertible or~$s$ is not
invertible. In the latter case~$s$ is nilpotent by
Proposition~\ref{prop:neginvnilpotent}.

We defer the converse direction to
Proposition~\ref{prop:boolean-dim0-continued} since we don't want to interrupt
the exposition here with a certain necessary technical condition.
\end{proof}


\subsection{Integrality} In intuitionistic logic, the notion of an integral
domain bifurcates into several inequivalent notions. The following two are
important for our purposes:
\begin{defn}A ring~$R$ is an \emph{integral domain in the weak sense} if and
only if~$1 \neq 0$ in~$R$ and
\[ \forall x,y\?R\_ xy = 0 \Longrightarrow (x = 0) \vee (y = 0). \]
A ring~$R$ is an \emph{integral domain in the strong sense} if and only if~$1
\neq 0$ in~$R$ and
\[ \forall x\?R\_ x = 0 \vee \speak{$x$ is regular}, \]
where~$\speak{$x$ is regular}$ means that~$xy = 0$ implies~$y = 0$ for any~$y \?
R$.\end{defn}

For the following result, recall that a scheme~$X$ (or ringed space) is
\emph{integral at a point~$x \in X$} if and only if~$\O_{X,x}$ is an integral
domain (in either sense, since we have adopted a classical metatheory).

\begin{prop}\label{prop:internal-integrality}
Let~$X$ be a ringed space. Then:
\begin{enumerate}
\item $X$ is integral at all points if and only if, internally,~$\O_X$ is an
integral domain in the weak sense.
\item If~$X$ is even a locally Noetherian scheme, then~$\O_X$ is an integral
domain in the weak sense iff it is an integral domain in the strong sense from
the internal point of view.
\end{enumerate}
\end{prop}
\begin{proof}The condition on a ring to be an integral domain in the weak sense
is a conjunction of two geometric implications,~``$1 = 0 \Rightarrow \bot$''
and the implication displayed in the definition. Therefore the first statement
is obvious.

For the second statement, note that the condition on a function~$f \in
\Gamma(U,\O_X)$ to be regular from the internal perspective is open: It holds
at a point~$x \in U$ if and only if it holds on some open neighbourhood of~$x$.
We will give a proof of this specific feature of locally Noetherian schemes
later on, when we have developed appropriate machinery to do so easily
(Proposition~\ref{prop:regularity-spreading}). In any case, this openness
property was the essential ingredient for the equivalence between ``holding
internally'' and ``holding at every point''
(Corollary~\ref{cor:geometric-implication}). Therefore~$\O_X$ is an integral
domain in the strong sense from the internal point of view if and only if all
local rings~$\O_{X,x}$ are integral domains. By the first statement, this is
equivalent to~$\O_X$ being an integral domain in the weak sense from the
internal point of view.
\end{proof}

We record the following lemma for later use. The proof presented here is
already simple, but a more conceptual proof is also possible (see
Section~\ref{sect:common-lemmas-transfer-principles}).
\begin{lemma}\label{lemma:regular-affine}
Let~$X = \Spec A$ be an affine scheme. Let~$f \in A$. Then~$f$ is
a regular element of~$A$ if and only if~$f$ is a regular element of~$\O_X$ from
the internal perspective.\end{lemma}
\begin{proof}The Kripke--Joyal translation of internal regularity is:
\begin{quote}For any (without loss of generality: standard) open subset~$U \subseteq X$ and any function~$g \in
\Gamma(U,\O_X)$, $fg = 0$ in~$\Gamma(U,\O_X)$ implies~$g = 0$
in~$\Gamma(U,\O_X)$.\end{quote}
So the ``only if'' direction is clear (use~$U \defeq X$). For the ``if'' direction,
note that~$\Gamma(U,\O_X)$ is a localization of~$A$ and that regular elements
remain regular in localizations.
\end{proof}


\subsection{Bézout property} Recall that a \emph{Bézout ring} is a ring in
which any finitely generated ideal is a principal ideal. In intuitionistic
mathematics, this is a better notion than that of a principal ideal ring: The
requirement that \emph{any} ideal is a principal ideal is far too strong.
Intuitively, this is because without any given generators to begin with, one
cannot hope to explicitly pinpoint a principal generator.
One can (provably) not even verify this property for the ring~$\ZZ$.\footnote{Assume
that any ideal of~$\ZZ$ is finitely generated. Let~$\varphi$ be an arbitrary
proposition; we want to intuitionistically deduce~$\varphi \vee \neg\varphi$.
Consider the ideal~$\aaa \defeq \{ x \in \ZZ \,|\, (x = 0) \vee \varphi \}
\subseteq \ZZ$. The definition is such that~$\varphi$ holds if and only
if~$\aaa$ contains an element other than zero; and that~$\neg\varphi$ holds if
and only if zero is the only element of~$\aaa$.
By assumption,~$\aaa$ is finitely generated. Since~$\ZZ$ is a
Bézout ring, it is therefore even principal:~$\aaa = (x_0)$ for some~$x_0 \in
\ZZ$. Even intuitionistically we have~$(x_0 = 0) \vee (x_0 \neq 0)$ (for the
natural numbers, this can be proven by induction). In the first case, it
follows that~$\aaa$ contains only zero; in the second case, it follows
that~$\aaa$ contains an element other than zero. Thus~$\neg\varphi \vee
\varphi$.

This kind of reasoning is called \emph{exhibiting a Brouwerian
counterexample}. The definition of~$\aaa$ may look slightly dubious,
considering that~$\varphi$ does not depend on~$x$; but we will see that such
definitions actually have a clear geometric meaning -- they can be used to
define extensions of sheaves by zero in the internal language
(Lemma~\ref{lemma:extension-by-zero}).}

\begin{prop}Let~$X$ be a scheme (or ringed space).
\begin{enumerate}
\item $\O_X$ is a Bézout ring from the internal perspective if and only if all
rings~$\O_{X,x}$ are Bézout rings.
\item $\O_X$ is such that, from the internal perspective, of any two elements,
one divides the other, if and only if all rings~$\O_{X,x}$ are such.
\end{enumerate}
\end{prop}
\begin{proof}Both properties can be formulated as geometric implications:
\begin{multline*}
  \text{(1)}\quad
  \forall f,g\?\O_X\_
  \top \Rightarrow
  \exists d\?\O_X\_
  (\exists a,b\?\O_X\_ d = af + bg) \wedge {} \\
  (\exists u\?\O_X\_ f = ud) \wedge
  (\exists v\?\O_X\_ g = vd)
\end{multline*}
\[
  \text{(2)}\quad
  \forall f,g\?\O_X\_
  \top \Rightarrow
  (\exists u\?\O_X\_ f = ug) \,\vee\,
  (\exists u\?\O_X\_ g = uf) \qquad\qquad\qquad\quad \qedhere
\]
\end{proof}

\begin{cor}\label{cor:dedekind-smith}
Let~$X$ be a Dedekind scheme, \ie a locally Noetherian normal scheme
of dimension~$\leq 1$. Then, from the internal perspective, any matrix
over~$\O_X$ can be put into Smith canonical form, \ie is equivalent to a
(rectangular) diagonal matrix with diagonal entries~$a_1|a_2|\cdots|a_n$
successively dividing each other.
\end{cor}
\begin{proof}It is well-known that such a scheme has principal ideal domains as
local rings~$\O_{X,x}$. For local domains, the Bézout condition is equivalent to the
property that of any two elements, one divides the other. Therefore all local
rings have this property, and by the previous proposition, the internal
ring~$\O_X$ has it as well. The statement thus follows from interpreting the
following fact of linear algebra in the internal universe: Let~$R$ be a ring
such that of any two elements, one divides the other. Then any matrix over~$R$
can be put into Smith canonical form.

The usual proof of this fact is indeed intuitionistically acceptable: Let a
matrix over~$R$ be given. By induction, one can show that for any finite family
of ring elements, one divides all the others. So some matrix entry is a factor
of all the others. We can put this entry to the upper left by row and column
transformations and then kill the other entries of the first row and the first
column. After these operations, it is still the case that the entry in the
first row and column is a factor of all other entries. Continuing in this
fashion, we obtain a diagonal matrix. Its diagonal entries already fulfill
the divisibility condition and thus do not have to be sorted.
\end{proof}

Note that phrases such as ``if by chance the entry in the upper left divides
all the others, we can directly proceed with the next step; otherwise, some
other entry must be a factor of all entries, so \ldots'' may not be included in
a proof which is intended to be intuitionistically acceptable.
Those phrases assume that one may make the case distinction that for
any two ring elements~$x,y$, either~$x$ divides~$y$ or not. Fortunately, those
case distinctions are in fact superfluous.

A consequence of the corollary is that internally to the sheaf topos of a
Dedekind scheme, the usual structure theorem on finitely
presented~$\O_X$-modules is available. We will exploit this in
Lemma~\ref{lemma:torsion-stuff}, where we give an internal proof of the
fact that on Dedekind schemes, torsion-free~$\O_X$-modules are locally free.


\subsection{Normality} We will discuss the property of a ring to be
\emph{normal}, \ie to be integrally closed in its total field of
fractions, in Section~\ref{sect:normality}, after giving an internal
characterization of the sheaf of rational functions.


\subsection{Special properties of constant sheaves of rings} Let~$R$ be an
ordinary ring and~$\ul{R}$ the associated sheaf of locally constant~$R$-valued
functions on a topological space. If~$R$ is reduced, local, or a field,
then~$\ul{R}$ is so as well, from the internal point of view.

We will prove this in greater generality: Appropriately formulated, a constant
sheaf~$\ul{R}$ has some property~$\varphi$ from the internal point of view if
and only if~$R$ has the property~$\varphi$ externally
(Lemma~\ref{lemma:properties-of-constant-sheaves}).


\section{Sheaves of modules}

From the internal perspective, a sheaf of~$\R$-modules, where~$\R$ is a sheaf
of rings, looks just like a plain module over the plain ring~$\R$. This is
proven just as the correspondence between sheaf of rings and internal rings
(Proposition~\ref{prop:rings-internally}).

% XXX: talk about modules over constant sheaves of rings as well


\subsection{Finite local freeness}

Recall that an~$\O_X$-module~$\F$ is \emph{finite locally free} if and only
if there exists a covering of~$X$ by open subsets~$U$ such that on each
such~$U$, the restricted module~$\F|_U$ is isomorphic as an~$\O_X|_U$-module
to~$(\O_X|_U)^n$ for some natural number~$n$ (which may depend on~$U$).

\begin{prop}\label{prop:locally-free}
Let~$X$ be a scheme (or ringed space). Let~$\F$ be
an~$\O_X$-module. Then~$\F$ is finite locally free if and only if, from the
internal perspective,~$\F$ is a finite free module, \ie
\[ \Sh(X) \models \bigvee_{n \geq 0} \speak{$\F \cong (\O_X)^n$}, \]
or more elementarily
\[ \Sh(X) \models \bigvee_{n \geq 0}
  \exists x_1,\ldots,x_n\?\F\_
  \forall x\?\F\_
  \exists! a_1,\ldots,a_n\?\O_X\_
  x = \textstyle\sum\limits_i a_i x_i. \]
\end{prop}
\begin{proof}By the expression~``$(\O_X)^n$'' in the internal language we mean
the internally constructed object~$\O_X \times \cdots \times \O_X$ with its
componentwise~$\O_X$-module structure. This coincides with the sheaf~$(\O_X)^n$ as
usually understood.

It is clear that the two stated internal conditions are equivalent, since the
corresponding proof in linear algebra is intuitionistically acceptable. The equivalence with
the external notion of finite local freeness follows because the
interpretation of the first condition with the Kripke--Joyal semantics is the
following: There exists a covering of~$X$ by open subsets~$U$ such that for
each such~$U$, there exists a natural number~$n$ and a morphism of
sheaves~$\varphi : \F|_U \to (\O_X|_U)^n$ such that
\[ U \models \speak{$\varphi$ is~$\O_X$-linear} \quad\text{and}\quad
  U \models \speak{$\varphi$ is bijective}. \]
The first subcondition means that~$\varphi$ is a morphism of sheaves
of~$\O_X|_U$-modules and the second one means that~$\varphi$ is an isomorphism of
sheaves.
\end{proof}


\subsection{Finite type, finite presentation, coherence}
Recall the conditions of an~$\O_X$-module~$\F$ on a scheme~$X$ (or ringed
space) to be of finite type, of finite presentation, and to be coherent:
\begin{itemize}
\item $\F$ is \emph{of finite type} if and only if there exists a covering of~$X$ by
open subsets~$U$ such that for each such~$U$, there exists an exact sequence
\[ (\O_X|_U)^n \longrightarrow \F|_U \longrightarrow 0 \]
of~$\O_X|_U$-modules.
\item $\F$ is \emph{of finite presentation} if and only if there exists a covering of~$X$ by
open subsets~$U$ such that for each such~$U$, there exists an exact sequence
\[ (\O_X|_U)^m \longrightarrow (\O_X|_U)^n \longrightarrow \F|_U \longrightarrow 0. \]
\item $\F$ is \emph{coherent} if and only if~$\F$ is of finite type and the
kernel of any~$\O_X|_U$-linear morphism~$(\O_X|_U)^n \to \F|_U$, where~$U \subseteq
X$ is any open subset, is of finite type.
\end{itemize}

The following proposition gives translations of these definitions into the
internal language.
\begin{prop}\label{prop:finite-type-and-co}
Let~$X$ be a scheme (or ringed space). Let~$\F$ be
an~$\O_X$-module. Then:
\begin{itemize}
\item $\F$ is of finite type if and only if~$\F$, considered as an ordinary
module from the internal perspective, is finitely generated, \ie if
\[ {\qquad\qquad} \Sh(X) \models
  \bigvee_{n \geq 0}
  \exists x_1,\ldots,x_n\?\F\_
  \forall x\?\F\_
  \exists a_1,\ldots,a_n\?\F\_
  x = \textstyle\sum\limits_i a_i x_i. \]
\item $\F$ is of finite presentation if and only if~$\F$ is a finitely
presented module from the internal perspective, \ie if
\[ {\qquad\qquad} \Sh(X) \models \bigvee_{n,m \geq 0}
  \speak{there is a short exact sequence $\O_X^m \to \O_X^n \to \F \to 0$}.
  \]
\item $\F$ is coherent if and only if~$\F$ is a coherent module from the
internal perspective, \ie if
\begin{multline*}
{\qquad\qquad\qquad}
  \Sh(X) \models \speak{$\F$ is finitely generated} \mathop{\wedge} \\
  \bigwedge_{n \geq 0} \forall \varphi \? \HOM_{\O_X}(\O_X^n,\F)\_
  \speak{$\Kernel \varphi$ is finitely generated}.
\end{multline*}
\end{itemize}
\end{prop}
\begin{proof}Straightforward -- the translations of the internal statements using
the Kripke--Joyal semantics are precisely the corresponding external
statements.
\end{proof}

Recall that an~$\O_X$-module~$\F$ is \emph{generated by global sections} if and
only if there exist global sections~$s_i \in \Gamma(X,\F)$ such that for any~$x
\in X$, the stalk~$\F_x$ is generated by the germs of the~$s_i$.
This condition is of course not local on the base. Therefore there cannot
exist a formula~$\varphi$ such that for any space~$X$ and
any~$\O_X$-module~$\F$ it holds that~$\F$ is generated by global sections if
and only if~$\Sh(X) \models \varphi(\F)$. But still, global generation can be
characterized by a mixed internal/external statement:

\begin{prop}Let~$X$ be a scheme (or ringed space). Let~$\F$ be
an~$\O_X$-module. Then~$\F$ is generated by global sections if and only if
there exist global sections~$s_i \in \Gamma(X,\F)$, $i \in I$ such that
\[ \Sh(X) \models \forall x\?\F\_ \bigvee\nolimits_{\textnormal{$J=\{i_1,\ldots,i_n\} \subseteq I$ finite}}
  \exists a_1,\ldots,a_n\?\O_X\_
  x = \sum_j a_j x_{i_j}. \]
Furthermore,~$\F$ is generated by finitely many global sections if and only if
there exist global sections~$s_1,\ldots,s_n \in \Gamma(X,\F)$ such that
\[ \Sh(X) \models \forall x\?\F\_ \exists a_1,\ldots,a_n\?\O_X\_ x = \sum_j a_j
x_j. \]
\end{prop}
\begin{proof}The given internal statements are geometric implications, their
validity can thus be checked stalkwise.\end{proof}


\subsection{Tensor product and flatness} Recall that the tensor product
of~$\O_X$-modules~$\F$ and~$\G$ on a scheme~$X$ (or ringed space) is usually
constructed as the sheafification of the presheaf
\[ \text{$U \subseteq X$ open} \quad\longmapsto\quad \Gamma(U,\F) \otimes_{\Gamma(U,\O_X)}
\Gamma(U,\G). \]
From the internal point of view,~$\F$ and~$\G$ look like ordinary modules, so
that we can consider their tensor product as usually constructed in
commutative algebra, as a certain quotient of the free module on the elements
of~$\F \times \G$:
\[ \O_X\langle x \otimes y \,|\, x\?\F, y\?\G \rangle / R, \]
where~$R$ is the submodule generated by
\begin{gather*}
  (x+x') \otimes y - x \otimes y - x' \otimes y, \\
  x \otimes (y+y') - x \otimes y - x \otimes y', \\
  (sx) \otimes y - s(x \otimes y), \\
  x \otimes (sy) - s(x \otimes y)
\end{gather*}
with~$x,x'\?\F$, $y,y'\?\G$, $s\?\O_X$.
This internal construction gives rise to the same sheaf
of modules as the externally defined tensor product:

\begin{prop}\label{prop:internal-tensor-product}
Let~$X$ be scheme (or a ringed space). Let~$\F$ and~$\G$
be~$\O_X$-modules. Then the internally constructed tensor product~$\F
\otimes_{\O_X} \G$ coincides with the external one.
\end{prop}
\begin{proof}
Since the proof of the corresponding fact of commutative algebra is
intuitionistically acceptable, the internally defined tensor product~$\F \otimes_{\O_X} \G$
has the following universal property: For any~$\O_X$-module~$H$,
any~$\O_X$-bilinear map~$\F \times \G \to H$ uniquely factors over the
canonical map~$\F \times \G \to \F \otimes_{\O_X} \G$.

Interpreting this property with the Kripke--Joyal semantics, we see that the
internally constructed tensor product has the following external property:
For any open subset~$U \subseteq X$ and any~$\O_X|_U$-module~$\H$ on~$U$,
any~$\O_X|_U$-bilinear morphism~$\F|_U \times \G|_U \to \H$ uniquely factors over the
canonical morphism~$\F|_U \times \G|_U \to (\F \otimes_{\O_X} \G)|_U$.

In particular, for~$U = X$, this property is well-known to be the universal
property of the externally constructed tensor product. Therefore the
claim follows.
\end{proof}

A description of the stalks of the tensor product
follows purely by considering the logical form of the construction:
\begin{cor}Let~$X$ be scheme (or a ringed space). Let~$\F$ and~$\G$
be~$\O_X$-modules. Then the stalks of the tensor product coincide with the
tensor products of the stalks: $(\F \otimes_{\O_X} \G)_x \cong \F_x
\otimes_{\O_{X,x}} \G_x$.\end{cor}
\begin{proof}
We constructed the tensor product using the following operations: product of
two sets, free module on a set, quotient module with respect to a submodule;
submodule generated by a set of elements given by a geometric formula.
All of these operations are geometric, so the tensor product construction is
geometric as well (see Section~\ref{sect:geometric-formulas-and-constructions}). Hence taking stalks commutes with performing the
construction.
\end{proof}

Recall that an~$\O_X$-module~$\F$ is \emph{flat} if and only if all
stalks~$\F_x$ are flat~$\O_{X,x}$-modules. We can characterize flatness in the
internal language.
\begin{prop}\label{prop:flatness}
Let~$X$ be a scheme (or ringed space). Let~$\F$ be
an~$\O_X$-module. Then~$\F$ is flat if and only if, from the internal
perspective,~$\F$ is a flat~$\O_X$-module.
\end{prop}
\begin{proof}
Recall that flatness of an~$A$-module~$M$ can be characterized without
reference to tensor products by the following condition (using
suggestive vector notation): For any natural number~$p$,
any $p$-tuple~$m \? M^p$ of elements of~$M$ and
any~$p$-tuple $a \? A^p$ of elements of~$A$, it should hold that
\[
  a^T m = 0 \ \Longrightarrow\
  \bigvee\limits_{q \geq 0} \exists n\?M^q, B\?A^{p \times q}\_
  Bn = m \wedge a^T B = 0. \]
The equivalence of this condition with tensoring being exact holds
intuitionistically as
well~\cite[Theorem~III.5.3]{mines-richman-ruitenburg:constructive-algebra}.
This formulation of flatness has the advantage that it is the conjunction of
geometric implications (one for each~$p \geq 0$); therefore it holds internally
if and only if it holds at any point.
\end{proof}


\subsection{Support} Recall that the \emph{support} of an~$\O_X$-module~$\F$ is
the subset~$\supp\F \defeq \{ x \in X \,|\, \F_x \neq 0 \} \subseteq X$. If~$\F$ is
of finite type, this set is closed, since its complement is then open by a
standard lemma. (We will give an internal proof of this fact in
Lemma~\ref{lemma:module-zero-point-neighbourhood}.)

\begin{prop}\label{prop:characterization-support}
Let~$X$ be a scheme (or ringed space). Let~$\F$ be
an~$\O_X$-module. Then the interior of the complement of the support of~$\F$
can be characterized as the largest open subset of~$X$ on which the internal
statement~$\F = 0$ holds.
\end{prop}
\begin{proof}
For any open subset~$U \subseteq X$, it holds that:
\begin{align*}
  &\ U \subseteq \Int(X \setminus \supp \F) \\
  \Longleftrightarrow&\ U \subseteq X \setminus \supp \F \\
  \Longleftrightarrow&\ U \subseteq \{ x \in X \,|\, \forall s \in \F_x\_ s = 0 \} \\
  \Longleftrightarrow&\ U \models \forall s\?\F\_ s = 0 \\
  \Longleftrightarrow&\ U \models \speak{$\F = 0$}
\end{align*}
The second to last equivalence is because~``$\forall s\?\F\_ s = 0$'' is a
geometric implication and can thus be checked stalkwise.
\end{proof}

\begin{rem}\label{rem:support-sheaf-of-sets}
The support of a sheaf of \emph{sets}~$\F$ is defined as the subset~$\{ x \in X \,|\,
\text{$\F_x$ is not a singleton} \}$. A similar proof shows that the interior
of its complement can be characterized as the largest open subset of~$X$ where
the internal statement~$\speak{$\F$ is a singleton}$ holds.\end{rem}


\subsection{Torsion} Let~$R$ be a ring. Recall that the
\emph{torsion submodule}~$M_\tors$ of an~$R$-module~$M$ is defined as
\[ M_\tors \defeq \{ x\?M \,|\, \exists a\?R\_ \speak{$a$ regular} \wedge ax = 0 \}
\subseteq M. \]
This definition is meaningful even if~$R$ is not an integral domain.
An~$R$-module~$M$ is \emph{torsion-free} if and only if~$M_\tors$ is
the zero submodule; an~$R$-module~$M$ is a \emph{torsion module} if and only
if~$M_\tors = M$.

Recall also that if~$\F$ is a sheaf of~$\O_X$-modules on an integral
scheme~$X$, there is a unique subsheaf~$\F_\tors \subseteq \F$ with the
property that~$\Gamma(U,\F_\tors) = \Gamma(U,\F)_\tors$ for all affine open
subsets~$U \subseteq X$. The content of the following proposition is that
internally constructing the torsion submodule of~$\F$, regarded as a plain
module from the internal perspective, gives exactly this subsheaf. There is
therefore no harm in using the same notation~``$\F_\tors$'' for the result of
the internal construction.

\begin{prop}\label{prop:torsion-int-ext}Let~$X$ be an integral scheme. Let~$\F$ be an~$\O_X$-module. Let~$U
= \Spec A \subseteq X$ be an affine open subset. Let~$s \in \Gamma(U,\F)$ be a local
section. Then
\[ s \in \Gamma(U,\F)_\tors \quad\text{if and only if}\quad
  U \models s \in \F_\tors. \]
\end{prop}
\begin{proof}
The ``only if'' direction is trivial in light of
Lemma~\ref{lemma:regular-affine}: If~$s$ is a torsion element
of~$\Gamma(U,\F)$, there exists a regular element~$a \in \Gamma(U,\O_X)$ such
that~$as = 0$. By the lemma, this element is regular from the internal
perspective as well, so~$U \models \speak{$a$ regular} \wedge as = 0$.

For the ``if'' direction, we may assume that there exists an open covering~$X =
\bigcup_i U_i$ be standard open subsets~$U_i = D(f_i)$ such that there are
sections~$a_i \in \Gamma(U_i,\O_X) = A[f_i^{-1}]$ with~$U_i \models \speak{$a_i$ regular}
\wedge a_i s = 0$. Without loss of generality, we may assume that the
denominators of the~$a_i$'s are ones, that the $f_i$ are
finite in number, and that the~$f_i$ are regular (\ie nonzero, since~$A$ is an
integral domain). By Lemma~\ref{lemma:regular-affine}, the~$a_i$ are
regular in~$A[f_i^{-1}]$ and by regularity of the~$f_i$ also regular in~$A$.
Therefore their product~$\prod_i a_i \in A$ is regular in~$A$ as well and
annihilates~$s$.
\end{proof}
% FUTURE:
% For quasicoherent sheaves, there is an alternative (and more conceptual)
% proof. Since X is integral, we have
%
%   Spec(A) |== (forall s:\ul{A}. s regular in A <==> s regular in O_{Spec A}).
%
% Therefore the internal statement
%
%   exists a:O_{Spec A}. a regular and as = 0 in \ul{M}[F^(-1)]
%
% is equivalent to
%
%   bigvee_{a in A, a regular} (exists u in F. uas = 0).
%
% This disjunction is directed.

\begin{prop}\label{prop:torsion-submodule-stalks}
Let~$X$ be a locally Noetherian scheme. Let~$\F$ be
an~$\O_X$-module. Let~$x \in X$ be a point. Then~$(\F_\tors)_x =
(\F_x)_\tors$.\end{prop}
\begin{proof}This would be obvious if the condition on an element~$s\?\F$ to
belong to~$\F_\tors$ were a geometric formula. Because of the universal
quantifier, it is not:
\[ s \in \F_\tors \quad\Longleftrightarrow\quad
  \exists a\?\O_X\_ (\forall b\?\O_X\_ ab = 0 \Rightarrow b = 0) \wedge as = 0. \]
But since~$X$ is assumed to be locally Noetherian, regularity is an open
property nonetheless (see Proposition~\ref{prop:regularity-spreading} for an
internal proof of this fact). Thus the claim still follows, just like in the
proof of Proposition~\ref{prop:internal-integrality}.
\end{proof}


\subsection{Internal proofs of common lemmas}

\begin{lemma}Let~$X$ be a scheme (or ringed space). Let
\[ 0 \lra \F \lra \G \lra \H \lra 0 \]
be a short exact sequence of~$\O_X$-modules. If~$\F$ and~$\H$ are of finite
type, so is~$\G$; similarly, if~$\F$ and~$\H$ are finite locally free, so
is~$\G$.
\end{lemma}
\begin{proof}From the internal perspective, we are given a short exact sequence
of modules with the outer ones being finitely generated (\resp finite free)
and we have to show that the middle one is finitely generated (\resp finite
free) as well. It is well-known that this follows; and since the usual proof of
this fact is intuitionistically acceptable, we are done.
\end{proof}

Note that the proof works very generally, in the context of arbitrary ringed
spaces, and is still very simple. This is common to proofs using the internal
language. Particular features of schemes enter only at clearly recognizable
points, for instance when an internal property specific to the structure sheaf
of schemes is used (such as in Proposition~\ref{prop:neginvnilpotent}).

\begin{lemma}\label{lemma:coherent-stuff}
Let~$X$ be a scheme (or ringed space).
\begin{itemize}
\item Let~$0 \to \F \to \G \to \H \to 0$ be an exact sequence
of~$\O_X$-modules. If two of the three modules are coherent, so is the third.
\item Let~$\F \to \G$ be a morphism of~$\O_X$-modules such that~$\F$ is
of finite type and~$\G$ is coherent. Then its kernel is of finite type as well.
\item If~$\F$ is a finitely presented~$\O_X$-module and~$\G$ is a
coherent~$\O_X$-module, the~$\O_X$-modules~$\HOM(\F,\G)$ and~$\F \otimes \G$
are coherent as well.
\end{itemize}
\end{lemma}
\begin{proof}These statements follow directly from interpreting the
corresponding standard proofs of commutative algebra in the internal language.
For those standard proofs, see for instance the lecture notes of Ravi
Vakil~\cite[Section~13.8]{vakil:foag}, where they are given as a series of
exercises.
\end{proof}

\begin{lemma}\label{lemma:kernel-of-epi-fingen}
Let~$X$ be a scheme (or locally ringed space). Let~$\alpha : \G
\to \H$ be an epimorphism of finite locally free~$\O_X$-modules. Then the
kernel of~$\alpha$ is finite locally free as well.\end{lemma}
\begin{proof}It suffices to give an intuitionistic proof of the following
statement: The kernel of a matrix over a local ring, which as a linear map is
surjective, is finite free.

Let~$M \? R^{n \times m}$ be such a matrix. Since by the surjectivity
assumption some linear combination of the columns is~$e_1$ (the first canonical
basis vector), some linear combination of the entries of the first row of~$M$
is~$1$. By locality of~$R$, at least one entry of the first row is invertible.
By applying appropriate column and row transformations, we may therefore assume that~$M$
is of the form
\[ \left(
  \begin{array}{c|ccc}
    1 & 0 & \cdots & 0 \\ \hline
    0 & \multicolumn{3}{c}{\multirow{3}{*}{\raisebox{-5mm}{\scalebox{1.2}{$\widetilde M$}}}} \\
    \raisebox{2pt}{\vdots} & & & \\
    0 & & &
  \end{array}
\right) \]
with the submatrix~$\widetilde M$ fulfilling the same condition as~$M$.
Continuing in this way, it follows that~$m \geq n$ and that we may assume
that~$M$ is of the form
\[ \left(
  \begin{array}{ccc|c}
    1 & & & \multirow{3}{*}{\raisebox{-3mm}{\scalebox{1.2}{\ 0}}} \\
    & \ddots && \\
    && 1
  \end{array}
\right)\!. \]
The kernel of such a matrix is obviously freely generated by the canonical
basis vectors corresponding to the zero columns. In particular, the rank of the
kernel is~$m-n$.
\end{proof}

\begin{rem}The internal language machinery gives no reason to believe that the
dual statement is true, \ie that the cokernel of a monomorphism of finite
locally free~$\O_X$-modules is finite locally free. This would follow from
an intuitionistic proof of the statement that the cokernel of an injective map
between finite free modules over a local ring is finite free. But this
statement is of course false (consider the exact sequence
$0 \lra \ZZ_{(2)} \stackrel{\cdot 2}{\lra} \ZZ_{(2)} \lra \FF_2 \lra 0$
of~$\ZZ_{(2)}$-modules).
\end{rem}

\begin{lemma}Let~$X$ be a scheme (or locally ringed space). Let~$\alpha : \G
\to \H$ be an epimorphism of finite locally free~$\O_X$-modules of the same
rank. Then~$\alpha$ is an isomorphism.\end{lemma}
\begin{proof}It suffices to give an intuitionistic proof of the following
statement: A square matrix over a local ring, which as a linear map is
surjective, is invertible.

This follows from the proof of the previous lemma, since it shows that the
kernel of such a matrix is finite free of rank zero.
\end{proof}

\begin{lemma}Let~$X$ be a scheme (or ringed space). Let
$0 \to \F \to \G \to \H \to 0$ be a short exact sequence of~$\O_X$-modules.
Then for the closures of the supports there holds the equation
$\Clos \supp \G = \Clos \supp \F \cup \Clos \supp \H$.
\end{lemma}
\begin{proof}Switching to complements, we have to prove that
\[ \Int(X \setminus \supp\G) = \Int(X \setminus \supp\F) \cap \Int(X \setminus
\supp\H). \]
By Proposition~\ref{prop:characterization-support}, it suffices to prove
\[ \Sh(X) \models (\G = 0\ \Longleftrightarrow\ \F = 0 \wedge \H = 0); \]
this is a basic observation in linear algebra, valid intuitionistically.
\end{proof}
Of course, a stronger version of this lemma -- about the supports themselves
instead of their closures -- is easily proven without using the internal
language. We included this example only for illustrative purposes.

\begin{lemma}Let~$X$ be a scheme (or locally ringed space). Let~$\L$ be a line
bundle on~$X$, \ie an~$\O_X$-module locally free of rank~1.
Let~$s_1,\ldots,s_n \in \Gamma(X,\L)$ be global sections. Then these sections
globally generate~$\F$ if and only if
\[ \Sh(X) \models \bigvee_i \speak{$\alpha(s_i)$ is invertible for some
isomorphism~$\alpha : \L \to \O_X$}. \]
\end{lemma}
\begin{proof}It suffices to give an intuitionistic proof of the following fact:
Let~$R$ be a local ring. Let~$L$ be a free~$R$-module of rank~1.
Let~$s_1,\ldots,s_n\?L$ be given elements. Then~$L$ is generated as
an~$R$-module by these elements if and only if for some~$i$, the image of~$s_i$
under some isomorphism~$L \to R$ is invertible.

Note that the choice of such an isomorphism does not matter, since any two such
isomorphisms~$\alpha, \beta : L \to R$ differ by a unit of~$R$: $\alpha(x) =
\alpha(\beta^{-1}(1)) \cdot \beta(x)$ for any~$x\?L$,
and~$\alpha(\beta^{-1}(1)) \cdot \beta(\alpha^{-1}(1)) = 1$ in~$R$.

For the ``if'' direction, we have that some~$\alpha(s_i)$ is a generator
of~$R$. Since~$\alpha$ is an isomorphism, it follows that~$s_i$ generates~$L$,
and thus that in particular, the family~$s_1,\ldots,s_n$ generates~$L$.

For the ``only if'' direction, we have that the unit of~$R$ can be expressed as
a linear combination of the~$\alpha(s_i)$, where~$\alpha : L \to R$ is some
isomorphism (whose existence is assured by the assumption on the rank of~$L$).
Since~$R$ is a local ring, it follows that one of the summands and thus one of
the~$\alpha(s_i)$ is invertible.
\end{proof}

\begin{rem}Note that the canonical ring homomorphism~$\O_{X,x}
\twoheadrightarrow k(x)$ is local. Therefore a germ in~$\O_{X,x}$ is invertible
if and only if its image in~$k(x)$ is not zero. From this one can follow that
global sections~$s_1,\ldots,s_n \in \Gamma(X,\F)$ generate~$\F$ if and only if,
for any point~$x \in X$, the images~$s_i \in \F|_x$ in the fibers do not vanish
simultaneously.
\end{rem}

\begin{lemma}Let~$X$ be a scheme (or ringed space). Let~$\L$ be a locally
free~$\O_X$-module of rank~1. Then~$\L^\vee \otimes_{\O_X} \L \cong \O_X$.\end{lemma}
\begin{proof}Recall that the dual is defined by~$\L^\vee \defeq
\HOM_{\O_X}(\L,\O_X)$. Since~``$\HOM$'' looks like~``$\Hom$'' from the internal
point of view, the dual sheaf~$\L^\vee$ looks just like the ordinary dual
module. However, to prove the claim, it does \emph{not} suffice to give an
intuitionistic proof of the following fact of linear algebra: ``Let~$L$ be a
free~$R$-module of rank~1. Then there exists an isomorphism~$L^\vee \otimes_R L
\to R$.'' Since the interpretation of ``$\exists$'' using the Kripke--Joyal
semantics is local existence, this would only show that~$\L^\vee \otimes_{\O_X}
\L$ is \emph{locally} isomorphic to~$\O_X$.

Instead, we have to actually \emph{write down} (\ie explicitly give) a
linear map in the internal language -- not using the assumption that~$L$ is
free of rank~1, as this would introduce an existential quantifier again (see
Section~\ref{sect:internal-constructions}).
So we have to prove the following fact: Let~$L$ be an~$R$-module. Then there
explicitly exists a linear map~$L^\vee \otimes_R L \to R$ such that this map is
an isomorphism if~$L$ is free of rank~1.

This is done as usual: Define~$\alpha : L^\vee \otimes_R L \to R$ by~$\lambda
\otimes x \mapsto \lambda(x)$. Since~$L$ is free of rank~1, there is an
isomorphism~$L \cong R$. Precomposing~$\alpha$
with the induced isomorphism~$R^\vee \otimes_R R \to L^\vee \otimes_R L$,
we obtain the linear map~$R^\vee \otimes_R R \to R$ given by the same
term: $\lambda \otimes x \mapsto \lambda(x)$. One can check that an inverse is given
by~$x \mapsto \id_R \otimes x$.
\end{proof}

\begin{lemma}\label{lemma:torsion-stuff}
Let~$X$ be a scheme (or ringed space). Let~$\F$ be
an~$\O_X$-module.
\begin{enumerate}
\item Assume~$X$ to be a locally Noetherian scheme. Then $\F$ is torsion-free
(meaning~$\F_\tors = 0$) if and only if all stalks~$\F_x$ are torsion-free.
\item The quotient sheaf~$\F/\F_\tors$ is torsion-free and the torsion
submodule~$\F_\tors$ is a torsion module.
\item The dual sheaf $\F^\vee$ is torsion-free.
\item If~$\F$ is reflexive (meaning that the canonical morphism~$\F \to
\F^{\vee\vee}$ is an isomorphism), it is torsion-free.
\item If~$\F$ is locally finitely free, it is reflexive.
\item Assume~$X$ to be a Dedekind scheme and~$\F$ to be of finite presentation.
If~$\F$ is torsion-free, then it is locally finitely free.
% change references below if numbering changes
\end{enumerate}
\end{lemma}
\begin{proof}The first statement follows from the observation that~$(\F_\tors)_x
= (\F_x)_\tors$ (Proposition~\ref{prop:torsion-submodule-stalks}). All the
others follow simply by interpreting the corresponding facts of linear algebra
in the internal universe. For concreteness, we give intuitionistic proofs of
the last three statements.

So let~$M$ be an reflexive~$R$-module. We have to show that~$M$ is
torsion-free. To this end, let an element~$x \? M$ and a regular element~$a \?
R$ such that~$ax = 0$ be given. For any~$\vartheta \? M^\vee$, it follows
that~$\vartheta(x) = 0$, since~$a \vartheta(x) = \vartheta(ax) = \vartheta(0) =
0$ and~$a$ is regular. Thus the image of~$x$ under the canonical map~$M \to
M^{\vee\vee}$ is zero. By reflexivity, this implies that~$x$ is zero.

For statement~(5), we have to prove that~$R$-modules of the form~$R^n$ are
reflexive. This is obvious, the required inverse map is~$(R^n)^{\vee\vee} \to
R^n,\ \lambda \mapsto \sum_i \lambda(\vartheta_i)$ where~$\vartheta_i : R^n \to
R,\ (x_j)_j \mapsto x_i$.

In light of Corollary~\ref{cor:dedekind-smith} we can put matrices over~$\O_X$
into Smith canonical form if~$X$ is a Dedekind scheme. Therefore it suffices
to give an intuitionistic proof of the following fact: Let~$R$ be an integral
domain in the strong sense such that matrices over~$R$ can be put into
Smith canonical form. Let~$M$ be a finitely presented torsion-free~$R$-module.
Then~$M$ is finite free.

This goes as follows: Since~$M$ is finitely presented, it is the cokernel of
some matrix. Without loss of generality, we may assume that it is a diagonal
matrix, so~$M$ is isomorphic to some (finite) direct sum~$\bigoplus_i R/(a_i)$.
Since~$M$ is torsion-free, all the summands~$R/(a_i)$ are torsion-free as well.
Since~$R$ is an integral domain in the strong sense, this holds if and only if
the~$a_i$ are either zero or invertible. Thus~$R/(a_i)$ is isomorphic to~$R$ or
to the zero module. In any case,~$R/(a_i)$ is finite free and therefore~$M$
is finite free as well.
\end{proof}


\section{Upper semicontinuous functions}
\label{sect:upper-semicontinuous-functions}

\subsection{Interlude on natural numbers}
In classical logic, the natural numbers are complete in the sense that any
inhabited set of natural numbers possesses a minimal element. This statement
can not be proven intuitionistically -- intuitively, this is because one cannot
explicitly pinpoint the (classically existing) minimal element of an arbitrary
inhabited set;\footnote{Let~$\varphi$ be an arbitrary formula. Assuming that
any inhabited subset of the natural numbers possesses a minimal element, we
want to show that~$\varphi \vee \neg\varphi$. Define the subset $U \defeq \{ n \in
\NN \,|\, (n = 1) \vee \varphi \} \subseteq \NN$, which surely is inhabited by~$1
\in U$. So by assumption, there exists a number~$z \in \NN$ which is the
minimum of~$U$. We have $z = 0$ or $z > 0$. If~$z = 0$, we have~$0 \in U$,
so~$(0 = 1) \vee \varphi$, so~$\varphi$ holds.  If~$z > 0$, then~$\neg\varphi$
holds: If~$\varphi$ were true, zero would be an element of~$U$, contradicting
the minimality of~$z$.} see below for a sheaf-theoretic interpretation.

In intuitionistic logic, the completeness principle can be salvaged in two
essentially different ways: either by strengthening the premise, or by
weakening the conclusion.

\begin{lemma}\label{lemma:minimum-subset-naturals}
Let~$U \subseteq \NN$ be an inhabited subset of the natural
numbers.
\begin{enumerate}
\item Assume~$U$ to be \emph{detachable}, \ie assume that for any natural
number~$n$, either~$n \in U$ or~$n \not\in U$. Then~$U$ possesses a minimal
element.
\item In any case,~$U$ does \emph{not not} possess a minimal element.
\end{enumerate}
\end{lemma}
\begin{proof}
The first statement can be proven by induction on the witness of inhabitation,
\ie the given number~$n$ such that~$n \in U$. We omit further details, since we will
not need this statement in our applications.

For the second statement, we give a careful proof since logical subtleties matter. To simplify the
exposition, we assume that~$U$ is upward-closed, \ie that any number
larger than some element of~$U$ lies in~$U$ as well. Any subset can be closed
in this way (by considering~$\{ n \in \NN \,|\, \exists m \in U\_ n \geq m \}$)
and a minimal element of the closure will be a minimal element for~$U$ as well.

We induct on the number~$n \in U$ given by the assumption that~$U$ is
inhabited. In the case~$n = 0$ we are done since~$0$ is a minimal element
of~$U$. For the induction step~$n \to n+1$, the intuitionistically valid double
negation of the law of excluded middle
gives
\[ \neg\neg(n \in U \vee n \not\in U). \]
Because of the tautologies $(\varphi \Rightarrow \psi) \Rightarrow
(\neg\neg\varphi \Rightarrow \neg\neg\psi)$ and~$\neg\neg\neg\neg\varphi \Rightarrow
\neg\neg\varphi$ (see Section~\ref{sect:appreciating-intuitionistic-logic}), it
suffices to show that~$n \in U \vee n \not\in U$ implies the conclusion.
So assume~$n \in U \vee n \not\in U$.
If~$n \in U$, then~$U$ does \notnot possess a minimal element by the induction
hypothesis. If~$n \not\in U$, then~$n+1$ is a minimal element (and so, in
particular,~$U$ does \notnot possess a minimal element): If~$m$ is
any element of~$U$, we have~$m \geq n+1$ or~$m \leq n$. In the first case,
we're done. In the second case, it follows that~$n \in U$ because~$U$ is
upward-closed and so we obtain a contradiction. From this contradiction we
can trivially deduce~$m \geq n+1$ as well. \qedhere
\end{proof}

If we want to work with a complete partially ordered set (poset) of natural numbers in intuitionistic
logic, we have to construct a suitable completion. The idea of the following
definition is to encode numbers as the (not necessarily existing) minimum of
inhabited upward-closed subsets.
\begin{defn}The \emph{completed poset of natural numbers}
the set~$\widehat{\NN}$ of all inhabited upward-closed subsets of~$\NN$, ordered by
reverse inclusion. The elements of~$\widehat{\NN}$ are called \emph{generalized natural numbers}.\end{defn}
\begin{lemma}The completed poset of natural numbers is the least poset
containing~$\NN$ and possessing minima
of arbitrary inhabited subsets.\end{lemma}
\begin{proof}
The embedding $\NN \hookrightarrow \widehat\NN$ is given by
\[ n \in \NN \quad\longmapsto\quad {\uparrow}(n) \defeq \{ m \in \NN \,|\, m \geq n \}. \]
If~$M \subseteq \widehat\NN$ is an inhabited subset, its minimum is
\[ \min M = \bigcup M \in \widehat\NN. \]
We omit the proof of the universal property.
\end{proof}

\begin{rem}\label{rem:surjectivity-embedding}
In classical logic, the map~$\widehat\NN \to \NN,\ U \mapsto \min U$
is a well-defined isomorphism of partially ordered sets. In fact, it is the
inverse of the canonical embedding~$\NN \hookrightarrow \widehat\NN$. In
intuitionistic logic, this embedding is still injective, but it can not be
shown to be surjective: It is only the case that any element of~$\widehat\NN$
does \notnot possess a preimage (by Lemma~\ref{lemma:minimum-subset-naturals}).
\end{rem}


\subsection{A geometric interpretation}
We are interested in the completed natural numbers for the following reason: A
generalized natural number in the topos of sheaves on a topological space~$X$ is
the same as an upper semicontinuous function~$X \to \NN$.

\begin{lemma}\label{lemma:upper-semicontinuous-functions}
Let~$X$ be a topological space. The sheaf~$\widehat\NN$ of
generalized natural numbers on~$X$ is canonically isomorphic to the sheaf of upper
semicontinuous~$\NN$-valued functions on~$X$.\end{lemma}
\begin{proof}
When referring to the natural numbers in the internal language, we actually
refer to the constant sheaf~$\ul{\NN}$ on~$X$. (This is because the
sheaf~$\ul{\NN}$ fulfills the axioms of a natural numbers object,
cf.\@~\cite[Section~VI.1]{moerdijk-maclane:sheaves-logic}.) Recall that its sections on an
open subset~$U \subseteq X$ are continuous functions~$U \to \NN$, where~$\NN$
is equipped with the discrete topology.

Therefore, a section of~$\widehat\NN$ on an open subset~$U \subseteq X$ is
given by a subsheaf~$\A \hookrightarrow \ul{\NN}|_U$ such that
\[ U \models \exists n\?\ul{\NN}\_ n \in \A
  \quad\text{and}\quad
  U \models \forall n,m\?\ul{\NN}\_ n \geq m \wedge n \in \A \Rightarrow m \in
  \A. \]
Since these conditions are geometric implications, they are satisfied if and only if any
stalk~$\A_x$ is an inhabited upward-closed subset of~$\ul{\NN}_x \cong \NN$.
The association
\[ x \in X \quad\longmapsto\quad \min\{ n \in \NN \,|\, n \in \A_x \} \]
thus defines a map~$X \to \NN$. This map is indeed upper semicontinuous, since
if~$n \in \A_x$, there exists an open neighbourhood~$V$ of~$x$ such that the constant
function with value~$n$ is an element of~$\Gamma(V,\A)$ and therefore~$n \in
\A_y$ for all~$y \in V$.

Conversely, let~$\alpha : U \to \NN$ be a upper semi-continous function. Then
\[ \text{$V \subseteq U$ open} \quad\longmapsto\quad \{ f : V \to \NN \,|\, \text{$f$
continuous,\ $f \geq \alpha$ on~$V$} \} \]
is a subobject of~$\ul{\NN}|_U$ which internally is inhabited and upward-closed.
Further details are left to the reader.
\end{proof}

Under the correspondence given by the lemma, locally \emph{constant}
functions map exactly to the (image of the) \emph{ordinary} internal natural numbers
(in the completed natural numbers).
In a similar vein, the sheaf given by the internal construction of
the set of \emph{all} upward-closed subsets of the natural numbers (not
only the inhabited ones) is canonically isomorphic to the sheaf of
upper semicontinuous functions with values in~$\NN \cup \{ +\infty
\}$.

Note that the correspondence can be used to understand classical facts about
upper semicontinuous functions as features of intuitionistic number theory. For
instance, it is well-known that any upper semicontinuous~$\mathbb{N}$-valued
function on an arbitrary topological space is locally constant on a dense open subset.
This can be explained as follows: The generalized natural number associated to such a
function is \notnot an ordinary natural number from the internal point of view.
Since ``not not'' translates to ``holding on a dense open subset''
(Proposition~\ref{prop:modops-kripke}), it follows that there is a dense open
subset on which the function corresponds to an ordinary internal natural
number, \ie is locally constant.


\subsection{The upper semicontinuous rank function}
Recall that the rank of an~$\O_X$-module~$\F$ on a scheme~$X$ (or
locally ringed space) at a point~$x \in X$ is defined as the~$k(x)$-dimension
of the vector space~$\F_x \otimes_{\O_{X,x}} k(x)$. If we assume that~$\F$ is
of finite type around~$x$, this dimension is finite and equals the minimal
number of elements needed to generate~$\F_x$ as an~$\O_{X,x}$-module (by
Nakayama's lemma).

In the internal language, we can define an element of~$\widehat\NN$ by
\begin{multline*}
  \rank\F \defeq \min\{ n \in \NN \,| \\
  \speak{there is a gen.\@ family for~$\F$ consisting of~$n$ elements} \} \in \widehat\NN.
\end{multline*}
If the module~$\F$ is finite locally free, it will be a finite free module from the
internal point of view and the rank defined in this way will be an
actual natural number (see below); but in general, the rank is really an element of the
completion.

\begin{prop}\label{prop:rank-function-internally}
Let~$\F$ be an~$\O_X$-module of finite type on a scheme~$X$ (or locally ringed
space). Under the correspondence given by the previous lemma, the internally
defined rank maps to the rank function of~$\F$.
\end{prop}
\begin{proof}
We have to show that for any point~$x \in X$ and natural number~$n$, there
exists a generating family for~$\F_x$ consisting of~$n$
elements if and only if there exists an open neighbourhood~$U$ of~$x$ such that
\[ U \models \speak{there exists a generating family
for~$\F$ consisting of~$n$ elements}. \]
The ``if'' direction is obvious. For the ``only if'' direction, consider
(liftings to local sections of a)
generating family~$s_1,\ldots,s_n$ of~$\F_x$. Since~$\F$ is of finite type,
there also exist sections~$t_1,\ldots,t_m$ on some neighbourhood~$V$ of~$x$ which
generate any stalk~$\F_y$, $y \in V$. Since the~$t_i$ can be expressed as a
linear combination of the~$s_j$ in~$\F_x$, the same is true on some open
neighbourhood~$U \subseteq V$ of~$x$. On this neighbourhood, the~$s_j$ generate
any stalk~$\F_y$, $y \in U$, so by geometricity we have
\[ U \models \speak{$s_1,\ldots,s_n$ generate~$\F$}. \qedhere \]
\end{proof}
\begin{rem}Once we understand when properties holding at a point spread to
neighbourhoods, we will be able to give a simpler proof of the proposition (see
Lemma~\ref{lemma:gen-family-n}).\end{rem}

\begin{lemma}Let~$X$ be a scheme (or a locally ringed space). Let~$\F$ be an~$\O_X$-module of
finite type. If~$\F$ is finite locally free, its rank function is locally
constant. The converse holds if~$X$ is a reduced scheme.
\end{lemma}
\begin{proof}The rank function is locally constant if and only if internally,
the rank of~$\F$ is an actual natural number. Also recall that the structure
sheaf fulfills a certain field condition if~$X$ is a reduced
scheme (Corollary~\ref{cor:field-reduced}). Therefore it suffices to give a
proof of the following fact of intuitionistic linear algebra: Let~$R$ be a
local ring. Let~$M$ be a finitely generated~$R$-module. If~$M$ is finite
free, its rank is an actual natural number. The converse holds if~$R$ fulfills
the field condition that any element which is not invertible is zero.

So assume that such a module~$M$ is finite free. Then it is isomorphic
to~$R^n$ for some actual natural number~$n$; by the internal proof in
Lemma~\ref{lemma:kernel-of-epi-fingen}, the rank of~$M$ is therefore this
number~$n$ (for any surjection~$R^m \twoheadrightarrow R^n$ it holds that~$m
\geq n$).

Conversely, assume that the rank of~$M$ is an actual natural number. Then
there exists a minimal generating family~$x_1,\ldots,x_n\?M$. We can verify that this family is
indeed linearly independent (and thus a basis, demonstrating that~$M$ is finite
free): Let~$\sum_i a_i x_i = 0$ with~$a_i\?R$. If any~$a_i$ were
invertible, the family~$x_1,\ldots,x_{i-1},x_{i+1},\ldots,x_n$ would too
generate~$M$, contradicting the minimality. So each~$a_i$ is not invertible.
By the field property of~$R$, each~$a_i$ is zero.
\end{proof}

\begin{lemma}\label{lemma:locally-free-dense}
Let~$X$ be a reduced scheme. Let~$\F$ be an~$\O_X$-module of
finite type. Then~$\F$ is finite locally free on a dense open subset.\end{lemma}
\begin{proof}Since ``dense open'' translates to ``not not'' in the internal
language (Proposition~\ref{prop:modops-kripke}), it suffices to give an
intuitionistic proof of the following fact: Let~$R$ be a local ring which fulfills an
appropriate field condition. Let~$M$ be a finitely generated~$R$-module.
Then~$R$ is \notnot finite free.

By Remark~\ref{rem:surjectivity-embedding}, the rank of such a module~$M$ is
\notnot an actual natural number. By the last part of the
previous proof, it thus follows that~$M$ is \notnot finite free.
\end{proof}

\begin{rem}Note that besides basics on natural numbers in an intuitionistic
setting and some dictionary terms (``reduced'', ``finite locally free'',
``finite type``, ``dense open''), this proof does not depend on any further
tools. In particular, Nakayama's lemma and facts about semicontinuous functions
do not enter. For the (more complex) standard proof of this fact, see for
instance~\cite{vakil:foag}, where the claim is dubbed an ``important hard
exercise'' (Exercise~13.7.K).\end{rem}


\subsection{The upper semicontinuous dimension function} Recall that the
dimension of a topological space~$X$ at a point~$x \in X$ is defined as the
infimum
\[ \dim_x X \defeq \inf\{ \dim U \,|\, \text{$U$ open neighbourhood of~$x$} \}. \]
One may restrict to open \emph{connected} neighbourhoods of~$X$, since an
arbitrary open neighbourhood of~$x$ contains such a one and the dimension
decreases (weakly) on subsets.

The map~$X \to \NN \cup \{+\infty\},\ x \mapsto \dim_x X$ is upper
semicontinuous and thus corresponds to an internal generalized (possibly
unbounded) natural number. The following proposition shows that this number has
an explicit description.

\begin{prop}Let~$X$ be a scheme. Then the upper semicontinuous function
associated to the internal number ``Krull dimension of~$\O_X$'' is the
dimension function~$x \mapsto \dim_x X$.\end{prop}
\begin{proof}Internally, we define the Krull dimension of~$\O_X$ as the infimum
over all natural numbers~$n$ such that~$\O_X$ is of Krull
dimension~$\leq n$. This infimum need not exist in the natural numbers, of
course; so we really mean the upward-closed set~$\A$ of all those numbers. (It
is inhabited if and only if, from the external perspective, the dimension
of~$X$ is locally finite. In this case, it defines a generalized natural number.)

We thus have to show for any point~$x \in X$:
\[ \inf\{ n \in \NN \cup \{+\infty\} \,|\, n \in \A_x \} =
  \dim_x X. \]
The condition on~$n$ can be expressed as follows, where we write~``$\ul{n}$''
to denote the constant function with value~$n$:
\begin{align*}
  &\ n \in \A_x \\
  \Longleftrightarrow &\
  \text{for some open neighbourhood~$U$ of~$x$, $\ul{n} \in \Gamma(U,\A)$} \\
  \Longleftrightarrow &\
  \text{for some open connected neighbourhood~$U$ of~$x$, $\ul{n} \in \Gamma(U,\A)$} \\
  \Longleftrightarrow &\
  \text{for some open connected neighbourhood~$U$ of~$x$,} \\
  & \qquad\qquad U \models \speak{$\O_X$ is of Krull dimension~$\leq n$} \\
  \Longleftrightarrow &\
  \text{for some open connected neighbourhood~$U$ of~$x$,} \\
  & \qquad\qquad \dim U \leq n
\end{align*}
We thus have:
\begin{align*}
  \inf\{ n \,|\, n \in \A_x \} &=
    \inf\{ \dim U \,|\, \text{$U$ open connected neighbourhood of~$x$} \} \\
  &= \dim_x X. \qedhere
\end{align*}
\end{proof}


\section{Modalities}
\label{sect:modalities}

Philosophers and logicians do not only study what is \emph{true}, but also what
is \emph{known}, what is \emph{believed}, what is \emph{possible}, and so on.
Such \emph{modalities} are absent from the usual mathematical practice.
However, it turns out that a specific kind of such modalities plays a role in
understanding when properties spread from points to neighbourhoods.

Briefly, this is because for any point~$x$ of a topological space~$X$, there
exists a modal operator~$\Box$ such that for any formula~$\varphi$ of the
internal language of the sheaf topos~$\Sh(X)$, the internal
statement~$\Box\varphi$ means that~$\varphi$ holds on some open neighbourhood
of the given point~$x$. In this way, we can reduce sheaf-theoretic questions to
questions of modal intuitionistic (non-sheafy) mathematics.

The techniques developed in this section also enable us to use the internal
language of~$\Sh(X)$ to talk about sheaves on \emph{subspaces} of~$X$ (and more
general \emph{sublocales} of~$X$).

Topological interpretations of modal logic were studied before, for instance by
Awodey and Kishida~\cite{awodey-kishida:modal}. However, they study a
different kind of modal operators, not corresponding to the Lawvere--Tierney
topologies of topos theory, and pursue different goals.


\subsection{Basics on truth values and modal operators}

\begin{defn}The \emph{set of truth values~$\Omega$} is the powerset of the
singleton set~$1 \defeq \{\star\}$, where~$\star$ is a formal symbol.\end{defn}

In classical logic, any subset of~$\{\star\}$ is either empty or inhabited, so
that~$\Omega$ contains exactly two elements, the empty set (``false'')
and~$\{\star\}$ (``true''). But
in intuitionistic logic, this can not be shown; indeed, if we interpret the
definition in the topos of sheaves on a space~$X$, we obtain a (large) sheaf~$\Omega$
with
\[ \text{$U \subseteq X$ open} \quad\longmapsto\quad \Gamma(U,\Omega) = \{ V \subseteq U \,|\, \text{$V$
open} \}. \]
(This is because by definition of~$\Omega$ as the power object of the terminal
sheaf~$1$, sections of~$\Omega$ on an open subset~$U$ correspond to
subsheaves~$\F \hookrightarrow 1|_U$, and those are given by the greatest open
subset~$V \subseteq U$ such that~$\Gamma(V,\F)$ is inhabited.)
Obviously, in general, this sheaf has many sections, in particular more than
the binary coproduct~$1 \amalg 1$ (unless any open subset of~$X$ is also
closed).

The \emph{truth value} of a formula~$\varphi$ is by definition the subset
$\{ x \in 1 \,|\, \varphi \} \in \Omega$, where~``$x$'' is a fresh variable not
appearing in~$\varphi$. This subset is inhabited if and only
if~$\varphi$ holds and is empty if and only if~$\neg\varphi$ holds.
Conversely, we can associate to a subset~$F \subseteq 1$ the
proposition~$\speak{$F$ is inhabited}$.

By the above description of~$\Omega$ in
a sheaf topos~$\Sh(X)$, the interpretation of the truth value
of a formula~$\varphi$ in the internal language of~$\Sh(X)$ is a certain open
subset of~$X$. Tracing the definitions, we see that this open subset is
precisely the largest open subset on which~$\varphi$ holds, \ie the union of
all open subsets~$U \subseteq X$ such that~$U \models \varphi$.

Under the correspondence of formulas with truth values, logical operations
like~$\wedge$ and~$\vee$ map to set-theoretic operations like~$\cap$ and~$\cup$
-- for instance, we have
\[ \{ x \in 1 \,|\, \varphi \} \cap \{ x \in 1 \,|\, \psi \} =
  \{ x \in 1 \,|\, \varphi \wedge \psi \}. \]
This justifies a certain abuse of notation: We will sometimes treat elements
of~$\Omega$ as propositions and use logical instead of set-theoretic
connectives. In particular, if~$\varphi$ and~$\psi$ are elements of~$\Omega$,
we will write~``$\varphi \Rightarrow \psi$'' to mean~$\varphi \subseteq \psi$;
``$\bot$'' to mean~$\emptyset$; and~``$\top$'' to mean~$1$.

\begin{defn}A \emph{modal operator} (or \emph{Lawvere--Tierney topology}) is a map~$\Box : \Omega \to \Omega$ such
that for all~$\varphi, \psi \in \Omega$,
\begin{enumerate}
\item $\varphi \Longrightarrow \Box\varphi$,
\item $\Box\Box\varphi \Longrightarrow \Box\varphi$,
\item $\Box(\varphi \wedge \psi) \Longleftrightarrow \Box\varphi \wedge \Box\psi$.
\end{enumerate}
\end{defn}

The intuition is that~$\Box\varphi$ is a certain weakening of~$\varphi$, where
the precise meaning of ``weaker'' depends on the modal operator. By the second
axiom, weakening twice is the same as weakening once.

In classical logic, where~$\Omega = \{ \bot, \top \}$, there are only two modal
operators: the identity map and the constant map with value~$\top$.
Both of these are not very interesting: The identity operator does not weaken
propositions at all, while the constant operator weakens every proposition to
the trivial statement~$\top$.

In intuitionistic logic, there can potentially exist further modal operators.
For applications to algebraic geometry, the following four operators will have
a clear geometric meaning and be of particular importance:
\begin{enumerate}
\item $\Box\varphi \defequiv (\alpha \Rightarrow \varphi)$, where~$\alpha$ is a
fixed proposition.
\item $\Box\varphi \defequiv (\varphi \vee \alpha)$, where~$\alpha$ is a
fixed proposition.
\item $\Box\varphi \defequiv \neg\neg\varphi$ (the \emph{double negation
modality}).
\item $\Box\varphi \defequiv ((\varphi \Rightarrow \alpha) \Rightarrow \alpha)$,
where~$\alpha$ is a fixed proposition.
\end{enumerate}

\begin{lemma}Any modal operator~$\Box$ is monotonic, \ie if~$\varphi
\Rightarrow \psi$, then~$\Box\varphi \Rightarrow \Box\psi$. Furthermore, there
holds a modus ponens rule: If~$\Box\varphi$ holds, and~$\varphi$
implies~$\Box\psi$, then~$\Box\psi$ holds as well.\end{lemma}
\begin{proof}Assume~$\varphi \Rightarrow \psi$. This is equivalent to
supposing~$\varphi \wedge \psi \Leftrightarrow \varphi$. We are to show
that~$\Box\varphi \Rightarrow \Box\psi$, \ie that~$\Box\varphi \wedge
\Box\psi \Leftrightarrow \Box\varphi$. This follows since by the third
axiom on a modal operator, we have~$\Box\varphi \wedge \Box\psi \Leftrightarrow
\Box(\varphi \wedge \psi)$, and~$\Box$ respects equivalence of propositions.

For the second statement, consider that if~$\varphi \Rightarrow \Box\psi$, by
monotonicity and the second axiom on a modal operator it follows
that~$\Box\varphi \Rightarrow \Box\Box\psi \Rightarrow \Box\psi$.
\end{proof}

The modus ponens rule justifies the following proof scheme: When trying to
show, given that some boxed statement~$\Box\varphi$ holds, that some further
boxed statement~$\Box\psi$ holds, we may give a proof of~$\Box\psi$ under the
stronger assumption~$\varphi$. Symbolically:
\[ (\Box\varphi \Rightarrow \Box\psi) \Longleftrightarrow
  (\varphi \Rightarrow \Box\psi). \]


\subsection{Geometric meaning}\label{sect:modalities-geometric-meaning}
Let~$X$ be a topological space. As discussed
above, an open subset~$U \subseteq X$ defines an internal truth value (a global
section of the sheaf~$\Omega$). We also denote it by~``$U$'', such that
\[ V \models U \quad\Longleftrightarrow\quad V \subseteq U \]
for any open subset~$V \subseteq X$. (Shortcutting the various intermediate
steps, this can also be taken as a definition of~``$V \models U$''.)
If~$A \subseteq X$ is a closed subset, there is thus an internal truth
value~$A^c$ corresponding to the open subset~$A^c = X \setminus A$. If~$x \in
X$ is a point, we define~``$\notat{x}$'' to denote the truth value
corresponding to~$\Int(X \setminus \{x\})$, such that
\[ V \models \notat{x} \quad\Longleftrightarrow\quad V \subseteq \Int(X
\setminus \{ x \}) \quad\Longleftrightarrow\quad x \not\in V. \]

\begin{prop}\label{prop:modops-kripke}
Let~$U \subseteq X$ be a fixed open and~$A \subseteq X$ be a fixed
closed subset. Let~$x \in X$. Then, for any open subset~$V \subseteq X$, it
holds that:
\[ \renewcommand{\arraystretch}{1.3}\begin{array}{@{}lcl@{}}
  V \models (U \Rightarrow \varphi) &\Longleftrightarrow&
    V \cap U \models \varphi. \\[0.3em]
  V \models (\varphi \vee A^c) &\Longleftrightarrow&
    \textnormal{there is an open subset~$W \subseteq V$} \\
  && \quad\quad \textnormal{containing~$A \cap V$ such that $W \models \varphi$.} \\[0.3em]
  V \models \neg\neg\varphi &\Longleftrightarrow&
    \textnormal{there is a dense open subset~$W \subseteq V$ s.\,th.\@ $W \models
    \varphi$.} \\[0.3em]
  V \models ((\varphi \Rightarrow \notat{x}) \Rightarrow \notat{x}) &\Longleftrightarrow&
    \textnormal{$x \not\in V$ or there is an open neighbourhood~$W \subseteq V$} \\
  && \quad\quad \textnormal{of~$x$ such that $W \models \varphi$.}
\end{array} \]
\end{prop}
\begin{proof}
\begin{enumerate}
\item Omitted.

\item Let~$V \models \varphi \vee A^c$. Then there exists an open covering~$V =
\bigcup_i V_i$ such that for each~$i$, $V_i \models \varphi$ or $V_i \subseteq
A^c$. Let~$W \subseteq V$ be the union of those~$V_i$ such that~$V_i \models \varphi$.
Then~$W \models \varphi$ by the locality of the internal language and~$A \cap V
\subseteq W$.

Conversely, let~$W \subseteq V$ be an open subset containing~$A \cap V$ such
that~$W \models \varphi$. Then~$V = W \cup (V \cap A^c)$ is an open covering
attesting~$V \models \varphi \vee A^c$.

\item For the ``only if'' direction, let~$W \subseteq V$ be the largest
open subset on which~$\varphi$ holds, \ie the union of all open subsets
of~$V$ on which~$\varphi$ holds. For the ``if'' direction, we may assume that
the given set~$W$ is also the largest open subset on which~$\varphi$ holds (by
enlarging~$W$ if necessary). The claim then follows by the following chain of
equivalences:
\begin{align*}
  &\ V \models \neg\neg\varphi \\
  \Longleftrightarrow&\ \forall \text{$Y \subseteq V$ open}\_
    \Bigl(\forall \text{$Z \subseteq Y$ open}\_ (Z \models \varphi) \Rightarrow Z
    = \emptyset\Bigr) \Longrightarrow Y = \emptyset \\
  \Longleftrightarrow&\ \forall \text{$Y \subseteq V$ open}\_
    \Bigl(\forall \text{$Z \subseteq Y$ open}\_ Z \subseteq W \Rightarrow Z
    = \emptyset\Bigr) \Longrightarrow Y = \emptyset \\
  \Longleftrightarrow&\ \forall \text{$Y \subseteq V$ open}\_
    Y \cap W = \emptyset \Longrightarrow Y = \emptyset \\
  \Longleftrightarrow&\ \text{$W$ is dense in~$V$.}
\end{align*}

\item Straightforward, since the interpretation of the internal statement with
the Kripke--Joyal semantics is
\[ \forall \text{$Y \subseteq V$ open}\_
  \Bigl(\forall \text{$Z \subseteq Y$ open}\_
    Z \models \varphi \Rightarrow x \not\in Z\Bigr) \Longrightarrow x \not\in
    Y. \qedhere \]
\end{enumerate}
\end{proof}


\subsection{The subspace associated to a modal operator}
\label{sect:subspace-to-modal-operator}
Any modal operator~$\Box : \Omega \to \Omega$ in the sheaf topos of~$X$ induces
on global sections a map
\[ j : \Open(X) \to \Open(X), \]
where~$\Open(X) = \Gamma(X,\Omega)$ is the set of open subsets of~$X$.
Explicitly, it is given by
\begin{align*}
  j(U) &= \text{largest open subset of~$X$ on which~$\Box U$ holds} \\
  &= \bigcup\ \{ V \subseteq X \ |\ \text{$V$ open},\ V \models \Box U \}.
\end{align*}
By the axioms on a modal operator, the map~$j$ fulfills similar such axioms: For any open
subsets~$U, V \subseteq X$,
\begin{enumerate}
\item $U \subseteq j(U)$,
\item $j(j(U)) \subseteq j(U)$,
\item $j(U \cap V) = j(U) \cap j(V)$.
\end{enumerate}
Such a map is called a \emph{nucleus} on~$\Open(X)$. Table~\ref{table:nuclei}
lists the nuclei associated to the four modal operators
of Proposition~\ref{prop:modops-kripke}.

\begin{table}
  \centering
  \renewcommand{\arraystretch}{1.3}
  \begin{tabular}{llll}
    \toprule
    Modal operator & associated nucleus &
      $j(V) = X$ iff \ldots &
      subspace \\\midrule
    $\Box\varphi \defequiv (U \Rightarrow \varphi)$ &
      $j(V) = \Int(U^c \cup V)$ & $U \subseteq V$ & $U$ \\
    $\Box\varphi \defequiv (\varphi \vee A^c)$ &
      $j(V) = V \cup A^c$ & $A \subseteq V$ & $A$ \\
    $\Box\varphi \defequiv \neg\neg\varphi$ &
      $j(V) = \Int(\Clos(V))$ & $V$ is dense in $X$ &
      \multicolumn{1}{p{1cm}}{smallest dense sublocale of~$X$} \\
    $\Box\varphi \defequiv ((\varphi \Rightarrow \notat{x}) \Rightarrow \notat{x})$ &
%      $\Int(\Clos(V \cap \Clos\{x\}) \cup (X \setminus \Clos\{x\}))$ &
      $\begin{array}{@{}ll@{}}
        j(V) = X \setminus \Clos\{x\}, & \text{if $x \not\in V$} \\
        j(V) = X, & \text{if $x \in V$}
      \end{array}$ &
      $x \in V$ & $\{x\}$ \\
    \bottomrule
  \end{tabular}
  \vspace{0.5em}

  \caption{\label{table:nuclei}List of important modal operators and their
  associated nuclei (notation as in Proposition~\ref{prop:modops-kripke}).}
\end{table}

Any nucleus~$j$ defines a subspace~$X_j$ of~$X$, to be described below, with a small caveat: In
general, the subspace~$X_j$ can not be realized as a topological subspace, but
only as a so-called \emph{sublocale}; the notion of a locale is a slight
generalization of the notion of a topological space, in which an underlying set
of points is not part of the definition. Instead, a locale is simply given by a
lattice of arbitrary \emph{opens} satisfying some axioms -- these opens may, but do not necessarily have to,
be sets of points. Sheaf theory carries over to locales essentially unchanged,
since the notions of presheaves and sheaves only refer to open sets and coverings,
but not points.
Accessible introductions to the theory of locales include two notes by
Johnstone~\cite{johnstone:art,johnstone:point}.

\begin{defn}\label{defn:subspace-by-nucleus}Let~$j$ be a nucleus on~$\Open(X)$.
Then the sublocale~$X_j$ of~$X$ is given by the lattice of opens
$\Open(X_j) \defeq \{ U \in \Open(X) \,|\, j(U) = U \}$.
\end{defn}
If~$j$ is induced by a modal operator~$\Box$, we also write~``$X_\Box$''
for~$X_j$. In three of the four cases listed in Table~\ref{table:nuclei}, the
sublocale~$X_\Box$ can indeed be realized as a topological subspace. The only
exception is the sublocale~$X_{\neg\neg}$ associated to the double negation
modality. It can be also be described as the \emph{smallest dense sublocale}
of~$X$; this is obviously a genuine locale-theoretic notion, since there
is (in general) no smallest dense topological subspace
(consider~$\RR$ and its dense subsets~$\QQ$ and~$\RR \setminus \QQ$).

The inclusion~$i : X_j \hookrightarrow X$ can not in general be described on the
level of points, since~$X_j$ might not be realizable as a topological subspace.
But for sheaf-theoretic purposes, it suffices to describe~$i$ on the level of
opens. This is done as follows:
\[ i^{-1} : \Open(X) \lra \Open(X_j),\ \quad U \longmapsto j(U). \]
Thus we can relate the toposes of sheaves on~$X_j$ and~$X$ by the usual
pullback and pushforward functors.
\begin{align*}
  i^{-1} \F &= \text{sheafification of $(U \mapsto \colim_{U \preceq i^{-1}V} \Gamma(V,\F))$} \\
  i_* \G &= (U \mapsto \Gamma(i^{-1}U, \G)) = (U \mapsto \Gamma(j(U), \G))
\end{align*}
As familiar from honest topological subspace inclusions, the pushforward
functor~$i_* : \Sh(X_j) \to \Sh(X)$ is fully faithful and the composition~$i^{-1}
\circ i_* : \Sh(X_j) \to \Sh(X_j)$ is (canonically isomorphic to) the identity.


\subsection{Internal sheaves and sheafification}\label{sect:internal-sheaves}
It turns out that the image of
the pushforward functor~$i_* : \Sh(X_\Box) \to \Sh(X)$, where~$\Box$ is a modal
operator in~$\Sh(X)$, can be explicitly described. Namely, it consists exactly
of those sheaves which from the internal point of view
are so-called~\emph{$\Box$-sheaves}, a notion explained below.

Furthermore, if we identify~$\Sh(X_\Box)$ with its image in~$\Sh(X)$, the
pullback functor is given by an internal sheafification process with respect to
the modality~$\Box$. Thus the external situation of pushforward/pullback
translates to forget/sheafify. This broadens the scope of the internal
language of~$\Sh(X)$: It can not only be used to talk about sheaves on~$X$ in a simple,
element-based language, but also to talk about sheaves on arbitrary subspaces
of~$X$.

To describe the notion of~$\Box$-sheaves and related ones, we switch to the internal
perspective and thus forget that we're working over a base space~$X$; we are simply given a modal operator~$\Box :
\Omega \to \Omega$ and have to take care that our proofs are intuitionistically acceptable. A
reference for the material in this subsection is a preprint by Fer-Jan de
Vries~\cite{vries:sheafification}.\footnote{Note that on page~5 of that
preprint there is a slight typing error: Fact~2.1(i) gives the
characterization of~$j$-closedness, not~$j$-denseness. The correct
characterization of~$j$-denseness in that context is~$\forall b \in B\_ j(b \in
A)$.}

Recall that a set~$S$ is a \emph{subsingleton} if and only if~$\forall x,y\?S\_
x = y$, and that a set~$S$ is a \emph{singleton} if and only if it is a subsingleton and
inhabited (\ie~$\exists x\?S$); this amounts to~$\exists!x\?S$.

\begin{defn}\label{defn:box-sheaves}
A set~$F$ is \emph{$\Box$-separated} if and only if
\[ \forall x,y\?F\_ \Box(x = y) \Longrightarrow x = y. \]
A set~$F$ is a \emph{$\Box$-sheaf} if and only if it is~$\Box$-separated and
\[ \forall S \subseteq F\_
  \Box(\speak{$S$ is a singleton}) \Longrightarrow
  \exists x\?F\_ \Box(x \in S). \]
\end{defn}

The two conditions can be combined: A set~$F$ is a~$\Box$-sheaf if and only if
\[ \forall S \subseteq F\_
  \Box(\speak{$S$ is a singleton}) \Longrightarrow
  \exists! x\?F\_ \Box(x \in S). \]
\XXX{explain how to read these definitions}

\begin{defn}\label{defn:plus-construction}
The \emph{plus construction} of a set~$F$ with respect to~$\Box$ is the set
\[ F^+ \defeq \{ S \subseteq F \,|\, \Box(\speak{$S$ is a singleton}) \}/{\sim},
\]
where the equivalence relation is defined by~$S \sim T \vcentcolon\Leftrightarrow
\Box(S = T)$. There is a canonical map~$F \to F^+$ given by~$x \mapsto
[\{x\}]$. The \emph{$\Box$-sheafi\-fi\-ca\-tion} of a set~$F$ is the
set~$F^{++}$.
\end{defn}

If~$F$ is~$\Box$-separated, then for any subset~$S \subseteq F$ it holds
that
\[ \Box(\speak{$S$ is a singleton}) \quad\Longleftrightarrow\quad
  \speak{$S$ is a subsingleton} \wedge \Box(\speak{$S$ is inhabited}). \]

\begin{rem}The topos of \emph{pre}sheaves on a topological space~$X$ admits an
internal language as well~\cite[Section~VI.7, discussion after
Theorem~1]{moerdijk-maclane:sheaves-logic}. In it, there
exists a modal operator~$\Box$ reflecting the topology of~$X$. A presheaf on~$X$ is separated
in the usual sense if, from the internal perspective of~$\PSh(X)$, it
is~$\Box$-separated; and it is a sheaf if, from the internal perspective, it
is a~$\Box$-sheaf. Furthermore, the~$\Box$-sheafification of a presheaf
(considered as a set from the internal perspective) coincides with the usual
sheafification.\end{rem}

\begin{ex}\label{ex:special-sets-sheaves}
Any singleton set is a~$\Box$-sheaf. The empty set is
always~$\Box$-separated (trivially) and is a~$\Box$-sheaf if and only
if~$\Box\bot \Rightarrow \bot$.\end{ex}

We will see geometric examples of~$\Box$-sheaves in further sections.
For instance, on an integral or locally Noetherian scheme~$X$, the structure sheaf~$\O_X$
is~$\neg\neg$-separated and its~$\neg\neg$-sheafification is the sheaf~$\K_X$
of rational functions (Proposition~\ref{prop:kx-is-negneg-sheafification}).

\begin{lemma}For any set~$F$, it holds that: \begin{enumerate}
\item $F^+$ is~$\Box$-separated.
\item The canonical map~$F \to F^+$ is injective if and only if~$F$
is~$\Box$-separated.
\item If~$F$ is~$\Box$-separated, then $F^+$ is a~$\Box$-sheaf.
\item If~$F$ is a~$\Box$-sheaf, then the canonical map~$F \to F^+$ is bijective.
\end{enumerate}
Let``$\Sh_\Box(\Set)$'' denote the full subcategory of~$\Set$ consisting of
the~$\Box$-sheaves. Then it holds that:
\begin{enumerate}
\addtocounter{enumi}{4}
\item The functor~$(\placeholder)^+ : \Set \to \Set$ is left exact.
\item The functor~$(\placeholder)^{++} : \Set \to \Sh_\Box(\Set)$ is left exact and left
adjoint to the forgetful functor~$\Sh_\Box(\Set) \to \Set,\ F \mapsto F$.
\end{enumerate}\end{lemma}
\begin{proof}These are all straightforward, and it fact simpler than their
classical counterparts, since there are no colimit constructions which would have to
be dealt with.
\end{proof}

\begin{rem}\label{rem:epi-in-box-sheaves}
As is to be expected from the familiar inclusion of sheaves in
presheaves on topological spaces, the forgetful functor~$\Sh_\Box(\Set) \to \Set$
does not in general preserve colimits. It is instructive to see why
epimorphisms in~$\Sh_\Box(\Set)$ need not be epimorphisms in~$\Set$: A map~$f:A
\to B$ between~$\Box$-sheaves is an epimorphism in~$\Sh_\Box(\Set)$ if and only
if
\[ \forall y\?B\_ \Box(\exists x\?X\_ f(x) = y), \]
\ie preimages do not need to exist, it suffices for them to~``$\Box$-exist''.
(Using results about the~$\Box$-translation, to be introduced below, this
characterization will be obvious.) This condition is intuitionistically weaker
than the condition that~$f$ is an epimorphism in~$\Set$, \ie that~$f$ is
surjective. Compare this to the failure of the forgetful functor~$\Sh(X)
\to \PSh(X)$ to preserve epimorphisms: A morphism of sheaves does not need to
have preimages for any local section in order to be an epimorphism. Instead, it
suffices for any local section to \emph{locally} have preimages.\end{rem}

\begin{prop}Let~$X$ be a topological space. Let~$\Box$ be a modal operator
in~$\Sh(X)$. Let~$i : X_\Box \hookrightarrow X$ be the inclusion of the
associated sublocale. Corestricting the pushforward functor~$i_* : \Sh(X_\Box) \to
\Sh(X)$ to its essential image, it induces an equivalence~$\Sh(X_\Box) \simeq
\Sh_\Box(\Sh(X))$ between the category of sheaves on~$X_\Box$ and the category
of~$\Box$-sheaves in~$\Sh(X)$.
\end{prop}
\begin{proof}For the further development of the theory, we need the statement
of this proposition, but not the proof, which really is routine in dealing with
subtoposes and modal operators. Nevertheless, a proof goes like follows:
Combine Example~A4.6.2(a) and Theorem~C1.4.7
of~\cite{johnstone:elephant} and note that for a topos of sheaves on a
locale~$Y$, it holds that~$\Open(Y) = \Gamma(Y, \Omega_{\Sh(Y)})$, and that the
subobject classifier of~$\Sh_\Box(\Sh(X))$ is~$\{ \varphi : \Omega_{\Sh(X)} \,|\,
\Box \varphi \Leftrightarrow \varphi \}$.
\end{proof}


\subsection{Sheaves for the double negation modality}
\label{sect:negneg-sheaves}

Recall that if~$\Box$ is the modal operator associated to a sub\emph{space}~$Y$
of a topological space~$X$, then the sheaves on~$X$ which are~$\Box$-sheaves
are easy to describe: These are precisely the sheaves in the essential image of
the pushforward functor~$\Sh(Y) \to \Sh(X)$. For the double negation modality,
the same is true, only that~$Y$ is then the perhaps unfamiliar \emph{smallest
dense sublocale} of~$X$.

The following proposition gives a characterization of~$\neg\neg$-separated
presheaves and~$\neg\neg$-sheaves in explicit terms.

\begin{prop}\label{prop:negneg-sheaves}
Let~$X$ be a topological space. Let~$\F$ be a sheaf on~$X$. Then:
\begin{enumerate}
\item $\F$ is~$\neg\neg$-separated if and only if any two local sections
of~$\F$, which are defined on a common domain and which agree on a dense open
subset of their domain, are already equal.
\item $\F$ is a~$\neg\neg$-sheaf if and only if it is~$\neg\neg$-separated and
for any open subset~$U \subseteq X$ and any open subset~$V \subseteq U$ dense
in~$U$, any~$V$-section of~$\F$ extends to a~$U$-section of~$\F$.
\item If~$\F$ is~$\neg\neg$-separated, the sections of $\F^+$ on an open
subset~$U \subseteq X$ can be described by pairs~$(V,s)$, where~$V$ is a dense
open subset of~$U$ and~$s$ is a section of~$\F$ on~$V$. Two such pairs~$(V,s),
(V',s')$ determine the same element in~$\Gamma(U,\F^+)$ if and only if~$s$ and~$s'$
agree on~$V \cap V'$.
\end{enumerate}
\end{prop}
\begin{proof}
The first statement is obvious from the definition of~$\neg\neg$-separatedness
(Definition~\ref{defn:box-sheaves} for~$\Box = \neg\neg$) and the geometric
interpretation of double negation (Proposition~\ref{prop:modops-kripke}).

For the second statement, we need to show that if~$\F$
is~$\neg\neg$-separated,~$\F$ has the extension property if and only if
\begin{multline*}
  \Sh(X) \models \forall \S \? \P(\F)\_
  \speak{$\S$ is a subsingleton} \wedge
  \neg\neg(\speak{$\S$ is inhabited}) \Longrightarrow \\
  \exists x\?\F\_ \neg\neg(x \in \S).
\end{multline*}
Note that a section~$\S \in \Gamma(U,\P(\F))$ which internally is a
subsingleton and \notnot inhabited is precisely a subsheaf~$\S \hookrightarrow
\F|_U$ such that all stalks~$\S_x$, $x \in U$ are subsingletons and such that for
some dense open subset~$V \subseteq U$, the stalks~$\S_x$, $x \in V$ are
inhabited. This is precisely the datum of a section of~$\F$ defined on some
dense open subset of~$U$: Consider the gluing of the unique germs in~$\S_x$ for
those points~$x$ such that~$\S_x$ is inhabited. (Conversely, a section~$s \in
\Gamma(V,\F)$ defines a subsheaf~$\S$ by setting~$\Gamma(W,\S) \defeq \{ s|_W \,|\,
W \subseteq V \}$.)

In view of this explicit description and the observation that the asserted
existence~(``$\exists x\?\F\_ \neg\neg(x \in \S)$'') is actually a question of
unique existence, the second statement follows.

For the third statement, one can check that the presheaf on~$X$ defined by
\[ \text{$U \subseteq X$ open} \quad\longmapsto\quad
  \{ (V,s) \,|\, \text{$V \subseteq U$ dense open},\ s \in \Gamma(V,\F)
  \}/{\sim} \]
is in fact a sheaf (with respect to the topology of~$X$), internally a $\neg\neg$-sheaf,
and that it has the universal property of the~$\neg\neg$-sheafification
of~$\F$.
\end{proof}

The conditions~(1) and~(2) of the previous proposition can be
summarized as follows: A sheaf~$\F$ on a topological space is
a~$\neg\neg$-sheaf if and only if, for any open subset~$U \subseteq X$, the
restriction map~$\Gamma(\Int\Clos U, \F) \to \Gamma(U,\F)$ is
bijective~\cite[Lemma~36]{jackson:sheaf-theoretic-measure-theory}.

In the case that~$X$ contains a \emph{generic point}, that is a point~$\xi \in X$
such that~$\Clos\{\xi\} = X$, we can describe the sublocale~$X_{\neg\neg}$ in
very explicit terms: In this case, it coincides with the subspace~$\{\xi\}$.
Such a point exists and is unique if~$X$ is an irreducible scheme and need not
exist otherwise.

\begin{lemma}\label{lemma:negneg-generic-point}
Let~$X$ be a topological space and~$\xi \in X$ a point such
that~$\Clos\{\xi\} = X$. Then the modal operator~$\Box \defequiv ((\placeholder
\Rightarrow \notat{\xi}) \Rightarrow \notat{\xi})$ coincides with the double
negation modality and~$X_{\neg\neg} = \{\xi\}$ as sublocales of~$X$.\end{lemma}
\begin{proof}The semantics of the formula~$\notat{\xi}$ was defined by the
equivalence
\[ U \models \notat{\xi} \quad\Longleftrightarrow\quad
  \xi \not\in U. \]
By the assumption on~$\xi$, this is equivalent to requiring~$U = \emptyset$.
Thus for any open subset~$U$ the formulas~$\notat{\xi}$ and~$\bot$ have the
same meaning; they are therefore logically equivalent from the internal point of
view. The given modal operator thus simplifies to
\[ \Box\varphi \quad\equiv\quad ((\varphi \Rightarrow \notat{\xi}) \Rightarrow \notat{\xi})
  \quad\Leftrightarrow\quad ((\varphi \Rightarrow \bot) \Rightarrow \bot)
  \quad\Leftrightarrow\quad \neg\neg\varphi. \]
The second claim follows Table~\ref{table:nuclei}.
\end{proof}

\begin{cor}\label{cor:negneg-generic-point-pushpull}
Let~$X$ be a topological space and~$\xi \in X$ a point such
that~$\Clos\{\xi\} = X$. Since~$X_{\neg\neg} = \{\xi\}$, the
category of~$\neg\neg$-sheaves in~$\Sh(X)$ coincides with the category of
sheaves on~$\{\xi\}$ and can therefore be identified with the category of sets.
Under this identification,
\begin{enumerate}
\item sheafifying an object~$\F \in \Sh(X)$ with respect
to the double negation modality (\ie pulling back to~$X_{\neg\neg}$) is the
same as calculating its generic stalk~$\F_\xi$ and
\item pushing forward a set~$M$ along~$X_{\neg\neg} \hookrightarrow X$ is the
same as calculating the constant sheaf associated to~$M$.
\end{enumerate}
\end{cor}
\begin{proof}The first statement follows because pulling back to~$X_{\neg\neg}$
is the same as pulling back to~$\{\xi\}$. The pushforward of a set~$M$,
considered as a sheaf on~$X_{\neg\neg}$, to~$M$ is explicitly given by
\[ U \quad\longmapsto\quad \begin{cases}
  M, & \text{if $U \neq \emptyset$,} \\
  \{\star\}, & \text{else.}
\end{cases} \]
We omit the routine verification that this sheaf coincides with the constant
sheaf~$\underline{M}$ associated to~$M$.
\end{proof}

The following technical lemma will occasionally be handy. It is an internal
reflection of the fact that an open subset of an affine scheme can always be
written as the union of standard open subsets. We will generalize it
to schemes which are not necessarily irreducible in
Section~\ref{sect:rational-functions} (see
Lemma~\ref{lemma:dense-standard-reflection-generalized}).

\begin{lemma}\label{lemma:dense-standard-reflection}
Let~$X$ be an irreducible scheme. Let~$\varphi$ be any formula
over~$X$. Then
\[ \Sh(X) \models \neg\neg\varphi \Longrightarrow \exists f\?\O_X\_
  \neg\neg(\speak{$f$ \inv}) \wedge (\speak{$f$ \inv} \Rightarrow \varphi). \]
\end{lemma}
\begin{proof}We may assume that~$X$ is the spectrum of an integral domain~$A$
and that there is a dense open subset~$U \subseteq X$ on which~$\varphi$ holds.
The open set~$U$ may be covered by standard open subsets~$D(f_i)$; by the
irreducibility hypothesis, at least one of these is itself
dense. We may take this~$f_i$ as the sought~$f$.
\end{proof}

We can now also follow up on a promise made earlier and prove the following
somewhat tangential lemma.
\begin{lemma}\label{lemma:boolean-dense}
Let~$X$ be a topological space. The internal language of~$\Sh(X)$ is Boolean if
and only if for any open subset~$U \subseteq X$ it holds that~$U$ is the only
dense open subset of~$U$.
\end{lemma}
\begin{proof}That the internal language of~$\Sh(X)$ is Boolean amounts to
\[ \Sh(X) \models \forall \varphi\?\Omega\_ \neg\neg\varphi \Rightarrow
\varphi. \]
This is equivalent to the external statement that for any open subset~$U
\subseteq X$ and for any open subset~$V \subseteq U$ it holds that: If~$V$ is
dense in~$U$, then~$V$ is equal to~$U$.
\end{proof}


\subsection{\texorpdfstring{The~$\Box$-translation}{The □-translation}}
There is certain well-known transformation~$\varphi
\mapsto \varphi^{\neg\neg}$ on formulas, the \emph{double negation
translation}, with the following curious property: A formula~$\varphi$ is
derivable in classical logic if and only if its
translation~$\varphi^{\neg\neg}$ is derivable in intuitionistic logic. The
translation~$\varphi^{\neg\neg}$ is obtained from~$\varphi$ by putting
``$\neg\neg$'' before any subformula, \ie before any~``$\exists$''
and~``$\forall$'', around any logical connective, and around any atomic
statement (``$x=y$'', ``$x \in A$'').

We will describe a slight generalization of the double negation translation,
the~$\Box$-translation for any modal operator~$\Box$. It will be pivotal
for using the internal language of a space~$X$ to express internal statements
about sheaves defined on subspaces of~$X$. The~$\Box$-translation has been studied
in other contexts
before~\cite{aczel:russell-prawitz,escardo:oliva:peirce-shift}. To the best of
my knowledge, this application -- expressing the internal language of
subtoposes in the internal language of the ambient topos -- is new.

\begin{defn}The~\emph{$\Box$-translation} is recursively defined as follows.
\newcommand{\optBox}{\textcolor{gray}{\Box}}
\begin{align*}
  (f = g)^\Box &\defequiv \Box(f = g) \\
  (x \in A)^\Box &\defequiv \Box(x \in A) \\
  \top^\Box &\defequiv \Box\top \quad \text{($\Leftrightarrow \top$)} \\
  \bot^\Box &\defequiv \Box\bot \\
  (\varphi \wedge \psi)^\Box &\defequiv \optBox(\varphi^\Box \wedge \psi^\Box) &
  \textstyle (\bigwedge_i \varphi_i)^\Box &\defequiv \textstyle \optBox(\bigwedge_i \varphi_i^\Box) \\
  (\varphi \vee \psi)^\Box &\defequiv \Box(\varphi^\Box \vee \psi^\Box) &
  \textstyle (\bigvee_i \varphi_i)^\Box &\defequiv \textstyle \Box(\bigvee_i \varphi_i^\Box) \\
  (\varphi \Rightarrow \psi)^\Box &\defequiv \optBox(\varphi^\Box \Rightarrow \psi^\Box) \\
  (\forall x\?X\_ \varphi)^\Box &\defequiv \optBox(\forall x\?X\_ \varphi^\Box) &
  (\forall X\_ \varphi)^\Box &\defequiv \optBox(\forall X\_ \varphi^\Box) \\
  (\exists x\?X\_ \varphi)^\Box &\defequiv \Box(\exists x\?X\_ \varphi^\Box) &
  (\exists X\_ \varphi)^\Box &\defequiv \Box(\exists X\_ \varphi^\Box)
\end{align*}
\end{defn}

\begin{defn}A formula~$\varphi$ is \emph{$\Box$-stable} if and only
if~$\Box\varphi$ implies~$\varphi$.\end{defn}

\begin{lemma}\begin{enumerate}
\item Formulas in the image of the $\Box$-translation are~$\Box$-stable,
\ie for any formula~$\varphi$ it holds that
$\Box(\varphi^\Box) \Longrightarrow \varphi^\Box$.
\item In the definition of the~$\Box$-translation, one may omit the boxes
printed in gray.
\end{enumerate}\end{lemma}
\begin{proof}The first statement is obvious, since one of the axioms on a modal
operator demands that~$\Box\Box\varphi \Rightarrow \Box\varphi$ for any
formula~$\varphi$. The second statement follows by an induction on the
formula structure. By way of example, we prove the case for~``$\Rightarrow$'':
\newcommand{\withgray}{\text{$\Box$ with the gray parts}}
\newcommand{\withoutgray}{\text{$\Box$ without the gray parts}}
\begin{align*}
  &\ (\varphi \Rightarrow \psi)^\withgray \\
  \Longleftrightarrow &\ \Box(\varphi^\withgray \Rightarrow \psi^\withgray) \\
  \Longleftrightarrow &\ (\varphi^\withgray \Rightarrow \psi^\withgray) \\
  \Longleftrightarrow &\ (\varphi^\withoutgray \Rightarrow \psi^\withoutgray) \\
  \Longleftrightarrow &\ (\varphi \Rightarrow \psi)^\withoutgray
\end{align*}
The first step is by definition; the second by~$\Box$-stability
of~$\psi^\withgray$ and the intuitionistic tautology~$\Box(\alpha \Rightarrow
\beta) \Leftrightarrow (\alpha \Rightarrow \beta)$ for~$\Box$-stable
formulas~$\beta$; the third by induction hypothesis; the fourth by
definition.
\end{proof}

\begin{lemma}\label{lemma:box-translation-sound}
The~$\Box$-translation is sound with respect to intuitionistic logic:
Assume that there exists an intuitionistic proof of an
implication~$\varphi \Rightarrow \psi$. Then there is also an intuitionistic
proof of the translated implication~$\varphi^\Box \Rightarrow \psi^\Box$.
\end{lemma}
\begin{proof}This follows by an induction on the structure of intuitionistic
proofs. We have to verify that we can mirror any inference rule of
intuitionistic logic in the translation. For instance, one of the disjunction
rules justifies the following proof scheme: In order to prove~$\varphi \vee
\psi \Rightarrow \chi$, it suffices to give proofs of~$\varphi \Rightarrow
\chi$ and~$\psi \Rightarrow \chi$. We have to justify the translated proof
scheme: In order to prove~$(\varphi \vee \psi)^\Box \Rightarrow \chi^\Box$, it
suffices to give proofs of~$\varphi^\Box \Rightarrow \chi^\Box$ and~$\psi^\Box
\Rightarrow \chi^\Box$.

So assume that proofs of the two implications are given. Further
assume~$(\varphi \vee \psi)^\Box$, \ie~$\Box(\varphi^\Box \vee \psi^\Box)$.
We want to show~$\chi^\Box$. Since this is a~$\Box$-stable statement, we may
assume that in fact~$\varphi^\Box \vee \psi^\Box$ holds. Then the claim is
obvious by the two given proofs.

The cases for the other rules (see Appendix~\ref{appendix:inference-rules} for
a list) are similar and left to the reader.\end{proof}

\begin{rem}The reader well-versed in formal logic will have noticed that we are
mixing syntax and semantics here. The proper way to state the lemma would be
to formally adjoin a box operator to the language of intuitionistic logic,
governed by three inference rules which are modeled on the three axioms on a
modal operator. This formal box operator could then be instantiated by any
concrete modal operator~$\Box : \Omega \to \Omega$.\end{rem}

Soundness of the~$\Box$-translation is important for the following reason.
If~$\varphi$ and~$\varphi'$ are equivalent formulas, we are
accustomed to be able to freely substitute~$\varphi$ by~$\varphi'$ anywhere we
want. Since a modal operator~$\Box$ is semantically defined as a map~$\Omega
\to \Omega$, it is trivially justified that~$\Box\varphi$ and~$\Box\varphi'$
are equivalent: The formulas~$\varphi$ and~$\varphi'$ give rise to the
\emph{same} element~$\{x \in 1 \,|\, \varphi\} = \{x \in 1 \,|\, \varphi'\}$
of~$\Omega$, and therefore their images under~$\Box$ are equal as well.

However, it is \emph{not} clear and in fact wrong in general that the translated formulas~$\varphi^\Box$
and~$(\varphi')^\Box$ are equivalent. This follows only if the soundness
lemma can be applied (two times, once for each direction). We should stress that to apply this
lemma, it is not enough to merely \emph{know} that~$\varphi$ and~$\varphi'$ are
equivalent; instead, there has to be an intuitionistic proof of this
equivalence. This is really a stronger requirement, since an
equivalence~$\varphi \Leftrightarrow \varphi'$ might
hold in a particular model, \ie in the internal language of some particular
topos, without possessing an intuitionistic proof, \ie holding in any topos. We
give an explicit example of this situation below
(Example~\ref{ex:translation-equivalence}).

\begin{lemma}\label{lemma:open-stalk}
Let~$\varphi$ be a formula such that for any subformulas~$\psi$
appearing as antecedents of implications, it holds that~$\psi^\Box \Rightarrow
\Box\psi$. (In particular, this condition is satisfied if there are
no~``$\Rightarrow$'' signs in~$\varphi$ or if~$\varphi$ is a geometric formula.) Then $\Box\varphi \Rightarrow
\varphi^\Box$.\end{lemma}
\begin{proof}We prove this by an induction on the formula structure. All cases
except for~``$\Rightarrow$'' are obvious. For this case, assume~$\Box(\psi
\Rightarrow \chi)$; we are to show that~$(\psi^\Box \Rightarrow \chi^\Box)$.
Since this is a~$\Box$-stable statement, we can in fact assume that~$(\psi
\Rightarrow \chi)$. We then have
\[ \psi^\Box \Longrightarrow \Box\psi \Longrightarrow \Box\chi
\Longrightarrow \chi^\Box, \]
with the first step being by the requirement on antecedents, the second by the
monotonicity of~$\Box$, and the third by the induction hypothesis.
\end{proof}

\begin{lemma}\label{lemma:stalk-open}
Let~$\varphi$ be a geometric formula.
Then $\varphi^\Box \Rightarrow \Box\varphi$.\end{lemma}
\begin{proof}By induction on the formula structure. By way of example, we verify
the case about~``$\bigvee$''. So assume~$\Box(\bigvee_i \varphi_i^\Box)$; we are
to show that~$\Box(\bigvee_i \varphi_i)$. Since this is a boxed statement, we
may in fact assume~$\bigvee_i \varphi_i^\Box$, so for some index~$j$, it holds
that~$\varphi_j^\Box$. By the induction hypothesis, it follows
that~$\Box\varphi_j$. By~$\varphi_j \Rightarrow \bigvee_i \varphi_i$ and the
monotonicity of~$\Box$, it follows that that~$\Box(\bigvee_i \varphi_i)$.
\end{proof}

Note that an analogous argument for infinite conjuctions is not valid:
Assume~$(\bigwedge_i \varphi_i)^\Box$. So for all~$j$,~$\varphi_j^\Box$ holds.
By the induction hypothesis,~$\Box\varphi_j$ holds for any~$j$. But from this
we may not deduce~$\Box\bigwedge_i \varphi_i$, since the axioms on a modal
operator only require commutativity with finite conjuctions. This failure also
has a geometric interpretation, for instance in the special case~$\Box =
\neg\neg$: Given dense open subsets~$U_i$ on which formulas~$\varphi_i$ hold,
we may not conclude that there exists a single dense open subset~$U$ on which
all the formulas~$\varphi_i$ hold.

\begin{rem}In the special case that~$\Box$ is the double negation modality, the
lemma holds with slightly weaker hypotheses: Namely, implications may occur
in~$\varphi$, provided that for their antecedents~$\psi$ it holds that~$\psi
\Rightarrow \psi^\Box$. This is because for the double negation modality,
the formula~$\Box(\psi \Rightarrow \chi)$ is equivalent to~$\psi \Rightarrow
\Box\chi$. (In general, for an arbitrary modality, only the former implies the latter, but not vice versa.) The case
for~``$\Rightarrow$'' in the inductive proof then goes as follows:
Assume~$(\psi \Rightarrow \chi)^\Box$. Then~$\psi \Rightarrow \psi^\Box
\Rightarrow \chi^\Box \Rightarrow \Box\chi$, so~$\Box(\psi \Rightarrow \chi)$.
\end{rem}

\begin{lemma}\label{lemma:stalk-open-with-hypothesis}
Let~$\varphi, \varphi', \psi$ be formulas. Assume that:
\begin{itemize}
\item The formula $\varphi'$ is geometric. (More generally, it suffices for~$(\varphi')^\Box$
to imply~$\Box\varphi'$.)
\item There is an intuitionistic proof that~$\varphi$
and~$\varphi'$ are equivalent under the (only) hypothesis~$\psi$.
\item Both~$\Box\psi$ and~$\psi^\Box$ hold.
\end{itemize}
Then $\varphi^\Box \Rightarrow \Box\varphi$.
\end{lemma}
\begin{proof}
Assume~$\varphi^\Box$. Since~$\psi^\Box$, $(\varphi \wedge \psi)^\Box$. Because
the~$\Box$-translation is sound with respect to intuitionistic logic
(Lemma~\ref{lemma:box-translation-sound})
it follows that~$(\varphi')^\Box$. As~$\varphi'$ is geometric, it follows
that~$\Box\varphi'$. Since~$\Box\psi$ holds, it follows that~$\Box\varphi$.
\end{proof}

\begin{ex}\label{ex:module-zero-geometric}
Let~$M$ be an~$R$-module. Then the statement that~$M$ is zero is not
geometric: $\varphi \defequiv (\forall x\?M\_ x = 0)$. But if~$M$ is generated by some finite
family~$x_1,\ldots,x_n\?M$, then~$\varphi$ is equivalent to the
statement~$\varphi' \defequiv (x_1 = 0
\wedge \cdots \wedge x_n = 0)$ which is geometric; and there is an
intuitionistic proof of this equivalence. Since no implication signs occur
in~$\psi \defequiv \speak{$M$ is generated by~$x_1,\ldots,x_n$}$, the lemma is
applicable and shows that~$\varphi^\Box$ implies~$\Box\varphi$.
This example will gain geometric meaning in
Lemma~\ref{lemma:module-zero-point-neighbourhood}.
\end{ex}

\begin{lemma}For the modality~$\Box$ defined by~$\Box\varphi \defequiv ((\varphi
\Rightarrow \alpha) \Rightarrow \alpha)$, where~$\alpha$ is a fixed
proposition, the~$\Box$-translation of the law of excluded middle holds.
In particular, this applies to the double negation modality~$\Box = \neg\neg$, where~$\alpha =
\bot$.\end{lemma}
\begin{proof}We are to show that~$(\varphi \vee \neg\varphi)^\Box$, \ie that
\[ (((\varphi^\Box \vee (\varphi^\Box \Rightarrow \alpha)) \Longrightarrow
\alpha) \Longrightarrow \alpha. \]
So assume that the antecedent holds. If~$\varphi^\Box$ would hold, then in
particular~$\varphi^\Box \vee (\varphi^\Box \Rightarrow \alpha)$ and thus~$\alpha$
would hold. Therefore it follows that~$(\varphi^\Box \Rightarrow \alpha)$. This
implies~$\varphi^\Box \vee (\varphi^\Box \Rightarrow \alpha)$ and
thus~$\alpha$.
\end{proof}


\subsection{\texorpdfstring{Truth at stalks \vs truth on neighbourhoods}{Truth
at stalks vs. truth on neighbourhoods}}\label{sect:spreading}
We now state the crucial property of the~$\Box$-translation. Recall
that~``$X_\Box$'' denotes the sublocale of~$X$ induced by~$\Box$
(Definition~\ref{defn:subspace-by-nucleus}).
\begin{thm}\label{thm:box-translation-semantically}
Let~$X$ be a topological space. Let~$\Box$ be a modal operator
in~$\Sh(X)$. Let~$\varphi$ be a formula over~$X$. Then
\[ \Sh(X) \models \varphi^\Box \quad\text{iff}\quad
  \Sh(X_\Box) \models \varphi, \]
where on the right hand side, all parameters occuring in~$\varphi$ were pulled
back to~$X_\Box$ along the inclusion~$X_\Box \hookrightarrow X$.
\end{thm}
\XXX{think about powersets appearing as domains of quantification}

We have not yet explicitly stated the Kripke--Joyal semantics for a sheaf topos
over a locale, which~$X_\Box$ is in general. The definition is exactly the same
as in the case for sheaf toposes over a topological space, only that any
mention of ``open sets'' has to be substituted by the more general ``opens''
and any mention of the union operator~``$\bigcup$'' has to be interpreted by
the supremum operator in the lattice of opens of the locale. For~$X_\Box$, this
is~$\sup U_i = j(\bigcup_i U_i)$. Before giving a proof of the theorem, we want
to discuss some of its consequences.

\begin{cor}\label{cor:spreading}
Let~$X$ be a topological space.
\begin{enumerate}
\item Let~$U \subseteq X$ be an open subset and let~$\Box\varphi \defequiv (U
\Rightarrow \varphi)$. Then
\[ \Sh(X) \models \varphi^\Box \quad\text{iff}\quad \Sh(U) \models \varphi. \]
\item Let~$A \subseteq X$ be a closed subset and let~$\Box\varphi \defequiv
(\varphi \vee A^c)$. Then
\[ \Sh(X) \models \varphi^\Box \quad\text{iff}\quad \Sh(A) \models \varphi. \]
\item Let~$\Box\varphi \defequiv \neg\neg\varphi$. Then
\[ \Sh(X) \models \varphi^\Box \quad\text{iff}\quad \Sh(X_{\neg\neg}) \models \varphi. \]
\item Let~$x \in X$ be a point and let~$\Box\varphi \defequiv ((\varphi
\Rightarrow \notat{x}) \Rightarrow \notat{x})$. Then
\[ \Sh(X) \models \varphi^\Box \quad\text{iff}\quad \text{$\varphi$ holds
at~$x$}. \]
% change text below ("discuss the third case") if numbering of the items changes
\end{enumerate}
\end{cor}
\begin{proof}Combine Theorem~\ref{thm:box-translation-semantically} and
Table~\ref{table:nuclei}.\end{proof}

We want to discuss the last case of the corollary in more detail. Let~$x$ be a
point of a topological space~$X$ and let~$\varphi$ be a formula. Let~$\Box$ be
the modal operator given in the corollary. Then~$\varphi$ \emph{holds at~$x$}
if and only if, from the internal perspective of~$\Sh(X)$, the translated
formula~$\varphi^\Box$ holds; and~$\varphi$ \emph{holds on some open
neighbourhood of~$x$} if and only if, from the internal perspective, the
formula~$\Box\varphi$ holds.

Thus the question whether the truth of~$\varphi$ at the point~$x$ spreads to
some open neighbourhood can be formulated in the following way:
\begin{quote}
\emph{Does~$\varphi^\Box$ imply~$\Box\varphi$ in the internal language
of~$\Sh(X)$?}
\end{quote}
Phrased this way, technicalities like appropriately shrinking open
neighbourhoods are blinded out. A purposefully trivial example to illustrate
this is the following. Let~$X$ be a scheme (or ringed space). Let~$f,g \in
\Gamma(X,\O_X)$ be global functions. Suppose that the germs of~$f$ and~$g$ are
zero in some stalk~$\O_{X,x}$; we want to show that they are zero on a common
open neighbourhood of~$x$.

\begin{proof}[Usual proof]Since the germ of~$f$ vanishes in~$\O_{X,x}$, there
is an open neighbourhood~$U_1$ of~$x$ such that~$f|_{U_1} = 0$
in~$\Gamma(U_1,\O_X)$. Since furthermore the germ of~$g$ vanishes in the same stalk,
there exists an open neighbourhood~$U_2$ of~$x$ such that~$g|_{U_2} = 0$. The
intersection of both neighbourhoods is still an open neighbourhood of~$x$; on
this it holds that~$f$ and~$g$ both vanish.
\end{proof}

\begin{proof}[Proof in the internal language]We may suppose that~$(f = 0 \wedge
g = 0)^\Box$, \ie $\Box(f=0) \wedge \Box(g=0)$, and have to prove
that~$\Box(f=0 \wedge g=0)$. (To this end, we could simply invoke the third
axiom on a modal operator, but we want to stay close to the given external
proof.) So by assumption, both~$\Box(f=0)$ and~$\Box(g=0)$ hold. Since our goal
is to prove a boxed statement, we may in fact assume that~$f = 0$ and~$g = 0$.
Thus~$f = 0 \wedge g = 0$.\end{proof}

By using the internal language with its modal operators, we can thus reduce
basic facts of scheme theory which deal with stalks and neighbourhoods to facts
of algebra in a \emph{modal intuitionistic context}. As with using the internal
language in its basic form without modalities, this brings conceptual clarity
and reduced technical overhead. There are however two more distinctive
advantages. Firstly, many internal proofs do not require specific properties of
the modal operator and thus work with any modal operator. By interpreting such
a proof using different operators, one obtains a whole family of external
statements without any additional work (see
Lemma~\ref{lemma:module-zero-point-neighbourhood} for an example).

Secondly, the following corollary gives a general metatheorem which is
applicable to a wide range of cases. It allows to decide whether spreading will
occur (or is likely not to occur) simply by looking at the \emph{logical form}
of the statement is question.

\begin{cor}\label{cor:geometric-spreading}
Let~$X$ be a topological space. Let~$\varphi$ be a formula.
If~$\varphi$ is geometric, truth of~$\varphi$ at a point~$x \in X$ implies
truth of~$\varphi$ on some open neighbourhood of~$x$, and vice versa.\end{cor}
\begin{proof}By the purely logical lemmas of the previous section, it holds
that~$\varphi^\Box \Leftrightarrow \Box\varphi$.
\end{proof}

\begin{cor}
Let~$X$ be a topological space. Let~$\varphi$ be a formula.
If~$\varphi$ is geometric, the property ``$\varphi$ holds at a point~$x \in
X$'' is open.
\end{cor}
\begin{proof}This is just a reformulation of the previous corollary:
If~$\varphi$ holds at a point~$x \in X$, it holds on some open
neighbourhood~$U$ of~$x$ as well. Going back to stalks, it follows
that~$\varphi$ holds at every point of~$U$.\end{proof}

\begin{ex}Let~$X$ be a scheme (or a ringed space). Since the condition for a
function~$f\?\O_X$ to be nilpotent is geometric (it is~$\bigvee_{n \geq 0} f^n
= 0$), nilpotency of~$f$ at a point is equivalent to nilpotency on some open
neighbourhood.\end{ex}

Combined with Lemma~\ref{lemma:stalk-open-with-hypothesis}, this metatheorem is
quite useful. We will illustrate it with many examples in the next subsection.

An important special case of spreading from stalks to neighbourhoods is the
case of spreading from the generic point (should it exist) to a dense open
subset. Whether this occurs can be phrased by
Lemma~\ref{lemma:negneg-generic-point} as follows:
\begin{quote}
\emph{Does~$\varphi^{\neg\neg}$ imply~$\neg\neg\varphi$ in the internal language
of~$\Sh(X)$?}
\end{quote}
This question is a question of ordinary (non-modal) intuitionistic algebra.

\begin{ex}We have seen in Remark~\ref{rem:epi-in-box-sheaves} that a
morphism~$f : A \to B$ in~$\Sh(X_\Box) \simeq \Sh_\Box(\Sh(X))$ is an
epimorphism if and only if~$\Sh(X) \models \forall y\?B\_ \Box(\exists x\?X\_
f(x) = y)$. We can now understand a simple proof of this fact:
\begin{align*}
  &\ \text{$f$ is an epimorphism in~$\Sh_\Box(\Sh(X))$} \\
  \Longleftrightarrow&\
    \Sh_\Box(\Sh(X)) \models \speak{$f$ is surjective} \\
  \Longleftrightarrow&\
    \Sh(X) \models \left(\speak{$f$ is surjective}\right)^\Box \\
  \Longleftrightarrow&\
    \Sh(X) \models \forall y\?B\_ \Box(\exists x\?X\_ f(x) = y).
\end{align*}
\end{ex}

\begin{rem}Theorem~\ref{thm:box-translation-semantically} can also be motivated
by purely logical considerations. Namely, one can check that interpreting a
formula~$\varphi$ by $\Sh(X) \models \varphi^\Box$ gives rise to a model of
intuitionistic logic -- if~$\varphi$ intuitionistically implies~$\psi$,
then~$\Sh(X) \models \varphi^\Box$ implies~$\Sh(X) \models \psi^\Box$. In
categorical logic, it is therefore a natural question whether there exists a
topos~$\E$ such that~$\E \models \varphi$ if and only if~$\Sh(X) \models
\varphi^\Box$. Theorem~\ref{thm:box-translation-semantically} gives an
affirmative answer to this question, explicitly stating that~$\E \defeq
\Sh(X_\Box)$ is such a topos.\end{rem}

\begin{proof}[Proof of Theorem~\ref{thm:box-translation-semantically}]
A fancy proof goes as follows. First, one shows intuitionistically that for a
modal operator~$\Box$ in~$\Set$, it holds that
\[ \Set \models \varphi^\Box \quad\Longleftrightarrow\quad
  \Sh_\Box(\Set) \models \varphi. \]
This can be done by an easy and nontechnical induction on the structure of
formulas~$\varphi$. Then one interprets this result in the sheaf topos~$\Sh(X)$:
\begin{align*}
  &\ \Sh(X) \models \varphi^\Box \\
  \Longleftrightarrow&\
  \Sh(X) \models \speak{$\Set \models \varphi^\Box$} &&\text{by idempotency}\\
  \Longleftrightarrow&\
  \Sh(X) \models \speak{$\Sh_\Box(\Set) \models \varphi$} &&\text{by the first step} \\
  \Longleftrightarrow&\
  \Sh_\Box(\Sh(X)) \models \varphi &&\text{by idempotency} \\
  \Longleftrightarrow&\
  \Sh(X_\Box) \models \varphi &&\text{since~$\Sh_\Box(\Sh(X)) \simeq
  \Sh(X_\Box)$}
\end{align*}
By \emph{idempotency}, we mean that internally employing the Kripke--Joyal
semantics to interpret doubly-internal statements is the same as using the
Kripke--Joyal semantics once. However, we do not want to discuss this here any further;
some details can be found in the original article on the stack
semantics~\cite[Lemma~7.20]{shulman:stack}, but the lemma given there is not
general enough to justify the second use of idempotency above. For this, one
would have to extend the stack semantics to support internal statements about
locally internal categories like~$\Sh(X_\Box) \hookrightarrow \Sh(X)$ (which
then look like locally small categories from the internal point of view). This
is worthwhile for other reasons too, but shall not be pursued in these notes.

Therefore, we give a more explicit proof. By induction, we are going to prove
that for any open subset~$U \subseteq X$ and any formula~$\varphi$ over~$U$, it
holds that
\[ U \models_X \varphi^\Box \quad\Longleftrightarrow\quad j(U) \models_{X_\Box}
\varphi, \]
where the internal statements are to be interpreted by the Kripke--Joyal
semantics of~$X$ and~$X_\Box$ respectively and~$j$ is the nucleus associated
to~$\Box$. We may assume that any sheaves occuring in~$\varphi$ as domains of
quantifications are in fact~$\Box$-sheaves; we justify this with a separate lemma
below.

The cases~$\varphi \equiv \top$,~$\varphi \equiv (\psi \wedge \chi)$,
and~$\varphi \equiv \bigwedge_i \psi_i$ are trivial. For~$\varphi \equiv \bot$,
the claim is that~$U \models_X \Box\bot$ if and only if~$j(U)
\models_{X_\Box} \bot$. The former means~$U \subseteq j(\emptyset)$ and the
latter means~$j(U) \leq \sup \emptyset = j(\emptyset)$, so the claim follows from
the first two axioms on a nucleus.
\end{proof}
% XXX: other cases

\begin{lemma}Let~$\Box$ be a modal operator. Let~$\varphi$ be a formula.
Let~$\psi \defequiv \varphi^\Box$ be the~$\Box$-translation of~$\varphi$.
Let~$\psi'$ be the formula obtained from~$\psi$ by substituting any occuring
domain of quantification by its~$\Box$-sheafification, as syntactically defined
in Definition~\ref{defn:plus-construction}. Then~$\psi$
and~$\psi'$ are intuitionistically equivalent.
\end{lemma}
\begin{proof}
For any formula~$\varphi$, we denote by~``$\varphi^\boxplus$'' the result of
first applying the~$\Box$-translation to~$\varphi$ and then substituting any
set~$F$ occuring in~$\varphi$ as a domain of quantification by the plus
construction~$F^+$. Recall that for any such~$F$ there is a canonical map~$F
\to F^+,\ x \mapsto [\{x\}]$. We are going to show by induction that for any
formula~$\varphi(x_1,\ldots,x_n)$ in which elements~$x_i\?F_i$ may occur as
terms, it holds that~$\varphi^\Box(x_1,\ldots,x_n)$ is equivalent
to~$\varphi^\boxplus([\{x_1\}],\ldots,[\{x_n\}])$. This suffices to prove the
lemma.

The cases for
\[ \top \quad \bot \quad \wedge \quad \bigwedge \quad \vee \quad \bigvee \quad \implies \]
are trivial. The cases for unbounded~``$\forall$'' and~``$\exists$'' are
trivial as well. The case for~``$=$'' is slightly more interesting; let~$\varphi(x,y)
\equiv (x = y)$. Then we are to show that~$\varphi^\Box(x,y) \equiv \Box(x=y)$
(equality in some set~$F$) is equivalent to~$\varphi^\boxplus([\{x\}],[\{y\}])
\equiv \Box([\{x\}] = [\{y\}])$ (equality in~$F^+$). This follows by the
definition of the plus construction. The case for~``$\in$'' is similar.

Let~$\varphi \equiv (\exists x\?F\_ \psi(x))$, where we have dropped further
variables occuring in~$\psi$ for simplicity. Then we are to show
that~$\varphi^\Box \equiv \Box(\exists x\?F\_ \psi^\Box(x))$ is equivalent
to~$\varphi^\boxplus \equiv \Box(\exists \bar x\?F^+\_ \psi^\boxplus(\bar x))$.
The ``only if'' direction is trivial (set~$\bar x \defeq [\{x\}]$). For the ``if''
direction, we may assume that there exists~$\bar x\?F^+$ such
that~$\psi^\boxplus(\bar x)$, since we want to prove a boxed statement. By
definition of the plus construction, it holds that~$\Box(\speak{$\bar x$ is a
singleton})$. So, again since we want to prove a boxed statement, we may assume
that~$\bar x$ is actually a singleton. Therefore there exists~$x\?F$ such
that~$\bar x = [\{x\}]$ and that~$\psi^\boxplus([\{x\})$ holds. By the induction
hypothesis, it follows that~$\psi^\Box(x)$. From this the claim follows.

The case for~``$\forall$'' is similar.
\end{proof}

\begin{ex}\label{ex:translation-equivalence}Let~$X$ be a scheme. Let~$f$ be a
global function on~$X$. Let~$\varphi \defequiv \neg(\speak{$f$ \inv})$
and~$\varphi' \defequiv \speak{$f$ nilpotent}$. Then, by Proposition~\ref{prop:cond-zero}, we
have~$\Sh(X) \models (\varphi \Leftrightarrow \varphi')$. But in general, this
does not imply that~$\Sh(X) \models (\varphi^\Box \Leftrightarrow
(\varphi')^\Box)$. Consider for instance the modal operator given by~$\Box\alpha
\defequiv ((\alpha \Rightarrow \notat{x}) \Rightarrow \notat{x})$ associated to a
point~$x \in X$. Then~$\Sh(X) \models (\varphi^\Box \Leftrightarrow
(\varphi')^\Box)$ means that the equivalence~$\varphi \Leftrightarrow \varphi'$
holds at the point~$x$. This is false for~$X = \Spec \ZZ$,~$f = 2$, and~$x =
(2)$, since in the local ring~$\O_{X,x} = \ZZ_{(2)}$, the element $f$ is not invertible
while also not being nilpotent.
\end{ex}


\subsection{Internal proofs of common lemmas}

\begin{lemma}\label{lemma:module-zero-point-neighbourhood}
Let~$X$ be a scheme (or ringed space). Let~$\F$ be an~$\O_X$-module
of finite type.
\begin{itemize}
\item Let~$x \in X$ be a point. Then the stalk~$\F_x$ is zero if and
only if~$\F$ is zero on some open neighbourhood of~$x$.
\item Let~$A \subseteq X$ be a closed subset. Then the restriction~$\F|_A$ (\ie
the pullback of~$\F$ to~$A$) is zero if and only if~$\F$ is zero on some open
subset of~$X$ containing~$A$.
\end{itemize}
\end{lemma}
\begin{proof}\emph{Both} statements are simply internalizations of
Example~\ref{ex:module-zero-geometric}, using the modal operators~$\Box =
(\placeholder \vee A^c)$ and~$\Box = ((\placeholder \Rightarrow
\notat{x}) \Rightarrow \notat{x})$.
\end{proof}

\begin{rem}Note that the proposition fails if one drops the hypothesis
that~$\F$ is of finite type. Indeed, in this case one cannot reformulate the
condition that~$\F$ is zero in a geometric way.\end{rem}

In a remark after the proof of Proposition~\ref{prop:rank-function-internally},
we promised to present a simpler proof of it once we would have developed the theory for
doing so. We can now follow up on this promise.
\begin{lemma}\label{lemma:gen-family-n}
Let~$X$ be a scheme (or ringed space). Let~$\F$ be an~$\O_X$-module
of finite type. Let~$x \in X$ be a point. Let~$n$ be a natural number. Then the
following statements are equivalent:
\begin{enumerate}
\item There exists a generating family for~$\F_x$ consisting of~$n$ elements.
\item There exists an open neighbourhood~$U$ of~$x$ such that
\[ U \models \speak{there exists a generating family for~$\F$ consisting of~$n$
elements}. \]
\end{enumerate}
\end{lemma}
\begin{proof}Using the modal operator~$\Box$ defined by~$\Box\varphi \defequiv
((\varphi \Rightarrow \notat{x}) \Rightarrow \notat{x})$, we have to show that
the following statements in the internal language are equivalent:
\begin{enumerate}
\item $\speak{there exists a generating family
for~$\F$ consisting of~$n$ elements}^\Box$.
\item $\Box(\speak{there exists a generating family
for~$\F$ consisting of~$n$ elements})$.
\end{enumerate}
By Lemma~\ref{lemma:open-stalk}, the second statement implies the first -- note
that in a formal spelling of the statement in quotes,
\begin{equation}\label{eqn:finitely-generated}
  \exists x_1,\ldots,x_n\?\F\_
  \forall x\?\F\_
  \exists a_1,\ldots,a_n\?\O_X\_
  x = \textstyle\sum_i a_i x_i,
\end{equation}
no implication signs occur. To show the converse direction,
we may assume that there is a generating family~$y_1,\ldots,y_m\?\F$ for~$\F$
(since~$\F$ is, externally speaking, of finite type). Then
the~$\Box$-translation of the statement that the~$y_i$ generate~$\F$ holds as
well (again by Lemma~\ref{lemma:open-stalk}). Since there is an intuitionistic
proof of
\begin{multline*}
  \speak{$y_1,\ldots,y_m$ generate~$\F$} \Longrightarrow \\
  \bigl(\speak{there exist $x_1,\ldots,x_n\?\F$ which generate~$\F$}
    \Longleftrightarrow \\
    \exists x_1,\ldots,x_n\?\F\_
    \exists A\?\O^{m \times n}\_ \speak{$\vec y = A \vec x$}\bigr),
\end{multline*}
we can substitute the non-geometric formula~\eqref{eqn:finitely-generated} by the geometric
formula
\[ \exists x_1,\ldots,x_n\?\F\_ \exists A\?\O^{m \times n}\_ \speak{$\vec
y = A \vec x$} \]
(Lemma~\ref{lemma:stalk-open-with-hypothesis}). Thus the claim follows.
\end{proof}

\begin{lemma}Let~$X$ be a scheme (or ringed space). Let~$\alpha : \F \to \G$ be
a morphism of~$\O_X$-modules. Let~$\G$ be of finite type and assume
that~$\alpha_x : \F_x \to \G_x$ is surjective for some point~$x \in X$.
Then~$\alpha$ is an epimorphism on some open neighbourhood of~$x$.\end{lemma}
\begin{proof}In the presence of generators~$y_1,\ldots,y_n\?\G$, the
non-geometric surjectivity condition ($\forall y\?\G\_ \exists x\?\F\_
\alpha(x) = y$) can be reformulated in a geometric way: $\bigwedge_{i=1}^n
\exists x\?\F\_ \alpha(x) = y_i$. Thus the claim follows by
Lemma~\ref{lemma:stalk-open-with-hypothesis}.\end{proof}

%\begin{lemma}Let~$X$ be a scheme (or ringed space). Let~$\alpha : \F \to \G$ be
%a morphism of~$\O_X$-modules. Let~$\F$ be of finite type and~$\G$ be coherent.
%Suppose that~$\alpha_x$ is injective at some point~$x \in X$. Then~$\alpha$ is
%a monomorphism on some open neighbourhood of~$x$.
%\end{lemma}
%\begin{proof}The kernel of~$\alpha$ is of finite type (by
%Lemma~\ref{lemma:coherent-stuff}) and zero at~$x$. By the previous lemma, it is
%therefore zero on some open neighbourhood of~$x$.
%\end{proof}
%This proof is precisely the classical one, so there is no need to include it here.

\begin{lemma}\label{lemma:pushforward-finite-type}
Let~$i : A \hookrightarrow X$ be a closed immersion of schemes (or
ringed spaces). Let~$\F$ be an~$\O_A$-module. Then~$i_*\F$ is of finite type if
and only if~$\F$ is of finite type.\end{lemma}
\begin{proof}
Let~$\Box$ be the modal operator defined by~$\Box\varphi \defequiv (\varphi \vee
A^c)$. From the internal perspective, we have a surjective ring homomorphism~$i^\sharp
: \O_X \to \O_A$, where we omit the forgetful functor~$i_*$ from~$\Box$-sheaves
to arbitrary sets in the notation, and an~$\O_A$-module~$\F$. Furthermore, we
may assume that~$\F$ is a~$\Box$-sheaf. We can regard~$\F$ as an~$\O_X$-module
by~$i^\sharp$.

Note that~$A^c \Rightarrow (\F = 0)$, by~$\Box$-separatedness of~$\F$.

We are to show that~$\F$ is a finitely generated~$\O_X$-module if and only if
the~$\Box$-translation of ``$\F$ is a finitely generated~$\O_A$-module'' holds.
In explicit terms, we have to show the equivalence of the following statements:
\begin{enumerate}
\item $\bigvee_{n \geq 0} \exists x_1,\ldots,x_n\?\F\_
  \forall x\?\F\_ \exists a_1,\ldots,a_n\?\O_X\_ x = \sum_i i^\sharp(a_i) x_i$.
\item $\Box(\bigvee_{n \geq 0} \Box(\exists x_1,\ldots,x_n\?\F\_
  \forall x\?\F\_ \Box(\exists b_1,\ldots,b_n\?\O_A\_ \Box(
    x = \sum_i b_i x_i))))$.
\end{enumerate}
It is clear that the first statement implies the second. For the converse
direction, we just have to repeatedly use the observation that~$\Box\varphi$
implies~$\varphi \vee (\F = 0)$ (once for each occurence of~$\Box$). So in each
step, we either obtain the statement we want or may assume
that~$\F$ is the trivial module, in which case any subclaim trivially follows. By
surjectivity of~$i^\sharp$, we may write any~$b\?\O_A$ as~$b =
i^\sharp(a)$ for some~$a\?\O_X$.
\end{proof}

\begin{lemma}Let~$X$ be a scheme (or a ringed space). Let~$\F$ and~$\G$ be~$\O_X$-modules. Let~$x
\in X$. Then $\HOM_{\O_X}(\F,\G)_x \cong \Hom_{\O_{X,x}}(\F_x,\G_x)$ if~$\F$ is
of finite presentation around~$x$.\end{lemma}
\begin{proof}It suffices to give an intuitionistic proof of the following fact:
The construction~$\Hom_R(M,\placeholder)$ is geometric if~$M$ is a finitely
presented~$R$-module. So assume that~$M$ is the cokernel of a presentation
matrix~$(a_{ij}) \? R^{n \times m}$. Then we can calculate the Hom with
any~$R$-module~$N$ as
\[ \Hom_R(M,N) \cong \Bigl\{ x \? N^n \ \Big|\ \bigwedge_{j=1}^m \sum_{i=1}^n a_{ij}
x_i = 0 \? N \Bigr\}, \]
and this construction is patently geometric, as a set comprehension with respect to
a geometric formula.
\end{proof}

\begin{lemma}Let~$X$ be a scheme (or a ringed space). Let~$\F$ be an~$\O_X$-module of finite
presentation. Let~$x \in X$. Then the stalk~$\F_x$ is a finite
free~$\O_{X,x}$-module if and only if~$\F$ is finite locally free on some open
neighbourhood of~$x$.\end{lemma}
\begin{proof}The internal statement that~$\F$ is a finite free module is not geometric:
\[ \bigvee_{n \geq 0}
  \exists x_1,\ldots,x_n\?\F\_
  \forall x\?\F\_
  \exists! a_1,\ldots,a_n\?\O_X\_
  x = \textstyle\sum_i a_i x_i. \]
But it can equivalently be reformulated as
\[ \bigvee_{n \geq 0}
  \exists \alpha\?\HOM_{\O_X}(\F,\O_X^n)\_
  \exists \beta\?\HOM_{\O_X}(\O_X^n,\F)\_
  \alpha \circ \beta = \id \wedge \beta \circ \alpha = \id. \]
This reformulation is geometric, therefore it holds at~$x$ if and only if it
holds on some open neighbourhood of~$x$. The claim follows since, by the
previous proposition, taking stalks commutes with
calculating~$\HOM_{\O_X}(\F,\placeholder)$ \resp~$\HOM_{\O_X}(\O_X^n,\placeholder)$;
thus the pulled back formula indeed expresses that~$\F_x$ is finite free as
an~$\O_{X,x}$-module.
\end{proof}

\begin{lemma}\label{lemma:torsion-module-generic-stalk}
Let~$X$ be an integral scheme with generic point~$\xi$. Let~$\F$
be a quasicoherent~$\O_X$-module. Then~$\F$ is a torsion module if and only if
its generic stalk~$\F_\xi$ vanishes.
\end{lemma}
\begin{proof}The generic stalk vanishes if and only if the internal
statement~``$(\F = 0)^{\neg\neg}$'' holds. Therefore it suffices to give an
intuitionistic proof of the following internal statement: The module~$\F$ is
torsion if and only if any element of~$\F$ is \notnot zero.

For the ``only if'' direction, let~$x\?\F$ be an arbitrary element. Since~$\F$
is a torsion module, there exists a regular element~$a\?\O_X$ such that~$ax =
0$. Since~$X$ is reduced, regularity is equivalent to not-not-invertibility.
Since we want to verify the~$\neg\neg$-stable statement~``$\neg\neg(x = 0)$'', we
may in fact assume that~$a$ is invertible. Then~$x = 0$ obviously follows.

For the ``if'' direction, let~$x\?\F$ be an arbitrary element; by assumption,~$x$
is \notnot zero. Since~$X$ is integral,
Lemma~\ref{lemma:dense-standard-reflection} is applicable. Therefore there
exists an element~$a\?\O_X$ such that~$a$ is \notnot invertible and such that
invertibility of~$a$ implies~$x = 0$. Since~$\F$ is quasicoherent, for some
natural number~$n$ it holds that~$a^n x = 0$ (Theorem~\ref{thm:qcoh-sheafchar}
below). Since~$a$ is \notnot invertible,
it is regular (see Lemma~\ref{lemma:regular-notnot-invertible} below for a short
and self-contained proof), and therefore~$a^n$ is regular. So~$x \in \F_\tors$.
\end{proof}

By simply using a different modal operator than~``\notnot'', we will -- without
any additional work -- obtain a more general form of this lemma, applicable to
non-integral schemes (see Lemma~\ref{lemma:torsion-module-generic-stalk-generalized}).

\begin{itemize}
\item general explanation of modalities (as for instance in philosophy)
\item explain that for some modal operators, the~$\Box$-translation of the law
of excluded middle is valid; explain consequences
\item spreading of properties from stalk to neighbourhood: give many examples
\item give proof of the expressions for the nuclei listed in the table
\end{itemize}


\section{Rational functions and Cartier divisors}
\label{sect:rational-functions}

\subsection{The sheaf of rational functions} Recall that the sheaf~$\K_X$ of rational
functions on a scheme~$X$ (or ringed space) can be defined as the sheaf
associated to the presheaf
\[ \text{$U \subseteq X$ open} \quad\longmapsto\quad \Gamma(U,\O_X)[\Gamma(U,\S)^{-1}], \]
where~$\Gamma(U,\S)$ is the multiplicative set of those sections of~$\O_X$ on~$U$,
which are regular in each stalk~$\O_{X,x}$, $x \in U$. Recall also that there are
some wrong definitions in the literature~\cite{kleiman:misconceptions}.

Using the internal language, we can give a simpler definition of~$\K_X$.
Recall that we can associate to any ring~$R$ its total quotient ring, \ie
its localization at the multiplicative subset of regular elements. Since from
the internal perspective~$\O_X$ is an ordinary ring, we can associate to it its
total quotient ring $\O_X[\S^{-1}]$,
where~$\S$ is internally defined by the formula
\[ \S \defeq \{ s\?\O_X \,|\, \speak{$s$ is regular} \} \subseteq \O_X. \]
Externally, this ring is the sheaf~$\K_X$.
\begin{prop}\label{prop:kx-internally}
Let~$X$ be a scheme (or a ringed space). The sheaf of rings defined
in the internal language by localizing~$\O_X$ at its set of regular elements is
(canonically isomorphic to) the sheaf~$\K_X$ of rational functions.
\end{prop}
\begin{proof}Internally, the ring~$\O_X[\S^{-1}]$ has the following
universal property: For any ring~$R$ and any homomorphism~$\O_X \to R$ which
maps the elements of~$\S$ to units, there exists exactly one
homomorphism~$\O_X[\S^{-1}] \to R$ which renders the evident diagram commutative.
\[ \xymatrix{
  \O_X \ar[rr] \ar[dr] && R \\
  & \O_X[\S^{-1}] \ar@{-->}[ru]
} \]
The translation using the Kripke--Joyal semantics gives the following universal
property: For any open subset~$U \subseteq X$, any sheaf of rings~$\R$ on~$U$ and any
homomorphism~$\O_X|_U \to \R$ which maps all elements of~$\Gamma(V,\S)$, $V
\subseteq U$ to units, there exists exactly one homomorphism~$\O_X[\S^{-1}]|_U \to
\R$ which renders the evident diagram commutative.
It is well-known that the sheaf~$\K_X$ as usually defined has
this universal property as well.
\end{proof}

\begin{prop}\label{prop:stalks-kx}
Let~$X$ be a scheme (or ringed space). Then the stalks of~$\K_X$
are given by
\[ \K_{X,x} = \O_{X,x}[\S_x^{-1}]. \]
The elements of~$\S_x$ are exactly the germs of those local sections which are
regular not only in~$\O_{X,x}$, but in all rings~$\O_{X,y}$ where~$y$
ranges over some open neighbourhood of~$x$ (depending on the section).\end{prop}
\begin{proof}
Since localization is a geometric construction, the first statement is made entirely
trivial by our framework. The second statement follows since
\[ \Gamma(U,\S) = \{ s\in\Gamma(U,\O_X) \,|\, U \models \speak{$s$ is regular}
\} \]
and regularity is a geometric implication, so that
$U \models \speak{$s$ is regular}$ if and only if the germ~$s_y$ is regular
in~$\O_{X,y}$ for all~$y \in U$.
\end{proof}

\begin{rem}Speaking internally, the multiplicative set~$\S$ is saturated.
Therefore an element~$s/t \? \K_X$ is invertible in~$\K_X$ if and only if the
numerator~$s$ belongs to~$\S$, \ie is an regular element of~$\O_X$.\end{rem}

% FUTURE:
% How can we prove that K_X(U) = K_X^pre(U) = Quot O_X(U) for U affine
% (and X locally Noetherian or X reduced and locally only finitely many
% irreducible components)? (cf. Kleiman!)
%
% Here is approximately how. Verify that (U |-> K_X^pre(U)) defines a sheaf, if
% U ranges only of the open affines. To do this, use the description of the
% subsheaf \T given in the section on "O_X = A[F^(-1)]".


\subsection{Regularity of local functions}
It is well-known that on a locally Noetherian scheme, regularity spreads from
stalks to neighbourhoods, \ie a section of~$\O_X$ is regular
in~$\O_{X,x}$ if and only if it is regular on some open neighbourhood of~$x$.
This fact has a simple proof in the internal language.
\begin{prop}\label{prop:regularity-spreading}
Let~$X$ be a locally Noetherian scheme. Let~$s \in \Gamma(U,\O_X)$
be a local function on~$X$. Let~$x \in U$. Then the following statements are
equivalent:
\begin{enumerate}
\item The section~$s$ is regular in~$\O_{X,x}$.
\item The section~$s$ is regular in all local rings~$\O_{X,y}$ where~$y$ ranges
over some open neighbourhood of~$x$.
\end{enumerate}
\end{prop}
\begin{proof}
Let~$\Box$ be the modal operator defined by~$\Box\varphi \defequiv ((\varphi
\Rightarrow {!x}) \Rightarrow {!x})$. By Corollary~\ref{cor:spreading}, we are
to show that the following statements of the internal language are equivalent:
\begin{enumerate}
\item $(\speak{$s$ is regular})^\Box$, \ie
$\forall t\?\O_X\_ st = 0 \Rightarrow \Box(t = 0)$.
\item $\Box(\speak{$s$ is regular})$, \ie
$\Box(\forall t\?\O_X\_ st = 0 \Rightarrow t = 0)$.
\end{enumerate}
It is clear that the second statement implies the first -- in fact, this is true
without any assumptions on~$X$: Let~$t\?\O_X$ be such that~$st = 0$. Since we want to
prove the boxed statement~$\Box(t=0)$, we may assume that~$s$ is regular and
prove~$t = 0$. This is immediate. (This direction also follows simply by
examining the logical form and applying Lemma~\ref{lemma:open-stalk}.)

For the converse direction, consider the annihilator of~$s$, \ie the ideal
\[ I \defeq \Ann_{\O_X}(s) = \{ t\?\O_X \,|\, st = 0 \} \subseteq \O_X. \]
This ideal satisfies the quasicoherence condition (Example~\ref{ex:annihilator-qcoh}),
thus~$I$ is a quasicoherent submodule of a finitely generated module. Since~$X$ is
locally Noetherian, it follows that~$I$ is finitely generated as well, say by~$x_1,\ldots,x_n \? I$. By
assumption, each generator~$x_i \? I$ fulfills~$\Box(x_i = 0)$. Since we want
to prove a boxed statement, we may in fact assume~$x_i = 0$. Thus~$I = (0)$ and
the assertion that~$s$ is regular follows.
\end{proof}

Note that the proof critically depends on the ideal~$I$ being finitely
generated, since a modal operator need only commute with finite
conjuctions. Intuitively, each time we use the modus ponens rule~$\Box\varphi \wedge
(\varphi \Rightarrow \psi) \Rightarrow \Box\psi$, we restrict to a smaller open
neighbourhood of~$x$. Since infinite intersections of open sets need not be
open, we cannot expect an infinitary modus ponens rule to hold.

\begin{cor}Let~$X$ be a locally Noetherian scheme. Then the stalks~$\K_{X,x}$
of the sheaf of rational functions are given by the total quotient rings of the
local rings~$\O_{X,x}$.\end{cor}
\begin{proof}Combine Proposition~\ref{prop:stalks-kx} and
Proposition~\ref{prop:regularity-spreading}.\end{proof}


\subsection{Normality}\label{sect:normality}
Recall that a ring~$R$ is \emph{normal} if and only if
it is integrally closed in its total quotient ring. Recall also that a
scheme~$X$ (or ringed space) is \emph{normal} if and only if all
rings~$\O_{X,x}$ are normal.

\begin{prop}\label{prop:normal-int-ext}A locally Noetherian scheme is normal if and only if the
ring~$\O_X$ is normal from the internal perspective.\end{prop}
\begin{proof}The condition of normality can be put into a form which is almost
a geometric implication:
\begin{multline*}
  \forall s,t\?\O_X\_
  \Bigl(\speak{$t$ regular} \wedge
  (\exists a_0,\ldots,a_{n-1}\?\O_X\_
  s^n + a_{n-1} t s^{n-1} + \cdots + a_1 t^{n-1} s + a_0 t^n = 0)
  \Longrightarrow \\
  \exists u\?\O_X\_ s = ut\Bigr).
\end{multline*}
The only non-geometric subpart is the condition on~$t$ to be regular. However,
by Proposition~\ref{prop:regularity-spreading}, for the purposes of comparing
its truth at points \vs on neighbourhoods, it behaves just like a geometric
formula. Therefore the claim follows.
\end{proof}


\subsection{Geometric interpretation of rational functions} Recall that on
integral schemes, rational functions (\ie sections of~$\K_X$) are the same
thing as regular functions defined on dense open subsets. This amounts to
saying that~\emph{$\K_X$ is the~$\neg\neg$-sheafification of~$\O_X$}
(see Proposition~\ref{prop:negneg-sheaves}). We want to rederive this result,
as far as possible in the internal language, and generalize it to arbitrary
(not necessarily locally Noetherian) schemes.

\begin{lemma}\label{lemma:regular-notnot-invertible}Let~$X$ be a reduced scheme. Then:
\begin{enumerate}
\item $\O_X$ is~$\neg\neg$-separated.
\item Internally, an element~$s\?\O_X$ is regular
if and only if it is \notnot invertible.
\end{enumerate}
\end{lemma}
\begin{proof}Recall from Corollary~\ref{cor:field-reduced} that
\begin{equation}\label{eqn:field-condition}
  \Sh(X) \models \forall s\?\O_X\_ \neg(\speak{$s$ invertible}) \Leftrightarrow
  s=0.
\end{equation}
From this we can deduce that~$\O_X$ is~$\neg\neg$-separated:
Assume~$\neg\neg(s=0)$ for~$s\?\O_X$. If~$s$ were invertible, we would
have~$\neg\neg(1=0)$ and thus~$\bot$. Therefore~$s$ is not invertible and thus
zero.

For the ``only if'' direction of the second statement,
note that a regular element is not zero (if it were, then the true statement~$0
\cdot 0 = 0 \cdot 1$ would imply the false statement~$0 = 1$) and thus \notnot
invertible (by the contrapositive of equivalence~\eqref{eqn:field-condition}). For the ``if''
direction, let~$st = 0$ in~$\O_X$. Since~$s$ is \notnot invertible, it follows
that~$t$ is \notnot zero. Since~$\O_X$ is~$\neg\neg$-separated, this implies
that~$t$ really is zero.
\end{proof}

For the following, we need two technical conditions. Say that an affine
scheme~$\Spec A$ has property~$(\star)$ if and only if:
\begin{quote}
Every open dense subset~$U \subseteq \Spec A$ contains a
\emph{standard open} dense subset.
\end{quote}
Say that~$\Spec A$ has property~$(\star\star)$ if and only if:
\begin{quote}
Every open scheme-theoretically dense subset~$U \subseteq \Spec A$ contains a
\emph{standard open} scheme-theoretically dense subset.
\end{quote}
The first condition is satisfied if~$A$ is an irreducible ring (\ie if~$\Spec A$
is irreducible) or more generally if~$A$ contains only finitely many minimal
prime ideals. Both conditions are satisfied if~$A$ is integral or if~$A$ is
Noetherian; for convenience, we give a proof in the
latter case.

\begin{prop}Let~$A$ be a Noetherian ring. Then~$\Spec A$ has properties~$(\star)$
and~$(\star\star)$.
\end{prop}
\begin{proof}Recall that, under the Noetherian hypothesis, an open subset of~$\Spec A$ is dense if and only if it
contains all minimal prime ideals and that it
is scheme-theoretically dense if and only if it contains all associated prime
ideals. There are only a finite number of these prime ideals. Therefore the
claim is reduced to the following statement:

Let~$\ppp_1,\ldots,\ppp_n$ be a
finite number of points of an open subset~$U \subseteq \Spec A$. Then there
exists a standard open subset~$D(f) \subseteq U$ which also contains these
points.

The proof of this statement is an easy application of the prime
avoidance lemma.
\end{proof}

\begin{prop}
\label{prop:kx-is-negneg-sheafification}
Let~$X$ be a reduced scheme. Assume that~$X$ can be covered by open affine
subsets which have property~$(\star)$. For instance, this condition is satisfied
if~$X$ is integral, the set of irreducible components is locally finite, or if~$X$ is locally Noetherian. Then~$\K_X$ is
the~$\neg\neg$-sheafification of~$\O_X$.\end{prop}
\begin{proof}
We first show that~$\K_X$ is~$\neg\neg$-separated,
so assume~$\neg\neg(a/s = 0)$ for~$a/s \? \K_X$. Since~$\K_X$ is obtained
from~$\O_X$ by localizing at regular elements, the fraction~$a/s$ vanishes
in~$\K_X$ if and only if~$a = 0$ in~$\O_X$. Thus it follows that~$\neg\neg(a =
0)$ in~$\O_X$ and therefore~$a = 0$ in~$\O_X$; in particular, $a/s = 0$ in~$\K_X$.

We defer the proof that~$\K_X$ is a~$\neg\neg$-sheaf to the end and first
verify the universal property of~$\neg\neg$-sheafification.
So let~$G$ be a~$\neg\neg$-sheaf and let~$\alpha : \O_X \to G$ be a map. We
can define an extension~$\bar\alpha : \K_X \to G$ in the following way:
Let~$f \? \K_X$. Define the subsingleton~$S \defeq \{ x \? G \,|\, \exists
b\?\O_X\_ f = b/1 \wedge x = \alpha(b) \} \subseteq G$. Since~$f$ can be
written in the form~$a/s$ with~$s$ \notnot invertible, it follows that~$S$
is \notnot inhabited. Since~$G$ is a~$\neg\neg$-sheaf, there exists a
unique~$x\?G$ such that~$\neg\neg(x \in S)$. We declare~$\bar\alpha(f)$ to be
this~$x$. It is straightforward to check that the composition~$\O_X \to \K_X
\to G$ equals~$\alpha$ and that~$\bar\alpha$ is unique with this property.

Up to this point, the proof did not need that~$X$ is a scheme -- it was enough
for~$X$ to be a ringed space such that equivalence~\eqref{eqn:field-condition} holds and
such that~$\neg(0 = 1)$ in~$\O_X$. Only now, in showing that~$\K_X$ is
a~$\neg\neg$-sheaf, the scheme condition enters. To this end, we first
reformulate the sheaf condition in a way such that it only refers to~$\O_X$,
not~$\K_X$: The quotient ring~$\K_X$ is a~$\neg\neg$-sheaf if and only if
\begin{multline*}
  \Sh(X) \models \forall T \subseteq \O_X\_
  \speak{$T$ subsingleton} \wedge \neg\neg(\speak{$T$ inhabited})
  \Longrightarrow \\
  \exists a,b\?\O_X\_ \speak{$b$ regular} \wedge \neg\neg(b^{-1} a \in T).
\end{multline*}
This is done just as in the proof of Theorem~\ref{thm:qcoh-sheafchar}. Note
% XXX: reorder qcoh before this because of the reference?
that~``$b^{-1}$'' refers to the inverse of~$b$ which indeed exists in a doubly
negated context, since~$b$ is assumed regular. More explicitly, we should write
\[ \neg\neg(\exists c\?\O_X\_ bc = 1 \wedge ca \in T)
  \quad\text{instead of}\quad
  \neg\neg(b^{-1} a \in T). \]
To verify the Kripke--Joyal interpretation of the rewritten sheaf condition, let
an affine open subset~$U = \Spec A \subseteq X$ (having property~$(\star)$) and a subsheaf~$T
\hookrightarrow \O_X|_U$ be given such that~$T$ is internally a subsingleton
and \notnot inhabited. We may glue the unique germs in the inhabited
stalks of~$T$ to obtain a section~$s \in \Gamma(V,\O_X)$ where~$V \subseteq U$
is a dense open subset. Since~$U$ has property~$(\star)$, we may assume that~$V
= D(f)$ is a standard open subset. Because~$V$ is dense and~$A$ is reduced, the
function~$f$ is a regular element of~$A$.
Since~$\Gamma(V,\O_X) = A[f^{-1}]$, we can write~$s = a/f^n$ with~$a \in A$
and~$n \geq 0$.

By Lemma~\ref{lemma:regular-affine}, the function~$b \defeq f^n$ is also regular as an
element of~$\O_U$ from the internal point of view. Note that~$b$ is invertible
on~$V$, since~$V = D(f) \subseteq D(b)$. It follows that on the dense
open subset~$V \subseteq U$, the sections~$s$ and~$b^{-1} a$ agree.
This observation concludes the proof.
\end{proof}

\begin{cor}Let~$X$ be a reduced scheme admitting a cover by affine open subschemes
with property~$(\star)$. Then~$\K_X$ is the result of
pulling back~$\O_X$ to the sublocale~$X_{\neg\neg}$ and then pushing forward
again. If~$X$ is irreducible with generic point~$\xi$, then~$\K_X$ is the
constant sheaf associated to the set~$\O_{X,\xi}$.\end{cor}
\begin{proof}Recall from Section~\ref{sect:internal-sheaves} that pulling back
to~$X_{\neg\neg}$ is equivalent to sheafifying with respect to the double
negation modality; and that pushing forward is equivalent to forgetting the
sheaf property. Therefore the first statement holds.

For the second statement, recall from Lemma~\ref{lemma:negneg-generic-point} that the
sublocale~$X_{\neg\neg}$ is given by the subspace~$\{\xi\}$; that the
sheafification functor~$\Sh(X) \to \Sh(\{\xi\}) \simeq \Set$ is given by
calculating the stalk at~$\xi$; and that the inclusion functor~$\Set \simeq
\Sh(\{\xi\}) \hookrightarrow \Sh(X)$ is given by the constant sheaf
construction.
\end{proof}

If~$X$ is a general scheme (not necessarily reduced),
we can describe~$\K_X$ in a similar way as a sheafification
of~$\O_X$; specifically, it is the sheafification with respect to the modal
operator defined by
\[ \sdense\varphi \defequiv \speak{$\O_X$ is~$(\varphi \Rightarrow
\placeholder)$-separated} \]
in the internal language of~$\Sh(X)$, \ie
\[ \sdense\varphi \defequiv \forall s\?\O_X\_ (\varphi \Rightarrow s = 0)
\Rightarrow s = 0. \]
This modal operator has a simple scheme-theoretical description.

\begin{lemma}\label{lemma:scheme-theoretical-density}
Let~$U$ be an open subset of a scheme~$X$. Then~$\Sh(X) \models
\sdense U$ if and only if~$U$ is scheme-theoretically dense in~$X$.
\end{lemma}
\begin{proof}We have the following chain of equivalences.
\begin{align*}
  &\ X \models \sdense U \\
  \Longleftrightarrow&\
    \speak{$\O_X$ is~$(U \Rightarrow \placeholder)$-separated} \\
  \Longleftrightarrow&\
    X \models \speak{$\O_X \to \O_X^{+}$ is injective} \\
  &\qquad\qquad\text{(where the plus construction is wrt.\@ the modality~$(U \Rightarrow \placeholder)$)} \\
  \Longleftrightarrow&\
    X \models \speak{$\O_X \to \O_X^{++}$ is injective} \\
  &\qquad\qquad\text{(by the factorization~$\O_X \to \O_X^{+} \to \O_X^{++}$)} \\
  \Longleftrightarrow&\
    \text{the canonical morphism~$\O_X \to j_* \O_U$
    (with $j : U \hookrightarrow X$) is injective} \\
  \Longleftrightarrow&\
    \text{$U$ is scheme-theoretically dense in~$X$.} \qedhere
\end{align*}
\end{proof}

Using the internal language of a scheme, talking about scheme-theoretically
dense open subsets is therefore just as easy as talking about ordinary
topologically dense open subsets; the difference simply amounts to using the
modal operator~``$\sdense$'' instead of~``\notnot''.

\begin{prop}\label{prop:kx-is-box-sheafification}
Let~$X$ be a ringed space. Then:
\begin{enumerate}
\item The operator~$\sdense$ fulfills the axioms on a modal operator.
\item $\O_X$ is~$\sdense$-separated.
\item $\K_X$ is~$\sdense$-separated.
\item Internally, it holds that~$\sdense(\speak{$f$ \inv})$ implies that~$f$ is
regular for any~$f\?\O_X$.
% update addtocounter below if numbering changes
\end{enumerate}
Suppose furthermore that~$X$ is a scheme. Then:
\begin{enumerate}
\addtocounter{enumi}{4}
\item The converse in~(4) holds.
\item If~$X$ can be covered by open affine subschemes with
property~$(\star\star)$, then $\K_X$ is the~$\sdense$-sheafification of~$\O_X$.
\end{enumerate}
\end{prop}
\begin{proof}The first four properties are entirely formal; we thus skip over
some details. For the first property, we verify the second axiom on a modal
operator. So we assume~$\sdense\sdense\varphi$ and have to show~$\sdense\varphi$. To
this end, let~$s\?\O_X$ be arbitrary such that~$\varphi \Rightarrow (s=0)$; we
have to prove that~$s = 0$. If~$\O_X$ were separated with respect to the modal
operator~$(\varphi \Rightarrow \placeholder)$, it would follow that~$s = 0$. So
unconditionally it holds that~$\sdense\varphi \Rightarrow (s=0)$. Since by
assumption~$\O_X$ is~$(\sdense\varphi \Rightarrow \placeholder)$-separated, the claim follows.

For the second property, let~$s\?\O_X$ be arbitrary such that~$\sdense(s = 0)$.
Obviously it holds that~$(s = 0) \Rightarrow (s = 0)$. Thus, since~$\O_X$ is
separated with respect to~$((s = 0) \Rightarrow \placeholder)$, it follows
that~$s = 0$. The proof of the third property is similar.

For the fourth property, assume~$\sdense(\speak{$f$ \inv})$ and let~$h\?\O_X$ be
arbitrary such that~$fh = 0$. Then, trivially, it holds that~$\speak{$f$ \inv}
\Rightarrow h = 0$. Since~$\O_X$ is separated with respect to~$(\speak{$f$
\inv} \Rightarrow \placeholder)$, it follows that~$h = 0$.

We may now suppose that~$X$ is a scheme. To verify the fifth property, let a
regular element~$f\?\O_X$ be given. We have to show that~$\O_X$ is separated
with respect to the modality~$(\speak{$f$ \inv} \Rightarrow \placeholder)$. So
assume that~$\speak{$f$ \inv} \Rightarrow (s = 0)$ for some~$s\?\O_X$. By
Proposition~\ref{prop:cond-zero} it follows that~$f^n s = 0$ for some natural
number~$n$. Since~$f$ is regular, we may conclude that~$s = 0$.

The verification of the universal property of~$\K_X$ is done analogously as in
the case that~$X$ is reduced: For the proof of
Proposition~\ref{prop:kx-is-negneg-sheafification}, it was critical that
regular elements of~$\O_X$ are \notnot invertible. We now need (and have) that
regular elements of~$\O_X$ are~$\sdense(\speak{invertible})$.

Thus it only remains to verify that~$\K_X$ is a~$\sdense$-sheaf. We may again imitate
the proof of Proposition~\ref{prop:kx-is-negneg-sheafification}; using the same
notation, we may now suppose that~$V$ is a standard open subset such that~$U \models \sdense
V$ (previously, we supposed that~$U \models \neg\neg V$). The proof that the
denominator~$b$ is regular (as seen from the internal perspective, as an
element of~$\O_U$) now goes as follows: We have~$V \subseteq
D(b)$. Therefore~$U \models \sdense V$ implies~$U \models \sdense(\speak{$b$ \inv})$. By
the fourth property, it follows that~$U \models \speak{$b$ is regular}$.
\end{proof}

\begin{rem}The modal operator~$\sdense$ is the largest (weakest) operator such
that~$\O_X$ is~$\sdense$-separated, \ie if~$\sdenseother$ is any modal operator
such that~$\O_X$ is~$\sdenseother$-separated, then~$\sdenseother\varphi
\Rightarrow \sdense\varphi$ for any proposition~$\varphi$.\end{rem}

In the special case that~$X$ is a reduced scheme,
Proposition~\ref{prop:kx-is-box-sheafification} recovers
the result of Proposition~\ref{prop:kx-is-negneg-sheafification}:

\begin{prop}Let~$X$ be a scheme. Then~$\sdense\varphi \Rightarrow \neg\neg\varphi$
for any formula~$\varphi$. The converse holds if~$X$ is reduced, so that in
this case the modal operator~$\sdense$ coincides with the double negation modality.\end{prop}
\begin{proof}Let~$\varphi$ be an arbitrary formula and assume~$\sdense\varphi$. Note that~$\neg\varphi$ is
equivalent to~$\varphi \Rightarrow (1 =
0)$. Since by assumption~$\O_X$ is separated with respect to the~$(\varphi
\Rightarrow \placeholder)$-modality, this in turn is equivalent to~$1 = 0 \?
\O_X$, \ie to~$\bot$. Thus~$\neg\neg\varphi$.

For the converse direction, let~$\varphi \Rightarrow (s = 0)$ for some~$s\?\O_X$;
we have to show that in fact~$s = 0$. Since by assumption~$\neg\neg\varphi$, it
follows that~$s$ is \notnot zero. Since~$X$ is reduced,~$\O_X$
is~$\neg\neg$-separated, so this implies that~$s$ is really zero.
\end{proof}

As a corollary, we can reprove the following basic lemma about
scheme-theoretical density.
\begin{lemma}Let~$U$ be an open subset of a scheme~$X$. If~$U$ is
scheme-theoretically dense, then~$U$ is also dense in the plain topological
sense. The converse holds if~$X$ is reduced.\end{lemma}
\begin{proof}The set~$U$ is scheme-theoretically dense if and only if~$\Sh(X)
\models \sdense U$ and is dense if and only if~$\Sh(X) \models \neg\neg U$.
Therefore the claim follows from the previous proposition.
\end{proof}

\begin{prop}\label{prop:kx-ass}
Let~$X$ be a scheme admitting a cover of open affine subsets with
property~$(\star\star)$. Then~$\K_X$ is the result of
pulling back~$\O_X$ to the sublocale~$X_\sdense$ associated to the modal
operator~$\sdense$ and then pushing forward again. If~$X$ is locally Noetherian,
this sublocale is the subspace of associated points in~$X$.
\end{prop}

In formulas, the proposition says that the canonical map
\[ \K_X \longrightarrow i_* i^{-1} \O_X \]
is an isomorphism, where~$i : X_\sdense \hookrightarrow X$ is the inclusion of
the sublocale~$X_\sdense$. This result requires a cover with
property~$(\star\star)$, but no Noetherian hypothesis.

\begin{proof}The first statement follows trivially by the results of
Section~\ref{sect:internal-sheaves} and the fact that~$\K_X$ is
the~$\sdense$-sheafification of~$\O_X$.

For the second statement, we need to verify that the nucleus~$j_{\Ass(\O_X)}$
associated to the subspace of associated points coincides with the
nucleus~$j_\sdense$ associated to the modal operator~$\sdense$. Recall from
Subsection~\ref{sect:subspace-to-modal-operator} that the latter is given by
\begin{align*}
  j_\sdense(U) &= \text{largest open subset of~$X$ on which~$\sdense U$ holds} \\
  &= \bigcup\ \{ V \subseteq X \ |\
  \text{$V$ open},\ V \models \sdense U \}
\intertext{and note that the former is given by}
  j_{\Ass(\O_X)}(U) &= \bigcup\ \{ V \subseteq X \ |\
  \text{$V$ open},\ V \cap \Ass(\O_X) \subseteq U \}.
\end{align*}
This is a general fact of locale theory, not depending on particular properties
of~$\Ass(\O_X)$. To verify this, check that~$j_{\Ass(\O_X)}$ is indeed a
nucleus and that~$\{ U \in \Open(X) \,|\,
j_{\Ass(\O_X)}(U) = U \} \to \Open(\Ass(\O_X)),\, U \mapsto \Ass(\O_X) \cap U$
is an isomorphism of frames with inverse given by~$\Ass(\O_X) \cap U \mapsto
j_{\Ass(\O_X)}(U)$.

The equivalence thus follows from a standard result on the set of associated
points on locally Noetherian schemes:
\begin{align*}
  &\ V \cap \Ass(\O_X) \subseteq U \\
  \Longleftrightarrow&\
    \Ass(\O_V) \subseteq U \\
  \Longleftrightarrow&\
    \text{$U \cap V$ is scheme-theoretically dense in~$U$} \\
  &\qquad\qquad\text{(this steps requires the Noetherian assumption)} \\
  \Longleftrightarrow&\
    V \models \sdense U. \qedhere
\end{align*}
\end{proof}

\begin{lemma}
Let~$X$ be a scheme admitting a cover of open affine subsets with
property~$(\star\star)$. Let~$j : U \hookrightarrow X$ be the inclusion of an
open subset containing the sublocale~$X_\sdense$. (If~$X$ is locally
Noetherian, this is equivalent to requiring that~$U$ contains~$\Ass(\O_X)$.)
Then the canonical morphism~$\K_X \to j_* \K_U$ is an isomorphism.
\end{lemma}
\begin{proof}Write~$i : X_\sdense \hookrightarrow X$ and~$i' : X_\sdense
\hookrightarrow U$ for the inclusions. By the previous proposition, the
sheaf~$\K_X$ is given by~$i_* i^{-1} \O_X$. Similarly, the sheaf~$j_* \K_U$ is
given by~$j_* i'_* i'^{-1} j^{-1} \O_X$. The claim follows since~$j \circ i' =
i$.
\end{proof}
% XXX: Does U admit a (**)-cover if X does?

\begin{lemma}\label{lemma:dense-standard-reflection-generalized}
Let~$X$ be scheme admitting a cover by affine open subschemes
with property~$(\star)$ respectively~$(\star\star)$. Let~$\varphi$ be any formula
over~$X$. Then
\[ \Sh(X) \models \neg\neg\varphi \Longrightarrow \exists f\?\O_X\_
  \neg\neg(\speak{$f$ \inv}) \wedge (\speak{$f$ \inv} \Rightarrow \varphi) \]
respectively
\[ \Sh(X) \models \sdense\varphi \Longrightarrow \exists f\?\O_X\_
  \sdense(\speak{$f$ \inv}) \wedge (\speak{$f$ \inv} \Rightarrow \varphi). \]
\end{lemma}
\begin{proof}The proof of Lemma~\ref{lemma:dense-standard-reflection} carries
over, \emph{mutatis mutandis}.
\end{proof}

\begin{prop}\label{prop:boolean-dim0-continued}
Let~$X$ be a scheme of dimension~$\leq 0$ such that the set of irreducible components is locally finite or such that~$X$ is locally Noetherian. Then
the internal language of~$\Sh(X)$ is Boolean. (The converse holds as well and
was already stated as Corollary~\ref{cor:boolean-dim0}.)
\end{prop}
\begin{proof}
It suffices to verify the principle of double negation elimination, since the
law of excluded middle is equivalent to it.\footnote{This is a standard fact of
intuitionistic logic. Assume that the principle of double negation elimination
holds. We want to verify the law of excluded middle, so let an arbitrary
formula~$\varphi$ be given. Even intuitionistically it holds
that~$\neg\neg(\varphi \vee \neg\varphi)$. By double negation elimination it
follows that~$\varphi \vee \neg\varphi$.}
So let~$\varphi$ be an arbitrary formula and assume~$\neg\neg\varphi$. By the
previous lemma there exists an element~$f\?\O_X$ such that~$f$ is \notnot
invertible and such that~$(\speak{$f$ \inv} \Rightarrow \varphi)$. Since~$\dim
X \leq 0$, this element is invertible or nilpotent
(Corollary~\ref{cor:scheme-dimension-zero}).  In the first case, we are done.
In the second case, some power~$f^n$ is zero and therefore in particular
\notnot zero. Since~$f$ is \notnot invertible, this implies that \notnot~$1 =
0$. On the other hand~$1 \neq 0$, so we obtain a contradiction; from this
contradiction~$\varphi$ trivially follows.
\end{proof}

\begin{lemma}\label{lemma:torsion-module-generic-stalk-generalized}
Let~$X$ be a locally Noetherian scheme. Let~$\F$ be a
quasicoherent~$\O_X$-module. Then~$\F$ is a torsion module if and only if the
restriction of~$\F$ to~$\Ass(\O_X)$ vanishes.
\end{lemma}
\begin{proof}By Proposition~\ref{prop:kx-ass} and
Lemma~\ref{lemma:dense-standard-reflection-generalized} it suffices to repeat
the proof of Lemma~\ref{lemma:torsion-module-generic-stalk} with
``\notnot'' substituted by~``$\sdense$''.
\end{proof}


\subsection{Cartier divisors} Let~$X$ be a scheme (or ringed space). Recall
that a \emph{Cartier divisor} on~$x$ is a global section of the sheaf of
groups~$\K_X^\times / \O_X^\times$. This sheaf can be constructed internally, with the
same notation: It is the quotient of the group of invertible elements of the
ring~$\K_X$ by the subgroup of invertible elements of the ring~$\O_X$. So an
arbitrary section of~$\K_X^\times/\O_X^\times$ is internally of the form~$[s/t]$
with~$s,t\?\O_X$ being regular elements; this is a simpler description than the
usual external one as a family~$(f_i)_i$ of functions~$f_i \in
\Gamma(U_i,\K_X^\times)$ such that~$f_i^{-1}|_{U_i \cap U_j} \cdot f_j|_{U_i \cap
U_j} \in \Gamma(U_i \cap U_j, \O_X^\times)$ for all~$i,j$.

We can sketch the basic theory of Cartier divisors completely from the internal
perspective. In accordance with common practice, we write the group
operation of~$\K_X^\times/\O_X^\times$ (which is induced by multiplication of elements
in~$\K_X^\times$) additively.

\begin{defn}\label{defn:effective-cartier-divisor}
A Cartier divisor is \emph{effective} if and only if, from the
internal perspective, it can be written in the form~$[s/1]$ with~$s\?\O_X$
being a regular element.\end{defn}

Thus a Cartier divisor~$[s/t]$ is effective if and only if~$s$ is
an~$\O_X$-multiple of~$t$.

\begin{defn}A Cartier divisor~$D$ is \emph{principal} if and only if there
exists a global section~$f \in \Gamma(X,\K_X^\times)$ such that internally,~$D = [f]$.
Two Cartier divisors are \emph{linearly equivalent} if and only if their
difference is a principal divisor.
\end{defn}

Note that decidedly, principality is a global notion: For \emph{any} divisor~$D$ it is
true that locally there exists sections~$f$ of~$\K_X^\times$ such that~$D = [f]$.

\begin{defn}\label{defn:line-bundle-of-divisor}
The \emph{line bundle associated to a Cartier divisor}~$D$
is the~$\O_X$-submodule
\[ \O_X(D) \defeq \{ g \in \K_X \,|\, g D \in \O_X \} = D^{-1} \O_X \subseteq \K_X
\]
of~$\K_X$. Here we are abusing language for~``$gD \in \O_X$'' to mean that~$gf
\in \O_X$ if~$D = [f]$ with~$f\?\K_X$; and for~``$D^{-1} \O_X$'' to
mean~$f^{-1}\O_X$. This condition respectively submodule does not depend on the
representative~$f$, since~$f$ is well-defined up to multiplication by an element
of~$\O_X^\times$.\end{defn}

The submodule~$\O_X(D)$ is indeed locally free of rank~$1$, since
internally~$f^{-1}$ gives an one-element basis. Note that~$D$ is effective if
and only if~$\O_X(-D)$ is a subset of~$\O_X$ from the internal perspective
(this comparison makes sense, since~$\O_X(-D)$ and~$\O_X$ are both canonically
embedded in~$\K_X$). In
this case, we can define the \emph{support} of~$D$ to be the closed subscheme
of~$X$ associated to the sheaf of ideals~$\O_X(-D) \subseteq \O_X$.

%\begin{prop}Let~$D$ be an effective divisor on~$X$. Then~$\O_X(D)$ is trivial
%outside of the support of~$D$.\end{prop}
%\begin{proof}We have to show that~$\O_X(D)|_U$ is trivial on~$U \defeq X \setminus
%V(\O_X(-D)) = \{ x \in X \,|\, 1 \in \O_X(-D) \}$. We do this by verifying ...
%D = [f], f : O_X regular. O_X(-D) = (f).
%Assume 1 in O_X(-D): Then f is invertible. Obviously O_X(-D) --> O_X
%is an isomorphism. So O_X --> O_X(D) is as well.
%\end{proof}

\begin{defn}The \emph{Cartier divisor associated to a free~$\O_X$-submodule~$\L \subseteq
\K_X$ of rank~1} is~$D \defeq [f^{-1}]$, where~$f\?\K_X$ is the unique element of
some one-element basis of~$\L$.\end{defn}

The basis element~$f\?\K_X$ does indeed lie in~$\K_X^\times$: Write~$f
= s/t$ with~$s,t \? \O_X$. It suffices to show that~$s$ is a regular element
of~$\O_X$. So let~$h\?\O_X$ such that~$sh = 0$ in~$\O_X$. Then in
particular~$hf = 0$ in~$\K_X$. By linear independence, it follows that~$h = 0$
in~$\K_X$ and thus~$h = 0$ in~$\O_X$.

Furthermore, the associated divisor does not depend on the choice of~$f$,
since~$f$ is well-defined up to multiplication by an element of~$\O_X^\times$: If~$f
\O_X = g \O_X \subseteq \K_X$, then there exist~$u,v\?\O_X$ such that~$fu = g$
and~$gv = f$ in~$\K_X$. It follows that~$uv = fuvf^{-1} = gvf^{-1} = ff^{-1} =
1$ in~$\K_X$ and thus in~$\O_X$, by injectivity of the canonical map~$\O_X \to
\K_X$. Therefore~$u$ and~$v$ are elements of~$\O_X^\times$.

\begin{lemma}Let~$D$ and~$D'$ be divisors on~$X$. Then~$\O_X(D) \otimes_{\O_X}
\O_X(D') \cong \O_X(D + D')$.\end{lemma}
\begin{proof}The wanted morphism of sheaves~$\O_X(D) \otimes \O_X(D') \to
\O_X(D + D')$ is given by multiplication. That this is well-defined and an
isomorphism can be checked from the internal point of view, where the claims
are obvious.\end{proof}

\begin{prop}The association~$D \mapsto \O_X(D)$ defines an one-to-one
correspondence between Cartier divisors on~$X$ and rank-one submodules
of~$\K_X$. This correpondence descends to an one-to-one correspondence between
Cartier divisiors up to linear equivalance and rank-one submodules of~$\K_X$ up
to isomorphism (as abstract~$\O_X$-modules, ignoring their embedding
into~$\K_X$).\end{prop}
\begin{proof}The first statement is obvious from the definitions. For the
second statement, it suffices to show that~$\O_X(D)$ is isomorphic to~$\O_X$ if
and only if~$D$ is principal. A given isomorphism~$\O_X \to \O_X(D)$ gives a
global section~$f \in \K_X^\times$ (by considering the image of the unit element)
such that internally,~$D = [f^{-1}]$; this shows that~$D$ is principal. The
converse is similar.
\end{proof}

%\begin{rem}Locally principal subschemes (closed subschemes which are locally
%the vanishing subscheme of a regular section of~$\O_X$) up to isomorphisms of
%subschemes are in one-to-one correspondence with rank-1 submodules of~$\O_X$
%(see~\xxx{ref}). Thus locally principal subschemes (up to isomorphisms of abstract
%schemes) are in one-to-one correspondence with effective Cartier divisors (up
%to linear equivalence).\end{rem}
%\xxx{check this.}

For the following definition, recall that we can localize an~$\O_X$-module~$\L$
away from the set~$\S \subseteq \O_X$ of regular elements to obtain
a~$\K_X$-module~$\L[S^{-1}]$.

\begin{defn}Let~$f\?\L[\S^{-1}]$ be a rational section of a line bundle~$\L$
on~$X$. Assume that~``$f$ is nontrivial'', that is multiplication by~$f$ is an
injective map~$\O_X \to \L[\S^{-1}]$. Then the \emph{associated divisor} of~$f$
is~$\operatorname{div}(f) \defeq [\psi(s)/t]$, where~$f = s/t$ with~$s\?\L$ and~$t\?\O_X$
and~$\psi : \L \to \O_X$ is an isomorphism.\end{defn}

One can check that~$\psi(s)$ is a regular element of~$\O_X$; this statement is
in fact equivalent to the multiplication map~$\O_X \to \L[\S^{-1}]$ being
injective. Furthermore one can check that~$[\psi(s)/t]$ does not depend on the
choice of~$s$,~$t$, and~$\psi$.

\begin{prop}Let~$f\?\L[\S^{-1}]$ be a nontrivial rational section of a line
bundle~$\L$ on~$X$. Then multiplication by~$f$ induces an
isomorphism~$\O_X(\operatorname{div}(f)) \to \L$.\end{prop}
\begin{proof}The isomorphism should map a rational function~$g$ to~$g \cdot f$. This
is a priori an element of~$\L[\S^{-1}]$; we have to check that it can be
regarded as an element of~$\L$. Just as in the definition
of~$\operatorname{div}(f)$, write~$f = s/t$ and fix an isomorphism~$\psi : \L
\to \O_X$. Write~$g = (t/\psi(s)) \cdot h$ for some function~$h\?\O_X$. Then~$g
\cdot f = sh/\psi(s) = h\psi^{-1}(1)$, since~$s = \psi^{-1}(\psi(s)) = \psi(s)
\cdot \psi^{-1}(1)$. The element~$h\psi^{-1}(1)$ can indeed be considered as an
element of~$\L$.

Injectivity of the map~$\O_X(\operatorname{div}(f)) \to \L$ is by the
nontriviality of~$f$. For surjectivity, note that~$(t/\psi(s)) \cdot \psi(v)$ is a
preimage to~$v\?\L$, since~$(t/\psi(s)) \cdot \psi(v) \cdot f = \psi(v) \psi(s)
\psi^{-1}(1) / \psi(s) = v$.
\end{proof}

\begin{prop}Let~$\L$ be a line bundle on~$X$. Assume that~$\L$ can be embedded
into~$\K_X$. Then~$\L$ possesses a nontrivial rational section.
\end{prop}
\begin{proof}Let~$i : \L \to \K_X$ denote the given injection. Let~$(v)$ be an
one-element basis for~$\L$. Write~$i(v) = s/t$. Then~$s$ is regular,
since~$hs = 0$ implies~$i(hv) = 0$ and thus~$h = 0$, for any~$h\?\O_X$.
Therefore~$f \defeq tv/s$ is a well-defined element of~$\L[\S^{-1}]$.
Furthermore it is nontrivial in the desired sense: If~$h \cdot (tv/s) = 0$,
then~$htv = 0$, thus~$ht = 0$ and~$h = 0$.

It remains to check that~$f$ is independent of the choice of~$v$ and of the
representation~$i(v) = s/t$; else we defined only local sections which might not
glue to a single nontrivial rational section (externally speaking). This is
obvious.
\end{proof}

\begin{prop}Let~$D$ be an effective divisor on~$X$. Then the complement of its
support is scheme-theoretically dense.\end{prop}
\begin{proof}The complement of the support of~$D$, that is~$D(\O_X(-D))$, is
the truth value associated to the statement~``$1 \in \O_X(-D)$''. By
Lemma~\ref{lemma:scheme-theoretical-density}, we therefore have to verify
that~$\O_X$ is separated with respect to the modal operator~$(1 \in \O_X(-D)
\Rightarrow \placeholder)$.

Let~$s \? \O_X$ be given such that~$1 \in \O_X(-D) \Rightarrow s = 0$; we have
to show that~$s = 0$. Writing~$D = [f/1]$ where~$f \? \O_X$ is a regular
element, this condition is equivalent to~$\speak{$f$ \inv} \Rightarrow s = 0$.
By Proposition~\ref{prop:cond-zero} it follows that~$f^n s = 0$ for some~$n
\geq 0$. Since~$f$ is regular, we may cancel~$f^n$ in this equation.
\end{proof}

\begin{prop}Assume that~$X$ is an integral scheme. Then any line bundle on~$X$
is (uncanonically) a submodule of~$\K_X$.\end{prop}
\begin{proof}Let~$\xi$ be the generic point of~$X$ and let~$\Box \defeq \neg\neg$
denote the modal operator such that internal sheafification with respect
to~$\Box$ is the same as pulling back to~$\{\xi\}$ and then pushing forward
to~$X$ again (see Section~\ref{sect:negneg-sheaves}). Let~$\L$ be a line bundle on~$X$. Since~$\L_\xi \cong
\O_{X,\xi}$ (uncanonically), there is some injection~$\L_\xi \to \K_{X,\xi}$;
this corresponds internally to an injection~$\L^{++} \to \K_X^{++}$.
Since~$\K_X$ is already a~$\Box$-sheaf (see
Proposition~\ref{prop:kx-is-negneg-sheafification}) and~$\L$ is~$\Box$-separated
(being isomorphic to~$\O_X$), we have the global injection
\[ \L \lhra \L^{++} \lhra \K_X^{++} \stackrel{({\cong})^{-1}}{\longrightarrow} \K_X. \qedhere \]
\end{proof}

\begin{itemize}
\item ``$\operatorname{div}(g) + D \geq 0$''
\end{itemize}


\section{Compactness and metaproperties}
\label{sect:compactness}

\subsection{Qausicompactness}

As stated in the introduction, quasicompactness of a space can not be detected
by the internal language: There cannot exist a formula~$\varphi$ such that a
topological space is quasicompact if and only if~$\Sh(X) \models \varphi$,
since the latter is always a local property on~$X$ while quasicompactness is not.
However, quasicompactness can be characterized by a \emph{metaproperty} of the
internal language.

This result is best stated in a way which does not explicitly refer to a notion
of finiteness. So recall that quasicompactness of a topological space~$X$ can
be phrased in the following way: For any directed set~$I$ and any monotone
family~$(U_i)_{i \in I}$ of open subsets, if~$X = \bigcup_i U_i$ then~$X = U_i$
for some~$i \in I$. As usual, a \emph{directed set} is an inhabited partially
ordered set such that for any two elements there exists a common upper bound.
A family~$(U_i)_{i \in I}$ is \emph{monotone} if and only if~$i \preceq j$
implies~$U_i \subseteq U_j$.

\begin{prop}\label{prop:quasicompact-meta}
Let~$X$ be a topological space. Then~$X$ is quasicompact if and
only if the internal language of~$\Sh(X)$ has the following metaproperty:
For any directed set~$I$ and any monotone family~$(\varphi_i)_{i \in I}$ of
formulas over~$X$,
\[ \Sh(X) \models \bigvee_{i \in I} \varphi_i
  \quad\text{implies}\quad
  \text{for some~$i \in I$, $\Sh(X) \models \varphi_i$}. \]
The monotonicity condition means that~$\Sh(X) \models (\varphi_i \Rightarrow
\varphi_j)$ for any~$i \preceq j$ in~$I$.
\end{prop}

Stated more succintly, a topological space~$X$ is quasicompact if and only
if~``$\Sh(X) \models$'' commutes with directed~``$\bigvee_{i \in I}$'''s.

\begin{proof}For the ``only if'' direction, let such a family of formulas be
given. Declare~$U_i$ to be the largest open subset of~$X$ where~$\varphi_i$
holds. Then by assumption, the~$U_i$ form a monotone family and cover~$X$. By
quasicompactness of~$X$, some single~$U_i$ covers~$X$ as well, such that the
corresponding formula~$\varphi_i$ holds on~$X$.

For the ``if'' direction, note that a monotone family~$(U_i)$ of open subsets
induces a monotone family of formulas by defining~$\varphi_i \defequiv U_i$. This
correspondence is such that~$\Sh(X) \models \bigvee_i \varphi_i$ holds if and
only if~$X = \bigcup_i U_i$ and such that~$\Sh(X) \models \varphi_i$ if and
only if~$X = U_i$. With these observations the claim is obvious.
\end{proof}

\XXX{formulate for locally constant index sheaves I as well.}

\begin{ex}\label{ex:nilpotency-directed}
Let~$X$ be a quasicompact scheme (or quasicompact ringed space).
Let~$f \in \Gamma(X,\O_X)$ be a global function. Endow the set of natural
numbers with the usual ordering. Then the family of formulas given by~$(f^n =
0)_{n \in \NN}$ is monotone. Thus, if it internally holds that~$f$ is
nilpotent, then~$f$ is nilpotent as an element of~$\Gamma(X,\O_X)$ as
well.\end{ex}

\begin{prop}Let~$X$ be a topological space. Let~$K \subseteq X$ be an open
subset which is \emph{locally quasicompact} in the sense that there exists an open
covering~$X = \bigcup_j U_j$ such that each~$K \cap U_j$ is quasicompact. Then the
internal language of~$\Sh(X)$ has the following metaproperty: For any
directed set~$I$ and monotone family~$(\varphi_i)_{i \in I}$ of formulas
over~$X$ it holds that
\[ \Sh(X) \models \bigl(K \Rightarrow \bigvee_i \varphi_i\bigr)
  \quad\text{implies}\quad
  \Sh(X) \models \bigvee_i (K \Rightarrow \varphi_i). \]
If additionally for any open subset~$V \subseteq X$ the set~$K \cap V$ is
locally quasicompact in~$V$, the following stronger and purely internal
statement holds:
\[ \Sh(X) \models \bigl(K \Rightarrow \bigvee_i \varphi_i\bigr)
  \Longrightarrow
  \bigvee_i (K \Rightarrow \varphi_i). \]
\end{prop}
\begin{proof}Assume that~$\Sh(X) \models (K \Rightarrow \bigvee_i \varphi_i)$.
This is equivalent to~$K \models \bigvee_i \varphi_i$. By the locality of the
internal language, it follows that~$K \cap U_j \models \bigvee_i \varphi_i$ for each~$j$.
Since~$K \cap U_j$ is quasicompact, it follows by the previous proposition that
there exists an index~$i_j \in I$ such that~$K \cap U_j \models \varphi_{i_j}$.
This is equivalent to~$U_j \models (K \Rightarrow \varphi_{i_j})$. In
particular, it holds that~$U_j \models \bigvee_i (K \Rightarrow \varphi_i)$.
Since this is true for any~$j$, it follows that~$X \models \bigvee_i (K
\Rightarrow \varphi_i)$, again by the locality of the internal language.

The second statement is a corollary of the first one.
\end{proof}

\begin{ex}Any retrocompact subset of a scheme is locally quasicompact in the
sense of the proposition.\end{ex}

\begin{ex}\label{ex:df-locally-compact}
Let~$X$ be a scheme and~$f \in \Gamma(X,\O_X)$ be a global function.
Then the open set~$D(f) = \{ x \in X \,|\, \text{$f_x$ is invertible in~$\O_{X,x}$}
\}$ is locally quasicompact in the sense of the proposition, even in the
stronger sense: Let~$V \subseteq X$ be any open set. Consider a covering~$V = \bigcup_i
U_i$ by open affine subsets~$U_i = \Spec A_i$. Then~$D(f) \cap U_i \cong \Spec
A_i[f^{-1}]$ is quasicompact.\end{ex}

From this example it will trivially follow that the nilradical~$\sqrt{(0)}
\subseteq \O_X$ of a scheme and indeed the radical of any quasicoherent ideal
sheaf is quasicoherent (Example~\ref{ex:radical-qcoh}). This example is also
pivotal for giving a simple description of the quasicoherator
(Proposition~\ref{prop:quasicoherator-arbitrary-algebra}), which in turn is
needed for an internal understanding of the relative
spectrum (Section~\ref{sect:relative-spectrum}).

\begin{rem}In applications, the open set~$K$ of the proposition is often given
as the largest open subset on which some formula~$\psi$ holds. (For instance,
in the previous example,~$K$ was given by the formula~$\speak{$f$
is invertible in $\O_X$}$.)
Then the conclusion of the proposition is that \emph{assuming that~$\psi$ holds commutes
with directed disjunctions}.\end{rem}


\subsection{Locality}

A stronger condition on a topological space~$X$ than quasicompactness is
locality: A topological space is \emph{local} if and only if for any open
covering~$X = \bigcup_i U_i$ (not necessarily directed) a certain single subset~$U_i$
covers~$X$ as well. For instance, the spectrum of a ring~$A$ is local if and only
if~$A$ is a local ring. Locality has the following characterization as a metaproperty
of~$\Sh(X)$.

\begin{prop}\label{prop:local-meta}Let~$X$ be a topological space. Then~$X$ is local if and
only if the internal language of~$\Sh(X)$ has the following metaproperty:
For any set~$I$ and any family~$(\varphi_i)_{i \in I}$ of
formulas over~$X$, it holds that
\[ \Sh(X) \models \bigvee_{i \in I} \varphi_i
  \quad\text{implies}\quad
  \text{for some~$i \in I$, $\Sh(X) \models \varphi_i$}. \]
In this case, the internal language has additionally the following (weaker) metaproperty: For any
sheaf~$\F$ on~$X$ and any formula~$\varphi(s)$ containing a variable~$s\?\F$,
it holds that
\[ \Sh(X) \models \exists s\?\F\_ \varphi(s)
  \quad\text{implies}\quad
  \text{for some~$s \in \Gamma(X,\F)$, $\Sh(X) \models \varphi(s)$}. \]
\end{prop}
\begin{proof}The proof of the first part is very similar to the proof of the
previous proposition. For the ``only if'' direction of the second part, note
that the antecedent implies that there exist local section~$s_i \in
\Gamma(U_i,\F)$ such that~$U_i \models \varphi(s_i)$ for some open covering~$X
= \bigcup_i U_i$. By locality of~$X$, one such~$U_i$ suffices to cover~$X$; so
the corresponding section~$s_i$ is actually a global section and verifies~$X
\models \varphi(s_i)$.
\end{proof}

\begin{rem}The second metaproperty stated in the proposition is indeed weaker
than the condition that~$X$ is local. For instance, let~$X$ be a space consisting
of two discrete points. Then~$\Sh(X)$ has the second metaproperty, but~$X$ is
not local.\end{rem}


\subsection{Irreducibility}

In intuitionistic logic, De Morgan's law~$\neg(\alpha \wedge \beta)
\Rightarrow \neg\alpha \vee \neg\beta$ is not generally justified; therefore we
can't use it when working internally to the topos of sheaves on a general scheme~$X$.
The following proposition demonstrates that if~$X$ is irreducible, the law
does hold.

\begin{prop}\label{prop:irreducibility-internally}
A topological space~$X$ is irreducible if and only if the internal
language of~$\Sh(X)$ has the following metaproperty: For any
formulas~$\varphi$ and~$\psi$
\[ \Sh(X) \models \neg(\varphi \wedge \psi)
  \quad\text{implies}\quad
  \Sh(X) \models \neg\varphi \text{ or }
  \Sh(X) \models \neg\psi. \]
Furthermore, in this case the following internal logical principle holds:
\[ \Sh(X) \models \forall \alpha,\beta \in \Omega\_
  \neg(\alpha \wedge \beta) \Rightarrow (\neg\alpha \vee \neg\beta). \]
\end{prop}
\begin{proof}The statement ``$\Sh(X) \models \neg(\varphi \wedge \psi)$'' means
that~$U \cap V = \emptyset$, where~$U$ and~$V$ are the largest open subsets on
which~$\varphi$ respectively~$\psi$ hold. The disjunction ``$\Sh(X) \models
\neg\varphi$ or $\Sh(X) \models \neg\psi$'' means that~$U = \emptyset$ or~$V =
\emptyset$.

Therefore, if~$X$ is irreducible, then the internal language has the claimed metaproperty. The converse
can be seen by instantiating~$\varphi$ and~$\psi$ with the formulas associated
to given open subsets having empty intersection. It then follows that one of
these formulas is false in the internal language; thus the associated subset is
empty.

The stated internal logical principle holds since nonempty open subsets of irreducible spaces are
irreducible.
\end{proof}


\subsection{Internal proofs of common lemmas}

\begin{lemma}Let~$X$ be an irreducible reduced scheme. Then all local
rings~$\O_{X,x}$ are integral domains.\end{lemma}
\begin{proof}It suffices to give a proof of the following statement: Let~$R$ be
a local ring such that elements which are not invertible are nilpotent. Further
assume that~$R$ is reduced. Then~$R$ is an integral domain in the weak sense.

This proof may, additionally to the rules of intuitionistic logic, use the
classical axiom given by Proposition~\ref{prop:irreducibility-internally}.

So let arbitrary elements~$x,y \? R$ with~$xy = 0$ be given. Then it is not the
case that~$x$ and~$y$ are both invertible: If they were, their product~$xy$
would be invertible as well, contradicting~$1 \neq 0$. By the classicality
principle, it follows that~$x$ is not invertible or that~$y$ is not invertible.
Thus~$x$ or~$y$ is nilpotent and therefore zero.
\end{proof}


\begin{itemize}
\item basic lemmas: filtered colimits, flatness, \ldots
\end{itemize}


\section{Quasicoherent sheaves of modules}
\label{sect:qcoh}

Recall that an~$\O_X$-module~$\F$ on a ringed space~$X$ is \emph{quasicoherent}
if and only if there exists a covering of~$X$ by open subsets~$U$ such that on
each such~$U$, there exists an exact sequence
\[ (\O_X|_U)^J \longrightarrow (\O_X|_U)^I \longrightarrow \F|_U \longrightarrow 0 \]
of~$\O_X|_U$-modules, where~$I$ and~$J$ are arbitrary sets (which may depend
on~$U$).

If~$X$ is indeed a scheme, quasicoherence can also be characterized in
terms of inclusions of distinguished open subsets of affines:
An~$\O_X$-module~$\F$ is quasicoherent if and only if for any open affine
subscheme~$U = \Spec A$ of~$X$ and any function~$f \in A$, the canonical map
\[ \Gamma(U,\F)[f^{-1}] \longrightarrow \Gamma(D(f),\F),\
  \tfrac{s}{f^n} \longmapsto f^{-n} s|_{D(f)} \]
is an isomorphism of~$A[f^{-1}]$-modules. Here~$D(f) \subseteq U$ denotes the
standard open subset~$\{ \ppp \in \Spec A \,|\, f \not\in \ppp \}$. Both
conditions can be internalized.

\begin{prop}Let~$X$ be a ringed space. Let~$\F$ be an~$\O_X$-module. Then~$\F$
is quasicoherent if and only if
\[ \Sh(X) \models \exists I,J\ \mathrm{lc}\_ \speak{there exists an
  exact sequence~$\O_X^J \to \O_X^I \to \F \to 0$}. \]
The ``\textnormal{lc}'' indicates that when interpreting this internal statement with the
Kripke--Joyal semantics,~$I$ and~$J$ should only be instantiated with
\emph{locally constant} sheaves.
\end{prop}
\begin{proof} We only sketch the proof.
The translation of the internal statement is that there exists a covering
of~$X$ by open subsets~$U$ such that for each such~$U$, there exist sets~$I,J$
and an exact sequence
\[ (\O_X|_U)^{\ul{J}} \longrightarrow (\O_X|_U)^{\ul{I}} \longrightarrow \F|_U
\longrightarrow 0 \]
where~$\ul{I}$ and~$\ul{J}$ are the constant sheaves associated to~$I$
respectively~$J$. The term~``$(\O_X|_U)^{\ul{I}}$'' refers to the internally
defined free~$\O_X$-module with basis the elements of~$\ul{I}$. By exploiting
that~$\ul{I}$ is a discrete set from the internal point of view (\ie any two
elements are either equal or not), one can show that this is the same
as~$(\O_X|_U)^I$; similarly for~$J$. With this observation, the statement
follows.
\end{proof}

\begin{rem}The restriction to locally constant sheaves is really necessary: The
internal statement~$\Sh(X) \models \exists I,J\_ \speak{there exists an
exact sequence~$\O_X^J \to \O_X^I \to \F \to 0$}$ is true for
\emph{any}~$\O_X$-module~$\F$. This is because the usual proof of the fact that
any module admits a resolution by (not necessarily finite) free modules is
intuitionistically acceptable and thus also valid in the internal
universe.\end{rem}

We don't think that there is a useful internal characterization of
locally constant sheaves. The alternative internal condition given by the following
theorem does not need such a characterization.

\begin{thm}\label{thm:qcoh-sheafchar}
Let~$X$ be scheme. Let~$\F$ be an~$\O_X$-module. Then~$\F$ is
quasicoherent if and only if, from the internal perspective, for any~$f\?\O_X$,
the localized module~$\F[f^{-1}]$ is a sheaf for the modal operator~$(\speak{$f$ \inv}
\Rightarrow \placeholder)$.
\end{thm}

In detail, the internal condition is that for any~$f\?\O_X$, it holds that
\[ \forall s\?\F[f^{-1}]\_
  (\speak{$f$ \inv} \Rightarrow s = 0) \Longrightarrow s = 0 \]
and for any subsingleton~$\S \subseteq \F[f^{-1}]$ it holds that
\[ (\speak{$f$ \inv} \Rightarrow \speak{$\S$ inhabited}) \Longrightarrow
  \exists s\?\F[f^{-1}]\_
  (\speak{$f$ \inv} \Rightarrow s \in \S). \]
Unlike with the internalizations of finite type, finite presentation and
coherence, this condition is \emph{not} a standard condition of commutative
algebra. In fact, in classical logic, this condition is always satisfied --
for trivial logical reasons if~$f$ is invertible, and because~$\F[f^{-1}]$ is
the zero module if~$f$ is not invertible (since~$f$ is nilpotent then, by
Proposition~\ref{prop:neginvnilpotent}).

That this condition in not known in commutative algebra is to be expected:
Quasicoherence is a condition on sheaves of modules, ensuring
that they are locally isomorphic to sheaves of the form~$M^\sim$,
where~$M$ is a plain module. But in commutative algebra, one \emph{only} studies plain
modules (and not sheaves of modules). The quasicoherence condition is imported
into the realm of commutative algebra only by the internal language.

We give the proof of the theorem below, after first discussing some examples
and consequences. The proof will explain the origin of this condition.

\begin{ex}The zero~$\O_X$-module is quasicoherent, since (it and) all
localizations of it are singleton sets from the internal perspective and
thus~$\Box$-sheaves for any modal operator~$\Box$
(Example~\ref{ex:special-sets-sheaves}).\end{ex}

\begin{cor}\label{cor:submodule-qcoh}
Let~$X$ be a scheme. Let~$\F$ be a quasicoherent~$\O_X$-module.
Let~$\G \subseteq \F$ be a submodule. Then~$\G$ is quasicoherent if and only
if
\[ \Sh(X) \models \forall f\?\O_X\_
  \forall s\?\F\_
  (\speak{$f$ \inv} \Rightarrow s \in \G) \Longrightarrow
  \bigvee_{n \geq 0} f^n s \in \G. \]
\end{cor}
\begin{proof}We can give a purely internal proof. Let~$f\?\O_X$.
Since subpresheaves of separated sheaves are separated, the module~$\G[f^{-1}]$
is in any case separated with respect to the modal operator~$\Box$
with~$\Box\varphi \defequiv (\speak{$f$ \inv} \Rightarrow \varphi)$.

Now suppose that~$\G$ is quasicoherent. Let~$f\?\O_X$. Let $s\?\F$ and assume that
if~$f$ were invertible,~$s$ would be an element of~$\G$. Define the
subsingleton~$S \defeq \{ t\?\G[f^{-1}] \,|\, \speak{$f$ \inv} \wedge t=s/1 \}$.
Then~$S$ would be inhabited by~$s/1$ if~$f$ were invertible. Since~$\G[f^{-1}]$
is a~$\Box$-sheaf, it follows that there exists an element~$u/f^n$ of~$\G[f^{-1}]$
such that, if~$f$ were invertible, it would be the case that~$u/f^n = s/1 \in
\G[f^{-1}] \subseteq \F[f^{-1}]$.
Since~$\F[f^{-1}]$ is~$\Box$-separated, it follows that it actually holds that~$u/f^n
= s/1 \in \F[f^{-1}]$. Therefore there exists~$m\?\NN$ such that $f^m f^n s =
f^m u \in \F$. Thus~$f^{m+n} s$ is an element of~$\G$.

For the converse direction, assume that~$\G$ fulfills the stated condition.
Let$f\?\O_X$. Let~$S \subseteq \G[f^{-1}]$ be a subsingleton which would be
inhabited if~$f$ were invertible. By regarding~$S$ as a subset of~$\F[f^{-1}]$,
it follows that there exists an element~$u/f^n \in \F[f^{-1}]$ such that,
if~$f$ were invertible, $u/f^n$ would be an element of~$S$. In particular,~$u$
would be an element of~$\G$. By assumption
it follows that there exists~$m\?\NN$ such that~$f^m u \in G$. Thus~$(f^m u) /
(f^m f^n)$ is an element of~$\G[f^{-1}]$ such that, if~$f$ were invertible, it
would be an element of~$S$.
\end{proof}

\begin{ex}\label{ex:annihilator-qcoh}
Let~$X$ be a scheme and~$s$ be a global section of~$\O_X$. Then the
annihilator of~$s$, \ie the sheaf of ideals internally defined by the
formula
\[ I \defeq \Ann_{\O_X}(s) = \{ t\?\O_X \,|\, st = 0 \} \subseteq \O_X \]
is quasicoherent. To prove this in the internal language it suffices to
verify the condition of the proposition.
So let~$f\?\O_X$ and~$t\?\O_X$ be arbitrary and assume~$\speak{$f$ \inv} \Rightarrow t \in I$,
\ie assume that if~$f$ were invertible, then~$st$ would be zero. By
Proposition~\ref{prop:cond-zero} it follows that~$f^n st = 0$ for
some~$n\?\NN$, \ie that~$f^n t \in I$.
\end{ex}

\begin{ex}\label{ex:radical-qcoh} Let~$X$ be a scheme and~$\I \subseteq \O_X$
be a quasicoherent ideal sheaf.  Then the radical of~$\I$, internally definable
as \[ \sqrt{\I} \defeq \Bigl\{ s\?\O_X \,\Big|\, \bigvee_{n \geq 0} s^n \in \I \Bigr\}, \] is
quasicoherent as well: Let~$f\?\O_X$ and~$s\?\O_X$ be arbitrary and
assume~$\speak{$f$ \inv} \Rightarrow s \in \sqrt{\I}$, \ie assume that if~$f$
were invertible, some power~$s^n$ would be an element of~$\I$. Since
\emph{assuming that~$f$ is invertible commutes with directed disjunctions}
(Example~\ref{ex:df-locally-compact}), it follows that for some natural
number~$n$, it holds that~$\speak{$f$ \inv} \Rightarrow s^n \in \I$. By
quasicoherence of~$\I$, we may deduce that for some natural number~$m$, it
holds that~$f^m s^n \in \I$. Thus~$fs \in \sqrt{\I}$.\end{ex}

\begin{ex}\label{ex:qcoh-single-radical}
Let~$X$ be a scheme and~$\A$ be a quasicoherent~$\O_X$-algebra. Let~$h \in
\Gamma(X,\A)$ be a global section of~$\A$. Then the radical ideal~$\sqrt{(h)}
\subseteq \A$ is quasicoherent.\end{ex}
% XXX: Fill in proof.
% Assume (f inv => exists u: s = uh).
% Set K := { z : O[(fs)^(-1)] | exists u: z = u/1, s = uh }.
% Then there exists z = a/(fs)^n such that (fs inv => s/1 = ah/(fs)^n).
% Therefore (fs)^m (fs)^n s = (fs)^m ah.
% XXX: This only works for A = O.

\begin{prop}Let~$X$ be a scheme of dimension~$\leq 0$. Then any~$\O_X$-module
is quasicoherent.\end{prop}
\begin{proof}By Corollary~\ref{cor:scheme-dimension-zero}, any
element~$f\?\O_X$ is invertible or nilpotent. Therefore the quasicoherence
condition of Theorem~\ref{thm:qcoh-sheafchar} is trivially satisfied for any~$\O_X$-module.
\end{proof}

\begin{rem}\label{rem:qcoh-in-constructive-mathematics}
In general intuitionistic mathematics -- not inside the internal universe of a
scheme -- the notion of quasicoherence as given by the internal condition of
Theorem~\ref{thm:qcoh-sheafchar}
does not seem to be very interesting: For many important rings, there are few
quasicoherent modules in this sense. For instance, let~$M$ be a module over a
ring~$R$ in which every element is invertible or not invertible. (The
ring~$\ZZ$ is such a ring.) Then~$M$ is quasicoherent if and only if for any~$f
\? R$ which is not invertible, the localized module~$M[f^{-1}]$ is the zero
module, \ie any element of~$M$ is annihilated by some power~$f^n$. As a
concrete example, any~$\ZZ$-submodule of~$\ZZ$ which contains a nonzero element
fails to be quasicoherent.
\end{rem}

\begin{proof}[Proof of Theorem~\ref{thm:qcoh-sheafchar}]
By the well-known characterization of quasicoherence in terms of inclusions of
distinguished open subsets, an~$\O_X$-module~$\F$ is quasicoherent if and only
if for any affine open subset~$U \subseteq X$ and any function~$f \in
\Gamma(U,\O_U)$, the canonical map
\begin{equation}\label{eqn:restr-map}
  \Gamma(U,\F)[f^{-1}] \lra \Gamma(D(f),\F), \ s/f^n \longmapsto
  f^{-n} s|_{D(f)}
\end{equation}
is bijective. We will see that this map is injective for all such~$U$ and~$f$
if and only if from the internal perspective, for any~$f\?\O_X$, the set~$\F[f^{-1}]$ is a
separated presheaf with respect to the modal operator~$(\speak{$f$ \inv}
\Rightarrow \placeholder)$; and we will see that in this
case, the map is additionally surjective for all such~$U$ and~$f$ if the full
sheaf condition is fulfilled.

Since the sheaf~$\F[f^{-1}]$ does not appear in the stated characterization, we
will first reformulate the separatedness and the sheaf condition in terms
of~$\F$ instead of~$\F[f^{-1}]$. To this end, note that the separatedness
condition is equivalent to
\begin{equation}\label{eqn:separated}
  \forall f\?\O_X\_ \forall s\?\F\_
  (\speak{$f$ \inv} \Rightarrow s = 0 \? \F) \Longrightarrow
  \bigvee_{n \geq 0} f^n s = 0 \? \F.
\end{equation}
The equivalence can easily be proven in the internal language. The sheaf
condition is equivalent to the conjunction of the separatedness condition and
\begin{multline}\label{eqn:sheaf}
  \forall f\?\O_X\_ \forall \K \subseteq \F\_
  (\speak{$f$ \inv} \Rightarrow \speak{$K$ is a singleton})
  \Longrightarrow \\
  \bigvee_{n \geq 0} \exists s\?\F\_
  \speak{$f$ \inv} \Rightarrow f^{-n} s \in \K.
\end{multline}
In one direction, a set~$\S \subseteq \F[f^{-1}]$ is given; construct~$K \defeq \{
s\?\F \,|\, s/1 \in \S \} \subseteq \F$. In the other direction, a set~$\K
\subseteq \F$ is given; construct~$S \defeq \{ s\?\F[f^{-1}] \,|\, \exists
s'\?\F\_ s' \in \K \wedge s = s'/1 \} \subseteq \F[f^{-1}]$. The remaining
details can easily be filled in.

We now interpret the internal statement~\eqref{eqn:separated} with the
Kripke--Joyal semantics. Using the simplification rules, the external meaning
is that for any affine open subset~$U \subseteq X$ and any function~$f \in
\Gamma(U,\O_U)$ the following condition is satisfied: For any section~$s \in
\Gamma(U,\F)$ it should hold that
\[ U \models (\speak{$f$ \inv} \Rightarrow s = 0) \quad\text{implies}\quad
  U \models \bigvee_{n \geq 0} f^n s = 0. \]
The antecedent is equivalent to saying that~$s$ is zero in~$\Gamma(D(f),\F)$.
The consequent is (by quasicompactness of~$U$, see
Example~\ref{ex:nilpotency-directed}) equivalent to saying that for some~$n \geq 0$, the
section~$f^n s$ is zero in~$\Gamma(U,\F)$, \ie that~$s$ is zero
in~$\Gamma(U,\F)[f^{-1}]$. So this condition is precisely the injectivity of
the canonical map~\eqref{eqn:restr-map}.

The external meaning of statement~\eqref{eqn:sheaf} is that for any affine open
subset~$U \subseteq X$ and any function~$f \in \Gamma(U,\O_U)$ the following
condition is satisfied: For any subsheaf~$\K \subseteq \F|_U$ it should hold
that
\begin{multline*}
  U \models (\speak{$f$ \inv} \Rightarrow \speak{$\K$ is a singleton})
  \quad\text{implies} \\
  U \models \bigvee_{n \geq 0} \exists s\?\F\_
  \speak{$f$ \inv} \Rightarrow f^{-n} s \in \K.
\end{multline*}
Given the injectivity of the canonical map~\eqref{eqn:restr-map} (for any
affine open subset, not only~$U$), this condition is equivalent to its
surjectivity: To see that surjectivity is sufficient, let a subsheaf~$\K
\subseteq \F|_U$ verifying the antecedent be given. Since~$\K|_{D(f)}$ is a
singleton sheaf, we can consider its unique section~$u \in \Gamma(D(f),\K)
\subseteq \Gamma(D(f),\F)$. By surjectivity, there exists a preimage, \ie a
fraction~$s/f^n \in \Gamma(U,\F)[f^{-1}]$ such that~$u = f^{-n} s|_{D(f)}$
in~$\Gamma(D(f),\F)$. Thus~$U \models f^{-n}s \in \K$ holds and the consequent
is verified.

To see that surjectivity is necessary, let a section~$u \in \Gamma(D(f),\F)$ be
given. Define a subsheaf~$\K \subseteq \F|_U$ by setting~$\Gamma(V,\K) \defeq \{
u|_V \,|\, V \subseteq D(f) \}$. Then~$\K$ verifies the antecedent. Thus the
consequent holds: There exists an open covering~$U = \bigcup_i U_i$ such that
for each~$i$, there exists a natural number~$n_i$ and a section~$s_i \in
\Gamma(U_i,\F)$ such that~$f^{-n_i} s_i = u$ on~$U_i \cap D(f)$. Without loss of
generality, we may assume that the~$U_i$ are distinguished open subsets~$D(g_i)
\subseteq U$; that they are finite in number; and that the natural
numbers~$n_i$ agree with each other and thus equal some number~$n$. Since~$s_i
= s_j$ in~$\Gamma(U_i \cap U_j \cap D(f), \F)$, injectivity of the canonical
map~\eqref{eqn:restr-map} (on the affine set~$U_i \cap U_j = D(g_i g_j)$)
implies that~$s_i = s_j$ in~$\Gamma(U_i \cap U_j, \F)[f^{-1}]$. Thus for
any indices~$i,j$ there exists a natural number~$m_{ij}$ such that~$f^{m_{ij}} s_i =
f^{m_{ij}} s_j$ in~$\Gamma(U_i \cap U_j, \F)$. We may assume that the
numbers~$m_{ij}$ equal some common number~$m$; thus the local sections~$f^m s_i$
glue to a section~$s \in \Gamma(U,\F)$. The sought preimage of~$u$ is the
fraction~$s/f^{n+m}$, since~$f^{-(n+m)} s|_{D(f)}$ equals~$u$
in~$\Gamma(D(f),\F)$ (as this is true on the covering~$D(f) = \bigcup_i (D(f)
\cap U_i)$).
\end{proof}
% FUTURE: Remark that Gamma(U, F[f^(-1)]) = Gamma(U, F)[f^(-1)],
% since F[f^(-1)] is a filtered colimit (see
% stacks-project/sites:lemma-directed-colimits-sections).
% This argument only works if multiplication by f is injective as a map F --> F.

For applications in Section~\ref{sect:relative-spectrum} about interpreting the
relative spectrum as an internal spectrum, we want to specialize to radical
ideal sheaves. In particular, we want to describe the \emph{quasicoherator} --
the left adjoint to the inclusion of the quasicoherent radical ideals in the
poset of all radical ideals -- in simple terms.

\begin{prop}\label{prop:quasicoherator-structure-sheaf}
Let~$X$ be a scheme. Let~$\I \subseteq \O_X$ be a radical ideal.
\begin{enumerate}
\item The ideal~$\I$ is quasicoherent if and only if
\[ \Sh(X) \models \forall s\?\O_X\_ (\speak{$s$ \inv} \Rightarrow s\in\I)
\Rightarrow s\in\I. \]
\item The reflection of~$\I$ in the poset of quasicoherent radical ideals is
the sheaf~$\overline{\I}$ given by the internal expression
\[ \overline{\I} \defeq \{ s\?\O_X \,|\, \speak{$s$ \inv} \Rightarrow s\in\I
\}. \]
\end{enumerate}
\end{prop}
\begin{proof}Both claims can be verified by purely internal reasoning. The
first claim is a straightforward calculation using the characterization given in
Corollary~\ref{cor:submodule-qcoh}. We discuss the second one in more detail.

Firstly, it's obvious that~$\overline{\I}$ contains~$\I$ and
that~$\overline{\I}$ is a radical ideal. To verify that~$\overline{\I}$ is
quasicoherent, let~$s\?\O_X$ be given such that, if~$s$ were invertible,
then~$s$ would be an element of~$\overline{\I}$. Symbolically, we have
\[ \speak{$s$ \inv} \Longrightarrow (\speak{$s$ \inv} \Rightarrow s\in\I), \]
which of course implies
\[ \speak{$s$ \inv} \Longrightarrow s\in\I. \]
This is precisely the condition for~$s$ to be an element of~$\overline{I}$.

To verify that the construction~$\I \mapsto \overline{\I}$ is really left
adjoint to the inclusion, let a quasicoherent radical ideal~$\J$ be given such
that~$\I \subseteq \J$. We have to show that~$\overline{\I} \subseteq \J$. This
is straightforward.
\end{proof}

For arbitrary~$\O_X$-algebras~$\A$, the description of the quasicoherator for
radical ideals of~$\A$ is more involved, but still sufficient for the
applications in Section~\ref{sect:relative-spectrum}.

\begin{prop}\label{prop:quasicoherator-arbitrary-algebra}
Let~$X$ be a scheme. Let~$\A$ be a quasicoherent~$\O_X$-algebra.
Then the reflection of a radical ideal~$\I \subseteq \A$ in the poset of
quasicoherent radical ideals of~$\A$ is given by the internal expression
\[ \overline{\I} \defeq \bigcup_{n \geq 0} \I_n, \]
where~$(\I_n)$ is the family of radical ideals defined recursively by
\begin{align*}
  \I_0 &\defeq \I, \\
  \I_{n+1} &\defeq \textnormal{the radical ideal generated by
  $\{ fs \,|\, f\?\O_X, s\?\A, (\speak{$f$ \inv} \Rightarrow s \in \I_n) \}$}.
\end{align*}
\end{prop}
\begin{proof}We argue internally. The set~$\overline{\I}$ contains~$\I$ and is
a radical ideal, as an ascending union of radical ideals. To verify
that~$\overline{\I}$ is quasicoherent, let~$f\?\O_X$ and~$s\?\A$ be given such
that, if~$f$ were invertible, then~$s$ would be an element of~$\overline{\I}$.
This means that we have
\[ \speak{$f$ \inv} \Longrightarrow \bigvee_{n \geq 0} s \in \I_n. \]
Since assuming that~$f$ is invertible commutes with directed disjunctions
(Example~\ref{ex:df-locally-compact}), there is a natural number~$n$ such that
\[ \speak{$f$ \inv} \Longrightarrow s \in \I_n. \]
Therefore~$fs \in \I_{n+1} \subseteq \overline{\I}$.

Finally, to verify that the construction~$\I \mapsto \overline{\I}$ is indeed
left adjoint to the inclusion of the quasicoherent radical ideals in all
radical ideals, let a quasicoherent radical ideal~$\J$ be given such that~$\I
\subseteq \J$. By induction we can show that~$\I_n \subseteq \J$ for all
natural numbers~$n$. Therefore~$\overline{\I} \subseteq \J$.
\end{proof}

There is also a purely formal description of the reflector, given by
\[ \I \longmapsto \bigcap \{ \J \subseteq \A \,|\,
  \text{$\J$ is a quasicoherent radical ideal such that~$\I \subseteq \J$} \}. \]
Verifying that this construction has the universal property of
the reflector is straightforward. However, it is not sufficiently concrete for
calculations. In particular, we don't see a way to prove the following
corollary without the explicit description given by the proposition.

\begin{cor}\label{cor:quasicoherator-meet}
Let~$X$ be a scheme. Let~$\A$ be a quasicoherent~$\O_X$-algebra. Let~$\I$
and~$\J$ be radical ideals of~$\A$. Then~$\overline{\I \cap \J} =
\overline{\I} \cap \overline{\J}$.
\end{cor}
\begin{proof}The claim is not purely formal. As a left adjoint, the reflector
preserves suprema; but the claim is that it preserves intersections.

Since the reflector is monotone, it is clear that~$\overline{\I
\cap \J} \subseteq \overline{\I} \cap \overline{\J}$.

To verify the converse
direction, we show by induction that~$\I_n \cap \J_m \subseteq \overline{\I
\cap \J}$ for all natural numbers~$n$ and~$m$. The base case is trivial,
since~$\I_0 \cap \J_0 = \I \cap \J$. For the induction step let~$x \in \I_{n+1}
\cap \J_m$. Then~$x^\ell = \sum_i f_i s_i$ for some natural number~$\ell$ and
elements~$f_i \? \O_X$, $s_i \? \A$ such that~$\speak{$f_i$ \inv} \Rightarrow
s_i \in \I_n$. In particular we have~$\speak{$f_i$ \inv} \Rightarrow s_i x \in
\I_n \cap \J_m$, so by the induction hypothesis~$\speak{$f_i$ \inv} \Rightarrow
s_i x \in \overline{\I \cap \J}$. This implies~$f_i s_i x \in \overline{\I \cap
\J}$, since~$\overline{\I \cap \J}$ is quasicoherent. Therefore~$x^{\ell+1} \in
\overline{\I \cap \J}$ and thus~$x \in \overline{\I \cap \J}$.
\end{proof}

\begin{rem}If in the situation of
Proposition~\ref{prop:quasicoherator-arbitrary-algebra} the algebra~$\A$ is not
quasicoherent, the construction~$\I \mapsto \overline{\I}$ is still left
adjoint to the inclusion of the radical ideal sheaves which satisfy the (then
somewhat unmotivated) internal condition given in
Corollary~\ref{cor:submodule-qcoh} in the poset of all radical ideal sheaves.
Also Corollary~\ref{cor:quasicoherator-meet} remains valid.
This is even the case if~$X$ is an arbitrary ringed and space; in this case,
the proofs of Proposition~\ref{prop:quasicoherator-arbitrary-algebra} and
Corollary~\ref{cor:submodule-qcoh} have to be modified, since then we may not
suppose that assuming that an element of~$\O_X$ is invertible commutes with
directed disjunctions. Instead, the reflector~$\I \mapsto \overline{\I}$ has to
be characterized by
\[ \overline{\I} \defeq \text{smallest fixed point of~$P$ above~$\I$}, \]
where~$P$ is the monotone operator on the set of radical ideals which takes a
radical ideal~$\I$ to the radical ideal generated by~$\{ fs \,|\, f\?\O_X,
s\?\A, (\speak{$f$ \inv} \Rightarrow s \in \I) \}$. The existence of these
fixed points is guaranteed by the Knaster--Tarski theorem, which is
intuitionistically valid in the version we need~\cite{bauer:lumsdaine:bourbaki-witt}.
\end{rem}
% The modified proof of the corollary goes as follows.
% First note that P(I cap J) = P(I) cap P(J).
% Then apply the general lemma that for monotone meet-preserving operators P,
% it holds that mu(P)_{>= x} wedge mu(P)_{>= y} = mu(P)_{>= x wedge y},
% when x and y are post-fixed points and mu(P)_{>= z} denotes the least fixed
% point of P above z.


% FUTURE:
% Think how to prove that modules of finite type are
% quasicoherent and that cokernels are quasicoherent.


\section{Subschemes}

\subsection{Sheaves on open and closed subspaces} It is well-known that sheaves
defined on open or closed subspaces of a topological space~$X$ can be related
with certain sheaves on~$X$, by using appropriate extension functors. We can
define these functors and show their basic properties in the internal
language. Recall from Section~\ref{sect:modalities-geometric-meaning} that we
have defined a formula~``$U$'' for any open subset~$U \subseteq X$ such that
$V \models U$ if and only if $V \subseteq U$.

\begin{lemma}\label{lemma:extension-by-empty-set}
Let~$X$ be a topological space. Let~$j : U \hookrightarrow X$ be the inclusion
of an open subspace. Then there is a canonical functor~$j_! : \Sh(U) \to
\Sh(X)$ called \emph{extension by the empty set} with the following properties:
\begin{enumerate}
\item The functor~$j_!$ is left adjoint to the restriction functor~$j^{-1} : \Sh(X) \to
\Sh(U)$.
\item The composition~$j^{-1} \circ j_! : \Sh(U) \to \Sh(U)$ is (canonically
isomorphic to) the identity.
\item The essential image of~$j_!$ consists of exactly those sheaves on~$X$
whose stalks are empty at all points of~$U^c$. For those sheaves~$\F$, it holds
that~$j_!j^{-1}\F \cong \F$ (canonically).
\end{enumerate}
\end{lemma}
\begin{proof}Internally, for a set~$\F$, we can define~$j_!(\F)$ simply be the
set comprehension
\[ j_!(\F) \defeq \{ x\?\F \,|\, U \}. \]
Externally, the sections of the thusly defined sheaf on an open subset~$V
\subseteq X$ are given by~$\{ x \in \Gamma(V,\F) \,|\, V \subseteq U \}$,
\ie the whole of~$\Gamma(V,\F)$ if~$V \subseteq U$ and the empty set otherwise.
With this short internal description, all of the stated properties can be
easily verified in the internal language.

For instance, recall that internally the functor~$j^{-1}$ is given by
sheafifying with respect to the modal operator~$\Box \defequiv (U \Rightarrow
\placeholder)$. Thus, to show the second statement, we have to give a
bijection~$(j_!(\F))^{++} \to \F$ for any~$\Box$-sheaf~$\F$. (This map has to
be given explicitly, to not only show a weaker statement about a local
isomorphism -- see Section~\ref{sect:internal-constructions}). To this end, we can use the composition
\[ (j_!(\F))^{++} \lhra \F^{++} \stackrel{({\cong})^{-1}}{\lra} \F, \]
where the first map is injective since sheafifying is exact. It is also
surjective, since the~$\Box$-translation of the statement~$\speak{$j_!(\F) \to
\F$ is surjective}$ holds: For any element~$x\?\F$, it holds
that~$\Box(\speak{$x$ possesses a preimage})$.

For the third property, note that a sheaf~$\F$ on~$X$ fulfills the stated
condition on stalks if and only if, from the internal perspective, it holds
that~$U \Rightarrow \speak{$\F$ is inhabited}$. We omit further details.
\end{proof}

\begin{lemma}\label{lemma:extension-by-zero}
Let~$X$ be a ringed space. Let~$j : U \hookrightarrow X$ be the inclusion
of an open subspace. Then there is a canonical functor~$j_! : \Mod_U(\O_U) \to
\Mod_X(\O_X)$ called \emph{extension by zero} with the following properties:
\begin{enumerate}
\item The functor~$j_!$ is left adjoint to the restriction functor~$j^{-1} :
\Mod_X(\O_X) \to \Mod_U(\O_U)$.
\item The composition~$j^{-1} \circ j_! : \Mod_U(\O_U) \to \Mod_U(\O_U)$ is (canonically
isomorphic to) the identity.
\item The essential image of~$j_!$ consists of exactly those~$\O_X$-modules
whose stalks are zero at all points of~$U^c$. For those sheaves~$\F$, it holds
that~$j_!j^{-1}\F \cong \F$ (canonically).
\end{enumerate}
\end{lemma}
\begin{proof}Internally, a sheaf of modules on~$\O_U$ is simply a module
on~$\O_X^{++}$ which is a~$\Box$-sheaf, where~$\Box \defequiv (U \Rightarrow
\placeholder)$. The suitable internal definition for the extension by zero of
such a module~$\F$ is
\[ j_!(\F) \defeq \{ x\?\F \,|\, (x = 0) \vee U \}. \]
With this description, all necessary verifications are easy. Note that
an~$\O_X$-module~$\F$ fulfills the stated condition on stalks if and only if
internally, it holds that~$\forall x\?\F\_ ((x = 0) \vee U)$.
\end{proof}

\begin{lemma}\label{lemma:essim-closed-immersion}
Let~$X$ be a topological space. Let~$i : A \hookrightarrow X$ be the inclusion
of a closed subspace. The essential image of the
inclusion~$i_* : \Sh(A) \to \Sh(X)$ consists of exactly those sheaves whose support
is a subset of~$A$. For those sheaves~$\F$, it holds that~$i_* i^{-1} \F \cong \F$
(canonically).\end{lemma}
\begin{proof}Recall that the modal operator associated to~$A$ is~$\Box\varphi
\defequiv (\varphi \vee A^c)$, and that by Section~\ref{sect:internal-sheaves} the
essential image of~$i_*$ consists of exactly those sheaves which
are~$\Box$-sheaves from the internal perspective. Let~$\F$ be a sheaf on~$X$.
Then it holds that
\[ \supp\F \subseteq A \quad\Longleftrightarrow\quad
  A^c \subseteq X \setminus \supp\F \quad\Longleftrightarrow\quad
  A^c \subseteq \Int(X \setminus \supp\F). \]
Since the interior of the complement of~$\supp\F$ can be characterized as the
largest open subset of~$X$ on which the internal statement~``$\F$ is a
singleton'' holds (Remark~\ref{rem:support-sheaf-of-sets}), the condition on
the support is fulfilled if and only if
\[ \Sh(X) \models (A^c \Rightarrow \speak{$\F$ is a singleton}). \]
We thus have to show that this internal condition is equivalent to~$\F$ being
a~$\Box$-sheaf. For the ``if'' direction, assume~$A^c$. Then the empty subset~$S
\subseteq \F$ trivially verifies the condition that~$\Box(\speak{$S$ is a
singleton})$. There thus exists an element~$x\?\F$ (such that~$\Box(x \in S)$).
If we're given a further element~$y\?\F$, it trivially holds that~$\Box(x =
y)$. By~$\Box$-separatedness, it thus follows that~$x = y$. Thus~$\F$ is the
singleton~$\{x\}$. The proof of the ``only if'' direction is similar.

The second statement says that internally, sheafifying a~$\Box$-sheaf with
respect to the modal operator~$\Box$ and then forgetting that the result is
a~$\Box$-sheaf amounts to doing nothing. This is obvious.
\end{proof}

\subsection{Closed subschemes} Let~$X$ be a ringed space. Recall
that an ideal sheaf~$\I \subseteq \O_X$ defines a closed subset~$V(\I) = \{ x
\in X \,|\, \I_x \neq (1) \subseteq \O_{X,x} \}$, a sheaf of
rings~$\O_X/\I$, and a ringed space~$(V(\I), \O_{V(\I)})$ where~$\O_{V(\I)}$ is
the pullback of~$\O_X/\I$ to~$V(\I)$. In the internal universe, we can
reify~$V(\I)$ by giving a modal operator~$\Box$ such that externally, the
subspace~$X_\Box$ coincides with~$V(\I)$.

\begin{prop}\label{prop:basics-closed-subspace}
Let~$X$ be a ringed space. Let~$\I \subseteq \O_X$ be an ideal
sheaf. Then:
\begin{enumerate}
\item The subspace of~$X$ associated to the modal operator~$\Box$ defined
by~$\Box\varphi \defequiv (\varphi \vee (1 \in \I))$ is~$V(\I)$.
\item The support of~$\O_X/\I$ is exactly~$V(\I)$.
\item The canonical morphism~$i : V(\I) \to X$ is a closed immersion
of ringed spaces.
\end{enumerate}\end{prop}
\begin{proof}For any open subset~$U \subseteq X$, it holds that~$U \models 1
\in \I$ if and only if~$U \subseteq D(\I) = X \setminus V(\I)$. Thus~$D(\I)$
can be characterized as the largest open subset on which~``$1 \in \I$'' holds.
According to Table~\ref{table:nuclei} on page~\pageref{table:nuclei}, the
stated modal operator thus defines the subspace~$D(\I)^c$, \ie~$V(\I)$.

For the second statement, note that since~$\O_X/\I$ is a sheaf of rings, its
support is closed. Therefore the largest open subset of~$X$ where the internal
statement~``$\O_X/\I = 0$'' holds is the complement of the support
(Proposition~\ref{prop:characterization-support}). Since~$D(\I)$ is the largest
open subset where the internal statement~``$\I = (1)$'' holds, it suffices to
show that internally,~$\O_X/\I = 0$ if and only if~$\I = (1)$. This is obvious.

The topological part of the third statement is clear. For the ring-theoretic
part, we have to show that the canonical ring homomorphism~$\O_X \to i_*
\O_{V(\I)}$, that is the canonical projection~$\O_X \to \O_X/(\I)$, is an
epimorphism of sheaves. This is obvious.
\end{proof}

By Lemma~\ref{lemma:essim-closed-immersion}, the sheaf~$\O_X/\I$ is
thus a~$\Box$-sheaf from the internal perspective.

\begin{prop}Let~$X$ be a locally ringed space. Let~$\I \subseteq \O_X$ be an
ideal sheaf. Then the ringed space~$(V(\I), \O_{V(\I)})$ is too locally
ringed.\end{prop}
\begin{proof}We have to show that
\[ \Sh(V(\I)) \models \speak{$\O_{V(\I)}$ is a local ring}. \]
By Theorem~\ref{thm:box-translation-semantically}, this is equivalent to
\[ \Sh(X) \models (\speak{$\O_X/\I$ is a local ring})^\Box, \]
where~$\Box$ is the modal operator given by~$\Box\varphi \defequiv (\varphi \vee
(1 \in \I))$. We therefore have to give an intuitionistic proof of the fact
\[ \forall x,y\?\O_X/\I\_ \speak{$x+y$ \inv} \Longrightarrow
  \Box(\speak{$x$ \inv} \vee \speak{$y$ \inv}). \]
So let~$x = [s], y = [t] \? \O_X/\I$ such that~$x + y$ is invertible
in~$\O_X/\I$. This means that there exists~$u\?\O_X$ and~$v\?\I$ such that~$us
+ ut + v = 1$ in~$\O_X$. Since~$\O_X$ is a local ring, it follows
that~$us$,~$ut$, or~$v$ is invertible. In the first two cases, it follows
that~$x$ respectively~$y$ are invertible in~$\O_X/\I$. In the third case, it
follows that~$1 \in \I$ and thus any boxed statement is trivially true.
\end{proof}

If~$X$ is a scheme and~$\I \subseteq \O_X$ is an ideal sheaf, it is well-known
that the locally ringed space~$V(\I)$ is a scheme if and only if~$\I$ is
quasicoherent. We cannot give an internal proof of this fact since we lack an
internal characterization of being a scheme.

\begin{lemma}\label{lemma:closed-subspace-reduced}
Let~$X$ be a scheme (or ringed space). Let~$\I \subseteq \O_X$ be
an ideal sheaf. The ringed space~$V(\I)$ is reduced if and only if, from the
internal perspective of~$\Sh(X)$, the ideal~$\I$ is a radical ideal.\end{lemma}
\begin{proof}The following chain of equivalences holds:
\begin{align*}
  &\ \Sh(V(\I)) \models \speak{$\O_{V(\I)}$ is a reduced ring} \\
  \Longleftrightarrow&\
    \Sh(V(\I)) \models \bigwedge_{n \geq 0} \forall s\?\O_{V(\I)}\_
      s^n = 0 \Longrightarrow s = 0 \\
  \Longleftrightarrow&\
    \Sh(X) \models \bigl(\bigwedge_{n \geq 0} \forall s\?\O_X/\I\_ s^n = 0
    \Rightarrow s = 0\bigr)^\Box \\
  \Longleftrightarrow&\
    \Sh(X) \models \bigwedge_{n \geq 0} \forall s\?\O_X/\I\_ s^n = 0 \Rightarrow \Box(s = 0) \\
  \Longleftrightarrow&\
    \Sh(X) \models \bigwedge_{n \geq 0} \forall s\?\O_X\_ s^n \in \I
    \Rightarrow \Box(s \in \I) \\
  \Longleftrightarrow&\
    \Sh(X) \models \bigwedge_{n \geq 0} \forall s\?\O_X\_ s^n \in \I
    \Rightarrow s \in \I \\
  \Longleftrightarrow&\
    \Sh(X) \models \speak{$\I$ is a radical ideal}
\end{align*}
In the second-to-last step, we used that~$\Box(s \in \I) \equiv ((s \in \I) \vee
(1 \in \I))$ implies~$s \in \I$. This is trivial in both cases of the
disjunction.
\end{proof}

\begin{lemma}\label{lemma:reduced-subspace}
Let~$X$ be a scheme (or ringed space).
\begin{enumerate}
\item There exists a reduced closed sub-ringed space~$X_\mathrm{red}
\hookrightarrow X$ having the same underlying topological space as~$X$ with
the following universal property: Any morphism~$Y \to X$
of (ringed or locally ringed) spaces such that~$Y$ is reduced factors uniquely
over the closed immersion~$X_\mathrm{red} \hookrightarrow X$.
\item Let~$A \subseteq X$ be a closed subset. Then there exists a structure of
a reduced closed ringed subspace on~$A$ with a similar universal
property.
\end{enumerate}
\end{lemma}
\begin{proof}For the first statement, let~$\N \subseteq \O_X$ be the nilradical
of~$\O_X$. This can internally be simply defined by~$\N \defeq \sqrt{(0)} = \{
s\?\O_X \,|\, \bigvee_{n \geq 0} s^n = 0 \}$. Define~$X_\mathrm{red}$ as the closed
subspace associated to this ideal sheaf. This ringed space is reduced by the
previous lemma. If~$X$ is a scheme, then quasicoherence of~$\N$ (which is
necessary and sufficient for~$X_\mathrm{red}$ to be a scheme) can be shown
internally (Example~\ref{ex:radical-qcoh}).
The proof of the universal property can also be done in the
internal language, by using that the well-known fact of locale theory that the
category of locales over~$X$ is equivalent to internal locales in~$\Sh(X)$; but
we do not want to discuss this further.

For the second statement, internally define the ideal~$\I \defeq \sqrt{\{ s\?\O_X \,|\, s = 0 \vee
A^c \}} \subseteq \O_X$. Then~$1 \in \I$ if and only if~$A^c$, thus by
Proposition~\ref{prop:basics-closed-subspace} the closed ringed subspace defined
by~$\I$ has~$A$ as underlying topological space. It is reduced since~$\I$ is a
radical ideal. \XXX{is~$\I$ quasicoherent, if~$X$ is a scheme?}\end{proof}

\begin{lemma}Let~$X$ be a scheme of dimension~$\leq n$. Let~$V(\I)
\hookrightarrow X$ be a closed subscheme which is locally cut out by a regular
equation. Then~$\dim V(\I) \leq n-1$.\end{lemma}
\begin{proof}By Proposition~\ref{prop:dimension-scheme-ox}, it suffices to give
an intuitionistic proof of the following fact of dimension theory: Let~$A$ be
an arbitrary ring of dimension~$\leq n$. Let~$I = (s) \subseteq A$ be an ideal
which is generated by a regular element~$s\?A$. Then the~$\Box$-translation
of~``$A/I$ is of dimension~$\leq n-1$'' holds. In fact, we can show that~$A/I$
really is of dimension~$\leq n-1$; since no implication signs occur in a formal rendering of ``being of dimension~$\leq n-1$'',
Lemma~\ref{lemma:open-stalk} is applicable and implies
that this a stronger statement.

For this, let a sequence~$([a_0],\ldots,[a_{n-1}])$ of elements in~$A/I$ be
given. We can lift and extend this sequence to the
sequence~$(a_0,\ldots,a_{n-1},s)$ of elements of~$A$. Since~$\dim A \leq n$,
there exists a complementary sequence~$(b_0,\ldots,b_{n-1},b_n)$.
Since~$s$ is regular, the inclusion~$\sqrt{(s b_n)} \subseteq \sqrt{(0)}$
given by the definition of complementarity implies that~$b_n$ is nilpotent.
Thus we have that~$\sqrt{(a_{n-1}b_{n-1})} \subseteq \sqrt{(s,b_n)} =
\sqrt{(s)}$ in~$A$, which translates to~$\sqrt{([a_{n-1}] [b_{n-1}])} \subseteq
\sqrt{(0)}$ in~$A/I$.  Therefore~$([b_0],\ldots,[b_{n-1}])$ is a complementary
sequence to~$([a_0],\ldots,[a_{n-1}])$ in~$A/I$.
\end{proof}

\begin{lemma}\label{lemma:dim-closed-subscheme}
Let~$X$ be a scheme. Let~$\I$ be a sheaf of~$\O_X$-modules. Then:
\[ \dim V(\I) \leq n \quad\Longleftrightarrow\quad
  \Sh(X) \models \speak{$\O_X/\I$ is of Krull dimension~$\leq n$}. \]
\end{lemma}
\begin{proof}By Proposition~\ref{prop:dimension-scheme-ox}, the condition~$\dim
V(\I) \leq n$ is equivalent to
\[ \Sh(V(\I)) \models \speak{$\O_{V(\I)}$ is of Krull dimension~$\leq n$}. \]
By Theorem~\ref{thm:box-translation-semantically} this is equivalent to
\[ \Sh(X) \models (\speak{$\O_X/\I$ is of Krull dimension~$\leq n$})^\Box, \]
where~$\Box$ is the modal operator given by~$\Box\varphi \defeq (\varphi \vee
(1\in\I))$. The claimed equivalence then follows by
Lemma~\ref{lemma:open-stalk} (for~``$\Leftarrow$'') and by direct inspection
similar to the proof of Lemma~\ref{lemma:pushforward-finite-type}
(for~``$\Rightarrow$'').
\end{proof}

\begin{itemize}
\item open subschemes
\item Koszul resolution
\end{itemize}


\section{Transfer principles}

Let~$M$ be an~$A$-module. A natural question is how properties of~$M$
relate to properties of the induced quasicoherent sheaf~$M^\sim$
on~$\Spec A$. For instance it is well-known that
\begin{itemize}
\item $M$ is finitely generated iff~$M^\sim$ is of finite type,
\item $M$ is flat over~$A$ iff~$M^\sim$ is flat over~$\O_{\Spec A}$, and
\item $M$ is torsion iff~$M^\sim$ is a torsion sheaf.
\end{itemize}
Using the internal language of the little Zariski topos of~$\Spec A$, we can
give a simple, conceptual, and uniform explanation of these equivalences.
Namely, from the internal point of view, the module~$M^\sim$ is obtained from
the constant sheaf~$\ul{M}$ by localizing at the \emph{generic filter}, a
particular multiplicative subset to be introduced below, and the set~$M$ and
the sheaf~$\ul{M}$ share the same properties (by
Lemma~\ref{lemma:properties-of-constant-sheaves} below).

This makes it obvious that, for instance, properties which are stable under
localization pass from~$M$ to~$M^\sim$.


\subsection{Internal properties of constant sheaves}

\begin{lemma}\label{lemma:properties-of-constant-sheaves}Let~$\varphi$ be a
formula in which arbitrary sets and elements may occur as parameters. Let~$X$
be a topological space and let~$U \subseteq X$ be an open subset. Then
\[ U \models \varphi \quad\text{iff}\quad (\text{$U$ inhabited} \Rightarrow
\varphi). \]
\end{lemma}
Note that we are abusing notation on the left hand side: The parameters
of~$\varphi$, which are sets and elements, must be read as the induced constant
sheaves and constant functions (sections of that sheaves). Possible
% XXX: "auftretend"
unbounded quantifiers have to be read as ranging only over locally constant
sheaves, not all sheaves.
\begin{proof}By induction on the structure of~$\varphi$. By way of example, we
give the argument in the case that~$\varphi \equiv (a = b)$, where~$a$ and~$b$ are
elements of some set~$M$. Then~$U \models \varphi$ means by definition that the
constant functions~$U \to M$ with value~$a$ respectively~$b$ coincide. This is
equivalent to saying that~$a$ and~$b$ coincide if~$U$ is inhabited.
\end{proof}

The lemma in particular implies that constant sheaves enjoy several
classical properties from the internal point of view, even though the internal
language only supports intuitionistic reasoning in general. For instance, for a
constant sheaf~$\ul{M}$ it holds that
\[ \Sh(X) \models \forall x,y\?\ul{M}\_ x = y \vee x \neq y
  \quad\text{and}\quad
  \Sh(X) \models \forall x,y\?\ul{M}\_ \neg\neg(x = y) \Rightarrow x = y. \]


\subsection{The generic filter}

Let~$A$ be a ring.

\begin{defn}A \emph{filter} in~$A$ is a subset~$F \subseteq A$ such that
\begin{itemize}
\item $0 \not\in F$,
\item $1 \in F$,
\item $x + y \in F \Longrightarrow (x \in F) \vee (y \in F)$, and
\item $xy \in F \Longleftrightarrow (x \in F) \wedge (y \in F)$
\end{itemize}
for all~$x,y \? A$.
\end{defn}

In classical logic, the complement of a prime ideal is a filter and
furthermore every filter is of such a form. In constructive mathematics however,
it is useful to axiomatize complements of prime ideals directly, avoiding
negations.

A filter is in particular a multiplicative subset. Inverting the
elements of a filter results in a local ring, while intuitionistically
the localization of a ring at a prime ideal might not be local.

\begin{defn}The \emph{generic filter}~$\F$ is the subsheaf of~$\ul{A}$
on~$\Spec A$ given by
\[ \Gamma(U, \F) \defeq \{ f : U \to A \,|\,
  \text{$f(\ppp) \not\in \ppp$ for all $\ppp \in U$} \}. \]
\end{defn}

\begin{prop}\label{prop:basics-univ-filter}\ \begin{enumerate}
\item Let~$f \in A$ and~$x \in A$. Then~$D(f) \models x \in \F$ if and only
if~$f \in \sqrt{(x)}$.
% If the numbering is changed, also update reference below.
\item The stalk~$\F_\ppp$ at a point~$\ppp \in \Spec A$
is in canonical bijection with~$A \setminus \ppp$.
\item From the internal point of view of~$\Sh(\Spec A)$, the generic
filter is indeed a filter in~$\ul{A}$.
\end{enumerate}
\end{prop}
\begin{proof}By definition~$D(f) \models x \in \F$ means that~$x \not\in \ppp$
for all prime ideals~$\ppp$ with~$f \not\in \ppp$. This is well-known to be
equivalent to~$f \in \sqrt{(x)}$.

For the claim about stalks, note that the canonical map~$\F_\ppp \to A
\setminus \ppp$ sending a germ~$[f]$ to~$f(\ppp)$ is invertible with inverse
being the map which sends an element~$x \not\in \ppp$ to the germ of the
constant function with value~$x$ (defined on~$D(x)$).

Regarding the third statement we only verify the axiom regarding sums, the
other verifications being easier. Interpreting this axiom with the Kripke--Joyal
semantics and restricting without loss of generality to open subsets where
given locally constant functions are constant, let elements~$x,y \in A$ be
given such that~$D(f) \models x+y \in \F$. By the first statement~$f \in
\sqrt{(x+y)}$. Therefore~$D(f) \subseteq D(x) \cup D(y)$, and on~$D(x)$ it
holds that~$x \in \F$ and on~$D(y)$ it holds that~$y \in \F$.
\end{proof}

The significance of the generic filter is given by the following proposition.
\begin{prop}\label{prop:tilde-construction-internally}
From the internal point of view of~$\Sh(\Spec A)$, \ldots
\begin{enumerate}
\item the structure
sheaf~$\O_{\Spec A}$ is the localization of the constant sheaf~$\ul{A}$ away from the
generic filter:~$\O_{\Spec A} = \ul{A}[\F^{-1}]$, and
\item the quasicoherent sheaf of modules~$M^\sim$ associated to
an~$A$-module~$M$ is the localization of the constant sheaf~$\ul{M}$ away from
the generic filter.
\end{enumerate}\end{prop}
\begin{proof}Ignoring the ring respectively module structure, the second
statement is more general; therefore we prove this one. We didn't discuss the
case of quotients in Section~\ref{sect:internal-constructions}. However it
should be perspicuous that the interpretation of~$\ul{M}[\F^{-1}]$ is defined
as the colimit of~$\E \twoheadrightarrow \ul{M} \times \F$, taken in the
category of sheaves on~$\Spec A$, where~$\E$ is the subsheaf of~$\F \times
(\ul{M} \times \F) \times (\ul{M} \times \F)$ given by~$\E(U) \defeq \{
(s,(x,t),(y,u)) \,|\, sux = sty \}$.

This colimit can be obtained as the sheafification of the similarly defined
presheaf colimit~$\E' \twoheadrightarrow \ul{M}_\mathrm{pre} \times \F$,
where~$\ul{M}_\mathrm{pre}$ is the constant \emph{presheaf} associated to~$M$.
On an open subset~$U$ this presheaf colimit is simply the
localization~$\Gamma(U, \ul{M}_\mathrm{pre})[\Gamma(U, \F)^{-1}] = M[\Gamma(U,
\F)^{-1}]$. In the special case that~$U = D(f)$ is a standard open subset,
Proposition~\ref{prop:basics-univ-filter}(a) shows that this module is
canonically isomorphic to~$M[f^{-1}]$. The quasicoherent sheaf~$M^\sim$ of
modules admits the same description.
\end{proof}
% Here we use the following technical lemma:
% Let E be a presheaf which, restricted to a basis of the topology,
% is even a sheaf. Then its sheafification coincides with the extension of
% the restricted sheaf.

Recognizing~$\O_{\Spec A}$ as a localization of~$\ul{A}$ fits nicely into the
following abstract algebraic motivation for schemes: Does the ring~$A$ admit a
\emph{universal localization}, \ie a homomorphism~$A \to A'$ into a local ring
such that every homomorphism~$A \to B$ into a local ring factors via a local
map over~$A \to A'$? Intuitively speaking, can we localize a ring at all prime
ideals at once, or equivalently away from all filters at once? The answer is \emph{no}
in general,\footnote{Assume that the universal localization~$A'$ of a ring~$A$ exists as an ordinary ring in~$\Set$. Then any
two prime ideals~$\ppp$ and~$\qqq$ of~$A$ are equal: Let~$s \not\in \ppp$.
Since~$s$ is invertible in the local ring~$A_\ppp$ and the map~$A' \to A_\ppp$
induced by~$A \to A_\ppp$ is local, it is also invertible in~$A'$. Therefore the image of~$s$ in~$A_\qqq$ is
invertible as well. Thus~$s \not\in \qqq$.} but always \emph{yes} if we are
willing to change the topos in which we look for a solution: The universal
localization of~$A$ is given by the ring~$\O_{\Spec A}$ in the topos~$\Sh(\Spec
A)$; this ring is constructed by localizing~$\ul{A}$ away from the generic
filter, a filter which exists in~$\Sh(\Spec A)$ but not in~$\Set$.

For transferring properties of~$M^\sim$ to~$M$, the following metatheorem is
crucial.
\begin{prop}\label{prop:metaproperty-of-the-generic-filter}
Let~$\I$ be an ideal in~$\ul{A}$ such that, for all inhabited
open subsets~$U \subseteq \Spec A$ and elements~$x \in A$, the set~$\Gamma(X,\I)$
contains the constant function with value~$x$ if~$\Gamma(U,\I)$ does. Then
\[ D(f) \models \speak{$\I \cap \F$ is inhabited}
  \quad\text{implies}\quad
  \text{for some~$n \geq 0$, $D(f) \models f^n \in \I$.} \]
\end{prop}

Lemma~\ref{lemma:properties-of-constant-sheaves} gives a simple and purely
syntactical criterion for the hypothesis on~$\I$: It suffices for~$\I$ to be
internally defined by an expression of the form~$\{ a\?\ul{A} \,|\, \varphi(a)
\}$, where~$\varphi$ is a formula which refers only to constant sheaves.

The metatheorem reflects the following well-known fact of classical
ring theory: If an ideal meets every filter (that is, the complement of every
prime ideal), it is the unit ideal. In this formulation the statement can't be
proven intuitionistically; the occurence of \emph{every filter} has to be
replaced by the \emph{generic filter}. Intuitively, the generic filter is a
reification of the abstract idea of an ``arbitrary filter'', a filter about
which nothing is known except that it satisfies the filter axioms.

\begin{proof}
Let~$D(f) \models \speak{$\I \cap \F$ is inhabited}$. Then there
exists an open cover~$D(f) = \bigcup_i D(f_i)$ and elements~$x_i \in A$ such
that~$D(f_i) \models x_i \in \F$ and~$D(f_i) \models x_i \in \I$. By
Proposition~\ref{prop:basics-univ-filter} we have that~$f_i \in \sqrt{(x_i)}$
and therefore~$D(f_i) \models f_i^{m_i} \in \I$ for some~$m_i \geq 0$. We may
assume that all the~$D(f_i)$ are inhabited and that the exponents~$m_i$ are all
equal to some number~$m$. The assumption on~$\I$ implies~$D(f) \models f_i^m
\in \I$ for all~$i$. By a standard argument we can write~$f^n = \sum_i a_i
f_i^m$ for some coefficients~$a_i$; thus~$D(f) \models f^n \in \I$.
\end{proof}

\begin{rem}The stronger statement
\[ D(f) \models (\speak{$\I \cap \F$ is inhabited} \Rightarrow \bigvee_{n \geq
0} (f^n \in \I)) \]
does not hold in general. Indeed, consider the example~$f \defeq 1$ and~$\I \defeq
\brak{(g)} \defeq \brak{\{ a\?\ul{A} \,|\, \exists b\?\ul{A}\_ a = bg \}}$,
where~$g$ is a fixed element of~$A$ which is not nilpotent and not invertible.
Since~$D(g) \models g \in \I \cap \F$, the stronger statement would imply~$D(g)
\models 1 \in \I$. By Lemma~\ref{lemma:properties-of-constant-sheaves}, this is
equivalent to~$g$ being invertible in~$A$.
\end{rem}

\begin{rem}Recall from Proposition~\ref{prop:kx-internally} that the
sheaf~$\K_{\Spec A}$ of rational functions can internally by obtained by
localizing~$\O_{\Spec A}$ at the set of regular elements. Since~$\O_{\Spec A}$
is itself a localization, the sheaf~$\K_{\Spec A}$ is therefore obtained by a
two-step process. It can also be obtained in a single step by
localizing~$\ul{A}$ at~$\T$, where~$\T$ is the subsheaf of~$\ul{A}$ defined
by
\[ \Gamma(U,\T) = \{ f : U \to A \,|\, \text{$f(\ppp)$ is regular in~$A_\ppp$
for all~$\ppp \in U$} \}. \]
This subsheaf is characterized by the property that, for all~$f \in A$ and~$x
\in A$,~$D(f) \models x \in \T$ if and only if~$x$ is regular in~$A[f^{-1}]$.
\end{rem}


\subsection{Internal proofs of common lemmas}
\label{sect:common-lemmas-transfer-principles}

\begin{lemma}Let~$A$ be a ring. Then~$A$ is reduced if and only if the
scheme~$\Spec A$ is reduced.\end{lemma}
\begin{proof}By Proposition~\ref{prop:reduced-ring} the scheme~$\Spec A$ is
reduced if and only if~$\O_{\Spec A}$ is a reduced ring
from the internal point of view of~$\Sh(\Spec A)$.

For the ``only if'' direction assume that~$A$ is reduced. Then~$\ul{A}$ is
reduced as well, by Lemma~\ref{lemma:properties-of-constant-sheaves}. Since
localizations of reduced rings are reduced (and this fact has an intuitionistic
proof), in particular~$\O_{\Spec A} = \ul{A}[\F^{-1}]$ is reduced.

For the ``if'' direction let~$x \in A$ be an element such that~$x^n = 0$.
Since~$\O_{\Spec A} = \ul{A}[\F^{-1}]$ is reduced from the internal point of
view, the element~$x$ is zero in that ring, that is
\[ \Sh(\Spec A) \models \exists s\?\F\_ sx = 0. \]
Therefore the ideal internally defined by
\[ \I \defeq \{ a\?\ul{A} \,|\, ax = 0 \} \]
meets the generic filter. By
Proposition~\ref{prop:metaproperty-of-the-generic-filter} it follows
that~$\Sh(\Spec A) \models 1 \in \I$. By
Lemma~\ref{lemma:properties-of-constant-sheaves} this is equivalent to~$1 \cdot
x = 0$ as elements of~$A$.
\end{proof}

Note that the ``if'' direction also admits a shorter proof, by simply
considering the Kripke--Joyal interpretation of~$\Sh(\Spec A) \models
\speak{$\O_{\Spec A}$ is reduced}$ and using~$\Gamma(\Spec A, \O_{\Spec A})
\cong A$. We included the given proof to give a simple example of the mixed
internal/external reasoning with the generic filter. In a similar way we could
reprove Lemma~\ref{lemma:regular-affine}, that is the statement that a ring
element~$f \in A$ is regular in~$A$ if and only if, from the internal point of
view, it is regular in~$\O_{\Spec A}$.

\begin{lemma}Let~$M$ be an~$A$-module. Then~$M^\sim$ is of finite type if and
only if~$M$ is finitely generated.\end{lemma}
\begin{proof}First assume that~$M$ is finitely generated over~$A$.
Then~$\ul{M}$ is finitely generated over~$\ul{A}$, by
Lemma~\ref{lemma:properties-of-constant-sheaves}. Since localizations of
finitely generated modules are finitely generated (over the localized ring),
the module~$M^\sim = \ul{M}[\F^{-1}]$ is finitely generated from the internal
point of view. By Proposition~\ref{prop:finite-type-and-co} this means
that~$M^\sim$ is of finite type from the external point of view.

For the ``only if`` direction, we assume that~$M^\sim$ is finitely generated
over~$\O_{\Spec A}$ from the internal point of view and have to verify that~$M$
is finitely generated over~$A$. So it holds that
\[ \Sh(\Spec A) \models \bigvee_{n \geq 0}
  \exists x_1,\ldots,x_n\?\ul{M}[\F^{-1}]\_
  \speak{the $x_i$ span~$\ul{M}[\F^{-1}]$ over~$\ul{A}[\F^{-1}]$}. \]
Since multiplying a generating family by an unit results again in a generating
family, we have in fact that
\[ \Sh(\Spec A) \models \bigvee_{n \geq 0}
  \exists x_1,\ldots,x_n\?\ul{M}\_
  \speak{the $x_i/1$ span~$\ul{M}[\F^{-1}]$ over~$\ul{A}[\F^{-1}]$} \]
or equivalently
\[ \Sh(\Spec A) \models \bigvee_{n \geq 0, x_1,\ldots,x_n \in M}
  \speak{the $x_i/1$ span~$\ul{M}[\F^{-1}]$ over~$\ul{A}[\F^{-1}]$}. \]
Since this is a directed disjunction and~$\Spec A$ is quasicompact,
Proposition~\ref{prop:quasicompact-meta} is applicable and shows that there
exists a natural number~$n \geq 0$ and elements~$x_1,\ldots,x_n \in M$ such
that
\[ \Sh(\Spec A) \models \speak{the $x_i/1$ span~$\ul{M}[\F^{-1}]$
over~$\ul{A}[\F^{-1}]$}. \]
We claim that these~$x_i$ also span~$M$ as an~$A$-module. So let~$x \in M$ be
arbitrary. By elementary linear algebra we can deduce that
\[ \Sh(\Spec A) \models \exists s\in\F\_ \exists a_1,\ldots,a_n\?\ul{A}\_
  sx = \sum_i a_i x_i. \]
Therefore the ideal internally defined by
\[ \I \defeq \{ s\?\ul{A} \,|\, \exists a_1,\ldots,a_n\?\ul{A}\_
  sx = \textstyle\sum_i a_i x_i \} \]
meets the generic filter.
Proposition~\ref{prop:metaproperty-of-the-generic-filter} shows that~$\Sh(\Spec A)
\models 1 \in \I$, that is~$x$ is an element of the~$A$-span of the~$x_i$.
\end{proof}

\begin{rem}If~$M^\sim$ can be generated by~$\leq n$ elements over~$\O_{\Spec
A}$ from the internal point of view, it needn't be the case that~$M$ can be
generated by~$\leq n$ elements over~$A$. It is instructive to see where the
appropriately modified version of the above proof fails: In this case we still
have
\[ \Sh(\Spec A) \models \bigvee_{x_1,\ldots,x_n \in M}
  \speak{the $x_i/1$ span~$\ul{M}[\F^{-1}]$ over~$\ul{A}[\F^{-1}]$}, \]
but this disjunction is no longer directed.
\end{rem}


\subsection{An application to constructive mathematics}
\label{sect:eliminating-prime-ideals}

The generic filter has a practical application in constructive mathematics.
Recall that intuitionistically prime and maximal ideals don't work very well,
since one often needs the axiom of choice or related set-theoretical principles
in dealing with them. This is unfortunate, since prime and maximal ideals are
very useful in some situations. For example:
\begin{itemize}
\item To verify that a ring element is nilpotent, it suffices to verify that it
is an element of every prime ideal. For instance, this is calculationally simpler
when proving that the coefficients of a nilpotent polynomial are
themselves nilpotent.
\item To verify that there is an relation of the form~$1 = p_1f_1 + \cdots +
p_mf_m$ among polynomials~$f_1,\ldots,f_m \in K[X_1,\ldots,X_n]$ where~$K$ is
an algebraically closed field, it suffices to show that the~$f_i$ don't have a
common zero.
\end{itemize}

One could of course simply switch to classical logic in this case. However this
might not be desirable, as a constructive proof would contain more information:
For instance, if we have classically proven that an element~$x$ is an element
of every prime ideal, then we know that some power~$x^n$ is zero. But from such
a proof we can't directly read off any upper bound on~$n$.

There is a way to combine some of the powerful tools of classical ring theory
with the advantages that constructive reasoning provides. Namely we can devise
a language in which we can usefully talk about prime ideals, but which
substitutes all non-constructive arguments by constructive arguments ``behind
the scenes''. The key idea is to substitute the phrase ``for all prime ideals''
(or equivalently ``for all filters'') by ``for the generic filter''.

This was already explored by Coquand, Coste, Lombardi, Roy, and
others under the theme of \emph{dynamical methods in
algebra}~\cite{clr:dynamicalmethod,cl:logical}. Here we show how one can use
the generic filter, as reified by a sheaf in the little Zariski topos, to
achieve similar effects.

\begin{prop}Let~$M$ and~$N$ be~$A$-modules. Let~$\alpha : M \to N$ be a linear
map. The interpretations of the statements in the second column of
Table~\ref{table:generic-filter-statements} in the internal language
of~$\Sh(\Spec A)$ are intuitionistically equivalent to the statements given in
the third column.\end{prop}
\begin{proof}To demonstrate the technique we verify the first and the last claim.
To make the following proofs constructive we would have to define~$\Spec A$
and its sheaf topos in a constructive fashion, not using prime ideals. This can
be done, by constructing~$\Spec A$ as a locale instead of a topological space
(see for instance Section~\ref{sect:spectrum-as-a-locale}
and~\cite[p.~743f.]{wraith:generic-galois-theory}),
but we won't discuss details here.

The interpretation of~$\Sh(\Spec A) \models x \not\in \F$ by the Kripke--Joyal
semantics is that~$D(f) \models x \in \F$ implies~$D(f) = \emptyset$ for
all~$f \in A$. By Proposition~\ref{prop:basics-univ-filter}(a) this is
equivalent to
\[ \forall f \in A\_ f \in \sqrt{(x)} \Rightarrow f \in \sqrt{(0)}, \]
that is the statement that~$x$ is nilpotent in~$A$.

Assume that~$\alpha : M \to N$ is surjective. By
Lemma~\ref{lemma:properties-of-constant-sheaves} the induced map~$\ul{M}
\to \ul{N}$ is surjective from the internal point of view. Since localization
preserves surjectivity, also the map~$\ul{M}[\F^{-1}] \to \ul{N}[\F^{-1}]$ is
surjective.

Conversely, assume that~$\ul{M}[\F^{-1}] \to \ul{N}[\F^{-1}]$ is surjective
from the internal point of view. To verify that~$\alpha : M \to N$ is
surjective, let~$y \in N$. The assumption implies that the ideal internally
defined by
\[ \I \defeq \{ s\?\ul{A} \,|\, \exists x\?\ul{A}\_ sy = \ul{\alpha}(x) \} \]
meets the generic filter. By
Proposition~\ref{prop:metaproperty-of-the-generic-filter} this implies
that~$\Sh(\Spec A) \models 1 \in \I$, that is there exists an element~$x \in A$
such that~$\alpha(x) = y$.
\end{proof}

\begin{table}
  \centering
  \renewcommand{\arraystretch}{1.3}
  \small
  % Change wording in proposition above if order changes.
  \begin{tabular}{lll}
    \toprule
    Statement & constructive substitution & meaning \\\midrule
    $x \in \ppp$ for all~$\ppp$. &
    $x \not\in \F$. &
    $x$ is nilpotent. \\
    $x \in \ppp$ for all~$\ppp$ such that~$y \in \ppp$. &
    $x \in \F \Rightarrow y \in \F$. &
    $x \in \sqrt{(y)}$. \\
    $x$ is regular in all stalks~$A_\ppp$. &
    $x$ is regular in~$\ul{A}[\F^{-1}]$. &
    $x$ is regular in~$A$. \\
    The stalks~$A_\ppp$ are reduced. &
    $\ul{A}[\F^{-1}]$ is reduced. &
    $A$ is reduced. \\
    The stalks~$M_\ppp$ vanish. &
    $\ul{M}[\F^{-1}] = 0$. &
    $M = 0$. \\
    The stalks~$M_\ppp$ are fin.\@ gen.\@ over~$A_\ppp$. &
    $\ul{M}[\F^{-1}]$ is fin.\@ gen.\@ over
    $\ul{A}[\F^{-1}]$. &
    $M$ is fin.\@ gen.\@ over~$A$. \\
    The stalks~$M_\ppp$ are flat over~$A_\ppp$. &
    $\ul{M}[\F^{-1}]$ is flat over~$\ul{A}[\F^{-1}]$. &
    $M$ is flat over~$A$. \\
    The maps~$M_\ppp \to N_\ppp$ are injective. &
    $\ul{M}[\F^{-1}] \to \ul{N}[\F^{-1}]$ is injective. &
    $M \to N$ is injective. \\
    The maps~$M_\ppp \to N_\ppp$ are surjective. &
    $\ul{M}[\F^{-1}] \to \ul{N}[\F^{-1}]$ is surjective. &
    $M \to N$ is surjective. \\
    \bottomrule
  \end{tabular}
  \vspace{0.5em}

  \caption{\label{table:generic-filter-statements}Substituting the use of prime
  ideals by the generic filter.}
\end{table}

The sheaf-theoretical approach using the generic filter is different from the
dynamical methods in the following aspect. We have to reword classical
arguments using (the generic) filter instead of (the generic) prime ideal.
Depending on the situation this might be a nuisance. One might be tempted to
employ the complement of the generic filter, but this is only an ideal, not a
prime ideal from the internal point of view.\footnote{One can check that the
complement of~$\F$ in~$\ul{A}$ is the subsheaf~$\P$ defined by~$\Gamma(U, \P)
\defeq \{ f : U \to A \,|\, \text{$f(\ppp) \in \ppp$ for all~$\ppp \in U$} \}$
and that~$D(f) \models x \in \P$ if and only if~$fx$ is nilpotent. This can be
used to show that the statement~$\Sh(\Spec A) \models \forall x,y\?\ul{A}\_ xy
\in \P \Rightarrow x \in \P \vee y \in \P$ is false in general.}
% XXX: Give counterexample.


\section{Relative spectrum}
\label{sect:relative-spectrum}

Recall that if~$\A$ is a quasicoherent~$\O_X$-algebra on a scheme~$X$, one can
construct the \emph{relative spectrum}~$\RelSpec{X}{\A}$ by appropriately
gluing the spectra~$\Spec \Gamma(U,\A)$ where~$U$ ranges over the affine opens
of~$X$. This relative spectrum comes equipped with a canonical
morphism~$\RelSpec{X}{\A} \to X$. We can give a simple construction of the
relative spectrum in the internal language of~$\Sh(X)$, imitating the usual
construction of the spectrum of a ring (as opposed to a sheaf of rings).

\subsection{Internal locales} Let~$X$ be a topological space (or a locale). A
fundamental fact in the theory of locales is that there is a canonical
equivalence between the category of \emph{locales over~$X$} -- that is
locales~$Y$ equipped with a morphism~$Y \to X$ -- and \emph{internal locales
in~$\Sh(X)$}~\cite[p.~49]{johnstone:point}. An internal locale in a topos~$\E$ is given by an object~$L$ of~$\E$
(the internal lattice of opens of the locale) together with a binary
relation~$(\preceq) \hookrightarrow L \times L$ such that the axioms on a
locale hold from the internal point of view. (In these notes, we do not need a
precise wording of these axioms.)

The equivalence is described as follows: A locale~$f : Y \to X$ over~$X$
induces an internal locale~$I(Y)$ with object of opens given by~$\Open(I(Y)) \defeq
f_* \Omega_{\Sh(Y)} \in \Sh(X)$, where~$f_*$ is the pushforward functor
and~$\Omega_{\Sh(Y)}$ is the object of truth values in the topos of sheaves
on~$Y$. Conversely, an internal locale~$\L \in \Sh(X)$ induces an (external)
locale~$E(\L)$ with lattice of opens given by~$\Open(E(\L)) \defeq \Gamma(X,\L)$.
This comes equipped with a canonical morphism~$Y \to X$ of locales which we do
not need to describe explicitly~\cite[Section~C1.6]{johnstone:elephant}.

As a special case, the internalization of the trivial locale~$\id : X \to X$
over~$X$ has as lattice of opens the object~$\id_* \Omega_{\Sh(X)} =
\Omega_{\Sh(X)} = \P(1)$. This is precisely the lattice of opens of the
one-point space. Thus~$I(X) \cong \pt$. This illustrates the intuition
behind working internally in~$\Sh(X)$: From the perspective
of~$\Sh(X)$, the space~$X$ looks like the one-point space (even if in fact it
is not).

One can associate to an internal locale~$L$ in a topos~$\E$ a topos of internal
sheaves on it:~$\Sh_\E(L)$. The correspondence is made in such a way that the topos of
sheaves on a locale~$Y$ over~$X$ is equivalent to the topos of sheaves on the
internal locale~$I(Y)$: $\Sh(Y) \simeq \Sh_{\Sh(X)}(I(Y))$.

There is no similarly nice correspondence between topological spaces
over~$X$ and internal topological spaces
in~$\Sh(X)$~\cite[Corollary~C1.6.7]{johnstone:elephant}. This is one of the
reasons why locales are better suited for working internally and for switching
between internal and external perspectives.


\subsection{The spectrum of a ring as a locale}
\label{sect:spectrum-as-a-locale}
Recall that the spectrum
of a ring~$A$ is usually constructed as the set
\[ \Spec A \defeq \{ \ppp \subseteq A \,|\,
  \text{$\ppp$ is a prime ideal} \} \]
endowed with a certain topology and a sheaf of rings~$\O_{\Spec A}$. From an
intuitionistic (and thus internal) point of view, this construction does not
work well: Prime ideals are intuitionistically much more elusive than
classically, where one can appeal to Zorn's lemma to obtain maximal (and thus
prime) ideals. More to the point, one cannot show that this construction of
the spectrum as a topological space verifies the expected universal property.

On the other hand, the lattice of opens of~$\Spec A$ admits a simple
description not needing the notion of prime ideals:
\[ \Open(\Spec A) \cong \{ \aaa \subseteq A \,|\,
  \text{$\aaa$ is a radical ideal} \}. \]
An open subset~$U \subseteq \Spec A$ corresponds to the radical ideal~$\{ h \in
A \,|\, D(h) \subseteq U \}$ (so in particular, the open subset~$D(f)$
corresponds to the radical ideal~$\sqrt{(f)}$); conversely, a radical ideal~$\aaa$
corresponds to the open subset~$\bigcup_{h \in \aaa} D(h)$.

Thus, in an intuitionistic context, we will construct the spectrum of a ring~$A$
only as a locale, not as a topological space; we will use the lattice of
radical ideals as the lattice of opens. This construction has the expected
universal property, namely that it is adjoint to the global functions functor:
\[ \Hom_{\LRL}(X, \Spec A) \cong \Hom_{\Ring}(A, \Gamma(X,\O_X)). \]
Here, ``$\LRL$'' refers to the category of \emph{locally ringed locales}, \ie
locales~$X$ equipped with a sheaf of rings~$\O_X$ such that from the internal point of
view of~$\Sh(X)$, the ring~$\O_X$ is a local ring. A morphism~$Y \to X$ of
locally ringed locales consists of a locale morphism~$f : Y \to X$ and a
morphism~$f^\sharp : f^{-1} \O_X \to \O_Y$ of sheaves of rings on~$Y$ such that, from the
internal point of view of~$\Sh(Y)$, the ring homomorphism~$f^\sharp$ is a local
homomorphism. The notion of a locally ringed locale is thus a straightforward
generalization of that of a locally ringed space.

The importance of a locale-theoretic approach to spectra of rings, especially in
relative situations, has also been stressed by Lurie~\cite[p.~37]{lurie:dag5}.

For later use, we study the question when the spectrum is the one-point space.
The answer is well-known classically, but since we want to use this result in
an internal context, we have to give an intuitionistic proof.
\begin{lemma}\label{lemma:spectrum-one-point}
Let~$A$ be a ring with~$0 \neq 1$. Its spectrum is a one-point
space (as a locale) if and only if any element of~$A$ is nilpotent or
invertible.\end{lemma}
\begin{proof}The locale~$\Spec A$ is a one-point space if and only if the
canonical map
\[ \begin{array}{@{}rcl@{}}
  \Omega = \P(1) &\longrightarrow& \Open(\Spec A) \\[0.1em]
  \varphi &\longmapsto& \aaa_\varphi \defeq \sup\{\sqrt{(1)} \,|\, \varphi \} =
  \{ x \? A \,|\, \speak{$x$ nilpotent} \vee \varphi \}
\end{array} \]
is bijective. It is always injective: If~$\aaa_\varphi = \aaa_\psi$,
then~$(1 \in \aaa_\varphi) \Leftrightarrow (1 \in \aaa_\psi)$. Since the unit
of~$A$ is not nilpotent, this amounts to~$\varphi \Leftrightarrow \psi$.

The map is surjective if and only if for any radical ideal~$\aaa \subseteq A$,
it holds that~$\aaa = \{ x \? A \,|\, \speak{$x$ nilpotent} \vee \varphi \}$
for some proposition~$\varphi$. By considering the condition~``$1 \in \aaa$'',
it follows that this proposition~$\varphi$ must be equivalent to the
proposition~``$1 \in \aaa$'' (if it is at all possible to write~$\aaa$ in such
a way).

So the map is surjective if and only if for any radical ideal~$\aaa \subseteq
A$ and any element~$x$ of~$A$ it holds that
\[ x \in \aaa \quad\Longleftrightarrow\quad
  \speak{$x$ nilpotent} \vee (1 \in \aaa). \]
The ``if'' direction always holds. If any element of~$A$ is nilpotent or
invertible, the ``only if'' direction holds as well (for any~$\aaa$ and any~$x$).
Considering the radical ideal~$\sqrt{(f)}$ for an element~$f \? A$, one
verifies that the converse holds as well.
\end{proof}


\subsection{The relative spectrum as an ordinary spectrum from the internal
point of view} Let~$X$ be a scheme and~$\A$ be a quasicoherent~$\O_X$-algebra.
Since~$\A$ looks like a plain algebra from the internal perspective
of~$\Sh(X)$, we can consider its internally defined spectrum. This is a locale
internal to~$\Sh(X)$; we might hope that its externalization is precisely the
relative spectrum of~$\A$ (considered as a locale):
\[ E(\Spec \A) \stackrel{?}{\cong} \RelSpec{X}{\A}. \]
However, this turns out to be too naive. The externalization of the internally
defined spectrum has the universal property
\begin{align*}
  \Hom_{\LRL/E(\Spec\O_X)}(Y, E(\Spec\A)) & \cong
  \Hom_{\LRL_{\Sh(X)}/\Spec\O_X}(I(Y), \Spec\A) \\
  & \cong \Hom_{\O_X}(\A, \mu_* \O_Y)
\end{align*}
for all locally ringed locales~$Y$ over~$E(\Spec\O_X)$. Here,~$\mu$ is the
structure morphism~$Y \to \Spec\O_X$, $E(\Spec\O_X)$ is the locally ringed
locale associated to the internally defined spectrum of~$\O_X$,
and~$\LRL_{\Sh(X)}$ is the category of locally ringed locales internal
to~$\Sh(X)$. However, the relative spectrum has the different universal property
\begin{align*}
  \Hom_{\LRL/X}(Y, \RelSpec{X}{\A}) & \cong
  \Hom_{\LRL_{\Sh(X)}/(\pt,\O_X)}(I(Y), I(\RelSpec{X}{\A})) \\
  & \cong \Hom_{\O_X}(\A, \mu_* \O_Y)
\end{align*}
for all locally ringed locales~$Y$ over~$X$. The crucial
difference is that in general, the internally defined locally ringed
locale~$\Spec\O_X$ does \emph{not} coincide with the internal locally ringed
locale~$(\pt,\O_X)$ (which is simply~$(X,\O_X)$ from the external point of
view). More succinctly, the functor~$E \circ \Spec$ is an adjoint to the
global sections functor~$\LRL/E(\Spec\O_X) \to \Alg(\O_X)^\op$, while the
relative spectrum functor is an adjoint to the global sections functor~$\LRL/X
\to \Alg(\O_X)^\op$.
\XXX{check details?}

The internal construction which correctly reflects the relative spectrum is the
following.
\begin{defn}Let~$A$ be an~$R$-algebra which is quasicoherent in the sense that
the condition given in Theorem~\ref{thm:qcoh-sheafchar} is fulfilled. The
lattice of opens of its \emph{quasicoherent spectrum} is given by
\[ \Open(\QcohSpec{R}{A}) \defeq \{ \aaa \subseteq A \,|\,
  \text{$\aaa$ is a quasicoherent radical ideal} \}, \]
where again ``quasicoherent'' is meant as in Theorem~\ref{thm:qcoh-sheafchar}.
If this lattice fulfills the axioms on a locale, we further define a
structure sheaf on a base by
$\Gamma(\sqrt{(f)}, \O_{\QcohSpec{R}{A}}) \defeq A[f^{-1}]$.
\end{defn}
% XXX: maybe use Borceux, p. 431, ex. 6.12.10 to find conditions when
% \QcohSpec{R}{A} is an internal locale

\begin{prop}Let~$X$ be a scheme and~$\A$ a quasicoherent~$\O_X$-algebra. Then:
\begin{enumerate}
\item The internal lattice~$\Open(\QcohSpec{\O_X}{\A})$ fulfills the axioms on an
internal locale.
\item The externalization~$E(\QcohSpec{\O_X}{\A})$ coincides with the relative
spectrum~$\RelSpec{\O_X}{\A}$.
\end{enumerate}
\end{prop}
\begin{proof}We first show that the internalization~$I(\RelSpec{\O_X}{\A})$
coincides with the internal lattice~$\Open(\QcohSpec{\O_X}{\A})$. Both of these
objects are sheaves on~$X$; so to show that they are canonically isomorphic, it
suffices to give compatible canonical isomorphisms on the base of~$X$
consisting of all affine opens. On such an open~$U = \Spec R$ the
quasicoherent algebra~$\A$ is given by a tilde construction:~$\A|_U =
S^{\sim}$ for some~$R$-algebra~$S$. Thus it holds that
\begin{align*}
  \Gamma(U, I(\RelSpec{\O_X}{\A})) &\cong
  \Open(\RelSpec{\O_X}{(\A)} \times_X U) \\
  & \cong
  \Open(\RelSpec{\O_X|_U}{\A|_U}) \\
  & \cong
  \Open(\Spec S) \\
  & \cong
  \{ \aaa \subseteq S \,|\, \text{$\aaa$ is a radical ideal} \}, \\
  \Gamma(U, \Open(\QcohSpec{\O_X}{\A})) &\cong
  \{ \I \hookrightarrow \A|_U \ |\ \text{$\I$ is a quasicoherent radical ideal sheaf} \}.
\end{align*}
Clearly, these lattices are isomorphic, by~$\aaa \mapsto \aaa^\sim$ and~$\I
\mapsto \Gamma(U,\I)$.

This is already enough to prove the claims: By a general lemma of locale
theory~\cite[discussion before Proposition~C1.6.1]{johnstone:elephant},
the internal lattice~$\Open(\QcohSpec{\O_X}{\A})$ defines an internal locale
(as the pushforward under~$\RelSpec{\O_X}{\A} \to X$ of a locale), and its
externalization coincides with~$\RelSpec{\O_X}{\A}$.
\end{proof}

There is also a general spectrum functor by Gillam~\cite{gillam:localization}
which works for arbitrary~$\O_X$-algebras (instead of only quasicoherent
ones). It coincides with the relative spectrum construction (and thus the
externalization of the internal quasicoherent spectrum) on
quasicoherent~$\O_X$-algebras. Furthermore, there is a general spectrum
construction by Hakim~\cite{hakim:relative-schemes} sending ringed toposes to
locally ringed toposes. We hope to study its relationship with the internal
spectrum construction in the future; it may turn out that they coincide.

\begin{prop}Let~$X$ be a scheme. Then~$E(\Spec\O_X) \cong X$ as locales
over~$X$ if and only if~$X$ is the empty scheme or~$X$ has dimension zero.\end{prop}
\begin{proof}The externalization of~$\Spec\O_X$ coincides with~$X$ if and only
if from the internal point of view, the locale~$\Spec\O_X$ coincides with the
one-point locale. By interpreting Lemma~\ref{lemma:spectrum-one-point} in the
internal language of~$\Sh(X)$, it follows that this is the case if and only if
\[ \Sh(X) \models \forall f\?\O_X\_ \speak{$f$ nilpotent} \vee \speak{$f$
invertible}. \]
(Internally, it always holds that~$\neg(1 = 0)$ in~$\O_X$, even if~$X$ happens
to be the empty scheme. Therefore the lemma is indeed applicable.) By
Corollary~\ref{cor:scheme-dimension-zero}, this condition is equivalent to the
dimension of~$X$ being less than or equal to zero, \ie to~$X$ being empty or
having dimension exactly zero.
% Since this condition is a geometric implication, it is fulfilled in the
% internal language if and only if it is fulfilled at every point~$x \in X$.
% This amounts to requiring that in any local ring~$\O_{X,x}$, the unique
% maximal ideal is the only prime ideal, \ie that all local rings~$\O_{X,x}$
% have Krull dimension zero. This is equivalent to~$X$ being empty or having
% dimension zero.
\end{proof}

\begin{cor}Let~$X$ be a scheme. Then the relative spectrum of
quasicoherent~$\O_X$-algebras can be calculated by
the internal spectrum (instead of the internal quasicoherent spectrum) if and
only if~$X$ is empty or zero-dimensional.\end{cor}
\begin{proof}The externalization of the internal spectrum of arbitrary
quasicoherent~$\O_X$-algebras~$\A$ coincides with the relative spectrum if and
only if it coincides in the special case~$\A = \O_X$. This is apparent by the
universal properties of both constructions. Thus the claim follows by the
previous proposition.
\end{proof}


\section{Outlook: the big Zariski topos}
\label{sect:big-zariski}

The preceding sections demonstrated that working in the internal universe of
the little Zariski topos of a scheme~$S$, the topos of sheaves on~$S$, is
useful for simplifying local work on~$S$. The basic tenet was that sheaves of
modules look just like plain modules and that theorems of intuitionistic
algebra yield theorems about sheaves.

But the little Zariski topos is not particularly suited for dealing with
schemes \emph{over~$S$}. For this, we need a related topos.

\begin{defn}The \emph{big Zariski topos}~$\Zar(S)$ of a scheme~$S$ is the
topos of sheaves on the Grothendieck site~$\Sch/S$ of schemes over~$S$.
\end{defn}

Explicitly, an object of~$\Zar(S)$ is a functor~$(\Sch/S)^\op \to \Set$
satisfying the gluing condition with respect to ordinary Zariski coverings:
If~$X = \bigcup_i U_i$ is a cover of an~$S$-scheme~$X$ by open subsets, the
diagram
\[ \F(X) \to \prod_i \F(U_i) \rightrightarrows \prod_{i,j} F(U_i \cap U_j)
\]
should be an equalizer diagram. Just like the topos of sheaves on a topological
space admits an internal language, so does the big Zariski topos; the necessary
modifications of the Kripke--Joyal semantics (Table~\ref{table:kripke-joyal}) are straightforward. For
instance, the rule for implication,
\[ U \models \varphi \Rightarrow \psi \quad\Ll\quad
  \text{for all open~$V \subseteq U$:
  $V \models \varphi$ implies $V \models \psi$}, \]
for open subsets~$U$ of~$S$, has to be rewritten as
\[ X \models \varphi \Rightarrow \psi \quad\Ll\quad
  \text{for all morphisms~$T \to X$ in~$\Sch/S$:
  $T \models \varphi$ implies $T \models \psi$}, \]
for~$S$-schemes~$X$.

The big Zariski topos is the proper context for the ``functor of
points'' approach to algebraic geometry. This is because the functor of points
associated to an~$S$-scheme~$X$,
\[ \ul{X} : \Sch/S \to \Set,\ T \mapsto \Hom_S(T,X), \]
fulfills the gluing condition and is therefore an object of~$\Zar(S)$.

From the internal point of view of~$\Zar(S)$, such a functor~$\ul{X}$ looks like a single
set. It can be pictured as the ``set of points of~$X$'', where~``point'' does
not mean ``point of the underlying topological space of~$X$'', but
rather~``$T$-point of~$X$'', where~$T$ varies over all~$S$-schemes. The
internal language of the big Zariski topos hides any explicit mentions of the
stage~$T$; it is therefore a device for reifying the multitude of points
of~$X$, defined on varying stages, as a single entity.

As a basic example, the functor~$\ul{S}$ looks like a singleton set from the
internal point of view of~$\Zar(S)$ and~$\ul{S \amalg S}$ looks like a
two-element set.

A particularly important object of~$\Zar(S)$ is the functor of points
associated to the affine line over~$S$, the~$S$-scheme~$\AA^1_S \defeq S
\times_{\Spec\ZZ} \ZZ[t]$. Its value on an~$S$-scheme~$T$ is given by
\[ \affl(T) = \Hom_S(T, \AA^1_S) \cong \Hom_{\Spec\ZZ}(T, \Spec\ZZ[t]) \cong
\Gamma(T,\O_T). \]
This object has an evident structure as a ring object in~$\Zar(S)$. In fact,
from the internal point of view, it is a local ring and even a field in the
sense that nonzero elements are invertible.
This is without any assumptions on
the base scheme~$S$; in the case~$S = \Spec\ZZ$, this was first observed by
Kock~\cite{kock:univ-proj-geometry}.\footnote{The proof goes as follows.
We want to verify that~$S \models \forall f\?\affl\_ \neg(f = 0) \Rightarrow
\speak{$f$ \inv}$. According to the Kripke--Joyal semantics of~$\Zar(S)$, we
thus have to show that for any~$S$-scheme~$T$ and any function~$f \in
\Gamma(T,\O_T)$ it holds that~$T \models \neg(f = 0)$ implies~$T \models
\speak{$f$ \inv}$. The antecedent means that for any~$T$-scheme~$T'$, if the
pullback of~$f$ to~$T'$ vanishes, then~$T'$ is the empty scheme. Similar to the
analogous statement of the little Zariski topos, the consequent means that~$f$
is invertible in~$\Gamma(T,\O_T)$. The claim follows by considering the
particular~$T$-scheme~$T' \defeq V(f)$. Since~$f$ vanishes on~$V(f)$, this
subscheme is empty and therefore its complement~$D(f)$ is the whole of~$T$.}

Using the internal language of the big Zariski topos, we can conduct
\emph{synthetic scheme theory}, or more precisely synthetic relative scheme
theory over a base scheme. As with other synthetic approaches to certain
subjects, such as synthetic differential geometry~\cite{kock:sdg}, synthetic
domain theory~\cite{hyland:synthetic-domain-theory}, or
synthetic computability theory~\cite{bauer:synthetic-computability-theory}, the
internal language lets us pretend that our
objects of study, schemes over a base scheme, are plain sets and that
morphisms between our objects are simply maps between those sets. We can therefore
employ a simple, element-based language to talk about schemes; the topological
and ring-theoretical structure is taken care of by the machinery.

As with the internal universe of any topos, we can perform the usual
set-theoretical constructions internally in the big Zariski topos, like
building products and coproducts, cutting out subsets and imposing equivalence
relations, and constructing sets of maps between given objects. Additionally,
the internal universe of the big Zariski topos has a distinctive algebraic
flavor, in that the curious statement
\begin{quote}\emph{Any} map~$\affl \to \affl$ is a polynomial function.\end{quote}
holds from the internal point of view.
Furthermore, even if the base scheme is the spectrum of a finite field, the
polynomial coefficients are unique.
These statements are related to the fact that the value of the internal
Hom~$[\affl,\affl]$ at an~$S$-scheme~$T$ is given
by~$\Gamma(\AA^1_T,\O_{\AA^1_T})$ and not by some larger ring consisting of,
say, all set-theoretical maps~$T \to T$. There is a similar phenomenon in
synthetic differential geometry, where any map~$\RR \to \RR$ is smooth (in an
appropriate sense).


\subsection{Internal descriptions of basic constructions of relative scheme theory}

With~$\affl$ at hand, we can perform many of the usual constructions of
(relative) scheme theory internally.

\subsubsection*{Group schemes} The functors associated to the standard group schemes~$\GG_\text{a}$, $\GG_\text{m}$,
$\mathrm{GL}_n$, and~$\mu_n$ are given by the internal expressions
\begin{align*}
  \GG_\text{a} &\defeq \affl \text{ (as an additive group)}, \\
  \GG_\text{m} &\defeq \{ x\?\affl \,|\, \speak{$x$ \inv} \}, \\
  \mathrm{GL}_n &\defeq \{ M \? (\affl)^{n \times n} \,|\, \speak{$M$ \inv} \}, \\
  \mu_n &\defeq \{ x \? \affl \,|\, x^n = 1 \}.
\end{align*}

\subsubsection*{Affine and projective space}
Affine~$n$-space over~$S$ is given by~$(\affl)^n$, \ie internally the set
of~$n$-tuples of elements of~$\affl$. The functor of points of
projective~$n$-space over~$X$, with all its nontrivial topological and
ring-theoretical structure, is described by the astoundingly naive expression
\[ \PP^n \defeq \{ (x_0,\ldots,x_n) \? (\affl)^{n+1} \,|\,
  x_0 \neq 0 \vee \cdots \vee x_n \neq 0 \}/{\sim}, \]
where the equivalence relation is the usual rescaling relation from the
internal point of view. This example was suggested by Zhen~Lin Low (private
communication).

More generally, for an~$S$-scheme~$X$, affine and projective~$n$-space
over~$X$ are given by~$\ul{X} \times (\affl^1)^n$ and~$\ul{X} \times \PP^n$,
respectively.

\subsubsection*{Open and closed subschemes}
Let~$X$ be an~$S$-scheme and~$f \in \O_X(X)$ be a global function. Since
the internal Hom~$[\ul{X},\affl] \in \Zar(S)$ has~$\O_X(X)$ as its set of global
sections, we can regard~$f$ as a map~$\ul{X} \to \affl$ from the internal point
of view. The internal expressions
\[ \{ x\?\ul{X} \,|\, f(x) \neq 0 \} \quad\text{and}\quad
  \{ x\?\ul{X} \,|\, f(x) = 0 \} \]
then describe the functor of points of the subscheme~$D(f)$
respectively~$V(f)$.

From the internal point of view,
the statement~$f(x)^n \neq 0$ is equivalent to~$f(x)^n$ being
invertible and therefore to~$f(x) \neq 0$. This reflects that~$D(f^n) = D(f)$.
However~$f(x)^n = 0$ does not imply~$f(x) = 0$, since~$\affl$ is only a field
in the sense that nonzero elements are invertible. Therefore there is no
contradiction with the basic fact that~$V(f^n)$ does not equal~$V(f)$ in
general.

As a special case, the set~$\Delta \defeq \{ \varepsilon\?\affl \,|\,
\varepsilon^2 = 0 \}$ describes an infinitesimal thickening of~$S$. It is the
vanishing scheme of~$\varepsilon^2$ in~$S \times_{\Spec \ZZ} \Spec
\ZZ[\varepsilon]$, so it has the same underlying topological space
as~$S$, but~$\O_S[\varepsilon]/(\varepsilon^2)$ instead of~$\O_S$ as structure
sheaf.

\subsubsection*{Tangent bundle}
For an~$S$-scheme~$X$, the internal Hom~$[\Delta,\ul{X}]$ describes the
``tangent bundle'' of~$X$, \ie the~$S$-scheme~$\RelSpec{X}{\operatorname{Sym}(\Omega^1_{X/S})} \to X \to S$, as can be seen by
chasing the definitions~\cite[Lemma~5.12.1]{brandenburg:tensor-foundations}.
Intuitively, a map~$f : \Delta \to \ul{X}$ from the internal point of view is
given by slightly more data than merely the point~$f(0)$; one also has to
specify first-order information.

This description of the (not necessarily
locally trivial) tangent bundle fits nicely with the intuition of tangent
vectors as infinitesimal curves, and in fact is precisely the definition of the
tangent bundle in synthetic differential geometry~\cite[Def.~7.1]{kock:sdg}.

\subsubsection*{Relative spectrum}
Let~$\E$ be an~$\affl$-algebra. Then~$[\E,\affl]_\affl$, \ie internally the set
of~$\affl$-algebra homomorphisms from~$\E$ to~$\affl$, can be pictured as ``the
relative spectrum of~$\E$'', even though of course this might fail to be
locally affine without further assumptions. For
instance, there are isomorphisms~$[\affl[x_1,\ldots,x_n],\affl]_\affl \to
(\affl)^n$ and~$[\affl \times \affl,\affl]_\affl \to \ul{S \amalg S}$.
In the special case that~$\E$ is induced by a quasicoherent algebra on~$S$, the
object~$[\E,\affl]_\affl$ is indeed the functor of points of the relative
spectrum of the quasicoherent algebra.

\begin{prop}Let~$S$ be a scheme. Let~$\A$ be a quasicoherent~$\O_S$-algebra.
Let~$\E \in \Zar(S)$ be the~$\affl$-algebra given by~$\E(T \xra{\mu} S) \defeq
\Gamma(T, \mu^*\A)$. The functor~$[\E,\affl]_\affl$ is the
functor of points of the~$S$-scheme~$\RelSpec{S}{\A}$.
\end{prop}
\begin{proof}

\end{proof}
% XXX

Conversely, if~$X \xra{\mu} S$ is an~$S$-scheme, the internal
expression~$[\ul{X},\affl]^\flat$ denotes the object of~$\Zar(S)$ induced by the
sheaf~$\mu_*\O_X$, \ie the functor given by~$(T \xra{\tau} S) \mapsto \Gamma(T,
\tau^* \mu_*\O_X)$. For an~$\affl$-module~$\E$ in~$\Zar(S)$ we mean by~$\E^\flat$
the~$\affl$-module obtained by restricting~$\E$ to~$\Sh(S)$ and then pulling back
to~$\Zar(S)$. Explicitly,~$\E^\flat(T \xra{\tau} S) = \Gamma(T, \tau^*
(\E|_{\Sh(S)}))$. (The symbol~``$\flat$'' is used because the operation~$\E
\mapsto \E^\flat$ is related to the \emph{flat modality} associated to the
local geometric morphism~$\Zar(S) \to \Sh(S)$ of toposes.)

Let~$\mu : X \to S$ be a quasicompact and quasiseparated morphism. It is well-known
that this morphism is affine if and only if the canonical morphism~$X \to
\RelSpec{S}(\mu_*\O_X)$ is an isomorphism. (This also holds if~$\mu$ is not
quasicompact and quasiseparated, but in this case~$\RelSpec{S}(\mu_*\O_X)$ is
only a locally ringed space, not a scheme.) We can reformulate this condition in
the internal language of~$\Zar(S)$ as saying that the canonical map
\[ \ul{X} \lra [ [\ul{X},\affl]^\flat, \affl ]_\affl,\ x \longmapsto
\placeholder(x) \]
into the ``double dual'' is bijective.


\subsection{Beyond the Zariski topology}

The Zariski topology is of course not the only interesting topology
on~$\Sch/S$. For any finer topology~$\tau$, such as the étale, smooth, or fppf
topology, the big~$\tau$-topos of~$S$, that is the topos of sheaves on~$\Sch/S$ with respect to~$\tau$, is a subtopos
of the big Zariski topos. Therefore there is a modal operator~$\Box_\tau$
in~$\Zar(S)$ reflecting the topology~$\tau$. Explicitly, for an~$S$-scheme~$T$
and a formula~$\varphi$ over~$T$, the meaning of
\[ T \models \Box_\tau \varphi \]
is that there exists a~$\tau$-covering~$(T_i \to T)_i$ of~$T$ such that~$T_i
\models \varphi$ for all~$i$ (where parameters appearing in~$\varphi$ have to
be pulled back along~$T_i \to T$). Succinctly, the formula ``$\Box_\tau
\varphi$'' means that~$\varphi$ holds~$\tau$-locally. Generalizing
Theorem~\ref{thm:box-translation-semantically} from sheaves on locales to
sheaves on arbitrary Grothendieck sites we also have
\[ \Zar(S) \models \varphi^{\Box_\tau} \qquad\text{iff}\qquad
  \Sh((\Sch/S)_\tau) \models \varphi. \]

A basic illustration of these modal operators is provided by the Kummer sequence, that is the short sequence
\[ 1 \lra \mu_n \lra \GG_\text{m} \stackrel{(\underline{\ })^n}{\lra} \GG_\text{m} \lra 1 \]
of multiplicatively-written commutative group objects in~$\Zar(S)$. With the
internal description of~$\mu_n$ and~$\GG_\text{m}$, there is a purely internal
and straightforward proof that this sequence is exact at the first two terms.
But except for trivial cases, the~$n$-th power map~$\GG_\text{m} \to
\GG_\text{m}$ will fail to be an epimorphism;
internally speaking, the statement
\[ \forall f\?(\affl)^\times\_ \phantom{\Box_\text{ét}(}\exists
g\?(\affl)^\times\_ f = g^n\phantom{)} \]
is false in general. However, if~$n$ is invertible in~$\Gamma(S,\O_S)$, the
internal statement
\[ \forall f\?(\affl)^\times\_ \Box_\text{ét}(\exists g\?(\affl)^\times\_ f = g^n) \]
\emph{is} true. In fact, the more general statement
\begin{multline*}
  \forall p\?\affl[x]\_ \speak{$p$ is monic, of positive degree, and separable}
  \Longrightarrow \\
  \Box_\text{ét}(\exists x\?\affl\_ p(x) = 0 \wedge \speak{$p'(x)$ \inv})
\end{multline*}
is true from the internal point of view, where a polynomial~$p$ is called
\emph{separable} if and only if there exists a Bézout representation~$ap + bp'
= 1$. After simplifying, the intepretation of that statement with the
Kripke--Joyal semantics is that for any~$S$-scheme~$T$ and any monic separable
polynomial~$p \in \Gamma(T,\O_T)[x]$ of positive degree there exists an étale
covering~$(T_i \to T)_i$ of~$T$ such that the pullbacks of~$p$ to each of
the~$T_i$ possess a simple zero. The required covering is given
by the canonical surjective étale map~$\RelSpec{T}{\O_T[x]/(p)} \to T$.

The following theorem shows that the modal operator~$\Box_\text{ét}$
corresponding to the étale topology admits a purely internal characterization
in~$\Zar(S)$, which furthermore resonates well with the intuition about the
étale topology.

\begin{thm}Let~$S$ be a scheme. The modal operator~$\Box_\text{ét}$
in~$\Zar(S)$ corresponding to the étale topology is the smallest
operator~$\Box$ such that the~$\Box$-translation of the statement~``$\affl$ is
separably closed'' holds.\end{thm}

Here, a ring~$A$ is \emph{separably closed} if and only if
\begin{multline*}
  \forall p\?A[x]\_ \speak{$p$ is monic, of positive degree, and unramifiable}
  \Longrightarrow \\
  \exists x\?A\_ p(x) = 0 \wedge \speak{$p'(x)$ \inv}.
\end{multline*}
We call a polynomial~$p$ over a ring~$A$ \emph{unramifiable} if and
only if it admits at least one simple root in every algebraically closed field
over~$A$. Since quantifying over algebraically closed fields raises red flags
from an intuitionistic point of view, just as quantifying over maximal ideals
does, this condition has to be formulated in a sensible way. One possibility is
to use the \emph{hyperdiscriminants} of~$p$, \ie the elementary symmetric
polynomials in the values of~$p'$ at the roots of~$p$, resulting in a simple
existential statement involving only the coefficients of~$p$; in particular,
the condition for a polynomial to be unramifiable is a geometric formula.
See~\cite[p.~751]{wraith:generic-galois-theory} for the precise formulation.

In more detail, the claim is that firstly~$\Box_\text{ét}$ is a modal operator
such that the displayed formula holds and that secondly, if~$\Box$ is any modal
operator verifying the formula, internally it holds that~$\Box_\text{ét}\varphi
\Rightarrow \Box\varphi$ for any truth value~$\varphi\?\Omega$.

\begin{proof}For the proof we require some familiarity with the concept of
classifying toposes. We are grateful to Felix Geißler for contributing a key step of
the argument.

To verify the first statement, note that the displayed formula is a geometric
implication and that the big étale topos~$\Et(S)$ has \emph{enough points}.
Therefore it suffices to show that for any~$S$-scheme~$T$ and any geometric
point~$\bar t$ of~$T$, the stalk~$\O_{T,\bar t}$ is separably closed. It is
well-known that this is true.

For the second statement we may assume without loss of generality that~$S =
\Spec A$ is affine. It is well-known that, for any cocomplete topos~$\E$,
geometric morphisms~$\E \to \Zar(\Spec A)$ are in canonical one-to-one correspondence
with local algebras over~$\ul{A}$ in~$\E$ (where~$\ul{A}$ denotes the pullback
of~$A$ along the unique geometric morphism~$\E \to \Set$) and that geometric
morphisms~$\E \to \Et(\Spec A)$ are in canonical one-to-one correspondence
with algebras over~$\ul{A}$ which are local and separably closed from the
internal point of view of~$\E$;
see~\cite[Section~VIII.6]{moerdijk-maclane:sheaves-logic}
and~\cite{anel:factorization-systems}.

Therefore a geometric morphism~$\E \to \Zar(\Spec A)$ factors over the
geometric embedding~$\Et(\Spec A) \hookrightarrow \Zar(\Spec A)$ if and only if
the pullback of~$\affla$ along~$\E \to \Zar(\Spec A)$ is separably closed.

Let~$\Box$ be a modal operator in~$\Zar(\Spec A)$ such that the~$\Box$-translation
of~``$\affla$ is separably closed'' holds. Then the pullback of~$\affla$
along~$\Zar(\Spec A)_\Box \hookrightarrow \Zar(\Spec A)$ is separably closed
and therefore this geometric embedding factors over~$\Et(\Spec A)
\hookrightarrow \Zar(\Spec A)$. This shows that any~$\Box$-sheaf is also
a~$\Box_\text{ét}$-sheaf.

The claim that~$\Box_\text{ét}\varphi \Rightarrow \Box\varphi$ for any truth
value~$\varphi\?\Omega$ then follows by combining the following two basic
observations of the theory of modal operators, valid for any modal
operator~$\Box$:
\begin{enumerate}
\item $\Box\varphi \Longleftrightarrow
  \forall \psi\?\Omega\_ ((\Box\psi\Rightarrow\psi) \wedge
  (\varphi\Rightarrow\psi)) \Rightarrow \psi$.
\item $(\Box\psi \Rightarrow \psi) \Longleftrightarrow
  \speak{$\{x \in 1 \,|\, \psi\}$ is a~$\Box$-sheaf}$. \qedhere
\end{enumerate}
\end{proof}

% XXX: other topologies?

% XXX: A note on tangent bundles in a category with a ring object.
% (http://www.mscand.dk/article/view/11736)

\begin{itemize}
\item basics: $\AA^1$ is a local ring and even a field, never reduced
\item synthetic topology?
\end{itemize}


\section{Unsorted}
\begin{itemize}
\item ``functoriality''
\item Kähler differentials
\item closed and open subschemes
\item $j_! \O_U$ flat over~$\O_X$, \ldots
\item Koszul resolution; Beilinson resolution?
\item meta properties: some lemmas about limits of modules
\item locally small categories
\item open/closed immersions
\item morphisms of schemes...
\item proper maps...
\item limits and colimits...
\item Kähler differentials; clear myth that the definition via free modules
``does not glue very well''
(\url{http://www.mathematik.uni-kl.de/~gathmann/class/alggeom-2002/chapter-7.pdf})
\end{itemize}


\appendix

\section{Noetherianity}

Recall the usual notion of a Noetherian ring: Any sequence~$\aaa_0 \subseteq
\aaa_1 \subseteq \cdots$ of ideals should stabilize, \ie there should exist a
natural number~$n$ such that~$\aaa_n = \aaa_{n+1} = \cdots$.

Intuitionistically, this definition has two problems. Firstly, without the
axiom of dependent choice, it is often not possible to construct a
\emph{sequence} of ideals: Often, it is only possible to show that there
\emph{exists} a suitable ideal~$\aaa_{n+1}$ depending on~$\aaa_n$. But since in
general there is no canonical choice for this successor ideal, the axiom of dependent choice
would be required to collect those into a sequence, \ie a function from~$\NN$
to the set of ideals.

Secondly, the conclusion that the sequence stabilizes is too strong:
Intuitionistically, one cannot even show that a weakly descending sequence of
natural numbers stabilizes in this sense; the statement that one could is
equivalent to the \emph{limited principle of omniscience for~$\NN$}.
Intuitionistically, it is only true that a weakly descending sequence~$a_0 \geq
a_1 \geq \cdots$ of natural numbers eventually \emph{halts} in the sense that
there exists an index~$n$ such that~$a_n = a_{n+1}$ (but~$a_{n+1} > a_{n+2}$ is
allowed).

We therefore adopt the following definitions.

\begin{defn}Let~$M$ be a partially ordered set. An \emph{ascending process
with values in~$M$} is a function~$f : \NN \to \P(M)$ such that
the subset~$f(0)$ is inhabited and such that for any~$x \in f(n)$, $n \in \NN$,
there exists an element~$y \in f(n+1)$ with~$x \preceq y$. (In particular, all
subsets~$f(n)$ are inhabited.) Such a process \emph{halts} if and only if there
exists a step~$n$ such that there is an element~$x \in f(n) \cap f(n+1)$. The
set~$M$ satisfies the \emph{ascending process condition} if and only if any
ascending process with values in~$M$ halts.
\end{defn}

\begin{defn}A ring~$R$ is \emph{processly Noetherian} if and only if the
set of finitely generated ideals in~$R$ satisfies the ascending process
condition.\end{defn}

Any ascending chain of elements~$a_0 \preceq a_1 \preceq \cdots$ in a partially
ordered set gives rise to an ascending process by setting~$f(n) \defeq \{ a_n
\}$. Conversely, the axiom of dependent choice would allow to construct an
ascending chain from an ascending process. In classical logic, a ring is
processly Noetherian if and only if it is Noetherian in the usual sense.

The notion of a processly Noetherian ring works well in an
intuitionistic context: Important rings such as~$\ZZ$ and more generally~$\O_K$
for any algebraic number field~$K$ are processly Noetherian, and matrices
over Bézout rings which are integral domains in the weak sense and
processly Noetherian can be put into Smith canonical form.

\begin{prop}\label{prop:internal-noetherianity}
A scheme~$X$ is locally Noetherian if and only if the ring~$\O_X$
is processly Noetherian from the internal point of view.\end{prop}
\begin{proof}We only prove the ``only if'' direction. We may assume that~$X =
\Spec A$ is affine with~$A$ a Noetherian ring and that internally, we are given an
ascending process on the set of finitely generated ideals of~$\O_X$.
Externally, this is a morphism~$\ul{\NN} \to \P(\M)$ where~$\NN$ is the
constant sheaf with value~$\NN$ and~$U$-sections of~$\M$ are finite type ideal
sheaves of~$\O_X|_U$.

Since~$X \models \speak{$f(0)$ is inhabited}$, there exists an open covering~$X
= \bigcup_i U_i$ and finite type ideal sheaves~$\I_i \hookrightarrow
\O_X|_{U_i}$ such that~$U_i \models \I_i \in f(0)$. Without loss of generality,
we may assume that the open sets~$U_i$ are standard open sets~$D(f_i)$ and that
the covering is finite. Since the sheaves~$\I_i$ are quasicoherent (being of finite type), they
correspond to ideals~$J_i \subseteq A[f_i^{-1}]$. Note for future reference
that for~$D(g) \subseteq D(f_i)$, the restricted ideal~$\I_i|_{D(g)}$ corresponds to the extension of~$J_i$ in the further
localized ring~$A[g^{-1}]$.

For each~$i$,~$D(f_i) \models \exists \aaa \in f(1)\_ \I_i \subseteq \aaa$.
Thus there exists an open covering~$D(f_i) = \bigcup_j D(g_{ij})$ and finite
type ideal sheaves~$\I_{ij} \hookrightarrow \O_X|_{D(g_{ij})}$; these
correspond to ideals~$J_{ij} \subseteq A[g_{ij}^{-1}]$ such that~$J_i
\subseteq J_{ij}$ (where we have suppressed the localization
morphism~$A[f_i^{-1}] \to A[g_{ij}^{-1}]$ in the notation). Equivalently,
writing~$J_i' \defeq A \cap J_i$ and~$J_{ij}' \defeq A \cap J_{ij}$ for the
contractions, we have the inclusions~$J_i' \subseteq J_{ij}'$ of ideals of~$A$.

Continuing in this fashion, we obtain a tree of ideals~$J_{i_1 \cdots i_n}'$.
Each path in this tree is a chain of ascending ideals and thus stabilizes
since~$A$ is Noetherian. Since only finitely many subtrees branch off at
each node, there appear only finitely many distinct ideals in this tree (this
is an application of the graph-theoretical \emph{König's lemma}).

There thus exists a natural number~$n$ such that~$J_{i_1 \cdots i_n}' = J_{i_1
\cdots i_n i_{n+1}}'$ for all appropriate indices~$i_1,\ldots,i_n,i_{n+1}$.
For this number~$n$, the internal statement~$X \models \exists \aaa \in f(n)
\cap f(n+1)$ holds; we leave further details to the reader.
\end{proof}

\begin{rem}The proof shows that, internally speaking, even
the set of all quasicoherent ideals (instead of merely the finitely generated
ones) fulfills the ascending process condition, if the base scheme is locally Noetherian. We have not taken this property
as the definition of a processly Noetherian ring since it is a notion not
usually studied in constructive mathematics (compare
Remark~\ref{rem:qcoh-in-constructive-mathematics}).
\end{rem}

\XXX{Find appropriate place for the material in this appendix}

\XXX{Give examples made possible by the internal Noetherianity}
% How about: Any quasicoherent submodule of a module of finite type is of
% finite type as well.


\section{Dictionary between internal and external notions}

{\small\renewcommand{\arraystretch}{1.3}
\begin{longtable}{@{}p{4.4cm}@{\qquad}p{6.7cm}@{\qquad}p{1.5cm}@{}}
  \toprule
  External & Internal & Reference \\ \midrule
  \textbf{Sheaves of sets} \\
  sheaf of sets & set \\
  $\alpha : \F \to \G$ monomorphism & $\alpha$ injective & Ex.\@~\ref{ex:injective-surjective} \\
  $\alpha : \F \to \G$ epimorphism & $\alpha$ surjective & Ex.\@~\ref{ex:injective-surjective} \\
  $\Int(X \setminus \supp\F)$ & truth value of ``$\F$ is a singleton'' & Rem.\@~\ref{rem:support-sheaf-of-sets} \\
  $f : X \to \NN$ upper semicont.\@ & element of~$\widehat\NN$ & Lemma~\ref{lemma:upper-semicontinuous-functions} \\
  $f : X \to \NN$ locally constant & element of~$\NN$ & Lemma~\ref{lemma:upper-semicontinuous-functions} \\\\

  \textbf{Sheaves of rings} \\
  sheaf of rings & ring & Prop.\@~\ref{prop:rings-internally} \\
  local sheaf of rings & local ring & Prop.\@~\ref{prop:local-ring} \\
  $X$ is reduced & $\O_X$ is reduced (and $\neg\text{invertible} \Rightarrow \text{zero}$) & Prop.\@~\ref{prop:reduced-ring} \\
  $\dim X \leq n$ & Krull dimension of~$\O_X$ is~$\leq n$ & Prop.\@~\ref{prop:dimension-scheme-ox} \\
  $X$ is integral at all points & $\O_X$ is a integral domain & Prop.\@~\ref{prop:internal-integrality} \\
  $X$ is locally Noetherian & $\O_X$ is processly Noetherian & Prop.\@~\ref{prop:internal-noetherianity} \\
  $X$ is normal & $\O_X$ is normal (assuming that~$X$ is locally Noetherian) & Prop.\@~\ref{prop:normal-int-ext} \\\\

  \textbf{Sheaves of modules} \\
  sheaf of modules & module \\
  $\F$ is finite locally free & $\F$ is finite free & Prop.\@~\ref{prop:locally-free} \\
  $\F$ is of finite type & $\F$ is finitely generated & Prop.\@~\ref{prop:finite-type-and-co} \\
  $\F$ is of finite presentation & $\F$ is finitely presented & Prop.\@~\ref{prop:finite-type-and-co} \\
  $\F$ is coherent & $\F$ is coherent & Prop.\@~\ref{prop:finite-type-and-co} \\
  $\F$ is quasicoherent & $\F[f^{-1}]$ is a sheaf wrt.\@~$(\speak{$f$ \inv} \Rightarrow \placeholder)$ for~$f\?\O_X$ & Thm.\@~\ref{thm:qcoh-sheafchar} \\
  $\F$ is flat & $\F$ is flat & Prop.\@~\ref{prop:flatness} \\
  $\F$ is torsion & $\F$ is torsion & Prop.\@~\ref{prop:torsion-int-ext} \\
  $M^\sim$ & $\ul{M}[\F^{-1}]$ (localization at generic filter) & Prop.\@~\ref{prop:tilde-construction-internally} \\
  tensor product $\F \otimes \G$ & tensor product $\F \otimes \G$ & Prop.\@~\ref{prop:internal-tensor-product} \\
  dual~$\F^\vee = \HOM_{\O_X}(\F,\O_X)$ & dual $\F^\vee = \Hom_{\O_X}(\F,\O_X)$ \\
  $\Int(X \setminus \supp\F)$ & truth value of~``$\F = 0$'' & Prop.\@~\ref{prop:characterization-support} \\
  quasicoherator of~$\I$ & $\{ s\?\O_X \,|\, \speak{$s$ \inv}
  \Rightarrow s \in \I \}$ ($\I$ a radical ideal) & Prop.\@~\ref{prop:quasicoherator-structure-sheaf} \\
  rank function of~$\F$ & minimal number of generators for~$\F$ & Prop.\@~\ref{prop:rank-function-internally} \\\\

  \multicolumn{3}{@{}l@{}}{\textbf{Subspaces} ($i : A \hookrightarrow X$ closed immersion, $j : U \hookrightarrow X$ open immersion)} \\
  sheaf supported on~$A$ & $\Box$-sheaf, where~$\Box = (\placeholder \vee A^c)$ & Lemma~\ref{lemma:essim-closed-immersion} \\
  sheaf of the form~$j_*(\F)$ & $\Box$-sheaf, where~$\Box = (U \Rightarrow
  \placeholder)$ & \\
  extension of~$\F$ by the empty set & $j_!(\F) = \{ x\?\F \,|\, U \}$ & Lemma~\ref{lemma:extension-by-empty-set} \\
  extension of~$\F$ by zero & $j_!(\F) = \{ x\?\F \,|\, (x = 0) \vee U \}$ & Lemma~\ref{lemma:extension-by-zero} \\
  sheaf with empty/zero stalks on~$U^c$ & sheaf of the form~$j_!(\F)$ \\
  sections of~$\F$ are equal if they agree on dense open & $\F$ is $\neg\neg$-separated & Prop.\@~\ref{prop:negneg-sheaves} \\
  sheaf of sections of~$\F$ defined on dense open subsets & $\F^{++}$ with respect to~$\Box = \neg\neg$ (assuming that~$\F$ is~$\neg\neg$-separated) & Prop.\@~\ref{prop:negneg-sheaves} \\
  $U$ is dense & $\neg\neg U$ & Prop.\@~\ref{prop:modops-kripke} \\
  $U$ is scheme-theoretically dense & $\sdense U$, \ie $\O_X$ is separated wrt.~$(U \Rightarrow \placeholder)$ & Lemma\@~\ref{lemma:scheme-theoretical-density} \\
  $V(\I)$ is reduced & $\I$ is a radical ideal & Lemma~\ref{lemma:closed-subspace-reduced} \\
  $\O_{X_\mathrm{red}}$ & $\O_X/\sqrt{(0)}$ & Lemma~\ref{lemma:reduced-subspace} \\\\

  \multicolumn{3}{@{}l@{}}{\textbf{Rational functions and Cartier divisors}} \\
  $\K_X$ & total quotient ring of~$\O_X$ & Prop.\@~\ref{prop:kx-internally} \\
  Cartier divisor & element of~$\K_X^\times/\O_X^\times$ \\
  effective Cartier divisor & $[s/1]$ with~$s\?\O_X$ regular & Def.\@~\ref{defn:effective-cartier-divisor} \\
  line bundle~$\O_X(D)$ & $D^{-1} \O_X \subseteq \K_X$ & Def.\@~\ref{defn:line-bundle-of-divisor} \\\\

  \multicolumn{3}{@{}l@{}}{\textbf{Topological properties}} \\
  $X$ is quasicompact & ``$\Sh(X) \models$'' commutes with directed disjunctions & Prop.\@~\ref{prop:quasicompact-meta} \\
  $X$ is local & ``$\Sh(X) \models$'' commutes with arbitrary disjunctions & Prop.\@~\ref{prop:local-meta} \\
  $X$ is irreducible & if $\neg(\varphi \wedge \psi)$, then $\neg\varphi$ or~$\neg\psi$ & Prop.\@~\ref{prop:irreducibility-internally} \\
  \bottomrule
\end{longtable}}
% missing: relative spectrum and big Zariski topos


\section{The inference rules of intuitionistic logic}
\label{appendix:inference-rules}

\XXX{cite \cite[Section~D1.3.1]{johnstone:elephant}, talk about~$\in$, and
explain contexts}

\begin{center}
  \textbf{Structural rules} \\
  \vspace{-0.5em}
  \phantom{a}\hfill
  \AxiomC{$\phantom{\seq{\vec x}}$}\UnaryInfC{$\varphi \seq{\vec x} \varphi$}\DisplayProof\hfill
  \AxiomC{$\varphi \seq{\vec x} \psi$}\UnaryInfC{$\varphi[\vec s/\vec x]
  \seq{\vec y} \psi[\vec s/\vec x]$}\DisplayProof\hfill
  \AxiomC{$\varphi \seq{\vec x} \psi$}\AxiomC{$\psi \seq{\vec x}
  \chi$}\BinaryInfC{$\varphi \seq{\vec x} \chi$}\DisplayProof
  \phantom{a}\hfill
  \vspace{2.0em}

  \textbf{Rules for nullary and binary conjunction} \\
  \vspace{-0.5em}
  \phantom{a}\hfill
  \AxiomC{$\phantom{\seq{\vec x}}$}\UnaryInfC{$\varphi \seq{\vec x} \top$}\DisplayProof\hfill
  \AxiomC{$\phantom{\seq{\vec x}}$}\UnaryInfC{$\varphi \wedge \psi \seq{\vec x} \varphi$}\DisplayProof\hfill
  \AxiomC{$\phantom{\seq{\vec x}}$}\UnaryInfC{$\varphi \wedge \psi \seq{\vec x} \psi$}\DisplayProof\hfill
  \AxiomC{$\varphi \seq{\vec x} \psi$}\AxiomC{$\varphi \seq{\vec x} \chi$}\BinaryInfC{$\varphi \seq{\vec x} \psi \wedge \chi$}\DisplayProof
  \phantom{a}\hfill
  \vspace{2em}

  \textbf{Rules for nullary and binary disjunction} \\
  \vspace{-0.5em}
  \phantom{a}\hfill
  \AxiomC{$\phantom{\seq{\vec x}}$}\UnaryInfC{$\bot \seq{\vec x} \varphi$}\DisplayProof\hfill
  \AxiomC{$\phantom{\seq{\vec x}}$}\UnaryInfC{$\varphi \seq{\vec x} \varphi \vee \psi$}\DisplayProof\hfill
  \AxiomC{$\phantom{\seq{\vec x}}$}\UnaryInfC{$\psi \seq{\vec x} \varphi \vee \psi$}\DisplayProof\hfill
  \AxiomC{$\varphi \seq{\vec x} \chi$}\AxiomC{$\psi \seq{\vec x} \chi$}\BinaryInfC{$\varphi \vee \psi \seq{\vec x} \chi$}\DisplayProof
  \phantom{a}\hfill
  \vspace{2em}

  \textbf{Rules for arbitrary set-indexed conjunction and disjunction} \\
  \vspace{-0.5em}
  \phantom{a}\hfill
  \AxiomC{$\phantom{\seq{\vec x}}$}\UnaryInfC{$\bigwedge_{i \in I} \varphi_i \seq{\vec x} \varphi_j$ for all~$j \in I$}\DisplayProof\hfill
  \AxiomC{$\varphi \seq{\vec x} \psi_j$ for all~$j \in I$}\UnaryInfC{$\varphi \seq{\vec x} \bigwedge_{i \in I} \psi_i$}\DisplayProof
  \phantom{a}\hfill
  \vspace{1em}

  \phantom{a}\hfill
  \AxiomC{$\phantom{\seq{\vec x}}$}\UnaryInfC{$\varphi_j \seq{\vec x} \bigvee_{i \in I} \varphi_i$ for all~$j \in I$}\DisplayProof\hfill
  \AxiomC{$\varphi_j \seq{\vec x} \psi$ for all~$j \in I$}\UnaryInfC{$\bigvee_{i \in I} \varphi_i \seq{\vec x} \psi$}\DisplayProof
  \phantom{a}\hfill
  \vspace{2em}

  \textbf{Double rule for implication} \\
  \vspace{-0.5em}
  \phantom{a}\hfill
  \Axiom$\varphi \wedge \psi\ \fCenter\seq{\vec x} \chi$
  \doubleLine
  \UnaryInf$\varphi\ \fCenter\seq{\vec x} \psi \Rightarrow \chi$
  \DisplayProof
  \phantom{a}\hfill
  \vspace{2em}

  \textbf{Double rules for bounded and unbounded quantification} \\
  \vspace{-0.5em}
  \phantom{a}\hfill
  \Axiom$\varphi\ \fCenter\seq{\vec x, y} \psi$
  \doubleLine
  \UnaryInf$\exists y\?Y\_\! \varphi\ \fCenter\seq{\vec x} \psi$
  \DisplayProof
  {\tiny ($y$ not occuring in~$\psi$)}
  \hfill
  \Axiom$\varphi\ \fCenter\seq{\vec x, y} \psi$
  \doubleLine
  \UnaryInf$\varphi\ \fCenter\seq{\vec x\phantom{, y}} \forall y\?Y\_\! \psi$
  \DisplayProof
  {\tiny ($y$ not occuring in~$\varphi$)}
  \hfill\phantom{a}
  \vspace{1em}

  \phantom{a}\hfill
  \Axiom$\varphi\ \fCenter\seq{\vec x, Y} \psi$
  \doubleLine
  \UnaryInf$\exists Y\_\! \varphi\ \fCenter\seq{\vec x} \psi$
  \DisplayProof
  {\tiny ($Y$ not occuring in~$\psi$)}
  \hfill
  \Axiom$\varphi\ \fCenter\seq{\vec x, Y} \psi$
  \doubleLine
  \UnaryInf$\varphi\ \fCenter\seq{\vec x\phantom{, Y}} \forall Y\_\! \psi$
  \DisplayProof
  {\tiny ($Y$ not occuring in~$\varphi$)}
  \hfill\phantom{a}
  \vspace{2em}

  \textbf{Rules for equality} \\
  \vspace{-0.5em}
  \phantom{a}\hfill
  \AxiomC{$\phantom{\seq{\vec x}}$}
  \UnaryInfC{$\top \seq{x} x = x$}
  \DisplayProof
  \hfill
  \AxiomC{$\phantom{\seq{\vec x}}$}
  \UnaryInfC{$(\vec x = \vec y) \wedge \varphi \seq{\vec z} \varphi[\vec y/\vec x]$}
  \DisplayProof
  \hfill\phantom{a} \\[0.5em]
  (``$\vec x = \vec y\,$'' is short for~``$x_1 = y_1 \wedge \cdots \wedge x_n =
  y_n$''.)
\end{center}

\nocite{*}
\printbibliography

\XXX{remark on possible pitfalls}

\end{document}

\YYY{typography of $\bigcup$}

Snippet that may be useful later:
By appealing to the axiom of unique choice (see~YYY), we can
define a morphism of sheaves~$\O_X(D) \otimes \O_X(D') \to \O_X(D + D')$ by
internally describing a suitable map using representatives~$D = [f]$,~$D' =
[f']$, as long as the resulting map does not depend on the choice of representatives.

For the third statement, note that it is equivalent to show that
\[ \Gamma(U,\F^+) = \{ (V,s) \,|\,
  \text{$V \subseteq U$ dense open},\
  s \in \Gamma(V,\F),\
  \text{$(V,s)$ maximal} \}, \]
where~``$(V,s)$ maximal'' means that for any other such pair~$(W,t)$ such
that~$V \subseteq W$ and~$t|_V = s$, it holds that~$V = W$. This follows from
the fact that the plus construction can also be defined as
\[ \F^+ \defeq \{ S \subseteq F \,|\,
  \speak{$S$ subsingleton},\
  \neg\neg(\speak{$S$ inhabited}),\
  \speak{$S$ $\neg\neg$-stable} \}. \]

What is a conceptual explanation for O_X(D(f)) = A[f^(-1)] and O_X(U) !=
A[S_U^(-1)] otherwise?

Constructible topology?

Big Zariski topos:
* A^1 local ring, field, but of unbounded Krull dimension

Reassure that constructing morphisms between spectra is *easy* in the localic
setting.

Sch/S or Aff/S
