\documentclass[10pt,reqno,a4paper]{amsbook}
\usepackage{etex}
\usepackage[utf8]{inputenc}
\usepackage[english]{babel}
\usepackage{etoolbox,chngcntr}
\usepackage{amsmath,amsthm,amssymb,array,stmaryrd,color,graphicx,mathtools,multirow,setspace}
\usepackage{soul}\setul{0.3ex}{}
\usepackage{bussproofs}
\usepackage{xspace}
\usepackage{longtable}
\usepackage{booktabs}
\usepackage[protrusion=true,expansion=true]{microtype}
\usepackage[bookmarksdepth=2,pdfencoding=auto]{hyperref}

%\addtolength{\oddsidemargin}{7mm}
%\addtolength{\evensidemargin}{-7mm}

% Hack to load extpfeil from http://tex.stackexchange.com/a/297109/32372
\expandafter\def\csname opt@stmaryrd.sty\endcsname
{only,shortleftarrow,shortrightarrow}
\usepackage{extpfeil}
\newextarrow{\xbigtoto}{{20}{20}{20}{20}}
   {\bigRelbar\bigRelbar{\bigtwoarrowsleft\rightarrow\rightarrow}}

\usepackage[all]{xy}
\usepackage{tikz}
\usetikzlibrary{calc,shapes.callouts,shapes.arrows,matrix,patterns}
\newcommand{\hcancel}[5]{%
    \tikz[baseline=(tocancel.base)]{
        \node[inner sep=0pt,outer sep=0pt] (tocancel) {#1};
        \draw[red, line width=0.3mm] ($(tocancel.south west)+(#2,#3)$) -- ($(tocancel.north east)+(#4,#5)$);
    }%
}

\usepackage[natbib=true,style=numeric]{biblatex}
\usepackage[babel]{csquotes}
\bibliography{bibliography}

\theoremstyle{definition}
\newtheorem{defn}{Definition}[section]
\newtheorem{ex}[defn]{Example}

\theoremstyle{plain}
\newtheorem{prop}[defn]{Proposition}
\newtheorem{cor}[defn]{Corollary}
\newtheorem{lemma}[defn]{Lemma}
\newtheorem{thm}[defn]{Theorem}
\newtheorem{scholium}[defn]{Scholium}

\theoremstyle{remark}
\newtheorem{rem}[defn]{Remark}
\newtheorem{question}[defn]{Question}

\newcommand{\ZZ}{\mathbb{Z}}
\newcommand{\FF}{\mathbb{F}}
\renewcommand{\AA}{\mathbb{A}}
\newcommand{\A}{\mathcal{A}}
\newcommand{\B}{\mathcal{B}}
\renewcommand{\C}{\mathcal{C}}
\newcommand{\D}{\mathcal{D}}
\newcommand{\E}{\mathcal{E}}
\newcommand{\F}{\mathcal{F}}
\renewcommand{\G}{\mathcal{G}}
\renewcommand{\H}{\mathcal{H}}
\renewcommand{\O}{\mathcal{O}}
\newcommand{\K}{\mathcal{K}}
\newcommand{\N}{\mathcal{N}}
\newcommand{\M}{\mathcal{M}}
\renewcommand{\L}{\mathcal{L}}
\renewcommand{\P}{\mathcal{P}}
\newcommand{\R}{\mathcal{R}}
\newcommand{\T}{\mathcal{T}}
\newcommand{\I}{\mathcal{I}}
\newcommand{\J}{\mathcal{J}}
\renewcommand{\S}{\mathcal{S}}
\renewcommand{\U}{\mathcal{U}}
\newcommand{\V}{\mathcal{V}}
\newcommand{\NN}{\mathbb{N}}
\newcommand{\PP}{\mathbb{P}}
\newcommand{\RR}{\mathbb{R}}
\newcommand{\CC}{\mathbb{C}}
\newcommand{\QQ}{\mathbb{Q}}
\newcommand{\GG}{\mathbb{G}}
\newcommand{\aaa}{\mathfrak{a}}
\newcommand{\bbb}{\mathfrak{b}}
\newcommand{\ccc}{\mathfrak{c}}
\newcommand{\ppp}{\mathfrak{p}}
\newcommand{\qqq}{\mathfrak{q}}
\newcommand{\mmm}{\mathfrak{m}}
\newcommand{\nnn}{\mathfrak{n}}
\newcommand{\Hom}{\mathrm{Hom}}
\newcommand{\HOM}{\mathcal{H}\mathrm{om}}
\newcommand{\id}{\mathrm{id}}
\newcommand{\GL}{\mathrm{GL}}
\newcommand{\placeholder}{\underline{\quad}}
\let\oldul\ul
\renewcommand{\ul}[1]{\text{\oldul{$#1$}}}
\newcommand{\Set}{\mathrm{Set}}
\newcommand{\Grp}{\mathrm{Grp}}
\newcommand{\Vect}{\mathrm{Vect}}
\newcommand{\Sh}{\mathrm{Sh}}
\newcommand{\PSh}{\mathrm{PSh}}
\newcommand{\Zar}{\mathrm{Zar}}
\newcommand{\Et}{\mathrm{\acute{E}t}}
\newcommand{\Sch}{\mathrm{Sch}}
\newcommand{\Aff}{\mathrm{Aff}}
\newcommand{\Mod}{\mathrm{Mod}}
\newcommand{\Alg}{\mathrm{Alg}}
\newcommand{\Ring}{\mathrm{Ring}}
\newcommand{\LocRing}{\mathrm{LocRing}}
\newcommand{\RL}{\mathrm{RL}}
\newcommand{\LRL}{\mathrm{LRL}}
\newcommand{\LRS}{\mathrm{LRS}}
\newcommand{\LRT}{\mathrm{LRT}}
\newcommand{\RT}{\mathrm{RT}}
\newcommand{\pt}{\mathrm{pt}}
\newcommand{\tors}{\mathrm{tors}}
\newcommand{\lfp}{\mathrm{lfp}}
\newcommand{\fp}{\mathrm{fp}}
\DeclareMathOperator{\Spec}{Spec}
\DeclareMathOperator{\Proj}{Proj}
\newcommand{\QcohSpec}[2]{\mathrm{Spec}^{\mathrm{qcoh}}_{#1}{#2}}
\newcommand{\RelSpec}{\operatorname{\ul{\mathrm{Spec}}}}
\newcommand{\RelProj}{\operatorname{\ul{\mathrm{Proj}}}}
\newcommand{\op}{\mathrm{op}}
\DeclareMathOperator{\colim}{colim}
\DeclareMathOperator{\rank}{rank}
\DeclareMathOperator{\Ann}{Ann}
\DeclareMathOperator{\Int}{int}
\DeclareMathOperator{\Clos}{cl}
\DeclareMathOperator{\Kernel}{ker}
\DeclareMathOperator{\im}{im}
\DeclareMathOperator{\supp}{supp}
\newcommand{\Ass}{\mathrm{Ass}}
\newcommand{\Sym}{\mathrm{Sym}}
\newcommand{\Gr}{\mathrm{Gr}}
\newcommand{\Open}{\T}
\newcommand{\?}{\,{:}\,}
\renewcommand{\_}{\mathpunct{.}\,}
\newcommand{\speak}[1]{\ulcorner\text{\textnormal{#1}}\urcorner}
\newcommand{\Ll}{\vcentcolon\Longleftrightarrow}
\newcommand{\notat}[1]{{!#1}}
\newcommand{\lra}{\longrightarrow}
\newcommand{\lhra}{\ensuremath{\lhook\joinrel\relbar\joinrel\rightarrow}}
\newcommand{\hra}{\hookrightarrow}
\newcommand{\brak}[1]{{\llbracket{#1}\rrbracket}}
\newcommand{\sdense}{{\widehat\Box}}
\newcommand{\sdenseother}{\Box}
\newcommand{\ie}{i.\,e.\@\xspace}
\newcommand{\eg}{e.\,g.\@\xspace}
\newcommand{\vs}{vs.\@\xspace}
\newcommand{\resp}{resp.\@\xspace}
\newcommand{\inv}{inv.\@}
\newcommand{\notnot}{\emph{not~not}\xspace}
\newcommand{\affl}{\ensuremath{{\ul{\AA}^1_S}}\xspace}
\newcommand{\afflx}{\ensuremath{{\ul{\AA}^1_X}}\xspace}
\newcommand{\afflt}{\ensuremath{{\ul{\AA}^1_T}}\xspace}
\newcommand{\affla}{\ensuremath{{\ul{\AA}^1_{\Spec A}}}\xspace}
\newcommand{\afflz}{\ensuremath{{\ul{\AA}^1_{\Spec Z}}}\xspace}
\newcommand{\xra}{\xrightarrow}
\newcommand{\stacksproject}[1]{\cite[{\href{http://stacks.math.columbia.edu/tag/#1}{Tag~#1}}]{stacks-project}}
\newcommand{\apart}{\mathrel{\#}}
\newcommand{\fieldext}{\mathrel{|}}

\newenvironment{indentblock}{%
  \list{}{\leftmargin\leftmargin}%
  \item\relax
}{%
  \endlist
}

\newcommand{\XXX}[1]{\textbf{XXX: #1}}
\newcommand{\XXXh}[1]{}

\newcommand{\defeq}{\vcentcolon=}
\newcommand{\defequiv}{\vcentcolon\equiv}
\newcommand{\seq}[1]{\mathrel{\vdash\!\!\!_{#1}}}

\definecolor{gray}{rgb}{0.7,0.7,0.7}

\title{Using the internal language of toposes in algebraic geometry}
\author{Ingo Blechschmidt}
\email{iblech@web.de}

\makeatletter
\counterwithout{section}{chapter}
\counterwithout{footnote}{chapter}
\counterwithout{table}{chapter}
\counterwithout{figure}{chapter}
\patchcmd{\@thm}{\let\thm@indent\indent}{\let\thm@indent\noindent}{}{}
\patchcmd{\@thm}{\thm@headfont{\scshape}}{\thm@headfont{\bfseries}}{}{}
\patchcmd{\@makechapterhead}{\chaptername}{Part}{}{}
\patchcmd{\@chapter}{\chaptername}{Part}{}{}
\patchcmd{\@schapter}{\chaptername}{Part}{}{}
\addto\captionsenglish{\renewcommand\chaptername{Part}}
\def\l@section{\@tocline{1}{12pt}{1pc}{}{\bfseries}}
\def\l@chapter{\@tocline{-1}{12pt}{0pt}{}{\bfseries}}
\renewcommand\thechapter{\Roman{chapter}}
\newcommand{\nocontentsline}[3]{}
\newcommand{\tocless}[1]{\let\addcontentsline=\nocontentsline}
\normalparindent=12pt
\parindent=\normalparindent
\renewenvironment{proof}[1][\proofname]{\par
  \pushQED{\qed}%
  \normalfont \topsep6\p@\@plus6\p@\relax
  \trivlist
  \item[\hskip\labelsep
        \itshape
    #1\@addpunct{.}]\ignorespaces
}{%
  \popQED\endtrivlist\@endpefalse
}
\let\@afterindenttrue\@afterindentfalse
\def\subsection{\@startsection{subsection}{2}%
  {0pt}{.5\linespacing\@plus.7\linespacing}{-.5em}%
  {\normalfont\bfseries}}
\makeatother

\begin{document}

\begin{abstract}
  There are several important toposes associated to a scheme, for instance the
  little and big Zariski toposes. These support an internal mathematical language
  which closely resembles the usual formal language of mathematics, but is ``local
  on the base scheme'':
  For example, from the internal perspective, the structure sheaf looks like an
  ordinary local ring (instead of a sheaf of rings with local stalks) and vector
  bundles look like ordinary free modules (instead of sheaves of modules
  satisfying a certain condition). The translation of internal statements and
  proofs is facilitated by an easy mechanical procedure.

  These expository notes give an introduction to this topic and show how the internal
  point of view can be exploited to give simpler definitions and more conceptual
  proofs of the basic notions and observations in algebraic geometry.
  % For instance, any theorem about modules yields a corresponding theorem about
  % sheaves of modules (with a small caveat).
  We also employ this framework to study the phenomenon that some properties
  spread from points to open neighbourhoods. We give a general sufficient
  condition for this spreading to occur, depending only on the logical
  form of the property in question.

  No prior knowledge about topos theory or formal logic is assumed.
\end{abstract}
% XXX: update

\newcommand{\HRule}{\rule{\linewidth}{.6pt}}

\begin{center}
\thispagestyle{empty}

\vspace*{.06\textheight}
{\scshape\LARGE University of Augsburg\par}\vspace{1.5cm}
\textsc{\Large Doctoral Thesis}\\[0.5cm]

\HRule \\[0.4cm]
{\huge \bfseries Using the internal language of toposes \\ in algebraic geometry\par}\vspace{0.4cm} % Thesis title
\HRule \\[0.5cm]

{\large Ingo Blechschmidt}

\vfill

\vspace{4cm}

\textbf{\textcolor{red}{-- preliminary version --}}
%\large \textit{A thesis submitted in fulfillment of the requirements} \\
%\textit{in the}\\[0.4cm]
%baz

\vfill

{\large April 2017}\\[4cm] % Date
%\includegraphics{Logo} % University/department logo - uncomment to place it

\vfill
\end{center}

%\begin{center}Rough draft in progress, missing explanations, references,
%proofs, and a proper copyediting. I am happy about comments of any
%kind; please direct them to \texttt{iblech@web.de}.\end{center}

\setcounter{tocdepth}{2}
\tableofcontents


\chapter{Basics}

\section{Introduction}

{\tocless

\subsection*{Internal language of toposes}

A \emph{topos} is a category which shares certain categorical properties with
the category of sets; the archetypical example is the category of sets, and
the most important example for the purposes of this thesis is the category of
set-valued sheaves on a topological space.

Any topos~$\E$ supports an \emph{internal language}. This is a device which
allows one to \emph{pretend} that the objects of~$\E$ are plain sets and that
the morphisms are plain maps between sets, even if in fact they are not. For
instance, consider a morphism~$\alpha : X \to Y$ in~$\E$. From the \emph{internal
point of view}, this looks like a map between sets, and we can formulate the
condition that this map is surjective; we write this as
\[ \E \models \forall y\?Y\_ \exists x\?X\_ \alpha(x) = y. \]
The appearance of the colons instead of the usual element signs reminds us that
this expression is not to be taken literally --~$X$ and~$Y$ are objects of~$\E$
and thus not necessarily sets. The definition of the internal language is made
in such a way so that the meaning of this internal statement is that~$\alpha$
is an epimorphism. Similarly, the translation of the internal statement
that~$\alpha$ is injective is that~$\alpha$ is a monomorphism.

Furthermore, we can \emph{reason} with the internal language. There is a
metatheorem to the effect that if some statement~$\varphi$ holds from the
internal point of view of a topos~$\E$ and if~$\varphi$ logically implies some
further statement~$\psi$, then~$\psi$ holds in~$\E$ as well. As a simple
example, consider the elementary fact that the composition of surjective maps
is surjective. Interpreting this statement in the internal language of~$\E$, we
obtain the more abstract result that the composition of epimorphisms in~$\E$ is
epic.

There is, however, a slight caveat to this metatheorem. Namely, the internal
language of a topos is in general only \emph{intuitionistic}, not
\emph{classical}. This means that internally, one can not use the law of
excluded middle~($\varphi \vee \neg\varphi$), the law of double negation
elimination~($\neg\neg\varphi \Rightarrow \varphi$), or the axiom of choice.
For instance, one rendition of the axiom of choice is that any vector space is
free. But it need not be the case that a vector space internal to a topos
is free as seen from the internal perspective: By the technique explained in
this thesis, this would imply the absurd statement that any sheaf of modules on
a reduced scheme is locally free.

The restriction to intuitionistic reasoning is not as confining as it might first
appear. We will discuss its practical consequences below (on
page~\pageref{sect:appreciating-intuitionistic-logic}).


\subsection*{Algebraic geometry}
We apply the internal language of toposes to algebraic geometry in two
different ways, corresponding to the two different toposes associated to a
scheme~$X$: the \emph{little Zariski topos} which is just the topos~$\Sh(X)$ of
set-valued sheaves on~$X$, and the \emph{big Zariski topos} which we introduce
below.

The internal language of the little Zariski topos can be applied as follows.
The structure sheaf~$\O_X$ of a scheme~$X$ is a sheaf of rings in that the sets of
local sections carry ring structures and these ring structures are compatible
with restriction. From the internal point of view of~$\Sh(X)$,
the structure
sheaf~$\O_X$ looks much simpler: It looks just like a plain ring (and
not a sheaf of rings). Similarly, a sheaf of~$\O_X$-modules looks just like a
plain module over that ring.

This allows to import notions and facts from basic linear and commutative
algebra into the sheaf setting. For instance, it turns out that a sheaf
of~$\O_X$-modules is of finite type if and only if, from the internal
perspective, it is finitely generated as an~$\O_X$-module. Now consider the
following fact of linear algebra: If in a short exact sequence of modules the two
outer ones are finitely generated, then the middle one is too. The usual proof of
this fact is intuitionistically acceptable and can thus be interpreted in the
internal language. It then \emph{automatically} yields the following more advanced
proposition: If in a short exact sequence of sheaves of~$\O_X$-modules the
two outer ones are of finite type, then the middle one is too.

This example was not special: \emph{Any (intuitionistically valid) theorem
about modules yields a corresponding theorem about sheaves of modules.}

The internal language machinery thus allows us to understand the basic notions
and statements of scheme theory as notions and statements of linear and
commutative algebra, interpreted in a suitable sheaf topos. This brings
conceptual clarity and reduces technical overhead.

In Section~\ref{sect:internal-language}, we explain how the internal language
machinery works, and then develop in Part~\ref{part:little-zariski} a
\emph{dictionary} between common notions of scheme theory and corresponding
notions of algebra. Once built, this dictionary can be used arbitrarily often.
We stress that no in-depth knowledge of topos theory or categorical logic is
necessary to apply this apparatus.

Two highlights of our approach are the following. If~$X$ is a reduced scheme,
the internal universe of~$\Sh(X)$ has the peculiar feature that~$\O_X$ is
Noetherian and a field, even if~$X$ is not locally Noetherian and (as will
almost always be the case) the local rings~$\O_{X,x}$ are not fields. Linear
and commutative algebra over~$\O_X$ are therefore particularly simple from the
internal point of view. For instance, Grothendieck's generic freeness lemma,
which is usually proved using a somewhat involved series of reduction steps,
admits a short, easy, and conceptual proof with this technique.

To briefly indicate a part of this, let~$\F$ be a sheaf of~$\O_X$-modules of finite
type. A basic version of Grothendieck's generic freeness lemma then states
that~$\F$ is locally free on some dense open subset of~$X$; this fact
is stated in Vakil's lecture notes as an ``important hard
exercise''~\cite[Exercise~13.7.K]{vakil:foag}. In fact, this proposition is just the
interpretation of the following basic statement of intuitionistic linear algebra in
the sheaf topos: Any finitely generated vector space is \emph{not not} free.
The proof of this statement is entirely straightforward.\footnote{Intuitionistically,
the statement that any finitely generated vector space is \emph{free} is stronger than
the doubly negated version and can not be shown. It would imply that any sheaf
of finite type is not only locally free on some dense open subset, but locally
free on the entire space. We discuss this example in more detail in
Section~\ref{sect:upper-semicontinuous-functions} and in particular in
Lemma~\ref{lemma:locally-free-dense}. A proof of Grothendieck's generic
freeness lemma in its full form is given in
Section~\ref{sect:generic-freeness}.}

The second highlight is that we can shed light on the phenomenon that
sometimes, truth of a property at a point~$x$ spreads to some open
neighbourhood of~$x$; and in particular that sometimes, truth of a property at
the generic point spreads to some dense open subset. For instance, if the stalk
of a sheaf of finite type is zero at some point, the sheaf is even zero on some
open neighbourhood; but this spreading does not occur for general sheaves which
may fail to be of finite type.

We formalize this by introducing a \emph{modal operator}~$\Box$ into the
internal language, such that the internal statement~$\Box\varphi$ means
that~$\varphi$ holds on some open neighbourhood of~$x$. Furthermore, we
introduce a simple operation on formulas, the~\emph{$\Box$-translation}
$\varphi \mapsto \varphi^\Box$, such that~$\varphi^\Box$ means that~$\varphi$
holds at the point~$x$. This translation is defined on a purely syntactical
level. The question whether truth at~$x$ spreads to truth on a
neighbourhood can then be formulated in the following way: Does~$\varphi^\Box$
intuitionistically imply~$\Box\varphi$?

This allows to deal with the question in a simpler, more logical way, with the
technicalities of sheaves blinded out. We also give a metatheorem which
covers a wide range of cases. Namely, spreading occurs for all those properties
which can be formulated in the internal language without
using~``$\Rightarrow$'',~``$\forall$'', and~``$\neg$''.

To take up the example above, consider the property of a module~$\F$ being
the zero module. In the internal language, it can be formulated as~$(\forall x\?\F\_ x = 0)$.
Because of the appearance of~``$\forall$'', the metatheorem is not
applicable to this statement. But if~$\F$ is of finite type, there are
generators~$x_1,\ldots,x_n\?\F$ from the internal point of view, and the
condition can be reformulated as~$x_1 = 0 \wedge \cdots \wedge x_n = 0$; the
metatheorem is applicable to this statement.


\subsection*{Synthetic algebraic geometry}
All of the applications mentioned above employ the little Zariski topos of the base
scheme~$X$, the topos of sheaves on the underlying topological space of~$X$.
Its internal language simplifies the treatment of sheaves of rings and modules
over~$X$, but the treatment of~\emph{schemes} over~$X$ is simplified only a
little bit: From the internal point of view of~$\Sh(X)$, a morphism~$T \to X$
of schemes looks like a morphism~$T \to \pt$. Therefore relative scheme theory
is turned into absolute scheme theory (over the ring~$\O_X$), but it still
requires the machinery of locally ringed spaces.

The internal language of the \emph{big} Zariski topos of~$X$ allows for a more far-reaching
change of perspective. It incorporates Grothendieck's functor-of-points
philosophy in order to cast modern algebraic geometry, relative to the arbitrary
base scheme~$X$, in a naive \emph{synthetic} language reminiscient of the classical
Italian school.

The synthetic approach is best explained by contrasting it with the usual
approach to scheme theory, which is to layer it upon some standard form of set theory:
to give a scheme means to firstly give a set of points; then
to describe a topology on this set; and finally to equip the resulting space
with a local sheaf of rings. Basic objects of study in algebraic geometry, such
as closed subschemes of projective spaces, are in this way encoded using a large
amount of machinery.

There is also a somewhat lesser used, but philosophically rewarding and more
``economical'' approach within set theory: Grothendieck's functorial approach.
In this account of scheme theory, to give a scheme means to give a functor
from the category of commutative rings to the category of sets. For instance,
the Fermat scheme is given by the functor
\[ A \longmapsto \{ (x,y,z) \in A^3 \,|\, x^n + y^n - z^n = 0 \}, \]
that is by a \emph{scheme} in the colloquial sense for prescribing a set of
solutions for any ring.

This approach requires fewer preparations and involves
only objects of intrinsic interest to algebraic geometry: $A$-valued points,
where~$A$ ranges over all rings. These tend to be better behaved, for instance in
that the set of~$A$-valued points of a product of schemes is isomorphic to the
product of the sets of~$A$-valued points, and are more fundamental from a
geometric point of view. In contrast, the set-theoretical points of a scheme in
the approach using locally ringed spaces actually parameterize irreducible
closed subsets, not points in an intuitive sense.

However, the description of basic objects can still be somewhat involved in the
functorial approach. For instance, while the functor associated to
projective~$n$-space is given on fields by the simple expression
\[ \begin{array}{r@{}c@{}l}
  K &{}\longmapsto{}& \text{the set of lines through the origin in~$K^{n+1}$} \\
  && \qquad \cong \{ [x_0:\cdots:x_n] \,|\, \text{$x_i \neq 0$ for some~$i$} \},
\end{array} \]
on general rings it is given by
\[ \begin{array}{r@{}c@{}l}
  A &{}\longmapsto{}& \text{the set of quotients~$A^n \twoheadrightarrow P$,
  where~$P$ is projective,} \\
  && \qquad \text{modulo isomorphism}.
\end{array} \]

On the one hand, typically only field-valued points admit a simple description.
On the other hand, the $A$-valued points for more general rings~$A$ are crucial
in order to impart a meaningful sense of cohesion on the field-valued
points and therefore can't simply be dropped.\footnote{For instance,
let~$\ul{\AA}^1 : A \mapsto A$ be the functor associated to the affine line.
Yoneda's lemma guarantees that the set of morphisms~$\ul{\AA}^1 \to \ul{\AA}^1$
in the functor category~$[\Ring,\Set]$ is in canonical bijection with the
set~$\ZZ[U]$, as one would expect: Algebraic functions~$\AA^1 \to \AA^1$ should
be given by polynomials. (The discussion could also be relativized so that the
answer is the polynomial ring~$k[U]$, where~$k$ is some base field.) However, if we calculate the
set of morphisms in~$[\mathrm{Field},\Set]$ we obtain~$\int_{K \in
\mathrm{Field}} \Hom(K,K)$, a set which contains pathological functions such as
some which permute the elements of the prime fields in arbitrary ways.}

We can resolve the tension by incorporating an automatic management of the
\emph{stage of definition}, the rings~$A$ such that we're
considering~$A$-valued points, into our language. Such a language is provided
by the internal language of the big Zariski topos. It allows for the Fermat
scheme to be given by the naive expression
\[ \{ (x,y,z) \? (\ul{\AA}^1)^3 \,|\, x^n + y^n - z^n = 0 \} \]
and for projective~$n$-space to be given by either of the expressions
\begin{multline*}
  \qquad\text{the set of lines through the origin in~$(\ul{\AA}^1)^{n+1}$}
  \quad\text{or} \\
  \{ [x_0:\cdots:x_n] \,|\, \text{$x_i \neq 0$ for some~$i$} \}.\qquad
\end{multline*}
This is not a specialized trick to give short descriptions of some schemes:
Like with the internal universe of any topos, the full power of intuitionistic
logic is available to reason about the objects constructed in this way.

We can thus add an approach to the list of ways of giving a rigorous foundation
to algebraic geometry, the synthetic approach which layers scheme theory not
upon a classical set theory, but rather directly encodes schemes as sets and
morphisms of schemes as maps of sets in the nonclassical universe provided by
the big Zariski topos of a base scheme. We can therefore use a simple,
element-based language to talk about schemes.

This is similar to synthetic approaches to other fields of mathematics, such as
differential geometry~\cite{kock:sdg}, domain
theory~\cite{hyland:synthetic-domain-theory}, computability
theory~\cite{bauer:synthetic-computability-theory}, and more recently and very successfully homotopy
theory~\cite{hott}. The synthetic approaches allow in each case to encode the
objects of study directly as (nonclassical) sets, with geometric,
domain-theoretic, computability-theoretic, or homotopic structure being
automatically provided for.

The implicit algebro-geometric structure has visible consequences on the
internal universe of the big Zariski topos and endows it with a distinctive
algebraic flavor. For instance, the statement
``\emph{any} map~$\ul{\AA}^1 \to \ul{\AA}^1$ is a polynomial function''
holds from the internal point of view. This is also a property which sets the
internal universe of the big Zariski topos apart from the toposes studied in
synthetic differential geometry.

If one is content with building upon classical scheme theory, the big Zariski
topos~$\Zar(X)$ of a base scheme~$X$ can be constructed as the topos of
sheaves on the Grothendieck site~$\Sch/X$ of~$X$-schemes.\footnote{Some care is
needed in order to avoid set-theoretical issues of size. We discuss this fine
point in Section~\ref{sect:proper-choice-of-site}. If one is interested in
foundational questions and doesn't simply want to use the big Zariski topos in
order to employ its convenient internal language, one can rest assured that
there's a way to construct it without resorting to classical scheme theory.} % XXX section
Explicitly, an object of~$\Zar(X)$ is a functor~$F : (\Sch/X)^\op \to \Set$
satisfying the gluing condition with respect to Zariski coverings:
If~$T = \bigcup_i U_i$ is a cover of an~$X$-scheme~$T$ by open subsets, the
diagram
\[ F(T) \longrightarrow \prod_i F(U_i) \xbigtoto{} \prod_{j,k} F(U_j \cap U_k) \]
should be an equalizer diagram. A premier example of an object of~$\Zar(X)$ is
the functor~$\ul{Y}$ of points associated to an~$X$-scheme~$Y$, mapping
an~$X$-scheme~$T$ to~$\Hom_X(T, Y)$. It satisfies the gluing condition since
one can glue morphisms of schemes in the Zariski topology.

The object~$\ul{\AA}^1$ which already appeared is the functor of points
of the affine line over~$X$, the~$X$-scheme~$\AA^1_X \defeq X \times_{\Spec\ZZ}
\ZZ[U]$. Its value on an~$X$-scheme~$T$ is
\[ \afflx(T) = \Hom_X(T, \AA^1_X) \cong \Hom_{\Spec\ZZ}(T, \Spec\ZZ[U]) \cong
\Gamma(T,\O_T). \]
This object has a canonical structure as a ring object in~$\Zar(X)$. In fact,
from the internal point of view of~$\Zar(X)$, it is a local ring and even a
field in the sense that nonzero elements are invertible. In the case~$X =
\Spec\ZZ$, this was first observed by Kock~\cite{kock:univ-proj-geometry}. At
the same time, it is not a reduced ring -- a feat possible only in an
intuitionistic context. This curious interplay is quite important, since the
sets
\[ \{ x\?\afflx \,|\, x = 0 \} \quad\text{and}\quad
  \{ x\?\afflx \,|\, x^2 = 0 \} \]
should and do describe two different~$X$-schemes: the first is isomorphic
to~$X$ while the second is an infinitesimal thickening of~$X$, the vanishing scheme
of~$U^2$ in~$\AA^1_X$. In contrast, the sets~$\{ x\?\afflx \,|\, x \neq 0 \}$
and~$\{ x\?\afflx \,|\, x^2 \neq 0 \}$ should and do coincide. By the field
property, both conditions are equivalent to~$x$ being invertible.
% XXX forward reference to tangent bundle

% XXX: Mention that all known cool properties of A^1 follow from synthetic
% quasicoherence! Maybe also in the introduction to Part III.

Modal operators are useful in the big topos setting as well. For instance,
there is a modal operator~$\Box_\text{ét}$ in the big Zariski topos such that
the internal statement~$\Box_\text{ét} \varphi$ roughly means that~$\varphi$
holds on an étale covering and such that the translated
formula~$\varphi^{\Box_\text{ét}}$ means that~$\varphi$ holds in the \emph{big étale
topos} familiar from étale cohomology. In this way, we can access the internal
universe of the big étale topos from within the big Zariski topos. The
ring~$\afflx$ enjoys additional properties when studied in the étale topos, for
instance it is separably closed.
\XXXh{wording}


\subsection*{Limitations} The internal language is \emph{local}, in the sense
that if~$X = \bigcup_i U_i$ is an open covering and an internal statement
holds in the sheaf toposes~$\Sh(U_i)$, it holds in~$\Sh(X)$ as well. On the one
hand, this property is very useful. But on the other hand, it causes an inherent
limitation of the internal language:
Global properties of sheaves of modules like ``being generated by global
sections'', ``being ample'', or ``having vanishing sheaf cohomology'' and global properties of schemes like ``being
quasicompact'' can \emph{not} be
expressed in the internal language.

Thus for global considerations, the internal language of~$\Sh(X)$ is only
useful in that local subparts can be simplified. Also, some global features
reflect themselves in certain metaproperties of the internal language. For
instance, a scheme is quasicompact if and only if the internal language
has a weak version of the so-called disjunction property of mathematical
logic (Section~\ref{sect:compactness}).

The locality limitation only refers to locality with respect to the base
scheme. For instance, the little and big Zariski toposes of~$X$ \emph{can}
distinguish between affine and projective~$n$-space over~$X$, even though these
are locally isomorphic.

The internal languages of both toposes can be used on a case-by-case
basis, employing them as part of longer arguments in the context of ordinary scheme
theory where it's useful to do so. However, if one wants to stay solely in one
of the provided internal universes and not use ordinary scheme theory at all,
then one will of course run into the further limitation that internal scheme
theory, as put forward in this thesis, is only developed to a small amount.


\subsection*{Introductory literature} This text is intended
to be self-contained, requiring only basic knowledge of scheme theory. In
particular, we assume no prior familiarity with topos theory or formal logic.
But if the interested reader is so inclined, she will find a gentle
introduction to topos theory in an article by
Leinster~\cite{leinster:introduction}. Standard references for the internal
language of a topos include the book of Mac~Lane and
Moerdijk~\cite[Chapter~VI]{moerdijk-maclane:sheaves-logic}, the book of
Borceux~\cite[Chapter~6]{borceux:handbook3}, and Part~D of
Johnstone's Elephant~\cite{johnstone:elephant}. In the 1970s, there was a
flurry of activity on applications of the internal language. An article by
Mulvey~\cite{mulvey:repr} of this time gives a very accessible
introduction to the topic, culminating in an internal proof of the Serre--Swan
theorem (with just one external ingredient needed).


\subsection*{Related work} The internal language of toposes was applied to algebraic geometry before. For
instance, Wraith used it to construct (and verify the universal property
of) the little étale topos of a scheme by internally developing the theory of
strict henselization~\cite{wraith:generic-galois-theory}. However, to the best
of my knowledge, systematically building a dictionary between external and
internal notions has not been attempted before, and the use of modal operators
to study the spreading of properties from points to neighbourhoods seems to be
new as well.

Brandenburg put forward a related program of internalization in his PhD
thesis~\cite{brandenburg:tensor-foundations}. However, he internalizes
constructions of algebraic geometry not in toposes, but in tensor categories.
There is some overlap in working out precise universal properties, particularly
when dealing with the big Zariski topos.

In other branches of mathematics, the internal language of toposes is used as well. For
instance, there is an ongoing effort in mathematical physics to understand
quantum mechanical systems from an internal point of view: To any quantum
mechanical system, one can associate a so-called Bohr topos containing an
internal mirror image of the system. This mirror image looks like a
system of classical mechanics from the internal perspective, and therefore
tools like Gelfand duality can be used to construct an internal
phase space for the system~\cite{bohr1,bohr2}.

In stochastics, the usefulness of an internal language was recently stressed by
Tao~\cite{tao:analysis-rel-measure-space}. Such a language makes the
common notational practice of dropping the explicit dependence of the
value~$X(\omega)$ of a random variable on the sample~$\omega$ completely
rigorous and simplifies the basic theory. Tao also highlighted how a suitable
language can be used to simplify ``$\varepsilon$/$\delta$ management'' in
analysis~\cite{tao:cheap-nsa}. Furthermore, there is a topos-theoretic approach to
measure theory, in which the sheaf of measurable real functions on
a~$\sigma$-algebra looks like the ordinary set of real numbers from an internal point
of view~\cite{jackson:sheaf-theoretic-measure-theory}; this has applications in
noncommutative geometry~\cite{henry:measure-theory-boolean-toposes}.

Intuitionistic methods have found many applications in computer science.
Recently, the internal language of a topos of trees and a suitable modal
operator was used to study guarded recursion, encompassing, for instance, an
internal Banach fixed-point theorem~\cite{birkedal:al:sgdt}.

In constructive mathematics, the internal language of toposes is routinely used
to obtain models of intuitionistic theories fulfilling certain anti-classical
axioms. For instance, there are toposes in which the axiom ``any map~$\RR \to
\RR$ is continuous'' (appropriately formulated) holds~\cite{kock:sdg,moerdijk:reyes:models}
and toposes in which the Church--Turing thesis ``any map~$\NN \to \NN$ is
computable'' holds (certain realizability toposes).
The internal language can also be used to extract computational content
out of classical constructions. To cite just one recent example, Mannaa and
Coquand used it to implement algorithms for working with the algebraic closure
of an arbitrary field of characteristic zero~\cite{mannaa:coquand:alg-closure}.

One way this thesis contributes to the program of constructive
mathematics is that intuitionistic mathematics gains new areas of application.
For instance, the constructive account of the theory of Krull dimension was
originally developed to remove Noetherian hypotheses, extract computational meaning, and
simplify proofs~\cite{dyn:krull-integral,dyn:char-krull}. It can now also be used to
reason about the dimension of schemes, since the topological dimension of a
scheme~$X$ coincides with the Krull dimension of the structure sheaf~$\O_X$
regarded as an ordinary ring from the internal perspective of~$\Sh(X)$
(Section~\ref{sect:krull-dimension}).

We obtained a second contribution to constructive mathematics as a byproduct of
deducing transfer principles which relate a module over a ring~$A$ with its
induced quasicoherent sheaf on~$\Spec A$: Using the internal language of the
little Zariski topos we can algorithmically turn certain non-constructive
arguments concerning prime ideals into constructive ones. We discuss this in
Section~\ref{sect:eliminating-prime-ideals}; it is related to the
\emph{dynamical methods in algebra} explored by Coquand, Coste, Lombardi, Roy,
and others~\cite{clr:dynamicalmethod,cl:logical}.

Caramello uses topos theory to build bridges between different mathematical
subjects, in a certain precise sense~\cite{caramello:1,caramello:2}. She
exploits that toposes can admit presentations by different sites. Our
contribution is certainly related to her grand research program in spirit, but since we
focus only on specific presentations of a few specific toposes associated to
schemes, there is as yet no direct technical connection.

\XXX{mention and explain: Mulvey/Burden, Vickers, Awodey, Coquand, ...}

\XXX{further work, ...}

\XXX{Mention insights on relative spectrum, Mulvey's "obscure statement", ...}

\XXX{Mention that the internal language unlocks new intrinsic
characterizations and descriptions, which would otherwise be too unwieldy to
formulate or think about.}


\subsection*{Notational convention} Occasionally, when quantifying, we use
colons instead of element signs not only in internal statements, but also in
external ones; particularly when we want to stress that a discussion takes place
in an intuitionistic context.

}


\section{The internal language of a sheaf topos}\label{sect:internal-language}

At its heart, the internal language of a topos provides a coherent way of
translating any mentions of set-theoretical elements to
\emph{generalized elements}, carefully keeping track of and adapting
the stage of definition. We want to illustrate this with a simple example
before giving the formal definition.

A map~$f : X \to Y$ of sets is injective if and only if
\begin{equation}\label{inj}
  \forall x,x' \in X\_ f(x) = f(x') \Longrightarrow x = x'.
\end{equation}
This condition can not only be interpreted in~$\Set$, but in any category~$\C$ whose
objects are structured sets and whose morphisms are maps between the underlying
sets. If we want to go beyond such kind of categories, we have to restate the
condition in purely category-theoretic language:
\begin{equation}\label{injcat}
  \forall (1 \xra{x} X), (1 \xra{x'} X)\_ f \circ x = f \circ x'
  \Longrightarrow x = x'.
\end{equation}
This condition makes sense in all categories which contain a terminal
object~$1$, and is equivalent to condition~\eqref{inj} in the case~$\C = \Set$.
This has a deeper reason: The one-element set~$1 = \{ \star \}$ is a
\emph{separator} of~$\Set$, that is objects of~$\Set$ are uniquely determined
by their \emph{global elements}, morphisms from the terminal object.

However, in categories in which the terminal object is not a separator,
condition~\eqref{injcat} is not very meaningful. This is for instance the case
if~$\C$ is the category~$\Sh(X)$ of set-valued sheaves on a topological
space~$X$. Global elements of a sheaf~$\F$ are in natural one-to-one
correspondence with global sections~$s \in \F(X)$ (hence the name), whereby
condition~\eqref{injcat} only states that~$f$ is \emph{injective on global
sections}. Since many interesting sheaves admit no or only few global sections,
this statement is typically not very substantial.

A basic tenet of category theory is therefore to not only refer to global
elements~$1 \to X$, but also to \emph{generalized elements}~$A \to X$,
where~$A$ ranges over all objects. The domain~$A$ is called the \emph{stage of
definition} in this context. Bearing this principle in mind, a better
translation of the injectivity condition is the statement
\begin{equation}\label{injgen}
  \forall \text{objects $A$ in~$\C$}\_ \forall (A \xra{x} X), (A \xra{x'\!} X) \text{ in~$\C$}\_\
  f \circ x = f \circ x' \ \Longrightarrow\ x = x'.
\end{equation}
This statement expresses that~$f$ is a monomorphism and therefore
correctly captures the structural essence of injectivity.

Unlike this manual translation guided by trial and error and categorical
philosophy, the internal language provides a purely mechanical translation
scheme.  It is fully formal, can be analyzed rigorously, works smoothly with
arbitrarily convoluted statements, and most importantly can be trusted to
support \emph{reasoning}: If a statement formulated in a naive element-based
language intuitionistically implies a further such statement, then the
translation of the former implies the translation of the latter.

The power of the internal language doesn't unfold in basic situations like with
the example above, where one can easily translate statements and even proofs by
hand. It unfolds when considering more complex statements. For instance, the
short proof of Grothendieck's generic freeness lemma promised in the
introduction rests on the internal statement~``any ideal
of~$\O_{\Spec(R)}[U_1,\ldots,U_n]$ is \notnot finitely generated'',
where~$R$ is a reduced ring. For the proof of Grothendieck's generic freeness
lemma it's not necessary to actually perform the translation of this statement into external
language, but for definiteness we display the translation here nevertheless:
\begin{indentblock}\label{page:convoluted-statement}
For any element~$f \in R$ and any (not necessarily quasicoherent) sheaf of
ideals~$\J \hookrightarrow \O_{\Spec(R)}[U_1,\ldots,U_n]|_{D(f)}$: If
\begin{indentblock}
for any element~$g \in R$ the condition that
\begin{indentblock}
the sheaf~$\J$ is of finite type on~$D(g)$
\end{indentblock}
implies that~$g = 0$,
\end{indentblock}
then~$f = 0$.
\end{indentblock}
This statement is obviously quite convoluted, and its proof is even more so;
therefore it probably wouldn't occur to one to base a proof of Grothendieck's
generic freeness lemma on this statement. The internal language is thus of real
use here. We'll expand on this example in Section~\ref{sect:noetherian} and in
Section~\ref{sect:generic-freeness}.\footnote{The statement can be proven by
hand, but it's much simpler to only verify the case~$n = 0$ (and even reduce
this case to simple other properties which~$\O_{\Spec(R)}$ enjoys from the
internal point of view) and then to apply Hilbert's basis theorem. Hilbert's
basis theorem is famous for admitting only a nonconstructive proof, and
nonconstructive proofs can't be translated by the internal language machinery;
but this is only true for the conclusion ``any ideal is finitely generated''.
The intuitionistically weaker conclusion ``any ideal is \notnot finitely
generated'' does admit a constructive proof, and is all what's needed here.}


\subsection{Internal statements}
Let~$X$ be a topological space. Later, $X$ will be the underlying space of a
scheme. The meaning of internal statements is given by a set of rules, the
\emph{Kripke--Joyal semantics} of the topos of sheaves on~$X$.

\begin{defn}\label{defn:kripke-joyal}The meaning of
\[ U \models \varphi \quad\text{(``$\varphi$ holds on $U$'')} \]
for open subsets~$U \subseteq X$ and formulas~$\varphi$ over~$U$ is given by
the rules listed in Table~\ref{table:kripke-joyal}, recursively in the structure of~$\varphi$.
In a \emph{formula over~$U$} there may appear sheaves defined on~$U$ as domains
of quantifications,~$U$-sections of sheaves as terms, and morphisms of sheaves
on~$U$ as function symbols. If~$V \subseteq U$ is an open subset, then formulas
over~$U$ can be pulled back to formulas over~$V$. The symbols~``$\top$'' and~``$\bot$'' denote truth
and falsehold, respectively. The universal and existential quantifiers come in
two flavors: for bounded and unbounded quantification.
The translation of~$U \models \neg\varphi$ does not have to be separately defined, since
negation can be expressed using other symbols: $\neg\varphi \defequiv (\varphi
\Rightarrow \bot)$. If we want to emphasize the particular topos, we write
\[ \Sh(X) \models \varphi \quad\Ll\quad X \models \varphi. \]
\end{defn}

\begin{table}
  \centering
  \[ \renewcommand{\arraystretch}{1.3}\begin{array}{@{}lcl@{}}
    U \models s = t \? \F &\Ll& s|_U = t|_U \in \Gamma(U, \F) \\
    U \models s \in \G &\Ll& s|_U \in \Gamma(U,\G) \quad\quad\text{($\G$ a
    subsheaf of~$\F$, $s$ a section of~$\F$)} \\
    U \models \top &\Ll& U = U \text{ (always fulfilled)} \\
    U \models \bot &\Ll& U = \emptyset \\
    U \models \varphi \wedge \psi &\Ll&
      \text{$U \models \varphi$ and $U \models \psi$} \\
    U \models \bigwedge_{j \in J} \varphi_j &\Ll&
      \text{for all~$j \in J$: $U \models \varphi_j$} \quad\quad\text{($J$ an
      index set)} \\
    U \models \varphi \vee \psi &\Ll&
      \hcancel{\text{$U \models \varphi$ or $U \models \psi$}}{0pt}{3pt}{0pt}{-2pt} \\
    && \text{there exists a covering $U = \bigcup_i U_i$ such that for all~$i$:} \\
    && \quad\quad \text{$U_i \models \varphi$ or $U_i \models \psi$} \\
    U \models \bigvee_{j \in J} \varphi_j &\Ll&
      \hcancel{\text{$U \models \varphi_j$ for some~$j \in J$}}{0pt}{3pt}{0pt}{-2pt}
      \quad\quad\text{($J$ an index set)} \\
    && \text{there exists a covering $U = \bigcup_i U_i$ such that for all~$i$:} \\
    && \quad\quad \text{$U_i \models \varphi_j$ for some~$j \in J$} \\
    U \models \varphi \Rightarrow \psi &\Ll&
      \hcancel{\text{$U \models \varphi$ implies $U \models \varphi$}}{0pt}{3pt}{0pt}{-2pt} \\
    && \text{for all open~$V \subseteq U$:
      $V \models \varphi$ implies $V \models \psi$} \\
    U \models \forall s \? \F\_ \varphi(s) &\Ll&
      \text{for all sections~$s \in \Gamma(V, \F)$ on open $V \subseteq U$: $V \models
      \varphi(s)$} \\
    U \models \exists s \? \F\_ \varphi(s) &\Ll&
      \hcancel{\text{there exists a section~$s \in \Gamma(U,\F)$ such that $U
      \models \varphi(s)$}}{0pt}{3pt}{0pt}{-2pt} \\
    &&
      \text{there exists an open covering $U = \bigcup_i U_i$ such that for all~$i$:} \\
    && \quad\quad \text{there exists~$s_i \in \Gamma(U_i, \F)$ such that
    $U_i \models \varphi(s_i)$} \\
    U \models \forall \F\_ \varphi(\F) &\Ll&
      \text{for all sheaves $\F$ on open $V \subseteq U$: $V \models \varphi(\F)$} \\
    U \models \exists \F\_ \varphi(\F) &\Ll&
      \text{there exists an open covering $U = \bigcup_i U_i$ such that for all~$i$:} \\
    && \quad\quad \text{there exists a sheaf~$\F_i$ on~$U_i$ such that
    $U_i \models \varphi(\F_i)$}
  \end{array} \]
  \caption{\label{table:kripke-joyal}The Kripke--Joyal semantics of a sheaf
  topos.}
\end{table}

\begin{rem}The last two rules in Table~\ref{table:kripke-joyal}, concerning
\emph{unbounded quantification}, are not part of the classical Kripke--Joyal
semantics. They are part of Mike Shulman's stack semantics~\cite{shulman:stack},
a slight extension. They are needed so that we can formulate universal
properties in the internal language.
\end{rem}

\begin{ex}\label{ex:injective-surjective}
Let~$\alpha : \F \to \G$ be a morphism of sheaves on~$X$. Then
$\alpha$ is a monomorphism of sheaves if and only if, from the internal
perspective,~$\alpha$ is simply an injective map:
\allowdisplaybreaks
\begin{align*}
  & X \models \speak{$\alpha$ is injective} \\[0.5em]
  \Longleftrightarrow\
  & X \models \forall s\?\F\_ \forall t\?\F\_ \alpha(s) = \alpha(t) \Rightarrow s = t \\[0.5em]
  \Longleftrightarrow\ &
    \text{for all open~$U \subseteq X$, sections $s \in \Gamma(U, \F)$:} \\
  &\qquad\qquad \text{for all open~$V \subseteq U$, sections $t \in \Gamma(V, \F)$:} \\
  &\qquad\qquad\qquad\qquad
      V \models \alpha(s) = \alpha(t) \Rightarrow s = t \\[0.5em]
  \Longleftrightarrow\ &
    \text{for all open~$U \subseteq X$, sections $s \in \Gamma(U, \F)$:} \\
  &\qquad\qquad \text{for all open~$V \subseteq U$, sections $t \in \Gamma(V, \F)$:} \\
  &\qquad\qquad\qquad\qquad
      \text{for all open~$W \subseteq V$:} \\
  &\qquad\qquad\qquad\qquad\qquad\qquad
        \text{$\alpha_W(s|_W) = \alpha_W(t|_W)$ implies $s|_W = t|_W$} \\[0.5em]
  \Longleftrightarrow\ &
    \text{for all open~$U \subseteq X$, sections $s, t \in \Gamma(U, \F)$:} \\
  &\qquad\qquad
        \text{$\alpha_U(s|_U) = \alpha_U(t|_U)$ implies $s|_U = t|_U$} \\[0.5em]
  \Longleftrightarrow\ &
    \text{$\alpha$ is a monomorphism of sheaves}
\end{align*}
The corner quotes ``$\speak{\ldots}$'' indicate that translation into formal
language is left to the reader. Similarly,~$\alpha$ is an epimorphism of
sheaves if and only if, from the internal perspective,~$\alpha$ is a
surjective map. Notice that injectivity and surjectivity are
notions of a simple element-based language. The Kripke--Joyal semantics
takes care to properly handle \emph{all} sections, not only global ones.
\end{ex}

The rules are not all arbitrary. They are finely concerted to make the
following two propositions true, which are crucial for a proper appreciation of the
internal language.

\begin{prop}[Locality of the internal language]
\label{prop:locality-of-the-internal-language}
Let~$U = \bigcup_i U_i$ be covered by open subsets. Let~$\varphi$
be a formula over~$U$. Then
\[ U \models \varphi \qquad\text{iff}\qquad
  \text{$U_i \models \varphi$ for each $i$}. \]
\end{prop}
\begin{proof}Induction on the structure of~$\varphi$. Note that the canceled
rules would make this proposition false.\end{proof}

As a corollary, one may restrict the open coverings and universal
quantifications in the the definition of the Kripke--Joyal semantics
(Table~\ref{table:kripke-joyal}) to open subsets of some basis of the topology.
For instance, if~$X$ is a scheme, one may restrict to affine open subsets.

Furthermore, the proposition shows that the internal language is monotone in
the following sense: If~$U \models \varphi$, and~$V$ is an open subset of~$U$,
then~$V \models \varphi$. (This follows by applying the proposition to the
trivial covering~$U = V \cup U$.)

\begin{prop}[Soundness of the internal language]
\label{prop:soundness-of-the-internal-language}
If a formula~$\varphi$ implies a further formula~$\psi$ in intuitionistic logic, then
$U \models \varphi$ implies $U \models \psi$.
\end{prop}
\begin{proof}
Proof by induction on the structure of formal intuitionistic proofs; we are to
show that any inference rule of intuitionistic logic is satisfied by the
Kripke--Joyal semantics. For instance, there is the following rule governing
disjunction:
\begin{quote}
If~$\varphi \vee \psi$ holds, and both $\varphi$ and $\psi$ imply a further
formula~$\chi$, then~$\chi$ holds.
\end{quote}
So we are to prove that if~$U \models \varphi \vee \psi$, $U \models (\varphi
\Rightarrow \chi)$, and $U \models (\psi \Rightarrow \chi)$, then $U \models \chi$.
This is done as follows: By assumption, there exists a covering~$U = \bigcup_i
U_i$ such that on each~$U_i$, $U_i \models \varphi$ or $U_i \models \psi$.
Again by assumption, we may conclude that~$U_i \models \chi$ for each~$i$. The statement
follows because of the locality of the internal language.

A complete list of which rules are to prove is
in Appendix~\ref{appendix:inference-rules}.
\end{proof}

In particular, if a formula~$\psi$ has an unconditional intuitionistic proof,
then~$U \models \psi$.

The restriction to intuitionistic logic is really necessary at this point. We
will encounter many examples of classically equivalent internal statements whose
translations using the Kripke--Joyal semantics are wildly different. To
anticipate just one example, the statement
\[ X \models \speak{$\F$ is finite free}, \]
referring to a sheaf~$\F$ of~$\O_X$-modules, means that~$\F$ is finite locally
free. The statement
\[ X \models \neg\neg(\speak{$\F$ is finite free}) \]
instead means that~$\F$ is finite locally free on a dense open subset of~$X$.

In
particular, our treatment of modal operators to understand spreading of
properties from points to neighbourhoods depends on having the ability to make
finer distinctions -- distinctions which are not visible in classical logic.
In Section~\ref{sect:appreciating-intuitionistic-logic} there is a discussion of what the restriction to
intuitionistic logic amounts to in practice.

\XXXh{Put rules into an appendix and give some explanation regarding contexts
etc. Don't forget the rules for $\in$, $\bigwedge$, $\bigvee$.}

Because of the multitude of quantifiers, literal translations of internal statements
can sometimes get slightly unwieldy. There are simplification rules for certain
often-occuring special cases:
\begin{prop}\label{prop:simplification}
    \[ \renewcommand{\arraystretch}{1.3}\begin{array}{@{}lcl@{}}
      U \models \forall s\?\F\_ \forall t\?\G\_ \varphi(s,t)
      &\Longleftrightarrow&
      \text{for all open~$V \subseteq U$,} \\
      && \text{sections~$s \in \Gamma(V,\F)$, $t \in \Gamma(V,\G)$:
      $V \models \varphi(s,t)$} \\[0.3em]
      U \models \forall s\?\F\_ \varphi(s) \Rightarrow \psi(s)
      &\Longleftrightarrow&
      \text{for all open~$V \subseteq U$, sections~$s \in \Gamma(V,\F)$:} \\
      &&\qquad\qquad \text{$V \models \varphi(s)$ implies $V \models \psi(s)$}
      \\[0.3em]
      U \models \exists!s\?\F\_ \varphi(s)
      &\Longleftrightarrow&
      \text{for all open~$V \subseteq U$,} \\
      &&
      \text{there is exactly one section~$s \in \Gamma(V,\F)$ with:} \\
      &&\qquad\qquad V \models \varphi(s)
    \end{array} \]
\end{prop}
\begin{proof}Straightforward. By way of example, we prove the existence claim
in the ``only if'' direction of the last rule. (This rule formalizes
the saying ``unique existence implies global existence''.) By definition of~$\exists!$, it
holds that
\[ U \models \exists s\?\F\_ \varphi(s)
  \qquad\text{and}\qquad
  U \models \forall s,t\?\F\_ \varphi(s) \wedge \varphi(t) \Rightarrow s = t. \]
Let~$V \subseteq U$ be an arbitrary open subset. Then there exist local
sections~$s_i \in \Gamma(V_i,\F)$ such that~$V_i \models \varphi(s_i)$, where~$V
= \bigcup_i V_i$ is an open covering. By the locality of the internal language,
on intersections it holds that~$V_i \cap V_j \models \varphi(s_i)$, so by the
uniqueness assumption, it follows that the local sections agree on intersections.
They therefore glue to a section~$s \in \Gamma(V,\F)$. Since~$V_i \models
\varphi(s)$ for all~$i$, the locality of the internal language allows us to
conclude that~$V \models \varphi(s)$.
\end{proof}

\begin{rem}Note that~$\Sh(X) \models \neg\varphi$ is in general a much stronger
statement than merely saying that~$\Sh(X) \models \varphi$ does not hold:
The former always implies the latter (unless~$X = \emptyset$, in which case
\emph{any} internal statement is true), but the converse does not hold: The
former statement means that~$U = \emptyset$ is the \emph{only} open subset on
which~$\varphi$ holds, that is that~$\varphi$ holds \emph{nowhere}. In
contrast, the statement~$\Sh(X) \not\models \varphi$ only means that~$\varphi$
does \emph{not hold everywhere}.\end{rem}


\subsection{Internal constructions}
\label{sect:internal-constructions}
The Kripke--Joyal semantics defines the
interpretation of internal \emph{statements}. The interpretation of internal
\emph{constructions} is given by the following definition.

\begin{defn}\label{defn:interpretation-internal-constructions}
The interpretation of an internal construction~$T$
is denoted by~$\brak{T} \in \Sh(X)$ and given by the following rules.
\begin{itemize}\item If~$\F$ and~$\G$ are sheaves, $\brak{\F \times \G}$ is the
categorical product of~$\F$ and~$\G$ (\ie their product as presheaves).
\item If~$\F$ and~$\G$ are sheaves, $\brak{\F \amalg \G}$ is the categorical
coproduct of~$\F$ and~$\G$, \ie the sheafification of the presheaf
$U \mapsto \Gamma(U,\F) \amalg \Gamma(U,\G)$.
\item If~$\F$ is a sheaf, the interpretation~$\brak{\P(\F)}$ of the power set
construction is the sheaf given by
\[ \text{$U \subseteq X$ open} \quad\longmapsto\quad \{ \G \hookrightarrow \F|_U \}, \]
\ie sections on an open set~$U$ are subsheaves of~$\F|_U$ (either literally
or isomorphism classes of arbitrary monomorphisms into~$\F|_U$).
\item If~$\F$ is a sheaf and~$\varphi(s)$ is a formula containing a free
variable~$s\?\F$, the interpretation~$\brak{\{s\?\F\,|\,\varphi(s)\}}$ is given
by the subpresheaf of~$\F$ defined by
\[ \text{$U \subseteq X$ open} \quad\longmapsto\quad \{ s \in \Gamma(U,\F) \ |\
  U \models \varphi(s) \}. \]
Note that by the locality of the internal language, this presheaf is in fact a
sheaf.
\end{itemize}
\end{defn}

The definition is made in such a way that, from the internal perspective, the
constructions enjoy their expected properties. For instance, it holds that
\[ \Sh(X) \models
  \bigl(\forall x\?\brak{\{s\?\F \,|\, \varphi(s)\}}\_ \psi(x)\bigr)
  \Longleftrightarrow
  \bigl(\forall x\?\F\_ \varphi(x) \Rightarrow \psi(x)\bigr). \]
We gloss over several details here. See~\cite[Section~D4.1]{johnstone:elephant} for
a proper treatment.

Morphisms can internally be constructed by appealing to the \emph{principle of
unique choice}: Let~$\varphi(s,t)$ be a formula with free variables of
type~$s\?\F$, $t\?\G$. Assume
\[ \Sh(X) \models \forall s\?\F\_ \exists!t\?\G\_ \varphi(s,t). \]
Then there is one and only one morphism~$\alpha : \F \to \G$ of sheaves such
that for any local section~$s \in \Gamma(U,\F)$, $\Sh(X) \models
\varphi(s,\alpha(s))$. This follows from the meaning of unique existence with
the Kripke--Joyal semantics (Proposition~\ref{prop:simplification}).

An important application is showing that two sheaves~$\F$ and~$\G$ are
isomorphic (usually as objects with more structure, for instance sheaves of
modules). To this end, it suffices to give a formula~$\varphi(s,t)$ satisfying,
in addition to the condition above, the condition
$\Sh(X) \models \forall t\?\G\_ \exists! s\?\F\_ \varphi(s,t)$,
expressing that the induced morphism~$\alpha$ is a bijective map from the
internal perspective. Note that this implies the statement
\[ \Sh(X) \models \exists \alpha\?\HOM(\F,\G)\_ \speak{$\alpha$ is bijective},
\]
but this statement is strictly weaker: Its interpretation with the
Kripke--Joyal semantics is that the sheaves~$\F$ and~$\G$ are \emph{locally}
isomorphic.


\subsection{Geometric formulas and
constructions}\label{sect:geometric-formulas-and-constructions}
In formal and categorical logic so-called geometric formulas play a
special role. They are named that way because, in a sense which can be made
precise, their meaning is preserved under pullback with geometric morphisms.
\begin{defn}\label{defn:geometric-formulas}
A formula is \emph{geometric} if and only if it consists only of
\[ {=} \quad {\in} \quad {\top} \quad {\bot} \quad {\wedge} \quad {\vee} \quad
{\bigvee} \quad {\exists}, \]
but not~``$\bigwedge$'' nor ``$\Rightarrow$'' nor~``$\forall$'' (and thus
not~``$\neg$'' either, since negation is defined using~``$\Rightarrow$'').
A \emph{geometric implication} is a formula of the form
\[ \forall \cdots \forall\_ (\cdots) \Rightarrow (\cdots) \]
with the bracketed subformulas being geometric.
\end{defn}
The \emph{parameters} of a formula~$\varphi$ are the sheaves
being quantified over, sections of sheaves appearing as terms, and morphisms of
sheaves appearing as function symbols in~$\varphi$.
We say that a formula~$\varphi$ holds \emph{at a point~$x \in X$} if and only
if the formula obtained by substituting all parameters in~$\varphi$ with their
stalks at~$x$ holds in the usual mathematical sense.

\begin{lemma}\label{lemma:geometric-stalk-neighbourhood}
Let~$x \in X$ be a point. Let~$\varphi$ be a geometric formula (over some open
neighbourhood~V of~$x$).
Then~$\varphi$ holds at~$x$ if and only if there exists an open neighbourhood~$U
\subseteq X$ of~$x$ (contained in~V) such that~$\varphi$ holds on~$U$.
\end{lemma}
\begin{proof}This is a very general instance of the phenomenon that sometimes,
truth at a point spreads to truth on a neighbourhood. It can be proven by
induction on the structure of~$\varphi$, but we will give a more conceptual
proof later (Corollary~\ref{cor:geometric-spreading}).
\end{proof}

This lemma is in fact a very useful metatheorem. We will properly discuss its
significance in Section~\ref{sect:spreading}. For now, we just use it to prove a
simple criterion for the internal truth of a geometric implication; we will
apply this criterion many times.

\begin{cor}\label{cor:geometric-implication}
A geometric implication holds on~$X$ if and only if it holds at
every point of~$X$.\end{cor}
\begin{proof}For notational simplicity, we consider a geometric implication of
the form
\[ \forall s\?\F\_ \varphi(s) \Rightarrow \psi(s). \]
For the ``only if'' direction, assume that this formula holds on~$X$ and let~$x
\in X$ be an arbitrary point. Let~$s_x \in \F_x$ be the germ of an arbitrary
local section~$s$ of~$\F$ and assume that~$\varphi(s)$ holds at~$x$. Then by
the lemma, it follows that~$\varphi(s)$ holds on some open neighbourhood of~$x$. By
assumption,~$\psi(s)$ holds on this neighbourhood as well. Again by the
lemma,~$\psi(s)$ holds at~$x$.

For the ``if'' direction, assume that the geometric implication holds at every
point. Let~$U \subseteq X$ be an arbitrary open subset and let~$s \in
\Gamma(U,\F)$ be a local section such that~$\varphi(s)$ holds on~$U$. By the
lemma and the locality of the internal language, to show that~$\psi(s)$ holds
on~$U$, it suffices to show that~$\psi(s)$
holds at every point of~$U$. This is clear, since again by the
lemma,~$\varphi(s)$ holds at every point of~$U$.
\end{proof}

\begin{ex}Injectivity and surjectivity are geometric implications (surjectivity
can be spelled~$\forall y\?\G\_ (\top \Rightarrow \exists x\?\F\_ \alpha(x) =
y)$). Thus the corollary gives a deeper reason for the well-known fact that a
morphism of sheaves is a monomorphism \resp an epimorphism if and only if it is
stalkwise injective \resp surjective.\end{ex}

A construction is \emph{geometric} if and only if it commutes with pullback
under arbitrary geometric morphisms. We do not want to discuss the notion of
geometric morphisms here; suffice it to say that calculating the stalk at a
point~$x \in X$ is an instance of such a pullback. Among others, the following
constructions are geometric:
\begin{itemize}
\item finite product: $(\F \times \G)_x \cong \F_x \times \G_x$
\item finite coproduct: $(\F \amalg \G)_x \cong \F_x \amalg \G_x$
\item arbitrary coproduct: $(\coprod_i \F_i)_x \cong \coprod_i (\F_i)_x$
\item set comprehension with respect to a \emph{geometric} formula~$\varphi$:
\[ \brak{\{ s\?\F \,|\, \varphi(s) \}}_x \cong \{ [s]\in\F_x \,|\,
\text{$\varphi(s)$ holds at $x$} \} \]
\item free module: $(\R\langle \F \rangle)_x \cong \R_x\langle \F_x
\rangle$ ($\R$ a sheaf of rings, $\F$ a sheaf of sets)
\item localization of a module: $\F[\S^{-1}]_x \cong \F_x[\S_x^{-1}]$
\end{itemize}
Note that compatibility with taking stalks is not sufficient for geometricity.
It is just the most easily visualized requirement.
The following constructions are not in general geometric:
\begin{itemize}
\item arbitrary product
\item set comprehension with respect to a non-geometric formula
\item powerset
\item internal Hom: $\HOM(\F,\G)_x \not\cong \Hom(\F_x,\G_x)$
\end{itemize}


\subsection{Appreciating intuitionistic logic}
\label{sect:appreciating-intuitionistic-logic}
The principal (and only) difference between classical and intuitionistic logic
is that in classical logic, the axioms schemes of \emph{excluded middle} and
\emph{double negation elimination} are added.
\[ \varphi \vee \neg\varphi \qquad\qquad \neg\neg\varphi \Rightarrow \varphi \]
A classically trained mathematician might legitimately wonder why one should
drop these axioms: Are they not obviously true? The pragmatic answer to this
question is that the translations of these axioms with the Kripke--Joyal
semantics are, except for uninteresting special cases of the base space~$X$,
plainly false -- irrespective of one's philosophical convictions. Therefore the
internal language is in general only sound with respect to intuitionistic logic and
not with respect to classical logic. Concretely, there is the following
proposition.
\begin{prop}The internal language of
a~$\mathrm{T}_1$-space~$X$ is \emph{Boolean}, \ie it verifies the classical
axiom schemes displayed above, if and only if~$X$ is discrete.
The internal language of an irreducible or locally Noetherian scheme~$X$ is Boolean if and only if~$X$ has
dimension~$\leq 0$.
\end{prop}
\begin{proof}\label{prop:lang-boolean}
The internal language of~$\Sh(X)$ is Boolean if and only if for
any open subset~$U \subseteq X$ it holds that~$U$ is the only dense open subset
of~$U$. This can be checked manually, by using the definition of the
Kripke--Joyal semantics, but we'll be able to give a more conceptual proof
later (Lemma~\ref{lemma:boolean-dense}). The first claim is then an exercise in
point-set topology, while the second is more difficult
(Corollary~\ref{cor:boolean-dim0}).
\end{proof}

However, there is also a more satisfying answer, which furthermore
illuminates how to intuitively picture intuitionistic mathematics.
Namely, when doing intuitionistic mathematics, we use the same formal symbols as classically, but with
\emph{a different intended meaning}. For instance, the classical reading of an
existential statement like~$\exists x\?A\_ \varphi(x)$ is that there exists
some element~$x \? A$ with the property~$\varphi(x)$. In contrast, its
intuitionistic reading is that such an element can actually be
\emph{constructed}, \ie explicitly given in some form. This is a much stronger
statement. Classically, a proof that it is \emph{not} the case that such an
element does \emph{not} exist -- formally $\neg\neg \exists x\?A\_ \varphi(x)$
(or, equivalently even in intuitionistic mathematics, $\neg\forall x\? A\_
\neg\varphi(x)$) -- suffices to demonstrate the existential statement; this is not
so in intuitionistic mathematics.

Similarly, the intuitionistic meaning of a disjunction~$\varphi \vee \psi$ is
not only that one of the disjuncts is true, but that one can explicitly state
which case holds. It is in general not enough to show that it is impossible
that both~$\varphi$ and~$\psi$ fail.

In this picture, it is obvious that one should not adopt the law of excluded
middle or the principle of double negation elimination as axioms. Note that we
do not \emph{reject} those axioms in the sense of postulating their
converses either, we simply don't use them. Therefore any intuitionistically
true result is also true classically. In fact, for some special instances,
these two classical axioms do hold intuitionistically. For example, any natural
number is zero or is not zero -- this is not a triviality, but can be proven by
induction.\footnote{The analogous statement about real numbers cannot be
shown. Intuitively, for a number given by a decimal expansion starting
with~$0.0000\ldots$ one cannot decide whether the string of zeros will continue
indefinitely or whether eventually a non-zero digit will occur. This argument
can be made rigorous. The analogous statement about algebraic numbers
\emph{can} be proven; the information contained in a witness of algebraicity (a
monic polynomial which the given number is a zero of) suffices to make the
case distiction~\cite[Chapter~VI.1, p.~140]{mines-richman-ruitenburg:constructive-algebra}.}

A consequence of not adopting these axioms is that proofs by contradiction are
not generally justified; they are intuitionistically acceptable only for those
statements which can be proven to be true or false. Note that a proof of a
\emph{negated formula} is not the same as a proof by contradiction. For
instance, the usual proof that~$\sqrt{2}$ is not rational is
intuitionistically perfectly fine: From the assumption that~$\sqrt{2}$ is
rational one deduces a contradiction~($\bot$). This is exactly the definition
of~$\neg(\speak{$\sqrt{2}$ is rational})$.

A more positive consequence of not adopting the law of excluded middle and the
principle of double negation elimination is that intuitionistically, we can
make \emph{finer distinctions}. For instance, for a formula~$\varphi$, the doubly
negated formula~$\neg\neg\varphi$ (``\notnot~$\varphi$'') is a certain kind of weakening of~$\varphi$:
If~$\varphi$ holds, then~$\neg\neg\varphi$ does as well, while the converse can
not be shown in general.\footnote{A detailed proof of the correct implication
goes as follows: Assume~$\varphi$. We are to show~$\neg\neg\varphi$, \ie
$(\neg\varphi \Rightarrow \bot)$. So assume~$\neg\varphi$, we are to
show~$\bot$. Since~$\varphi$ and~$\varphi \Rightarrow \bot$,~$\bot$ indeed
follows.} An example from everday life runs as follows: If in the morning you
can't find the key for your appartment, but you know that it must hide
somewhere since you used it to open the door in the evening before, you
intuitionistically know~$(\neg\neg\exists x\_ \speak{the key is at position
$x$})$, but you cannot claim the unnegated proposition. One cannot model this
distinction with pure classical logic.

Double negation also has a concrete geometric meaning with the
Kripke--Joyal semantics. Namely,~$X \models \neg\neg\varphi$ holds if and
only if there is a dense open subset~$U$ of~$X$ such that~$U \models \varphi$.
This is of course a weaker statement than~$X \models \varphi$.
In Section~\ref{sect:modalities}, we will discuss this fact and other
\emph{modal operators} in more detail. For instance, there is a similarly defined modal
operator~$\Box$ such that~$X \models \Box\varphi$ if and only if there is an
open neighbourhood~$U$ of a given point~$x$ such that~$U \models \varphi$. Also
there is a different operator~$\Box$ such that~$X \models \Box\varphi$ if and only
if~$\varphi$ holds on a scheme-theoretically dense open subset.

For future reference, note that if~$\varphi \Rightarrow \psi$,
then also~$\neg\neg\varphi \Rightarrow \neg\neg\psi$; and note that weakening
twice has no further effect, \ie~$\neg\neg\neg\neg\varphi \Leftrightarrow
\neg\neg\varphi$.\footnote{In fact, negating thrice is the same as negating
once: Assume~$\neg\neg\neg\varphi$. We are to show~$\neg\varphi$. So
assume~$\varphi$, we are to show~$\bot$. Since~$\varphi$,~$\neg\neg\varphi$.
By~$\neg\neg\neg\varphi$,~$\bot$ follows.}

A classical mathematician might then ask which classical results are valid
intuitionistically. The answer is that in linear and commutative algebra, most
of the basic theorems stay valid, provided one exercises some caution in
formulating them (for instance, one should not arbitrarily weaken assumptions
by introducing double negations). This is because the proofs of these
statements are usually direct; if intuitionistically unacceptable case
distictions do occur, they can often be eliminated by streamlining the proof.

Consider as a simple example the proposition that the kernel of a linear map is
a linear subspace. The case distiction ``either the kernel consists just of the
zero vector, in which case the claim is trivial, or otherwise \ldots'' is not
intuitionistically acceptable, but it can be entirely dispensed with: The proof
for the general case works in the special case just as well.

Finally, we should clarify the status of the axiom of choice. This axiom, which
is strictly speaking not part of classical logic, but of a classical set
theory, is not accepted in an intuitionistic context: By \emph{Diaconescu's
theorem}, it implies the law of excluded middle in presence of the other axioms
of set theory.

Standard references for intuitionistic algebra are a textbook by Mines,
Richman and
Ruitenburg~\cite{mines-richman-ruitenburg:constructive-algebra} and a textbook
by Lombardi~\cite{lombardi:quitte:constructive-algebra}, the standard
reference for intuitionistic analysis is a book by Bishop and
Bridges~\cite{bishop-bridges:constructive-analysis}. Further explanations and
pointers to relevant literature can be found in an expository article and a
recorded lecture by Bauer~\cite{bauer:int-mathematics,bauer:video}. A
recent survey of intuitionistic logic from a historical and logical point of
view is~\cite{melikhov:intuitionistic-logic}.

\begin{rem}For ease of exposition, we work in a classical metatheory. This
means that we allow ourselves to occasionally use the law of excluded middle
and the axiom of choice when reasoning \emph{about} the internal language. In
particular, we have the theory of schemes as commonly presented at our
disposal. But we should note that this concession is really a cop out, and that
it would be better to develop an intuitionistic theory of schemes. If this were
done, one could extend our approach to understand morphisms of schemes from an
internal point of view -- a morphism~$Y \to X$ would internally look like a
morphism~$Y \to \pt$. See Section~\ref{sect:relative-spectrum} for details.\end{rem}


\chapter{The little Zariski topos}\label{part:little-zariski}

\section{Sheaves of rings}

Recall that a \emph{sheaf of rings} can be categorically described as a
sheaf of sets~$\R$ together with maps of sheaves $+, \cdot : \R \times \R \to
\R$, $- : \R \to \R$, and global elements~$0, 1$ such that certain axioms hold.
For instance, the axiom on the commutativity of addition is rendered in
diagrammatic form as follows:
\[ \xymatrix{
  \R \times \R \ar[rr]^{\mathrm{swap}} \ar[rd]_{+} && \R \times \R \ar[ld]^{+} \\
  & \R
} \]

From the internal perspective, a sheaf of rings looks just like a plain ring.
This is the content of the following proposition.

\begin{prop}\label{prop:rings-internally}
Let~$X$ be a topological space. Let~$\R$ be a sheaf of sets on~$X$.
Let~$+, \cdot : \R \times \R \to \R$ and $- : \R \to \R$ be maps of sheaves and let~$0, 1$ be
global elements of~$\R$. Then these data define a sheaf of rings if and only
if, from the internal perspective, these data fulfill the usual equational ring
axioms.\end{prop}
\begin{proof}We only discuss the commutativity axiom. The internal statement
\[ \Sh(X) \models \forall x,y\?\R\_ x + y = y + x \]
means that for any open subset~$U \subseteq X$ and any local sections~$x,y \in
\Gamma(U,\R)$, it holds that~$x + y = y + x \in \Gamma(U,\R)$. This is
precisely the external commutativity condition.
\end{proof}

\begin{lemma}\label{lemma:internal-invertibility}
Let~$X$ be a topological space. Let~$\R$ be a sheaf of rings
on~$X$. Let~$f$ be a global section of~$\R$. Then the following statements are
equivalent:
\begin{enumerate}
\item $f$ is invertible from the internal point of view, \ie $\Sh(X) \models
\exists g\?\R\_ fg = 1$.
\item $f$ is invertible in all stalks~$\R_x$.
\item $f$ is invertible in~$\Gamma(X,\R)$.
\end{enumerate}
\end{lemma}
\begin{proof}Since invertibility is a geometric implication, the equivalence of
the first two statements is clear. Also, it is obvious that the third statement
implies the other two. For the remaining direction, note that the
uniqueness of inverses in rings can be proven intuitionistically. Therefore, if~$f$ is invertible
from the internal point of view, it actually holds that
\[ \Sh(X) \models \exists! g\?\R\_ fg = 1. \]
Since unique internal existence implies global existence
(Proposition~\ref{prop:simplification}), this shows that the first statement
implies the third.
\end{proof}


\subsection{Reducedness}\label{sect:reducedness} Recall that a scheme~$X$ is \emph{reduced} if and only
if all stalks~$\O_{X,x}$ are reduced rings. Since the condition on a ring~$R$
to be reduced is a geometric implication,
\[ \forall s\?R\_ \Bigl(\bigvee_{n \geq 0} s^n = 0\Bigr) \Longrightarrow s = 0, \]
we immediately obtain the following characterization of reducedness in the
internal language:
\begin{prop}\label{prop:reduced-ring}
A scheme~$X$ is reduced iff, from the internal point of view, the
ring~$\O_X$ is reduced.\end{prop}


\subsection{Locality} Recall the usual definition of a local ring: a ring
possessing exactly one maximal ideal. This is a so-called \emph{higher-order
condition} since it involves quantification over subsets. It is also not of a
geometric form. Therefore, for our purposes, it is better to
adopt the following elementary definition of a local ring.
\begin{defn}A \emph{local ring} is a ring~$R$ such that~$1 \neq 0$ in~$R$ and
for all~$x,y \? R$
\[ \text{$x+y$ invertible} \quad\Longrightarrow\quad
  \text{$x$ invertible}\ \vee\ \text{$y$ invertible}. \]
\end{defn}
In classical logic, it is an easy exercise to show that this definition is
equivalent to the usual one. In intuitionistic logic, we would need to be
more precise in order to even state the question of equivalence, since
intuitionistically, the notion of a maximal ideal bifurcates into several
non-equivalent notions.\footnote{For instance, should a maximal ideal~$\mmm$ be
such that if~$\nnn$ is any ideal with~$\mmm \subseteq \nnn \subsetneq (1)$,
then~$\mmm = \nnn$? Or should the condition be that if~$\nnn$ is any ideal
with~$\mmm \subseteq \nnn$, then~$\mmm = \nnn$ or~$\nnn = (1)$?
Intuitionistically, the latter condition is stronger than the former.}
This is a common phenomenon in intuitionistic
mathematics: Classically equivalent notions may bifurcate into related but
inequivalent notions intuitionistically, each having a unique character and
yielding slightly different theories.


\begin{prop}\label{prop:local-ring}
In the internal language of a scheme~$X$ (or a locally ringed
space), the ring~$\O_X$ is a local ring.\end{prop}
\begin{proof}The stated locality condition is a conjunction of two geometric
implications (the first one being~$1 = 0 \Rightarrow \bot$, the second being
the displayed one) and holds on each stalk.\end{proof}

\begin{rem}When first exposed to locally ringed spaces, one might ask why the
requirement is that the \emph{stalks}~$\O_{X,x}$ are local rings, instead of the
easier-to-define sets of sections~$\O_X(U)$. This question has of course a good
geometric answer. Using the internal language, it also has a purely formal
answer: The requirement that the stalks are local rings is precisely the
requirement that the ring~$\O_X$ is a local ring from the perspective of the
internal language of~$X$.
\end{rem}


\subsection{Field properties} From the internal point of view, the structure
sheaf~$\O_X$ of a scheme~$X$ is \emph{almost} a field, in the sense that any
element which is not invertible is nilpotent. This is a genuine property of
schemes, not shared with arbitrary locally ringed spaces. It is also a specific
feature of the internal universe: Neither the local rings~$\O_{X,x}$ nor the
rings of local sections~$\Gamma(U,\O_X)$ have this property in general.

\begin{prop}\label{prop:neginvnilpotent}Let~$X$ be a scheme. Then
\[ \Sh(X) \models \forall s\?\O_X\_ \neg(\speak{$s$ invertible}) \Rightarrow
\speak{$s$ nilpotent}. \]
\end{prop}
\begin{proof}By the locality of the internal language and since~$X$ can be
covered by open affine subsets, it is enough to show that for any affine
scheme~$X = \Spec A$ and any global function~$s \in \Gamma(X,\O_X) = A$ it holds
that
\[ X \models \neg(\speak{$s$ invertible}) \quad\text{implies}\quad
  X \models \speak{$s$ nilpotent}. \]
The meaning of the antecedent is that any open subset on which~$s$ is
invertible is empty. This implies in particular that the standard open subset~$D(s)$ is
empty. This means that~$s$ is an element of any prime ideal of~$A$, thus
nilpotent, and therefore implies the a priori weaker statement~$X \models \speak{$s$
nilpotent}$ (which would allow~$s$ to have different indices of nilpotency on
an open covering).
\end{proof}

\begin{rem}In classical logic, the statement ``not invertible implies
nilpotent'' is equivalent to ``any element is invertible or nilpotent''.
However, in intuitionistic logic, the latter is strictly stronger than the
former. We will see in the next section
(Corollary~\ref{cor:scheme-dimension-zero}) that the structure sheaf of a
scheme fulfills the latter condition if and only if the scheme is
zero-dimensional (or empty).\end{rem}

\begin{cor}\label{cor:field-reduced}
Let~$X$ be a scheme. If~$X$ is reduced, the ring~$\O_X$ is a field
from the internal point of view, in the sense that
\[ \Sh(X) \models \forall s\?\O_X\_ \neg(\speak{$s$ invertible}) \Rightarrow
s=0. \]
Conversely, if~$\O_X$ is a field in this internal sense, then~$X$ is reduced.\end{cor}
\begin{proof}We can prove this purely in the internal language: It suffices to
give an intuitionistic proof of the fact that a local ring which satisfies the
condition of the previous proposition fulfills the stated field condition if
and only if it is reduced. This is straightforward.
\end{proof}

This field property is very useful. We will put it to good use when giving a
simple proof of the fact that~$\O_X$-modules of finite type on a reduced scheme
are locally free on a dense open subset (Lemma~\ref{lemma:locally-free-dense}).
Note that the field property only holds in the precise form as stated;
the classically equivalent condition that any element is invertible or zero is
intuitionistically stronger. This is an instance of the already remarked upon
phenomenon of intuitionistic bifurcation of notions.

The observation that the structure sheaf is (almost) a field is attributed by
Tierney to Mulvey~\cite[p.~209]{tierney:spectrum}.
Tierney also states that ``its precise significance is still somewhat
obscure'' (ibid). We think that it's significant as a special case of the
following more general proposition,
which says that we can deduce a certain unconditional
statement from the premise that, under the assumption that some element~$f\?\O_X$ is invertible, an element~$s\?\O_X$ is zero. This is
interesting on its own, but will be of particular importance in understanding
quasicoherence from the internal point of view (Section~\ref{sect:qcoh}) and
interpreting the relative spectrum as an internal spectrum
(Section~\ref{sect:relative-spectrum}).

\begin{prop}\label{prop:cond-zero}
Let~$X$ be a scheme. Then
\[ \Sh(X) \models
  \forall f\?\O_X\_
  \forall s\?\O_X\_
  (\speak{$f$ \inv} \Rightarrow s = 0) \Longrightarrow
  \textstyle
  \bigvee_{n \geq 0} f^n s = 0. \]
\end{prop}
\begin{proof}It is enough to show that for any affine scheme~$X = \Spec A$ and
any global functions~$f, s \in A$ such that
\[ X \models (\speak{$f$ \inv} \Rightarrow s = 0), \]
it holds that $X \models \textstyle \bigvee_{n \geq 0} f^n s = 0$. This
indeed follows, since by assumption such a function~$s$ is zero on~$D(f)$, \ie $s$
is zero as an element of~$A[f^{-1}]$.
\end{proof}

Proposition~\ref{prop:neginvnilpotent} follows from this proposition by
setting~$s \defeq 1$.


\subsection{Krull dimension}\label{sect:krull-dimension}
Recall that the \emph{Krull dimension} of a
ring is usually defined as the supremum of the lengths of strictly
ascending chains of prime ideals. As with the classical definition of a local ring,
this definition does not lead to a well-behaved notion in an intuitionistic
context. Furthermore, it is a higher-order condition, so interpreting it
with the Kripke--Joyal semantics is a bit unwieldy.

Luckily, there is an elementary definition of the Krull dimension which works
intuitionistically and which is classically equivalent to the usual notion. It
was found by Coquand and Lombardi, building upon work by
Joyal and Español~\cite{dyn:krull-integral,dyn:char-krull}, and can be
used to give a short proof that~$\dim k[X_1,\ldots,X_n] = n$, where~$k$ is a
field~\cite{dyn:krull-dim-polynomial-ring}.

\begin{defn}Let~$R$ be a ring. A \emph{complementary sequence} for a
sequence~$(a_0,\ldots,a_n)$ of elements of~$R$ is a sequence~$(b_0,\ldots,b_n)$
such that the following inclusions of radical ideals hold:
\[ \renewcommand{\arraystretch}{1.3}
\left\{\begin{array}{@{}rcl@{}}
  \sqrt{(1)} &\subseteq& \sqrt{(a_0,b_0)} \\
  \sqrt{(a_0 b_0)} &\subseteq& \sqrt{(a_1,b_1)} \\
  \sqrt{(a_1 b_1)} &\subseteq& \sqrt{(a_2,b_2)} \\
  &\vdots \\
  \sqrt{(a_{n-1} b_{n-1})} &\subseteq& \sqrt{(a_n,b_n)} \\
  \sqrt{(a_n b_n)} &\subseteq& \sqrt{(0)}
\end{array}\right. \]
The ring~$R$ is \emph{of Krull dimension~$\leq n$} if
and only if for any sequence~$(a_0,\ldots,a_n)$ there exists a
complementary sequence. (The ring~$R$ is trivial if and only if it is
of Krull dimension~$\leq -1$.)
\end{defn}
Note that unlike the usual definition, this definition posits only a condition
on elements and not on ideals. It is thus of a simpler logical form.
(The radical ideals appear only for convenience. We will dispose of them in the
proof of Proposition~\ref{prop:dimension-scheme-ox}.)
Also note that we do not define the Krull dimension of a ring as some natural
number (this is intuitionistically not possible for general rings). Instead, we
only define what it means for the Krull dimension to be less than or equal to
a given natural number.

For the following, no intuition about the definition is needed; however, we
feel that some motivation might be of use. Recall that we can picture inclusions of
radical ideals geometrically by considering standard open subsets~$D(f) = \{
\ppp \in \Spec R \,|\, f \not\in \ppp \}$: The inclusion~$\sqrt{(f)} \subseteq
\sqrt{(g,h)}$ holds if and only if~$D(f) \subseteq D(g) \cup D(h)$, and
intersections are calculated by products, \ie~$D(f) \cap D(g) = D(fg)$.

The condition that~$(b_0,\ldots,b_n)$ is complementary to~$(a_0,\ldots,a_n)$
thus means that~$D(a_0)$ and~$D(b_0)$ cover all of~$\Spec R$; that their
intersection is covered by~$D(a_1)$ and~$D(b_1)$; that in turn their
intersection is covered by~$D(a_2)$ and~$D(b_2)$; \ldots; and that finally, the
intersection of~$D(a_n)$ and~$D(b_n)$ is empty.

For the special case~$n = 0$, the condition that~$R$ is of Krull
dimension~$\leq 0$ means that for any element~$a_0$ there exists an
element~$b_0$ such that~$D(a_0)$ and~$D(b_0)$ cover~$\Spec R$ and are disjoint.

The definition of the Krull dimension can be written in such a way as to mimic the
definition of the inductive Menger--Urysohn dimension of topological
spaces~\cite[Section~1]{dyn:krull-integral}.

\begin{thm}Let~$R$ be a ring.
\begin{enumerate}
\item In classical logic, the ring~$R$ is
of Krull dimension~$\leq n$ if and only if its Krull dimension
as usually defined using chains of prime ideals is less than or equal to~$n$.
\item If the ring~$R$ is
of Krull dimension~$\leq n$, the radical of any finitely generated ideal is
equal to the radical of some ideal which can be generated by~$n+1$ elements.
This holds intuitionistically, and there is an explicit algorithm for computing
the reduced set of generators from the given ones. (Kronecker's theorem)
\end{enumerate}
\end{thm}
\begin{proof}See~\cite[Theorem~1.2]{dyn:krull-integral} for the first
statement. The proof relies on the observation that~$\dim R \leq n$ if and only
if~$\dim R[S_x^{-1}] \leq n-1$ for all~$x \in R$, where~$S_x = x^\NN (1+xR)
\subseteq R$. We put the second statement only to demonstrate that the
definition of the Krull dimension is constructively sensible. It follows from
the identity~$\sqrt{(x,a_0,\ldots,a_n)} =
\sqrt{(a_0-xb_0,\ldots,a_n-xb_n)}$, where~$(b_0,\ldots,b_n)$ is a complementary
sequence for~$(a_0,\ldots,a_n)$.
\end{proof}

We can apply the constructive theory of Krull dimension to the structure
sheaf~$\O_X$ of a scheme~$X$ as follows. Note that the condition that a
scheme~$X$ has dimension exactly~$n$ (in the usual sense using ascending chains
of closed irreducible subsets) is not local -- the dimension may vary on
an open cover; therefore it is not possible to characterize this condition in
the internal language. However, the condition that the dimension of~$X$ is less
than or equal to~$n$ \emph{is} local, thus there is hope that it can be
internalized. And indeed, this is the case.

\begin{prop}\label{prop:dimension-scheme-ox}
Let~$X$ be a scheme. Then:
\[ \dim X \leq n \quad\Longleftrightarrow\quad
  \Sh(X) \models \speak{$\O_X$ is of Krull dimension~$\leq n$}
  \]
\end{prop}
\begin{proof}
% Recall that the topological dimension of~$X$ is defined as the
% supremum of the lengths of strictly ascending chains of irreducible closed
% subsets. It can be calculated as the supremum of the local dimensions~$\dim_x X
% \defeq \inf\{\dim U \,|\, \text{$U$ open neighbourhood of~$x$} \}$, where~$x$
% ranges over all points of~$X$. The local dimension can be characterized
% algebraically: $\dim_x X = \dim \O_{X,x}$.

A condition of the form~``$\sqrt{(f)} \subseteq \sqrt{(g,h)}$''
like in the constructive definition of the Krull dimension is not a geometric
formula when taken on face value. However, it is equivalent to a geometric
condition, namely to
\[ \exists a,b\?\O_X\_ \bigvee_{m \geq 0} f^m = ag + bh. \]
Therefore the condition~$\speak{$\O_X$ is of Krull
dimension~$\leq n$}$ is (equivalent to) a geometric implication and thus holds
internally if and only if it holds at every point~$x \in X$. This in turn means that the
Krull dimension of any stalk~$\O_{X,x}$ is less than or equal to~$n$. This is
equivalent to the (Krull) dimension of~$X$ being less than or equal to~$n$.
\end{proof}

We will state and prove a generalization of this lemma about the dimension of closed
subschemes later, as Lemma~\ref{lemma:dim-closed-subscheme}.

If~$X$ is a reduced scheme, we have seen in Corollary~\ref{cor:field-reduced}
that~$\O_X$ is a field from the internal perspective, in the sense that
non-invertible elements are zero. But fields are well-known to be of Krull
dimension zero. Why is this not a contradiction to the proposition just proven?
Intuitionistically, the notion of a field bifurcates into several
non-equivalent notions:
\begin{enumerate}
\item ``Any element which is not invertible is zero.''
\item ``Any element which is not zero is invertible.''
\item ``Any element is either zero or invertible.''
\end{enumerate}
Only fields in the sense~(3) are automatically of Krull dimension zero.
Fields in the weaker senses can have higher Krull dimension, as exhibited by
the structure sheaf of reduced schemes with positive dimension.

For the following corollary, note that if a scheme~$X$ is not empty, $\dim X
\leq 0$ is equivalent to~$\dim X = 0$.
\begin{cor}\label{cor:scheme-dimension-zero}
Let~$X$ be a scheme. Then:
\[ \dim X \leq 0 \quad\Longleftrightarrow\quad
  \Sh(X) \models \forall s\?\O_X\_ \speak{$s$ \inv} \vee \speak{$s$ nilpotent}.
  \]
If furthermore~$X$ is reduced, this is further equivalent to~$\O_X$ being a
field in the strong sense that any element of~$\O_X$ is invertible or zero.
\end{cor}
\begin{proof}By the proposition and the fact that~$\O_X$ is a local ring from
the internal perspective, this is an immediate consequence of
interpreting the following standard fact of ring theory in the internal
language of~$\Sh(X)$: A local ring~$R$ is of Krull
dimension~$\leq 0$ if and only if any element of~$R$ is invertible or
nilpotent.

It is well-known that this holds classically; to make sure that it
holds intuitionistically as well (so that it can be used in the internal
universe), we give a proof of the ``only if'' direction. Let~$a \? R$ be
arbitrary. By assumption on the Krull dimension, there exists an element~$b \?
R$ such that~$\sqrt{(1)} \subseteq \sqrt{(a,b)}$ and~$\sqrt{(ab)} =
\sqrt{(0)}$. The latter means that~$ab$ is nilpotent. Since~$R$ is local, the
former implies that~$a$ is invertible or that~$b$ is invertible. In the first
case, we are done. In the second case, it follows that~$a$ is nilpotent, so we
are done as well.
\end{proof}

As a further corollary note the curious fact that the classicality of the
internal language of~$\Sh(X)$, where~$X$ is a scheme, is tightly coupled with
the properties of the ring~$\O_X$: Internally, the law of excluded middle and
the principle of double negation elimination are ``almost equivalent'' to the
Krull dimension of~$\O_X$ being~$\leq 0$.
\begin{cor}\label{cor:boolean-dim0}
Let~$X$ be a scheme. If the internal language of~$\Sh(X)$ is Boolean, then
$\dim X \leq 0$. The converse holds if~$X$ is irreducible or locally Noetherian.
\end{cor}
\begin{proof}
We show that any element of~$\O_X$ is invertible or nilpotent, therefore
verifying the hypothesis of the previous corollary.
Let~$s\?\O_X$ be given. By assumption, either~$s$ is invertible or~$s$ is not
invertible. In the latter case~$s$ is nilpotent by
Proposition~\ref{prop:neginvnilpotent}.

We defer the converse direction to
Proposition~\ref{prop:boolean-dim0-continued} since we don't want to interrupt
the exposition here with a certain necessary technical condition.
\end{proof}


\subsection{Integrality}\label{sect:integrality}
In intuitionistic logic, the notion of an integral
domain bifurcates into several inequivalent notions. The following two are
important for our purposes:
\begin{defn}A ring~$R$ is an \emph{integral domain in the weak sense} if and
only if~$1 \neq 0$ in~$R$ and
\[ \forall x,y\?R\_ xy = 0 \Longrightarrow (x = 0) \vee (y = 0). \]
A ring~$R$ is an \emph{integral domain in the strong sense} if and only if~$1
\neq 0$ in~$R$ and
\[ \forall x\?R\_ x = 0 \vee \speak{$x$ is regular}, \]
where~$\speak{$x$ is regular}$ means that~$xy = 0$ implies~$y = 0$ for any~$y \?
R$.\end{defn}

For the following result, recall that a scheme~$X$ (or a ringed space) is
\emph{integral at a point~$x \in X$} if and only if~$\O_{X,x}$ is an integral
domain (in either sense, since we have adopted a classical metatheory).

\begin{prop}\label{prop:internal-integrality}
Let~$X$ be a ringed space. Then:
\begin{enumerate}
\item $X$ is integral at all points if and only if, internally,~$\O_X$ is an
integral domain in the weak sense.
\item If~$X$ is even a locally Noetherian scheme, then~$\O_X$ is an integral
domain in the weak sense iff it is an integral domain in the strong sense from
the internal point of view.
\end{enumerate}
\end{prop}
\begin{proof}The condition on a ring to be an integral domain in the weak sense
is a conjunction of two geometric implications,~``$1 = 0 \Rightarrow \bot$''
and the implication displayed in the definition. Therefore the first statement
is obvious.

For the second statement, note that the condition on a function~$f \in
\Gamma(U,\O_X)$ to be regular from the internal perspective is open: It holds
at a point~$x \in U$ if and only if it holds on some open neighbourhood of~$x$.
We will give a proof of this specific feature of locally Noetherian schemes
later on, when we have developed appropriate machinery to do so easily
(Proposition~\ref{prop:regularity-spreading}). In any case, this openness
property was the essential ingredient for the equivalence between ``holding
internally'' and ``holding at every point''
(Corollary~\ref{cor:geometric-implication}). Therefore~$\O_X$ is an integral
domain in the strong sense from the internal point of view if and only if all
local rings~$\O_{X,x}$ are integral domains. By the first statement, this is
equivalent to~$\O_X$ being an integral domain in the weak sense from the
internal point of view.
\end{proof}

We record the following lemma for later use. The proof presented here is
already simple, but a more conceptual proof is also possible (see
Section~\ref{sect:common-lemmas-transfer-principles}).
\begin{lemma}\label{lemma:regular-affine}
Let~$X = \Spec A$ be an affine scheme. Let~$f \in A$. Then~$f$ is
a regular element of~$A$ if and only if~$f$ is a regular element of~$\O_X$ from
the internal perspective.\end{lemma}
\begin{proof}The Kripke--Joyal translation of internal regularity is:
\begin{quote}For any (without loss of generality: standard) open subset~$U \subseteq X$ and any function~$g \in
\Gamma(U,\O_X)$, $fg = 0$ in~$\Gamma(U,\O_X)$ implies~$g = 0$
in~$\Gamma(U,\O_X)$.\end{quote}
So the ``if'' direction is clear (use~$U \defeq X$). For the ``only if'' direction,
note that~$\Gamma(U,\O_X)$ is a localization of~$A$ and that regular elements
remain regular in localizations.
\end{proof}


\subsection{Bézout property} Recall that a \emph{Bézout ring} is a ring in
which any finitely generated ideal is a principal ideal. In intuitionistic
mathematics, this is a better notion than that of a principal ideal ring: The
requirement that \emph{any} ideal is a principal ideal is far too strong.
Intuitively, this is because without any given generators to begin with, one
cannot hope to explicitly pinpoint a principal generator.
One can (provably) not even verify this property for the ring~$\ZZ$.\footnote{\label{fn:z-principal-ideal-domain}Assume
that any ideal of~$\ZZ$ is finitely generated. Let~$\varphi$ be an arbitrary
statement; we want to intuitionistically deduce~$\varphi \vee \neg\varphi$.
Consider the ideal~$\aaa \defeq \{ x \in \ZZ \,|\, (x = 0) \vee \varphi \}
\subseteq \ZZ$. The definition is such that~$\varphi$ holds if and only
if~$\aaa$ contains an element other than zero; and that~$\neg\varphi$ holds if
and only if zero is the only element of~$\aaa$.
By assumption,~$\aaa$ is finitely generated. Since~$\ZZ$ is a
Bézout ring, it is therefore even principal:~$\aaa = (x_0)$ for some~$x_0 \in
\ZZ$. Even intuitionistically we have~$(x_0 = 0) \vee (x_0 \neq 0)$ (for the
natural numbers, this can be proven by induction). In the first case, it
follows that~$\aaa$ contains only zero; in the second case, it follows
that~$\aaa$ contains an element other than zero. Thus~$\neg\varphi \vee
\varphi$.

This kind of reasoning is called \emph{exhibiting a Brouwerian
counterexample}. The definition of~$\aaa$ may look slightly dubious,
considering that~$\varphi$ does not depend on~$x$; but we will see that such
definitions actually have a clear geometric meaning -- they can be used to
define extensions of sheaves by zero in the internal language
(Lemma~\ref{lemma:extension-by-zero}).}

\begin{prop}Let~$X$ be a scheme (or a ringed space).
\begin{enumerate}
\item $\O_X$ is a Bézout ring from the internal perspective if and only if all
rings~$\O_{X,x}$ are Bézout rings.
\item $\O_X$ is such that, from the internal perspective, of any two elements,
one divides the other, if and only if all rings~$\O_{X,x}$ are such.
\end{enumerate}
\end{prop}
\begin{proof}Both properties can be formulated as geometric implications:
\begin{multline*}
  \text{(1)}\quad
  \forall f,g\?\O_X\_
  \top \Rightarrow
  \exists d\?\O_X\_
  (\exists a,b\?\O_X\_ d = af + bg) \wedge {} \\
  (\exists u\?\O_X\_ f = ud) \wedge
  (\exists v\?\O_X\_ g = vd)
\end{multline*}
\[
  \text{(2)}\quad
  \forall f,g\?\O_X\_
  \top \Rightarrow
  (\exists u\?\O_X\_ f = ug) \,\vee\,
  (\exists u\?\O_X\_ g = uf) \qquad\qquad\qquad\quad \qedhere
\]
\end{proof}

\begin{cor}\label{cor:dedekind-smith}
Let~$X$ be a Dedekind scheme, \ie a locally Noetherian normal scheme
of dimension~$\leq 1$. Then, from the internal perspective, any matrix
over~$\O_X$ can be put into Smith canonical form, \ie is equivalent to a
(rectangular) diagonal matrix with diagonal entries~$a_1|a_2|\cdots|a_n$
successively dividing each other.
\end{cor}
\begin{proof}It is well-known that such a scheme has principal ideal domains as
local rings~$\O_{X,x}$. For local domains, the Bézout condition is equivalent to the
property that of any two elements, one divides the other. Therefore all local
rings have this property, and by the previous proposition, the internal
ring~$\O_X$ has it as well. The statement thus follows from interpreting the
following fact of linear algebra in the internal universe: Let~$R$ be a ring
such that of any two elements, one divides the other. Then any matrix over~$R$
can be put into Smith canonical form.

The usual proof of this fact is indeed intuitionistically acceptable: Let a
matrix over~$R$ be given. By induction, one can show that for any finite family
of ring elements, one divides all the others. So some matrix entry is a factor
of all the others. We can put this entry to the upper left by row and column
transformations and then kill the other entries of the first row and the first
column. After these operations, it is still the case that the entry in the
first row and column is a factor of all other entries. Continuing in this
fashion, we obtain a diagonal matrix. Its diagonal entries already fulfill
the divisibility condition and thus do not have to be sorted.
\end{proof}

Note that phrases such as ``if by chance the entry in the upper left divides
all the others, we can directly proceed with the next step; otherwise, some
other entry must be a factor of all entries, so \ldots'' may not be included in
a proof which is intended to be intuitionistically acceptable.
Those phrases assume that one may make the case distinction that for
any two ring elements~$x,y$, either~$x$ divides~$y$ or not. Fortunately, those
case distinctions are in fact superfluous.

A consequence of the corollary is that internally to the sheaf topos of a
Dedekind scheme, the usual structure theorem on finitely
presented~$\O_X$-modules is available. We will exploit this in
Lemma~\ref{lemma:torsion-stuff}, where we give an internal proof of the
fact that on Dedekind schemes, torsion-free~$\O_X$-modules are locally free.


\subsection{Normality} We will discuss the property of a ring to be
\emph{normal}, \ie to be integrally closed in its total field of
fractions, in Section~\ref{sect:normality}, after giving an internal
characterization of the sheaf of rational functions.


\subsection{Special properties of constant sheaves of rings} Let~$R$ be an
ordinary ring and~$\ul{R}$ the associated sheaf of locally constant~$R$-valued
functions on a topological space. If~$R$ is reduced, local, or a field,
then~$\ul{R}$ is so as well, from the internal point of view.

We will prove this in greater generality: Appropriately formulated, a constant
sheaf~$\ul{R}$ has some property~$\varphi$ from the internal point of view if
and only if~$R$ has the property~$\varphi$ externally
(Lemma~\ref{lemma:properties-of-constant-sheaves}).


\subsection{Noetherian properties}
\label{sect:noetherian}

Recall the usual notion of a Noetherian ring: Any sequence~$\aaa_0 \subseteq
\aaa_1 \subseteq \cdots$ of ideals should stabilize, \ie there should exist a
natural number~$n$ such that~$\aaa_n = \aaa_{n+1} = \cdots$.

Intuitionistically, this definition has two problems. Firstly, without the
axiom of dependent choice, it is often not possible to construct a
\emph{sequence} of ideals: Often, it is only possible to show that there
\emph{exists} a suitable ideal~$\aaa_{n+1}$ depending on~$\aaa_n$. But since in
general there is no canonical choice for this successor ideal, the axiom of dependent choice
would be required to collect those into a sequence, \ie a function from~$\NN$
to the set of ideals.

Secondly, the conclusion that the sequence stabilizes is too strong.
Intuitionistically, one cannot even show that a weakly descending sequence of
natural numbers stabilizes in this sense; the statement that one could is
equivalent to the \emph{limited principle of omniscience for~$\NN$}.
Intuitionistically, it is only true that a weakly descending sequence~$a_0 \geq
a_1 \geq \cdots$ of natural numbers eventually \emph{halts} in the sense that
there exists an index~$n$ such that~$a_n = a_{n+1}$ (but~$a_{n+1} > a_{n+2}$ is
allowed).\footnote{Classically, the following three statements on a ring are
equivalent: (1)~Every ascending chain of ideals stabilizes. (2)~Every ascending
chain of finitely generated ideals stabilizes. (3)~Every ascending chain of
finitely generated ideals halts.}

We give two constructively inequivalent notions of Noetherian rings. The first
one is of independent constructive interest and enjoys the property that the structure sheaf
of a scheme~$X$ satisfies the Noetherian condition from the internal point of
view of~$\Sh(X)$ if and only if~$X$ is locally Noetherian.

The second one is quite weak from a constructive point of view, but still
interesting from a geometric point of view and useful enough to derive
nontrivial consequences. It is satisfied by the structure sheaf of any (not
necessarily locally Noetherian) reduced scheme.


{\tocless

\subsection*{Processly Noetherian rings}

\begin{defn}Let~$M$ be a partially ordered set. An \emph{ascending process
with values in~$M$} consists of an initial value~$x_0 \in M$ and a function~$f
: M \to \P(M)$ such that:
\begin{itemize}
\item For any~$x \in M$ and any~$y \in f(x)$, $x \preceq y$.
\item The set $f(x_0)$ is inhabited.
\item For any~$x_1 \in f(x_0)$, the set $f(x_1)$ is inhabited.
\item For any~$x_1 \in f(x_0)$ and any~$x_2 \in f(x_1)$, the set $f(x_2)$ is inhabited.
\item And so on.
\end{itemize}
Such a process \emph{halts} if and only if there
exists a step~$n$ and elements~$x_1, \ldots, x_n$ such that~$x_{i+1}
\in f(x_i)$ for~$i = 0,\ldots,n-1$ and such that~$x_n \in f(x_n)$.
The set~$M$ satisfies the \emph{ascending process condition} if and only if all
ascending processes with values in~$M$ halt.
\end{defn}

Intuitively, we picture~$f(x)$ as the set of all possible results of running
the process for a single step, starting with the value~$x$. This set could
be a singleton, but it may also contain more than one element, for instance if
the process cannot provide the next value in a canonical way. Instead of
arbitrarily choosing a definitive value for its result, the process may instead
collect all the possible values in the set~$f(x)$.

\begin{defn}A ring~$A$ is \emph{processly Noetherian} if and only if the
set of finitely generated ideals in~$A$ satisfies the ascending process
condition.\end{defn}

An ascending chain of elements~$a_0 \preceq a_1 \preceq \cdots$ in a partially
ordered set gives rise to an ascending process by setting~$x_0 \defeq a_0$
and~$f(x) \defeq \{ y \,|\, \exists n\_ x = a_n \wedge y = a_{n+1} \}$.
(This process halts iff there is an index~$n$ such that~$a_n = a_{n+1}$.)
Conversely, the axiom of dependent choice would allow to construct an
ascending chain from an ascending process. In the presence of this axiom, for
instance in a classical context, a ring is therefore
processly Noetherian if and only if it is Noetherian in the usual sense.

The notion of a processly Noetherian ring works well in an
intuitionistic context: Important rings such as~$\ZZ$ and more generally~$\O_K$
for any algebraic number field~$K$ are processly Noetherian, and matrices
over Bézout rings which are integral domains in the weak sense and
processly Noetherian can be put into Smith canonical form.

Richman also studied Noetherian rings in a constructive context without
dependent choice~\cite{richman:noetherian}. His notion of \emph{ascending tree
condition} is equivalent to our ascending process condition. His condition
emphasizes the branching nature of a non-deterministic computation, while ours
emphasizes the step-for-step picture of computation.

There are three reasons why we did not define a ring to be processly Noetherian
if and only if the set of all (not only finitely generated) ideals satisfies
the ascending process condition. Firstly, this stricter condition excludes
rings as~$\ZZ$.\footnote{The main ingredient in the proof that~$\ZZ$ is
Noetherian is that any ideal of~$\ZZ$ is a principal ideal, since (looking at
the prime factor decomposition) one can give explicit bounds on the length of
strictly ascending chains of principal ideals. However, as detailed in the
footnote on page~\pageref{fn:z-principal-ideal-domain}, constructively one
cannot show that every ideal of~$\ZZ$ is a principal ideal; one can only verify
that finitely generated ideals are principal. Geometrically, ideals which are
not finitely generated correspond to sheaves of ideals which may fail to
be quasicoherent.} Secondly, restricting to finitely generated
ideals in this context is a well-established procedure in constructive
mathematics~\cite{mines-richman-ruitenburg:constructive-algebra,richman:noetherian}
and suffices for the applications of the Noetherian condition one typically expects.
Thirdly, our definition provides a link to the external condition on a scheme
to be locally Noetherian, as shown by the following proposition.

\XXX{Adapt the proof of the proposition to conform to the new (corrected)
definition.}

\begin{prop}\label{prop:internal-noetherianity}
A scheme~$X$ is locally Noetherian if and only if the ring~$\O_X$
is processly Noetherian from the internal point of view.\end{prop}
\begin{proof}We only prove the ``only if'' direction. We may assume that~$X =
\Spec A$ is affine with~$A$ a Noetherian ring and that internally, we are given an
ascending process on the set of finitely generated ideals of~$\O_X$.
Externally, this is a morphism~$\ul{\NN} \to \P(\M)$ where~$\NN$ is the
constant sheaf with value~$\NN$ and~$U$-sections of~$\M$ are finite type ideal
sheaves of~$\O_X|_U$.

Since~$X \models \speak{$f(0)$ is inhabited}$, there exists an open covering~$X
= \bigcup_i U_i$ and finite type ideal sheaves~$\I_i \hookrightarrow
\O_X|_{U_i}$ such that~$U_i \models \I_i \in f(0)$. Without loss of generality,
we may assume that the open sets~$U_i$ are standard open sets~$D(f_i)$ and that
the covering is finite. Since the sheaves~$\I_i$ are quasicoherent (being of
finite type, they are images of morphisms of the form~$(\O_X|_{U_i}^n \to
\O_X|_{U_i}$), they correspond to ideals~$J_i \subseteq A[f_i^{-1}]$. Note for future reference
that for~$D(g) \subseteq D(f_i)$, the restricted ideal~$\I_i|_{D(g)}$ corresponds to the extension of~$J_i$ in the further
localized ring~$A[g^{-1}]$.

For each~$i$,~$D(f_i) \models \exists \aaa \in f(1)\_ \I_i \subseteq \aaa$.
Thus there exists an open covering~$D(f_i) = \bigcup_j D(g_{ij})$ and finite
type ideal sheaves~$\I_{ij} \hookrightarrow \O_X|_{D(g_{ij})}$; these
correspond to ideals~$J_{ij} \subseteq A[g_{ij}^{-1}]$ such that~$J_i
\subseteq J_{ij}$ (where we have suppressed the localization
morphism~$A[f_i^{-1}] \to A[g_{ij}^{-1}]$ in the notation). Equivalently,
writing~$J_i' \defeq A \cap J_i$ and~$J_{ij}' \defeq A \cap J_{ij}$ for the
contractions, we have the inclusions~$J_i' \subseteq J_{ij}'$ of ideals of~$A$.

Continuing in this fashion, we obtain a tree of ideals~$J_{i_1 \cdots i_n}'$.
Each path in this tree is a chain of ascending ideals and thus stabilizes
since~$A$ is Noetherian. Since only finitely many subtrees branch off at
each node, there appear only finitely many distinct ideals in this tree (this
is an application of the graph-theoretical \emph{König's lemma}).

There thus exists a natural number~$n$ such that~$J_{i_1 \cdots i_n}' = J_{i_1
\cdots i_n i_{n+1}}'$ for all appropriate indices~$i_1,\ldots,i_n,i_{n+1}$.
For this number~$n$, the internal statement~$X \models \exists \aaa \in f(n)
\cap f(n+1)$ holds; we leave further details to the reader.
\end{proof}

\begin{rem}The proof shows that, internally speaking, even
the set of all quasicoherent ideals (instead of merely the finitely generated
ones) fulfills the ascending process condition, if the base scheme is locally Noetherian. We have not taken this property
as the definition of a processly Noetherian ring since it is a notion not
usually studied in constructive mathematics (compare
Remark~\ref{rem:qcoh-in-constructive-mathematics}).
\end{rem}

\XXX{Give examples made possible by the internal Noetherianity}
\XXX{How about: Any quasicoherent submodule of a module of finite type is of
finite type as well.}


\subsection*{Weakly Noetherian rings}

Classically, there is a characterization of Noetherian rings which doesn't
involve ascending sequences: A ring is Noetherian if and only if any of its
ideals is finitely generated. We mentioned in the footnote on
page~\pageref{fn:z-principal-ideal-domain} that this condition is far too
strong from a constructive point of view; not even the ring~$\ZZ$ verifies it.
However, it can be weakened to yield a constructively sensible notion:

\begin{defn}A ring~$A$ is \emph{weakly Noetherian} if and only if any ideal
of~$A$ is \notnot finitely generated. A module~$M$ is \emph{weakly Noetherian}
if and only if any submodule of~$M$ is \notnot finitely generated.\end{defn}

\begin{ex}There is an intuitionistic proof that the ring~$\ZZ$ is weakly
Noetherian: Let~$\aaa \subseteq \ZZ$ be any ideal. Under the assumption that
either there exists a nonzero element in~$\aaa$ or not, the ideal~$\aaa$ is
\notnot finitely generated, even \notnot principal: For in the first case, a
minimal element~$d$ of~$\aaa \cap \NN^+$ (which \notnot exists) witnesses~$\aaa
= (d)$. In the second case the ideal~$\aaa$ is the zero ideal. Since the
assumption is \notnot satisfied, the ideal~$\aaa$ is \notnot \notnot finitely
generated, so \notnot finitely generated. (We remark on this proof scheme on
page~\pageref{proof-scheme-boxed-statements}.) \end{ex}

\begin{thm}\label{thm:hilbert}
Let~$A$ be a weakly Noetherian ring. Then the polynomial
algebra~$A[X]$ is weakly Noetherian as well, intuitionistically.
\end{thm}
\begin{proof}Classically, this is precisely the statement of Hilbert's basis
theorem, whose usual accounts do not care about the sensibilities of
constructive mathematics. However, a careful reading of for instance the proof
given in~\cite[Theorem~7.5]{atiyah:macdonald:commutative-algebra} shows that
the theorem holds intuitionistically as stated.
\end{proof}

\begin{lemma}Let~$0 \to M' \to M \to M'' \to 0$ be a short exact sequence of
modules. Intuitionistically, the module~$M$ is weakly Noetherian if and only
if~$M'$ and~$M''$ are.
\end{lemma}

\begin{proof}The usual proof applies.\end{proof}

\begin{prop}\label{prop:ox-weakly-noetherian}
Let~$X$ be an arbitrary reduced scheme (not necessarily locally
Noetherian). Then~$\O_X$ is weakly Noetherian from the internal point of view
of~$\Sh(X)$.\end{prop}
\begin{proof}By Corollary~\ref{cor:field-reduced}, the ring~$\O_X$ fulfills a
suitable field condition from the internal point of view. Therefore it suffices
to give an intuitionistic proof of the following statement: Let~$k$ be a ring such that
any element of~$k$ which is not invertible is zero. Then any ideal of~$k$ is
\notnot finitely generated.

Let~$\aaa \subseteq k$ be an arbitrary ideal. We have~$\neg\neg(1 \in \aaa \vee
1 \not\in \aaa)$. Therefore~$\neg\neg(\aaa = (1) \vee \aaa = (0))$. Thus~$\aaa$
is \notnot finitely generated (even \notnot principal).
\end{proof}

The fact that~$\O_X$ is weakly Noetherian from the internal point of view will
allow for a quick proof of Grothendieck's generic freeness lemma in
Section~\ref{sect:generic-freeness}. The translation of the statement
that~$\O_X[U_1,\ldots,U_n]$ is weakly Noetherian was displayed on
page~\pageref{page:convoluted-statement}.

}


\section{Sheaves of modules}

From the internal perspective, a sheaf of~$\R$-modules, where~$\R$ is a sheaf
of rings, looks just like a plain module over the plain ring~$\R$. This is
proven just as the correspondence between sheaf of rings and internal rings
(Proposition~\ref{prop:rings-internally}).

\XXX{talk about modules over constant sheaves of rings as well}


\subsection{Finite local freeness}

Recall that an~$\O_X$-module~$\F$ is \emph{finite locally free} if and only
if there exists a covering of~$X$ by open subsets~$U$ such that on each
such~$U$, the restricted module~$\F|_U$ is isomorphic as an~$\O_X|_U$-module
to~$(\O_X|_U)^n$ for some natural number~$n$ (which may depend on~$U$).

\begin{prop}\label{prop:locally-free}
Let~$X$ be a scheme (or a ringed space). Let~$\F$ be
an~$\O_X$-module. Then~$\F$ is finite locally free if and only if, from the
internal perspective,~$\F$ is a finite free module, \ie
\[ \Sh(X) \models \bigvee_{n \geq 0} \speak{$\F \cong (\O_X)^n$}, \]
or more elementarily
\[ \Sh(X) \models \bigvee_{n \geq 0}
  \exists x_1,\ldots,x_n\?\F\_
  \forall x\?\F\_
  \exists! a_1,\ldots,a_n\?\O_X\_
  x = \textstyle\sum\limits_i a_i x_i. \]
\end{prop}
\begin{proof}By the expression~``$(\O_X)^n$'' in the internal language we mean
the internally constructed object~$\O_X \times \cdots \times \O_X$ with its
componentwise~$\O_X$-module structure. This coincides with the sheaf~$(\O_X)^n$ as
usually understood.

It is clear that the two stated internal conditions are equivalent, since the
corresponding proof in linear algebra is intuitionistically acceptable. The equivalence with
the external notion of finite local freeness follows because the
interpretation of the first condition with the Kripke--Joyal semantics is the
following: There exists a covering of~$X$ by open subsets~$U$ such that for
each such~$U$, there exists a natural number~$n$ and a morphism of
sheaves~$\varphi : \F|_U \to (\O_X|_U)^n$ such that
\[ U \models \speak{$\varphi$ is~$\O_X$-linear} \quad\text{and}\quad
  U \models \speak{$\varphi$ is bijective}. \]
The first subcondition means that~$\varphi$ is a morphism of sheaves
of~$\O_X|_U$-modules and the second one means that~$\varphi$ is an isomorphism of
sheaves.
\end{proof}

\begin{rem}There are intuitionistic proofs of the following facts:
An~$R$-module is a dualizable object in the monoidal category of all
$R$-modules if and only if it is finitely generated and projective. If~$R$ is
local, then an~$R$-module is finitely generated and projective if and only if
it is finite free. Therefore an~$\O_X$-module is internally dualizable if and
only if is finite locally free.
\end{rem}


\XXX{algebras of finite type, ...}

\subsection{Finite type, finite presentation, coherence}
Recall the conditions of an~$\O_X$-module~$\F$ on a scheme~$X$ (or a ringed
space) to be of finite type, of finite presentation, and to be coherent:
\begin{itemize}
\item $\F$ is \emph{of finite type} if and only if there exists a covering of~$X$ by
open subsets~$U$ such that for each such~$U$, there exists an exact sequence
\[ (\O_X|_U)^n \longrightarrow \F|_U \longrightarrow 0 \]
of~$\O_X|_U$-modules.
\item $\F$ is \emph{of finite presentation} if and only if there exists a covering of~$X$ by
open subsets~$U$ such that for each such~$U$, there exists an exact sequence
\[ (\O_X|_U)^m \longrightarrow (\O_X|_U)^n \longrightarrow \F|_U \longrightarrow 0. \]
\item $\F$ is \emph{coherent} if and only if~$\F$ is of finite type and the
kernel of any~$\O_X|_U$-linear morphism~$(\O_X|_U)^n \to \F|_U$, where~$U \subseteq
X$ is any open subset, is of finite type.
\end{itemize}

The following proposition gives translations of these definitions into the
internal language.
\begin{prop}\label{prop:finite-type-and-co}
Let~$X$ be a scheme (or a ringed space). Let~$\F$ be
an~$\O_X$-module. Then:
\begin{itemize}
\item $\F$ is of finite type if and only if~$\F$, considered as an ordinary
module from the internal perspective, is finitely generated, \ie if
\[ {\qquad\qquad} \Sh(X) \models
  \bigvee_{n \geq 0}
  \exists x_1,\ldots,x_n\?\F\_
  \forall x\?\F\_
  \exists a_1,\ldots,a_n\?\F\_
  x = \textstyle\sum\limits_i a_i x_i. \]
\item $\F$ is of finite presentation if and only if~$\F$ is a finitely
presented module from the internal perspective, \ie if
\[ {\qquad\qquad} \Sh(X) \models \bigvee_{n,m \geq 0}
  \speak{there is a short exact sequence $\O_X^m \to \O_X^n \to \F \to 0$}.
  \]
\item $\F$ is coherent if and only if~$\F$ is a coherent module from the
internal perspective, \ie if
\begin{multline*}
{\qquad\qquad\qquad}
  \Sh(X) \models \speak{$\F$ is finitely generated} \mathop{\wedge} \\
  \bigwedge_{n \geq 0} \forall \varphi \? \HOM_{\O_X}(\O_X^n,\F)\_
  \speak{$\Kernel \varphi$ is finitely generated}.
\end{multline*}
\end{itemize}
\end{prop}
\begin{proof}Straightforward -- the translations of the internal statements using
the Kripke--Joyal semantics are precisely the corresponding external
statements.
\end{proof}

Recall that an~$\O_X$-module~$\F$ is \emph{generated by global sections} if and
only if there exist global sections~$s_i \in \Gamma(X,\F)$ such that for any~$x
\in X$, the stalk~$\F_x$ is generated by the germs of the~$s_i$.
This condition is of course not local on the base. Therefore there cannot
exist a formula~$\varphi$ such that for any space~$X$ and
any~$\O_X$-module~$\F$ it holds that~$\F$ is generated by global sections if
and only if~$\Sh(X) \models \varphi(\F)$. But still, global generation can be
characterized by a mixed internal/external statement:

\begin{prop}Let~$X$ be a scheme (or a ringed space). Let~$\F$ be
an~$\O_X$-module. Then~$\F$ is generated by global sections if and only if
there exist global sections~$s_i \in \Gamma(X,\F)$, $i \in I$ such that
\[ \Sh(X) \models \forall x\?\F\_ \bigvee\nolimits_{\textnormal{$J=\{i_1,\ldots,i_n\} \subseteq I$ finite}}
  \exists a_1,\ldots,a_n\?\O_X\_
  x = \sum_j a_j x_{i_j}. \]
Furthermore,~$\F$ is generated by finitely many global sections if and only if
there exist global sections~$s_1,\ldots,s_n \in \Gamma(X,\F)$ such that
\[ \Sh(X) \models \forall x\?\F\_ \exists a_1,\ldots,a_n\?\O_X\_ x = \sum_j a_j
x_j. \]
\end{prop}
\begin{proof}The given internal statements are geometric implications, their
validity can thus be checked stalkwise.\end{proof}


\subsection{Tensor product and flatness} Recall that the tensor product
of~$\O_X$-modules~$\F$ and~$\G$ on a scheme~$X$ (or a ringed space) is usually
constructed as the sheafification of the presheaf
\[ \text{$U \subseteq X$ open} \quad\longmapsto\quad \Gamma(U,\F) \otimes_{\Gamma(U,\O_X)}
\Gamma(U,\G). \]
From the internal point of view,~$\F$ and~$\G$ look like ordinary modules, so
that we can consider their tensor product as usually constructed in
commutative algebra, as a certain quotient of the free module on the elements
of~$\F \times \G$:
\[ \O_X\langle x \otimes y \,|\, x\?\F, y\?\G \rangle / R, \]
where~$R$ is the submodule generated by
\begin{gather*}
  (x+x') \otimes y - x \otimes y - x' \otimes y, \\
  x \otimes (y+y') - x \otimes y - x \otimes y', \\
  (sx) \otimes y - s(x \otimes y), \\
  x \otimes (sy) - s(x \otimes y)
\end{gather*}
with~$x,x'\?\F$, $y,y'\?\G$, $s\?\O_X$.
This internal construction gives rise to the same sheaf
of modules as the externally defined tensor product:

\begin{prop}\label{prop:internal-tensor-product}
Let~$X$ be a scheme (or a ringed space). Let~$\F$ and~$\G$
be~$\O_X$-modules. Then the internally constructed tensor product~$\F
\otimes_{\O_X} \G$ coincides with the external one.
\end{prop}
\begin{proof}
Since the proof of the corresponding fact of commutative algebra is
intuitionistically acceptable, the internally defined tensor product~$\F \otimes_{\O_X} \G$
has the following universal property: For any~$\O_X$-module~$H$,
any~$\O_X$-bilinear map~$\F \times \G \to H$ uniquely factors over the
canonical map~$\F \times \G \to \F \otimes_{\O_X} \G$.

Interpreting this property with the Kripke--Joyal semantics, we see that the
internally constructed tensor product has the following external property:
For any open subset~$U \subseteq X$ and any~$\O_X|_U$-module~$\H$ on~$U$,
any~$\O_X|_U$-bilinear morphism~$\F|_U \times \G|_U \to \H$ uniquely factors over the
canonical morphism~$\F|_U \times \G|_U \to (\F \otimes_{\O_X} \G)|_U$.

In particular, for~$U = X$, this property is well-known to be the universal
property of the externally constructed tensor product. Therefore the
claim follows.
\end{proof}

A description of the stalks of the tensor product
follows purely by considering the logical form of the construction:
\begin{cor}Let~$X$ be a scheme (or a ringed space). Let~$\F$ and~$\G$
be~$\O_X$-modules. Then the stalks of the tensor product coincide with the
tensor products of the stalks: $(\F \otimes_{\O_X} \G)_x \cong \F_x
\otimes_{\O_{X,x}} \G_x$.\end{cor}
\begin{proof}
We constructed the tensor product using the following operations: product of
two sets, free module on a set, quotient module with respect to a submodule;
submodule generated by a set of elements given by a geometric formula.
All of these operations are geometric, so the tensor product construction is
geometric as well (see Section~\ref{sect:geometric-formulas-and-constructions}). Hence taking stalks commutes with performing the
construction.
\end{proof}

Recall that an~$\O_X$-module~$\F$ is \emph{flat} if and only if all
stalks~$\F_x$ are flat~$\O_{X,x}$-modules. We can characterize flatness in the
internal language.
\begin{prop}\label{prop:flatness}
Let~$X$ be a scheme (or a ringed space). Let~$\F$ be
an~$\O_X$-module. Then~$\F$ is flat if and only if, from the internal
perspective,~$\F$ is a flat~$\O_X$-module.
\end{prop}
\begin{proof}
Recall that flatness of an~$A$-module~$M$ can be characterized without
reference to tensor products by the following condition (using
suggestive vector notation): For any natural number~$p$,
any $p$-tuple~$m \? M^p$ of elements of~$M$ and
any~$p$-tuple $a \? A^p$ of elements of~$A$, it should hold that
\[
  a^T m = 0 \ \Longrightarrow\
  \bigvee\limits_{q \geq 0} \exists n\?M^q, B\?A^{p \times q}\_
  Bn = m \wedge a^T B = 0. \]
The equivalence of this condition with tensoring being exact holds
intuitionistically as
well~\cite[Theorem~III.5.3]{mines-richman-ruitenburg:constructive-algebra}.
This formulation of flatness has the advantage that it is the conjunction of
geometric implications (one for each~$p \geq 0$); therefore it holds internally
if and only if it holds at any point.
\end{proof}


\subsection{Support} Recall that the \emph{support} of an~$\O_X$-module~$\F$ is
the subset~$\supp\F \defeq \{ x \in X \,|\, \F_x \neq 0 \} \subseteq X$. If~$\F$ is
of finite type, this set is closed, since its complement is then open by a
standard lemma. (We will give an internal proof of this fact in
Lemma~\ref{lemma:module-zero-point-neighbourhood}.)

\begin{prop}\label{prop:characterization-support}
Let~$X$ be a scheme (or a ringed space). Let~$\F$ be
an~$\O_X$-module. Then the interior of the complement of the support of~$\F$
can be characterized as the largest open subset of~$X$ on which the internal
statement~$\F = 0$ holds.
\end{prop}
\begin{proof}
For any open subset~$U \subseteq X$, it holds that:
\begin{align*}
  &\ U \subseteq \Int(X \setminus \supp \F) \\
  \Longleftrightarrow&\ U \subseteq X \setminus \supp \F \\
  \Longleftrightarrow&\ U \subseteq \{ x \in X \,|\, \forall s \in \F_x\_ s = 0 \} \\
  \Longleftrightarrow&\ U \models \forall s\?\F\_ s = 0 \\
  \Longleftrightarrow&\ U \models \speak{$\F = 0$}
\end{align*}
The second to last equivalence is because~``$\forall s\?\F\_ s = 0$'' is a
geometric implication and can thus be checked stalkwise.
\end{proof}

\begin{rem}\label{rem:support-sheaf-of-sets}
The support of a sheaf of \emph{sets}~$\F$ is defined as the subset~$\{ x \in X \,|\,
\text{$\F_x$ is not a singleton} \}$. A similar proof shows that the interior
of its complement can be characterized as the largest open subset of~$X$ where
the internal statement~$\speak{$\F$ is a singleton}$ holds.\end{rem}


\subsection{Torsion} Let~$R$ be a ring. Recall that the
\emph{torsion submodule}~$M_\tors$ of an~$R$-module~$M$ is defined as
\[ M_\tors \defeq \{ x\?M \,|\, \exists a\?R\_ \speak{$a$ regular} \wedge ax = 0 \}
\subseteq M. \]
This definition is meaningful even if~$R$ is not an integral domain.
An~$R$-module~$M$ is \emph{torsion-free} if and only if~$M_\tors$ is
the zero submodule; an~$R$-module~$M$ is a \emph{torsion module} if and only
if~$M_\tors = M$.

Recall also that if~$\F$ is a sheaf of~$\O_X$-modules on an integral
scheme~$X$, there is a unique subsheaf~$\F_\tors \subseteq \F$ with the
property that~$\Gamma(U,\F_\tors) = \Gamma(U,\F)_\tors$ for all affine open
subsets~$U \subseteq X$. The content of the following proposition is that
internally constructing the torsion submodule of~$\F$, regarded as a plain
module from the internal perspective, gives exactly this subsheaf. There is
therefore no harm in using the same notation~``$\F_\tors$'' for the result of
the internal construction.

\begin{prop}\label{prop:torsion-int-ext}Let~$X$ be an integral scheme. Let~$\F$ be an~$\O_X$-module. Let~$U
= \Spec A \subseteq X$ be an affine open subset. Let~$s \in \Gamma(U,\F)$ be a local
section. Then
\[ s \in \Gamma(U,\F)_\tors \quad\text{if and only if}\quad
  U \models s \in \F_\tors. \]
\end{prop}
\begin{proof}
The ``only if'' direction is trivial in view of
Lemma~\ref{lemma:regular-affine}: If~$s$ is a torsion element
of~$\Gamma(U,\F)$, there exists a regular element~$a \in \Gamma(U,\O_X)$ such
that~$as = 0$. By the lemma, this element is regular from the internal
perspective as well, so~$U \models \speak{$a$ regular} \wedge as = 0$.

For the ``if'' direction, we may assume that there exists an open covering~$X =
\bigcup_i U_i$ by standard open subsets~$U_i = D(f_i)$ such that there are
sections~$a_i \in \Gamma(U_i,\O_X) = A[f_i^{-1}]$ with~$U_i \models \speak{$a_i$ regular}
\wedge a_i s = 0$. Without loss of generality, we may assume that the
denominators of the~$a_i$'s are ones, that the $f_i$ are
finite in number, and that the~$f_i$ are regular (\ie nonzero, since~$A$ is an
integral domain). By Lemma~\ref{lemma:regular-affine}, the~$a_i$ are
regular in~$A[f_i^{-1}]$ and by regularity of the~$f_i$ also regular in~$A$.
Therefore their product~$\prod_i a_i \in A$ is regular in~$A$ as well and
annihilates~$s$.
\end{proof}
% FUTURE:
% For quasicoherent sheaves, there is an alternative (and more conceptual)
% proof. Since X is integral, we have
%
%   Spec(A) |== (forall s:\ul{A}. s regular in A <==> s regular in O_{Spec A}).
%
% Therefore the internal statement
%
%   exists a:O_{Spec A}. a regular and as = 0 in \ul{M}[F^(-1)]
%
% is equivalent to
%
%   bigvee_{a in A, a regular} (exists u in F. uas = 0).
%
% This disjunction is directed.

\begin{prop}\label{prop:torsion-submodule-stalks}
Let~$X$ be a locally Noetherian scheme. Let~$\F$ be
an~$\O_X$-module. Let~$x \in X$ be a point. Then~$(\F_\tors)_x =
(\F_x)_\tors$.\end{prop}
\begin{proof}This would be obvious if the condition on an element~$s\?\F$ to
belong to~$\F_\tors$ were a geometric formula. Because of the universal
quantifier, it is not:
\[ s \in \F_\tors \quad\Longleftrightarrow\quad
  \exists a\?\O_X\_ (\forall b\?\O_X\_ ab = 0 \Rightarrow b = 0) \wedge as = 0. \]
But since~$X$ is assumed to be locally Noetherian, regularity is an open
property nonetheless (see Proposition~\ref{prop:regularity-spreading} for an
internal proof of this fact). Thus the claim still follows, just like in the
proof of Proposition~\ref{prop:internal-integrality}.
\end{proof}


\subsection{Internal proofs of common lemmas}

\begin{lemma}Let~$X$ be a scheme (or a ringed space). Let
\[ 0 \lra \F \lra \G \lra \H \lra 0 \]
be a short exact sequence of~$\O_X$-modules. If~$\F$ and~$\H$ are of finite
type, so is~$\G$; similarly, if~$\F$ and~$\H$ are finite locally free, so
is~$\G$.
\end{lemma}
\begin{proof}From the internal perspective, we are given a short exact sequence
of modules with the outer ones being finitely generated (\resp finite free)
and we have to show that the middle one is finitely generated (\resp finite
free) as well. It is well-known that this follows; and since the usual proof of
this fact is intuitionistically acceptable, we are done.
\end{proof}

Note that the proof works very generally, in the context of arbitrary ringed
spaces, and is still very simple. This is common to proofs using the internal
language. Particular features of schemes enter only at clearly recognizable
points, for instance when an internal property specific to the structure sheaf
of schemes is used (such as in Proposition~\ref{prop:neginvnilpotent}).

\begin{lemma}\label{lemma:coherent-stuff}
Let~$X$ be a scheme (or a ringed space).
\begin{itemize}
\item Let~$0 \to \F \to \G \to \H \to 0$ be an exact sequence
of~$\O_X$-modules. If two of the three modules are coherent, so is the third.
\item Let~$\F \to \G$ be a morphism of~$\O_X$-modules such that~$\F$ is
of finite type and~$\G$ is coherent. Then its kernel is of finite type as well.
\item If~$\F$ is a finitely presented~$\O_X$-module and~$\G$ is a
coherent~$\O_X$-module, the~$\O_X$-modules~$\HOM_{\O_X}(\F,\G)$ and~$\F \otimes_{\O_X} \G$
are coherent as well.
\end{itemize}
\end{lemma}
\begin{proof}These statements follow directly from interpreting the
corresponding standard proofs of commutative algebra in the internal language.
For those standard proofs, see for instance the lecture notes of Ravi
Vakil~\cite[Section~13.8]{vakil:foag}, where they are given as a series of
exercises.
\end{proof}

\begin{lemma}\label{lemma:kernel-of-epi-fingen}
Let~$X$ be a scheme (or locally ringed space). Let~$\alpha : \G
\to \H$ be an epimorphism of finite locally free~$\O_X$-modules. Then the
kernel of~$\alpha$ is finite locally free as well.\end{lemma}
\begin{proof}It suffices to give an intuitionistic proof of the following
statement: The kernel of a matrix over a local ring, which as a linear map is
surjective, is finite free.

Let~$M \? R^{n \times m}$ be such a matrix. Since by the surjectivity
assumption some linear combination of the columns is~$e_1$ (the first canonical
basis vector), some linear combination of the entries of the first row of~$M$
is~$1$. By locality of~$R$, at least one entry of the first row is invertible.
By applying appropriate column and row transformations, we may therefore assume that~$M$
is of the form
\[ \left(
  \begin{array}{c|ccc}
    1 & 0 & \cdots & 0 \\ \hline
    0 & \multicolumn{3}{c}{\multirow{3}{*}{\raisebox{-5mm}{\scalebox{1.2}{$\widetilde M$}}}} \\
    \raisebox{2pt}{\vdots} & & & \\
    0 & & &
  \end{array}
\right) \]
with the submatrix~$\widetilde M$ fulfilling the same condition as~$M$.
Continuing in this way, it follows that~$m \geq n$ and that we may assume
that~$M$ is of the form
\[ \left(
  \begin{array}{ccc|c}
    1 & & & \multirow{3}{*}{\raisebox{-3mm}{\scalebox{1.2}{\ 0}}} \\
    & \ddots && \\
    && 1
  \end{array}
\right)\!. \]
The kernel of such a matrix is obviously freely generated by the canonical
basis vectors corresponding to the zero columns. In particular, the rank of the
kernel is~$m-n$.
\end{proof}

\begin{rem}The internal language machinery gives no reason to believe that the
dual statement is true, \ie that the cokernel of a monomorphism of finite
locally free~$\O_X$-modules is finite locally free. This would follow from
an intuitionistic proof of the statement that the cokernel of an injective map
between finite free modules over a local ring is finite free. But this
statement is of course false (consider the exact sequence
$0 \lra \ZZ_{(2)} \stackrel{\cdot 2}{\lra} \ZZ_{(2)} \lra \FF_2 \lra 0$
of~$\ZZ_{(2)}$-modules).
\end{rem}

\begin{lemma}\label{lemma:epi-iso}Let~$X$ be a scheme (or locally ringed space). Let~$\alpha : \G
\to \H$ be an epimorphism of finite locally free~$\O_X$-modules of the same
rank. Then~$\alpha$ is an isomorphism.\end{lemma}
\begin{proof}It suffices to give an intuitionistic proof of the following
statement: A square matrix over a local ring, which as a linear map is
surjective, is invertible.

This follows from the proof of the previous lemma, since it shows that the
kernel of such a matrix is finite free of rank zero.
\end{proof}

\begin{rem}The conclusion of Lemma~\ref{lemma:epi-iso} also holds if~$X$ is
only assumed to be a ringed space. To show this, it suffices to give an
intuitionistic proof of the following statement: A square matrix over a (not
necessarily local) ring, which as a linear map is surjective, is invertible.
Such a matrix~$A$ possesses a right inverse. Therefore~$\det A$ is invertible.
Thus~$A$ is invertible with inverse~$(\det A)^{-1} \cdot \operatorname{ad} A$.
\end{rem}

\begin{lemma}Let~$X$ be a scheme (or a ringed space). Let
$0 \to \F \to \G \to \H \to 0$ be a short exact sequence of~$\O_X$-modules.
Then for the closures of the supports there holds the equation
$\Clos \supp \G = \Clos \supp \F \cup \Clos \supp \H$.
\end{lemma}
\begin{proof}Switching to complements, we have to prove that
\[ \Int(X \setminus \supp\G) = \Int(X \setminus \supp\F) \cap \Int(X \setminus
\supp\H). \]
By Proposition~\ref{prop:characterization-support}, it suffices to prove
\[ \Sh(X) \models (\G = 0\ \Longleftrightarrow\ \F = 0 \wedge \H = 0); \]
this is a basic observation in linear algebra, valid intuitionistically.
\end{proof}
Of course, a stronger version of this lemma -- about the supports themselves
instead of their closures -- is easily proven without using the internal
language. We included this example only for illustrative purposes.

\begin{lemma}Let~$X$ be a scheme (or locally ringed space). Let~$\L$ be a line
bundle on~$X$, \ie an~$\O_X$-module locally free of rank~1.
Let~$s_1,\ldots,s_n \in \Gamma(X,\L)$ be global sections. Then these sections
globally generate~$\F$ if and only if
\[ \Sh(X) \models \bigvee_i \speak{$\alpha(s_i)$ is invertible for some
isomorphism~$\alpha : \L \to \O_X$}. \]
\end{lemma}
\begin{proof}It suffices to give an intuitionistic proof of the following fact:
Let~$R$ be a local ring. Let~$L$ be a free~$R$-module of rank~1.
Let~$s_1,\ldots,s_n\?L$ be given elements. Then~$L$ is generated as
an~$R$-module by these elements if and only if for some~$i$, the image of~$s_i$
under some isomorphism~$L \to R$ is invertible.

Note that the choice of such an isomorphism does not matter, since any two such
isomorphisms~$\alpha, \beta : L \to R$ differ by a unit of~$R$: $\alpha(x) =
\alpha(\beta^{-1}(1)) \cdot \beta(x)$ for any~$x\?L$,
and~$\alpha(\beta^{-1}(1)) \cdot \beta(\alpha^{-1}(1)) = 1$ in~$R$.

For the ``if'' direction, we have that some~$\alpha(s_i)$ is a generator
of~$R$. Since~$\alpha$ is an isomorphism, it follows that~$s_i$ generates~$L$,
and thus that in particular, the family~$s_1,\ldots,s_n$ generates~$L$.

For the ``only if'' direction, we have that the unit of~$R$ can be expressed as
a linear combination of the~$\alpha(s_i)$, where~$\alpha : L \to R$ is some
isomorphism (whose existence is assured by the assumption on the rank of~$L$).
Since~$R$ is a local ring, it follows that one of the summands and thus one of
the~$\alpha(s_i)$ is invertible.
\end{proof}

\begin{rem}Note that the canonical ring homomorphism~$\O_{X,x}
\twoheadrightarrow k(x)$ is local. Therefore a germ in~$\O_{X,x}$ is invertible
if and only if its image in~$k(x)$ is not zero. From this one can conclude that
global sections~$s_1,\ldots,s_n \in \Gamma(X,\F)$ generate~$\F$ if and only if,
for any point~$x \in X$, the images~$s_i \in \F|_x$ in the fibers do not vanish
simultaneously.
\end{rem}

\begin{lemma}Let~$X$ be a scheme (or a ringed space). Let~$\L$ be a locally
free~$\O_X$-module of rank~1. Then~$\L^\vee \otimes_{\O_X} \L \cong \O_X$.\end{lemma}
\begin{proof}Recall that the dual is defined by~$\L^\vee \defeq
\HOM_{\O_X}(\L,\O_X)$. Since~``$\HOM$'' looks like~``$\Hom$'' from the internal
point of view, the dual sheaf~$\L^\vee$ looks just like the ordinary dual
module. However, to prove the claim, it does \emph{not} suffice to give an
intuitionistic proof of the following fact of linear algebra: ``Let~$L$ be a
free~$R$-module of rank~1. Then there exists an isomorphism~$L^\vee \otimes_R L
\to R$.'' Since the interpretation of ``$\exists$'' using the Kripke--Joyal
semantics is local existence, this would only show that~$\L^\vee \otimes_{\O_X}
\L$ is \emph{locally} isomorphic to~$\O_X$.

Instead, we have to actually \emph{write down} (\ie explicitly give) a
linear map in the internal language -- not using the assumption that~$L$ is
free of rank~1, as this would introduce an existential quantifier again (see
Section~\ref{sect:internal-constructions}).
So we have to prove the following fact: Let~$L$ be an~$R$-module. Then there
explicitly exists a linear map~$L^\vee \otimes_R L \to R$ such that this map is
an isomorphism if~$L$ is free of rank~1.

This is done as usual: Define~$\alpha : L^\vee \otimes_R L \to R$ by~$\lambda
\otimes x \mapsto \lambda(x)$. Since~$L$ is free of rank~1, there is an
isomorphism~$L \cong R$. Precomposing~$\alpha$
with the induced isomorphism~$R^\vee \otimes_R R \to L^\vee \otimes_R L$,
we obtain the linear map~$R^\vee \otimes_R R \to R$ given by the same
term: $\lambda \otimes x \mapsto \lambda(x)$. One can check that an inverse is given
by~$x \mapsto \id_R \otimes x$.
\end{proof}

\begin{lemma}\label{lemma:torsion-stuff}
Let~$X$ be a scheme (or a ringed space). Let~$\F$ be
an~$\O_X$-module.
\begin{enumerate}
\item Assume~$X$ to be a locally Noetherian scheme. Then $\F$ is torsion-free
(meaning~$\F_\tors = 0$) if and only if all stalks~$\F_x$ are torsion-free.
\item The quotient sheaf~$\F/\F_\tors$ is torsion-free and the torsion
submodule~$\F_\tors$ is a torsion module.
\item The dual sheaf $\F^\vee$ is torsion-free.
\item If~$\F$ is reflexive (meaning that the canonical morphism~$\F \to
\F^{\vee\vee}$ is an isomorphism), it is torsion-free.
\item If~$\F$ is finite locally free, it is reflexive.
\item Assume~$X$ to be a Dedekind scheme and~$\F$ to be of finite presentation.
If~$\F$ is torsion-free, then it is finite locally free.
% change references below if numbering changes
\end{enumerate}
\end{lemma}
\begin{proof}The first statement follows from the observation that~$(\F_\tors)_x
= (\F_x)_\tors$ (Proposition~\ref{prop:torsion-submodule-stalks}). All the
others follow simply by interpreting the corresponding facts of linear algebra
in the internal universe. For concreteness, we give intuitionistic proofs of
the last three statements.

So let~$M$ be an reflexive~$R$-module. We have to show that~$M$ is
torsion-free. To this end, let an element~$x \? M$ and a regular element~$a \?
R$ such that~$ax = 0$ be given. For any~$\vartheta \? M^\vee$, it follows
that~$\vartheta(x) = 0$, since~$a \vartheta(x) = \vartheta(ax) = \vartheta(0) =
0$ and~$a$ is regular. Thus the image of~$x$ under the canonical map~$M \to
M^{\vee\vee}$ is zero. By reflexivity, this implies that~$x$ is zero.

For statement~(5), we have to prove that~$R$-modules of the form~$R^n$ are
reflexive. This is obvious, the required inverse map is~$(R^n)^{\vee\vee} \to
R^n,\ \lambda \mapsto \sum_i \lambda(\vartheta_i)$ where~$\vartheta_i : R^n \to
R,\ (x_j)_j \mapsto x_i$.

In view of Corollary~\ref{cor:dedekind-smith} we can put matrices over~$\O_X$
into Smith canonical form if~$X$ is a Dedekind scheme. Therefore it suffices
to give an intuitionistic proof of the following fact: Let~$R$ be an integral
domain in the strong sense such that matrices over~$R$ can be put into
Smith canonical form. Let~$M$ be a finitely presented torsion-free~$R$-module.
Then~$M$ is finite free.

This goes as follows: Since~$M$ is finitely presented, it is the cokernel of
some matrix. Without loss of generality, we may assume that it is a diagonal
matrix, so~$M$ is isomorphic to some (finite) direct sum~$\bigoplus_i R/(a_i)$.
Since~$M$ is torsion-free, all the summands~$R/(a_i)$ are torsion-free as well.
Since~$R$ is an integral domain in the strong sense, this holds if and only if
the~$a_i$ are either zero or invertible. Thus~$R/(a_i)$ is isomorphic to~$R$ or
to the zero module. In any case,~$R/(a_i)$ is finite free and therefore~$M$
is finite free as well.
\end{proof}


\section{Upper semicontinuous functions}
\label{sect:upper-semicontinuous-functions}

\subsection{Interlude on natural numbers}
In classical logic, the natural numbers are complete in the sense that any
inhabited set of natural numbers possesses a minimal element. This statement
can not be proven intuitionistically -- intuitively, this is because one cannot
explicitly pinpoint the (classically existing) minimal element of an arbitrary
inhabited set;\footnote{Let~$\varphi$ be an arbitrary formula. Assuming that
any inhabited subset of the natural numbers possesses a minimal element, we
want to show that~$\varphi \vee \neg\varphi$. Define the subset $U \defeq \{ n \in
\NN \,|\, (n = 1) \vee \varphi \} \subseteq \NN$, which surely is inhabited by~$1
\in U$. So by assumption, there exists a number~$z \in \NN$ which is the
minimum of~$U$. We have $z = 0$ or $z > 0$. If~$z = 0$, we have~$0 \in U$,
so~$(0 = 1) \vee \varphi$, so~$\varphi$ holds.  If~$z > 0$, then~$\neg\varphi$
holds: If~$\varphi$ were true, zero would be an element of~$U$, contradicting
the minimality of~$z$.} see below for a sheaf-theoretic interpretation.

In intuitionistic logic, the completeness principle can be salvaged in two
essentially different ways: either by strengthening the premise, or by
weakening the conclusion.

\begin{lemma}\label{lemma:minimum-subset-naturals}
Let~$U \subseteq \NN$ be an inhabited subset of the natural
numbers.
\begin{enumerate}
\item Assume~$U$ to be \emph{detachable}, \ie assume that for any natural
number~$n$, either~$n \in U$ or~$n \not\in U$. Then~$U$ possesses a minimal
element.
\item In any case,~$U$ does \emph{not not} possess a minimal element.
\end{enumerate}
\end{lemma}
\begin{proof}
The first statement can be proven by induction on the witness of inhabitation,
\ie the given number~$n$ such that~$n \in U$. We omit further details, since we will
not need this statement in our applications.

For the second statement, we give a careful proof since logical subtleties matter. To simplify the
exposition, we assume that~$U$ is upward-closed, \ie that any number
larger than some element of~$U$ lies in~$U$ as well. Any subset can be closed
in this way (by considering~$\{ n \in \NN \,|\, \exists m \in U\_ n \geq m \}$)
and a minimal element of the closure will be a minimal element for~$U$ as well.

We induct on the number~$n \in U$ given by the assumption that~$U$ is
inhabited. In the case~$n = 0$ we are done since~$0$ is a minimal element
of~$U$. For the induction step~$n \to n+1$, the intuitionistically valid double
negation of the law of excluded middle
gives
\[ \neg\neg(n \in U \vee n \not\in U). \]
Because of the tautologies $(\varphi \Rightarrow \psi) \Rightarrow
(\neg\neg\varphi \Rightarrow \neg\neg\psi)$ and~$\neg\neg\neg\neg\varphi \Rightarrow
\neg\neg\varphi$ (see Section~\ref{sect:appreciating-intuitionistic-logic}), it
suffices to show that~$n \in U \vee n \not\in U$ implies the conclusion.
So assume~$n \in U \vee n \not\in U$.
If~$n \in U$, then~$U$ does \notnot possess a minimal element by the induction
hypothesis. If~$n \not\in U$, then~$n+1$ is a minimal element (and so, in
particular,~$U$ does \notnot possess a minimal element): If~$m$ is
any element of~$U$, we have~$m \geq n+1$ or~$m \leq n$. In the first case,
we're done. In the second case, it follows that~$n \in U$ because~$U$ is
upward-closed and so we obtain a contradiction. From this contradiction we
can trivially deduce~$m \geq n+1$ as well. \qedhere
\end{proof}

If we want to work with a complete partially ordered set (poset) of natural numbers in intuitionistic
logic, we have to construct a suitable completion. The idea of the following
definition is to encode numbers as the (not necessarily existing) minimum of
inhabited upward-closed subsets.
\begin{defn}The \emph{completed poset of natural numbers} is
the set~$\widehat{\NN}$ of all inhabited upward-closed subsets of~$\NN$, ordered by
reverse inclusion. The elements of~$\widehat{\NN}$ are called \emph{generalized natural numbers}.\end{defn}
\begin{lemma}The completed poset of natural numbers is the least poset
containing~$\NN$ and possessing minima
of arbitrary inhabited subsets.\end{lemma}
\begin{proof}
The embedding $\NN \hookrightarrow \widehat\NN$ is given by
\[ n \in \NN \quad\longmapsto\quad {\uparrow}(n) \defeq \{ m \in \NN \,|\, m \geq n \}. \]
If~$M \subseteq \widehat\NN$ is an inhabited subset, its minimum is
\[ \min M = \bigcup M \in \widehat\NN. \]
We omit the proof of the universal property.
\end{proof}

\begin{rem}\label{rem:surjectivity-embedding}
In classical logic, the map~$\widehat\NN \to \NN,\ U \mapsto \min U$
is a well-defined isomorphism of partially ordered sets. In fact, it is the
inverse of the canonical embedding~$\NN \hookrightarrow \widehat\NN$. In
intuitionistic logic, this embedding is still injective, but it can not be
shown to be surjective: It is only the case that any element of~$\widehat\NN$
does \notnot possess a preimage (by Lemma~\ref{lemma:minimum-subset-naturals}).
\end{rem}


\subsection{A geometric interpretation}
We are interested in the completed natural numbers for the following reason: A
generalized natural number in the topos of sheaves on a topological space~$X$ is
the same as an upper semicontinuous function~$X \to \NN$.

\begin{lemma}\label{lemma:upper-semicontinuous-functions}
Let~$X$ be a topological space. The sheaf~$\widehat\NN$ of
generalized natural numbers on~$X$ is canonically isomorphic to the sheaf of upper
semicontinuous~$\NN$-valued functions on~$X$.\end{lemma}
\begin{proof}
When referring to the natural numbers in the internal language, we actually
refer to the constant sheaf~$\ul{\NN}$ on~$X$. (This is because the
sheaf~$\ul{\NN}$ fulfills the axioms of a natural numbers object,
cf.\@~\cite[Section~VI.1]{moerdijk-maclane:sheaves-logic}.) Recall that its sections on an
open subset~$U \subseteq X$ are continuous functions~$U \to \NN$, where~$\NN$
is equipped with the discrete topology.

Therefore, a section of~$\widehat\NN$ on an open subset~$U \subseteq X$ is
given by a subsheaf~$\A \hookrightarrow \ul{\NN}|_U$ such that
\[ U \models \exists n\?\ul{\NN}\_ n \in \A
  \quad\text{and}\quad
  U \models \forall n,m\?\ul{\NN}\_ n \geq m \wedge n \in \A \Rightarrow m \in
  \A. \]
Since these conditions are geometric implications, they are satisfied if and only if any
stalk~$\A_x$ is an inhabited upward-closed subset of~$\ul{\NN}_x \cong \NN$.
The association
\[ x \in X \quad\longmapsto\quad \min\{ n \in \NN \,|\, n \in \A_x \} \]
thus defines a map~$X \to \NN$. This map is indeed upper semicontinuous, since
if~$n \in \A_x$, there exists an open neighbourhood~$V$ of~$x$ such that the constant
function with value~$n$ is an element of~$\Gamma(V,\A)$ and therefore~$n \in
\A_y$ for all~$y \in V$.

Conversely, let~$\alpha : U \to \NN$ be a upper semi-continous function. Then
\[ \text{$V \subseteq U$ open} \quad\longmapsto\quad \{ f : V \to \NN \,|\, \text{$f$
continuous,\ $f \geq \alpha$ on~$V$} \} \]
is a subobject of~$\ul{\NN}|_U$ which internally is inhabited and upward-closed.
Further details are left to the reader.
\end{proof}

Under the correspondence given by the lemma, locally \emph{constant}
functions map exactly to the (image of the) \emph{ordinary} internal natural numbers
(in the completed natural numbers).
In a similar vein, the sheaf given by the internal construction of
the set of \emph{all} upward-closed subsets of the natural numbers (not
only the inhabited ones) is canonically isomorphic to the sheaf of
upper semicontinuous functions with values in~$\NN \cup \{ +\infty
\}$.

Note that the correspondence can be used to understand classical facts about
upper semicontinuous functions as features of intuitionistic number theory. For
instance, it is well-known that any upper semicontinuous~$\mathbb{N}$-valued
function on an arbitrary topological space is locally constant on a dense open subset.
This can be explained as follows: The generalized natural number associated to such a
function is \notnot an ordinary natural number from the internal point of view.
Since ``not not'' translates to ``holding on a dense open subset''
(Proposition~\ref{prop:modops-kripke}), it follows that there is a dense open
subset on which the function corresponds to an ordinary internal natural
number, \ie is locally constant.


\subsection{The upper semicontinuous rank function}
Recall that the rank of an~$\O_X$-module~$\F$ on a scheme~$X$ (or
locally ringed space) at a point~$x \in X$ is defined as the~$k(x)$-dimension
of the vector space~$\F_x \otimes_{\O_{X,x}} k(x)$. If we assume that~$\F$ is
of finite type around~$x$, this dimension is finite and equals the minimal
number of elements needed to generate~$\F_x$ as an~$\O_{X,x}$-module by
Nakayama's lemma.

In the internal language, we can define an element of~$\widehat\NN$ by
\begin{multline*}
  \rank\F \defeq \min\{ n \in \NN \,| \\
  \speak{there is a gen.\@ family for~$\F$ consisting of~$n$ elements} \} \in \widehat\NN.
\end{multline*}
If the module~$\F$ is finite locally free, it will be a finite free module from the
internal point of view and the rank defined in this way will be an
actual natural number (see below); but in general, the rank is really an element of the
completion.

\begin{prop}\label{prop:rank-function-internally}
Let~$\F$ be an~$\O_X$-module of finite type on a scheme~$X$ (or locally ringed
space). Under the correspondence given by the Lemma~\ref{lemma:upper-semicontinuous-functions}, the internally
defined rank maps to the rank function of~$\F$.
\end{prop}
\begin{proof}
We have to show that for any point~$x \in X$ and natural number~$n$, there
exists a generating family for~$\F_x$ consisting of~$n$
elements if and only if there exists an open neighbourhood~$U$ of~$x$ such that
\[ U \models \speak{there exists a generating family
for~$\F$ consisting of~$n$ elements}. \]
The ``if'' direction is obvious. For the ``only if'' direction, consider
(liftings to local sections of a)
generating family~$s_1,\ldots,s_n$ of~$\F_x$. Since~$\F$ is of finite type,
there also exist sections~$t_1,\ldots,t_m$ on some neighbourhood~$V$ of~$x$ which
generate any stalk~$\F_y$, $y \in V$. Since the~$t_i$ can be expressed as a
linear combination of the~$s_j$ in~$\F_x$, the same is true on some open
neighbourhood~$U \subseteq V$ of~$x$. On this neighbourhood, the~$s_j$ generate
any stalk~$\F_y$, $y \in U$, so by geometricity we have
\[ U \models \speak{$s_1,\ldots,s_n$ generate~$\F$}. \qedhere \]
\end{proof}
\begin{rem}Once we understand when properties holding at a point spread to
neighbourhoods, we will be able to give a simpler proof of the proposition (see
Lemma~\ref{lemma:gen-family-n}).\end{rem}

\begin{lemma}\label{lemma:rank-functor-locally-constant}
Let~$X$ be a scheme (or a locally ringed space). Let~$\F$ be an~$\O_X$-module of
finite type. If~$\F$ is finite locally free, its rank function is locally
constant. The converse holds if~$X$ is a reduced scheme.
\end{lemma}
\begin{proof}The rank function is locally constant if and only if internally,
the rank of~$\F$ is an actual natural number. Also recall that the structure
sheaf fulfills a certain field condition if~$X$ is a reduced
scheme (Corollary~\ref{cor:field-reduced}). Therefore it suffices to give a
proof of the following fact of intuitionistic linear algebra: Let~$R$ be a
local ring. Let~$M$ be a finitely generated~$R$-module. If~$M$ is finite
free, its rank is an actual natural number. The converse holds if~$R$ fulfills
the field condition that any element which is not invertible is zero.

So assume that such a module~$M$ is finite free. Then it is isomorphic
to~$R^n$ for some actual natural number~$n$; by the internal proof in
Lemma~\ref{lemma:kernel-of-epi-fingen}, the rank of~$M$ is therefore this
number~$n$ (for any surjection~$R^m \twoheadrightarrow R^n$ it holds that~$m
\geq n$).

Conversely, assume that the rank of~$M$ is an actual natural number. Then
there exists a minimal generating family~$x_1,\ldots,x_n\?M$. We can verify that this family is
indeed linearly independent (and thus a basis, demonstrating that~$M$ is finite
free): Let~$\sum_i a_i x_i = 0$ with~$a_i\?R$. If any~$a_i$ were
invertible, the family~$x_1,\ldots,x_{i-1},x_{i+1},\ldots,x_n$ would too
generate~$M$, contradicting the minimality. So each~$a_i$ is not invertible.
By the field property of~$R$, each~$a_i$ is zero.
\end{proof}

\begin{lemma}\label{lemma:locally-free-dense}
Let~$X$ be a reduced scheme. Let~$\F$ be an~$\O_X$-module of
finite type. Then~$\F$ is finite locally free on a dense open subset.\end{lemma}
\begin{proof}Since ``dense open'' translates to ``not not'' in the internal
language (Proposition~\ref{prop:modops-kripke}), it suffices to give an
intuitionistic proof of the following fact: Let~$R$ be a local ring which fulfills an
appropriate field condition. Let~$M$ be a finitely generated~$R$-module.
Then~$M$ is \notnot finite free.

By Remark~\ref{rem:surjectivity-embedding}, the rank of such a module~$M$ is
\notnot an actual natural number. By the last part of the
previous proof, it thus follows that~$M$ is \notnot finite free.
\end{proof}

\begin{rem}Note that besides basics on natural numbers in an intuitionistic
setting and some dictionary terms (``reduced'', ``finite locally free'',
``finite type``, ``dense open''), this proof does not depend on any further
tools. In particular, Nakayama's lemma and facts about semicontinuous functions
do not enter. For the (more complex) standard proof of this fact, see for
instance~\cite{vakil:foag}, where the claim is dubbed an ``important hard
exercise'' (Exercise~13.7.K).\end{rem}


\subsection{The upper semicontinuous dimension function} Recall that the
dimension of a topological space~$X$ at a point~$x \in X$ is defined as the
infimum
\[ \dim_x X \defeq \inf\{ \dim U \,|\, \text{$U$ open neighbourhood of~$x$} \}. \]

The map~$X \to \NN \cup \{+\infty\},\ x \mapsto \dim_x X$ is upper
semicontinuous and thus corresponds to an internal generalized (possibly
unbounded) natural number. The following proposition shows that this number has
an explicit description.

\begin{prop}Let~$X$ be a scheme. Then the upper semicontinuous function
associated to the internal number ``Krull dimension of~$\O_X\!$'' is the
dimension function~$x \mapsto \dim_x X$.\end{prop}
\begin{proof}Internally, we define the Krull dimension of~$\O_X$ as the infimum
over all natural numbers~$n$ such that~$\O_X$ is of Krull
dimension~$\leq n$. This infimum need not exist in the natural numbers, of
course; so we really mean the upward-closed set~$\A$ of all those numbers. (It
is inhabited if and only if, from the external perspective, the dimension
of~$X$ is locally finite. In this case, it defines a generalized natural number.)

We thus have to show for any point~$x \in X$:
\[ \inf\{ n \in \NN \cup \{+\infty\} \,|\, n \in \A_x \} =
  \dim_x X. \]
The condition on~$n$ can be expressed as follows, where we write~``$\ul{n}$''
to denote the constant function with value~$n$:
\begin{align*}
  &\ n \in \A_x \\
  \Longleftrightarrow &\
  \text{for some open neighbourhood~$U$ of~$x$, $\ul{n} \in \Gamma(U,\A)$} \\
  \Longleftrightarrow &\
  \text{for some open neighbourhood~$U$ of~$x$,} \\
  & \qquad\qquad U \models \speak{$\O_X$ is of Krull dimension~$\leq n$} \\
  \Longleftrightarrow &\
  \text{for some open neighbourhood~$U$ of~$x$,} \\
  & \qquad\qquad \dim U \leq n
\end{align*}
The second equivalence follows from the external description of internally-defined
subsheaves given in Section~\ref{sect:internal-constructions}.
We thus have:
\[
  \inf\{ n \,|\, n \in \A_x \} =
    \inf\{ \dim U \,|\, \text{$U$ open neighbourhood of~$x$} \}
  = \dim_x X. \qedhere
\]
\end{proof}


\section{Modalities}
\label{sect:modalities}

Philosophers and logicians do not only study what is \emph{true}, but also what
is \emph{known}, what is \emph{believed}, what is \emph{possible}, and so on.
Such \emph{modalities} are absent from the usual mathematical practice.
However, it turns out that a specific kind of such modalities plays a role in
understanding when properties spread from points to neighbourhoods.

Briefly, this is because for any point~$x$ of a topological space~$X$, there
exists a modal operator~$\Box$ such that for any formula~$\varphi$ of the
internal language of the sheaf topos~$\Sh(X)$, the internal
statement~$\Box\varphi$ means that~$\varphi$ holds on some open neighbourhood
of the given point~$x$. In this way, we can reduce sheaf-theoretic questions to
questions of modal intuitionistic (non-sheafy) mathematics.

The techniques developed in this section also enable us to use the internal
language of~$\Sh(X)$ to talk about sheaves on \emph{subspaces} of~$X$ (and more
general \emph{sublocales} of~$X$).

Topological interpretations of modal logic were studied before, for instance by
Awodey and Kishida~\cite{awodey-kishida:modal}. However, they study a
different kind of modal operators, not corresponding to the Lawvere--Tierney
topologies of topos theory, and pursue different goals.


\subsection{Basics on truth values and modal operators}
\label{sect:basics-on-truth-values}

\begin{defn}The \emph{set of truth values~$\Omega$} is the powerset of the
singleton set~$1 \defeq \{\star\}$, where~$\star$ is a formal symbol.\end{defn}

In classical logic, any subset of~$\{\star\}$ is either empty or inhabited, so
that~$\Omega$ contains exactly two elements, the empty set (``false'')
and~$\{\star\}$ (``true''). But
in intuitionistic logic, this can not be shown; indeed, if we interpret the
definition in the topos of sheaves on a space~$X$, we obtain a (large) sheaf~$\Omega$
with
\[ \text{$U \subseteq X$ open} \quad\longmapsto\quad \Gamma(U,\Omega) = \{ V \subseteq U \,|\, \text{$V$
open} \}. \]
(This is because by definition of~$\Omega$ as the power object of the terminal
sheaf~$1$, sections of~$\Omega$ on an open subset~$U$ correspond to
subsheaves~$\F \hookrightarrow 1|_U$, and those are given by the greatest open
subset~$V \subseteq U$ such that~$\Gamma(V,\F)$ is inhabited.)
Obviously, in general, this sheaf has many sections, in particular more than
the binary coproduct~$1 \amalg 1$ (unless any open subset of~$X$ is also
closed).

The \emph{truth value} of a formula~$\varphi$ is by definition the subset
$\{ x \in 1 \,|\, \varphi \} \in \Omega$, where~``$x$'' is a fresh variable not
appearing in~$\varphi$. This subset is inhabited if and only
if~$\varphi$ holds and is empty if and only if~$\neg\varphi$ holds.
Conversely, we can associate to a subset~$F \subseteq 1$ the
proposition~$\speak{$F$ is inhabited}$.

By the above description of~$\Omega$ in
a sheaf topos~$\Sh(X)$, the interpretation of the truth value
of a formula~$\varphi$ in the internal language of~$\Sh(X)$ is a certain open
subset of~$X$. Tracing the definitions, we see that this open subset is
precisely the largest open subset on which~$\varphi$ holds, \ie the union of
all open subsets~$U \subseteq X$ such that~$U \models \varphi$.

Under the correspondence of formulas with truth values, logical operations
like~$\wedge$ and~$\vee$ map to set-theoretic operations like~$\cap$ and~$\cup$
-- for instance, we have
\[ \{ x \in 1 \,|\, \varphi \} \cap \{ x \in 1 \,|\, \psi \} =
  \{ x \in 1 \,|\, \varphi \wedge \psi \}. \]
This justifies a certain abuse of notation: We will sometimes treat elements
of~$\Omega$ as propositions and use logical instead of set-theoretic
connectives. In particular, if~$\varphi$ and~$\psi$ are elements of~$\Omega$,
we will write~``$\varphi \Rightarrow \psi$'' to mean~$\varphi \subseteq \psi$;
``$\bot$'' to mean~$\emptyset$; and~``$\top$'' to mean~$1$.

\begin{defn}A \emph{modal operator} (or \emph{Lawvere--Tierney topology}) is a map~$\Box : \Omega \to \Omega$ such
that for all~$\varphi, \psi \in \Omega$,
\begin{enumerate}
\item $\varphi \Longrightarrow \Box\varphi$,
\item $\Box\Box\varphi \Longrightarrow \Box\varphi$,
\item $\Box(\varphi \wedge \psi) \Longleftrightarrow \Box\varphi \wedge \Box\psi$.
\end{enumerate}
\end{defn}

The intuition is that~$\Box\varphi$ is a certain weakening of~$\varphi$, where
the precise meaning of ``weaker'' depends on the modal operator. By the second
axiom, weakening twice is the same as weakening once.

In classical logic, where~$\Omega = \{ \bot, \top \}$, there are only two modal
operators: the identity map and the constant map with value~$\top$.
Both of these are not very interesting: The identity operator does not weaken
propositions at all, while the constant operator weakens every proposition to
the trivial statement~$\top$.

In intuitionistic logic, there can potentially exist further modal operators.
For applications to algebraic geometry, the following four operators will have
a clear geometric meaning and be of particular importance:
\begin{enumerate}
\item $\Box\varphi \defequiv (\alpha \Rightarrow \varphi)$, where~$\alpha$ is a
fixed proposition.
\item $\Box\varphi \defequiv (\varphi \vee \alpha)$, where~$\alpha$ is a
fixed proposition.
\item $\Box\varphi \defequiv \neg\neg\varphi$ (the \emph{double negation
modality}).
\item $\Box\varphi \defequiv ((\varphi \Rightarrow \alpha) \Rightarrow \alpha)$,
where~$\alpha$ is a fixed proposition.
\end{enumerate}

\begin{lemma}Any modal operator~$\Box$ is monotonic, \ie if~$\varphi
\Rightarrow \psi$, then~$\Box\varphi \Rightarrow \Box\psi$. Furthermore, there
holds a modus ponens rule: If~$\Box\varphi$ holds, and~$\varphi$
implies~$\Box\psi$, then~$\Box\psi$ holds as well.\end{lemma}
\begin{proof}Assume~$\varphi \Rightarrow \psi$. This is equivalent to
supposing~$\varphi \wedge \psi \Leftrightarrow \varphi$. We are to show
that~$\Box\varphi \Rightarrow \Box\psi$, \ie that~$\Box\varphi \wedge
\Box\psi \Leftrightarrow \Box\varphi$. This follows since by the third
axiom on a modal operator, we have~$\Box\varphi \wedge \Box\psi \Leftrightarrow
\Box(\varphi \wedge \psi)$, and~$\Box$ respects equivalence of propositions.

For the second statement, consider that if~$\varphi \Rightarrow \Box\psi$, by
monotonicity and the second axiom on a modal operator it follows
that~$\Box\varphi \Rightarrow \Box\Box\psi \Rightarrow \Box\psi$.
\end{proof}

\label{proof-scheme-boxed-statements}The modus ponens rule justifies the
following proof scheme: When trying to show, given that some boxed
statement~$\Box\varphi$ holds, that some further boxed statement~$\Box\psi$
holds, we may give a proof of~$\Box\psi$ under the stronger
assumption~$\varphi$. Symbolically:
\[ (\Box\varphi \Rightarrow \Box\psi) \Longleftrightarrow
  (\varphi \Rightarrow \Box\psi). \]


\subsection{Geometric meaning}\label{sect:modalities-geometric-meaning}
Let~$X$ be a topological space. As discussed
above, an open subset~$U \subseteq X$ defines an internal truth value (a global
section of the sheaf~$\Omega$). We also denote it by~``$U$'', such that
\[ V \models U \quad\Longleftrightarrow\quad V \subseteq U \]
for any open subset~$V \subseteq X$. (Shortcutting the various intermediate
steps, this can also be taken as a definition of~``$V \models U$''.)
If~$A \subseteq X$ is a closed subset, there is thus an internal truth
value~$A^c$ corresponding to the open subset~$A^c = X \setminus A$. If~$x \in
X$ is a point, we define~``$\notat{x}$'' to denote the truth value
corresponding to~$\Int(X \setminus \{x\})$, such that
\[ V \models \notat{x} \quad\Longleftrightarrow\quad V \subseteq \Int(X
\setminus \{ x \}) \quad\Longleftrightarrow\quad x \not\in V. \]

\begin{prop}\label{prop:modops-kripke}
Let~$U \subseteq X$ be a fixed open and~$A \subseteq X$ be a fixed
closed subset. Let~$x \in X$. Then, for any open subset~$V \subseteq X$, it
holds that:
\[ \renewcommand{\arraystretch}{1.3}\begin{array}{@{}lcl@{}}
  V \models (U \Rightarrow \varphi) &\Longleftrightarrow&
    V \cap U \models \varphi. \\[0.3em]
  V \models (\varphi \vee A^c) &\Longleftrightarrow&
    \textnormal{there is an open subset~$W \subseteq V$} \\
  && \quad\quad \textnormal{containing~$A \cap V$ such that $W \models \varphi$.} \\[0.3em]
  V \models \neg\neg\varphi &\Longleftrightarrow&
    \textnormal{there is a dense open subset~$W \subseteq V$ s.\,th.\@ $W \models
    \varphi$.} \\[0.3em]
  V \models ((\varphi \Rightarrow \notat{x}) \Rightarrow \notat{x}) &\Longleftrightarrow&
    \textnormal{$x \not\in V$ or there is an open neighbourhood~$W \subseteq V$} \\
  && \quad\quad \textnormal{of~$x$ such that $W \models \varphi$.}
\end{array} \]
\end{prop}
\begin{proof}
\begin{enumerate}
\item Omitted.

\item Let~$V \models \varphi \vee A^c$. Then there exists an open covering~$V =
\bigcup_i V_i$ such that for each~$i$, $V_i \models \varphi$ or $V_i \subseteq
A^c$. Let~$W \subseteq V$ be the union of those~$V_i$ such that~$V_i \models \varphi$.
Then~$W \models \varphi$ by the locality of the internal language and~$A \cap V
\subseteq W$.

Conversely, let~$W \subseteq V$ be an open subset containing~$A \cap V$ such
that~$W \models \varphi$. Then~$V = W \cup (V \cap A^c)$ is an open covering
attesting~$V \models \varphi \vee A^c$.

\item For the ``only if'' direction, let~$W \subseteq V$ be the largest
open subset on which~$\varphi$ holds, \ie the union of all open subsets
of~$V$ on which~$\varphi$ holds. For the ``if'' direction, we may assume that
the given set~$W$ is also the largest open subset on which~$\varphi$ holds (by
enlarging~$W$ if necessary). The claim then follows by the following chain of
equivalences:
\begin{align*}
  &\ V \models \neg\neg\varphi \\
  \Longleftrightarrow&\ \forall \text{$Y \subseteq V$ open}\_
    \Bigl(\forall \text{$Z \subseteq Y$ open}\_ (Z \models \varphi) \Rightarrow Z
    = \emptyset\Bigr) \Longrightarrow Y = \emptyset \\
  \Longleftrightarrow&\ \forall \text{$Y \subseteq V$ open}\_
    \Bigl(\forall \text{$Z \subseteq Y$ open}\_ Z \subseteq W \Rightarrow Z
    = \emptyset\Bigr) \Longrightarrow Y = \emptyset \\
  \Longleftrightarrow&\ \forall \text{$Y \subseteq V$ open}\_
    Y \cap W = \emptyset \Longrightarrow Y = \emptyset \\
  \Longleftrightarrow&\ \text{$W$ is dense in~$V$.}
\end{align*}

\item Straightforward, since the interpretation of the internal statement with
the Kripke--Joyal semantics is
\[ \forall \text{$Y \subseteq V$ open}\_
  \Bigl(\forall \text{$Z \subseteq Y$ open}\_
    Z \models \varphi \Rightarrow x \not\in Z\Bigr) \Longrightarrow x \not\in
    Y. \qedhere \]
\end{enumerate}
\end{proof}


\subsection{The subspace associated to a modal operator}
\label{sect:subspace-to-modal-operator}
Any modal operator~$\Box : \Omega \to \Omega$ in the sheaf topos of~$X$ induces
on global sections a map
\[ j : \Open(X) \to \Open(X), \]
where~$\Open(X) = \Gamma(X,\Omega)$ is the set of open subsets of~$X$.
Explicitly, it is given by
\begin{align*}
  j(U) &= \text{largest open subset of~$X$ on which~$\Box U$ holds} \\
  &= \bigcup\ \{ V \subseteq X \ |\ \text{$V$ open},\ V \models \Box U \}.
\end{align*}
By the axioms on a modal operator, the map~$j$ fulfills similar such axioms: For any open
subsets~$U, V \subseteq X$,
\begin{enumerate}
\item $U \subseteq j(U)$,
\item $j(j(U)) \subseteq j(U)$,
\item $j(U \cap V) = j(U) \cap j(V)$.
\end{enumerate}
Such a map is called a \emph{nucleus} on~$\Open(X)$. Table~\ref{table:nuclei}
lists the nuclei associated to the four modal operators
of Proposition~\ref{prop:modops-kripke}.

\begin{table}
  \centering
  \renewcommand{\arraystretch}{1.3}
  \begin{tabular}{llll}
    \toprule
    Modal operator & associated nucleus &
      $j(V) = X$ iff \ldots &
      subspace \\\midrule
    $\Box\varphi \defequiv (U \Rightarrow \varphi)$ &
      $j(V) = \Int(U^c \cup V)$ & $U \subseteq V$ & $U$ \\
    $\Box\varphi \defequiv (\varphi \vee A^c)$ &
      $j(V) = V \cup A^c$ & $A \subseteq V$ & $A$ \\
    $\Box\varphi \defequiv \neg\neg\varphi$ &
      $j(V) = \Int(\Clos(V))$ & $V$ is dense in $X$ &
      \multicolumn{1}{p{1cm}}{smallest dense sublocale of~$X$} \\
    $\Box\varphi \defequiv ((\varphi \Rightarrow \notat{x}) \Rightarrow \notat{x})$ &
%      $\Int(\Clos(V \cap \Clos\{x\}) \cup (X \setminus \Clos\{x\}))$ &
      $\begin{array}{@{}ll@{}}
        j(V) = X \setminus \Clos\{x\}, & \text{if $x \not\in V$} \\
        j(V) = X, & \text{if $x \in V$}
      \end{array}$ &
      $x \in V$ & $\{x\}$ \\
    \bottomrule
  \end{tabular}
  \vspace{0.5em}

  \caption{\label{table:nuclei}List of important modal operators and their
  associated nuclei (notation as in Proposition~\ref{prop:modops-kripke}).}
\end{table}

Any nucleus~$j$ defines a subspace~$X_j$ of~$X$, to be described below, with a small caveat: In
general, the subspace~$X_j$ can not be realized as a topological subspace, but
only as a so-called \emph{sublocale}; the notion of a locale is a slight
generalization of the notion of a topological space, in which an underlying set
of points is not part of the definition. Instead, a locale is simply given by a
lattice of arbitrary \emph{opens} satisfying some axioms -- these opens may, but do not necessarily have to,
be sets of points. Sheaf theory carries over to locales essentially unchanged,
since the notions of presheaves and sheaves only refer to open sets and coverings,
but not points.
Accessible introductions to the theory of locales include two notes by
Johnstone~\cite{johnstone:art,johnstone:point}.

\begin{defn}\label{defn:subspace-by-nucleus}Let~$j$ be a nucleus on~$\Open(X)$.
Then the sublocale~$X_j$ of~$X$ is given by the lattice of opens
$\Open(X_j) \defeq \{ U \in \Open(X) \,|\, j(U) = U \}$.
\end{defn}
If~$j$ is induced by a modal operator~$\Box$, we also write~``$X_\Box$''
for~$X_j$. In three of the four cases listed in Table~\ref{table:nuclei}, the
sublocale~$X_\Box$ can indeed be realized as a topological subspace. The only
exception is the sublocale~$X_{\neg\neg}$ associated to the double negation
modality. It can also be described as the \emph{smallest dense sublocale}
of~$X$; this is obviously a genuine locale-theoretic notion, since there
is (in general) no smallest dense topological subspace
(consider~$\RR$ and its dense subsets~$\QQ$ and~$\RR \setminus \QQ$).

The inclusion~$i : X_j \hookrightarrow X$ can not in general be described on the
level of points, since~$X_j$ might not be realizable as a topological subspace.
But for sheaf-theoretic purposes, it suffices to describe~$i$ on the level of
opens. This is done as follows:
\[ i^{-1} : \Open(X) \lra \Open(X_j),\ \quad U \longmapsto j(U). \]
Thus we can relate the toposes of sheaves on~$X_j$ and~$X$ by the usual
pullback and pushforward functors.
\begin{align*}
  i^{-1} \F &= \text{sheafification of $(U \mapsto \colim_{U \preceq i^{-1}V} \Gamma(V,\F))$} \\
  i_* \G &= (U \mapsto \Gamma(i^{-1}U, \G)) = (U \mapsto \Gamma(j(U), \G))
\end{align*}
As familiar from honest topological subspace inclusions, the pushforward
functor~$i_* : \Sh(X_j) \to \Sh(X)$ is fully faithful and the composition~$i^{-1}
\circ i_* : \Sh(X_j) \to \Sh(X_j)$ is (canonically isomorphic to) the identity.


\subsection{Internal sheaves and sheafification}\label{sect:internal-sheaves}
It turns out that the image of
the pushforward functor~$i_* : \Sh(X_\Box) \to \Sh(X)$, where~$\Box$ is a modal
operator in~$\Sh(X)$, can be explicitly described. Namely, it consists exactly
of those sheaves which from the internal point of view
are so-called~\emph{$\Box$-sheaves}, a notion explained below.

Furthermore, if we identify~$\Sh(X_\Box)$ with its image in~$\Sh(X)$, the
pullback functor is given by an internal sheafification process with respect to
the modality~$\Box$. Thus the external situation of pushforward/pullback
translates to forget/sheafify. This broadens the scope of the internal
language of~$\Sh(X)$: It can not only be used to talk about sheaves on~$X$ in a simple,
element-based language, but also to talk about sheaves on arbitrary subspaces
of~$X$.

To describe the notion of~$\Box$-sheaves and related ones, we switch to the internal
perspective and thus forget that we're working over a base space~$X$; we are simply given a modal operator~$\Box :
\Omega \to \Omega$ and have to take care that our proofs are intuitionistically acceptable. A
reference for the material in this subsection is a preprint by Fer-Jan de
Vries~\cite{vries:sheafification}.\footnote{Note that on page~5 of that
preprint there is a slight typing error: Fact~2.1(i) gives the
characterization of~$j$-closedness, not~$j$-denseness. The correct
characterization of~$j$-denseness in that context is~$\forall b \in B\_ j(b \in
A)$.}

Recall that a set~$S$ is a \emph{subsingleton} if and only if~$\forall x,y\?S\_
x = y$, and that a set~$S$ is a \emph{singleton} if and only if it is a subsingleton and
inhabited (\ie~$\exists x\?S\_ \top$); this amounts to~$\exists!x\?S\_ \top$.

\begin{defn}\label{defn:box-sheaves}
A set~$F$ is \emph{$\Box$-separated} if and only if
\[ \forall x,y\?F\_ \Box(x = y) \Longrightarrow x = y. \]
A set~$F$ is a \emph{$\Box$-sheaf} if and only if it is~$\Box$-separated and
\[ \forall S \subseteq F\_
  \Box(\speak{$S$ is a singleton}) \Longrightarrow
  \exists x\?F\_ \Box(x \in S). \]
\end{defn}

The two conditions can be combined: A set~$F$ is a~$\Box$-sheaf if and only if
\[ \forall S \subseteq F\_
  \Box(\speak{$S$ is a singleton}) \Longrightarrow
  \exists! x\?F\_ \Box(x \in S). \]
\XXX{explain how to read these definitions}

\begin{defn}\label{defn:plus-construction}
The \emph{plus construction} of a set~$F$ with respect to~$\Box$ is the set
\[ F^+ \defeq \{ S \subseteq F \,|\, \Box(\speak{$S$ is a singleton}) \}/{\sim},
\]
where the equivalence relation is defined by~$S \sim T \vcentcolon\Leftrightarrow
\Box(S = T)$. There is a canonical map~$F \to F^+$ given by~$x \mapsto
[\{x\}]$. The \emph{$\Box$-sheafi\-fi\-ca\-tion} of a set~$F$ is the
set~$F^{++}$.
\end{defn}

If~$F$ is~$\Box$-separated, then for any subset~$S \subseteq F$ it holds
that
\[ \Box(\speak{$S$ is a singleton}) \quad\Longleftrightarrow\quad
  \speak{$S$ is a subsingleton} \wedge \Box(\speak{$S$ is inhabited}). \]

\begin{rem}The topos of \emph{pre}sheaves on a topological space~$X$ admits an
internal language as well~\cite[Section~VI.7, discussion after
Theorem~1]{moerdijk-maclane:sheaves-logic}. In it, there
exists a modal operator~$\Box$ reflecting the topology of~$X$. A presheaf on~$X$ is separated
in the usual sense if, from the internal perspective of~$\PSh(X)$, it
is~$\Box$-separated; and it is a sheaf if, from the internal perspective, it
is a~$\Box$-sheaf. Furthermore, the~$\Box$-sheafification of a presheaf
(considered as a set from the internal perspective) coincides with the usual
sheafification.\end{rem}

\begin{ex}\label{ex:special-sets-sheaves}
Any singleton set is a~$\Box$-sheaf. The empty set is
always~$\Box$-separated (trivially) and is a~$\Box$-sheaf if and only
if~$\Box\bot \Rightarrow \bot$.\end{ex}

We will see geometric examples of~$\Box$-sheaves in further sections.
For instance, on an integral or locally Noetherian scheme~$X$, the structure sheaf~$\O_X$
is~$\neg\neg$-separated and its~$\neg\neg$-sheafification is the sheaf~$\K_X$
of rational functions (Proposition~\ref{prop:kx-is-negneg-sheafification}).

\begin{lemma}For any set~$F$, it holds that: \begin{enumerate}
\item $F^+$ is~$\Box$-separated.
\item The canonical map~$F \to F^+$ is injective if and only if~$F$
is~$\Box$-separated.
\item If~$F$ is~$\Box$-separated, then $F^+$ is a~$\Box$-sheaf.
\item If~$F$ is a~$\Box$-sheaf, then the canonical map~$F \to F^+$ is bijective.
\end{enumerate}
Let $\Sh_\Box(\Set)$ be the full subcategory of~$\Set$ consisting of
the~$\Box$-sheaves. Then it holds that:
\begin{enumerate}
\addtocounter{enumi}{4}
\item The functor~$(\placeholder)^+ : \Set \to \Set$ is left exact.
\item The functor~$(\placeholder)^{++} : \Set \to \Sh_\Box(\Set)$ is left exact and left
adjoint to the forgetful functor~$\Sh_\Box(\Set) \to \Set,\ F \mapsto F$.
\end{enumerate}\end{lemma}
\begin{proof}These are all straightforward, and in fact simpler than their
classical counterparts, since there are no colimit constructions which would have to
be dealt with.
\end{proof}

\begin{rem}\label{rem:epi-in-box-sheaves}
As is to be expected from the familiar inclusion of sheaves in
presheaves on topological spaces, the forgetful functor~$\Sh_\Box(\Set) \to \Set$
does not in general preserve colimits. It is instructive to see why
epimorphisms in~$\Sh_\Box(\Set)$ need not be epimorphisms in~$\Set$: A map~$f:A
\to B$ between~$\Box$-sheaves is an epimorphism in~$\Sh_\Box(\Set)$ if and only
if
\[ \forall y\?B\_ \Box(\exists x\?X\_ f(x) = y), \]
\ie preimages do not need to exist, it suffices for them to~``$\Box$-exist''.
(Using results about the~$\Box$-translation, to be introduced below, this
characterization will be obvious.) This condition is intuitionistically weaker
than the condition that~$f$ is an epimorphism in~$\Set$, \ie that~$f$ is
surjective. Compare this to the failure of the forgetful functor~$\Sh(X)
\to \PSh(X)$ to preserve epimorphisms: A morphism of sheaves does not need to
have preimages for any local section in order to be an epimorphism. Instead, it
suffices for any local section to \emph{locally} have preimages.\end{rem}

\begin{prop}Let~$X$ be a topological space. Let~$\Box$ be a modal operator
in~$\Sh(X)$. Let~$i : X_\Box \hookrightarrow X$ be the inclusion of the
associated sublocale. Corestricting the pushforward functor~$i_* : \Sh(X_\Box) \to
\Sh(X)$ to its essential image, it induces an equivalence~$\Sh(X_\Box) \simeq
\Sh_\Box(\Sh(X))$ between the category of sheaves on~$X_\Box$ and the category
of~$\Box$-sheaves in~$\Sh(X)$.
\end{prop}
\begin{proof}For the further development of the theory, we need the statement
of this proposition, but not the proof, which really is routine in dealing with
subtoposes and modal operators. Nevertheless, a proof goes like follows:
Combine Example~A4.6.2(a) and Theorem~C1.4.7
of~\cite{johnstone:elephant} and note that for a topos of sheaves on a
locale~$Y$, it holds that~$\Open(Y) = \Gamma(Y, \Omega_{\Sh(Y)})$, and that the
subobject classifier of~$\Sh_\Box(\Sh(X))$ is~$\{ \varphi : \Omega_{\Sh(X)} \,|\,
\Box \varphi \Leftrightarrow \varphi \}$.
\end{proof}

\begin{rem}It's possible to rewrite the sheaf condition in the following form.
A set~$F$ is~$\Box$-separated if and only if, for any truth value~$\varphi \?
\Omega$ such that~$\Box\varphi$, the canonical map
\[ F \lra F^\varphi, \]
which maps an element~$x\?F$ to the constant map~$\varphi \to X$ with value~$x$
(where~$\varphi$ is considered as a subset of the terminal set~$1$), is
injective. The set~$F$ is a~$\Box$-sheaf if and only if furthermore this map is
surjective for all such truth values.
\end{rem}


\subsection{Sheaves for the double negation modality}
\label{sect:negneg-sheaves}

Recall that if~$\Box$ is the modal operator associated to a sub\emph{space}~$Y$
of a topological space~$X$, then the sheaves on~$X$ which are~$\Box$-sheaves
are easy to describe: These are precisely the sheaves in the essential image of
the pushforward functor~$\Sh(Y) \to \Sh(X)$. For the double negation modality,
the same is true, only that~$Y$ is then the perhaps unfamiliar \emph{smallest
dense sublocale} of~$X$.

The following proposition gives a characterization of~$\neg\neg$-separated
presheaves and~$\neg\neg$-sheaves in explicit terms.

\begin{prop}\label{prop:negneg-sheaves}
Let~$X$ be a topological space. Let~$\F$ be a sheaf on~$X$. Then:
\begin{enumerate}
\item $\F$ is~$\neg\neg$-separated if and only if any two local sections
of~$\F$, which are defined on a common domain and which agree on a dense open
subset of their domain, are already equal.
\item $\F$ is a~$\neg\neg$-sheaf if and only if it is~$\neg\neg$-separated and
for any open subset~$U \subseteq X$ and any open subset~$V \subseteq U$ dense
in~$U$, any~$V$-section of~$\F$ extends to a~$U$-section of~$\F$.
\item If~$\F$ is~$\neg\neg$-separated, the sections of $\F^+$ on an open
subset~$U \subseteq X$ can be described by pairs~$(V,s)$, where~$V$ is a dense
open subset of~$U$ and~$s$ is a section of~$\F$ on~$V$. Two such pairs~$(V,s),
(V',s')$ determine the same element in~$\Gamma(U,\F^+)$ if and only if~$s$ and~$s'$
agree on~$V \cap V'$.
\end{enumerate}
\end{prop}
\begin{proof}
The first statement is obvious from the definition of~$\neg\neg$-separatedness
(Definition~\ref{defn:box-sheaves} for~$\Box = \neg\neg$) and the geometric
interpretation of double negation (Proposition~\ref{prop:modops-kripke}).

For the second statement, we need to show that if~$\F$
is~$\neg\neg$-separated,~$\F$ has the extension property if and only if
\begin{multline*}
  \Sh(X) \models \forall \S \? \P(\F)\_
  \speak{$\S$ is a subsingleton} \wedge
  \neg\neg(\speak{$\S$ is inhabited}) \Longrightarrow \\
  \exists x\?\F\_ \neg\neg(x \in \S).
\end{multline*}
Note that a section~$\S \in \Gamma(U,\P(\F))$ which internally is a
subsingleton and \notnot inhabited is precisely a subsheaf~$\S \hookrightarrow
\F|_U$ such that all stalks~$\S_x$, $x \in U$ are subsingletons and such that for
some dense open subset~$V \subseteq U$, the stalks~$\S_x$, $x \in V$ are
inhabited. This is precisely the datum of a section of~$\F$ defined on some
dense open subset of~$U$: Consider the gluing of the unique germs in~$\S_x$ for
those points~$x$ such that~$\S_x$ is inhabited. (Conversely, a section~$s \in
\Gamma(V,\F)$ defines a subsheaf~$\S$ by setting~$\Gamma(W,\S) \defeq \{ s|_W \,|\,
W \subseteq V \}$.)

In view of this explicit description and the observation that the asserted
existence~(``$\exists x\?\F\_ \neg\neg(x \in \S)$'') is actually a question of
unique existence, the second statement follows.

For the third statement, one can check that the presheaf on~$X$ defined by
\[ \text{$U \subseteq X$ open} \quad\longmapsto\quad
  \{ (V,s) \,|\, \text{$V \subseteq U$ dense open},\ s \in \Gamma(V,\F)
  \}/{\sim} \]
is in fact a sheaf (with respect to the topology of~$X$), internally a $\neg\neg$-sheaf,
and that it has the universal property of the~$\neg\neg$-sheafification
of~$\F$.
\end{proof}

The conditions~(1) and~(2) of the previous proposition can be
summarized as follows: A sheaf~$\F$ on a topological space is
a~$\neg\neg$-sheaf if and only if, for any open subset~$U \subseteq X$, the
restriction map~$\Gamma(\Int\Clos U, \F) \to \Gamma(U,\F)$ is
bijective~\cite[Lemma~36]{jackson:sheaf-theoretic-measure-theory}.

In the case that~$X$ contains a \emph{generic point}, that is a point~$\xi \in X$
such that~$\Clos\{\xi\} = X$, we can describe the sublocale~$X_{\neg\neg}$ in
very explicit terms: In this case, it coincides with the subspace~$\{\xi\}$.
Such a point exists and is unique if~$X$ is an irreducible scheme and need not
exist otherwise.

\begin{lemma}\label{lemma:negneg-generic-point}
Let~$X$ be a topological space and~$\xi \in X$ a point such
that~$\Clos\{\xi\} = X$. Then the modal operator~$\Box \defequiv ((\placeholder
\Rightarrow \notat{\xi}) \Rightarrow \notat{\xi})$ coincides with the double
negation modality and~$X_{\neg\neg} = \{\xi\}$ as sublocales of~$X$.\end{lemma}
\begin{proof}The semantics of the formula~$\notat{\xi}$ was defined by the
equivalence
\[ U \models \notat{\xi} \quad\Longleftrightarrow\quad
  \xi \not\in U. \]
By the assumption on~$\xi$, this is equivalent to requiring~$U = \emptyset$.
Thus for any open subset~$U$ the formulas~$\notat{\xi}$ and~$\bot$ have the
same meaning; they are therefore logically equivalent from the internal point of
view. The given modal operator thus simplifies to
\[ \Box\varphi \quad\equiv\quad ((\varphi \Rightarrow \notat{\xi}) \Rightarrow \notat{\xi})
  \quad\Leftrightarrow\quad ((\varphi \Rightarrow \bot) \Rightarrow \bot)
  \quad\Leftrightarrow\quad \neg\neg\varphi. \]
The second claim follows Table~\ref{table:nuclei}.
\end{proof}

\begin{cor}\label{cor:negneg-generic-point-pushpull}
Let~$X$ be a topological space and~$\xi \in X$ a point such
that~$\Clos\{\xi\} = X$. Since~$X_{\neg\neg} = \{\xi\}$, the
category of~$\neg\neg$-sheaves in~$\Sh(X)$ coincides with the category of
sheaves on~$\{\xi\}$ and can therefore be identified with the category of sets.
Under this identification,
\begin{enumerate}
\item sheafifying an object~$\F \in \Sh(X)$ with respect
to the double negation modality (\ie pulling back to~$X_{\neg\neg}$) is the
same as calculating its generic stalk~$\F_\xi$ and
\item pushing forward a set~$M$ along~$X_{\neg\neg} \hookrightarrow X$ is the
same as calculating the constant sheaf associated to~$M$.
\end{enumerate}
\end{cor}
\begin{proof}The first statement follows because pulling back to~$X_{\neg\neg}$
is the same as pulling back to~$\{\xi\}$. The pushforward of a set~$M$,
considered as a sheaf on~$X_{\neg\neg}$, to~$X$ is explicitly given by
\[ U \quad\longmapsto\quad \begin{cases}
  M, & \text{if $U \neq \emptyset$,} \\
  \{\star\}, & \text{else.}
\end{cases} \]
We omit the routine verification that this sheaf coincides with the constant
sheaf~$\underline{M}$ associated to~$M$.
\end{proof}

The following technical lemma will occasionally be handy. It is an internal
reflection of the fact that an open subset of an affine scheme can always be
written as the union of standard open subsets. We will generalize it
to schemes which are not necessarily integral in
Section~\ref{sect:rational-functions} (see
Lemma~\ref{lemma:dense-standard-reflection-generalized}).

\begin{lemma}\label{lemma:dense-standard-reflection}
Let~$X$ be an integral scheme. Let~$\varphi$ be any formula
over~$X$. Then
\[ \Sh(X) \models \neg\neg\varphi \Longrightarrow \exists f\?\O_X\_
  \neg\neg(\speak{$f$ \inv}) \wedge (\speak{$f$ \inv} \Rightarrow \varphi). \]
\end{lemma}
\begin{proof}We may assume that~$X$ is the spectrum of an integral domain~$A$
and that there is a dense open subset~$U \subseteq X$ on which~$\varphi$ holds.
The open set~$U$ may be covered by standard open subsets~$D(f_i)$; since~$X$ is
irreducible, at least one of these is itself
dense. We may take this~$f_i$ as the sought~$f$.
\end{proof}

We can now also follow up on a promise made earlier and prove the following
somewhat tangential lemma.
\begin{lemma}\label{lemma:boolean-dense}
Let~$X$ be a topological space. The internal language of~$\Sh(X)$ is Boolean if
and only if for any open subset~$U \subseteq X$ it holds that~$U$ is the only
dense open subset of~$U$.
\end{lemma}
\begin{proof}That the internal language of~$\Sh(X)$ is Boolean amounts to
\[ \Sh(X) \models \forall \varphi\?\Omega\_ \neg\neg\varphi \Rightarrow
\varphi. \]
This is equivalent to the external statement that for any open subset~$U
\subseteq X$ and for any open subset~$V \subseteq U$ it holds that: If~$V$ is
dense in~$U$, then~$V$ is equal to~$U$.
\end{proof}


\subsection{\texorpdfstring{The~$\Box$-translation}{The □-translation}}
There is certain well-known transformation~$\varphi
\mapsto \varphi^{\neg\neg}$ on formulas, the \emph{double negation
translation}, with the following curious property: A formula~$\varphi$ is
derivable in classical logic if and only if its
translation~$\varphi^{\neg\neg}$ is derivable in intuitionistic logic. The
translation~$\varphi^{\neg\neg}$ is obtained from~$\varphi$ by putting
``$\neg\neg$'' before any subformula, \ie before any~``$\exists$''
and~``$\forall$'', around any logical connective, and around any atomic
statement (``$x=y$'', ``$x \in A$''). For instance, the double negation
translation of ``$f$ is surjective'' is
\[ \neg\neg \forall y\?Y\_ \neg\neg \exists x\?X\_ \neg\neg f(x) = y. \]

We will describe a slight generalization of the double negation translation,
the~$\Box$-translation for any modal operator~$\Box$. It will be pivotal
for using the internal language of a space~$X$ to express internal statements
about sheaves defined on subspaces of~$X$. The~$\Box$-translation has been studied
in other contexts
before~\cite{aczel:russell-prawitz,escardo:oliva:peirce-shift}. To the best of
my knowledge, this application -- expressing the internal language of
subtoposes in the internal language of the ambient topos -- is new.

\begin{defn}The~\emph{$\Box$-translation} is recursively defined as follows.
\newcommand{\optBox}{\textcolor{gray}{\Box}}
\begin{align*}
  (f = g)^\Box &\defequiv \Box(f = g) \\
  (x \in A)^\Box &\defequiv \Box(x \in A) \\
  \top^\Box &\defequiv \Box\top \quad \text{($\Leftrightarrow \top$)} \\
  \bot^\Box &\defequiv \Box\bot \\
  (\varphi \wedge \psi)^\Box &\defequiv \optBox(\varphi^\Box \wedge \psi^\Box) &
  \textstyle (\bigwedge_i \varphi_i)^\Box &\defequiv \textstyle \optBox(\bigwedge_i \varphi_i^\Box) \\
  (\varphi \vee \psi)^\Box &\defequiv \Box(\varphi^\Box \vee \psi^\Box) &
  \textstyle (\bigvee_i \varphi_i)^\Box &\defequiv \textstyle \Box(\bigvee_i \varphi_i^\Box) \\
  (\varphi \Rightarrow \psi)^\Box &\defequiv \optBox(\varphi^\Box \Rightarrow \psi^\Box) \\
  (\forall x\?X\_ \varphi)^\Box &\defequiv \optBox(\forall x\?X\_ \varphi^\Box) &
  (\forall X\_ \varphi)^\Box &\defequiv \optBox(\forall X\_ \varphi^\Box) \\
  (\exists x\?X\_ \varphi)^\Box &\defequiv \Box(\exists x\?X\_ \varphi^\Box) &
  (\exists X\_ \varphi)^\Box &\defequiv \Box(\exists X\_ \varphi^\Box)
\end{align*}
\end{defn}

\begin{defn}A formula~$\varphi$ is \emph{$\Box$-stable} if and only
if~$\Box\varphi$ implies~$\varphi$.\end{defn}

\begin{lemma}\begin{enumerate}
\item Formulas in the image of the $\Box$-translation are~$\Box$-stable,
\ie for any formula~$\varphi$ it holds that
$\Box(\varphi^\Box) \Longrightarrow \varphi^\Box$.
\item In the definition of the~$\Box$-translation, one may omit the boxes
printed in gray.
\end{enumerate}\end{lemma}
\begin{proof}The first statement is obvious, since one of the axioms on a modal
operator demands that~$\Box\Box\varphi \Rightarrow \Box\varphi$ for any
formula~$\varphi$. The second statement follows by an induction on the
formula structure. By way of example, we prove the case for~``$\Rightarrow$'':
\newcommand{\withgray}{\text{$\Box$ with the gray parts}}
\newcommand{\withoutgray}{\text{$\Box$ without the gray parts}}
\begin{align*}
  &\ (\varphi \Rightarrow \psi)^\withgray \\
  \Longleftrightarrow &\ \Box(\varphi^\withgray \Rightarrow \psi^\withgray) \\
  \Longleftrightarrow &\ (\varphi^\withgray \Rightarrow \psi^\withgray) \\
  \Longleftrightarrow &\ (\varphi^\withoutgray \Rightarrow \psi^\withoutgray) \\
  \Longleftrightarrow &\ (\varphi \Rightarrow \psi)^\withoutgray
\end{align*}
The first step is by definition; the second by~$\Box$-stability
of~$\psi^\withgray$ and the intuitionistic tautology~$\Box(\alpha \Rightarrow
\beta) \Leftrightarrow (\alpha \Rightarrow \beta)$ for~$\Box$-stable
formulas~$\beta$; the third by induction hypothesis; the fourth by
definition.
\end{proof}

\begin{lemma}\label{lemma:box-translation-sound}
The~$\Box$-translation is sound with respect to intuitionistic logic:
Assume that there exists an intuitionistic proof of an
implication~$\varphi \Rightarrow \psi$. Then there is also an intuitionistic
proof of the translated implication~$\varphi^\Box \Rightarrow \psi^\Box$.
\end{lemma}
\begin{proof}This follows by an induction on the structure of intuitionistic
proofs. We have to verify that we can mirror any inference rule of
intuitionistic logic in the translation. For instance, one of the disjunction
rules justifies the following proof scheme: In order to prove~$\varphi \vee
\psi \Rightarrow \chi$, it suffices to give proofs of~$\varphi \Rightarrow
\chi$ and~$\psi \Rightarrow \chi$. We have to justify the translated proof
scheme: In order to prove~$(\varphi \vee \psi)^\Box \Rightarrow \chi^\Box$, it
suffices to give proofs of~$\varphi^\Box \Rightarrow \chi^\Box$ and~$\psi^\Box
\Rightarrow \chi^\Box$.

So assume that proofs of the two implications are given. Further
assume~$(\varphi \vee \psi)^\Box$, \ie~$\Box(\varphi^\Box \vee \psi^\Box)$.
We want to show~$\chi^\Box$. Since this is a~$\Box$-stable statement, we may
assume that in fact~$\varphi^\Box \vee \psi^\Box$ holds. Then the claim is
obvious by the two given proofs.

The cases for the other rules (see Appendix~\ref{appendix:inference-rules} for
a list) are similar and left to the reader.\end{proof}

\begin{rem}The reader well-versed in formal logic will have noticed that we are
mixing syntax and semantics here. The proper way to state the lemma would be
to formally adjoin a box operator to the language of intuitionistic logic,
governed by three inference rules which are modeled on the three axioms on a
modal operator. This formal box operator could then be instantiated by any
concrete modal operator~$\Box : \Omega \to \Omega$.\end{rem}

Soundness of the~$\Box$-translation is important for the following reason.
If~$\varphi$ and~$\varphi'$ are equivalent formulas, we are
accustomed to be able to freely substitute~$\varphi$ by~$\varphi'$ anywhere we
want. Since a modal operator~$\Box$ is semantically defined as a map~$\Omega
\to \Omega$, it is trivially justified that~$\Box\varphi$ and~$\Box\varphi'$
are equivalent: The formulas~$\varphi$ and~$\varphi'$ give rise to the
\emph{same} element~$\{x \in 1 \,|\, \varphi\} = \{x \in 1 \,|\, \varphi'\}$
of~$\Omega$, and therefore their images under~$\Box$ are equal as well.

However, it is \emph{not} clear and in fact wrong in general that the translated formulas~$\varphi^\Box$
and~$(\varphi')^\Box$ are equivalent. This follows only if the soundness
lemma can be applied (two times, once for each direction). We should stress that to apply this
lemma, it is not enough to merely \emph{know} that~$\varphi$ and~$\varphi'$ are
equivalent; instead, there has to be an intuitionistic proof of this
equivalence. This is really a stronger requirement, since an
equivalence~$\varphi \Leftrightarrow \varphi'$ might
hold in a particular model, \ie in the internal language of some particular
topos, without possessing an intuitionistic proof, \ie holding in any topos. We
give an explicit example of this situation below
(Example~\ref{ex:translation-equivalence}).

\begin{lemma}\label{lemma:open-stalk}
Let~$\varphi$ be a formula such that for any subformulas~$\psi$
appearing as antecedents of implications, it holds that~$\psi^\Box \Rightarrow
\Box\psi$. (In particular, this condition is satisfied if there are
no~``$\Rightarrow$'' signs in~$\varphi$ or if~$\varphi$ is a geometric formula.) Then $\Box\varphi \Rightarrow
\varphi^\Box$.\end{lemma}
\begin{proof}We prove this by an induction on the formula structure. All cases
except for~``$\Rightarrow$'' are obvious. For this case, assume~$\Box(\psi
\Rightarrow \chi)$; we are to show that~$(\psi^\Box \Rightarrow \chi^\Box)$.
Since this is a~$\Box$-stable statement, we can in fact assume that~$(\psi
\Rightarrow \chi)$. We then have
\[ \psi^\Box \Longrightarrow \Box\psi \Longrightarrow \Box\chi
\Longrightarrow \chi^\Box, \]
with the first step being by the requirement on antecedents, the second by the
monotonicity of~$\Box$, and the third by the induction hypothesis.
\end{proof}

\begin{lemma}\label{lemma:stalk-open}
Let~$\varphi$ be a geometric formula.
Then $\varphi^\Box \Leftrightarrow \Box\varphi$.\end{lemma}
\begin{proof}The~``$\Leftarrow$'' direction is by Lemma~\ref{lemma:open-stalk}.
The~``$\Rightarrow$'' direction is an induction on the formula structure. By way of example, we verify
the case about~``$\bigvee$''. So assume~$\Box(\bigvee_i \varphi_i^\Box)$; we are
to show that~$\Box(\bigvee_i \varphi_i)$. Since this is a boxed statement, we
may in fact assume~$\bigvee_i \varphi_i^\Box$, so for some index~$j$, it holds
that~$\varphi_j^\Box$. By the induction hypothesis, it follows
that~$\Box\varphi_j$. By~$\varphi_j \Rightarrow \bigvee_i \varphi_i$ and the
monotonicity of~$\Box$, it follows that~$\Box(\bigvee_i \varphi_i)$.
\end{proof}

Note that an analogous argument for infinite conjunctions is not valid:
Assume~$(\bigwedge_i \varphi_i)^\Box$. So for all~$j$,~$\varphi_j^\Box$ holds.
By the induction hypothesis,~$\Box\varphi_j$ holds for any~$j$. But from this
we may not deduce~$\Box\bigwedge_i \varphi_i$, since the axioms on a modal
operator only require commutativity with finite conjunctions. This failure also
has a geometric interpretation, for instance in the special case~$\Box =
\neg\neg$: Given dense open subsets~$U_i$ on which formulas~$\varphi_i$ hold,
we may not conclude that there exists a single dense open subset~$U$ on which
all the formulas~$\varphi_i$ hold.

\begin{rem}In the special case that~$\Box$ is the double negation modality,
Lemma~\ref{lemma:stalk-open} holds with slightly weaker hypotheses: Namely, implications may occur
in~$\varphi$, provided that for their antecedents~$\psi$ it holds that~$\psi
\Rightarrow \psi^\Box$. This is because for the double negation modality,
the formula~$\Box(\psi \Rightarrow \chi)$ is equivalent to~$\psi \Rightarrow
\Box\chi$. (In general, for an arbitrary modality, only the former implies the latter, but not vice versa.) The case
for~``$\Rightarrow$'' in the inductive proof then goes as follows:
Assume~$(\psi \Rightarrow \chi)^\Box$. Then~$\psi \Rightarrow \psi^\Box
\Rightarrow \chi^\Box \Rightarrow \Box\chi$, so~$\Box(\psi \Rightarrow \chi)$.
\end{rem}

\begin{lemma}\label{lemma:stalk-open-with-hypothesis}
Let~$\varphi, \varphi', \psi$ be formulas. Assume that:
\begin{itemize}
\item The formula $\varphi'$ is geometric. (More generally, it suffices for~$(\varphi')^\Box$
to imply~$\Box\varphi'$.)
\item There is an intuitionistic proof that~$\varphi$
and~$\varphi'$ are equivalent under the (only) hypothesis~$\psi$.
\item Both~$\Box\psi$ and~$\psi^\Box$ hold.
\end{itemize}
Then $\varphi^\Box \Rightarrow \Box\varphi$.
\end{lemma}
\begin{proof}
Assume~$\varphi^\Box$. Since~$\psi^\Box$, $(\varphi \wedge \psi)^\Box$. Because
the~$\Box$-translation is sound with respect to intuitionistic logic
(Lemma~\ref{lemma:box-translation-sound})
it follows that~$(\varphi')^\Box$. As~$\varphi'$ is geometric, it follows
that~$\Box\varphi'$. Since~$\Box\psi$ holds, it follows that~$\Box\varphi$.
\end{proof}

\begin{ex}\label{ex:module-zero-geometric}
Let~$M$ be an~$R$-module. The statement that~$M$ is zero is not
geometric: $\varphi \defequiv (\forall x\?M\_ x = 0)$. But if~$M$ is generated by some finite
family~$x_1,\ldots,x_n\?M$, then~$\varphi$ is equivalent to the
statement~$\varphi' \defequiv (x_1 = 0
\wedge \cdots \wedge x_n = 0)$ which is geometric; and there is an
intuitionistic proof of this equivalence. Since no implication signs occur
in~$\psi \defequiv \speak{$M$ is generated by~$x_1,\ldots,x_n$}$, Lemma~\ref{lemma:stalk-open-with-hypothesis} is
applicable and shows that~$\varphi^\Box$ implies~$\Box\varphi$.
This example will gain geometric meaning in
Lemma~\ref{lemma:module-zero-point-neighbourhood}.
\end{ex}

\begin{lemma}For the modality~$\Box$ defined by~$\Box\varphi \defequiv ((\varphi
      \Rightarrow \alpha) \Rightarrow \alpha)$, where~$\alpha$ is a fixed
proposition, the~$\Box$-translation of the law of excluded middle holds.
In particular, this applies to the double negation modality~$\Box = \neg\neg$, where~$\alpha =
\bot$.\end{lemma}
\begin{proof}We are to show that~$(\varphi \vee \neg\varphi)^\Box$, \ie that
\[ ((\varphi^\Box \vee (\varphi^\Box \Rightarrow \alpha)) \Longrightarrow
    \alpha) \Longrightarrow \alpha. \]
So assume that the antecedent holds. If~$\varphi^\Box$ holds, then in
particular~$\varphi^\Box \vee (\varphi^\Box \Rightarrow \alpha)$ and thus~$\alpha$
hold. Therefore it follows that~$(\varphi^\Box \Rightarrow \alpha)$. This
implies~$\varphi^\Box \vee (\varphi^\Box \Rightarrow \alpha)$ and
thus~$\alpha$.
\end{proof}


\subsection{\texorpdfstring{Truth at stalks \vs truth on neighbourhoods}{Truth
at stalks vs. truth on neighbourhoods}}\label{sect:spreading}
We now state the crucial property of the~$\Box$-translation. Recall
that~``$X_\Box$'' denotes the sublocale of~$X$ induced by~$\Box$
(Definition~\ref{defn:subspace-by-nucleus}).
\begin{thm}\label{thm:box-translation-semantically}
Let~$X$ be a topological space. Let~$\Box$ be a modal operator
in~$\Sh(X)$. Let~$\varphi$ be a formula over~$X$. Then
\[ \Sh(X) \models \varphi^\Box \quad\text{iff}\quad
  \Sh(X_\Box) \models \varphi, \]
where on the right hand side, all parameters occuring in~$\varphi$ were pulled
back to~$X_\Box$ along the inclusion~$X_\Box \hookrightarrow X$.
\end{thm}
\XXXh{think about powersets appearing as domains of quantification}

We have not yet explicitly stated the Kripke--Joyal semantics for a sheaf topos
over a locale, which~$X_\Box$ is in general. The definition is exactly the same
as in the case for sheaf toposes over a topological space, only that any
mention of ``open sets'' has to be substituted by the more general ``opens''
and any mention of the union operator~``$\bigcup$'' has to be interpreted by
the supremum operator in the lattice of opens of the locale. For~$X_\Box$, this
is~$\sup U_i = j(\bigcup_i U_i)$. Before giving a proof of the theorem, we want
to discuss some of its consequences.

\begin{cor}\label{cor:spreading}
Let~$X$ be a topological space.
\begin{enumerate}
\item Let~$U \subseteq X$ be an open subset and let~$\Box\varphi \defequiv (U
\Rightarrow \varphi)$. Then
\[ \Sh(X) \models \varphi^\Box \quad\text{iff}\quad \Sh(U) \models \varphi. \]
\item Let~$A \subseteq X$ be a closed subset and let~$\Box\varphi \defequiv
(\varphi \vee A^c)$. Then
\[ \Sh(X) \models \varphi^\Box \quad\text{iff}\quad \Sh(A) \models \varphi. \]
\item Let~$\Box\varphi \defequiv \neg\neg\varphi$. Then
\[ \Sh(X) \models \varphi^\Box \quad\text{iff}\quad \Sh(X_{\neg\neg}) \models \varphi. \]
\item Let~$x \in X$ be a point and let~$\Box\varphi \defequiv ((\varphi
\Rightarrow \notat{x}) \Rightarrow \notat{x})$. Then
\[ \Sh(X) \models \varphi^\Box \quad\text{iff}\quad \text{$\varphi$ holds
at~$x$}. \]
% change text below ("discuss the third case") if numbering of the items changes
\end{enumerate}
\end{cor}
\begin{proof}Combine Theorem~\ref{thm:box-translation-semantically} and
Table~\ref{table:nuclei}.\end{proof}

We want to discuss the last case of the corollary in more detail. Let~$x$ be a
point of a topological space~$X$ and let~$\varphi$ be a formula. Let~$\Box$ be
the modal operator given in the corollary. Then~$\varphi$ \emph{holds at~$x$}
if and only if, from the internal perspective of~$\Sh(X)$, the translated
formula~$\varphi^\Box$ holds; and~$\varphi$ \emph{holds on some open
neighbourhood of~$x$} if and only if, from the internal perspective, the
formula~$\Box\varphi$ holds.

Thus the question whether the truth of~$\varphi$ at the point~$x$ spreads to
some open neighbourhood can be formulated in the following way:
\begin{quote}
\emph{Does~$\varphi^\Box$ imply~$\Box\varphi$ in the internal language
of~$\Sh(X)$?}
\end{quote}
Phrased this way, technicalities like appropriately shrinking open
neighbourhoods are blinded out. A purposefully trivial example to illustrate
this is the following. Let~$X$ be a scheme (or a ringed space). Let~$f,g \in
\Gamma(X,\O_X)$ be global functions. Suppose that the germs of~$f$ and~$g$ are
zero in some stalk~$\O_{X,x}$; we want to show that they are zero on a common
open neighbourhood of~$x$.

\begin{proof}[Usual proof]Since the germ of~$f$ vanishes in~$\O_{X,x}$, there
is an open neighbourhood~$U_1$ of~$x$ such that~$f|_{U_1} = 0$
in~$\Gamma(U_1,\O_X)$. Since furthermore the germ of~$g$ vanishes in the same stalk,
there exists an open neighbourhood~$U_2$ of~$x$ such that~$g|_{U_2} = 0$. The
intersection of both neighbourhoods is still an open neighbourhood of~$x$; on
this it holds that~$f$ and~$g$ both vanish.
\end{proof}

\begin{proof}[Proof in the internal language]We may suppose that~$(f = 0 \wedge
g = 0)^\Box$, that is $\Box(f=0) \wedge \Box(g=0)$, and have to prove
that~$\Box(f=0 \wedge g=0)$. (To this end, we could simply invoke the third
axiom on a modal operator, but we want to stay close to the given external
proof.) So by assumption, both~$\Box(f=0)$ and~$\Box(g=0)$ hold. Since our goal
is to prove a boxed statement, we may in fact assume that~$f = 0$ and~$g = 0$.
Thus~$f = 0 \wedge g = 0$.\end{proof}

By using the internal language with its modal operators, we can thus reduce
basic facts of scheme theory which deal with stalks and neighbourhoods to facts
of algebra in a \emph{modal intuitionistic context}. As with using the internal
language in its basic form without modalities, this brings conceptual clarity
and reduced technical overhead. There are, however, two more distinctive
advantages. Firstly, many internal proofs do not require specific properties of
the modal operator and thus work with any modal operator. By interpreting such
a proof using different operators, one obtains an entire family of external
statements without any additional work (see
Lemma~\ref{lemma:module-zero-point-neighbourhood} for an example).

Secondly, the following corollary gives a general metatheorem which is
applicable to a wide range of cases. It allows to decide whether spreading will
occur (or is likely not to occur) simply by looking at the \emph{logical form}
of the statement in question.

\begin{cor}\label{cor:geometric-spreading}
Let~$X$ be a topological space. Let~$\varphi$ be a formula.
If~$\varphi$ is geometric, truth of~$\varphi$ at a point~$x \in X$ implies
truth of~$\varphi$ on some open neighbourhood of~$x$, and vice versa.\end{cor}
\begin{proof}By the purely logical lemmas of the previous section, it holds
that~$\varphi^\Box \Leftrightarrow \Box\varphi$.
\end{proof}

\begin{cor}
Let~$X$ be a topological space. Let~$\varphi$ be a formula.
If~$\varphi$ is geometric, the property ``$\varphi$ holds at a point~$x \in
X$'' is open.
\end{cor}
\begin{proof}This is just a reformulation of the previous corollary:
If~$\varphi$ holds at a point~$x \in X$, it holds on some open
neighbourhood~$U$ of~$x$ as well. Going back to stalks, it follows
that~$\varphi$ holds at every point of~$U$.\end{proof}

\begin{ex}Let~$X$ be a scheme (or a ringed space). Since the condition for a
function~$f\?\O_X$ to be nilpotent is geometric (it is~$\bigvee_{n \geq 0} f^n
= 0$), nilpotency of~$f$ at a point is equivalent to nilpotency on some open
neighbourhood.\end{ex}

Combined with Lemma~\ref{lemma:stalk-open-with-hypothesis}, this metatheorem is
quite useful. We will illustrate it with many examples in the next subsection.

An important special case of spreading from stalks to neighbourhoods is the
case of spreading from the generic point (should it exist) to a dense open
subset. Whether this occurs can be phrased by
Lemma~\ref{lemma:negneg-generic-point} as follows:
\begin{quote}
\emph{Does~$\varphi^{\neg\neg}$ imply~$\neg\neg\varphi$ in the internal language
of~$\Sh(X)$?}
\end{quote}
This question is a question of ordinary (non-modal) intuitionistic algebra.

\begin{ex}We have seen in Remark~\ref{rem:epi-in-box-sheaves} that a
morphism~$f : A \to B$ in~$\Sh(X_\Box) \simeq \Sh_\Box(\Sh(X))$ is an
epimorphism if and only if~$\Sh(X) \models \forall y\?B\_ \Box(\exists x\?X\_
f(x) = y)$. We can now understand a simple proof of this fact:
\begin{align*}
  &\ \text{$f$ is an epimorphism in~$\Sh_\Box(\Sh(X))$} \\
  \Longleftrightarrow&\
    \Sh_\Box(\Sh(X)) \models \speak{$f$ is surjective} \\
  \Longleftrightarrow&\
    \Sh(X) \models \left(\speak{$f$ is surjective}\right)^\Box \\
  \Longleftrightarrow&\
    \Sh(X) \models \forall y\?B\_ \Box(\exists x\?X\_ \Box(f(x) = y)) \\
  \Longleftrightarrow&\
    \Sh(X) \models \forall y\?B\_ \Box(\exists x\?X\_ f(x) = y).
\end{align*}
The ultimate equivalence is by Lemma~\ref{lemma:stalk-open}, applied to the
geometric subformula~``$\exists x\?X\_ f(x) = y$''.
\end{ex}

\begin{rem}Theorem~\ref{thm:box-translation-semantically} can also be motivated
by purely logical considerations. Namely, one can check that interpreting a
formula~$\varphi$ by $\Sh(X) \models \varphi^\Box$ gives rise to a model of
intuitionistic logic -- if~$\varphi$ intuitionistically implies~$\psi$,
then~$\Sh(X) \models \varphi^\Box$ implies~$\Sh(X) \models \psi^\Box$. In
categorical logic, it is therefore a natural question whether there exists a
topos~$\E$ such that~$\E \models \varphi$ if and only if~$\Sh(X) \models
\varphi^\Box$. Theorem~\ref{thm:box-translation-semantically} gives an
affirmative answer to this question, explicitly stating that~$\E \defeq
\Sh(X_\Box)$ is such a topos.\end{rem}

\begin{proof}[Proof of Theorem~\ref{thm:box-translation-semantically}]
A fancy proof goes as follows. First, one shows intuitionistically that for a
modal operator~$\Box$ in~$\Set$, it holds that
\[ \Set \models \varphi^\Box \quad\Longleftrightarrow\quad
  \Sh_\Box(\Set) \models \varphi. \]
This can be done by an easy and nontechnical induction on the structure of
formulas~$\varphi$. Then one interprets this result in the sheaf topos~$\Sh(X)$:
\begin{align*}
  &\ \Sh(X) \models \varphi^\Box \\
  \Longleftrightarrow&\
  \Sh(X) \models \speak{$\Set \models \varphi^\Box$} &&\text{by idempotency}\\
  \Longleftrightarrow&\
  \Sh(X) \models \speak{$\Sh_\Box(\Set) \models \varphi$} &&\text{by the first step} \\
  \Longleftrightarrow&\
  \Sh_\Box(\Sh(X)) \models \varphi &&\text{by idempotency} \\
  \Longleftrightarrow&\
  \Sh(X_\Box) \models \varphi &&\text{since~$\Sh_\Box(\Sh(X)) \simeq
  \Sh(X_\Box)$}
\end{align*}
By \emph{idempotency}, we mean that internally employing the Kripke--Joyal
semantics to interpret doubly-internal statements is the same as using the
Kripke--Joyal semantics once. However, we do not want to discuss this here any further;
some details can be found in the original article on the stack
semantics~\cite[Lemma~7.20]{shulman:stack}, but the lemma given there is not
general enough to justify the second use of idempotency above. For this, one
would have to extend the stack semantics to support internal statements about
locally internal categories like~$\Sh(X_\Box) \hookrightarrow \Sh(X)$ (which
then look like locally small categories from the internal point of view). This
is worthwhile for other reasons too, but shall not be pursued here.

Therefore, we give a more explicit proof. By induction, we are going to prove
that for any open subset~$U \subseteq X$ and any formula~$\varphi$ over~$U$, it
holds that
\[ U \models_X \varphi^\Box \quad\Longleftrightarrow\quad j(U) \models_{X_\Box}
\varphi, \]
where the internal statements are to be interpreted by the Kripke--Joyal
semantics of~$X$ and~$X_\Box$ respectively and~$j$ is the nucleus associated
to~$\Box$. We may assume that any sheaves occuring in~$\varphi$ as domains of
quantifications are in fact~$\Box$-sheaves; we justify this with a separate lemma
below.

The cases~$\varphi \equiv \top$,~$\varphi \equiv (\psi \wedge \chi)$,
and~$\varphi \equiv \bigwedge_i \psi_i$ are trivial. For~$\varphi \equiv \bot$,
the claim is that~$U \models_X \Box\bot$ if and only if~$j(U)
\models_{X_\Box} \bot$. The former means~$U \subseteq j(\emptyset)$ and the
latter means~$j(U) \leq \sup \emptyset = j(\emptyset)$, so the claim follows from
the first two axioms on a nucleus.
\end{proof}
\XXX{include proof for other cases}

\begin{lemma}Let~$\Box$ be a modal operator. Let~$\varphi$ be a formula.
Let~$\psi \defequiv \varphi^\Box$ be the~$\Box$-translation of~$\varphi$.
Let~$\psi'$ be the formula obtained from~$\psi$ by substituting any occuring
domain of quantification by its~$\Box$-sheafification, as syntactically defined
in Definition~\ref{defn:plus-construction}. Then~$\psi$
and~$\psi'$ are intuitionistically equivalent.
\end{lemma}
\begin{proof}
For any formula~$\varphi$, we denote by~``$\varphi^\boxplus$'' the result of
first applying the~$\Box$-translation to~$\varphi$ and then substituting any
set~$F$ occuring in~$\varphi$ as a domain of quantification by the plus
construction~$F^+$. Recall that for any such~$F$ there is a canonical map~$F
\to F^+,\ x \mapsto [\{x\}]$. We are going to show by induction that for any
formula~$\varphi(x_1,\ldots,x_n)$ in which elements~$x_i\?F_i$ may occur as
terms, it holds that~$\varphi^\Box(x_1,\ldots,x_n)$ is equivalent
to~$\varphi^\boxplus([\{x_1\}],\ldots,[\{x_n\}])$. This suffices to prove the
lemma.

The cases for
\[ \top \quad \bot \quad \wedge \quad \bigwedge \quad \vee \quad \bigvee \quad \implies \]
are trivial. The cases for unbounded~``$\forall$'' and~``$\exists$'' are
trivial as well. The case for~``$=$'' is slightly more interesting; let~$\varphi(x,y)
\equiv (x = y)$. Then we are to show that~$\varphi^\Box(x,y) \equiv \Box(x=y)$
(equality in some set~$F$) is equivalent to~$\varphi^\boxplus([\{x\}],[\{y\}])
\equiv \Box([\{x\}] = [\{y\}])$ (equality in~$F^+$). This follows by the
definition of the plus construction. The case for~``$\in$'' is similar.

Let~$\varphi \equiv (\exists x\?F\_ \psi(x))$, where we have dropped further
variables occuring in~$\psi$ for simplicity. Then we are to show
that~$\varphi^\Box \equiv \Box(\exists x\?F\_ \psi^\Box(x))$ is equivalent
to~$\varphi^\boxplus \equiv \Box(\exists \bar x\?F^+\_ \psi^\boxplus(\bar x))$.
The ``only if'' direction is trivial (set~$\bar x \defeq [\{x\}]$). For the ``if''
direction, we may assume that there exists~$\bar x\?F^+$ such
that~$\psi^\boxplus(\bar x)$, since we want to prove a boxed statement. By
definition of the plus construction, it holds that~$\Box(\speak{$\bar x$ is a
singleton})$. So, again since we want to prove a boxed statement, we may assume
that~$\bar x$ is actually a singleton. Therefore there exists~$x\?F$ such
that~$\bar x = [\{x\}]$ and that~$\psi^\boxplus([\{x\})$ holds. By the induction
hypothesis, it follows that~$\psi^\Box(x)$. From this the claim follows.

The case for~``$\forall$'' is similar.
\end{proof}

\begin{ex}\label{ex:translation-equivalence}Let~$X$ be a scheme. Let~$f$ be a
global function on~$X$. Let~$\varphi \defequiv \neg(\speak{$f$ \inv})$
and~$\varphi' \defequiv \speak{$f$ nilpotent}$. Then, by Proposition~\ref{prop:cond-zero}, we
have~$\Sh(X) \models (\varphi \Leftrightarrow \varphi')$. But in general, this
does not imply that~$\Sh(X) \models (\varphi^\Box \Leftrightarrow
(\varphi')^\Box)$. Consider for instance the modal operator given by~$\Box\alpha
\defequiv ((\alpha \Rightarrow \notat{x}) \Rightarrow \notat{x})$ associated to a
point~$x \in X$. Then~$\Sh(X) \models (\varphi^\Box \Leftrightarrow
(\varphi')^\Box)$ means that the equivalence~$\varphi \Leftrightarrow \varphi'$
holds at the point~$x$. This is false for~$X = \Spec \ZZ$,~$f = 2$, and~$x =
(2)$, since in the local ring~$\O_{X,x} = \ZZ_{(2)}$, the element $f$ is not invertible
while also not being nilpotent.
\end{ex}
% Adapt rem:closed-geometric-morphism in case the example changes.


\subsection{Internal proofs of common lemmas}

\begin{lemma}\label{lemma:module-zero-point-neighbourhood}
Let~$X$ be a scheme (or a ringed space). Let~$\F$ be an~$\O_X$-module
of finite type.
\begin{itemize}
\item Let~$x \in X$ be a point. Then the stalk~$\F_x$ is zero if and
only if~$\F$ is zero on some open neighbourhood of~$x$.
\item Let~$A \subseteq X$ be a closed subset. Then the restriction~$\F|_A$ (\ie
the pullback of~$\F$ to~$A$) is zero if and only if~$\F$ is zero on some open
subset of~$X$ containing~$A$.
\end{itemize}
\end{lemma}
\begin{proof}\emph{Both} statements are simply internalizations of
Example~\ref{ex:module-zero-geometric}, using the modal operators~$\Box =
(\placeholder \vee A^c)$ and~$\Box = ((\placeholder \Rightarrow
\notat{x}) \Rightarrow \notat{x})$.
\end{proof}

\begin{rem}Note that the proposition fails if one drops the hypothesis
that~$\F$ is of finite type. Indeed, in this case one cannot reformulate the
condition that~$\F$ is zero in a geometric way.\end{rem}

In a remark after the proof of Proposition~\ref{prop:rank-function-internally},
we promised to present a simpler proof of it once we would have developed the theory for
doing so. We can now follow up on this promise.
\begin{lemma}\label{lemma:gen-family-n}
Let~$X$ be a scheme (or a ringed space). Let~$\F$ be an~$\O_X$-module
of finite type. Let~$x \in X$ be a point. Let~$n$ be a natural number. Then the
following statements are equivalent:
\begin{enumerate}
\item There exists a generating family for~$\F_x$ consisting of~$n$ elements.
\item There exists an open neighbourhood~$U$ of~$x$ such that
\[ U \models \speak{there exists a generating family for~$\F$ consisting of~$n$
elements}. \]
\end{enumerate}
\end{lemma}
\begin{proof}Using the modal operator~$\Box$ defined by~$\Box\varphi \defequiv
((\varphi \Rightarrow \notat{x}) \Rightarrow \notat{x})$, we have to show that
the following statements in the internal language are equivalent:
\begin{enumerate}
\item $\speak{there exists a generating family
for~$\F$ consisting of~$n$ elements}^\Box$.
\item $\Box(\speak{there exists a generating family
for~$\F$ consisting of~$n$ elements})$.
\end{enumerate}
By Lemma~\ref{lemma:open-stalk}, the second statement implies the first -- note
that in a formal spelling of the statement in quotes,
\begin{equation}\label{eqn:finitely-generated}
  \exists x_1,\ldots,x_n\?\F\_
  \forall x\?\F\_
  \exists a_1,\ldots,a_n\?\O_X\_
  x = \textstyle\sum_i a_i x_i,
\end{equation}
no implication signs occur. To show the converse direction,
we may assume that there is a generating family~$y_1,\ldots,y_m\?\F$ for~$\F$
(since~$\F$ is, externally speaking, of finite type). Then
the~$\Box$-translation of the statement that the~$y_i$ generate~$\F$ holds as
well (again by Lemma~\ref{lemma:open-stalk}). Since there is an intuitionistic
proof of
\begin{multline*}
  \speak{$y_1,\ldots,y_m$ generate~$\F$} \Longrightarrow \\
  \bigl(\speak{there exist $x_1,\ldots,x_n\?\F$ which generate~$\F$}
    \Longleftrightarrow \\
    \exists x_1,\ldots,x_n\?\F\_
    \exists A\?\O^{m \times n}\_ \speak{$\vec y = A \vec x$}\bigr),
\end{multline*}
we can substitute the non-geometric formula~\eqref{eqn:finitely-generated} by the geometric
formula
\[ \exists x_1,\ldots,x_n\?\F\_ \exists A\?\O^{m \times n}\_ \speak{$\vec
y = A \vec x$} \]
(Lemma~\ref{lemma:stalk-open-with-hypothesis}). Thus the claim follows.
\end{proof}

\begin{lemma}Let~$X$ be a scheme (or a ringed space). Let~$\alpha : \F \to \G$ be
a morphism of~$\O_X$-modules. Let~$\G$ be of finite type and assume
that~$\alpha_x : \F_x \to \G_x$ is surjective for some point~$x \in X$.
Then~$\alpha$ is an epimorphism on some open neighbourhood of~$x$.\end{lemma}
\begin{proof}In the presence of generators~$y_1,\ldots,y_n\?\G$, the
non-geometric surjectivity condition ($\forall y\?\G\_ \exists x\?\F\_
\alpha(x) = y$) can be reformulated in a geometric way: $\bigwedge_{i=1}^n
\exists x\?\F\_ \alpha(x) = y_i$. Thus the claim follows by
Lemma~\ref{lemma:stalk-open-with-hypothesis}.\end{proof}

%\begin{lemma}Let~$X$ be a scheme (or a ringed space). Let~$\alpha : \F \to \G$ be
%a morphism of~$\O_X$-modules. Let~$\F$ be of finite type and~$\G$ be coherent.
%Suppose that~$\alpha_x$ is injective at some point~$x \in X$. Then~$\alpha$ is
%a monomorphism on some open neighbourhood of~$x$.
%\end{lemma}
%\begin{proof}The kernel of~$\alpha$ is of finite type (by
%Lemma~\ref{lemma:coherent-stuff}) and zero at~$x$. By the previous lemma, it is
%therefore zero on some open neighbourhood of~$x$.
%\end{proof}
%This proof is precisely the classical one, so there is no need to include it here.

\begin{lemma}\label{lemma:pushforward-finite-type}
Let~$i : A \hookrightarrow X$ be a closed immersion of schemes (or
ringed spaces). Let~$\F$ be an~$\O_A$-module. Then~$i_*\F$ is of finite type if
and only if~$\F$ is of finite type.\end{lemma}
\begin{proof}
Let~$\Box$ be the modal operator defined by~$\Box\varphi \defequiv (\varphi \vee
A^c)$. From the internal perspective, we have a surjective ring homomorphism~$i^\sharp
: \O_X \to \O_A$, where we omit the forgetful functor~$i_*$ from~$\Box$-sheaves
to arbitrary sets in the notation, and an~$\O_A$-module~$\F$. Furthermore, we
may assume that~$\F$ is a~$\Box$-sheaf. We can regard~$\F$ as an~$\O_X$-module
by~$i^\sharp$.

Note that~$A^c \Rightarrow (\F = 0)$, by~$\Box$-separatedness of~$\F$.

We are to show that~$\F$ is a finitely generated~$\O_X$-module if and only if
the~$\Box$-translation of ``$\F$ is a finitely generated~$\O_A$-module'' holds.
In explicit terms, we have to show the equivalence of the following statements:
\begin{enumerate}
\item $\bigvee_{n \geq 0} \exists x_1,\ldots,x_n\?\F\_
  \forall x\?\F\_ \exists a_1,\ldots,a_n\?\O_X\_ x = \sum_i i^\sharp(a_i) x_i$.
\item $\Box(\bigvee_{n \geq 0} \Box(\exists x_1,\ldots,x_n\?\F\_
  \forall x\?\F\_ \Box(\exists b_1,\ldots,b_n\?\O_A\_ \Box(
    x = \sum_i b_i x_i))))$.
\end{enumerate}
It is clear that the first statement implies the second. For the converse
direction, we just have to repeatedly use the observation that~$\Box\varphi$
implies~$\varphi \vee (\F = 0)$ (once for each occurence of~$\Box$). So in each
step, we either obtain the statement we want or may assume
that~$\F$ is the trivial module, in which case any subclaim trivially follows. By
surjectivity of~$i^\sharp$, we may write any~$b\?\O_A$ as~$b =
i^\sharp(a)$ for some~$a\?\O_X$.
\end{proof}

\begin{lemma}\label{lemma:fp-hom-geometric}
Let~$X$ be a scheme (or a ringed space). Let~$\F$ and~$\G$ be~$\O_X$-modules. Let~$x
\in X$. Then $\HOM_{\O_X}(\F,\G)_x \cong \Hom_{\O_{X,x}}(\F_x,\G_x)$ if~$\F$ is
of finite presentation around~$x$.\end{lemma}
\begin{proof}It suffices to give an intuitionistic proof of the following fact:
The construction~$\Hom_R(M,\placeholder)$ is geometric if~$M$ is a finitely
presented~$R$-module. So assume that~$M$ is the cokernel of a presentation
matrix~$(a_{ij}) \? R^{n \times m}$. Then we can calculate the Hom with
any~$R$-module~$N$ as
\[ \Hom_R(M,N) \cong \Bigl\{ x \? N^n \ \Big|\ \bigwedge_{j=1}^m \sum_{i=1}^n a_{ij}
x_i = 0 \? N \Bigr\}, \]
and this construction is patently geometric, as a set comprehension with respect to
a geometric formula.
\end{proof}

\begin{lemma}Let~$X$ be a scheme (or a ringed space). Let~$\F$ be an~$\O_X$-module of finite
presentation. Let~$x \in X$. Then the stalk~$\F_x$ is a finite
free~$\O_{X,x}$-module if and only if~$\F$ is finite locally free on some open
neighbourhood of~$x$.\end{lemma}
\begin{proof}The internal statement that~$\F$ is a finite free module is not geometric:
\[ \bigvee_{n \geq 0}
  \exists x_1,\ldots,x_n\?\F\_
  \forall x\?\F\_
  \exists! a_1,\ldots,a_n\?\O_X\_
  x = \textstyle\sum_i a_i x_i. \]
But it can equivalently be reformulated as
\[ \bigvee_{n \geq 0}
  \exists \alpha\?\HOM_{\O_X}(\F,\O_X^n)\_
  \exists \beta\?\HOM_{\O_X}(\O_X^n,\F)\_
  \alpha \circ \beta = \id \wedge \beta \circ \alpha = \id. \]
This reformulation is geometric, therefore it holds at~$x$ if and only if it
holds on some open neighbourhood of~$x$. The claim follows since, by the
previous proposition, taking stalks commutes with
calculating~$\HOM_{\O_X}(\F,\placeholder)$ \resp~$\HOM_{\O_X}(\O_X^n,\placeholder)$;
thus the pulled back formula indeed expresses that~$\F_x$ is finite free as
an~$\O_{X,x}$-module.
\end{proof}

\begin{lemma}\label{lemma:torsion-module-generic-stalk}
Let~$X$ be an integral scheme with generic point~$\xi$. Let~$\F$
be a quasicoherent~$\O_X$-module. Then~$\F$ is a torsion module if and only if
its generic stalk~$\F_\xi$ vanishes.
\end{lemma}
\begin{proof}The generic stalk vanishes if and only if the internal
statement~``$(\F = 0)^{\neg\neg}$'' holds. Therefore it suffices to give an
intuitionistic proof of the following internal statement: The module~$\F$ is
torsion if and only if any element of~$\F$ is \notnot zero.

For the ``only if'' direction, let~$x\?\F$ be an arbitrary element. Since~$\F$
is a torsion module, there exists a regular element~$a\?\O_X$ such that~$ax =
0$. Since~$X$ is reduced, regularity is equivalent to not-not-invertibility.
Since we want to verify the~$\neg\neg$-stable statement~``$\neg\neg(x = 0)$'', we
may in fact assume that~$a$ is invertible. Then~$x = 0$ obviously follows.

For the ``if'' direction, let~$x\?\F$ be an arbitrary element; by assumption,~$x$
is \notnot zero. Since~$X$ is integral,
Lemma~\ref{lemma:dense-standard-reflection} is applicable. Therefore there
exists an element~$a\?\O_X$ such that~$a$ is \notnot invertible and such that
invertibility of~$a$ implies~$x = 0$. Since~$\F$ is quasicoherent, for some
natural number~$n$ it holds that~$a^n x = 0$ (Theorem~\ref{thm:qcoh-sheafchar}
below). Since~$a$ is \notnot invertible,
it is regular (see Lemma~\ref{lemma:regular-notnot-invertible} below for a short
and self-contained proof), and therefore~$a^n$ is regular. So~$x \in \F_\tors$.
\end{proof}

By simply using a different modal operator than~``\notnot'', we will -- without
any additional work -- obtain a more general form of this lemma, applicable to
non-integral schemes (see Lemma~\ref{lemma:torsion-module-generic-stalk-generalized}).

\begin{itemize}
\item general explanation of modalities (as for instance in philosophy)
\item explain that for some modal operators, the~$\Box$-translation of the law
of excluded middle is valid; explain consequences
\item spreading of properties from stalk to neighbourhood: give many examples
\item give proof of the expressions for the nuclei listed in the table
\end{itemize}


\section{Rational functions and Cartier divisors}
\label{sect:rational-functions}

\subsection{The sheaf of rational functions} Recall that the sheaf~$\K_X$ of rational
functions on a scheme~$X$ (or a ringed space) can be defined as the sheaf
associated to the presheaf
\[ \text{$U \subseteq X$ open} \quad\longmapsto\quad \Gamma(U,\O_X)[\Gamma(U,\S)^{-1}], \]
where~$\Gamma(U,\S)$ is the multiplicative set of those sections of~$\O_X$ on~$U$,
which are regular in each stalk~$\O_{X,x}$, $x \in U$. Recall also that there are
some wrong definitions in the literature~\cite{kleiman:misconceptions}.

Using the internal language, we can give a simpler definition of~$\K_X$.
Recall that we can associate to any ring~$R$ its total quotient ring, \ie
its localization at the multiplicative subset of regular elements. Since from
the internal perspective~$\O_X$ is an ordinary ring, we can associate to it its
total quotient ring $\O_X[\S^{-1}]$,
where~$\S$ is internally defined by the formula
\[ \S \defeq \{ s\?\O_X \,|\, \speak{$s$ is regular} \} \subseteq \O_X. \]
Externally, this ring is the sheaf~$\K_X$.
\begin{prop}\label{prop:kx-internally}
Let~$X$ be a scheme (or a ringed space). The sheaf of rings defined
in the internal language by localizing~$\O_X$ at its set of regular elements is
(canonically isomorphic to) the sheaf~$\K_X$ of rational functions.
\end{prop}
\begin{proof}Internally, the ring~$\O_X[\S^{-1}]$ has the following
universal property: For any ring~$R$ and any homomorphism~$\O_X \to R$ which
maps the elements of~$\S$ to units, there exists exactly one
homomorphism~$\O_X[\S^{-1}] \to R$ which renders the evident diagram commutative.
\[ \xymatrix{
  \O_X \ar[rr] \ar[dr] && R \\
  & \O_X[\S^{-1}] \ar@{-->}[ru]
} \]
The translation using the Kripke--Joyal semantics gives the following universal
property: For any open subset~$U \subseteq X$, any sheaf of rings~$\R$ on~$U$ and any
homomorphism~$\O_X|_U \to \R$ which maps all elements of~$\Gamma(V,\S)$, $V
\subseteq U$ to units, there exists exactly one homomorphism~$\O_X[\S^{-1}]|_U \to
\R$ which renders the evident diagram commutative.
It is well-known that the sheaf~$\K_X$ as usually defined has
this universal property as well.
\end{proof}

\begin{prop}\label{prop:stalks-kx}
Let~$X$ be a scheme (or a ringed space). Then the stalks of~$\K_X$
are given by
\[ \K_{X,x} = \O_{X,x}[\S_x^{-1}]. \]
The elements of~$\S_x$ are exactly the germs of those local sections which are
regular not only in~$\O_{X,x}$, but in all rings~$\O_{X,y}$ where~$y$
ranges over some open neighbourhood of~$x$ (depending on the section).\end{prop}
\begin{proof}
Since localization is a geometric construction, the first statement is made entirely
trivial by our framework. The second statement follows since
\[ \Gamma(U,\S) = \{ s\in\Gamma(U,\O_X) \,|\, U \models \speak{$s$ is regular}
\} \]
and regularity is a geometric implication, so that
$U \models \speak{$s$ is regular}$ if and only if the germ~$s_y$ is regular
in~$\O_{X,y}$ for all~$y \in U$.
\end{proof}

\begin{rem}Speaking internally, the multiplicative set~$\S$ is saturated.
Therefore an element~$s/t \? \K_X$ is invertible in~$\K_X$ if and only if the
numerator~$s$ belongs to~$\S$, \ie is an regular element of~$\O_X$.\end{rem}

% FUTURE:
% How can we prove that K_X(U) = K_X^pre(U) = Quot O_X(U) for U affine
% (and X locally Noetherian or X reduced and locally only finitely many
% irreducible components)? (cf. Kleiman!)
%
% Here is approximately how. Verify that (U |-> K_X^pre(U)) defines a sheaf, if
% U ranges only of the open affines. To do this, use the description of the
% subsheaf \T given in the section on "O_X = A[F^(-1)]".


\subsection{Regularity of local functions}
It is well-known that on a locally Noetherian scheme, regularity spreads from
stalks to neighbourhoods, \ie a section of~$\O_X$ is regular
in~$\O_{X,x}$ if and only if it is regular on some open neighbourhood of~$x$.
This fact has a simple proof in the internal language.
\begin{prop}\label{prop:regularity-spreading}
Let~$X$ be a locally Noetherian scheme. Let~$s \in \Gamma(U,\O_X)$
be a local function on~$X$. Let~$x \in U$. Then the following statements are
equivalent:
\begin{enumerate}
\item The section~$s$ is regular in~$\O_{X,x}$.
\item The section~$s$ is regular in all local rings~$\O_{X,y}$ where~$y$ ranges
over some open neighbourhood of~$x$.
\end{enumerate}
\end{prop}
\begin{proof}
Let~$\Box$ be the modal operator defined by~$\Box\varphi \defequiv ((\varphi
\Rightarrow {!x}) \Rightarrow {!x})$. By Corollary~\ref{cor:spreading}, we are
to show that the following statements of the internal language are equivalent:
\begin{enumerate}
\item $(\speak{$s$ is regular})^\Box$, \ie
$\forall t\?\O_X\_ st = 0 \Rightarrow \Box(t = 0)$.
\item $\Box(\speak{$s$ is regular})$, \ie
$\Box(\forall t\?\O_X\_ st = 0 \Rightarrow t = 0)$.
\end{enumerate}
It is clear that the second statement implies the first -- in fact, this is true
without any assumptions on~$X$: Let~$t\?\O_X$ be such that~$st = 0$. Since we want to
prove the boxed statement~$\Box(t=0)$, we may assume that~$s$ is regular and
prove~$t = 0$. This is immediate. (This direction also follows simply by
examining the logical form and applying Lemma~\ref{lemma:open-stalk}.)

For the converse direction, consider the annihilator of~$s$, \ie the ideal
\[ I \defeq \Ann_{\O_X}(s) = \{ t\?\O_X \,|\, st = 0 \} \subseteq \O_X. \]
This ideal satisfies the quasicoherence condition (Example~\ref{ex:annihilator-qcoh}),
thus~$I$ is a quasicoherent submodule of a finitely generated module. Since~$X$ is
locally Noetherian, it follows that~$I$ is finitely generated as well, say by~$x_1,\ldots,x_n \? I$. By
assumption, each generator~$x_i \? I$ fulfills~$\Box(x_i = 0)$. Since we want
to prove a boxed statement, we may in fact assume~$x_i = 0$. Thus~$I = (0)$ and
the assertion that~$s$ is regular follows.
\end{proof}

Note that the proof critically depends on the ideal~$I$ being finitely
generated, since a modal operator need only commute with finite
conjuctions. Intuitively, each time we use the modus ponens rule~$\Box\varphi \wedge
(\varphi \Rightarrow \psi) \Rightarrow \Box\psi$, we restrict to a smaller open
neighbourhood of~$x$. Since infinite intersections of open sets need not be
open, we cannot expect an infinitary modus ponens rule to hold.

\begin{cor}Let~$X$ be a locally Noetherian scheme. Then the stalks~$\K_{X,x}$
of the sheaf of rational functions are given by the total quotient rings of the
local rings~$\O_{X,x}$.\end{cor}
\begin{proof}Combine Proposition~\ref{prop:stalks-kx} and
Proposition~\ref{prop:regularity-spreading}.\end{proof}


\subsection{Normality}\label{sect:normality}
Recall that a ring~$R$ is \emph{normal} if and only if
it is integrally closed in its total quotient ring. Recall also that a
scheme~$X$ (or a ringed space) is \emph{normal} if and only if all
rings~$\O_{X,x}$ are normal.

\begin{prop}\label{prop:normal-int-ext}A locally Noetherian scheme is normal if and only if the
ring~$\O_X$ is normal from the internal perspective.\end{prop}
\begin{proof}The condition of normality can be put into a form which is almost
a geometric implication:
\begin{multline*}
  \forall s,t\?\O_X\_
  \Bigl(\speak{$t$ regular} \wedge {} \\
  (\exists a_0,\ldots,a_{n-1}\?\O_X\_
  s^n + a_{n-1} t s^{n-1} + \cdots + a_1 t^{n-1} s + a_0 t^n = 0)
  \Longrightarrow \\
  \exists u\?\O_X\_ s = ut\Bigr).
\end{multline*}
The only non-geometric subpart is the condition on~$t$ to be regular. However,
by Proposition~\ref{prop:regularity-spreading}, for the purposes of comparing
its truth at points \vs on neighbourhoods, it behaves just like a geometric
formula. Therefore the claim follows.
\end{proof}


\subsection{Geometric interpretation of rational functions} Recall that on
integral schemes, rational functions (\ie sections of~$\K_X$) are the same
thing as regular functions defined on dense open subsets. This amounts to
saying that~\emph{$\K_X$ is the~$\neg\neg$-sheafification of~$\O_X$}
(see Proposition~\ref{prop:negneg-sheaves}). We want to rederive this result,
as far as possible in the internal language, and generalize it to arbitrary
(not necessarily locally Noetherian) schemes.

\begin{lemma}\label{lemma:regular-notnot-invertible}Let~$X$ be a reduced scheme. Then:
\begin{enumerate}
\item $\O_X$ is~$\neg\neg$-separated.
\item Internally, an element~$s\?\O_X$ is regular
if and only if it is \notnot invertible.
\end{enumerate}
\end{lemma}
\begin{proof}Recall from Corollary~\ref{cor:field-reduced} that
\begin{equation}\label{eqn:field-condition}
  \Sh(X) \models \forall s\?\O_X\_ \neg(\speak{$s$ invertible}) \Leftrightarrow
  s=0.
\end{equation}
From this we can deduce that~$\O_X$ is~$\neg\neg$-separated:
Assume~$\neg\neg(s=0)$ for~$s\?\O_X$. If~$s$ were invertible, we would
have~$\neg\neg(1=0)$ and thus~$\bot$. Therefore~$s$ is not invertible and thus
zero.

For the ``only if'' direction of the second statement,
note that a regular element is not zero (if it were, then the true statement~$0
\cdot 0 = 0 \cdot 1$ would imply the false statement~$0 = 1$) and thus \notnot
invertible (by the contrapositive of equivalence~\eqref{eqn:field-condition}). For the ``if''
direction, let~$st = 0$ in~$\O_X$. Since~$s$ is \notnot invertible, it follows
that~$t$ is \notnot zero. Since~$\O_X$ is~$\neg\neg$-separated, this implies
that~$t$ really is zero.
\end{proof}

For the following, we need two technical conditions. Say that an affine
scheme~$\Spec A$ has property~$(\star)$ if and only if:
\begin{quote}
Every open dense subset~$U \subseteq \Spec A$ contains a
\emph{standard open} dense subset.
\end{quote}
Say that~$\Spec A$ has property~$(\star\star)$ if and only if:
\begin{quote}
Every open scheme-theoretically dense subset~$U \subseteq \Spec A$ contains a
\emph{standard open} scheme-theoretically dense subset.
\end{quote}
The first condition is satisfied if~$A$ is an irreducible ring (\ie if~$\Spec A$
is irreducible) or more generally if~$A$ contains only finitely many minimal
prime ideals. Both conditions are satisfied if~$A$ is integral or if~$A$ is
Noetherian; for convenience, we give a proof in the
latter case.

\begin{prop}Let~$A$ be a Noetherian ring. Then~$\Spec A$ has properties~$(\star)$
and~$(\star\star)$.
\end{prop}
\begin{proof}Recall that, under the Noetherian hypothesis, an open subset of~$\Spec A$ is dense if and only if it
contains all minimal prime ideals and that it
is scheme-theoretically dense if and only if it contains all associated prime
ideals. There are only a finite number of these prime ideals. Therefore the
claim is reduced to the following statement:

Let~$\ppp_1,\ldots,\ppp_n$ be a
finite number of points of an open subset~$U \subseteq \Spec A$. Then there
exists a standard open subset~$D(f) \subseteq U$ which also contains these
points.

The proof of this statement is an easy application of the prime
avoidance lemma.
\end{proof}

\begin{prop}
\label{prop:kx-is-negneg-sheafification}
Let~$X$ be a reduced scheme. Assume that~$X$ can be covered by open affine
subsets which have property~$(\star)$. For instance, this condition is satisfied
if~$X$ is integral, the set of irreducible components is locally finite, or if~$X$ is locally Noetherian. Then~$\K_X$ is
the~$\neg\neg$-sheafification of~$\O_X$.\end{prop}
\begin{proof}
We first show that~$\K_X$ is~$\neg\neg$-separated,
so assume~$\neg\neg(a/s = 0)$ for~$a/s \? \K_X$. Since~$\K_X$ is obtained
from~$\O_X$ by localizing at regular elements, the fraction~$a/s$ vanishes
in~$\K_X$ if and only if~$a = 0$ in~$\O_X$. Thus it follows that~$\neg\neg(a =
0)$ in~$\O_X$ and therefore~$a = 0$ in~$\O_X$; in particular, $a/s = 0$ in~$\K_X$.

We defer the proof that~$\K_X$ is a~$\neg\neg$-sheaf to the end and first
verify the universal property of~$\neg\neg$-sheafification.
So let~$G$ be a~$\neg\neg$-sheaf and let~$\alpha : \O_X \to G$ be a map. We
can define an extension~$\bar\alpha : \K_X \to G$ in the following way:
Let~$f \? \K_X$. Define the subsingleton~$S \defeq \{ x \? G \,|\, \exists
b\?\O_X\_ f = b/1 \wedge x = \alpha(b) \} \subseteq G$. Since~$f$ can be
written in the form~$a/s$ with~$s$ \notnot invertible, it follows that~$S$
is \notnot inhabited. Since~$G$ is a~$\neg\neg$-sheaf, there exists a
unique~$x\?G$ such that~$\neg\neg(x \in S)$. We declare~$\bar\alpha(f)$ to be
this~$x$. It is straightforward to check that the composition~$\O_X \to \K_X
\to G$ equals~$\alpha$ and that~$\bar\alpha$ is unique with this property.

Up to this point, the proof did not need that~$X$ is a scheme -- it was enough
for~$X$ to be a ringed space such that equivalence~\eqref{eqn:field-condition} holds and
such that~$\neg(0 = 1)$ in~$\O_X$. Only now, in showing that~$\K_X$ is
a~$\neg\neg$-sheaf, the scheme condition enters. To this end, we first
reformulate the sheaf condition in a way such that it only refers to~$\O_X$,
not~$\K_X$: The quotient ring~$\K_X$ is a~$\neg\neg$-sheaf if and only if
\begin{multline*}
  \Sh(X) \models \forall T \subseteq \O_X\_
  \speak{$T$ subsingleton} \wedge \neg\neg(\speak{$T$ inhabited})
  \Longrightarrow \\
  \exists a,b\?\O_X\_ \speak{$b$ regular} \wedge \neg\neg(b^{-1} a \in T).
\end{multline*}
This is done just as in the proof of Theorem~\ref{thm:qcoh-sheafchar}.
\XXX{reorder qcoh before this because of the reference?}
Note
that~``$b^{-1}$'' refers to the inverse of~$b$ which indeed exists in a doubly
negated context, since~$b$ is assumed regular. More explicitly, we should write
\[ \neg\neg(\exists c\?\O_X\_ bc = 1 \wedge ca \in T)
  \quad\text{instead of}\quad
  \neg\neg(b^{-1} a \in T). \]
To verify the Kripke--Joyal interpretation of the rewritten sheaf condition, let
an affine open subset~$U = \Spec A \subseteq X$ (having property~$(\star)$) and a subsheaf~$T
\hookrightarrow \O_X|_U$ be given such that~$T$ is internally a subsingleton
and \notnot inhabited. We may glue the unique germs in the inhabited
stalks of~$T$ to obtain a section~$s \in \Gamma(V,\O_X)$ where~$V \subseteq U$
is a dense open subset. Since~$U$ has property~$(\star)$, we may assume that~$V
= D(f)$ is a standard open subset. Because~$V$ is dense and~$A$ is reduced, the
function~$f$ is a regular element of~$A$.
Since~$\Gamma(V,\O_X) = A[f^{-1}]$, we can write~$s = a/f^n$ with~$a \in A$
and~$n \geq 0$.

By Lemma~\ref{lemma:regular-affine}, the function~$b \defeq f^n$ is also regular as an
element of~$\O_U$ from the internal point of view. Note that~$b$ is invertible
on~$V$, since~$V = D(f) \subseteq D(b)$. It follows that on the dense
open subset~$V \subseteq U$, the sections~$s$ and~$b^{-1} a$ agree.
This observation concludes the proof.
\end{proof}

\begin{cor}Let~$X$ be a reduced scheme admitting a cover by affine open subschemes
with property~$(\star)$. Then~$\K_X$ is the result of
pulling back~$\O_X$ to the sublocale~$X_{\neg\neg}$ and then pushing forward
again. If~$X$ is irreducible with generic point~$\xi$, then~$\K_X$ is the
constant sheaf associated to the set~$\O_{X,\xi}$.\end{cor}
\begin{proof}Recall from Section~\ref{sect:internal-sheaves} that pulling back
to~$X_{\neg\neg}$ is equivalent to sheafifying with respect to the double
negation modality; and that pushing forward is equivalent to forgetting the
sheaf property. Therefore the first statement holds.

For the second statement, recall from Lemma~\ref{lemma:negneg-generic-point} that the
sublocale~$X_{\neg\neg}$ is given by the subspace~$\{\xi\}$; that the
sheafification functor~$\Sh(X) \to \Sh(\{\xi\}) \simeq \Set$ is given by
calculating the stalk at~$\xi$; and that the inclusion functor~$\Set \simeq
\Sh(\{\xi\}) \hookrightarrow \Sh(X)$ is given by the constant sheaf
construction.
\end{proof}

If~$X$ is a general scheme (not necessarily reduced),
we can describe~$\K_X$ in a similar way as a sheafification
of~$\O_X$; specifically, it is the sheafification with respect to the modal
operator defined by
\[ \sdense\varphi \defequiv \speak{$\O_X$ is~$(\varphi \Rightarrow
\placeholder)$-separated} \]
in the internal language of~$\Sh(X)$, \ie
\[ \sdense\varphi \defequiv \forall s\?\O_X\_ (\varphi \Rightarrow s = 0)
\Rightarrow s = 0. \]
This modal operator has an explicit scheme-theoretic description.

\begin{lemma}\label{lemma:scheme-theoretical-density}
Let~$U$ be an open subset of a scheme~$X$. Then~$\Sh(X) \models
\sdense U$ if and only if~$U$ is scheme-theoretically dense in~$X$.
\end{lemma}
\begin{proof}We have the following chain of equivalences.
\begin{align*}
  &\ X \models \sdense U \\
  \Longleftrightarrow&\
    \speak{$\O_X$ is~$(U \Rightarrow \placeholder)$-separated} \\
  \Longleftrightarrow&\
    X \models \speak{$\O_X \to \O_X^{+}$ is injective} \\
  &\qquad\qquad\text{(where the plus construction is wrt.\@ the modality~$(U \Rightarrow \placeholder)$)} \\
  \Longleftrightarrow&\
    X \models \speak{$\O_X \to \O_X^{++}$ is injective} \\
  &\qquad\qquad\text{(by the factorization~$\O_X \to \O_X^{+} \to \O_X^{++}$)} \\
  \Longleftrightarrow&\
    \text{the canonical morphism~$\O_X \to j_* \O_U$
    (with $j : U \hookrightarrow X$) is injective} \\
  \Longleftrightarrow&\
    \text{$U$ is scheme-theoretically dense in~$X$.} \qedhere
\end{align*}
\end{proof}

Using the internal language of a scheme, talking about scheme-theoretically
dense open subsets is therefore just as easy as talking about ordinary
topologically dense open subsets; the difference simply amounts to using the
modal operator~``$\sdense$'' instead of~``\notnot''.

\begin{prop}\label{prop:kx-is-box-sheafification}
Let~$X$ be a ringed space. Then:
\begin{enumerate}
\item The operator~$\sdense$ fulfills the axioms on a modal operator.
\item $\O_X$ is~$\sdense$-separated.
\item $\K_X$ is~$\sdense$-separated.
\item Internally, it holds that~$\sdense(\speak{$f$ \inv})$ implies that~$f$ is
regular for any~$f\?\O_X$.
% update addtocounter below if numbering changes
\end{enumerate}
Suppose furthermore that~$X$ is a scheme. Then:
\begin{enumerate}
\addtocounter{enumi}{4}
\item The converse in~(4) holds.
\item If~$X$ can be covered by open affine subschemes with
property~$(\star\star)$, then $\K_X$ is the~$\sdense$-sheafification of~$\O_X$.
\end{enumerate}
\end{prop}
\begin{proof}The first four properties are entirely formal; we thus skip over
some details. For the first property, we verify the second axiom on a modal
operator. So we assume~$\sdense\sdense\varphi$ and have to show~$\sdense\varphi$. To
this end, let~$s\?\O_X$ be arbitrary such that~$\varphi \Rightarrow (s=0)$; we
have to prove that~$s = 0$. If~$\O_X$ were separated with respect to the modal
operator~$(\varphi \Rightarrow \placeholder)$, it would follow that~$s = 0$. So
unconditionally it holds that~$\sdense\varphi \Rightarrow (s=0)$. Since by
assumption~$\O_X$ is~$(\sdense\varphi \Rightarrow \placeholder)$-separated, the claim follows.

For the second property, let~$s\?\O_X$ be arbitrary such that~$\sdense(s = 0)$.
Obviously it holds that~$(s = 0) \Rightarrow (s = 0)$. Thus, since~$\O_X$ is
separated with respect to~$((s = 0) \Rightarrow \placeholder)$, it follows
that~$s = 0$. The proof of the third property is similar.

For the fourth property, assume~$\sdense(\speak{$f$ \inv})$ and let~$h\?\O_X$ be
arbitrary such that~$fh = 0$. Then, trivially, it holds that~$\speak{$f$ \inv}
\Rightarrow h = 0$. Since~$\O_X$ is separated with respect to~$(\speak{$f$
\inv} \Rightarrow \placeholder)$, it follows that~$h = 0$.

We may now suppose that~$X$ is a scheme. To verify the fifth property, let a
regular element~$f\?\O_X$ be given. We have to show that~$\O_X$ is separated
with respect to the modality~$(\speak{$f$ \inv} \Rightarrow \placeholder)$. So
assume that~$\speak{$f$ \inv} \Rightarrow (s = 0)$ for some~$s\?\O_X$. By
Proposition~\ref{prop:cond-zero} it follows that~$f^n s = 0$ for some natural
number~$n$. Since~$f$ is regular, we may conclude that~$s = 0$.

The verification of the universal property of~$\K_X$ is done analogously as in
the case that~$X$ is reduced: For the proof of
Proposition~\ref{prop:kx-is-negneg-sheafification}, it was critical that
regular elements of~$\O_X$ are \notnot invertible. We now need (and have) that
regular elements of~$\O_X$ are~$\sdense(\speak{invertible})$.

Thus it only remains to verify that~$\K_X$ is a~$\sdense$-sheaf. We may again imitate
the proof of Proposition~\ref{prop:kx-is-negneg-sheafification}; using the same
notation, we may now suppose that~$V$ is a standard open subset such that~$U \models \sdense
V$ (previously, we supposed that~$U \models \neg\neg V$). The proof that the
denominator~$b$ is regular (as seen from the internal perspective, as an
element of~$\O_U$) now goes as follows: We have~$V \subseteq
D(b)$. Therefore~$U \models \sdense V$ implies~$U \models \sdense(\speak{$b$ \inv})$. By
the fourth property, it follows that~$U \models \speak{$b$ is regular}$.
\end{proof}

\begin{rem}The modal operator~$\sdense$ is the largest (weakest) operator such
that~$\O_X$ is~$\sdense$-separated, \ie if~$\sdenseother$ is any modal operator
such that~$\O_X$ is~$\sdenseother$-separated, then~$\sdenseother\varphi
\Rightarrow \sdense\varphi$ for any proposition~$\varphi$.\end{rem}

In the special case that~$X$ is a reduced scheme,
Proposition~\ref{prop:kx-is-box-sheafification} recovers
the result of Proposition~\ref{prop:kx-is-negneg-sheafification}:

\begin{prop}Let~$X$ be a scheme. Then~$\sdense\varphi \Rightarrow \neg\neg\varphi$
for any formula~$\varphi$. The converse holds if~$X$ is reduced, so that in
this case the modal operator~$\sdense$ coincides with the double negation modality.\end{prop}
\begin{proof}Let~$\varphi$ be an arbitrary formula and assume~$\sdense\varphi$. Note that~$\neg\varphi$ is
equivalent to~$\varphi \Rightarrow (1 =
0)$. Since by assumption~$\O_X$ is separated with respect to the~$(\varphi
\Rightarrow \placeholder)$-modality, this in turn is equivalent to~$1 = 0 \?
\O_X$, \ie to~$\bot$. Thus~$\neg\neg\varphi$.

For the converse direction, let~$\varphi \Rightarrow (s = 0)$ for some~$s\?\O_X$;
we have to show that in fact~$s = 0$. Since by assumption~$\neg\neg\varphi$, it
follows that~$s$ is \notnot zero. Since~$X$ is reduced,~$\O_X$
is~$\neg\neg$-separated, so this implies that~$s$ is really zero.
\end{proof}

As a corollary, we can reprove the following basic lemma about
scheme-theoretical density.
\begin{lemma}Let~$U$ be an open subset of a scheme~$X$. If~$U$ is
scheme-theoretically dense, then~$U$ is also dense in the plain topological
sense. The converse holds if~$X$ is reduced.\end{lemma}
\begin{proof}The set~$U$ is scheme-theoretically dense if and only if~$\Sh(X)
\models \sdense U$ and is dense if and only if~$\Sh(X) \models \neg\neg U$.
Therefore the claim follows from the previous proposition.
\end{proof}

\begin{prop}\label{prop:kx-ass}
Let~$X$ be a scheme admitting a cover of open affine subsets with
property~$(\star\star)$. Then~$\K_X$ is the result of
pulling back~$\O_X$ to the sublocale~$X_\sdense$ associated to the modal
operator~$\sdense$ and then pushing forward again. If~$X$ is locally Noetherian,
this sublocale is the subspace of associated points in~$X$.
\end{prop}

In formulas, the proposition says that the canonical map
\[ \K_X \longrightarrow i_* i^{-1} \O_X \]
is an isomorphism, where~$i : X_\sdense \hookrightarrow X$ is the inclusion of
the sublocale~$X_\sdense$. This result requires a cover with
property~$(\star\star)$, but no Noetherian hypothesis.

\begin{proof}The first statement follows trivially by the results of
Section~\ref{sect:internal-sheaves} and the fact that~$\K_X$ is
the~$\sdense$-sheafification of~$\O_X$.

For the second statement, we need to verify that the nucleus~$j_{\Ass(\O_X)}$
associated to the subspace of associated points coincides with the
nucleus~$j_\sdense$ associated to the modal operator~$\sdense$. Recall from
Subsection~\ref{sect:subspace-to-modal-operator} that the latter is given by
\begin{align*}
  j_\sdense(U) &= \text{largest open subset of~$X$ on which~$\sdense U$ holds} \\
  &= \bigcup\ \{ V \subseteq X \ |\
  \text{$V$ open},\ V \models \sdense U \}
\intertext{and note that the former is given by}
  j_{\Ass(\O_X)}(U) &= \bigcup\ \{ V \subseteq X \ |\
  \text{$V$ open},\ V \cap \Ass(\O_X) \subseteq U \}.
\end{align*}
This is a general fact of locale theory, not depending on particular properties
of~$\Ass(\O_X)$. To verify this, one needs to check that~$j_{\Ass(\O_X)}$ is indeed a
nucleus and that the canonical map
\[ \{ U \in \Open(X) \,|\, j_{\Ass(\O_X)}(U) = U \} \longrightarrow \Open(\Ass(\O_X)),\ U \longmapsto \Ass(\O_X) \cap U \]
is an isomorphism of frames with inverse given by~$\Ass(\O_X) \cap U \mapsto
j_{\Ass(\O_X)}(U)$.

The equivalence thus follows from a standard result on the set of associated
points on locally Noetherian schemes:
\begin{align*}
  &\ V \cap \Ass(\O_X) \subseteq U \\
  \Longleftrightarrow&\
    \Ass(\O_V) \subseteq U \\
  \Longleftrightarrow&\
    \text{$U \cap V$ is scheme-theoretically dense in~$U$} \\
  &\qquad\qquad\text{(this step requires the Noetherian assumption)} \\
  \Longleftrightarrow&\
    V \models \sdense U. \qedhere
\end{align*}
\end{proof}

\begin{lemma}
Let~$X$ be a scheme admitting a cover of open affine subsets with
property~$(\star\star)$. Let~$j : U \hookrightarrow X$ be the inclusion of an
open subset containing the sublocale~$X_\sdense$. (If~$X$ is locally
Noetherian, this is equivalent to requiring that~$U$ contains~$\Ass(\O_X)$.)
Then the canonical morphism~$\K_X \to j_* \K_U$ is an isomorphism.
\end{lemma}
\begin{proof}Write~$i : X_\sdense \hookrightarrow X$ and~$i' : X_\sdense
\hookrightarrow U$ for the inclusions. By the previous proposition, the
sheaf~$\K_X$ is given by~$i_* i^{-1} \O_X$. Similarly, the sheaf~$j_* \K_U$ is
given by~$j_* i'_* i'^{-1} j^{-1} \O_X$. The claim follows since~$j \circ i' =
i$.
\end{proof}
\XXXh{Does U admit a (**)-cover if X does?}

\begin{lemma}\label{lemma:dense-standard-reflection-generalized}
Let~$X$ be a scheme admitting a cover by affine open subschemes
with property~$(\star)$ respectively~$(\star\star)$. Let~$\varphi$ be any formula
over~$X$. Then
\[ \Sh(X) \models \neg\neg\varphi \Longrightarrow \exists f\?\O_X\_
  \neg\neg(\speak{$f$ \inv}) \wedge (\speak{$f$ \inv} \Rightarrow \varphi) \]
respectively
\[ \Sh(X) \models \sdense\varphi \Longrightarrow \exists f\?\O_X\_
  \sdense(\speak{$f$ \inv}) \wedge (\speak{$f$ \inv} \Rightarrow \varphi). \]
\end{lemma}
\begin{proof}The proof of Lemma~\ref{lemma:dense-standard-reflection} carries
over, \emph{mutatis mutandis}.
\end{proof}

\begin{prop}\label{prop:boolean-dim0-continued}
Let~$X$ be a scheme of dimension~$\leq 0$ such that the set of irreducible components is locally finite or such that~$X$ is locally Noetherian. Then
the internal language of~$\Sh(X)$ is Boolean. (The converse holds as well and
was already stated as Corollary~\ref{cor:boolean-dim0}.)
\end{prop}
\begin{proof}
It suffices to verify the principle of double negation elimination, since the
law of excluded middle is equivalent to it.\footnote{This is a standard fact of
intuitionistic logic. Assume that the principle of double negation elimination
holds. We want to verify the law of excluded middle, so let an arbitrary
formula~$\varphi$ be given. Even intuitionistically it holds
that~$\neg\neg(\varphi \vee \neg\varphi)$. By double negation elimination it
follows that~$\varphi \vee \neg\varphi$.}
So let~$\varphi$ be an arbitrary formula and assume~$\neg\neg\varphi$. By the
previous lemma there exists an element~$f\?\O_X$ such that~$f$ is \notnot
invertible and such that~$(\speak{$f$ \inv} \Rightarrow \varphi)$. Since~$\dim
X \leq 0$, this element is invertible or nilpotent
(Corollary~\ref{cor:scheme-dimension-zero}).  In the first case, we are done.
In the second case, some power~$f^n$ is zero and therefore in particular
\notnot zero. Since~$f$ is \notnot invertible, this implies that \notnot~$1 =
0$. On the other hand~$1 \neq 0$, so we obtain a contradiction; from this
contradiction~$\varphi$ trivially follows.
\end{proof}

\begin{lemma}\label{lemma:torsion-module-generic-stalk-generalized}
Let~$X$ be a locally Noetherian scheme. Let~$\F$ be a
quasicoherent~$\O_X$-module. Then~$\F$ is a torsion module if and only if the
restriction of~$\F$ to~$\Ass(\O_X)$ vanishes.
\end{lemma}
\begin{proof}By Proposition~\ref{prop:kx-ass} and
Lemma~\ref{lemma:dense-standard-reflection-generalized} it suffices to repeat
the proof of Lemma~\ref{lemma:torsion-module-generic-stalk} with
``\notnot'' substituted by~``$\sdense$''.
\end{proof}


\subsection{Cartier divisors} Let~$X$ be a scheme (or a ringed space). Recall
that a \emph{Cartier divisor} on~$X$ is a global section of the sheaf of
groups~$\K_X^\times / \O_X^\times$. This sheaf can be constructed internally, with the
same notation: It is the quotient of the group of invertible elements of the
ring~$\K_X$ by the subgroup of invertible elements of the ring~$\O_X$. So an
arbitrary section of~$\K_X^\times/\O_X^\times$ is internally of the form~$[s/t]$
with~$s,t\?\O_X$ being regular elements; this is a simpler description than the
usual external one as a family~$(f_i)_i$ of functions~$f_i \in
\Gamma(U_i,\K_X^\times)$ such that~$f_i^{-1}|_{U_i \cap U_j} \cdot f_j|_{U_i \cap
U_j} \in \Gamma(U_i \cap U_j, \O_X^\times)$ for all~$i,j$.

We can sketch the basic theory of Cartier divisors completely from the internal
perspective. In accordance with common practice, we write the group
operation of~$\K_X^\times/\O_X^\times$ (which is induced by multiplication of elements
in~$\K_X^\times$) additively.

\begin{defn}\label{defn:effective-cartier-divisor}
A Cartier divisor is \emph{effective} if and only if, from the
internal perspective, it can be written in the form~$[s/1]$ with~$s\?\O_X$
being a regular element.\end{defn}

Thus a Cartier divisor~$[s/t]$ is effective if and only if~$s$ is
an~$\O_X$-multiple of~$t$.

\begin{defn}A Cartier divisor~$D$ is \emph{principal} if and only if there
exists a global section~$f \in \Gamma(X,\K_X^\times)$ such that internally,~$D = [f]$.
Two Cartier divisors are \emph{linearly equivalent} if and only if their
difference is a principal divisor.
\end{defn}

Note that decidedly, principality is a global notion: For \emph{any} divisor~$D$ it is
true that locally there exists sections~$f$ of~$\K_X^\times$ such that~$D = [f]$.

\begin{defn}\label{defn:line-bundle-of-divisor}
The \emph{line bundle associated to a Cartier divisor}~$D$
is the~$\O_X$-submodule
\[ \O_X(D) \defeq \{ g \in \K_X \,|\, g D \in \O_X \} = D^{-1} \O_X \subseteq \K_X
\]
of~$\K_X$. Here we are abusing language for~``$gD \in \O_X$'' to mean that~$gf
\in \O_X$ if~$D = [f]$ with~$f\?\K_X$; and for~``$D^{-1} \O_X$'' to
mean~$f^{-1}\O_X$. This condition respectively submodule does not depend on the
representative~$f$, since~$f$ is well-defined up to multiplication by an element
of~$\O_X^\times$.\end{defn}

The submodule~$\O_X(D)$ is indeed locally free of rank~$1$, since
internally~$f^{-1}$ gives a one-element basis. Note that~$D$ is effective if
and only if~$\O_X(-D)$ is a subset of~$\O_X$ from the internal perspective
(this comparison makes sense, since~$\O_X(-D)$ and~$\O_X$ are both canonically
embedded in~$\K_X$). In
this case, we can define the \emph{support} of~$D$ to be the closed subscheme
of~$X$ associated to the sheaf of ideals~$\O_X(-D) \subseteq \O_X$.

%\begin{prop}Let~$D$ be an effective divisor on~$X$. Then~$\O_X(D)$ is trivial
%outside of the support of~$D$.\end{prop}
%\begin{proof}We have to show that~$\O_X(D)|_U$ is trivial on~$U \defeq X \setminus
%V(\O_X(-D)) = \{ x \in X \,|\, 1 \in \O_X(-D) \}$. We do this by verifying ...
%D = [f], f : O_X regular. O_X(-D) = (f).
%Assume 1 in O_X(-D): Then f is invertible. Obviously O_X(-D) --> O_X
%is an isomorphism. So O_X --> O_X(D) is as well.
%\end{proof}

\begin{defn}The \emph{Cartier divisor associated to a free~$\O_X$-submodule~$\L \subseteq
\K_X$ of rank~1} is~$D \defeq [f^{-1}]$, where~$f\?\K_X$ is the unique element of
some one-element basis of~$\L$.\end{defn}

The basis element~$f\?\K_X$ does indeed lie in~$\K_X^\times$: Write~$f
= s/t$ with~$s,t \? \O_X$. It suffices to show that~$s$ is a regular element
of~$\O_X$. So let~$h\?\O_X$ such that~$sh = 0$ in~$\O_X$. Then in
particular~$hf = 0$ in~$\K_X$. By linear independence, it follows that~$h = 0$
in~$\K_X$ and thus~$h = 0$ in~$\O_X$.

Furthermore, the associated divisor does not depend on the choice of~$f$,
since~$f$ is well-defined up to multiplication by an element of~$\O_X^\times$: If~$f
\O_X = g \O_X \subseteq \K_X$, then there exist~$u,v\?\O_X$ such that~$fu = g$
and~$gv = f$ in~$\K_X$. It follows that~$uv = fuvf^{-1} = gvf^{-1} = ff^{-1} =
1$ in~$\K_X$ and thus in~$\O_X$, by injectivity of the canonical map~$\O_X \to
\K_X$. Therefore~$u$ and~$v$ are elements of~$\O_X^\times$.

\begin{lemma}Let~$D$ and~$D'$ be divisors on~$X$. Then~$\O_X(D) \otimes_{\O_X}
\O_X(D') \cong \O_X(D + D')$.\end{lemma}
\begin{proof}The wanted morphism of sheaves~$\O_X(D) \otimes \O_X(D') \to
\O_X(D + D')$ is given by multiplication. That this is well-defined and an
isomorphism can be checked from the internal point of view, where the claims
are obvious.\end{proof}

\begin{prop}The association~$D \mapsto \O_X(D)$ defines a one-to-one
correspondence between Cartier divisors on~$X$ and rank-one submodules
of~$\K_X$. This correpondence descends to a one-to-one correspondence between
Cartier divisiors up to linear equivalance and rank-one submodules of~$\K_X$ up
to isomorphism (as abstract~$\O_X$-modules, ignoring their embedding
into~$\K_X$).\end{prop}
\begin{proof}The first statement is obvious from the definitions. For the
second statement, it suffices to show that~$\O_X(D)$ is isomorphic to~$\O_X$ if
and only if~$D$ is principal. An isomorphism~$\O_X \to \O_X(D)$ gives a
global section~$f \in \K_X^\times$ (by considering the image of the unit element)
such that internally,~$D = [f^{-1}]$; this shows that~$D$ is principal. The
converse is similar.
\end{proof}

%\begin{rem}Locally principal subschemes (closed subschemes which are locally
%the vanishing subscheme of a regular section of~$\O_X$) up to isomorphisms of
%subschemes are in one-to-one correspondence with rank-1 submodules of~$\O_X$
%(see~\XXX{ref}). Thus locally principal subschemes (up to isomorphisms of abstract
%schemes) are in one-to-one correspondence with effective Cartier divisors (up
%to linear equivalence).\end{rem}
%\XXX{check this.}

For the following definition, recall that we can localize an~$\O_X$-module~$\L$
away from the set~$\S \subseteq \O_X$ of regular elements to obtain
a~$\K_X$-module~$\L[S^{-1}]$.

\begin{defn}Let~$f\?\L[\S^{-1}]$ be a rational section of a line bundle~$\L$
on~$X$. Assume that~``$f$ is nontrivial'', that is multiplication by~$f$ is an
injective map~$\O_X \to \L[\S^{-1}]$. Then the \emph{associated divisor} of~$f$
is~$\operatorname{div}(f) \defeq [\psi(s)/t]$, where~$f = s/t$ with~$s\?\L$ and~$t\?\O_X$
and~$\psi : \L \to \O_X$ is an isomorphism.\end{defn}

One can check that~$\psi(s)$ is a regular element of~$\O_X$; this statement is
in fact equivalent to the multiplication map~$\O_X \to \L[\S^{-1}]$ being
injective. Furthermore one can check that~$[\psi(s)/t]$ does not depend on the
choice of~$s$,~$t$, and~$\psi$.

\begin{prop}Let~$f\?\L[\S^{-1}]$ be a nontrivial rational section of a line
bundle~$\L$ on~$X$. Then multiplication by~$f$ induces an
isomorphism~$\O_X(\operatorname{div}(f)) \to \L$.\end{prop}
\begin{proof}The isomorphism should map a rational function~$g$ to~$g \cdot f$. This
is a priori an element of~$\L[\S^{-1}]$; we have to check that it can be
regarded as an element of~$\L$. Just as in the definition
of~$\operatorname{div}(f)$, write~$f = s/t$ and fix an isomorphism~$\psi : \L
\to \O_X$. Write~$g = (t/\psi(s)) \cdot h$ for some function~$h\?\O_X$. Then~$g
\cdot f = sh/\psi(s) = h\psi^{-1}(1)$, since~$s = \psi^{-1}(\psi(s)) = \psi(s)
\cdot \psi^{-1}(1)$. The element~$h\psi^{-1}(1)$ can indeed be considered as an
element of~$\L$.

Injectivity of the map~$\O_X(\operatorname{div}(f)) \to \L$ is by the
nontriviality of~$f$. For surjectivity, note that~$(t/\psi(s)) \cdot \psi(v)$ is a
preimage to~$v\?\L$, since~$(t/\psi(s)) \cdot \psi(v) \cdot f = \psi(v) \psi(s)
\psi^{-1}(1) / \psi(s) = v$.
\end{proof}

\begin{prop}Let~$\L$ be a line bundle on~$X$. Assume that~$\L$ can be embedded
into~$\K_X$. Then~$\L$ possesses a nontrivial rational section.
\end{prop}
\begin{proof}Let~$i : \L \to \K_X$ denote the given injection. Let~$(v)$ be an
one-element basis for~$\L$. Write~$i(v) = s/t$. Then~$s$ is regular,
since~$hs = 0$ implies~$i(hv) = 0$ and thus~$h = 0$, for any~$h\?\O_X$.
Therefore~$f \defeq tv/s$ is a well-defined element of~$\L[\S^{-1}]$.
Furthermore it is nontrivial in the desired sense: If~$h \cdot (tv/s) = 0$,
then~$htv = 0$, thus~$ht = 0$ and~$h = 0$.

It remains to check that~$f$ is independent of the choice of~$v$ and of the
representation~$i(v) = s/t$; else we defined only local sections which might not
glue to a single nontrivial rational section (externally speaking). This is
obvious.
\end{proof}

\begin{prop}Let~$D$ be an effective divisor on~$X$. Then the complement of its
support is scheme-theoretically dense.\end{prop}
\begin{proof}The complement of the support of~$D$, that is~$D(\O_X(-D))$, is
the truth value associated to the statement~``$1 \in \O_X(-D)$''. By
Lemma~\ref{lemma:scheme-theoretical-density}, we therefore have to verify
that~$\O_X$ is separated with respect to the modal operator~$(1 \in \O_X(-D)
\Rightarrow \placeholder)$.

Let~$s \? \O_X$ be given such that~$1 \in \O_X(-D) \Rightarrow s = 0$; we have
to show that~$s = 0$. Writing~$D = [f/1]$ where~$f \? \O_X$ is a regular
element, this condition is equivalent to~$\speak{$f$ \inv} \Rightarrow s = 0$.
By Proposition~\ref{prop:cond-zero} it follows that~$f^n s = 0$ for some~$n
\geq 0$. Since~$f$ is regular, we may cancel~$f^n$ in this equation.
\end{proof}

\begin{prop}Assume that~$X$ is an integral scheme. Then any line bundle on~$X$
is (uncanonically) a submodule of~$\K_X$.\end{prop}
\begin{proof}Let~$\xi$ be the generic point of~$X$ and let~$\Box \defeq \neg\neg$
denote the modal operator such that internal sheafification with respect
to~$\Box$ is the same as pulling back to~$\{\xi\}$ and then pushing forward
to~$X$ again (see Section~\ref{sect:negneg-sheaves}). Let~$\L$ be a line bundle on~$X$. Since~$\L_\xi \cong
\O_{X,\xi}$ (uncanonically), there is some injection~$\L_\xi \to \K_{X,\xi}$;
this corresponds internally to an injection~$\L^{++} \to \K_X^{++}$.
Since~$\K_X$ is already a~$\Box$-sheaf (see
Proposition~\ref{prop:kx-is-negneg-sheafification}) and~$\L$ is~$\Box$-separated
(being isomorphic to~$\O_X$), we have the global injection
\[ \L \lhra \L^{++} \lhra \K_X^{++} \stackrel{({\cong})^{-1}}{\longrightarrow} \K_X. \qedhere \]
\end{proof}

\begin{itemize}
\item ``$\operatorname{div}(g) + D \geq 0$''
\end{itemize}


\section{Compactness and metaproperties}
\label{sect:compactness}

\subsection{Quasicompactness}

As stated in the introduction, quasicompactness of a space can not be detected
by the internal language: There cannot exist a formula~$\varphi$ such that a
topological space is quasicompact if and only if~$\Sh(X) \models \varphi$,
since the latter is always a local property on~$X$ while quasicompactness is not.
However, quasicompactness can be characterized by a \emph{metaproperty} of the
internal language.

This result is best stated in a way which does not explicitly refer to a notion
of finiteness. So recall that quasicompactness of a topological space~$X$ can
be phrased in the following way: For any directed set~$I$ and any monotone
family~$(U_i)_{i \in I}$ of open subsets, if~$X = \bigcup_i U_i$ then~$X = U_i$
for some~$i \in I$. As usual, a \emph{directed set} is an inhabited partially
ordered set such that for any two elements there exists a common upper bound.
A family~$(U_i)_{i \in I}$ is \emph{monotone} if and only if~$i \preceq j$
implies~$U_i \subseteq U_j$.

\begin{prop}\label{prop:quasicompact-meta}
Let~$X$ be a topological space. Then~$X$ is quasicompact if and
only if the internal language of~$\Sh(X)$ has the following metaproperty:
For any directed set~$I$ and any monotone family~$(\varphi_i)_{i \in I}$ of
formulas over~$X$,
\[ \Sh(X) \models \bigvee_{i \in I} \varphi_i
  \quad\text{implies}\quad
  \text{for some~$i \in I$, $\Sh(X) \models \varphi_i$}. \]
The monotonicity condition means that~$\Sh(X) \models (\varphi_i \Rightarrow
\varphi_j)$ for any~$i \preceq j$ in~$I$.
\end{prop}

Stated more succintly, a topological space~$X$ is quasicompact if and only
if~``$\Sh(X) \models$'' commutes with directed~``$\bigvee_{i \in I}$'''s.

\begin{proof}For the ``only if'' direction, let such a family of formulas be
given. Declare~$U_i$ to be the largest open subset of~$X$ where~$\varphi_i$
holds. Then by assumption, the~$U_i$ form a monotone family and cover~$X$. By
quasicompactness of~$X$, some single~$U_i$ covers~$X$ as well, such that the
corresponding formula~$\varphi_i$ holds on~$X$.

For the ``if'' direction, note that a monotone family~$(U_i)$ of open subsets
induces a monotone family of formulas by defining~$\varphi_i \defequiv U_i$. This
correspondence is such that~$\Sh(X) \models \bigvee_i \varphi_i$ holds if and
only if~$X = \bigcup_i U_i$ and such that~$\Sh(X) \models \varphi_i$ if and
only if~$X = U_i$. With these observations the claim is obvious.
\end{proof}

\begin{ex}\label{ex:nilpotency-directed}
Let~$X$ be a quasicompact scheme (or quasicompact ringed space).
Let~$f \in \Gamma(X,\O_X)$ be a global function. Endow the set of natural
numbers with the usual ordering. Then the family of formulas given by~$(f^n =
0)_{n \in \NN}$ is monotone. Thus, if it internally holds that~$f$ is
nilpotent, then~$f$ is nilpotent as an element of~$\Gamma(X,\O_X)$ as
well.\end{ex}

\begin{prop}Let~$X$ be a topological space. Let~$K \subseteq X$ be an open
subset which is \emph{locally quasicompact} in the sense that there exists an open
covering~$X = \bigcup_j U_j$ such that each~$K \cap U_j$ is quasicompact. Then the
internal language of~$\Sh(X)$ has the following metaproperty: For any
directed set~$I$ and monotone family~$(\varphi_i)_{i \in I}$ of formulas
over~$X$ it holds that
\[ \Sh(X) \models \bigl(K \Rightarrow \bigvee_i \varphi_i\bigr)
  \quad\text{implies}\quad
  \Sh(X) \models \bigvee_i (K \Rightarrow \varphi_i). \]
If additionally for any open subset~$V \subseteq X$ the set~$K \cap V$ is
locally quasicompact in~$V$, the following stronger and purely internal
statement holds:
\[ \Sh(X) \models \bigl(K \Rightarrow \bigvee_i \varphi_i\bigr)
  \Longrightarrow
  \bigvee_i (K \Rightarrow \varphi_i). \]
\end{prop}
\begin{proof}Assume that~$\Sh(X) \models (K \Rightarrow \bigvee_i \varphi_i)$.
This is equivalent to~$K \models \bigvee_i \varphi_i$. By the locality of the
internal language, it follows that~$K \cap U_j \models \bigvee_i \varphi_i$ for each~$j$.
Since~$K \cap U_j$ is quasicompact, it follows by the previous proposition that
there exists an index~$i_j \in I$ such that~$K \cap U_j \models \varphi_{i_j}$.
This is equivalent to~$U_j \models (K \Rightarrow \varphi_{i_j})$. In
particular, it holds that~$U_j \models \bigvee_i (K \Rightarrow \varphi_i)$.
Since this is true for any~$j$, it follows that~$X \models \bigvee_i (K
\Rightarrow \varphi_i)$, again by the locality of the internal language.

The second statement is a corollary of the first one.
\end{proof}

\begin{ex}Any retrocompact subset of a scheme is locally quasicompact in the
sense of the proposition.\end{ex}

\begin{ex}\label{ex:df-locally-compact}
Let~$X$ be a scheme and~$f \in \Gamma(X,\O_X)$ be a global function.
Then the open set~$D(f) = \{ x \in X \,|\, \text{$f_x$ is invertible in~$\O_{X,x}$}
\}$ is locally quasicompact in the sense of the proposition, even in the
stronger sense: Let~$V \subseteq X$ be any open set. Consider a covering~$V = \bigcup_i
U_i$ by open affine subsets~$U_i = \Spec A_i$. Then~$D(f) \cap U_i \cong \Spec
A_i[f^{-1}]$ is quasicompact.\end{ex}

From this example it will trivially follow that the nilradical~$\sqrt{(0)}
\subseteq \O_X$ of a scheme and indeed the radical of any quasicoherent ideal
sheaf is quasicoherent (Example~\ref{ex:radical-qcoh}). This example is also
pivotal for giving a simple description of the quasicoherator
(Proposition~\ref{prop:quasicoherator-arbitrary-algebra}), which in turn is
needed for an internal understanding of the relative
spectrum (Section~\ref{sect:relative-spectrum}).

\begin{rem}In applications, the open set~$K$ of the proposition is often given
as the largest open subset on which some formula~$\psi$ holds. (For instance,
in the previous example,~$K$ was given by the formula~$\speak{$f$
is invertible in $\O_X$}$.)
Then the conclusion of the proposition is that \emph{assuming that~$\psi$ holds commutes
with directed disjunctions}.\end{rem}


\subsection{Locality}

A stronger condition on a topological space~$X$ than quasicompactness is
locality: A topological space is \emph{local} if and only if for any open
covering~$X = \bigcup_i U_i$ (not necessarily directed) a certain single subset~$U_i$
covers~$X$ as well. For instance, the spectrum of a ring~$A$ is local if and only
if~$A$ is a local ring. Locality has the following characterization as a metaproperty
of~$\Sh(X)$.

\begin{prop}\label{prop:local-meta}Let~$X$ be a topological space. Then~$X$ is local if and
only if the internal language of~$\Sh(X)$ has the following metaproperty:
For any set~$I$ and any family~$(\varphi_i)_{i \in I}$ of
formulas over~$X$, it holds that
\[ \Sh(X) \models \bigvee_{i \in I} \varphi_i
  \quad\text{implies}\quad
  \text{for some~$i \in I$, $\Sh(X) \models \varphi_i$}. \]
In this case, the internal language has additionally the following (weaker) metaproperty: For any
sheaf~$\F$ on~$X$ and any formula~$\varphi(s)$ containing a variable~$s\?\F$,
it holds that
\[ \Sh(X) \models \exists s\?\F\_ \varphi(s)
  \quad\text{implies}\quad
  \text{for some~$s \in \Gamma(X,\F)$, $\Sh(X) \models \varphi(s)$}. \]
\end{prop}
\begin{proof}The proof of the first part is very similar to the proof of the
previous proposition. For the ``only if'' direction of the second part, note
that the antecedent implies that there exist local section~$s_i \in
\Gamma(U_i,\F)$ such that~$U_i \models \varphi(s_i)$ for some open covering~$X
= \bigcup_i U_i$. By locality of~$X$, one such~$U_i$ suffices to cover~$X$; so
the corresponding section~$s_i$ is actually a global section and verifies~$X
\models \varphi(s_i)$.
\end{proof}

\begin{rem}The second metaproperty stated in the proposition is indeed weaker
than the condition that~$X$ is local. For instance, let~$X$ be a space consisting
of two discrete points. Then~$\Sh(X)$ has the second metaproperty, but~$X$ is
not local.\end{rem}


\subsection{Irreducibility}

In intuitionistic logic, De Morgan's law~$\neg(\alpha \wedge \beta)
\Rightarrow \neg\alpha \vee \neg\beta$ is not generally justified; therefore we
can't use it when working internally to the topos of sheaves on a general scheme~$X$.
The following proposition demonstrates that if~$X$ is irreducible, the law
does hold.

\begin{prop}\label{prop:irreducibility-internally}
A topological space~$X$ is irreducible if and only if the internal
language of~$\Sh(X)$ has the following metaproperty: For any
formulas~$\varphi$ and~$\psi$
\[ \Sh(X) \models \neg(\varphi \wedge \psi)
  \quad\text{implies}\quad
  \Sh(X) \models \neg\varphi \text{ or }
  \Sh(X) \models \neg\psi, \]
and not $\Sh(X) \models \bot$.
Furthermore, in this case the following internal logical principle holds:
\[ \Sh(X) \models \forall \alpha,\beta \in \Omega\_
  \neg(\alpha \wedge \beta) \Rightarrow (\neg\alpha \vee \neg\beta). \]
\end{prop}
\begin{proof}The statement ``$\Sh(X) \models \neg(\varphi \wedge \psi)$'' means
that~$U \cap V = \emptyset$, where~$U$ and~$V$ are the largest open subsets on
which~$\varphi$ respectively~$\psi$ hold. The disjunction ``$\Sh(X) \models
\neg\varphi$ or $\Sh(X) \models \neg\psi$'' means that~$U = \emptyset$ or~$V =
\emptyset$. And ``$\Sh(X) \models \bot$'' is equivalent to~$X = \emptyset$.

Therefore, if~$X$ is irreducible, then the internal language has the claimed metaproperty. The converse
can be seen by instantiating~$\varphi$ and~$\psi$ with the formulas associated
to given open subsets having empty intersection. It then follows that one of
these formulas is false in the internal language; thus the associated subset is
empty.

The stated internal logical principle holds since nonempty open subsets of irreducible spaces are
irreducible.
\end{proof}


\subsection{Internal proofs of common lemmas}

\begin{lemma}Let~$X$ be an irreducible reduced scheme. Then all local
rings~$\O_{X,x}$ are integral domains.\end{lemma}
\begin{proof}It suffices to give a proof of the following statement: Let~$R$ be
a local ring such that elements which are not invertible are nilpotent. Further
assume that~$R$ is reduced. Then~$R$ is an integral domain in the weak sense.

This proof may, additionally to the rules of intuitionistic logic, use the
classical axiom given by Proposition~\ref{prop:irreducibility-internally}.

So let arbitrary elements~$x,y \? R$ with~$xy = 0$ be given. Then it is not the
case that~$x$ and~$y$ are both invertible: If they were, their product~$xy$
would be invertible as well, contradicting~$1 \neq 0$. By the classicality
principle, it follows that~$x$ is not invertible or that~$y$ is not invertible.
Thus~$x$ or~$y$ is nilpotent and therefore zero.
\end{proof}


\begin{itemize}
\item basic lemmas: filtered colimits, flatness, \ldots
\end{itemize}


\section{Quasicoherent sheaves of modules}
\label{sect:qcoh}

Recall that an~$\O_X$-module~$\F$ on a ringed space~$X$ is \emph{quasicoherent}
if and only if there exists a covering of~$X$ by open subsets~$U$ such that on
each such~$U$, there exists an exact sequence
\[ (\O_X|_U)^J \longrightarrow (\O_X|_U)^I \longrightarrow \F|_U \longrightarrow 0 \]
of~$\O_X|_U$-modules, where~$I$ and~$J$ are arbitrary sets (which may depend
on~$U$).

If~$X$ is indeed a scheme, quasicoherence can also be characterized in
terms of inclusions of distinguished open subsets of affines:
An~$\O_X$-module~$\F$ is quasicoherent if and only if for any open affine
subscheme~$U = \Spec A$ of~$X$ and any function~$f \in A$, the canonical map
\[ \Gamma(U,\F)[f^{-1}] \longrightarrow \Gamma(D(f),\F),\
  \tfrac{s}{f^n} \longmapsto f^{-n} s|_{D(f)} \]
is an isomorphism of~$A[f^{-1}]$-modules. Here~$D(f) \subseteq U$ denotes the
standard open subset~$\{ \ppp \in \Spec A \,|\, f \not\in \ppp \}$. Both
conditions can be internalized.

\begin{prop}Let~$X$ be a ringed space. Let~$\F$ be an~$\O_X$-module. Then~$\F$
is quasicoherent if and only if
\[ \Sh(X) \models \exists I,J\ \mathrm{lc}\_ \speak{there exists an
  exact sequence~$\O_X^J \to \O_X^I \to \F \to 0$}. \]
The ``\textnormal{lc}'' indicates that when interpreting this internal statement with the
Kripke--Joyal semantics,~$I$ and~$J$ should only be instantiated with
\emph{locally constant} sheaves.
\end{prop}
\begin{proof} We only sketch the proof.
The translation of the internal statement is that there exists a covering
of~$X$ by open subsets~$U$ such that for each such~$U$, there exist sets~$I,J$
and an exact sequence
\[ (\O_X|_U)^{\ul{J}} \longrightarrow (\O_X|_U)^{\ul{I}} \longrightarrow \F|_U
\longrightarrow 0 \]
where~$\ul{I}$ and~$\ul{J}$ are the constant sheaves associated to~$I$
respectively~$J$. The term~``$(\O_X|_U)^{\ul{I}}$'' refers to the internally
defined free~$\O_X$-module with basis the elements of~$\ul{I}$. By exploiting
that~$\ul{I}$ is a discrete set from the internal point of view (\ie any two
elements are either equal or not), one can show that this is the same
as~$(\O_X|_U)^I$; similarly for~$J$. With this observation, the statement
follows.
\end{proof}

\begin{rem}The restriction to locally constant sheaves is really necessary: The
internal statement~$\Sh(X) \models \exists I,J\_ \speak{there exists an
exact sequence~$\O_X^J \to \O_X^I \to \F \to 0$}$ is true for
\emph{any}~$\O_X$-module~$\F$. This is because the usual proof of the fact that
any module admits a resolution by (not necessarily finite) free modules is
intuitionistically acceptable and thus also valid in the internal
universe.\end{rem}

We don't think that there is a useful internal characterization of
locally constant sheaves. The alternative internal condition given by the following
theorem does not need such a characterization.

\begin{thm}\label{thm:qcoh-sheafchar}
Let~$X$ be a scheme. Let~$\F$ be an~$\O_X$-module. Then~$\F$ is
quasicoherent if and only if, from the internal perspective, for any~$f\?\O_X$,
the localized module~$\F[f^{-1}]$ is a sheaf for the modal operator~$(\speak{$f$ \inv}
\Rightarrow \placeholder)$.
\end{thm}

In detail, the internal condition is that for any~$f\?\O_X$, it holds that
\[ \forall s\?\F[f^{-1}]\_
  (\speak{$f$ \inv} \Rightarrow s = 0) \Longrightarrow s = 0 \]
and for any subsingleton~$\S \subseteq \F[f^{-1}]$ it holds that
\[ (\speak{$f$ \inv} \Rightarrow \speak{$\S$ inhabited}) \Longrightarrow
  \exists s\?\F[f^{-1}]\_
  (\speak{$f$ \inv} \Rightarrow s \in \S). \]
Unlike with the internalizations of finite type, finite presentation and
coherence, this condition is \emph{not} a standard condition of commutative
algebra. In fact, in classical logic, this condition is always satisfied --
for trivial logical reasons if~$f$ is invertible, and because~$\F[f^{-1}]$ is
the zero module if~$f$ is not invertible (since~$f$ is nilpotent then, by
Proposition~\ref{prop:neginvnilpotent}).

That this condition in not known in commutative algebra is to be expected:
Quasicoherence is a condition on sheaves of modules, ensuring
that they are locally isomorphic to sheaves of the form~$M^\sim$,
where~$M$ is a plain module. But in commutative algebra, one \emph{only} studies plain
modules (and not sheaves of modules). The quasicoherence condition is imported
into the realm of commutative algebra only by the internal language.

We give the proof of the theorem below, after first discussing some examples
and consequences. The proof will explain the origin of this condition.

\begin{ex}The zero~$\O_X$-module is quasicoherent, since (it and) all
localizations of it are singleton sets from the internal perspective and
thus~$\Box$-sheaves for any modal operator~$\Box$
(Example~\ref{ex:special-sets-sheaves}).\end{ex}

\begin{cor}\label{cor:submodule-qcoh}
Let~$X$ be a scheme. Let~$\F$ be a quasicoherent~$\O_X$-module.
Let~$\G \subseteq \F$ be a submodule. Then~$\G$ is quasicoherent if and only
if
\[ \Sh(X) \models \forall f\?\O_X\_
  \forall s\?\F\_
  (\speak{$f$ \inv} \Rightarrow s \in \G) \Longrightarrow
  \bigvee_{n \geq 0} f^n s \in \G. \]
\end{cor}
\begin{proof}We can give a purely internal proof. Let~$f\?\O_X$.
Since subpresheaves of separated sheaves are separated, the module~$\G[f^{-1}]$
is in any case separated with respect to the modal operator~$\Box$
with~$\Box\varphi \defequiv (\speak{$f$ \inv} \Rightarrow \varphi)$.

Now suppose that~$\G$ is quasicoherent. Let~$f\?\O_X$. Let $s\?\F$ and assume that
if~$f$ were invertible,~$s$ would be an element of~$\G$. Define the
subsingleton~$S \defeq \{ t\?\G[f^{-1}] \,|\, \speak{$f$ \inv} \wedge t=s/1 \}$.
Then~$S$ would be inhabited by~$s/1$ if~$f$ were invertible. Since~$\G[f^{-1}]$
is a~$\Box$-sheaf, it follows that there exists an element~$u/f^n$ of~$\G[f^{-1}]$
such that, if~$f$ were invertible, it would be the case that~$u/f^n = s/1 \in
\G[f^{-1}] \subseteq \F[f^{-1}]$.
Since~$\F[f^{-1}]$ is~$\Box$-separated, it follows that it actually holds that~$u/f^n
= s/1 \in \F[f^{-1}]$. Therefore there exists~$m\?\NN$ such that $f^m f^n s =
f^m u \in \F$. Thus~$f^{m+n} s$ is an element of~$\G$.

For the converse direction, assume that~$\G$ fulfills the stated condition.
Let$f\?\O_X$. Let~$S \subseteq \G[f^{-1}]$ be a subsingleton which would be
inhabited if~$f$ were invertible. By regarding~$S$ as a subset of~$\F[f^{-1}]$,
it follows that there exists an element~$u/f^n \in \F[f^{-1}]$ such that,
if~$f$ were invertible, $u/f^n$ would be an element of~$S$. In particular,~$u$
would be an element of~$\G$. By assumption
it follows that there exists~$m\?\NN$ such that~$f^m u \in G$. Thus~$(f^m u) /
(f^m f^n)$ is an element of~$\G[f^{-1}]$ such that, if~$f$ were invertible, it
would be an element of~$S$.
\end{proof}

\begin{ex}\label{ex:annihilator-qcoh}
Let~$X$ be a scheme and~$s$ be a global section of~$\O_X$. Then the
annihilator of~$s$, \ie the sheaf of ideals internally defined by the
formula
\[ I \defeq \Ann_{\O_X}(s) = \{ t\?\O_X \,|\, st = 0 \} \subseteq \O_X \]
is quasicoherent. To prove this in the internal language it suffices to
verify the condition of the proposition.
So let~$f\?\O_X$ and~$t\?\O_X$ be arbitrary and assume~$\speak{$f$ \inv} \Rightarrow t \in I$,
\ie assume that if~$f$ were invertible, then~$st$ would be zero. By
Proposition~\ref{prop:cond-zero} it follows that~$f^n st = 0$ for
some~$n\?\NN$, \ie that~$f^n t \in I$.
\end{ex}

\begin{ex}\label{ex:radical-qcoh} Let~$X$ be a scheme and~$\I \subseteq \O_X$
be a quasicoherent ideal sheaf.  Then the radical of~$\I$, internally definable
as \[ \sqrt{\I} \defeq \Bigl\{ s\?\O_X \,\Big|\, \bigvee_{n \geq 0} s^n \in \I \Bigr\}, \] is
quasicoherent as well: Let~$f\?\O_X$ and~$s\?\O_X$ be arbitrary and
assume~$\speak{$f$ \inv} \Rightarrow s \in \sqrt{\I}$, \ie assume that if~$f$
were invertible, some power~$s^n$ would be an element of~$\I$. Since
\emph{assuming that~$f$ is invertible commutes with directed disjunctions}
(Example~\ref{ex:df-locally-compact}), it follows that for some natural
number~$n$, it holds that~$\speak{$f$ \inv} \Rightarrow s^n \in \I$. By
quasicoherence of~$\I$, we may deduce that for some natural number~$m$, it
holds that~$f^m s^n \in \I$. Thus~$fs \in \sqrt{\I}$.\end{ex}

%\begin{ex}\label{ex:qcoh-single-radical}
%Let~$X$ be a scheme and~$\A$ be a quasicoherent~$\O_X$-algebra. Let~$h \in
%\Gamma(X,\A)$ be a global section of~$\A$. Then the radical ideal~$\sqrt{(h)}
%\subseteq \A$ is quasicoherent.\end{ex}
% A proof could look like this:
% Assume (f inv => exists u: s = uh).
% Set K := { z : O[(fs)^(-1)] | exists u: z = u/1, s = uh }.
% Then there exists z = a/(fs)^n such that (fs inv => s/1 = ah/(fs)^n).
% Therefore (fs)^m (fs)^n s = (fs)^m ah.
% XXX: This only works for A = O.

\begin{prop}Let~$X$ be a scheme of dimension~$\leq 0$. Then any~$\O_X$-module
is quasicoherent.\end{prop}
\begin{proof}By Corollary~\ref{cor:scheme-dimension-zero}, any
element~$f\?\O_X$ is invertible or nilpotent. Therefore the quasicoherence
condition of Theorem~\ref{thm:qcoh-sheafchar} is trivially satisfied for any~$\O_X$-module.
\end{proof}

\begin{rem}\label{rem:qcoh-in-constructive-mathematics}
In general intuitionistic mathematics -- not inside the internal universe of a
scheme -- the notion of quasicoherence as given by the internal condition of
Theorem~\ref{thm:qcoh-sheafchar}
does not seem to be very interesting: For many important rings, there are few
quasicoherent modules in this sense. For instance, let~$M$ be a module over a
ring~$R$ in which every element is invertible or not invertible. (The
ring~$\ZZ$ is such a ring.) Then~$M$ is quasicoherent if and only if for any~$f
\? R$ which is not invertible, the localized module~$M[f^{-1}]$ is the zero
module, \ie any element of~$M$ is annihilated by some power~$f^n$. As a
concrete example, any~$\ZZ$-submodule of~$\ZZ$ which contains a nonzero element
fails to be quasicoherent.
\end{rem}

\begin{proof}[Proof of Theorem~\ref{thm:qcoh-sheafchar}]
By the well-known characterization of quasicoherence in terms of inclusions of
distinguished open subsets, an~$\O_X$-module~$\F$ is quasicoherent if and only
if for any affine open subset~$U \subseteq X$ and any function~$f \in
\Gamma(U,\O_U)$, the canonical map
\begin{equation}\label{eqn:restr-map}
  \Gamma(U,\F)[f^{-1}] \lra \Gamma(D(f),\F), \ s/f^n \longmapsto
  f^{-n} s|_{D(f)}
\end{equation}
is bijective. We will see that this map is injective for all such~$U$ and~$f$
if and only if from the internal perspective, for any~$f\?\O_X$, the set~$\F[f^{-1}]$ is a
separated presheaf with respect to the modal operator~$(\speak{$f$ \inv}
\Rightarrow \placeholder)$; and we will see that in this
case, the map is additionally surjective for all such~$U$ and~$f$ if the full
sheaf condition is fulfilled.

Since the sheaf~$\F[f^{-1}]$ does not appear in the stated characterization, we
will first reformulate the separatedness and the sheaf condition in terms
of~$\F$ instead of~$\F[f^{-1}]$. To this end, note that the separatedness
condition is equivalent to
\begin{equation}\label{eqn:separated}
  \forall f\?\O_X\_ \forall s\?\F\_
  (\speak{$f$ \inv} \Rightarrow s = 0 \? \F) \Longrightarrow
  \bigvee_{n \geq 0} f^n s = 0 \? \F.
\end{equation}
The equivalence can easily be proven in the internal language. The sheaf
condition is equivalent to the conjunction of the separatedness condition and
\begin{multline}\label{eqn:sheaf}
  \forall f\?\O_X\_ \forall \K \subseteq \F\_
  (\speak{$f$ \inv} \Rightarrow \speak{$K$ is a singleton})
  \Longrightarrow \\
  \bigvee_{n \geq 0} \exists s\?\F\_
  \speak{$f$ \inv} \Rightarrow f^{-n} s \in \K.
\end{multline}
In one direction, a set~$\S \subseteq \F[f^{-1}]$ is given; construct~$K \defeq \{
s\?\F \,|\, s/1 \in \S \} \subseteq \F$. In the other direction, a set~$\K
\subseteq \F$ is given; construct~$S \defeq \{ s\?\F[f^{-1}] \,|\, \exists
s'\?\F\_ s' \in \K \wedge s = s'/1 \} \subseteq \F[f^{-1}]$. The remaining
details can easily be filled in.

We now interpret the internal statement~\eqref{eqn:separated} with the
Kripke--Joyal semantics. Using the simplification rules, the external meaning
is that for any affine open subset~$U \subseteq X$ and any function~$f \in
\Gamma(U,\O_U)$ the following condition is satisfied: For any section~$s \in
\Gamma(U,\F)$ it should hold that
\[ U \models (\speak{$f$ \inv} \Rightarrow s = 0) \quad\text{implies}\quad
  U \models \bigvee_{n \geq 0} f^n s = 0. \]
The antecedent is equivalent to saying that~$s$ is zero in~$\Gamma(D(f),\F)$.
The consequent is (by quasicompactness of~$U$, see
Example~\ref{ex:nilpotency-directed}) equivalent to saying that for some~$n \geq 0$, the
section~$f^n s$ is zero in~$\Gamma(U,\F)$, \ie that~$s$ is zero
in~$\Gamma(U,\F)[f^{-1}]$. So this condition is precisely the injectivity of
the canonical map~\eqref{eqn:restr-map}.

The external meaning of statement~\eqref{eqn:sheaf} is that for any affine open
subset~$U \subseteq X$ and any function~$f \in \Gamma(U,\O_U)$ the following
condition is satisfied: For any subsheaf~$\K \subseteq \F|_U$ it should hold
that
\begin{multline*}
  U \models (\speak{$f$ \inv} \Rightarrow \speak{$\K$ is a singleton})
  \quad\text{implies} \\
  U \models \bigvee_{n \geq 0} \exists s\?\F\_
  \speak{$f$ \inv} \Rightarrow f^{-n} s \in \K.
\end{multline*}
Given the injectivity of the canonical map~\eqref{eqn:restr-map} (for any
affine open subset, not only~$U$), this condition is equivalent to its
surjectivity: To see that surjectivity is sufficient, let a subsheaf~$\K
\subseteq \F|_U$ verifying the antecedent be given. Since~$\K|_{D(f)}$ is a
singleton sheaf, we can consider its unique section~$u \in \Gamma(D(f),\K)
\subseteq \Gamma(D(f),\F)$. By surjectivity, there exists a preimage, \ie a
fraction~$s/f^n \in \Gamma(U,\F)[f^{-1}]$ such that~$u = f^{-n} s|_{D(f)}$
in~$\Gamma(D(f),\F)$. Thus~$U \models f^{-n}s \in \K$ holds and the consequent
is verified.

To see that surjectivity is necessary, let a section~$u \in \Gamma(D(f),\F)$ be
given. Define a subsheaf~$\K \subseteq \F|_U$ by setting~$\Gamma(V,\K) \defeq \{
u|_V \,|\, V \subseteq D(f) \}$. Then~$\K$ verifies the antecedent. Thus the
consequent holds: There exists an open covering~$U = \bigcup_i U_i$ such that
for each~$i$, there exists a natural number~$n_i$ and a section~$s_i \in
\Gamma(U_i,\F)$ such that~$f^{-n_i} s_i = u$ on~$U_i \cap D(f)$. Without loss of
generality, we may assume that the~$U_i$ are distinguished open subsets~$D(g_i)
\subseteq U$; that they are finite in number; and that the natural
numbers~$n_i$ agree with each other and thus equal some number~$n$. Since~$s_i
= s_j$ in~$\Gamma(U_i \cap U_j \cap D(f), \F)$, injectivity of the canonical
map~\eqref{eqn:restr-map} (on the affine set~$U_i \cap U_j = D(g_i g_j)$)
implies that~$s_i = s_j$ in~$\Gamma(U_i \cap U_j, \F)[f^{-1}]$. Thus for
any indices~$i,j$ there exists a natural number~$m_{ij}$ such that~$f^{m_{ij}} s_i =
f^{m_{ij}} s_j$ in~$\Gamma(U_i \cap U_j, \F)$. We may assume that the
numbers~$m_{ij}$ equal some common number~$m$; thus the local sections~$f^m s_i$
glue to a section~$s \in \Gamma(U,\F)$. The sought preimage of~$u$ is the
fraction~$s/f^{n+m}$, since~$f^{-(n+m)} s|_{D(f)}$ equals~$u$
in~$\Gamma(D(f),\F)$ (as this is true on the covering~$D(f) = \bigcup_i (D(f)
\cap U_i)$).
\end{proof}
% FUTURE: Remark that Gamma(U, F[f^(-1)]) = Gamma(U, F)[f^(-1)],
% since F[f^(-1)] is a filtered colimit (see
% stacks-project/sites:lemma-directed-colimits-sections).
% This argument only works if multiplication by f is injective as a map F --> F.

For applications in Section~\ref{sect:relative-spectrum} about interpreting the
relative spectrum as an internal spectrum, we want to specialize to radical
ideal sheaves. In particular, we want to describe the \emph{quasicoherator} --
the left adjoint to the inclusion of the quasicoherent radical ideals in the
poset of all radical ideals -- in simple terms.

\begin{prop}\label{prop:quasicoherator-structure-sheaf}
Let~$X$ be a scheme. Let~$\I \subseteq \O_X$ be a radical ideal.
\begin{enumerate}
\item The ideal~$\I$ is quasicoherent if and only if
\[ \Sh(X) \models \forall s\?\O_X\_ (\speak{$s$ \inv} \Rightarrow s\in\I)
\Rightarrow s\in\I. \]
\item The reflection of~$\I$ in the poset of quasicoherent radical ideals is
the sheaf~$\overline{\I}$ given by the internal expression
\[ \overline{\I} \defeq \{ s\?\O_X \,|\, \speak{$s$ \inv} \Rightarrow s\in\I
\}. \]
\end{enumerate}
\end{prop}
\begin{proof}Both claims can be verified by purely internal reasoning. The
first claim is a straightforward calculation using the characterization given in
Corollary~\ref{cor:submodule-qcoh}. We discuss the second one in more detail.

Firstly, it's obvious that~$\overline{\I}$ contains~$\I$ and
that~$\overline{\I}$ is a radical ideal. To verify that~$\overline{\I}$ is
quasicoherent, let~$s\?\O_X$ be given such that, if~$s$ were invertible,
then~$s$ would be an element of~$\overline{\I}$. Symbolically, we have
\[ \speak{$s$ \inv} \Longrightarrow (\speak{$s$ \inv} \Rightarrow s\in\I), \]
which of course implies
\[ \speak{$s$ \inv} \Longrightarrow s\in\I. \]
This is precisely the condition for~$s$ to be an element of~$\overline{I}$.

To verify that the construction~$\I \mapsto \overline{\I}$ is really left
adjoint to the inclusion, let a quasicoherent radical ideal~$\J$ be given such
that~$\I \subseteq \J$. We have to show that~$\overline{\I} \subseteq \J$. This
is straightforward.
\end{proof}

For arbitrary~$\O_X$-algebras~$\A$, the description of the quasicoherator for
radical ideals of~$\A$ is more involved, but still sufficiently explicit for the
applications in Section~\ref{sect:relative-spectrum}.

\begin{prop}\label{prop:quasicoherator-arbitrary-algebra}
Let~$X$ be a scheme. Let~$\A$ be a quasicoherent~$\O_X$-algebra.
Then the reflection of a radical ideal~$\I \subseteq \A$ in the poset of
quasicoherent radical ideals of~$\A$ is given by the internal expression
\[ \overline{\I} \defeq \bigcup_{n \geq 0} \I_n, \]
where~$(\I_n)$ is the family of radical ideals defined recursively by
\begin{align*}
  \I_0 &\defeq \I, \\
  \I_{n+1} &\defeq \textnormal{the radical ideal generated by
  $\{ fs \,|\, f\?\O_X, s\?\A, (\speak{$f$ \inv} \Rightarrow s \in \I_n) \}$}.
\end{align*}
\end{prop}
\begin{proof}We argue internally. The set~$\overline{\I}$ contains~$\I$ and is
a radical ideal, as an ascending union of radical ideals. To verify
that~$\overline{\I}$ is quasicoherent, let~$f\?\O_X$ and~$s\?\A$ be given such
that, if~$f$ were invertible, then~$s$ would be an element of~$\overline{\I}$.
This means that we have
\[ \speak{$f$ \inv} \Longrightarrow \bigvee_{n \geq 0} s \in \I_n. \]
Since assuming that~$f$ is invertible commutes with directed disjunctions
(Example~\ref{ex:df-locally-compact}), there is a natural number~$n$ such that
\[ \speak{$f$ \inv} \Longrightarrow s \in \I_n. \]
Therefore~$fs \in \I_{n+1} \subseteq \overline{\I}$.

Finally, to verify that the construction~$\I \mapsto \overline{\I}$ is indeed
left adjoint to the inclusion of the quasicoherent radical ideals in all
radical ideals, let a quasicoherent radical ideal~$\J$ be given such that~$\I
\subseteq \J$. By induction we can show that~$\I_n \subseteq \J$ for all
natural numbers~$n$. Therefore~$\overline{\I} \subseteq \J$.
\end{proof}

\begin{rem}\label{rem:reflector-single-element}
If the goal was to close a given radical ideal under the condition
\[ \forall s\?\A\_ (\speak{$f$ \inv} \Rightarrow s \in \I) \Longrightarrow fs \in \I, \]
where~$f \? \O_X$ is a fixed element, no infinite iteration would be necessary.
The closure would in this case simply be given by
\[ \overline{\I}^f \defeq \text{the radical ideal generated by the set~$\{ fs
\,|\, s \? \A, (\speak{$f$ \inv} \Rightarrow s \in \I) \}$}. \]\end{rem}

There is also a purely formal description of the reflector, given by
\[ \I \longmapsto \bigcap \{ \J \subseteq \A \,|\,
  \text{$\J$ is a quasicoherent radical ideal such that~$\I \subseteq \J$} \}. \]
Verifying that this construction has the universal property of
the reflector is straightforward. However, it is not sufficiently concrete for
calculations. In particular, we don't see a way to prove the following
corollary without the explicit description given by
Proposition~\ref{prop:quasicoherator-arbitrary-algebra}.

\begin{cor}\label{cor:quasicoherator-meet}
Let~$X$ be a scheme. Let~$\A$ be a quasicoherent~$\O_X$-algebra. Let~$\I$
and~$\J$ be radical ideals of~$\A$. Then~$\overline{\I \cap \J} =
\overline{\I} \cap \overline{\J}$.
\end{cor}
\begin{proof}The claim is not purely formal. As a left adjoint, the reflector
preserves arbitrary suprema (as a map from the poset of all radical ideals into the poset
of all quasicoherent radical ideals); but the claim is that it preserves (finite) intersections.

Since the reflector is monotone, it is clear that~$\overline{\I
\cap \J} \subseteq \overline{\I} \cap \overline{\J}$.

To verify the converse
direction, we show by induction that~$\I_n \cap \J_m \subseteq \overline{\I
\cap \J}$ for all natural numbers~$n$ and~$m$. The base case is trivial,
since~$\I_0 \cap \J_0 = \I \cap \J$. For the induction step let~$x \in \I_{n+1}
\cap \J_m$. Then~$x^\ell = \sum_i f_i s_i$ for some natural number~$\ell$ and
elements~$f_i \? \O_X$, $s_i \? \A$ such that~$\speak{$f_i$ \inv} \Rightarrow
s_i \in \I_n$. In particular we have~$\speak{$f_i$ \inv} \Rightarrow s_i x \in
\I_n \cap \J_m$, so by the induction hypothesis~$\speak{$f_i$ \inv} \Rightarrow
s_i x \in \overline{\I \cap \J}$. This implies~$f_i s_i x \in \overline{\I \cap
\J}$, since~$\overline{\I \cap \J}$ is quasicoherent. Therefore~$x^{\ell+1} \in
\overline{\I \cap \J}$ and thus~$x \in \overline{\I \cap \J}$.
\end{proof}

\begin{rem}\label{rem:quasicoherator-knaster-tarski}
If in the situation of
Proposition~\ref{prop:quasicoherator-arbitrary-algebra} the algebra~$\A$ is not
quasicoherent, the construction~$\I \mapsto \overline{\I}$ is still left
adjoint to the inclusion of the radical ideal sheaves which satisfy the (then
somewhat unmotivated) internal condition given in
Corollary~\ref{cor:submodule-qcoh} in the poset of all radical ideal sheaves.
Also Corollary~\ref{cor:quasicoherator-meet} remains valid.
This is even the case if~$X$ is an arbitrary ringed space; in this case,
the proofs of Proposition~\ref{prop:quasicoherator-arbitrary-algebra} and
Corollary~\ref{cor:submodule-qcoh} have to be modified, since then we may not
suppose that assuming that an element of~$\O_X$ is invertible commutes with
directed disjunctions.

Instead, the reflector~$\I \mapsto \overline{\I}$ has to
be characterized by
\[ \overline{\I} \defeq \text{least fixed point of~$P$ above~$\I$}, \]
where~$P$ is the monotone operator on the set of radical ideals which takes a
radical ideal~$\I$ to the radical ideal generated by~$\{ fs \,|\, f\?\O_X,
s\?\A, (\speak{$f$ \inv} \Rightarrow s \in \I) \}$. The existence of these
fixed points is guaranteed by the Knaster--Tarski theorem, which is
intuitionistically valid in the version we need~\cite{bauer:lumsdaine:bourbaki-witt}.

%The following proof scheme is useful for verifying properties of the least
%fixed point. Let~$\varphi(\J)$ be a statement on radical ideals~$\J$ such that
%$\varphi(\sup_i \J_i) \Leftrightarrow \bigvee_i \varphi(\J_i)$ for arbitrary
%families~$(\J_i)_i$ of radical ideals. Let~$\psi$ be a further statement. If
%$\varphi(\I) \Rightarrow \psi$ and
%\[ (\varphi(\J) \Rightarrow \psi) \Longrightarrow (\varphi(P(\J)) \Rightarrow \psi) \]
%for all radical ideals~$\J$ containing~$\I$, then~$\varphi(\overline{\I}) \Rightarrow \psi$.
The following proof scheme is useful for verifying properties of the least
fixed point. Let~$\varphi(\J)$ be a statement on radical ideals~$\J$ such that
$\varphi(\sup_i \J_i) \Leftrightarrow \bigvee_i \varphi(\J_i)$ for every
family~$(\J_i)_i$ of radical ideals. If
\[ \varphi(P(\J)) \Longrightarrow \varphi(\J) \]
for all radical ideals~$\J$ containing~$\I$, then~$\varphi(\overline{\I})
\Rightarrow \varphi(\I)$. This proof scheme is a special case of the following
more general scheme, which is also sometimes needed for reasoning about the
least fixed point.

Let~$L$ be a complete partial order. Let~$\alpha$ be a map from the set of
radical ideals to~$L$ such that $\alpha(\sup_i \J_i) = \sup_i \alpha(\J_i)$ for
every family~$(\J_i)_i$ of radical ideals. If
\[ \alpha(P(\J)) \preceq \alpha(\J) \]
for all radical ideals~$\J$ containing~$\I$, then~$\alpha(\overline{\I})
\preceq \alpha(\I)$.
\end{rem}
% The modified proof of the corollary goes as follows.
% First note that P(I cap J) = P(I) cap P(J).
% Then apply the general lemma that for monotone meet-preserving operators P,
% it holds that mu(P)_{>= x} wedge mu(P)_{>= y} = mu(P)_{>= x wedge y},
% when x and y are post-fixed points and mu(P)_{>= z} denotes the least fixed
% point of P above z.

\begin{rem}The reflector can also be given by the formula
\[ \overline{\I} = \bigcap_\J\,
  \Bigl(\J : \bigcap_{f\?\O_X} (\J : \overline{\I}^f)\Bigr), \]
where~$\overline{\I}^f$ is as in Remark~\ref{rem:reflector-single-element} and
the first intersection is indexed by all radical ideals~$\J \subseteq \A$.
This identity follows by the description of~$\overline{\I}$ as a least fixed
point and the explicit formula for the least fixed point from the proof of its
existence~\cite{bauer:lumsdaine:bourbaki-witt}. It also follows from the
observation that the operation~$\J \mapsto \overline{\J}$ is the nucleus
associated to the intersection of the sublocales given by the nuclei~$\J
\mapsto \overline{\J}^f$, which in turn is evident from the description of the
relative spectrum as a classifying locale given in
Proposition~\ref{prop:local-spectrum-classify}.
\end{rem}

% FUTURE:
% Think how to prove that modules of finite type are
% quasicoherent and that cokernels are quasicoherent.


\section{Subschemes}

\subsection{Sheaves on open and closed subspaces} It is well-known that sheaves
defined on open or closed subspaces of a topological space~$X$ can be related
with certain sheaves on~$X$, by using appropriate extension functors. We can
define these functors and show their basic properties in the internal
language. Recall from Section~\ref{sect:modalities-geometric-meaning} that we
have defined a formula~``$U$'' for any open subset~$U \subseteq X$ such that
$V \models U$ if and only if $V \subseteq U$.

\begin{lemma}\label{lemma:extension-by-empty-set}
Let~$X$ be a topological space. Let~$j : U \hookrightarrow X$ be the inclusion
of an open subspace. Then there is a canonical functor~$j_! : \Sh(U) \to
\Sh(X)$ called \emph{extension by the empty set} with the following properties:
\begin{enumerate}
\item The functor~$j_!$ is left adjoint to the restriction functor~$j^{-1} : \Sh(X) \to
\Sh(U)$.
\item The composition~$j^{-1} \circ j_! : \Sh(U) \to \Sh(U)$ is (canonically
isomorphic to) the identity.
\item The essential image of~$j_!$ consists of exactly those sheaves on~$X$
whose stalks are empty at all points of~$U^c$. For those sheaves~$\F$ it holds
that~$j_!j^{-1}\F \cong \F$ (canonically).
\end{enumerate}
\end{lemma}
\begin{proof}Internally, for a set~$\F$, we can define~$j_!(\F)$ simply to be the
set comprehension
\[ j_!(\F) \defeq \{ x\?\F \,|\, U \}. \]
Externally, the sections of the thus defined sheaf on an open subset~$V
\subseteq X$ are given by~$\{ x \in \Gamma(V,\F) \,|\, V \subseteq U \}$,
\ie all of~$\Gamma(V,\F)$ if~$V \subseteq U$ and the empty set otherwise.
With this short internal description, all of the stated properties can be
easily verified in the internal language.

For instance, recall that internally the functor~$j^{-1}$ is given by
sheafifying with respect to the modal operator~$\Box \defequiv (U \Rightarrow
\placeholder)$. Thus, to show the second statement, we have to give a
bijection~$(j_!(\F))^{++} \to \F$ for any~$\Box$-sheaf~$\F$. (This map has to
be given explicitly, to not only show a weaker statement about a local
isomorphism -- see Section~\ref{sect:internal-constructions}). To this end, we can use the composition
\[ (j_!(\F))^{++} \lhra \F^{++} \stackrel{({\cong})^{-1}}{\lra} \F, \]
where the first map is injective since sheafifying is exact. It is also
surjective, since the~$\Box$-translation of the statement~$\speak{$j_!(\F) \to
\F$ is surjective}$ holds: For any element~$x\?\F$, it holds
that~$\Box(\speak{$x$ possesses a preimage})$.

For the third property, note that a sheaf~$\F$ on~$X$ fulfills the stated
condition on stalks if and only if, from the internal perspective, it holds
that~$U \Rightarrow \speak{$\F$ is inhabited}$. We omit further details.
\end{proof}

\begin{lemma}\label{lemma:extension-by-zero}
Let~$X$ be a ringed space. Let~$j : U \hookrightarrow X$ be the inclusion
of an open subspace. Then there is a canonical functor~$j_! : \Mod_U(\O_U) \to
\Mod_X(\O_X)$ called \emph{extension by zero} with the following properties:
\begin{enumerate}
\item The functor~$j_!$ is left adjoint to the restriction functor~$j^{-1} :
\Mod_X(\O_X) \to \Mod_U(\O_U)$.
\item The composition~$j^{-1} \circ j_! : \Mod_U(\O_U) \to \Mod_U(\O_U)$ is (canonically
isomorphic to) the identity.
\item The essential image of~$j_!$ consists of exactly those~$\O_X$-modules
whose stalks are zero at all points of~$U^c$. For those sheaves~$\F$ it holds
that~$j_!j^{-1}\F \cong \F$ (canonically).
\end{enumerate}
\end{lemma}
\begin{proof}Internally, a sheaf of modules on~$\O_U$ is simply a module
on~$\O_X^{++}$ which is a~$\Box$-sheaf, where~$\Box \defequiv (U \Rightarrow
\placeholder)$. The suitable internal definition for the extension by zero of
such a module~$\F$ is
\[ j_!(\F) \defeq \{ x\?\F \,|\, (x = 0) \vee U \}. \]
With this description, all necessary verifications are easy. Note that
an~$\O_X$-module~$\F$ fulfills the stated condition on stalks if and only if
internally, it holds that~$\forall x\?\F\_ ((x = 0) \vee U)$.
\end{proof}

\begin{lemma}\label{lemma:essim-closed-immersion}
Let~$X$ be a topological space. Let~$i : A \hookrightarrow X$ be the inclusion
of a closed subspace. The essential image of the
inclusion~$i_* : \Sh(A) \to \Sh(X)$ consists of exactly those sheaves whose support
is a subset of~$A$. For those sheaves~$\F$ it holds that~$i_* i^{-1} \F \cong \F$
(canonically).\end{lemma}
\begin{proof}Recall that the modal operator associated to~$A$ is~$\Box\varphi
\defequiv (\varphi \vee A^c)$, and that by Section~\ref{sect:internal-sheaves} the
essential image of~$i_*$ consists of exactly those sheaves which
are~$\Box$-sheaves from the internal perspective. Let~$\F$ be a sheaf on~$X$.
Then it holds that
\[ \supp\F \subseteq A \quad\Longleftrightarrow\quad
  A^c \subseteq X \setminus \supp\F \quad\Longleftrightarrow\quad
  A^c \subseteq \Int(X \setminus \supp\F). \]
Since the interior of the complement of~$\supp\F$ can be characterized as the
largest open subset of~$X$ on which the internal statement~``$\F$ is a
singleton'' holds (Remark~\ref{rem:support-sheaf-of-sets}), the condition on
the support is fulfilled if and only if
\[ \Sh(X) \models (A^c \Rightarrow \speak{$\F$ is a singleton}). \]
We thus have to show that this internal condition is equivalent to~$\F$ being
a~$\Box$-sheaf. For the ``if'' direction, assume~$A^c$. Then the empty subset~$S
\subseteq \F$ trivially verifies the condition that~$\Box(\speak{$S$ is a
singleton})$. There thus exists an element~$x\?\F$ (such that~$\Box(x \in S)$).
If we're given a further element~$y\?\F$, it trivially holds that~$\Box(x =
y)$. By~$\Box$-separatedness, it thus follows that~$x = y$. Thus~$\F$ is the
singleton~$\{x\}$. The proof of the ``only if'' direction is similar.

The second statement says that internally, sheafifying a~$\Box$-sheaf with
respect to the modal operator~$\Box$ and then forgetting that the result is
a~$\Box$-sheaf amounts to doing nothing. This is obvious.
\end{proof}

\subsection{Closed subschemes} Let~$X$ be a ringed space. Recall
that an ideal sheaf~$\I \subseteq \O_X$ defines a closed subset~$V(\I) = \{ x
\in X \,|\, \I_x \neq (1) \subseteq \O_{X,x} \}$, a sheaf of
rings~$\O_X/\I$, and a ringed space~$(V(\I), \O_{V(\I)})$ where~$\O_{V(\I)}$ is
the pullback of~$\O_X/\I$ to~$V(\I)$. In the internal universe, we can
reify~$V(\I)$ by giving a modal operator~$\Box$ such that externally, the
subspace~$X_\Box$ coincides with~$V(\I)$.

\begin{prop}\label{prop:basics-closed-subspace}
Let~$X$ be a ringed space. Let~$\I \subseteq \O_X$ be an ideal
sheaf. Then:
\begin{enumerate}
\item The subspace of~$X$ associated to the modal operator~$\Box$ defined
by~$\Box\varphi \defequiv (\varphi \vee (1 \in \I))$ is~$V(\I)$.
\item The support of~$\O_X/\I$ is exactly~$V(\I)$.
\item The canonical morphism~$i : V(\I) \to X$ is a closed immersion
of ringed spaces.
\end{enumerate}\end{prop}
\begin{proof}For any open subset~$U \subseteq X$, it holds that~$U \models 1
\in \I$ if and only if~$U \subseteq D(\I) = X \setminus V(\I)$. Thus~$D(\I)$
can be characterized as the largest open subset on which~``$1 \in \I$'' holds.
According to Table~\ref{table:nuclei} on page~\pageref{table:nuclei}, the
stated modal operator thus defines the subspace~$D(\I)^c$, \ie~$V(\I)$.

For the second statement, note that since~$\O_X/\I$ is a sheaf of rings, its
support is closed. Therefore the largest open subset of~$X$ where the internal
statement~``$\O_X/\I = 0$'' holds is the complement of the support
(Proposition~\ref{prop:characterization-support}). Since~$D(\I)$ is the largest
open subset where the internal statement~``$\I = (1)$'' holds, it suffices to
show that internally,~$\O_X/\I = 0$ if and only if~$\I = (1)$. This is obvious.

The topological part of the third statement is clear. For the ring-theoretic
part, we have to show that the canonical ring homomorphism~$\O_X \to i_*
\O_{V(\I)}$, that is the canonical projection~$\O_X \to \O_X/(\I)$, is an
epimorphism of sheaves. This is obvious.
\end{proof}

By Lemma~\ref{lemma:essim-closed-immersion}, the sheaf~$\O_X/\I$ is
thus a~$\Box$-sheaf from the internal perspective.

\begin{prop}Let~$X$ be a locally ringed space. Let~$\I \subseteq \O_X$ be an
ideal sheaf. Then the ringed space~$(V(\I), \O_{V(\I)})$ is locally
ringed as well.\end{prop}
\begin{proof}We have to show that
\[ \Sh(V(\I)) \models \speak{$\O_{V(\I)}$ is a local ring}. \]
By Theorem~\ref{thm:box-translation-semantically}, this is equivalent to
\[ \Sh(X) \models (\speak{$\O_X/\I$ is a local ring})^\Box, \]
where~$\Box$ is the modal operator given by~$\Box\varphi \defequiv (\varphi \vee
(1 \in \I))$. We therefore have to give an intuitionistic proof of the fact
\[ \forall x,y\?\O_X/\I\_ \speak{$x+y$ \inv} \Longrightarrow
  \Box(\speak{$x$ \inv} \vee \speak{$y$ \inv}). \]
So let~$x = [s], y = [t] \? \O_X/\I$ such that~$x + y$ is invertible
in~$\O_X/\I$. This means that there exists~$u\?\O_X$ and~$v\?\I$ such that~$us
+ ut + v = 1$ in~$\O_X$. Since~$\O_X$ is a local ring, it follows
that~$us$,~$ut$, or~$v$ is invertible. In the first two cases, it follows
that~$x$ respectively~$y$ are invertible in~$\O_X/\I$. In the third case, it
follows that~$1 \in \I$ and thus any boxed statement is trivially true.
\end{proof}

If~$X$ is a scheme and~$\I \subseteq \O_X$ is an ideal sheaf, it is well-known
that the locally ringed space~$V(\I)$ is a scheme if and only if~$\I$ is
quasicoherent. We cannot give an internal proof of this fact since we lack an
internal characterization of being a scheme.

\begin{lemma}\label{lemma:closed-subspace-reduced}
Let~$X$ be a scheme (or a ringed space). Let~$\I \subseteq \O_X$ be
an ideal sheaf. The ringed space~$V(\I)$ is reduced if and only if, from the
internal perspective of~$\Sh(X)$, the ideal~$\I$ is a radical ideal.\end{lemma}
\begin{proof}The following chain of equivalences holds:
\begin{align*}
  &\ \Sh(V(\I)) \models \speak{$\O_{V(\I)}$ is a reduced ring} \\
  \Longleftrightarrow&\
    \Sh(V(\I)) \models \bigwedge_{n \geq 0} \forall s\?\O_{V(\I)}\_
      s^n = 0 \Longrightarrow s = 0 \\
  \Longleftrightarrow&\
    \Sh(X) \models \bigl(\bigwedge_{n \geq 0} \forall s\?\O_X/\I\_ s^n = 0
    \Rightarrow s = 0\bigr)^\Box \\
  \Longleftrightarrow&\
    \Sh(X) \models \bigwedge_{n \geq 0} \forall s\?\O_X/\I\_ s^n = 0 \Rightarrow \Box(s = 0) \\
  \Longleftrightarrow&\
    \Sh(X) \models \bigwedge_{n \geq 0} \forall s\?\O_X\_ s^n \in \I
    \Rightarrow \Box(s \in \I) \\
  \Longleftrightarrow&\
    \Sh(X) \models \bigwedge_{n \geq 0} \forall s\?\O_X\_ s^n \in \I
    \Rightarrow s \in \I \\
  \Longleftrightarrow&\
    \Sh(X) \models \speak{$\I$ is a radical ideal}
\end{align*}
In the second-to-last step, we used that~$\Box(s \in \I) \equiv ((s \in \I) \vee
(1 \in \I))$ implies~$s \in \I$. This is trivial in both cases of the
disjunction.
\end{proof}

\begin{lemma}\label{lemma:reduced-subspace}
Let~$X$ be a scheme (or a ringed space).
\begin{enumerate}
\item There exists a reduced closed sub-ringed space~$X_\mathrm{red}
\hookrightarrow X$ having the same underlying topological space as~$X$ with
the following universal property: Any morphism~$Y \to X$
of (ringed or locally ringed) spaces such that~$Y$ is reduced factors uniquely
over the closed immersion~$X_\mathrm{red} \hookrightarrow X$.
\item Let~$A \subseteq X$ be a closed subset. Then there exists a structure of
a reduced closed ringed subspace on~$A$ with a similar universal
property.
\end{enumerate}
\end{lemma}
\begin{proof}For the first statement, let~$\N \subseteq \O_X$ be the nilradical
of~$\O_X$. This can internally be simply defined by~$\N \defeq \sqrt{(0)} = \{
s\?\O_X \,|\, \bigvee_{n \geq 0} s^n = 0 \}$. Define~$X_\mathrm{red}$ as the closed
subspace associated to this ideal sheaf. This ringed space is reduced by the
previous lemma. If~$X$ is a scheme, then quasicoherence of~$\N$ (which is
necessary and sufficient for~$X_\mathrm{red}$ to be a scheme) can be shown
internally (Example~\ref{ex:radical-qcoh}).
The proof of the universal property can also be done in the
internal language, by using that the well-known fact of locale theory that the
category of locales over~$X$ is equivalent to internal locales in~$\Sh(X)$; but
we do not want to discuss this further.

For the second statement, internally define the ideal~$\I \defeq \sqrt{\{ s\?\O_X \,|\, s = 0 \vee
A^c \}} \subseteq \O_X$. Then~$1 \in \I$ if and only if~$A^c$, thus by
Proposition~\ref{prop:basics-closed-subspace} the closed ringed subspace defined
by~$\I$ has~$A$ as underlying topological space. It is reduced since~$\I$ is a
radical ideal. \XXXh{is~$\I$ quasicoherent, if~$X$ is a scheme?}\end{proof}

\begin{lemma}Let~$X$ be a scheme of dimension~$\leq n$. Let~$V(\I)
\hookrightarrow X$ be a closed subscheme which is locally cut out by a regular
equation. Then~$\dim V(\I) \leq n-1$.\end{lemma}
\begin{proof}By Proposition~\ref{prop:dimension-scheme-ox}, it suffices to give
an intuitionistic proof of the following fact of dimension theory: Let~$A$ be
an arbitrary ring of dimension~$\leq n$. Let~$I = (s) \subseteq A$ be an ideal
which is generated by a regular element~$s\?A$. Then the~$\Box$-translation
of~``$A/I$ is of dimension~$\leq n-1$'' holds. In fact, we can show that~$A/I$
really is of dimension~$\leq n-1$; since no implication signs occur in a formal rendering of ``being of dimension~$\leq n-1$'',
Lemma~\ref{lemma:open-stalk} is applicable and implies
that this a stronger statement.

For this, let a sequence~$([a_0],\ldots,[a_{n-1}])$ of elements in~$A/I$ be
given. We can lift and extend this sequence to the
sequence~$(a_0,\ldots,a_{n-1},s)$ of elements of~$A$. Since~$\dim A \leq n$,
there exists a complementary sequence~$(b_0,\ldots,b_{n-1},b_n)$.
Since~$s$ is regular, the inclusion~$\sqrt{(s b_n)} \subseteq \sqrt{(0)}$
given by the definition of complementarity implies that~$b_n$ is nilpotent.
Thus we have that~$\sqrt{(a_{n-1}b_{n-1})} \subseteq \sqrt{(s,b_n)} =
\sqrt{(s)}$ in~$A$, which translates to~$\sqrt{([a_{n-1}] [b_{n-1}])} \subseteq
\sqrt{(0)}$ in~$A/I$.  Therefore~$([b_0],\ldots,[b_{n-1}])$ is a complementary
sequence to~$([a_0],\ldots,[a_{n-1}])$ in~$A/I$.
\end{proof}

\begin{lemma}\label{lemma:dim-closed-subscheme}
Let~$X$ be a scheme. Let~$\I$ be a sheaf of~$\O_X$-modules. Then:
\[ \dim V(\I) \leq n \quad\Longleftrightarrow\quad
  \Sh(X) \models \speak{$\O_X/\I$ is of Krull dimension~$\leq n$}. \]
\end{lemma}
\begin{proof}By Proposition~\ref{prop:dimension-scheme-ox}, the condition~$\dim
V(\I) \leq n$ is equivalent to
\[ \Sh(V(\I)) \models \speak{$\O_{V(\I)}$ is of Krull dimension~$\leq n$}. \]
By Theorem~\ref{thm:box-translation-semantically} this is equivalent to
\[ \Sh(X) \models (\speak{$\O_X/\I$ is of Krull dimension~$\leq n$})^\Box, \]
where~$\Box$ is the modal operator given by~$\Box\varphi \defeq (\varphi \vee
(1\in\I))$. The claimed equivalence then follows by
Lemma~\ref{lemma:open-stalk} (for~``$\Leftarrow$'') and by direct inspection
similar to the proof of Lemma~\ref{lemma:pushforward-finite-type}
(for~``$\Rightarrow$'').
\end{proof}

\begin{itemize}
\item open subschemes
\item Koszul resolution
\end{itemize}


\section{Transfer principles}

Let~$M$ be an~$A$-module. A natural question is how properties of~$M$
relate to properties of the induced quasicoherent sheaf~$M^\sim$
on~$\Spec A$. For instance it is well-known that
\begin{itemize}
\item $M$ is finitely generated iff~$M^\sim$ is of finite type,
\item $M$ is flat over~$A$ iff~$M^\sim$ is flat over~$\O_{\Spec A}$, and
\item $M$ is torsion iff~$M^\sim$ is a torsion sheaf.
\end{itemize}
Using the internal language of the little Zariski topos of~$\Spec A$, we can
give a simple, conceptual, and uniform explanation of these equivalences.
Namely, from the internal point of view, the module~$M^\sim$ is obtained from
the constant sheaf~$\ul{M}$ by localizing at the \emph{generic filter}, a
particular multiplicative subset to be introduced below, and the set~$M$ and
the sheaf~$\ul{M}$ share the same properties (by
Lemma~\ref{lemma:properties-of-constant-sheaves} below).

This makes it obvious that, for instance, properties which are stable under
localization pass from~$M$ to~$M^\sim$.


\subsection{Internal properties of constant sheaves}

\begin{lemma}\label{lemma:properties-of-constant-sheaves}Let~$\varphi$ be a
formula in which arbitrary sets and elements may occur as parameters. Let~$X$
be a topological space and let~$U \subseteq X$ be an open subset. Then
\[ U \models \varphi \quad\text{iff}\quad (\text{$U$ inhabited} \Rightarrow
\varphi). \]
\end{lemma}
Note that we are abusing notation on the left hand side: The parameters
of~$\varphi$, which are sets and elements, must be read as the induced constant
sheaves and constant functions (sections of that sheaves).
Unbounded quantifiers have to be read as ranging only over locally constant
sheaves, not all sheaves.
\begin{proof}By induction on the structure of~$\varphi$. By way of example, we
give the argument in the case that~$\varphi \equiv (a = b)$, where~$a$ and~$b$ are
elements of some set~$M$. Then~$U \models \varphi$ means by definition that the
constant functions~$U \to M$ with value~$a$ respectively~$b$ coincide. This is
equivalent to saying that~$a$ and~$b$ coincide if~$U$ is inhabited.
\end{proof}

The lemma in particular implies that constant sheaves enjoy several
classical properties from the internal point of view, even though the internal
language only supports intuitionistic reasoning in general. For instance, for a
constant sheaf~$\ul{M}$ it holds that
\[ \Sh(X) \models \forall x,y\?\ul{M}\_ \neg\neg(x = y) \Rightarrow x = y \]
and even
\[ \Sh(X) \models \forall x,y\?\ul{M}\_ x = y \vee x \neq y. \]


\subsection{The generic filter}
\label{sect:generic-filter}

Let~$A$ be a ring.

\begin{defn}\label{defn:filter}
A \emph{filter} of~$A$ is a subset~$F \subseteq A$ such that
\begin{itemize}
\item $0 \not\in F$,
\item $1 \in F$,
\item $x + y \in F \Longrightarrow (x \in F) \vee (y \in F)$, and
\item $xy \in F \Longleftrightarrow (x \in F) \wedge (y \in F)$
\end{itemize}
for all~$x,y \? A$.
\end{defn}

In classical logic, the complement of a prime ideal is a filter and
furthermore every filter is of such a form. In constructive mathematics however,
it is useful to axiomatize complements of prime ideals directly, avoiding
negations. Intuitionistically, since De Morgan's law~$\neg(\alpha \wedge \beta)
\Rightarrow \neg\alpha \vee \neg\beta$ is not available, one can neither show
that the complement of a prime ideal is a filter nor that the complement of a
filter is a prime ideal.

A filter is in particular a multiplicative subset. Inverting the
elements of a filter results in a local ring, while intuitionistically
the localization of a ring at a prime ideal cannot in general be verified to be local.

\begin{defn}The \emph{generic filter}~$\F$ is the subsheaf of~$\ul{A}$
on~$\Spec A$ given by
\[ \Gamma(U, \F) \defeq \{ f : U \to A \,|\,
  \text{$f(\ppp) \not\in \ppp$ for all $\ppp \in U$} \}. \]
\end{defn}

\begin{prop}\label{prop:basics-univ-filter}\ \begin{enumerate}
\item Let~$f \in A$ and~$x \in A$. Then~$D(f) \models x \in \F$ if and only
if~$f \in \sqrt{(x)}$.
% If the numbering is changed, also update reference below.
\item The stalk~$\F_\ppp$ at a point~$\ppp \in \Spec A$
is in canonical bijection with~$A \setminus \ppp$.
\item From the internal point of view of~$\Sh(\Spec A)$, the generic
filter is indeed a filter of~$\ul{A}$.
\end{enumerate}
\end{prop}
\begin{proof}By definition~$D(f) \models x \in \F$ means that~$x \not\in \ppp$
for all prime ideals~$\ppp$ with~$f \not\in \ppp$. This is well-known to be
equivalent to~$f \in \sqrt{(x)}$.

For the claim about stalks, note that the canonical map~$\F_\ppp \to A
\setminus \ppp$ sending a germ~$[f]$ to~$f(\ppp)$ is invertible with inverse
being the map which sends an element~$x \not\in \ppp$ to the germ of the
constant function with value~$x$ (defined on~$D(x)$).

Regarding the third statement we only verify the axiom regarding sums, the
other verifications being easier. Interpreting this axiom with the Kripke--Joyal
semantics and restricting without loss of generality to open subsets where
given locally constant functions are constant, let elements~$x,y \in A$ be
given such that~$D(f) \models x+y \in \F$. By the first statement~$f \in
\sqrt{(x+y)}$. Therefore~$D(f) \subseteq D(x) \cup D(y)$, and on~$D(x)$ it
holds that~$x \in \F$ and on~$D(y)$ it holds that~$y \in \F$.
\end{proof}

The significance of the generic filter is given by the following proposition.
\begin{prop}\label{prop:tilde-construction-internally}
From the internal point of view of~$\Sh(\Spec A)$,
\begin{enumerate}
\item the structure
sheaf~$\O_{\Spec A}$ is the localization of the constant sheaf~$\ul{A}$ at the
generic filter:~$\O_{\Spec A} = \ul{A}[\F^{-1}]$, and
\item the quasicoherent sheaf of modules~$M^\sim$ associated to
an~$A$-module~$M$ is the localization of the constant sheaf~$\ul{M}$ at the
generic filter.
\end{enumerate}\end{prop}
\begin{proof}Ignoring the ring respectively module structure, the second
statement is more general; therefore we prove this one. We didn't discuss the
case of quotients in Section~\ref{sect:internal-constructions}. However it
should be perspicuous that the interpretation of~$\ul{M}[\F^{-1}]$ is defined
as the colimit of~$\E \twoheadrightarrow \ul{M} \times \F$, taken in the
category of sheaves on~$\Spec A$, where~$\E$ is the subsheaf of~$\F \times
(\ul{M} \times \F) \times (\ul{M} \times \F)$ given by~$\E(U) \defeq \{
(s,(x,t),(y,u)) \,|\, sux = sty \}$.

This colimit can be obtained as the sheafification of the similarly defined
presheaf colimit~$\E' \twoheadrightarrow \ul{M}_\mathrm{pre} \times \F$,
where~$\ul{M}_\mathrm{pre}$ is the constant \emph{presheaf} associated to~$M$.
On an open subset~$U$ this presheaf colimit is simply the
localization~$\Gamma(U, \ul{M}_\mathrm{pre})[\Gamma(U, \F)^{-1}] = M[\Gamma(U,
\F)^{-1}]$. In the special case that~$U = D(f)$ is a standard open subset,
Proposition~\ref{prop:basics-univ-filter}(a) shows that this module is
canonically isomorphic to~$M[f^{-1}]$. The quasicoherent sheaf~$M^\sim$ of
modules admits the same description.
\end{proof}
% Here we use the following technical lemma:
% Let E be a presheaf which, restricted to a basis of the topology,
% is even a sheaf. Then its sheafification coincides with the extension of
% the restricted sheaf.

Recognizing~$\O_{\Spec A}$ as a localization of~$\ul{A}$ fits nicely into the
following abstract algebraic motivation for schemes: Does the ring~$A$ admit a
\emph{universal localization}, \ie a homomorphism~$A \to A'$ into a local ring
such that every homomorphism~$A \to B$ into a local ring factors via a local
map over~$A \to A'$? Intuitively speaking, can we localize a ring at all prime
ideals at once, or equivalently at all filters at once? The answer is \emph{no}
in general,\footnote{Assume that the universal localization~$A'$ of a ring~$A$ exists as an ordinary ring in~$\Set$. Then any
two prime ideals~$\ppp$ and~$\qqq$ of~$A$ are equal: Let~$s \not\in \ppp$.
Since~$s$ is invertible in the local ring~$A_\ppp$ and the map~$A' \to A_\ppp$
induced by~$A \to A_\ppp$ is local, it is also invertible in~$A'$. Therefore the image of~$s$ in~$A_\qqq$ is
invertible as well. Thus~$s \not\in \qqq$.} but always \emph{yes} if we are
willing to change the topos in which we look for a solution: The universal
localization of~$A$ is given by the ring~$\O_{\Spec A}$ in the topos~$\Sh(\Spec
A)$; this ring is constructed by localizing~$\ul{A}$ at the generic
filter, a filter which exists in~$\Sh(\Spec A)$ but not in~$\Set$.

We expand on this point of view in Section~\ref{sect:relative-spectrum} on the
relative spectrum.

For transferring properties of~$M^\sim$ to~$M$, the following metatheorem is
crucial.
\begin{prop}\label{prop:metaproperty-of-the-generic-filter}
Let~$\I$ be an ideal in~$\ul{A}$ such that, for all inhabited
open subsets~$U \subseteq \Spec A$ and elements~$x \in A$, the set~$\Gamma(X,\I)$
contains the constant function with value~$x$ if~$\Gamma(U,\I)$ does. Then
\[ D(f) \models \speak{$\I \cap \F$ is inhabited}
  \quad\text{implies}\quad
  \text{for some~$n \geq 0$, $D(f) \models f^n \in \I$.} \]
\end{prop}

Lemma~\ref{lemma:properties-of-constant-sheaves} gives a simple and purely
syntactical criterion for the hypothesis on~$\I$: It suffices for~$\I$ to be
internally defined by an expression of the form~$\{ a\?\ul{A} \,|\, \varphi(a)
\}$, where~$\varphi$ is a formula which refers only to constant sheaves.

The metatheorem reflects the following well-known fact of classical
ring theory: If an ideal meets every filter (that is, the complement of every
prime ideal), it is the unit ideal. In this formulation the statement can't be
proven intuitionistically; the occurence of \emph{every filter} has to be
replaced by \emph{generic filter}. Intuitively, the generic filter is a
reification of the abstract idea of an ``arbitrary filter'', a filter about
which nothing is known except that it satisfies the filter axioms.

\begin{proof}
Let~$D(f) \models \speak{$\I \cap \F$ is inhabited}$. Then there
exists an open cover~$D(f) = \bigcup_i D(f_i)$ and elements~$x_i \in A$ such
that~$D(f_i) \models x_i \in \F$ and~$D(f_i) \models x_i \in \I$. By
Proposition~\ref{prop:basics-univ-filter} we have that~$f_i \in \sqrt{(x_i)}$
and therefore~$D(f_i) \models f_i^{m_i} \in \I$ for some~$m_i \geq 0$. We may
assume that all the~$D(f_i)$ are inhabited and that the exponents~$m_i$ are all
equal to some number~$m$. The assumption on~$\I$ implies~$D(f) \models f_i^m
\in \I$ for all~$i$. By a standard argument we can write~$f^n = \sum_i a_i
f_i^m$ for some coefficients~$a_i$; thus~$D(f) \models f^n \in \I$.
\end{proof}

\begin{rem}The stronger statement
\[ D(f) \models (\speak{$\I \cap \F$ is inhabited} \Rightarrow \bigvee_{n \geq
0} (f^n \in \I)) \]
does not hold in general. Indeed, consider the example~$f \defeq 1$ and~$\I \defeq
\brak{(g)} \defeq \brak{\{ a\?\ul{A} \,|\, \exists b\?\ul{A}\_ a = bg \}}$,
where~$g$ is a fixed element of~$A$ which is not nilpotent and not invertible.
Since~$D(g) \models g \in \I \cap \F$, the stronger statement would imply~$D(g)
\models 1 \in \I$. By Lemma~\ref{lemma:properties-of-constant-sheaves}, this is
equivalent to~$g$ being invertible in~$A$.
\end{rem}

\begin{rem}Recall from Proposition~\ref{prop:kx-internally} that the
sheaf~$\K_{\Spec A}$ of rational functions can internally by obtained by
localizing~$\O_{\Spec A}$ at the set of regular elements. Since~$\O_{\Spec A}$
is itself a localization, the sheaf~$\K_{\Spec A}$ is therefore obtained by a
two-step process. It can also be obtained in a single step by
localizing~$\ul{A}$ at~$\T$, where~$\T$ is the subsheaf of~$\ul{A}$ defined
by
\[ \Gamma(U,\T) = \{ f : U \to A \,|\, \text{$f(\ppp)$ is regular in~$A_\ppp$
for all~$\ppp \in U$} \}. \]
This subsheaf is characterized by the property that, for all~$f \in A$ and~$x
\in A$,~$D(f) \models x \in \T$ if and only if~$x$ is regular in~$A[f^{-1}]$.
\end{rem}


\subsection{Internal proofs of common lemmas}
\label{sect:common-lemmas-transfer-principles}

\begin{lemma}Let~$A$ be a ring. Then~$A$ is reduced if and only if the
scheme~$\Spec A$ is reduced.\end{lemma}
\begin{proof}By Proposition~\ref{prop:reduced-ring} the scheme~$\Spec A$ is
reduced if and only if~$\O_{\Spec A}$ is a reduced ring
from the internal point of view of~$\Sh(\Spec A)$.

For the ``only if'' direction assume that~$A$ is reduced. Then~$\ul{A}$ is
reduced as well, by Lemma~\ref{lemma:properties-of-constant-sheaves}. Since
localizations of reduced rings are reduced (and this fact has an intuitionistic
proof), in particular~$\O_{\Spec A} = \ul{A}[\F^{-1}]$ is reduced.

For the ``if'' direction let~$x \in A$ be an element such that~$x^n = 0$.
Since~$\O_{\Spec A} = \ul{A}[\F^{-1}]$ is reduced from the internal point of
view, the element~$x$ is zero in that ring, that is
\[ \Sh(\Spec A) \models \exists s\?\F\_ sx = 0. \]
Therefore the ideal internally defined by
\[ \I \defeq \{ a\?\ul{A} \,|\, ax = 0 \} \]
meets the generic filter. By
Proposition~\ref{prop:metaproperty-of-the-generic-filter} it follows
that~$\Sh(\Spec A) \models 1 \in \I$. By
Lemma~\ref{lemma:properties-of-constant-sheaves} this is equivalent to~$1 \cdot
x = 0$ as elements of~$A$.
\end{proof}

Note that the ``if'' direction also admits a shorter proof, by simply
considering the Kripke--Joyal interpretation of~$\Sh(\Spec A) \models
\speak{$\O_{\Spec A}$ is reduced}$ and using~$\Gamma(\Spec A, \O_{\Spec A})
\cong A$. We included the given proof to give a simple example of the mixed
internal/external reasoning with the generic filter. In a similar way we could
reprove Lemma~\ref{lemma:regular-affine}, that is the statement that a ring
element~$f \in A$ is regular in~$A$ if and only if, from the internal point of
view, it is regular in~$\O_{\Spec A}$.

\begin{lemma}Let~$M$ be an~$A$-module. Then~$M^\sim$ is of finite type if and
only if~$M$ is finitely generated.\end{lemma}
\begin{proof}First assume that~$M$ is finitely generated over~$A$.
Then~$\ul{M}$ is finitely generated over~$\ul{A}$, by
Lemma~\ref{lemma:properties-of-constant-sheaves}. Since localizations of
finitely generated modules are finitely generated (over the localized ring),
the module~$M^\sim = \ul{M}[\F^{-1}]$ is finitely generated from the internal
point of view. By Proposition~\ref{prop:finite-type-and-co} this means
that~$M^\sim$ is of finite type from the external point of view.

For the ``only if`` direction, we assume that~$M^\sim$ is finitely generated
over~$\O_{\Spec A}$ from the internal point of view and have to verify that~$M$
is finitely generated over~$A$. So it holds that
\[ \Sh(\Spec A) \models \bigvee_{n \geq 0}
  \exists x_1,\ldots,x_n\?\ul{M}[\F^{-1}]\_
  \speak{the $x_i$ span~$\ul{M}[\F^{-1}]$ over~$\ul{A}[\F^{-1}]$}. \]
Since multiplying a generating family by an unit results again in a generating
family, we have in fact that
\[ \Sh(\Spec A) \models \bigvee_{n \geq 0}
  \exists x_1,\ldots,x_n\?\ul{M}\_
  \speak{the $x_i/1$ span~$\ul{M}[\F^{-1}]$ over~$\ul{A}[\F^{-1}]$} \]
or equivalently
\[ \Sh(\Spec A) \models \bigvee_{n \geq 0, x_1,\ldots,x_n \in M}
  \speak{the $x_i/1$ span~$\ul{M}[\F^{-1}]$ over~$\ul{A}[\F^{-1}]$}. \]
Since this is a directed disjunction and~$\Spec A$ is quasicompact,
Proposition~\ref{prop:quasicompact-meta} is applicable and shows that there
exists a natural number~$n \geq 0$ and elements~$x_1,\ldots,x_n \in M$ such
that
\[ \Sh(\Spec A) \models \speak{the $x_i/1$ span~$\ul{M}[\F^{-1}]$
over~$\ul{A}[\F^{-1}]$}. \]
We claim that these~$x_i$ also span~$M$ as an~$A$-module. So let~$x \in M$ be
arbitrary. By elementary linear algebra we can deduce that
\[ \Sh(\Spec A) \models \exists s\in\F\_ \exists a_1,\ldots,a_n\?\ul{A}\_
  sx = \sum_i a_i x_i. \]
Therefore the ideal internally defined by
\[ \I \defeq \{ s\?\ul{A} \,|\, \exists a_1,\ldots,a_n\?\ul{A}\_
  sx = \textstyle\sum_i a_i x_i \} \]
meets the generic filter.
Proposition~\ref{prop:metaproperty-of-the-generic-filter} shows that~$\Sh(\Spec A)
\models 1 \in \I$, that is~$x$ is an element of the~$A$-span of the~$x_i$.
\end{proof}

\begin{rem}If~$M^\sim$ can be generated by~$\leq n$ elements over~$\O_{\Spec
A}$ from the internal point of view, it needn't be the case that~$M$ can be
generated by~$\leq n$ elements over~$A$. It is instructive to see where the
appropriately modified version of the above proof fails: In this case we still
have
\[ \Sh(\Spec A) \models \bigvee_{x_1,\ldots,x_n \in M}
  \speak{the $x_i/1$ span~$\ul{M}[\F^{-1}]$ over~$\ul{A}[\F^{-1}]$}, \]
but this disjunction is no longer directed.
\end{rem}

\begin{lemma}Let~$X$ be a scheme. Then kernels and cokernels of morphisms
between quasicoherent~$\O_X$-modules are quasicoherent.\end{lemma}

\begin{proof}We may assume that~$X = \Spec A$ is affine. A morphism between
quasicoherent~$\O_X$-modules is of the form~$\ul{\varphi}[\F^{-1}] :
\ul{M}[\F^{-1}] \to \ul{N}[\F^{-1}]$, where~$\varphi : M \to N$ is a linear map
between~$A$-modules. Since taking constant shaves and localization are exact,
we have the chain of isomorphisms
\[ (\ul{\Kernel(\varphi)})[\F^{-1}] =
  (\Kernel(\ul{\varphi}))[\F^{-1}] =
  \Kernel(\ul{\varphi}[\F^{-1}]), \]
and similarly for the cokernel.
\end{proof}


\subsection{An application to constructive mathematics}
\label{sect:eliminating-prime-ideals}

The generic filter has a practical application in constructive mathematics.
Recall that intuitionistically prime and maximal ideals don't work very well,
since one often needs the axiom of choice or related set-theoretical principles
in dealing with them. This is unfortunate, since prime and maximal ideals are
very useful in some situations. For example:
\begin{itemize}
\item To verify that a ring element is nilpotent, it suffices to verify that it
is an element of every prime ideal. For instance, this is calculationally simpler
when proving that the coefficients of a nilpotent polynomial are
themselves nilpotent.
\item To verify that there is an relation of the form~$1 = p_1f_1 + \cdots +
p_mf_m$ among polynomials~$f_1,\ldots,f_m \in K[X_1,\ldots,X_n]$ where~$K$ is
an algebraically closed field, it suffices to show that the~$f_i$ don't have a
common zero.
\end{itemize}

One could of course simply switch to classical logic in this case. However this
might not be desirable, as a constructive proof would contain more information:
For instance, if we have classically proven that an element~$x$ is an element
of every prime ideal, then we know that some power~$x^n$ is zero. But from such
a proof we can't directly read off any upper bound on~$n$.

There is a way to combine some of the powerful tools of classical ring theory
with the advantages that constructive reasoning provides. Namely we can devise
a language in which we can usefully talk about prime ideals, but which
substitutes all non-constructive arguments by constructive arguments ``behind
the scenes''. The key idea is to substitute the phrase ``for all prime ideals''
(or equivalently ``for all filters'') by ``for the generic filter''.

This was already explored by Coquand, Coste, Lombardi, Roy, and
others under the theme of \emph{dynamical methods in
algebra}~\cite{clr:dynamicalmethod,cl:logical}. Here we show how one can use
the generic filter, as reified by a sheaf in the little Zariski topos, to
achieve similar effects.

\begin{prop}Let~$M$ and~$N$ be~$A$-modules. Let~$\alpha : M \to N$ be a linear
map. The interpretations of the statements in the second column of
Table~\ref{table:generic-filter-statements} in the internal language
of~$\Sh(\Spec A)$ are intuitionistically equivalent to the statements given in
the third column.\end{prop}
\begin{proof}To demonstrate the technique we verify the first and the last claim.
To make the following proofs constructive we would have to define~$\Spec A$
and its sheaf topos in a constructive fashion, not using prime ideals. This can
be done, by constructing~$\Spec A$ as a locale instead of a topological space
(see for instance Section~\ref{sect:spectrum-as-a-locale}
and~\cite[p.~743f.]{wraith:generic-galois-theory}),
but we won't discuss details here.

The interpretation of~$\Sh(\Spec A) \models x \not\in \F$ by the Kripke--Joyal
semantics is that~$D(f) \models x \in \F$ implies~$D(f) = \emptyset$ for
all~$f \in A$. By Proposition~\ref{prop:basics-univ-filter}(a) this is
equivalent to
\[ \forall f \in A\_ f \in \sqrt{(x)} \Rightarrow f \in \sqrt{(0)}, \]
that is the statement that~$x$ is nilpotent in~$A$.

Assume that~$\alpha : M \to N$ is surjective. By
Lemma~\ref{lemma:properties-of-constant-sheaves} the induced map~$\ul{M}
\to \ul{N}$ is surjective from the internal point of view. Since localization
preserves surjectivity, also the map~$\ul{M}[\F^{-1}] \to \ul{N}[\F^{-1}]$ is
surjective.

Conversely, assume that~$\ul{M}[\F^{-1}] \to \ul{N}[\F^{-1}]$ is surjective
from the internal point of view. To verify that~$\alpha : M \to N$ is
surjective, let~$y \in N$. The assumption implies that the ideal internally
defined by
\[ \I \defeq \{ s\?\ul{A} \,|\, \exists x\?\ul{A}\_ sy = \ul{\alpha}(x) \} \]
meets the generic filter. By
Proposition~\ref{prop:metaproperty-of-the-generic-filter} this implies
that~$\Sh(\Spec A) \models 1 \in \I$, that is there exists an element~$x \in A$
such that~$\alpha(x) = y$.
\end{proof}

\begin{table}
  \centering
  \renewcommand{\arraystretch}{1.3}
  \small
  % Change wording in proposition above if order changes.
  \begin{tabular}{lll}
    \toprule
    Statement & constructive substitution & meaning \\\midrule
    $x \in \ppp$ for all~$\ppp$. &
    $x \not\in \F$. &
    $x$ is nilpotent. \\
    $x \in \ppp$ for all~$\ppp$ such that~$y \in \ppp$. &
    $x \in \F \Rightarrow y \in \F$. &
    $x \in \sqrt{(y)}$. \\
    $x$ is regular in all stalks~$A_\ppp$. &
    $x$ is regular in~$\ul{A}[\F^{-1}]$. &
    $x$ is regular in~$A$. \\
    The stalks~$A_\ppp$ are reduced. &
    $\ul{A}[\F^{-1}]$ is reduced. &
    $A$ is reduced. \\
    The stalks~$M_\ppp$ vanish. &
    $\ul{M}[\F^{-1}] = 0$. &
    $M = 0$. \\
    The stalks~$M_\ppp$ are fin.\@ gen.\@ over~$A_\ppp$. &
    $\ul{M}[\F^{-1}]$ is fin.\@ gen.\@ over
    $\ul{A}[\F^{-1}]$. &
    $M$ is fin.\@ gen.\@ over~$A$. \\
    The stalks~$M_\ppp$ are flat over~$A_\ppp$. &
    $\ul{M}[\F^{-1}]$ is flat over~$\ul{A}[\F^{-1}]$. &
    $M$ is flat over~$A$. \\
    The maps~$M_\ppp \to N_\ppp$ are injective. &
    $\ul{M}[\F^{-1}] \to \ul{N}[\F^{-1}]$ is injective. &
    $M \to N$ is injective. \\
    The maps~$M_\ppp \to N_\ppp$ are surjective. &
    $\ul{M}[\F^{-1}] \to \ul{N}[\F^{-1}]$ is surjective. &
    $M \to N$ is surjective. \\
    \bottomrule
  \end{tabular}
  \vspace{0.5em}

  \caption{\label{table:generic-filter-statements}Substituting the use of prime
  ideals by the generic filter.}
\end{table}
\XXX{Discuss "finitely generated" example in more detail.}

The sheaf-theoretical approach using the generic filter is different from the
dynamical methods in the following aspect. We have to reword classical
arguments using (the generic) filter instead of (the generic) prime ideal.
Depending on the situation this might be a nuisance. One might be tempted to
employ the complement of the generic filter, but this is only an ideal, not a
prime ideal from the internal point of view.\footnote{One can check that the
complement of~$\F$ in~$\ul{A}$ is the subsheaf~$\P$ defined by~$\Gamma(U, \P)
\defeq \{ f : U \to A \,|\, \text{$f(\ppp) \in \ppp$ for all~$\ppp \in U$} \}$
and that~$D(f) \models x \in \P$ if and only if~$fx$ is nilpotent. This can be
used to show that the statement~$\Sh(\Spec A) \models \forall x,y\?\ul{A}\_ xy
\in \P \Rightarrow x \in \P \vee y \in \P$ is false in general.}
\XXX{Give counterexample.}


\subsection{An internal proof of Grothendieck's generic freeness lemma}\label{sect:generic-freeness}
The goal of this subsection is to give a simple proof of Grothendieck's generic
freeness lemma in the following general form.

\begin{figure}[h]
  \centering
  \begin{tikzpicture}[scale=1.4]
    \draw[step=1cm,gray,very thin] (0,0) grid (8,8);

    \draw
      (0,8) -- (0,0) -- (8,0);

    \fill[pattern=north east lines,pattern color=gray,opacity=0.5]
      (2,8) -- (2,5) -- (5,5) -- (5,3) -- (6,3) -- (6,2) -- (8,2) -- (8,8);
    \draw
      (2,8) -- (2,5) -- (5,5) -- (5,3) -- (6,3) -- (6,2) -- (8,2);

    \fill[fill=blue!40!white,opacity=0.5]
      (0,8) -- (0,0) -- (4,0) -- (4,3) -- (3,3) -- (3,5) -- (2,5) -- (2,8);

    \fill[fill=red!40!white,opacity=0.5]
      (3,3) -- (4,3) -- (4,4) -- (3,4);

    % !perl -we 'for my $y (reverse 0..7) { for my $x (0..7) { print % "\$x^$x\Ey^$y\$ & " } print "\\\\ \n" }'
    \begin{scope}[yshift=4cm, xshift=4cm]
      \matrix[matrix of nodes,nodes={inner sep=0pt,text width=1.4cm,align=center,minimum height=1.4cm}]{
        $x^0y^7v_1$ & $x^1y^7v_1$ & $x^2y^7v_1$ & $x^3y^7v_1$ & $x^4y^7v_1$ & $x^5y^7v_1$ & $x^6y^7v_1$ & $x^7y^7v_1$ & \\
        $x^0y^6v_1$ & $x^1y^6v_1$ & $x^2y^6v_1$ & $x^3y^6v_1$ & $x^4y^6v_1$ & $x^5y^6v_1$ & $x^6y^6v_1$ & $x^7y^6v_1$ & \\
        $x^0y^5v_1$ & $x^1y^5v_1$ & $x^2y^5v_1$ & $x^3y^5v_1$ & $x^4y^5v_1$ & $x^5y^5v_1$ & $x^6y^5v_1$ & $x^7y^5v_1$ & \\
        $x^0y^4v_1$ & $x^1y^4v_1$ & $x^2y^4v_1$ & $x^3y^4v_1$ & $x^4y^4v_1$ & $x^5y^4v_1$ & $x^6y^4v_1$ & $x^7y^4v_1$ & \\
        $x^0y^3v_1$ & $x^1y^3v_1$ & $x^2y^3v_1$ & $x^3y^3v_1$ & $x^4y^3v_1$ & $x^5y^3v_1$ & $x^6y^3v_1$ & $x^7y^3v_1$ & \\
        $x^0y^2v_1$ & $x^1y^2v_1$ & $x^2y^2v_1$ & $x^3y^2v_1$ & $x^4y^2v_1$ & $x^5y^2v_1$ & $x^6y^2v_1$ & $x^7y^2v_1$ & \\
        $x^0y^1v_1$ & $x^1y^1v_1$ & $x^2y^1v_1$ & $x^3y^1v_1$ & $x^4y^1v_1$ & $x^5y^1v_1$ & $x^6y^1v_1$ & $x^7y^1v_1$ & \\
        $x^0y^0v_1$ & $x^1y^0v_1$ & $x^2y^0v_1$ & $x^3y^0v_1$ & $x^4y^0v_1$ & $x^5y^0v_1$ & $x^6y^0v_1$ & $x^7y^0v_1$ & \\
      };
    \end{scope}
  \end{tikzpicture}
  \caption{\label{fig:single-step}A single step in the iterative process used
  in the proof of Theorem~\ref{thm:generic-freeness}, in the special case~$n = 2, m = 1$. The hatched cells
  indicate vectors which have already been removed from the generating family.
  The vector in the red cell was found to be expressible as a linear
  combination of vectors with smaller index (blue cells). It is therefore
  about to be removed, along with the vectors in all cells to the top and to
  the right of the red cell.}
\end{figure}

\begin{figure}[h]
  \centering
  \begin{tikzpicture}[scale=0.4]
    \begin{scope}
      \draw[step=1cm,gray,very thin] (0,0) grid (8,8);

      \draw
        (0,8) -- (0,0) -- (8,0);

      \fill[fill=blue!40!white,opacity=0.5]
        (0,8) -- (0,0) -- (6,0) -- (6,5) -- (5,5) -- (5,8);

      \fill[fill=red!40!white,opacity=0.5]
        (5,5) -- (6,5) -- (6,6) -- (5,6);

      \node[shape=circle,draw,inner sep=2pt] at (1,1) {1};
    \end{scope}

    \begin{scope}[xshift=9cm]
      \draw[step=1cm,gray,very thin] (0,0) grid (8,8);

      \draw
        (0,8) -- (0,0) -- (8,0);

      \fill[pattern=north east lines,pattern color=gray,opacity=0.5]
        (5,8) -- (5,5) -- (8,5) -- (8,8);
      \draw
        (5,8) -- (5,5) -- (8,5);

      \fill[fill=blue!40!white,opacity=0.5]
        (0,8) -- (0,0) -- (3,0) -- (3,5) -- (2,5) -- (2,8);

      \fill[fill=red!40!white,opacity=0.5]
        (2,5) -- (3,5) -- (3,6) -- (2,6);

      \node[shape=circle,draw,inner sep=2pt] at (1,1) {2};
    \end{scope}

    \begin{scope}[xshift=18cm]
      \draw[step=1cm,gray,very thin] (0,0) grid (8,8);

      \draw
        (0,8) -- (0,0) -- (8,0);

      \fill[pattern=north east lines,pattern color=gray,opacity=0.5]
        (2,8) -- (2,5) -- (8,5) -- (8,8);
      \draw
        (2,8) -- (2,5) -- (8,5);

      \fill[fill=blue!40!white,opacity=0.5]
        (0,8) -- (0,0) -- (7,0) -- (7,2) -- (6,2) -- (6,5) -- (2,5) -- (2,8);

      \fill[fill=red!40!white,opacity=0.5]
        (6,2) -- (7,2) -- (7,3) -- (6,3);

      \node[shape=circle,draw,inner sep=2pt] at (1,1) {3};
    \end{scope}

    \begin{scope}[yshift=-9cm]
      \draw[step=1cm,gray,very thin] (0,0) grid (8,8);

      \draw
        (0,8) -- (0,0) -- (8,0);

      \fill[pattern=north east lines,pattern color=gray,opacity=0.5]
        (2,8) -- (2,5) -- (6,5) -- (6,2) -- (8,2) -- (8,8);
      \draw
        (2,8) -- (2,5) -- (6,5) -- (6,2) -- (8,2);

      \fill[fill=blue!40!white,opacity=0.5]
        (0,8) -- (0,0) -- (6,0) -- (6,3) -- (5,3) -- (5,5) -- (2,5) -- (2,8);

      \fill[fill=red!40!white,opacity=0.5]
        (5,3) -- (6,3) -- (6,4) -- (5,4);

      \node[shape=circle,draw,inner sep=2pt] at (1,1) {4};
    \end{scope}

    \begin{scope}[yshift=-9cm,xshift=9cm]
      \draw[step=1cm,gray,very thin] (0,0) grid (8,8);

      \draw
        (2,8) -- (2,5) -- (5,5) -- (5,3) -- (6,3) -- (6,2) -- (8,2);

      \draw
        (0,8) -- (0,0) -- (8,0);

      \fill[pattern=north east lines,pattern color=gray,opacity=0.5]
        (2,8) -- (2,5) -- (5,5) -- (5,3) -- (6,3) -- (6,2) -- (8,2) -- (8,8);

      \fill[fill=blue!40!white,opacity=0.5]
        (0,8) -- (0,0) -- (4,0) -- (4,3) -- (3,3) -- (3,5) -- (2,5) -- (2,8);

      \fill[fill=red!40!white,opacity=0.5]
        (3,3) -- (4,3) -- (4,4) -- (3,4);

      \node[shape=circle,draw,inner sep=2pt] at (1,1) {5};
    \end{scope}

    \begin{scope}[yshift=-9cm,xshift=18cm]
      \draw[step=1cm,gray,very thin] (0,0) grid (8,8);

      \draw
        (0,8) -- (0,0) -- (8,0);

      \fill[pattern=north east lines,pattern color=gray,opacity=0.5]
        (2,8) -- (2,5) -- (3,5) -- (3,3) -- (6,3) -- (6,2) -- (8,2) -- (8,8);
      \draw
        (2,8) -- (2,5) -- (3,5) -- (3,3) -- (6,3) -- (6,2) -- (8,2);

      \node[shape=circle,draw,inner sep=2pt] at (1,1) {6};
    \end{scope}
  \end{tikzpicture}
  \caption{\label{fig:iteration}The iterative process used
  in the proof of Theorem~\ref{thm:generic-freeness}, in the special case~$n = 2, m = 1$.
  The process terminates after reducing the generating family
  a finite number of times.}
\end{figure}

\begin{thm}\label{thm:generic-freeness}
Let~$A$ be a reduced ring. Let~$B$ be an~$A$-algebra of finite type.
Let~$M$ be a finitely generated~$B$-module. Then there is a dense open
subset~$U \subseteq \Spec(A)$ such that over~$U$,
\begin{enumerate}
\item $B^\sim$ is finitely presented as an~$\O_{\Spec(A)}$-algebra,
\item $M^\sim$ is of finite presentation over~$B^\sim$, and
\item $M^\sim$ is (not necessarily finite) locally free as an~$\O_{\Spec(A)}$-module.
\end{enumerate}
\end{thm}

The usual proofs of Grothendieck's generic freeness lemma proceed using a series
of reduction steps which are arguably not very memorable or
straightforward, see for instance~\stacksproject{051Q}
or~\cite{staats:generic-freeness}. In particular, there doesn't seem to be a
published proof which tackles the Noetherian and non-Noetherian cases in one go.
Employing the internal language, Grothendieck's generic freeness lemma can be
proved in a simple, conceptual, and constructive way without any reduction steps.

This section was prompted by a MathOverflow thread~\cite{mo:kernel} and
greatly benefited from discussions with Brandenburg.

\begin{proof}[Proof of Theorem~\ref{thm:generic-freeness}]
Since ``dense open'' translates to ``not not'' in the internal language
(Proposition~\ref{prop:modops-kripke}), it suffices to prove that, from the
internal point of view of~$\Sh(\Spec(A))$, it's \notnot the case that
\begin{enumerate}
\item $B^\sim$ is of finite presentation over~$\O_{\Spec(A)}$,
\item $M^\sim$ is finitely presented as a~$B^\sim$-module, and
\item $M^\sim$ is (not necessarily finite) free over~$\O_{\Spec(A)}$.
\end{enumerate}

Since~$B^\sim$ is finitely generated as an~$\O_{\Spec(A)}$-algebra, it is
isomorphic to an algebra of the form~$\O_{\Spec(A)}[X_1,\ldots,X_n]/\aaa$ for
some number~$n \geq 0$ and some ideal~$\aaa$. By
Proposition~\ref{prop:ox-weakly-noetherian} and Theorem~\ref{thm:hilbert}, the
ring~$\O_{\Spec(A)}[X_1,\ldots,X_n]$ is weakly Noetherian. Therefore~$\aaa$ is
\notnot finitely generated, showing that~$B^\sim$ is \notnot of finite
presentation over~$\O_{\Spec(A)}$.

Similarly, the module~$M^\sim$ is of the form~$(B^\sim)^m/U$ for some number~$m
\geq 0$ and some submodule~$U$. Since~$(B^\sim)^m$ is weakly Noetherian as a
direct sum of weakly Noetherian modules, the submodule~$U$ is \notnot finitely
generated. Thus~$M^\sim$ is \notnot a finitely presented~$B^\sim$-module.

The basic idea to show that~$M^\sim$ is \notnot free over~$\O_{\Spec(A)}$ is as
follows. Since~$\O_{\Spec(A)}$ is a field in the sense that noninvertible
elements are zero, minimal generating families are already linearly independent;
we observed this in the proof of Lemma~\ref{lemma:rank-functor-locally-constant}.
By the finiteness hypotheses, the module~$M^\sim$ admits a countable
generating family. It's \notnot the case that either one of these vectors can be
expressed as a linear combination of the others, or not. In the second case
we're done; in the first case, we remove the redundant vector and continue in
the same fashion.

However, if we shrink the given generating family in this naive fashion, the
process may not terminate in finitely many steps. In a classical context,
Zorn's lemma could be used to iterate the process transfinitely and eventually
obtain a minimal generating family, but Zorn's lemma is not available in the
internal universe of the little Zariski topos. We therefore have to pick the
vectors we'll remove in a more systematic fashion.

Let~$(x_1,\ldots,x_n)$ be a generating family for~$B^\sim$ as
an~$\O_{\Spec(A)}$-algebra and let $(v_1,\ldots,v_m)$ be a generating
family for~$B^\sim$ as a~$B^\sim$-module. We endow the set
\[ I \defeq \{ (j, i_1,\ldots,i_n) \,|\,
  j \in \{ 1,\ldots,m \},
  i_1,\ldots,i_n \in \{ 0,1,\ldots \} \} \]
with the lexicographic order. We choose the family~$(x_1^{i_1} \cdots x_n^{i_n}
v_j)_{j,i_1,\ldots,i_n}$ as the starting point of the shrinking process. In
each step, we use that it's \notnot the case that
\begin{itemize}
\item either one of the vectors of the generating family can be expressed as a
linear combination of vectors in the family with a smaller index,
\item or not.
\end{itemize}

In the second case, the generating family is linearly independent: For any
linear combination summing to zero, we can show that all coefficients are zero,
beginning with the coefficient which is paired with the vector of greatest
index.

Figure~\ref{fig:single-step} illustrates our action in the first case. We remove the redundant vector~$x_1^{i_1} \cdots x_n^{i_n}
v_j$ \emph{and also any vector with greater powers of the~$x_1,\ldots,x_n$}
from the generating family. The resulting family will still be a generating
family, since the linear combination witnessing the redundancy of~$x_1^{i_1}
\cdots x_n^{i_n} v_j$ successively gives rise to linear combinations witnessing
the redundancy of the vectors~$x_1^{\geq i_1} \cdots x_n^{\geq i_n} v_j$;
we maintain the invariant that any member of the starting generating family can
be expressed as a linear combination of vectors of the current generating
family with smaller or equal index.

As indicated in Figure~\ref{fig:iteration}, this process terminates after
finitely many steps. This fact is related to the fact that the ordinal~$\omega^n$ is
well-founded.
\end{proof}

Since the given internal proof was (necessarily) intuitionistically valid, the
internal language machinery is intuitionistically valid, and the construction
of the spectrum can be set up in an intuitionistically sensible way
(Section~\ref{sect:relative-spectrum}), an intuitionistic external proof not
employing the topos machinery can be extracted from the given argument.
The resulting proof will verify Grothendieck's generic freeness lemma in the
following form.

\begin{thm}\label{thm:generic-freeness-constructively}
Let~$A$ be a reduced ring. Let~$B$ be an~$A$-algebra of finite type.
Let~$M$ be a finitely generated~$B$-module. Assume that the only element~$f \in
A$ such that
\begin{enumerate}
\item $B[f^{-1}]$ is of finite presentation over~$A[f^{-1}]$,
\item $M[f^{-1}]$ is finitely presented as a~$B[f^{-1}]$-module, and
\item $M[f^{-1}]$ is free over~$A[f^{-1}]$
\end{enumerate}
is~$f = 0$. Then~$A = 0$.
\end{thm}

In classical logic, this form implies Grothendieck's generic freeness lemma in its more abstract
formulation by a routine argument: Let~$U \subseteq \Spec(A)$ be the union over
all standard open subsets~$D(f)$ such that the statements~(1),~(2), and~(3) in
Theorem~\ref{thm:generic-freeness-constructively} hold. The
statements~(1),~(2), and~(3) of Theorem~\ref{thm:generic-freeness} hold on this
open subset, therefore it remains to show that~$U$ is dense.

So let a nonempty open subset~$V$ of~$\Spec(A)$ be given. This
contains a standard open subset~$D(g) \subseteq V$ such that~$g$ is not
nilpotent. Therefore the localized ring~$A[g^{-1}]$ is not zero. Thus the
conclusion of Theorem~\ref{thm:generic-freeness-constructively} is not satisfied.
Since we assume classical logic, there is a nonzero element~$f \in A[g^{-1}]$
such that statements~(1),~(2), and~(3) in
Theorem~\ref{thm:generic-freeness-constructively} hold for~$A[g^{-1}][f^{-1}]$,
$B[g^{-1}][f^{-1}]$, and~$M[g^{-1}][f^{-1}]$.
Writing~$f = h/g^n$, we see that~$U \cap V$ contains the nonzero open
subset~$D(gh)$.

We refrain from giving the resulting explicit proof of
Theorem~\ref{thm:generic-freeness-constructively} here, but will report on it
in the future~\cite{blechschmidt:generic-freeness}. A part of the proof was
included by Brandenburg in a paper of his~\cite{brandenburg:schur}.

\begin{rem}There is no hope that there is an intuitionistic proof of
Gro\-then\-dieck's generic freeness lemma in the form of
Theorem~\ref{thm:generic-freeness} even if the spectrum is constructed in an
intuitionistically sensible way, since there is the following Brouwerian
counterexample. Let~$\varphi$ be an arbitrary statement. Then
the~$\ZZ$-module~$M \defeq \ZZ/\aaa$, where~$\aaa \defeq \{ x \in \ZZ \,|\, (x
= 0) \vee \varphi \}$ as in Footnote~\ref{fn:z-principal-ideal-domain} on
page~\pageref{fn:z-principal-ideal-domain}, is finitely generated. By
assumption, there exists a nonzero element~$f \in \ZZ$ such that~$M[f^{-1}]$ is
a finite free module over~$A[f^{-1}]$ of some rank~$n$. If~$n = 0$, then~$f^m
\in \aaa$ for some~$m \geq 0$, therefore~$\varphi$ holds. If~$n \geq 1$,
then~$\neg\varphi$ holds, since~$\varphi$ would imply~$\aaa = \ZZ$ and
therefore~$M[f^{-1}] = 0$. Since~$n = 0 \vee n \geq 1$, it follows
that~$\varphi \vee \neg\varphi$.
\end{rem}


\section{Relative spectrum}
\label{sect:relative-spectrum}

Recall that if~$\A$ is a quasicoherent~$\O_X$-algebra on a scheme~$X$, one can
construct the \emph{relative spectrum}~$\RelSpec_X{\A}$ by appropriately
gluing the spectra~$\Spec \Gamma(U,\A)$ where~$U$ ranges over the affine opens
of~$X$. This relative spectrum comes equipped with a canonical
morphism~$\RelSpec_X{\A} \to X$.

From the internal point of view of~$\Sh(X)$, the sheaf~$\A$ looks just like a
plain algebra, to which therefore the usual (absolute) spectrum construction
can be applied. One could hope that this construction yields the relative
spectrum.

In this section, we discuss generalities on how to make sense of this internal
construction; we show that this proposed construction is too naive and doesn't
yield the relative spectrum; we give a refined internal construction which does
yield the relative spectrum, discuss its relation to the naive construction,
and phrase it in topos-theoretic terms; and we deduce, as an application, a
description of limits in the category of locally ringed spaces. We also cover
the relative Proj construction.

In much of the following, it's not actually necessary that~$X$ is a scheme
and~$\A$ is a quasicoherent algebra. If~$X$ is not a scheme or~$\A$ is not
quasicoherent, then~$\RelSpec_X(\A)$ might fail to be a scheme and can of
course not be constructed by gluing usual spectra, but it still exists as
a more general kind of space and still verifies a meaningful universal
property. We give details on this generalization below.


\subsection{Internal locales} Let~$X$ be a topological space (or a locale). A
fundamental fact in the theory of locales is that there is a canonical
equivalence between the category of \emph{locales over~$X$} -- that is
locales~$Y$ equipped with a morphism~$Y \to X$ -- and \emph{internal locales
in~$\Sh(X)$}~\cite[p.~49]{johnstone:point}. An internal locale in a topos~$\E$ is given by an object~$L$ of~$\E$
(the internal lattice of opens of the locale) together with a binary
relation~$(\preceq) \hookrightarrow L \times L$ such that the axioms on a
locale hold from the internal point of view. (For our purposes, we do not need a
precise wording of these axioms.)

The equivalence is described as follows: A locale~$f : Y \to X$ over~$X$
induces an internal locale~$I(Y)$ with object of opens given by~$\Open(I(Y)) \defeq
f_* \Omega_{\Sh(Y)} \in \Sh(X)$, where~$f_*$ is the pushforward functor
and~$\Omega_{\Sh(Y)}$ is the object of truth values in the topos of sheaves
on~$Y$. Conversely, an internal locale given by an internal frame~$\L \in \Sh(X)$ induces an (external)
locale~$E(\L)$ with lattice of opens given by~$\Open(E(\L)) \defeq \Gamma(X,\L)$.
This comes equipped with a canonical morphism~$Y \to X$ of locales which we do
not need to describe explicitly~\cite[Section~C1.6]{johnstone:elephant}.

As a special case, the internalization of the trivial locale~$\id : X \to X$
over~$X$ has as lattice of opens the object~$\id_* \Omega_{\Sh(X)} =
\Omega_{\Sh(X)} = \P(1)$. This is precisely the lattice of opens of the
one-point space. Thus~$I(X) \cong \pt$. This illustrates the intuition
behind working internally in~$\Sh(X)$: From the perspective
of~$\Sh(X)$, the space~$X$ looks like the one-point space (even if in fact it
is not).

One can associate to an internal locale~$T$ in a topos~$\E$ a topos of internal
sheaves on it:~$\Sh_\E(T)$. The correspondence is made in such a way that the topos of
sheaves on a locale~$Y$ over~$X$ is equivalent to the topos of sheaves on the
internal locale~$I(Y)$: $\Sh(Y) \simeq \Sh_{\Sh(X)}(I(Y))$.

There is no similarly nice correspondence between topological spaces
over~$X$ and internal topological spaces
in~$\Sh(X)$~\cite[Corollary~C1.6.7]{johnstone:elephant}. This is one of the
reasons why locales are better suited for working internally and for switching
between internal and external perspectives.

For verification of properties of such sheaves, the \emph{idempotency} of the
internal language is useful: If~$\varphi$ is a formula over~$Y$, then
\[ \Sh(Y) \models \varphi \qquad\text{if and only if}\qquad
  \Sh(X) \models \speak{$\Sh(I(Y)) \models \varphi$}. \]
Here we're abusing notation in two ways. Firstly, the formula~$\varphi$ has to
be appropriately interpreted in the expression~``$\Sh(I(Y)) \models \varphi$''.
Secondly, the expression~``$\Sh(I(Y))$'' doesn't actually refer to the
category~$\Sh_{\Sh(X)}(I(Y))$, but to the \emph{locally internal} category induced by
the canonical geometric morphism~$\Sh_{\Sh(X)}(I(Y)) \to \Sh(X)$. We give some
details on this point in Section~\ref{sect:relation-big-little}. However, in
the situations encountered in this section, the meaning will always be
reasonably clear.


\subsection{The spectrum of a ring as a locale}
\label{sect:spectrum-as-a-locale}
Recall that the spectrum
of a ring~$A$ is usually constructed as the set
\[ \Spec A \defeq \{ \ppp \subseteq A \,|\,
  \text{$\ppp$ is a prime ideal} \} \]
endowed with a certain topology and a sheaf of rings~$\O_{\Spec A}$. From an
intuitionistic (and thus internal) point of view, this construction does not
work well: Prime ideals are intuitionistically much more elusive than
classically, where one can appeal to Zorn's lemma to obtain maximal (and thus
prime) ideals. More to the point, one cannot show that this construction of
the spectrum as a topological space verifies the expected universal property,
namely
\[ \Hom_\LRS(X, \Spec A) \cong \Hom_\Ring(A, \Gamma(X, \O_X)) \]
for all locally ringed spaces~$X$ (or some variant of this property involving
more general kinds of spaces).

On the other hand, the lattice of opens of~$\Spec A$ admits a simple
description not requiring the notion of prime ideals:
\[ \Open(\Spec A) \cong \{ \aaa \subseteq A \,|\,
  \text{$\aaa$ is a radical ideal} \}. \]
An open subset~$U \subseteq \Spec A$ corresponds to the radical ideal~$\{ h \in
A \,|\, D(h) \subseteq U \}$ (so in particular, the open subset~$D(f)$
corresponds to the radical ideal~$\sqrt{(f)}$); conversely, a radical ideal~$\aaa$
corresponds to the open subset~$\bigcup_{h \in \aaa} D(h)$.

Thus, in an intuitionistic context, we will construct the spectrum of a ring~$A$
as a locale, not as a topological space, and adopt the following definition.

\begin{defn}\label{defn:spectrum-as-a-locale}
The \emph{spectrum}~$\Spec(A)$ of a ring~$A$ is the locale whose lattice of
opens is the lattice of radical ideals of~$A$. We endow it with the structure
sheaf~$\O_{\Spec(A)} \defeq \ul{A}[\F^{-1}]$, where~$\F$ is the generic filter
as described in Section~\ref{sect:generic-filter}.\end{defn}

This construction has the expected
universal property, namely that it is adjoint to the global functions functor:
\[ \Hom_{\LRL}(X, \Spec A) \cong \Hom_{\Ring}(A, \Gamma(X,\O_X)). \]
Here, ``$\LRL$'' refers to the category of \emph{locally ringed locales}, \ie
locales~$X$ equipped with a sheaf of rings~$\O_X$ such that from the internal point of
view of~$\Sh(X)$, the ring~$\O_X$ is a local ring. A morphism~$Y \to X$ of
locally ringed locales consists of a locale morphism~$f : Y \to X$ and a
morphism~$f^\sharp : f^{-1} \O_X \to \O_Y$ of sheaves of rings on~$Y$ such that, from the
internal point of view of~$\Sh(Y)$, the ring homomorphism~$f^\sharp$ is a local
homomorphism. The notion of a locally ringed locale is thus a straightforward
generalization of that of a locally ringed space.

Schemes are usually regarded as locally ringed spaces, not
as locally ringed locales. However, in a classical
context where the axiom of choice is available, schemes are \emph{sober}
topological spaces~\stacksproject{01IS}. For sober topological spaces, the passage from the space to
its induced locale (forgetting the set of points and only keeping the frame of
open subsets) doesn't lose information: The category of sober topological
spaces with arbitrary continuous maps embeds into the category of locales as a
full subcategory. Therefore the category of schemes can just as well be viewed
as a full subcategory of the category of locally ringed locales.

The importance of a locale-theoretic approach to spectra of rings, especially in
relative situations, has also been stressed by Lurie~\cite[p.~37]{lurie:dag5}.


{\tocless

\subsection*{Points of the locale-theoretic spectrum}
Constructing the spectrum as a locale instead of a topological space
sidesteps any issues with prime ideals, since points are not a defining
ingredient of a locale. However, points are still meaningful as a \emph{derived
concept}: A point of locale~$X$ is a morphism~$1 \to X$, where~$1$ is the
terminal locale, the locale corresponding to the one-point topological space
with lattice of opens~$\P(1) = \Omega$.
Therefore it's still an interesting question what the points of the
locale~$\Spec(A)$ look like.

\begin{prop}\label{prop:points-spectrum}
Let~$A$ be a ring. Then the points of the locale~$\Spec(A)$ are in
canonical one-to-one correspondence with the filters of~$A$
(as in Definition~\ref{defn:filter}), even intuitionistically.\end{prop}

\begin{proof}The points of a locale~$X$ are in canonical one-to-one
correspondence with the \emph{completely prime filters} of~$\Open(X)$,
subsets~$K \subseteq \Open(X)$ which are upward-closed, downward-directed, and
have the property that, whenever a supremum of a set~$M \subseteq \Open(X)$ is
contained in~$K$, then so is some element of~$M$.

Such a completely prime filter~$K \subseteq \Open(\Spec(A))$ corresponds to the
ring-theoretic filter
\[ F \defeq \{ s \? A \,|\, \sqrt{(s)} \in K \} \subseteq A, \]
and a ring-theoretic filter~$F \subseteq A$ corresponds to the completely prime
filter
\[ K \defeq \{ \aaa \? \Open(\Spec(A)) \,|\,
  \text{$\aaa \cap F$ is inhabited} \}. \]
We omit the required routine verifications.
\end{proof}

In classical logic, where complementation yields a one-to-one correspondence
between filters and prime ideals, the points of~$\Spec(A)$ are therefore in
canonical bijection with the prime ideals of~$A$, just as one would expect.

Observing that intuitionistically the points of the locale~$\Spec(A)$ are
filters, not prime ideals, one might wonder: Is the locale-theoretic approach
really necessary? Wouldn't it suffice to define~$\Spec(A)$ as the topological
space of filters of~$A$? Indeed, for some time this was
believed~\cite[Section~3]{lawvere:icm-address}; however, this hope turned out
to be too naive: Joyal gave an explicit example of a nontrivial ring in a
certain topos without any filters~\cite[pp.~200f.]{tierney:spectrum}, thus
showing that the construction can't have the expected universal property and
that therefore a true pointfree approach as provided by lattice theory/locale theory~\cite{cls:spectral-schemes}, topos
theory, or formal topology~\cite{schuster:formal-zariski} is necessary to construct the spectrum in an
intuitionistic context.\footnote{When following
reference~\cite{tierney:spectrum}, note that Tierney calls ``primes'' what we
call ``filters''. Joyal's example was none other than the ring~$\affl$
in the functor category~$[\Ring_\mathrm{fp}, \Set]$. The big Zariski topos
of~$\Spec(\ZZ)$, when defined using the parsimonious sites, is a subtopos of
that topos; in it, the ring~$\affl$ does have filters, for instance the filter
of units. These two facts are not contradictory, since not having any filters
is not a geometric implication and is therefore not preserved by inverse image
parts of geometric morphisms.}


\subsection*{The spectrum as a classifying locale}

The fact that the points of~$\Spec(A)$ are in canonical one-to-one
correspondence with the filters of~$A$ is a shadow of a more general fact.
Namely, for any locale~$X$ (and in fact any topos), maps~$X \to \Spec(A)$ are
in canonical one-to-one correspondence with the internal filters of~$A$
in~$\Sh(X)$, that is subsheaves of the constant sheaf~$\ul{A}$ satisfying the
filter axioms from the point of view of the internal language of~$\Sh(X)$:
The locale~$\Spec(A)$ is the \emph{classifying locale of the theory of filters
of~$A$}.

The fact about the points of~$\Spec(A)$ can be recovered from this
observation as follows. A point of~$\Spec(A)$ is a map~$1 \to \Spec(A)$ and
therefore corresponds to a subsheaf of the constant sheaf~$\ul{A}$
satisfying the filter axioms from the point of view of~$\Sh(1)$. Since~$\Sh(1)
\simeq \Set$, such a subsheaf amounts to a subset of~$A$ satisfying the filter
axioms.

The notion of classifying locales provides a pleasant way of approaching the
problem of constructing a space of models of a \emph{propositional geometric
theory} (in the case of the spectrum the theory of filters), simultaneously
streamlining the usual topological approach and generalizing it to work in an
intuitionistic context: Instead of first constructing the \emph{set} of models
(filters of~$A$) and then manually endowing this set with a suitable topology
(the Zariski topology), one can simply consider the \emph{locale} of models,
that is the classifying locale of the theory. Its sets of points coincides with
the set of models of the topological approach, but the locale is not determined
by its sets of points, facilitating a better behavior in contexts where the
points might be elusive.

Put more concisely, the topological space of filters doesn't work well in an
intuitionistic context, but the locale of filters does.

A lucid expository account of the theory of classifying locales can be found in
a survey article by Vickers~\cite{vickers:locales-toposes}.

\begin{rem}\label{rem:theory-of-filters}
For comparison with a refined geometric theory discussed below, we describe the
geometric theory of filters of~$A$ here explicitly. It has one atomic
proposition~``$s \in F$'' for each element~$s \? A$, and its axioms are given by the
following axiom schemes:
\begin{enumerate}
\item $\top \vdash 1 \in F$
\item $st \in F \dashv\vdash s \in F \wedge t \in F$ (two axioms for each $s,t\?A$)
\item $0 \in F \vdash \bot$
\item $s+t \in F \vdash s \in F \vee t \in F$ (one axiom for each $s,t\?A$)
\end{enumerate}
\end{rem}


\subsection*{A trivial case}
For later use, we study the question when the spectrum is the one-point space.
The answer is well-known classically, but since we want to use this result in
an internal context, we have to give an intuitionistic proof.
\begin{lemma}\label{lemma:spectrum-one-point}
Let~$A$ be a ring. Its spectrum is a one-point
space (as a locale) if and only if~$1 \neq 0$ in~$A$ any element of~$A$ is nilpotent or
invertible.\end{lemma}
\begin{proof}The locale~$\Spec A$ is a one-point space if and only if the
unique continuous map~$\Spec(A) \to \pt$ of locales is an isomorphism. This is the case if and only
if the canonical frame homomorphism
\[ \begin{array}{@{}rcl@{}}
  \Omega = \P(1) &\longrightarrow& \Open(\Spec A) \\[0.1em]
  \varphi &\longmapsto& \aaa_\varphi \defeq \sup\{\sqrt{(1)} \,|\, \varphi \} =
  \{ x \? A \,|\, \speak{$x$ nilpotent} \vee \varphi \}
\end{array} \]
is surjective and reflects the ordering (and is therefore automatically
injective). If~$1 = 0$ in~$A$, this homomorphism is not injective, since~$\bot$
and~$\top$ get both mapped to~$\sqrt{(0)}$. For the rest of the proof, we'll
therefore assume that~$1 \neq 0$ in~$A$.

Under this assumption, the homomorphism reflects the ordering: If~$\aaa_\varphi \subseteq \aaa_\psi$,
then~$(1 \in \aaa_\varphi) \Rightarrow (1 \in \aaa_\psi)$. Since the unit
of~$A$ is not nilpotent, this amounts to~$\varphi \Rightarrow \psi$.

The homomorphism is surjective if and only if for any radical ideal~$\aaa \subseteq A$,
it holds that~$\aaa = \{ x \? A \,|\, \speak{$x$ nilpotent} \vee \varphi \}$
for some proposition~$\varphi$. By considering the condition~``$1 \in \aaa$'',
it follows that this proposition~$\varphi$ must be equivalent to the
proposition~``$1 \in \aaa$'' (if it is at all possible to write~$\aaa$ in such
a way).

So the map is surjective if and only if for any radical ideal~$\aaa \subseteq
A$ and any element~$x$ of~$A$ it holds that
\[ x \in \aaa \quad\Longleftrightarrow\quad
  \speak{$x$ nilpotent} \vee (1 \in \aaa). \]
The ``if'' direction always holds. If any element of~$A$ is nilpotent or
invertible, the ``only if'' direction holds as well (for any~$\aaa$ and any~$x$).
Conversely, if the ``only if'' direction holds, then any element of~$A$ is
nilpotent or invertible. This follows by
considering the radical ideal~$\sqrt{(f)}$ for an element~$f \? A$.
\end{proof}

\begin{rem}The structure sheaf~$\O_X$ of a scheme fulfills almost, but not
quite, the condition given in Lemma~\ref{lemma:spectrum-one-point}: By
Proposition~\ref{prop:neginvnilpotent}, it has the property that
any element which is not invertible is nilpotent. In classical logic, this
statement is equivalent to the statement that every element is nilpotent or invertible.
However, intuitionistically the former is a weaker statement than the latter.
This observation entails that the internally constructed spectrum does
\emph{not} coincide with the relative spectrum, and that instead a refined
approach is necessary. Section~\ref{sect:rel-spec-as-ordinary-spec} is devoted
to studying this difference.
\end{rem}

}


\subsection{Digression: Further topologies on the set of prime ideals}
\label{sect:flat-constructible-topologies}

The Zariski topology is not the only interesting topology on the set of prime
ideals. For instance, the constructible topology and the flat topology studied
by Tarizadeh~\cite{tarizadeh:flat} too have their uses. While the contents of
Section~\ref{sect:spectrum-as-a-locale} are well-known, the locale-theoretic
approach to these variants of the spectrum and their universal properties
appear to not have been studied much.

The universal properties given in the following two propositions should be
compared with the following way of phrasing the universal property of the
ordinary locale-theoretic spectrum. The usual phrasing employs the
categories~$\RL$ and~$\LRL$ of (locally) ringed locales, therefore emphasizing
the spatial character. But the dual categories~$\RL^\op$ and~$\LRL^\op$ can be
used just as well; since the morphisms in~$\RL^\op$ and~$\LRL^\op$ go in the
direction of the ring-theoretic parts, they can be thought of as the category
of \emph{all} rings respectively \emph{all} local rings, where~``all'' refers
to the fact that these categories don't only include the (local) rings
in~$\Set$, but the (local) rings in arbitrary localic sheaf toposes.

Formulated using~$\RL^\op$ and~$\LRL^\op$, and adopting the notation to
suppress mention of the involved spaces (instead of the involved sheaves
of rings), the universal property of~$\Spec(A)$ reads as follows: For any
local ring~$\O_Y$ over any locale~$Y$,
\[ \Hom_{\LRL^\op}(\O_{\Spec(A)}, \O_Y) \cong
  \Hom_{\RL^\op}(A, \O_Y). \]
The morphism~$A \to \O_{\Spec(A)}$ in~$\RL^\op$ is therefore the
\emph{universal localization} of~$A$.

\begin{prop}Let~$A$ be a ring. The locale given by the space of prime ideals
of~$A$ with the flat topology is the classifying locale of prime ideals of~$A$.
Equipped with~$\ul{A}/\P$ as structure sheaf, where~$\P$ is the generic prime
ideal, it is the universal way of mapping~$A$ to an integral domain in the weak
sense (as defined in Section~\ref{sect:integrality}).
\end{prop}

\begin{prop}Let~$A$ be a ring. The locale given by the space of prime ideals
of~$A$ with the constructible topology is the classifying locale of detachable
prime ideals (or equivalently detachable filters) of~$A$. Equipped
with~$\ul{A}/\P$ as structure sheaf, where~$\P$ is the generic prime ideal, it
is the universal way of mapping~$A$ to an integral domain in the strong sense.
Equipped with~$\ul{A}[\F^{-1}]$, where~$\F$ is the generic filter, is is the
universal way of mapping~$A$ to a local ring in which invertibility is
decidable.\end{prop}

In constructive mathematics, a subset~$U \subseteq A$ is \emph{detachable} if
and only if for every element~$a\?A$, either $a \in U$ or~$a \not\in U$. While
intuitionistically the complement of a filter might fail to be a prime ideal
and the complement of a prime ideal might fail to be a filter, the complement
of a detachable filter is a detachable prime ideal, and vice versa.

\XXX{write down proof}


\subsection{The relative spectrum as an ordinary spectrum\except{toc}{ from the internal
point of view}}\label{sect:rel-spec-as-ordinary-spec}

Let~$X$ be a scheme and~$\A$ be a quasicoherent~$\O_X$-algebra.
Since~$\A$ looks like a plain algebra from the internal perspective
of~$\Sh(X)$, we can consider its internally defined spectrum. This is a locale
internal to~$\Sh(X)$; we might hope that its externalization is precisely the
relative spectrum of~$\A$ (considered as a locale):
\[ E(\Spec \A) \stackrel{?}{\cong} \RelSpec_X{\A}. \]
However, this turns out to be too naive. The locale~$E(\Spec(\A))$ is equipped
with a map to~$X$, being an externalization of a locale internal to~$\Sh(X)$,
and it is equipped with a sheaf of rings (because we can transport the
internally defined structure sheaf along the
equivalence~$\Sh_{\Sh(X)}(\Spec(\A)) \simeq \Sh(E(\Spec(A)))$. Furthermore,
this sheaf of rings is local, since we know
\[ \Sh(X) \models \speak{$\Sh(\Spec(\A)) \models \speak{%
  $\O_{\Spec(\A)}$ is a local ring}$} \]
which by idempotency of the internal language is equivalent to
\[ \Sh(E(\Spec(\A))) \models \speak{$\O_{\Spec(\A)}$ is a local ring}. \]

However, the map~$E(\Spec(\A)) \to X$ is only part of a morphism of ringed
locales, not of locally ringed locales (even though domain and codomain happen
to be locally ringed): Internally, the morphism~$(\Spec(\A),\O_{\Spec(\A)}) \to
(\pt,\O_X)$ of ringed locales, which is defined using the~$\O_X$-algebra
structure of~$\A$, is not a morphism of locally ringed locales (even though
domain and codomain happen
to be locally ringed).

In contrast, the true relative spectrum~$\RelSpec_X(\A)$ is equipped with a
morphism of locally ringed locales to~$X$.

It's illuminating to compare the different universal properties
of~$E(\Spec(\A))$ and~$\RelSpec_X(\A)$. There is a canonical
morphism~$E(\Spec(\A)) \to E(\Spec(\O_X))$ of locally ringed locales (the
externalization of the canonical morphism~$\Spec(\A) \to \Spec(\O_X)$ given by
the~$\O_X$-algebra structure of~$\A$), but in general, the
locales $E(\Spec(\O_X))$ and~$X$ are not isomorphic.

As we justify below, the externalization of the internally
defined spectrum has the universal property
\[
  \Hom_{\LRL/E(\Spec\O_X)}(Y, E(\Spec\A)) \cong
    \Hom_{\O_X}(\A, \mu_* \O_Y)
\]
for all locally ringed locales~$Y$ over~$E(\Spec\O_X)$. Here,~$\mu$ is the
structure morphism~$Y \to \Spec\O_X$, $E(\Spec\O_X)$ is the locally ringed
locale associated to the internally defined spectrum of~$\O_X$,
and~$\LRL_{\Sh(X)}$ is the category of locally ringed locales internal
to~$\Sh(X)$. In contrast, the relative spectrum has the different universal property
\[
  \Hom_{\LRL/X}(Y, \RelSpec_X{\A}) \cong
    \Hom_{\O_X}(\A, \mu_* \O_Y)
\]
for all locally ringed locales~$Y$ over~$X$.\footnote{If~$X$ is a scheme
and~$\A$ is quasicoherent, this universal property is well-known, even though
it's usually only stated for schemes~$Y$ over~$X$ instead of general locally
ringed locales over~$X$. In any case, we take this universal property as the
definition of what the relative spectrum should be.} The crucial
difference is that in general, the internally defined locally ringed
locale~$\Spec\O_X$ does \emph{not} coincide with the internal locally ringed
locale~$(\pt,\O_X)$ (which is simply~$(X,\O_X)$ from the external point of
view). More succinctly, the functor~$E \circ \Spec$ is an adjoint to the
pushforward-of-sheaf-of-functions functor~$\LRL/E(\Spec\O_X) \to \Alg(\O_X)^\op$, while the
relative spectrum functor is an adjoint to the analogous functor~$\LRL/X
\to \Alg(\O_X)^\op$.

The universal property of~$E(\Spec(\A))$ can be determined as follows.
From the internal point of view of~$\Sh(X)$, the locally ringed
locale~$E(\Spec(\A))$ looks like the ordinary locale-theoretic
spectrum~$\Spec(\A)$ and therefore has the universal property
\[ \Hom_{\LRL}(Y, \Spec(\A)) \cong
  \Hom_{\Ring}(\A, \Gamma(Y, \O_Y)) \]
for any locally ringed locale~$Y$.\footnote{Externally, this implies that for any
locally ringed locale over the underlying locale of~$X$ (that is, for any
locale~$Y$ equipped with a morphism~$\mu : Y \to X$ and a local sheaf of
rings), we have
\[ \Hom_{\mathrm{LR}(\mathrm{L}/X)}(Y, E(\Spec(\A))) \cong
  \Hom_{\Ring_{\Sh(X))}}(\A, \mu_*\O_Y). \]}
If we restrict the right-hand side to the
set of~$\O_X$-algebra homomorphisms, the left-hand side restricts to the set of
morphisms~$Y \to \Spec(\A)$ of locally ringed locales over the locally ringed
locale~$\Spec(\O_X)$. So we have
\[ \Hom_{\LRL/\Spec(\O_X)}(Y, \Spec(\A)) \cong \Hom_{\Alg(\O_X)}(\A,
\Gamma(Y,\O_Y)). \]

This discussion took place in the internal universe of~$\Sh(X)$. Externally,
the displayed universal property implies that for any locally ringed
locale~$\mu : Y \to X$ over~$E(\Spec(\O_X))$,
\[ \Hom_{\LRL/E(\Spec(\O_X))}(Y, E(\Spec(\A))) \cong
  \Hom_{\Alg(\O_X)}(\A, \mu_*\O_Y), \]
as claimed above.

\begin{defn}\label{defn:local-spectrum}
Let~$R$ be a ring. Let~$A$ be an~$R$-algebra. The \emph{local
spectrum} of~$A$ over~$R$ is the locale~$\Spec(A|R)$ with lattice of opens
given by
\begin{multline*}
  \quad\qquad\Open(\Spec(A|R)) \defeq
    \{ \aaa \subseteq A \,|\,
      \text{$\aaa$ is a radical ideal such that} \\
  \text{$\forall f\?R\_ \forall s\?A\_
    (\speak{$f$ \inv} \Rightarrow s \in \aaa) \Rightarrow fs \in \aaa$} \}.\qquad\quad
\end{multline*}
\end{defn}

We'll equip the local spectrum with the structure of a locally ringed locale
below. It is this refined construction which correctly internalizes the
relative spectrum:

\begin{thm}\label{thm:local-spectrum-yields-relative-spectrum}
Let~$X$ be a scheme (or a locally ringed locale). Let~$\A$ be
an~$\O_X$-algebra. Then the externalization~$E(\Spec(\A|\O_X))$ coincides
with~$\RelSpec_X(\A)$ as locally ringed locales over~$X$.\end{thm}

Before giving the proof, we want to clarify some details of the
construction.

Firstly, the base ring~$R$ directly enters the construction. This is in
contrast to the usual spectrum: If~$A$ is an~$R$-algebra, the construction
of~$\Spec(A)$ does not depend on the~$R$-algebra structure of~$A$. The algebra
structure only enters in the construction of a morphism~$\Spec(A) \to
\Spec(R)$.

Secondly, in the case that~$X$ is a scheme and~$\A$ is a
quasicoherent~$\O_X$-algebra, we can compare the externalization
of~$\Spec(\A|\O_X)$ with the result of the construction of~$\RelSpec_X(\A)$
by gluing spectra:

\begin{prop}Let~$X$ be a scheme. Let~$\A$ be a quasicoherent~$\O_X$-algebra.
Then~$E(\Spec(\A|\O_X))$ coincides with~$\RelSpec_X(\A)$ as locales over~$X$.
\end{prop}

\begin{proof}The condition
\[ \forall f\?\O_X\_ \forall s\?\A\_
    (\speak{$f$ \inv} \Rightarrow s \in \aaa) \Longrightarrow fs \in \aaa \]
appearing in Definition~\ref{defn:local-spectrum} is precisely the internal
quasicoherence condition of Corollary~\ref{cor:submodule-qcoh} (slightly
simplified in view that~$\aaa$ is a radical ideal). The sections of the
sheaf~$\brak{\Open(\Spec(\A|\O_X))}$ on an open subset~$U \subseteq X$ are
therefore precisely the quasicoherent sheaves of radical ideals~$\aaa
\hookrightarrow \A|_U$. Let~$\pi : \RelSpec_X(\A) \to X$ be the canonical
morphism. If~$U$ is affine, then
\[ \pi^{-1}U \cong \RelSpec_X(\A) \times_X U \cong \RelSpec_U(\A|_U) \cong
  \Spec(\Gamma(U,\A)) \]
is affine as well and
\begin{align*}
  \Gamma(U, \Open(I(\RelSpec_X(\A)))) &=
  \Gamma(U, \pi_* \Omega_{\RelSpec_X(\A)}) =
  \Omega_{\RelSpec_X(\A)}(\pi^{-1}U) \\
  &\cong \text{set of open subsets of~$\pi^{-1}U$} \\
  &\cong \text{set of open subsets of~$\Spec(\Gamma(U,\A))$} \\
  &\cong \text{set of radical ideals of~$\Gamma(U,\A)$} \\
  &\cong \text{set of quasicoherent sheaves of radical ideals of~$\A|_U$} \\
  &\cong \Gamma(U, \brak{\Open(\Spec(\A|\O_X))}).
\end{align*}
Therefore~$I(\RelSpec_X(\A))$ is canonically isomorphic to~$\Spec(\A|\O_X)$ as
locales internal to~$\Sh(X)$. Expressed externally: The relative
spectrum~$\RelSpec_X(\A)$ coincides with the externalization
of~$\Spec(\A|\O_X)$ as locales over~$X$, as claimed.
\end{proof}

Thirdly, the partial order~$\Open(\Spec(A|R))$ is indeed a frame. A quick
way to verify this is to recognize that it is related to the frame of opens
of~$\Spec(A)$ by the formula
\[ \Open(\Spec(A|R)) =
  \{ \aaa \? \Open(\Spec(A)) \,|\, \aaa = \overline{\aaa} \}, \]
where~$(\aaa \mapsto \overline{\aaa})$ is the quasicoherator described in
Remark~\ref{rem:quasicoherator-knaster-tarski}. Since the quasicoherator
satisfies the axioms on a nucleus, this formula exhibits~$\Spec(A|R)$ as a
sublocale of~$\Spec(A)$. In particular, suprema are calculated
in~$\Open(\Spec(A|R))$ by applying the quasicoherator to the suprema calculated
in~$\Open(\Spec(A))$. We denote the inclusion~$\Spec(A|R) \hookrightarrow
\Spec(A)$ by~``$i$''.

Lastly, it's interesting to know the points of~$\Spec(A|R)$, even though these
don't determine~$\Spec(A)$.

\begin{defn}\label{defn:over-the-filter-of-units}
Let~$R$ be a ring. Let~$\varphi : R \to A$ be an algebra. A
filter~$F \subseteq A$ \emph{lies over the filter of units} if and only if
$\varphi^{-1}F \subseteq R^\times$, that is if
\[ \varphi(r) \in F \Longrightarrow \text{$r$ is invertible in $R$} \]
for all~$r \? R$. (The reverse inclusion~``$\varphi^{-1}F \supseteq R^\times$''
holds automatically.)\end{defn}

This definition will mostly be used in situations where the ring~$R$ is local,
in which case the subset~$R^\times$ is actually a filter and the phrase
``filter of units'' is therefore justified.

It's illuminating to consider Definition~\ref{defn:over-the-filter-of-units} in
a classical context, even though the use case we have in mind is to apply it in
the internal language of the little Zariski topos of a base scheme.
Classically, a filter~$F$ lies over the filter of units if and only
if~$\varphi^{-1}\ppp \supseteq R \setminus R^\times$, where~$\ppp \defeq
F^c = A \setminus F$ is the prime ideal associated to~$F$. If~$R$ is local, the
set~$R \setminus R^\times$ is the unique maximal ideal~$\mmm$ of~$R$. Thus~$F$
lies over the filter of units if and only if~$\ppp$ lies over the maximal
ideal.

\begin{prop}Let~$R$ be a ring. Let~$\varphi : R \to A$ be an~$R$-algebra. Then
the points of~$\Spec(A|R)$ are intuitionistically in canonical one-to-one
correspondence with those filters of~$A$ which lie over the filter of units.
\end{prop}

\begin{proof}The correspondence outlined in
Proposition~\ref{prop:points-spectrum} can be adapted to the situation at hand.
A completely prime filter~$K \subseteq \Open(\Spec(A|R))$ corresponds to the
ring-theoretic filter
\[ F \defeq \{ s\?A \,|\, \overline{\sqrt{(s)}} \in K \} \]
and a ring-theoretic filter~$F$ corresponds to the completely prime filter
\[ K \defeq \{ \aaa\?\Open(\Spec(A|R)) \,|\, \text{$\aaa \cap F$ is inhabited} \}. \]
It's instructive to perform some of the necessary verifications, to see how
the quasicoherator is used, even though
Proposition~\ref{prop:local-spectrum-classify} will subsume this
correspondence.

The filter~$F$ corresponding to~$K$ has the displayed property for the
following reason. Let~$\varphi(r) \in F$. We want to verify that~$r$ is
invertible in~$R$. Under the assumption that~$r$ is invertible in~$R$,
it's trivial that~$1$ is an element of
\begin{align*}
  \aaa &\defeq \sup \{ \sqrt{(1)} \,|\, \text{$r$ is invertible in $R$} \} \\
  &\phantom{\vcentcolon}= \{ s\?A \,|\, \text{$s$ is nilpotent or $r$ is invertible in $R$} \}
  \in \Open(\Spec(A)).
\end{align*}
Therefore, without any assumption on~$r$, we have that~$r \cdot 1 = \varphi(r)$ is an
element of~$\overline{\aaa}$ and therefore~$\overline{\sqrt{(\varphi(r))}} \subseteq
\overline{\aaa}$. Since~$K$ is upward-closed, it follows that~$\overline{\aaa}
\in K$. Since~$\overline{\aaa}$ is the supremum of the set~$\{ \sqrt{(1)} \,|\,
\text{$r$ is invertible} \}$ in~$\Open(\Spec(A|R))$ and~$K$ is completely prime, it
follows that this set is inhabited. Thus~$r$ is invertible in~$R$.

The set~$K$ corresponding to a ring-theoretic filter~$F$ is completely prime
for the following reason. Let~$\sup_i \aaa_i = \overline{\sqrt{\sum_i \aaa_i}}
\in K$. Then~$\overline{\sqrt{\sum_i \aaa_i}} \cap F$ is inhabited. By the
special assumption on~$F$, the intersection~$\sqrt{\sum_i \aaa_i} \cap F$ is inhabited
as well: In the case that~$X$ is a scheme, this follows easily using the
description of the quasicoherator given in
Proposition~\ref{prop:quasicoherator-arbitrary-algebra}. In the general case,
we use the proof scheme outlined in
Remark~\ref{rem:quasicoherator-knaster-tarski} -- using the notation of that
remark, if~$P(\bbb) \cap F$ is inhabited, then~$\bbb \cap F$ is as well.

A short calculation using the filter axioms then shows that there
exists an index~$i$ such that~$\aaa_i \cap F$ is inhabited.
\end{proof}

\begin{prop}\label{prop:local-spectrum-classify}
Let~$R$ be a ring. Let~$\varphi : R \to A$ be an algebra. Then~$\Spec(A|R)$ is
the classifying locale of the theory of filters of~$A$ which lie over the
filter of units, that is of the geometric theory with atomic propositions~``$s
\in F$'' for~$s\?A$ and axioms given by the following axiom schemes:
\begin{enumerate}
% Adapt proof if numbering changes
\item $\top \vdash 1 \in F$
\item $st \in F \dashv\vdash s \in F \wedge t \in F$ (two axioms for each $s,t\?A$)
\item $0 \in F \vdash \bot$
\item $s+t \in F \vdash s \in F \vee t \in F$ (one axiom for each $s,t\?A$)
\item $\varphi(r) \in F \vdash \bigvee \{ \top \,|\, \text{$r$ is invertible in
$R$} \}$ (one axiom for each $r\?R$)
\end{enumerate}
\end{prop}

\begin{proof}The frame of the classifying locale of the given theory~$T$ is the
free frame on generators~``$s \in F$'' for~$s\?A$ subject to the relations
given by the axioms of the theory. More explicitly, it's the Lindenbaum
algebra~$L(T)$ of the theory, so its elements are the formulas of the theory up
to provable equivalence and the ordering is defined by~$[\varphi] \preceq
[\psi] \vcentcolon\Leftrightarrow (\varphi \vdash \psi)$. We want to verify that this frame is
isomorphic to~$\Open(\Spec(A|R))$.

We define a frame homomorphism~$L(T) \to \Open(\Spec(A|R))$ by sending the
generators~$[s \in F]$ to the radical ideal~$\overline{\sqrt{(s)}}$. This
respects the relations and therefore gives a well-defined map. The map is
surjective, since a preimage to~$\aaa \? \Open(\Spec(A|R))$
is~$[\bigvee_{s\in\aaa} (s \in F)]$. To verify that it is an isomorphism of
frames, we therefore only have to verify that it reflects the ordering.

By the axiom schemes~(1) and~(2), any formula of~$T$ is provably equivalent to
a formula of the form~$\bigvee_i (s_i \in F)$. It therefore suffices to verify
that, for any families~$(s_i)_i$ and~$(t_j)_j$ such that
$\overline{\sqrt{(s_i)_i}} \subseteq \overline{\sqrt{(t_j)_j}}$, the sequent
$\bigvee_i (s_i \in F) \vdash \bigvee_j (t_j \in F)$ is derivable. We'll show
more generally: If~$\aaa$ and~$\bbb$ are radical ideals such
that~$\overline{\aaa} \subseteq \overline{\bbb}$,
then~$\bigvee_{s\in\aaa}(s\in F) \vdash \bigvee_{t\in\bbb}(t\in F)$. This
follows from the following chain of deductions:
\[ \bigvee_{s\in\aaa}(s\in F) \vdash
  \bigvee_{s\in\overline{\aaa}}(s\in F) \vdash
  \bigvee_{s\in\overline{\bbb}}(s\in F) \vdash
  \bigvee_{s\in\bbb}(s\in F). \]
All but the final step are trivial. The final step is an application of the
general proof scheme outlined in
Remark~\ref{rem:quasicoherator-knaster-tarski}. In the notation of that remark,
we set~$\alpha(\J) \defeq [\bigvee_{s\in\J}(s\in F)]$ and exploit that, if~$s
\in P(\J)$, then~$s \in F \vdash \bigvee_{t\in\J} (t \in F)$. This is
because~$s$ can be written as~$s^n = \sum_j a_j f_j u_j$ such that, for each~$j$,
if~$f_j$ is invertible in~$R$ then~$u_j \in \J$, and we have the
following chain of deductions.
\begin{align*}
  s \in F &\vdash s^n \in F \\
  &\vdash \bigvee_j (t_j f_j u_j \in F) \\
  &\vdash \bigvee_j (\varphi(f_j) \in F \wedge u_j \in F) \\
  &\vdash \bigvee_j \bigl(\bigvee\{\top\,|\,\text{$f_j$ invertible in $R$}\} \wedge u_j \in F\bigr) \\
  &\vdash \bigvee_j \bigvee\{(u_j\in F) \,|\, \text{$f_j$ invertible in $R$}\} \\
  &\vdash \bigvee_{t \in \J} (t \in F). \qedhere
\end{align*}
\end{proof}

\begin{lemma}\label{lemma:universal-property-local-spectrum}
Let~$R$ be a local ring. Let~$\varphi : R \to A$ be an~$R$-algebra.
Then, intuitionistically, the locale~$\Spec(A|R)$ carries a canonical structure
as a locally ringed locale over~$(\pt,R)$ and has the following universal
property: For any locally ringed locale~$(Y,\O_Y)$ over~$(\pt,R)$,
\[ \Hom_{\LRL/(\pt,R)}(Y, \Spec(A|R)) \cong \Hom_{\Alg(R)}(A, \Gamma(Y,\O_Y)). \]
\end{lemma}

\begin{proof}Since~$\Spec(A|R)$ is a sublocale of~$\Spec(A)$, we can
equip~$\Spec(A|R)$ with the restriction of~$\O_{\Spec(A)}$ to~$\Spec(A|R)$ as
the structure sheaf:
\[ \O_{\Spec(A|R)} \defeq
  i^{-1}\O_{\Spec(A)} =
  i^{-1}(\ul{A}[\F^{-1}]) \cong
  (i^{-1}\ul{A})[(i^{-1}\F)^{-1}] \cong
  \ul{A}[(i^{-1}\F)^{-1}]. \]
The generic filter~$\F$ was described in Section~\ref{sect:generic-filter}.
The penultimate isomorphism is because localizing is a geometric construction.
Since locality of a ring is a geometric implication, this structure sheaf is
indeed a local sheaf of rings. Thus~$\Spec(A|R)$ is a locally ringed locale.

Next, we have to describe a morphism~$(\Spec(A|R), \O_{\Spec(A|R)}) \to
(\pt,R)$. Locale-theoretically, this morphism is given by the unique map~$! :
\Spec(A|R) \to \pt$. The ring-theoretic part is given by the composition
\[ !^{-1}R = \ul{R} \longrightarrow
  \ul{A} \longrightarrow
  \ul{A}[(i^{-1}\F)^{-1}] =
  \O_{\Spec(A|R)}. \]
This homomorphism of rings which happen to be local is indeed a local
homomorphism, that is, it reflects invertibility. More precisely,
\[ \Spec(A|R) \models
  \forall f\?\ul{R}\_
  \speak{$\ul{\varphi}(f)$ is \inv\@ in~$\O_{\Spec(A|R)}$} \Rightarrow
  \speak{$f$ is \inv\@ in~$\ul{R}$}. \]
Denoting the modal operator associated to the sublocale inclusion~$\Spec(A|R)
\hookrightarrow \Spec(A)$ by~``$\Box$'', this statement is equivalent to
\[ \Spec(A) \models (\forall f\?\ul{R}\_
  \ul{\varphi}(f) \in \F \Rightarrow \speak{$f$ is \inv\@ in~$\ul{R}$})^\Box \]
by Theorem~\ref{thm:box-translation-semantically} and
Lemma~\ref{lemma:box-translation-sound}. To verify this, let~$s\?A$ and~$f\?R$
be given such that~$\sqrt{(s)} \models \varphi(f) \in \F$, that is,~$s \in
\sqrt{(\varphi(f))}$. We are to show that~$\sqrt{(s)} \models \Box(\speak{$f$
is invertible in~$\ul{R}$})$.

The largest open in~$\Spec(A)$ on which~$\speak{$f$ is invertible in~$\ul{R}$}$
holds is
\begin{align*}
  \aaa &\defeq \sup \{ \sqrt{(1)} \,|\, \text{$f$ is invertible in~$R$} \} \\
  &\phantom{\vcentcolon}=
  \{ t\?A \,|\, \text{$t$ is nilpotent or $f$ is invertible in $R$} \} \in
  \Open(\Spec(A)),
\end{align*}
by Lemma~\ref{lemma:properties-of-constant-sheaves}. Under the assumption
that~$f$ is invertible in~$R$, trivially~$1 \in \aaa$. Therefore, without any
assumptions on~$f$, we have that~$\varphi(f) \in \overline{\aaa}$.
Thus~$\sqrt{(\varphi(f))} \subseteq \overline{\aaa}$ and
therefore~$\sqrt{(\varphi(f))} \models \Box(\speak{$f$ is invertible
in~$\ul{R}$})$. Since~$\sqrt{(s)} \subseteq \sqrt{(\varphi(f))}$, the
monotonicity of the internal language implies~$\sqrt{(s)} \models
\Box(\speak{$f$ is invertible
in~$\ul{R}$})$.

Finally, we verify the universal property. Let~$Y$ be a locally ringed locale
over~$(\pt,R)$ and let a morphism~$A \to \Gamma(Y,\O_Y)$ of~$R$-algebras be
given. We like this data to uniquely induce a morphism~$Y \to \Spec(A|R)$ of
locally ringed locales over~$(\pt,R)$.

To obtain a locale-theoretic map~$f : Y \to \Spec(A|R)$, by
Proposition~\ref{prop:local-spectrum-classify} we need to specify a
filter of~$\ul{A}$ in~$\Sh(Y)$ which lies over the filter of units.
The given morphism~$A \to \Gamma(Y,\O_Y)$ induces a morphism~$\alpha : \ul{A}
\to \O_Y$ in~$\Sh(Y)$. Since~$\O_Y$ is a local ring, the subsheaf~$\O_Y^\times$
is a filter. Its preimage~$F \defeq \alpha^{-1}\O_Y^\times$ is the sought filter of~$\ul{A}$.
It lies over the filter of units because the composition~$\ul{R} \to \ul{A} \to
\O_Y$ is local. By the general theory, the pullback of the generic filter
in~$\Sh(\Spec(A|R))$ to~$\Sh(Y)$ along~$f$ is~$F$.

The ring-theoretic part of the sought morphism~$Y \to \Spec(A|R)$ of locally ringed
locales over~$(\pt,R)$ is the canonical homomorphism
\[ f^{-1}\O_{\Spec(A|R)} =
  f^{-1}(\ul{A}[(i^{-1}\F)^{-1}]) =
  \ul{A}[F^{-1}] \longrightarrow \O_Y \]
of local rings.

This finishes the description of the construction. We omit further
verifications that the construction works as claimed.
\end{proof}

\begin{rem}The modal operator~$\Box$ associated to the inclusion~$\Spec(A|R)
\hookrightarrow \Spec(A)$ can be defined in the internal language
of~$\Sh(\Spec(A))$. Namely, it's the smallest operator such that
the~$\Box$-translated statement
\[ (\speak{the morphism $\ul{R} \to \O_{\Spec(A)}$ is local})^\Box \]
holds. It is thus the smallest operator such that for
any~$f\?\ul{R}$ with~$\ul{\varphi}(f) \in \F$, $\Box(\speak{$f$ is invertible
in $\ul{R}$})$. The sublocale~$\Spec(A|R)$ is therefore the largest
sublocale of~$\Spec(A)$ on which the morphism~$\ul{R} \to \O_{\Spec(A)}$ is
local.
\end{rem}

\begin{proof}[Proof of Theorem~\ref{thm:local-spectrum-yields-relative-spectrum}]
Follows immediately by interpreting the intuitionistic proof of
Lemma~\ref{lemma:universal-property-local-spectrum} in the internal language
of~$\Sh(X)$, applied to~$R \defeq \O_X$ and~$A \defeq \A$.
Then~``$(\pt,\O_X)$'' actually refers to the locally ringed locale~$(X,\O_X)$
and ``$\Gamma(Y,\O_Y)$'' refers to~$\mu_*\O_Y$, where~$\mu : (Y,\O_Y) \to
(X,\O_X)$ is a locally ringed locale over~$(X,\O_X)$.
\end{proof}

Theorem~\ref{thm:local-spectrum-yields-relative-spectrum} settles the question
how the the little Zariski topos of~$\RelSpec_X(\A)$ looks like from the
internal point of view of~$\Sh(X)$. A related question is how the big Zariski
topos looks like. We give the answer in
Theorem~\ref{thm:big-zariski-topos-of-relative-spectrum}.


\subsection{Comparing the different spectrum constructions}

For rings and algebras, there are at least the following spectrum
constructions.

\begin{itemize}
\item The ordinary spectrum of a ring, possibly realized as a locale instead of
a topological space in order to work in an intuitionistic setting: $\Ring^\op \to
\LRS$ or $\Ring^\op \to \LRL$
\item The local spectrum of an algebra: $\Alg(R)^\op \to \LRL/(\pt,R)$
\item Gillam's spectrum of a sheaf of algebras~\cite{gillam:localization}:
$\Alg(\O_X)^\op \to \LRS/(X,\O_X)$
\item Hakim's spectrum of a ringed topos~\cite{hakim:relative-schemes},
yielding a locally ringed topos: $\RT \to \LRT$.
\item Cole's general framework for spectrum constructions~\cite{cole:spectra}
(also reported on at~\cite[Theorem~6.58]{johnstone:topos-theory})
\end{itemize}

These are related as follows.

As described in Section~\ref{sect:rel-spec-as-ordinary-spec}, the ordinary
spectrum construction can not only be applied to rings, but also to sheaves of
rings and indeed ring objects internal to arbitrary elementary toposes equipped
with a natural numbers object, by employing the internal language. Applied to a
ring~$\O$ internal to such a topos~$\E$, it yields a locally ringed locale
internal to~$\E$, or equivalently a locally ringed localic topos internal
to~$\E$.  Externally, this corresponds to a locally ringed topos which is
equipped with a localic geometric morphism to~$\E$.

The ordinary spectrum construction can therefore be used to turn a ringed
topos~$(\E,\O)$ (with a natural numbers object) into a locally ringed topos
(which will be equipped with a morphism of ringed toposes to~$(\E,\O)$, but
which will, even if~$\O$ happens to be a local ring, not be equipped with a
morphism of locally ringed toposes to~$(\E,\O)$).

By comparing the universal properties one sees that this kind of internal
application of the ordinary spectrum construction coincides with the result of
Hakim's spectrum construction. In fact, it can be interpreted as a simultaneous
simplification and generalization of Hakim's construction: It's simpler, since
it's just the familiar spectrum construction and no explicit site calculations
are required; and it's more general, since Hakim's construction only applies to
ringed Grothendieck toposes whereas the internally-performed construction of
the ordinary spectrum applies to ringed elementary toposes with natural numbers
object.

Gillam's spectrum coincides with internally performing the construction of the
local spectrum, with the caveat that Gillam's construction starts with and
yields a locally ringed space, whereas ours starts with and yields a locally
ringed locale.\footnote{More generally, the local spectrum construction can be
applied to any algebra over a local ring~$\O$ internal to an elementary
topos~$\E$ with a natural numbers objects and yields a locally ringed topos
equipped with a morphism of locally ringed toposes to~$(\E,\O)$.} More
precisely:

For a locale~$Y$, let~$Y_P$ be the topological space of points of~$Y$, and for
a topological space~$T$, let~$T_L$ be the induced locale.  Let~$(X,\O_X)$ be a
sober locally ringed topological space. Let~$\A$ be an~$\O_X$-algebra. Then we
have a morphism~$E(\Spec(\A|\O_X)) \to X_L$ of locally ringed locales.
Since~$X \cong (X_L)_P$, there is an induced morphism~$E(\Spec(\A|\O_X))_P \to X$ of
locally ringed spaces. The adjunction~$(\placeholder)_L \dashv
(\placeholder)_P$ relating locales and topological spaces then yields, for any
locally ringed space~$\mu : Y \to X$ over~$X$,
\begin{align*}
  \Hom_{\LRS/X}(Y, E(\Spec(\A|\O_X))_P) &\cong
  \Hom_{\LRL/X_L}(Y_L, E(\Spec(\A|\O_X))) \\
  &\cong \Hom_{\Alg(\O_X)}(\A, \mu_*\O_Y).
\end{align*}
This is precisely the universal property which Gillam's spectrum enjoys.

Cole's framework for spectrum constructions is sufficiently general to
encompass both the ordinary spectrum and the local spectrum, and by extension
Hakim's spectrum and Gillam's spectrum. As is well-known, the ordinary spectrum
can be obtained from Cole's framework by applying it to the geometric
theory~$\mathbb{S}$ of rings, its quotient theory~$\mathbb{T}$ of local rings,
and the admissible class~$\mathbb{A}$ of local homomorphisms (notation as
in~\cite[Theorem~6.58]{johnstone:topos-theory}). The local spectrum can be
obtained by applying it to the geometric theory~$\mathbb{S}$
of~$\O_X$-algebras, its quotient theory~$\mathbb{T}$ of local~$\O_X$-algebras
which are local over~$\O_X$, and the admissible class of local homomorphisms.
For this to make sense, one has to interpret Cole's framework in the internal
language of~$\Sh(X)$, since there are no external geometric theories of
(local)~$\O_X$-algebras.

In general, the local spectrum doesn't coincide with the usual spectrum and
Gillam's spectrum doesn't coincide with Hakim's spectrum. However, if the base
space is a scheme of dimension~$\leq 0$, they do coincide.

\begin{prop}\label{prop:local-spectrum-full-spectrum}
Let~$X$ be a scheme. Then~$E(\Spec(\O_X)) \cong X$ as locales
over~$X$ if and only if~$\dim X \leq 0$.\end{prop}
\begin{proof}The externalization of~$\Spec\O_X$ coincides with~$X$ if and only
if from the internal point of view, the locale~$\Spec\O_X$ coincides with the
one-point locale. By interpreting Lemma~\ref{lemma:spectrum-one-point} in the
internal language of~$\Sh(X)$, it follows that this is the case if and only if
\[ \Sh(X) \models \forall f\?\O_X\_ \speak{$f$ nilpotent} \vee \speak{$f$
invertible}. \]
(Internally, it always holds that~$\neg(1 = 0)$ in~$\O_X$, even if~$X$ happens
to be the empty scheme. Therefore the lemma is indeed applicable.) By
Corollary~\ref{cor:scheme-dimension-zero}, this condition is equivalent to the
dimension of~$X$ being less than or equal to zero (\ie to~$X$ being empty or
having dimension exactly zero).
% Since this condition is a geometric implication, it is fulfilled in the
% internal language if and only if it is fulfilled at every point~$x \in X$.
% This amounts to requiring that in any local ring~$\O_{X,x}$, the unique
% maximal ideal is the only prime ideal, \ie that all local rings~$\O_{X,x}$
% have Krull dimension zero. This is equivalent to~$X$ being empty or having
% dimension zero.
\end{proof}

\begin{cor}Let~$X$ be a scheme. Then the relative spectrum of~$\O_X$-algebras
can be calculated by the internal spectrum (instead of the internal local
spectrum) if and only if~$\dim X \leq 0$.\end{cor}
\begin{proof}The externalization of the internal spectrum of
arbitrary~$\O_X$-algebras~$\A$ coincides with the relative spectrum if and
only if it coincides in the special case~$\A = \O_X$. This is apparent by the
universal properties of both constructions. Thus the claim follows from
Proposition~\ref{prop:local-spectrum-full-spectrum}.
\end{proof}

Which construction is more fundamental, the ordinary spectrum of a ring or the
local spectrum of an algebra? The ordinary spectrum~$\Spec(A)$ can be expressed
as the local spectrum~$\Spec(A^\sim|\O_{\Spec(\ZZ)})$, where~$A^\sim$ is the
induced quasicoherent algebra on~$\Sh(\Spec(\ZZ))$. This fact is well-known
in the alternate form~``$\RelSpec_{\Spec(\ZZ)}(A^\sim) \cong \Spec(A)$''.

Fast and loose reasoning as follows could lead one to believe that it's
similarly possible to express the local spectrum as an ordinary spectrum.
Let~$R$ be a local ring. Let~$\varphi : R \to A$ be an algebra. The points
of~$\Spec(A|R)$ are those filters~$F \subseteq A$ such that~$\varphi^{-1}F =
R^\times$. Illicitly assuming classical logic, the points of~$\Spec(A|R)$ are in
canonical one-to-one correspondence with those prime ideals~$\ppp \subseteq A$
such that~$\varphi^{-1}\ppp = \mmm_R$. The points of~$\Spec(A|R)$ are therefore
in canonical one-to-one correspondence with the points of~$\Spec(A \otimes_R
k)$, where~$k = R/\mmm_R$ is the residue field of~$R$. Therefore~$\Spec(A|R)$
and~$\Spec(A \otimes_R k)$ might coincide.

However, we have the following negative result.\footnote{Intuitionistically,
it's still true that the prime ideals of a quotient ring~$A/\ppp$ are in
one-to-one correspondence with those prime ideals of~$A$ which contain~$\ppp$.
However, the analogous statement ``filters of~$A/F$ correspond to those filters
of~$A$ which are contained in~$F$'' can't be shown intuitionistically, if~$A/F$
is defined as~$A/F^c$. However, informally speaking, this failure is not the
fault of the statement, but of the definition of~$A/F$. The definition
raises red flags from an intuitionistic point of view, since not~$F$, but only
its complement~$F^c$ enters the construction.

The statement can be salvaged by defining~``$A/F$'' to mean the set~$A$
equipped with a new \emph{apartness relation} defined by $a \apart b
\vcentcolon\Leftrightarrow a-b \in F$. (A basic example for a
ring-with-apartness-relation is the field of real numbers equipped with~$x
\apart y \vcentcolon\Leftrightarrow \exists q \in \QQ\_ |x-y| \geq q > 0$.) A filter~$G$ of this
ring-with-apartness-relation~$A$ is by definition a subset~$G \subseteq A$ which
verifies the filter axioms and which is \emph{open with respect to the
apartness relation} in that for any elements~$a,b \? A$, the implication~$a \in
G \Rightarrow (b \in G) \vee (a \apart b)$ holds.

This construction provides one of several motivations for developing the theory
of rings using apartness relations and anti-ideals; one can even define the
spectrum of a ring-with-apartness-relation. However, we'll not pursue these
ideas further here.}

\begin{prop}In general, the local spectrum of an algebra can't be expressed as
an ordinary spectrum.\end{prop}

\begin{proof}It is well-known that the ordinary spectrum is always quasicompact. The local spectrum,
however, can fail to be quasicompact. A quick way to see this is to notice
that, if that was the case, the locale-theoretic part of the projection
morphism~$\RelSpec_X(\A) \to X$ would always be a proper map of
locales~\cite{vermeulen:locales}.

There's also a more direct way of seeing this, which in fact proves a slightly
stronger statement. Let~$X$ be a scheme. Let~$f\in\Gamma(X,\O_X)$.
From the internal point of view of~$\Sh(X)$, the local
spectrum~$\Spec(\O_X[f^{-1}]|\O_X) \hookrightarrow \Spec(\O_X|\O_X) \cong \pt$
is the open sublocale of~$\pt$ corresponding to the truth value of~``$f$~is
invertible''.  Explicitly, the frame of opens of~$\Spec(\O_X[f^{-1}]|\O_X)$ is
isomorphic to $\{ \psi \? \Omega \,|\, \psi \Rightarrow \text{$f$ is
invertible} \}$.

The ordinary spectrum always has the Frobenius reciprocity property, being
quasicompact. In contrast, the locale~$\Spec(\O_X[f^{-1}]|\O_X)$ has this
property if and only if~$f$ is nilpotent or invertible.
\end{proof}

Finally, we want to restate the universal properties of the ordinary spectrum
and the local spectrum in ring-theoretic language, employing the dual
categories~$\RL^\op$ and~$\LRL^\op$, as in
Section~\ref{sect:flat-constructible-topologies}.

Let~$A$ be a ring. The morphism~$A \to \O_{\Spec(A)}$ in~$\RL^\op$ (the
ring-theoretic part of the canonical morphism~$(\Spec(A), \O_{\Spec(A)}) \to
(\Set, A)$) is the \emph{universal localization} of~$A$: The
ring~$\O_{\Spec(A)}$ is local, and for any morphism~$A \to \B$ into a local
ring~$\B$ (over any locale), there is a unique local morphism~$\O_{\Spec(A)}
\to \B$ rendering the diagram
\[ \xymatrix{
  A \ar[rd] \ar[rrr] &&& {\substack{\text{local}\\\text{\normalsize$\B$}\\\phantom{\text{local}}}} \\
  & {\substack{\text{\normalsize$\O_{\Spec(A)}$}\\\text{local}}} \ar@{-->}_[@!29]{\text{local}}[rru]
} \]
commutative. In contrast, the universal property of the local spectrum is as
follows. Let~$R$ be a ring. Let~$A$ be an~$R$-algebra. The morphism~$A \to
\O_{\Spec(A|R)}$ is the universal way of turning~$A$ into a local ring
\emph{which is local over~$R$}: The ring~$\O_{\Spec(A|R)}$ is local, the
composition~$R \to A \to \O_{\Spec(A|R)}$ is local, and for any morphism~$A
\to \B$ into a local ring (over any locale) such that the composition~$R \to A
\to \B$ is local, there is a unique local morphism~$\O_{\Spec(A|R)} \to \B$
such that the diagram
\[ \xymatrix{
  R \ar[r]\ar@/^2pc/[rrrr]^{\text{local}}\ar@/_/[rrd]_[@!-26]{\text{local}} &
    A \ar[rd] \ar[rrr] &&&
    {\substack{\text{local}\\\text{\normalsize$\B$}\\\phantom{\text{local}}}} \\
  && {\substack{\text{\normalsize$\O_{\Spec(A|R)}$}\\\text{local}}} \ar@{-->}[rru]_[@!28]{\text{local}}
} \]
commutes.

\begin{rem}It's possible to state the universal property of the structure sheaf
of the big Zariski topos of a ring~$A$, more precisely of the canonical
morphism~$(\Zar(A),\affla) \to (\Set,A)$ of ringed toposes, in a similar manner,
employing the dual categories~$\RT^\op$ and~$\LRT^\op$ of the categories of
(locally) ringed toposes. However, unlike the universal property of the
spectrum, this universal property looks slightly odd from an algebraic point of
view: For any morphism~$A \to \B$ into a local ring (over any topos~$\E$), there is
a unique bijective homomorphism~$\affla \to \B$ rendering the diagram
\[ \xymatrix{
  A \ar[rd] \ar[rrr] &&& {\substack{\text{local}\\\text{\normalsize$\B$}\\\phantom{\text{local}}}} \\
  & {\substack{\text{\normalsize$\affla$}\\\text{local}}} \ar@{-->}_[@!31]{\text{bijective}}[rru]
} \]
commutative. By ``bijective'' we mean that the ring-theoretic part~$f^\sharp :
f^{-1}\affla \to \B$ of the morphism~$f:(\E,\B) \to (\Zar(A),\affla)$ is
bijective as seen from the internal point of view of~$\E$.
\end{rem}


\subsection{The spectrum of the generic ring}

Let~$\Set[\Ring]$ be the classifying topos of the theory of
rings; explicitly, it's the topos of preshaves on~$\Ring_\fp^\op$, the dual of
the category of finitely presented rings. This topos contains the \emph{generic
ring}~$U$ (explicitly the presheaf~$R \mapsto R$): any ring in any topos is the
pullback of~$U$ along a suitable geometric morphism.

Let~$\Set[\LocRing]$ be the classifying topos of the theory of local rings.
Explicitly, it's the big Zariski topos~$\Zar(\Spec(\ZZ))$ (built using one of
the \emph{parsimonious sites}, as described in
Section~\ref{sect:proper-choice-of-site}). This topos contains the
\emph{generic local ring}~$U'$: any local ring in any topos is the pullback
of~$U'$ along a suitable geometric morphism.

Let~$A$ be a ring. By the universal property of~$\Set[\Ring]$, there is a
geometric morphism~$g : \Set \to \Set[\Ring]$ such that~$g^{-1}U \cong A$.
Since~$U'$ is in particular a ring, again by the universal property
of~$\Set[\Ring]$, there is a geometric morphism~$f : \Set[\LocRing] \to
\Set[\Ring]$ such that~$f^{-1}U \cong U'$.
By the universal property of~$\Set[\LocRing]$, the topos of
sheaves over the spectrum of~$A$ admits a geometric morphism~$g'$
to~$\Set[\LocRing]$ such that~$(g')^{-1}U' \cong \O_{\Spec(A)}$.

The resulting solid diagram
\[ \xymatrix{
  \E \ar@/^1pc/@{.>}[rrd]^{\tilde g} \ar@/_1pc/@{.>}[ddr]_{\tilde f} \ar@{.>}[rd]^h \\
  & \Sh(\Spec(A)) \ar[r]^{g'} \ar[d]_{f'} & \Set[\LocRing] \ar[d]^f \\
  & \Set \ar[r]_g \ar@{}[ur]^(.3){}="a"^(.7){}="b" \ar@{=>}_\eta "a";"b" & \Set[\Ring]
} \]
commutes up to a non-invertible natural transformation~$\eta$; under the
equivalence
\begin{multline*}
  \qquad\text{category of geometric morphisms~$\Sh(\Spec(A)) \to \Set[\Ring]$} \simeq \\
  \text{category of ring objects in~$\Sh(\Spec(A))$}\qquad
\end{multline*}
this transformation corresponds to the non-invertible localization homomorphism~$\ul{A} \to
\ul{A}[\F^{-1}] = \O_{\Spec(A)}$. It is folklore that this square is a lax
pullback square in the 2-category of Grothendieck toposes (for instance, this
is reported on at~\cite{arndt:lax-pullback}); however, this is not true.

Given a topos~$\E$ together with geometric morphisms~$\tilde f : \E \to \Set$
and~$\tilde g : \E \to \Set[\LocRing]$ and a natural transformation~$\tilde
\eta : \tilde f^{-1} \circ g^{-1} \Rightarrow \tilde g^{-1} \circ f^{-1}$
(these data correspond to a local ring~$\O_\E$ in~$\E$ together with a ring
homomorphism~$\varphi : \ul{A} \to \O_\E$), there is a canonical geometric morphism~$h :
\E \to \Sh(\Spec(A))$ (determined by requiring that~$h^{-1}\F \cong \F_0 \defeq
\varphi^{-1}[\O_\E^\times]$), and this morphism renders the lower left triangle
commutative up to a natural isomorphism, but it renders the upper right
triangle commutative only up to a non-invertible natural transformation
(corresponding to the non-invertible ring homomorphism~$\ul{A}[\F_0^{-1}] \to
\O_E$).

The observation that the square is not a lax pullback is joint with Peter Arndt
and Matthias Hutzler. The observation raises two questions: What is the lax
pullback (which exists by general theory), if it's not~$\Sh(\Spec(A))$?
And how can~$\Sh(\Spec(A))$ be described as a pullback? The following two
propositions answer these questions. The geometric morphism~$\Set \to
\Set[\Ring]$ which they implicitly refer to is the morphism~$g$ mentioned
above.

\begin{prop}Let~$A$ be a ring. The lax pullback~$(\Set \Rightarrow_{\Set[\Ring]}
\Set[\LocRing])$ is the big Zariski topos of~$\Spec(A)$ (built using one of the
parsimonious sites, as described in Section~\ref{sect:proper-choice-of-site}).
\end{prop}

\begin{proof}The claim can be checked by hand, but it's more instructive to
employ the general theory of classifying toposes. In the situation
\[ \xymatrix{
  (\Set[T] \Rightarrow_{\Set[T_0]} \Set[T']) \ar[r] \ar[d] & \Set[T'] \ar[d]^f \\
  \Set[T] \ar[r]_g \ar@{}[ur]^(.3){}="a"^(.7){}="b" \ar@{=>}_\eta "a";"b" & \Set[T_0],
} \]
where~$T_0$, $T$, and~$T'$ are arbitrary geometric theories, the lax pullback
classifies the geometric theory whose models consist of a model~$M$ of~$T$, a
model~$N$ of~$T'$, and a homomorphism~$G(M) \to F(N)$ of~$T_0$-models. The
constructions~$G$ and~$F$ are given by the geometric morphisms~$g$ and~$f$:

Any object of~$\Set[T]$ can be obtained by geometric constructions from~$U_T$,
the universal model of~$T$ in~$\Set[T]$. In particular, the pullback~$g^{-1}
U_{T_0}$, which is a model of~$T_0$, can be obtained by geometric constructions
from~$U_T$. Therefore the geometric morphism~$g$ displays a way to turn the
generic model of~$T$ into a model of~$T_0$ using only geometric constructions.
The same constructions can be applied to any model~$M$ of~$T$, yielding a
model~$G(M)$ of~$T_0$.

In the concrete situation at hand, the theory~$T$ is the empty theory
(admitting in any topos a unique model~$M$), the theory~$T'$ is the theory of local
rings, and~$T_0$ is the theory of rings. The~$T_0$-model~$G(M)$ is the
ring~$A$. The~$T_0$-model~$F(N)$ of a local ring~$N$ is the underlying ring
of~$N$.

Therefore the lax pullback~$(\Set \Rightarrow_{\Set[\Ring]} \Set[\LocRing])$
classifies ring homomorphisms~$A \to R$ where~$R$ is a local ring, that is,
local~$A$-algebras. It's well-known that~$\Zar(\Spec(A))$ classifies these as
well.
\end{proof}

\begin{prop}\label{prop:spectrum-as-pullback}
Let~$A$ be a ring. The pullback of the spectrum of the generic ring
along~$\Set \to \Set[\Ring]$ is the spectrum of~$A$.
\end{prop}

\begin{proof}There are two related ways of making the statement precise.
Firstly, the spectrum of the generic ring~$U$ can be interpreted as a (locally
ringed) locale internal to~$\Set[\Ring]$. Locales can be pulled back along
geometric morphisms (even though the pullback of a frame along a geometric
morphism typically fails to be a frame)~\cite{vickers:case-study}. In this
way~$\Spec(U)$ pulls back to a locale internal to~$\Set$, that is an ordinary
external locale. The claim is that this locale is canonically isomorphic
to~$\Spec(A)$.

A second way to interpret the statement of the proposition is to regard the
spectrum of the generic ring as a localic geometric morphism with
codomain~$\Set[\Ring]$. The claim is then that the diagram
\[ \xymatrix{
  \Sh(\Spec(A)) \ar[r] \ar[d] & \Sh_{\Set[\Ring]}(\Spec(U)) \ar[d] \\
  \Set \ar[r]_g & \Set[\Ring]
} \]
is a pullback diagram in the 2-category of toposes.

Using the language of classifying locales and classifying toposes, both claims
are easy to establish. The pulled-back locale (or topos) classifies the
pulled-back geometric theory~\cite[Corollary~5.4]{vickers:case-study}. Since
the description of the theory which~$\Spec(U)$ classifies -- the theory of
filters of~$U$ -- is itself geometric, the pulled-back theory is the theory of
filters of~$g^{-1}U \cong A$.\footnote{In the notation
of~\cite[Section~5]{vickers:case-study}, the theory of filters of~$U$ is
represented by a GRD-system with~$G = U$ and $R = 1 \amalg U^2 \amalg U^2 \amalg 1
\amalg U^2$ (one summand for each axiom scheme).}
\end{proof}

\begin{prop}\label{prop:local-spectrum-generic}\begin{enumerate}
\item Let~$A$ be an~$R$-algebra. The local spectrum~$\Spec(A|R)$ is the
pullback of~$\Spec(U''|R)$, where~$U''$ is the \emph{generic~$R$-algebra}
contained in the classifying topos~$\E$ of~$R$-algebras, along the geometric
morphism~$\Set \to \E$ given by~$A$.
\item Let~$X$ be a scheme (or a locally ringed locale). Let~$\A$ be
an~$\O_X$-algebra. The relative spectrum~$\RelSpec_X(\A)$ is the pullback
of~$\Spec(U''|\O_X)$, where~$U''$ is the generic~$\O_X$-algebra contained in
the classifying $\Sh(X)$-topos~$\E$ of~$\O_X$-algebras, along the geometric
morphism~$\Sh(X) \to \E$ given by~$\A$.
\end{enumerate}
\end{prop}

\begin{proof}Straightforward modification of the proof of
Proposition~\ref{prop:spectrum-as-pullback}.
\end{proof}

\begin{rem}The big Zariski topos~$\Zar(\Spec(A))$ can be obtained as the
pullback of the big Zariski topos of the generic ring~$U$, if both toposes
are understood to be defined using the parsimonious sites as described in
Section~\ref{sect:proper-choice-of-site}.
\end{rem}


\subsection{Limits in the category of locally ringed locales}

The category of ringed locales has small limits, by the naive construction. For
instance, the fiber product~$X \times_Z Y$ of ringed locales is given by the
fiber product of the underlying locales and the
structure sheaf~$\pi_X^{-1}\O_X \otimes_{\pi_Z^{-1}\O_Z} \pi_Y^{-1}\O_Y$. More
generally, the limit of a small diagram of ringed locales is given by the limit~$L$
of the underlying locales and the colimit of the pulled-back structure sheaves
(calculated in the category of sheaves of rings on~$L$).

However, when applied to a diagram of locally ringed locales, the ringed locale
which this simple construction yields is in general not locally ringed. This
can be nicely understood from the internal point of view: Let~$R$ be a local
ring. Let~$R \to A$ and~$R \to B$ be local~$R$-algebras which are furthermore
local over~$R$. Then the tensor product~$A \otimes_R B$ is in general not a
local ring. Indeed, this fails even in the easiest case, where all rings
involved are fields: The rings~$\RR$ and~$\CC$ are local, and the
inclusion~$\RR \to \CC$ is local, but~$\CC \otimes_\RR \CC \cong \CC
\otimes_\RR \RR[X]/(X^2+1) \cong \CC[X]/(X^2+1) \cong \CC \times
\CC$ is not.

The following proposition explains that the true limit in the category of
locally ringed locales is obtained by \emph{relocalizing} the limit in the
category of ringed locales.

\begin{prop}\label{prop:lrl-complete}
The category of locally ringed locales has all small limits.
\end{prop}

\begin{proof}For notational simplicity, we describe how products in the
category of locally ringed locales can be constructed. The general case is
entirely analogous.

Let~$X$ and~$Y$ be locally ringed locales. Their product~$P$ as ringed locales has
two defects: Firstly, it's not locally ringed. Secondly, the ring-theoretic
parts of the projection morphisms~$\pi_X : P \to X$ and~$\pi_Y : P \to Y$
aren't local, that is, don't reflect invertibility.

The first issue could be solved by constructing, internally to~$\Sh(P)$, the
ordinary spectrum of~$\O_P$. From the external point of view, this would yield
a locally ringed locale equipped with morphisms of ringed, but not of locally
ringed, locales to~$X$ and~$Y$.

To solve both issues, we need to employ a refined spectrum construction,
similar to the modification required by the internal account of the relative
spectrum: Internally to~$\Sh(P)$, we construct the classifying locale of the
theory of those filters of~$\O_P$ which simultaneously lie over the filter of
units of~$\pi_X^{-1}\O_X$ and which lie over the filter of units
of~$\pi_Y^{-1}\O_Y$. This locale is a sublocale of~$\Spec(\O_P)$, the largest
such that the morphisms to~$(\pt,\pi_X^{-1}\O_X)$ and to~$(\pt,\pi_Y^{-1}\O_Y)$ are
morphisms of locally ringed locales.

The externalization of the internal locally ringed locale obtained in this way
is the sought product of~$X$ and~$Y$ in the category of locally ringed locales.
\end{proof}

\begin{rem}The category of locally ringed locales embeds as a (non-full)
coreflective subcategory into the category of ringed locales; the coreflector
maps a ringed locale~$(X,\O_X)$ to the externalization of~$\Spec(\O_X)$
(constructed internally to~$\Sh(X)$). However, as is familiar in situations where the embedding
is not full~\cite{adamek:rosicky:reflective}, it's in general not the case that limits in~$\LRL$ are calculated by
applying the coreflector to the limit calculated in~$\RL$. Employing the language
of the proof of Proposition~\ref{prop:lrl-complete}, applying the coreflector only
solves the first issue, but not the second.
\end{rem}

It's instructive to determine the points of limits in~$\LRL$, even though a
locale is of course not determined by its points. For instance, the construction in
Proposition~\ref{prop:lrl-complete} shows that the points of the product~$X
\times Y$ of locally ringed locales in~$\LRL$ are in canonical one-to-one
correspondence with tuples~$(x,y,F)$, where~$x$ is a point of~$X$,~$y$ is a
point of~$y$, and~$F$ is a filter of~$\O_{X,x} \otimes_\ZZ \O_{Y,y}$ which lies
over the filter of units of~$\O_{X,x}$ and of~$\O_{Y,y}$. In classical logic,
those tuples are in canonical one-to-one correspondence with
tuples~$(x,y,\ppp)$, where~$x$ and~$y$ are as before and~$\ppp$ is a prime
ideal of~$k(x) \otimes_\ZZ k(y)$.

Similarly, points of the fiber product~$X \times_Z Y$ are in canonical
one-to-one correspondence with tuples~$(x,y,F)$, where~$x$ is a point of~$X$
and $y$ is a point of~$y$ such that both map to the same point~$z$ of~$Z$,
and~$F$ is a filter of~$\O_{X,x} \otimes_{\O_{Z,z}} \O_{Y,y}$ lying over the
filter of units of~$\O_{X,x}$ and of~$\O_{Y,y}$ (and therefore automatically
of~$\O_{Z,z}$). In classical logic, those tuples are in canonical one-to-one
correspondence with tuples~$(x,y,\ppp)$, where~$x$ and~$y$ are as before
and~$\ppp$ is a prime ideal of~$k(x) \otimes_{k(z)} k(y)$.

\begin{rem}By the adjunction~$(\placeholder)_L \dashv (\placeholder)_P$
relating locales and topological spaces, limits of locally ringed spaces which
happen to be sober can be calculated by regarding them as locally ringed
locales by~$(\placeholder)_L$, calculating their limit in~$\LRL$, and taking
the associated topological space of the limit by~$(\placeholder)_P$.

Small diagrams of arbitrary locally ringed spaces admit limits as well.
Indeed, the proof of Proposition~\ref{prop:lrl-complete} was adapted from
Gillam's proof of this fact~\cite[Corollary~5]{gillam:localization}.\end{rem}


\subsection{Relative Proj construction} Similar issues as with the relative
spectrum arise with the Proj construction: The standard definition of the Proj
construction as a topological space of homogeneous prime ideals gives rise to a
space which can't intuitionistically be shown to satisfy the expected
universal property. The construction has to be reimagined as a locale
instead of a topological space. A certain sublocale of this locale then yields
the relative Proj construction when interpreted in the internal language of
the little Zariski topos of a base scheme (or a locally ringed locale).

\begin{defn}The \emph{Proj construction} of an~$\NN$-graded ring~$S$ is the
locale with frame of opens given by
\begin{multline*}
  \qquad\Open(\Proj(S)) \defeq
    \{ \aaa \subseteq S \,|\,
      \text{$\aaa$ is a homogeneous radical ideal such that} \\
  \forall x\?S\_
    x S_+ \subseteq \aaa \Rightarrow x \in \aaa \},\qquad
\end{multline*}
where~$S_+ = \bigoplus_{i > 0} S_i$ is the irrelevant ideal.
\end{defn}

A quick way to see that the partial order~$\Open(\Proj(S))$ is a frame is to
recognize that it's the frame of opens of a sublocale of~$\Spec(S)$.
The associated nucleus~$j : \Open(\Spec(S)) \to \Open(\Spec(S))$ is given by
\[ j(\aaa) \defeq
  (\sqrt{\aaa^h} : S_+), \]
where~$\aaa^h$ is the homogenization of~$\aaa$, the ideal of~$S$ generated by
all homogeneous components of the elements of~$\aaa$. Since~$\aaa \subseteq
\aaa^h \subseteq \sqrt{\aaa^h} \subseteq j(\aaa)$, a radical ideal~$\aaa$
is an element of~$\Open(\Proj(S))$ if and only if~$\aaa = j(\aaa)$.

One way to derive this definition is to start, within a classical context, with
the general expression for the nucleus associated to the subspace of~$\Spec(S)$
consisting of those prime ideals which are homogeneous and don't
contain~$S_+$, and then rewrite this expression to not refer to prime ideals.

\begin{defn}A filter~$F \subseteq S$ in an~$\NN$-graded ring~$S$ is
\emph{homogeneous} if and only if, for any element~$a \? S$, the filter~$F$
contains~$a$ if it contains at least one of the homogeneous components of~$a$.
It \emph{meets the irrelevant ideal} if and only if~$F \cap S_+$ is
inhabited.\end{defn}

In classical logic, a subset is a homogeneous filter meeting the irrelevant
ideal if and only if its complement is a homogeneous prime ideal not containing
the irrelevant ideal. Intuitionistically, neither direction can be shown.

\begin{prop}\label{prop:proj-classifying-locale}
Let~$S$ be an~$\NN$-graded ring. Then~$\Proj(S)$ is the classifying
locale of any of the following geometric theories.
\begin{enumerate}
\item The theory of homogeneous filters of~$S$ meeting the irrelevant ideal,
that is the theory of Remark~\ref{rem:theory-of-filters} supplemented by the
following two axiom schemes:
\begin{itemize}
\item $\bigvee_i (a_i \in F) \vdash a \in F$ (one axiom for each
decomposition~$a = \sum_i a_i$ of an element of~$S$ into homogeneous components)
\item $\top \vdash \bigvee_{a \in S_+} (a \in F)$ (one axiom)
\end{itemize}
% Change text below if numbering changes
\item The theory given by one atomic proposition~``$a \in F_i$'' for each
homogeneous element~$a$ of degree~$i$ in~$S$ and axioms given by the following
axiom schemes:
\begin{itemize}
\item $\top \vdash 1 \in F_0$ (one axiom)
\item $st \in F_{i+j} \dashv\vdash s \in F_i \wedge t \in F_j$ (two axioms for
each $i, j \geq 0$, $s \in S_i$, $t \in S_j$)
\item $0 \in F_i \vdash \bot$ (one axiom for each~$i \geq 0$)
\item $s+t \in F_i \vdash s \in F_i \vee t \in F_i$ (one axiom for each $i \geq
0$, $s,t \in A_i$)
\item $\top \vdash \bigvee_{i \geq 1} \bigvee_{a \in S_i} (a \in F_i)$ (one
axiom)
\end{itemize}
\item The same theory as in~(2), but with atomic propositions only for
homogeneous elements of degree~$\geq 1$ and without the first axiom~``$\top
\vdash 1 \in F_0$''.
\end{enumerate}
\end{prop}

\begin{proof}That~$\Proj(S)$ coincides with the classifying locale of the
theory given in~(1), can be verified by a direct calculation. By the general
theory, the nucleus associated to the quotient theory given in~(1) maps a
radical ideal~$\aaa \? \Open(\Spec(S))$ to the least fixed point above
of~$\aaa$ of the map
\[ \bbb \longmapsto
  \bbb \vee
  \bigvee_{a \? S} \Bigl(\sqrt{(a_i)_i} \cap \bigl(\sqrt{(a)} \rightarrow \bbb\bigr)\Bigr) \vee
  \Bigl(\sqrt{(a)_{a \in S_+}} \to \bbb\Bigr), \]
where~$(\ccc \to \bbb) = (\bbb : \ccc)$ is the Heyting implication
and~``$\vee$'' is the join in~$\Open(\Spec(S))$. We omit the intermediate steps
of the calculation.

The theories given in~(1) and in~(2) are bi-interpretable. The interpretation
of the atomic propositions~``$a \in F_i$'' of theory~(2) using the signature of
theory~(1) is~``$a \in F$''. Verifying the axioms is straightforward.
Conversely, the interpretation of~``$a \in F$'' in the signature of theory~(2)
is~``$\bigvee_i (a_i \in F_i)$'', where~$a = \sum_i a_i$ is the decomposition
into homogeneous components. For verifying the axioms, one needs the lemma that
\[ \bigvee_i (s_i \in F_i)  \wedge  \bigvee_j (t_j \in F_j)
    \ \dashv\vdash\ \bigvee_n \Bigl(\sum_{i+j=n} s_i t_j \in F_n\Bigr) \]
is derivable in theory~(2), for any decompositions~$s = \sum_i s_i$ and~$t =
\sum_j t_j$ of elements of~$S$ into homogeneous components. In the
guise~``$\sqrt{(s_i)_i} \cap \sqrt{(t_j)_j} = \sqrt{(\sum_{i+j=n} s_i t_j)_n}$''
this is a familiar fact on the content of
polynomials~\cite[Proposition~1]{banaschewski:vermeulen:radical-content}.

Also theories~(2) and~(3) are bi-interpretable. The interpretation of~``$a \in
F_0$'' in the signature of theory~(3) is~``$\bigvee_{i \geq 1} \bigvee_{h \in
S_i} (ha \in F_i)$''.
\end{proof}

\begin{cor}Let~$S$ be an~$\NN$-graded ring. The points of~$\Proj(S)$ are in
canonical one-to-one correspondence with the homogeneous filters
of~$S$ meeting the irrelevant ideal.\end{cor}

\begin{proof}Points of~$\Proj(S)$ are given by models of the theory of
homogeneous filters of~$S$ meeting the irrelevant ideal in~$\Set$.
\end{proof}

\begin{rem}The same presentation as in
Proposition~\ref{prop:proj-classifying-locale}(3) has been used to
construct~$\Proj(S)$ not as a locale, but as a distributive
lattice~\cite{cls:projective-spectrum}.\end{rem}

\begin{defn}Let~$S$ be an~$\NN$-graded ring. The \emph{generic homogeneous
filter meeting the irrelevant ideal} is the subsheaf~$\F \hookrightarrow
\ul{S}$ over~$\Proj(S)$ generated by the sections~$a$ over~$D_+(a) \defeq j(\sqrt{(a)})$.
\end{defn}

Equivalently, the generic homogeneous filter meeting the irrelevant ideal is
the pullback of the generic filter in~$\Sh(\Spec(S))$ to~$\Sh(\Proj(S))$.

\begin{defn}Let~$S$ be an~$\NN$-graded ring. The structure sheaf of~$\Proj(S)$ is the homogeneous
localization~$\ul{S}[\F^{-1}]_0$ of the ring~$\ul{S}$ at the generic
homogeneous filter meeting the irrelevant ideal, that is the degree-zero part
of~$\ul{S}[\F^{-1}]$. The \emph{tilde construction} of a graded~$S$-module~$M$
is~$M^\sim \defeq \ul{M}[\F^{-1}]_0$.
\end{defn}

The locally ringed locale~$\Proj(S)$ and the tilde construction defined in this
way enjoy their familiar properties. For instance, we have the following lemma.

\begin{lemma}Let~$S$ be an~$\NN$-graded ring.
\begin{enumerate}
\item Let~$f \? S$ be homogeneous of degree~$d \geq 1$. Then~$D_+(h) \cong
\Spec(S[f^{-1}]_0)$.
\item Assume that~$S$ is generated as an~$S_0$-algebra by~$S_1$. Let~$M$
and~$N$ be graded~$S$-modules. Then~$M^\sim \otimes_{\O_{\Proj(S)}} N^\sim
\cong (M \otimes_S N)^\sim$.
\item Under the same assumption as in~(2), the twisting sheaves~$\O(m) \defeq
(S(m))^\sim$ are finite locally free of rank~$1$.
\end{enumerate}
\end{lemma}

\begin{proof}For the first statement, it suffices to verify that the theories
of homogeneous filters of~$S$ meeting the irrelevant ideal and containing~$h$
and of filters of~$S[f^{-1}]_0$ are bi-interpretable. It's slightly more
convenient to use the presentation given by
Proposition~\ref{prop:proj-classifying-locale}(2) for the former theory.

The interpretation of~``$q \in F$'' for~$q \? S[f^{-1}]_0$ in the signature of
the theory given by Proposition~\ref{prop:proj-classifying-locale}(2) is
\[ \bigvee\{ (x \in F_{di}) \,|\, \text{$q = x/f^i$ for some $x \? S$, $i \geq
0$} \}. \]
Conversely, the interpretation of~``$a \in F_i$'' in the signature of the
theory of filters of~$S[f^{-1}]_0$ is~``$x^d / h^i \in F$''.

The second statement follows from the calculation
\begin{align*}
  M^\sim \otimes_{\O_{\Proj(S)}} N^\sim &=
  \ul{M}[\F^{-1}]_0 \otimes_{\ul{S}[\F^{-1}]_0} \ul{N}[\F^{-1}]_0 \\
  &\cong (\ul{M} \otimes_{\ul{S}} \ul{N})[\F^{-1}]_0 
  \cong (\ul{M \otimes_S N})[\F^{-1}]_0
  = (M \otimes_S N)^\sim.
\end{align*}
The first isomorphism maps~$x/s \otimes y/t$ to~$(x \otimes y)/(st)$. By the
assumption that~$S$ is generated as an~$S_0$-algebra by~$S_1$, the generic
filter contains a homogeneous element~$h$ of degree~$1$ from the internal point
of view of~$\Sh(\Proj(S))$. Therefore the
map has an inverse sending~$(a \otimes b) / u$, where~$a$ and~$b$ are
homogeneous of degrees~$i$ and~$j$, to~$(h^j a)/u \otimes b/h^j$. The second
isomorphism is because the tensor product is a geometric construction and
therefore commutes with constructing the constant sheaf.

For the proof of the third statement, we show that~$(S(m))^\sim$ is a finite
free module of rank~$1$ from the internal point of view. We again use that the
generic filter contains a homogeneous element~$h \? \ul{S}$ of degree~$1$ from
the internal point of view. Such an element allows to define an
isomorphism~$\O_{\Proj(S)} = \ul{S}[\F^{-1}]_0 \to \ul{S(m)}[\F^{-1}]_0 = \O(m)$ by
mapping~$x/s$ to~$(h^m x)/s$ if~$m \geq 0$ and to~$x/(h^{-m} s)$ otherwise.
\end{proof}

\begin{defn}Let~$R$ be a ring. Let~$S$ be an~$\NN$-graded~$R$-algebra. The
\emph{local Proj construction} of~$S$ over~$R$ is the sublocale~$\Proj(S|R)$
of~$\Proj(S)$ with frame of opens given by
\[
  \Open(\Proj(S|R)) \defeq
    \{ \aaa \? \Open(\Proj(S)) \,|\,
      \forall f\?R\_ \forall s\?S\_
        (\speak{$f$ \inv} \Rightarrow s \in \aaa) \Rightarrow fs \in \aaa \}
\]
and with the pullback of~$\O_{\Proj(S)}$ as the structure sheaf.
\end{defn}

\begin{prop}\label{prop:local-proj-classifying-locale}
Let~$R$ be a ring. Let~$S$ be an~$\NN$-graded~$R$-algebra.
Then the local Proj construction~$\Proj(S|R)$ is the classifying locale of the theory of homogeneous
filters of~$S$ meeting the irrelevant ideal and lying over the filter of
units.\end{prop}

\begin{proof}Direct calculation similar to the proof of
Proposition~\ref{prop:proj-classifying-locale}.\end{proof}

Since pullback and localization commute, the structure sheaf of~$\Proj(S|R)$
can also be described as~$\ul{S}[\F^{-1}]_0$, where by abuse of notation we
mean by~``$\F$'' the pullback of the generic filter on~$\Proj(S)$
to~$\Proj(S|R)$. This filter has the special property
\[ \Sh(\Proj(S|R)) \models
  \forall r\?\ul{R}\_
    r \in \F \Rightarrow \speak{$r$ \inv\@ in $\ul{R}$}. \]

\begin{thm}Let~$X$ be a scheme (or a locally ringed locale). Let~$\S$ be
an~$\NN$-graded $\O_X$-algebra. Then the externalization~$E(\Proj(\S|\O_X))$ coincides
with the relative Proj construction~$\RelProj_X(\S)$ as locally ringed locales
over~$X$.\end{thm}

\begin{proof}For simplicity, we assume that~$\S$ is generated as
an~$\S_0$-algebra by~$\S_1$. In this case, the expected universal property
of the relative Proj construction is that it's a locally ringed locale over~$X$ such
that, for all locally ringed locales~$\mu : Y \to X$ over~$X$, the
set~$\Hom_{\LRL/X}(Y, \RelProj_X(\S))$ is canonically isomorphic (by pullback
of the standard such datum on~$\RelProj_X(\S)$) to the set of pairs~$(\L,
\psi)$ such that
\begin{itemize}
\item $\L$ is a line bundle on~$Y$ and
\item $\psi : \mu^*\S \to \bigoplus_{n\geq0} \L^{\otimes n}$ is a graded
morphism of~$\O_Y$-algebras such that the degree-$1$ part of~$\psi$ is a
surjective morphism~$\mu^*\S_1 \to \L$
\end{itemize}
modulo equivalence. For instance, it is known that this property is satisfied
in the case that~$X$ is a scheme and~$\S$ is
quasicoherent~\stacksproject{01O4}.

We verify that~$E(\Proj(\S|\O_X))$ enjoys the same property, even if~$X$ is not
a scheme or~$\S$ is not quasicoherent. For the rest of the proof, we switch to
the internal universe of~$\Sh(X)$.

The local Proj construction is a locally ringed locale over~$(\pt, \O_X)$ by
the unique morphism~$! : \Proj(\S|\O_X) \to \pt$ of locales and by the canonical
morphism~$!^\sharp : \ul{\O_X} \to \ul{\S}_0 \to \ul{S}[\F^{-1}]_0 = \O_{\Proj(\S|\O_X)}$
of local rings.

As the standard datum on~$\Proj(\S|\O_X)$, we choose the line bundle~$\O(1)$
(pulled back to~$\Proj(\S|\O_X)$) together with the canonical morphism~$!^* \S
\to \oplus_{n \geq 0} \O(1)^{\otimes n}$.

Let~$Y$ be a locally ringed locale over~$(\pt, \O_X)$. Let a pair~$(\L,\psi)$
be given. In the internal language of~$\Sh(Y)$, we define a filter by the formula
\[ \F' \defeq \{ s \? \ul{S} \,|\,
  \speak{there exists~$i$ such that~$(\psi(s_i \otimes 1))$ is a basis
  of~$\L^{\otimes i}$} \} \subseteq \ul{S}, \]
where~$s_i$ refers to the homogeneous component of~$s$ of degree~$i$.
Since~$\L^{\otimes i}$ is finite free of rank~$1$, a one-element family
in~$\L^{\otimes i}$ is a basis if and only if it's a generating family. This
observation can be repeatedly used to verify that~$\F'$ is
homogeneous, meets the irrelevant ideal, and lies over the filter of
units. Since~$\Proj(\S|\O_X)$ is the classifying locale of such filters
(Proposition~\ref{prop:local-proj-classifying-locale}), we obtain a
morphism~$f : Y \to \Proj(\S|\O_X)$ of locales which is unique with the property
that~$f^{-1}\F = \F'$.

To obtain a morphism~$Y \to \Proj(\S|\O_X)$ of locally ringed locales, it
remains to define a morphism~$f^\sharp : f^{-1}\O_{\Proj(\S|\O_X)} =
\ul{S}[\F'^{-1}]_0 \to \O_Y$. A canonical choice is
\[ x/s \mapsto \speak{
  the coefficient of~$\psi(x \otimes 1)$
  with respect to the basis~$(\psi(s \otimes 1))$}. \]
We omit further verifications.
\end{proof}


\section{Higher direct images and other derived functors}

\subsection{Flabby sheaves}

Recall that a sheaf~$\F$ of sets on a topological space (or a locale)~$X$ is
\emph{flabby} if and only if, for any open subset~$U \subseteq X$ the
restriction map~$\F(X) \to \F(U)$ is surjective.

Flabbiness of a sheaf is a local property, even though it doesn't seem like that
at first sight: If the restrictions~$\F|_{U_i}$ of~$\F$ to the members of an
open covering~$X = \bigcup_i U_i$ are flabby, then the verification that~$\F$ is
flabby can't proceed as follows. ``Let~$s \in \F(U)$ be an arbitrary section.
Since each~$\F|_{U_i}$ is flabby, the section~$s|_{U \cap U_i}$ extends to a
section on~$U_i$.'' The reason is that the individual extensions obtained in
this way might not glue.

A correct proof employs Zorn's lemma in a typical way, considering a maximal
extension and then verifying that the subset this maximal extension is defined
on is all of~$X$.

Since flabbiness is a local property, it's not unreasonable to expect that
flabbiness can be characterized in the internal language. The following
proposition shows that this is indeed the case.

\begin{prop}\label{prop:internal-char-flabbiness}
Let~$\F$ be a sheaf of sets on a topological space~$X$ (or a locale).
Then the following statements are equivalent:
\begin{enumerate}
\item $\F$ is flabby.

\item ``Any section of~$\F$ can be locally extended'':
For any open subset $U \subseteq X$ and any section $s \in \F(U)$ there is
an open covering $X = \bigcup_i V_i$ such that, for each $i$, there is an
extension of $s$ to $U \cup V_i$ (that is, a section $s' \in \F(U \cup V_i)$
such that $s'|_U = s$).

(If $X$ is a space instead of a locale, this can be equivalently formulated as
follows: For any open subset $U \subseteq X$, any section $s \in \F(U)$, and any
point $x \in X$, there is an open neighbourhood $V$ of $x$ and an extension of
$s$ to $U \cup V$.)

% Proof of "internally injective ==> externally injective" references the
% following condition by number.
\item From the point of view of the internal language of~$\Sh(X)$, for any
subsingleton $K \subseteq \F$ there exists an element $s \? \F$ such that~$s \in
K$ if~$K$ is inhabited. More precisely,
\begin{multline*}
  \qquad\qquad\Sh(X) \models
  \forall K \subseteq \F\_
  (\forall s,s'\?K\_ s = s') \Longrightarrow \\
  \exists s\?\F\_ (\text{$K$ is inhabited} \Rightarrow s \in K).\qquad\qquad
\end{multline*}

\item The canonical map $\F \to \P_{\leq 1}(\F), s \mapsto \{s\}$ is
final from the internal point of view, that is
\[ \Sh(X) \models
  \forall K \? \P_{\leq 1}(\F)\_
  \exists s \? \F\_
  K \subseteq \{s\}, \]
where $\P_{\leq 1}(\F)$ is the object of subsingletons of $\F$.
\end{enumerate}
\end{prop}

\begin{proof}
The implication~``(1)~$\Rightarrow$~(2)'' is trivial. The converse direction uses a
typical argument with Zorn's lemma, considering a maximal extension. The
equivalence~``(2)~$\Leftrightarrow$~(3)'' is routine, using the Kripke--Joyal
semantics to interpret the internal statement. Condition~4 is a straightforward
reformulation of Condition~3.
\end{proof}

Condition~2 of the proposition is, unlike the standard definition of flabbiness,
manifestly local. Also its equivalence with Condition~3 and Condition~4 is
intuitionistically valid; therefore one might consider to adopt Condition~2 as
the definition of flabbiness.

The object~$\P_{\leq 1}(\F)$ of subsingletons of $\F$ can be
interpreted as the object of \emph{partially-defined
elements} of $\F$. In this view, the empty subset is the maximally undefined
element and a singleton is a maximally defined element. The proposition shows
that~$\F$ is flabby if and only if any such partially-defined element can be
refined to an honest element of~$\F$.


\subsection{Injective sheaves}

Recall that an object~$I$ of a category~$\C$ is \emph{injective} if and only if,
for any monomorphism~$X \to Y$ in~$\C$ and any morphism~$X \to I$, there
is a lifting such that the diagram
\[ \xymatrix{
  X \ar@{^{(}->}[r]\ar[d] & Y \ar@{-->}[dl] \\
  I
} \]
commutes. Equivalently, an object~$I$ is injective if and only if the Hom
functor~$\Hom_\C(\placeholder, I) : \C^\op \to \Set$ maps monomorphisms in~$\C$ to
surjective maps. This general definition is often specialized to one of these cases:
to the category of modules over a ring, to the category of set-valued sheaves
on a topological space, and to the category of sheaves of~$\O_X$-modules on a
ringed space~$(X,\O_X)$.

The definition is seldomly applied in the category of sets, since in a classical
context it's easy to show that a set is injective if and only if it's
inhabited, thereby completely settling the question which objects are
injective in a trivial manner.

The question is more interesting in an intuitionistic setting, since
intuitionistically one cannot prove that inhabited sets are
injective~\cite{aczel-et-al:injective}; but one can still verify that any set embeds
into an injective set: The powerset~$\P(X)$ and even the smaller
set~$\P_{\leq1}(X)$ of subsingletons of a given set~$X$ are injective. We verify
this in the proof of Lemma~\ref{lemma:enough-flabby}.

For a cartesian closed category~$\C$, there is also the notion of an
\emph{internally injective} object. This is an object~$I$ such that the
internal Hom functor~$[\placeholder, I] : \C^\op \to \C$ maps monomorphisms
in~$\C$ to epimorphisms. In the specical case that~$\C$ is a elementary topos
with a natural numbers object, such as the topos of set-valued sheaves on a
space, this condition can be rephrased in several ways. The following
proposition lists five of these conditions. The equivalence of the first four is
due to Harting~\cite{harting}.

\begin{prop}\label{prop:notions-of-internal-injectivity}
Let~$\E$ be an elementary topos with a natural numbers object. Then
the following statements about an object~$I \in \E$ are equivalent.
\begin{enumerate}
\item $I$ is internally injective.
\item The functor~$[\placeholder, I] : \E^\op \to \E$ maps monomorphisms in $\E$
to morphisms for which any global element of the target locally (after change of
base along an epimorphism) possesses a preimage.
\item For any morphism $p : A \to 1$ in $\E$, the object $p^*I$ has property~(1)
as an object of $\E/A$.
\item For any morphism $p : A \to 1$ in $\E$, the object $p^*I$ has property~(2)
as an object of $\E/A$.
\item From the point of view of the internal language of~$\E$, the object~$I$
is injective.\footnote{In Section~\ref{sect:internal-language}, we have only
introduced the internal language for sheaf toposes. The general definition is
in~\cite[Section~7]{shulman:stack}.}
\end{enumerate}
\end{prop}

\begin{proof}
The implications ``(1)~$\Rightarrow$~(2)'', ``(3) $\Rightarrow$~(4)'', ``(3)
$\Rightarrow$~(1)'', and ``(4)~$\Rightarrow$~(2)'' are trivial.

The equivalence ``(3)~$\Leftrightarrow$~(5)'' follows directly from the
interpretation rules of the stack semantics.

The implication ``(2)~$\Rightarrow$~(4)'' employs the
extra left adjoint $p_! : \E/A \to \E$ of $p^* : \E
\to \E/A$~(which maps an object~$(X \to A)$ to~$X$), as in the usual proof that
injective sheaves remain injective when
restricted to smaller open subsets: We have that $p_* \circ [\placeholder, p^*I]_{\E/A}
\cong [\placeholder, I]_\E \circ p_!$, the functor $p_!$ preserves monomorphisms, and one
can check that $p_*$ reflects the property that global elements locally possess
preimages. Details are in~\cite[Thm.~(1)1]{harting}.

The implication ``(4)~$\Rightarrow$~(3)'' follows by performing an extra change of
base, since any non-global element becomes a global element after a suitable
change of base.
\end{proof}

Somewhat surprisingly, and in stark contrast with the situation for internally
projective objects (which are defined dually), internal injectivity coincides
with external injecticity for sheaf toposes over spaces.

\begin{thm}\label{thm:char-injectivity}
Let~$X$ be a topological space (or a locale). An object~$\I \in \Sh(X)$ is
injective if and only if it is internally injective.
\end{thm}

\begin{proof}For the ``only if'' direction, let~$\I$ be an injective sheaf of
sets. Then~$\I$ satisfies Condition~(2) in
Proposition~\ref{prop:notions-of-internal-injectivity}, even without having to
pass to covers.

For the ``if'' direction, let~$\I$ be an internally injective object. Let~$m :
\E \to \F$ be a monomorphism in~$\Sh(X)$ and let~$k : \E \to \I$ be an arbitrary
morphism. We want to show that there exists an extension $\F \to \I$ of~$k$
along~$m$. To this end, we consider the sheaf defined by the internal expression
\[ \G \defeq \brak{\{ k' \? [\F,\I] | k' \circ m = k \}}. \]
Global sections of~$\G$ are extensions of the kind we're looking for.
Therefore it suffices to show that~$\G$ is flabby. We do this by verifying
Condition~(3) of Proposition~\ref{prop:internal-char-flabbiness} in the internal
language of~$\Sh(X)$.

Let~$K \subseteq \G$ be a subsingleton. We consider the injectivity diagram
\[ \xymatrix{
  m[\E] \cup \F' \ar@{^{(}->}[r]\ar[d] & \F \ar@{-->}[ld] \\
  \I,
} \]
where~$\F' \defeq \{ s \? \F \,|\, \text{$K$ is inhabited} \}$ and the solid
vertical arrow is defined in the following way: It should map an element~$s \in
\F'$ to~$k'(s)$, where~$k'$ is any element of~$K$; and it should map an
element~$m[u] \in m[\E]$ to~$k(u)$. These prescriptions determine a well-defined
map.

Since~$\I$ is injective from the internal point of view we're taking up here,
there exists a dotted map rendering the diagram commutative. This map is an
element of~$\G$. Furthermore, this map is an element of~$K$, if~$K$ is
inhabited.
\end{proof}

\begin{thm}\label{thm:char-injectivity-modules}
Let~$(X,\O_X)$ be a ringed topological space (or a ringed locale).
An~$\O_X$-module~$\I$ is injective if and only if it is internally injective.
\end{thm}

\begin{proof}Proposition~\ref{prop:notions-of-internal-injectivity} can be
adapted from sheaves to sets to sheaves of modules, with the same proof.
\XXXh{check that.}

The proof of Theorem~\ref{thm:char-injectivity} can be adopted as well.
It suffices to change~``$m[\E] \cup \F'$'' to~``$m[\E] + \F''$'', where~$\F''
\defeq \{ s \? \F \,|\, \text{$s = 0$ or $K$ is inhabited} \}$.
\end{proof}

\XXX{Remark that proof requires AxC and that it's a unique feature of sheaf toposes.
Counterexample?}


\subsection{Internal proofs of common lemmas}

\begin{lemma}A sheaf of sets or a sheaf of modules is injective if and only if it
is locally injective.\end{lemma}

\begin{proof}By Theorem~\ref{thm:char-injectivity} respectively
Theorem~\ref{thm:char-injectivity-modules}, injectivity can be characterized in the
internal language. Any such property is local.\end{proof}

\begin{lemma}Let~$X$ be a topological space (or a locale).
\begin{enumerate}
\item Let~$\I$ be an injective sheaf of sets over~$X$. Let~$\F$ be an arbitrary
sheaf of sets. Then~$\HOM(\F,\I)$ is flabby.
\item Let~$\I$ be an injective sheaf of modules over some sheaf~$\O_X$ of rings
over~$X$. Let~$\F$ be an arbitrary sheaf of modules.
Then~$\HOM_{\O_X}(\F,\I)$ is flabby.
\end{enumerate}
\end{lemma}

\begin{proof}
We cover the case of sheaf of sets first. By Theorem~\ref{thm:char-injectivity}
and Proposition~\ref{prop:internal-char-flabbiness}, it suffices to give an
intuitionistic proof of the following statement: If~$I$ is an injective set
and~$F$ is an arbitrary set, then partially defined elements of the set~$[F,I]$
of all maps~$F \to I$ can be refined to honest elements.

Thus let a subsingleton~$K \subseteq [F,I]$ be given. We consider the
injectivity diagram
\[ \xymatrix{
  F' \ar[r]\ar[d] & F \ar@{-->}[ld] \\
  I
} \]
where~$F'$ is the subset~$\{ s \? F \,|\, \text{$K$ is inhabited} \} \subseteq F$ and the
solid vertical map sends~$s \in F'$ to~$f(s)$, where~$f$ is an arbitrary element
of~$K$. This association is well-defined. Since~$I$ is injective, a dotted lift
as indicated exists. If~$K$ is inhabited, this lift is an element of~$K$.

The same kind of argument applies to the case of sheaves of modules, relying on
Theorem~\ref{thm:char-injectivity-modules} and defining~$F'$ as the
submodule~``$\{ s \? F \,|\, \text{$s = 0$ or $K$ is inhabited} \}$''.
\end{proof}

\begin{cor}Injective sheaves of sets and injective sheaves of modules are
flabby.\end{cor}

\begin{proof}Follows from the previous lemma by considering the special cases~$\F
\defeq 1$ respectively~$\F \defeq \O_X$.\end{proof}

\begin{lemma}\label{lemma:enough-flabby}
Let~$X$ be a topological space (or a locale). Any sheaf of sets
over~$X$ can be embedded into an injective (therefore flabby) sheaf of sets.
\end{lemma}

\begin{proof}By Proposition~\ref{prop:internal-char-flabbiness}, it suffices to
give an intuitionistic proof of the following statement: Any set~$F$ can be
embedded into an injective set.

As already indicated there at least two simple ways that~$F$ can be embedded
into an injective set: by embedding~$F$ in its powerset~$\P(F)$ or by
embedding~$F$ in~$\P_{\leq1}(F)$, the set of subsingletons of~$F$. For
conciseness, we only verify that~$\P_{\leq1}(F)$ is injective.

So let~$m : A \hookrightarrow B$ be an injective map and let~$k : A \to
\P_{\leq1}(F)$ be an arbitrary map. Then we can extend~$k$ to a map~$k' : B \to
\P_{\leq1}(F)$ by defining for~$y \? B$
\begin{align*}
  k'(y) &\defeq \bigcup k[m^{-1}[\{y\}]] \\
  &\phantom{\vcentcolon}= \{ s\?F \,|\, \text{$s \in k(x)$ for some~$x \in A$ such that~$m(x) = y$} \}.
  \qedhere
\end{align*}
\end{proof}

\begin{rem}The \emph{Godement construction} is a well-known way of embedding an
inhabited sheaf of sets~$\F$ into an injective sheaf, namely embedding it into
the sheaf of not necessarily continuous sections of the étale space of~$\F$:
\[ U \subseteq X \quad\longmapsto\quad
  \prod_{x \in U} \F_x. \]
The sheaf~$\P_{\leq1}(\F)$ does not coincide with this construction.
Instead by Definition~\ref{defn:interpretation-internal-constructions}, it is the sheaf with
\[ U \subseteq X \quad\longmapsto\quad
  \{ \langle V, s \rangle \,|\,
    \text{$V \subseteq U$ open, $s \in \F(V)$} \}. \]
It's not possible to describe the Godement construction in the internal language
of~$\Sh(X)$, since in the Godement construction the underlying set of~$X$
enters. But the sheaf topos of~$X$ doesn't remember this set. For instance, if~$X$
is an inhabited indiscrete topological spac, then~$\Sh(X)$ is equivalent
to~$\Set$.
\end{rem}

\begin{rem}It's not known to me whether it's possible to intuitionistically
prove that any module can be embedded into a module which satisfies the internal
flabbiness criterion of Proposition~\ref{prop:internal-char-flabbiness}. This
would give an internal proof that any sheaf of modules can be embedded into a
flabby sheaf of modules. The naive candidates don't work: The
set~$\P_{\leq1}(F)$ doesn't admit a canonical module structure, and the free
module over that set is not flabby in general.

Since by the Godement construction the statement is true in many models of
intuitionistic logic, the sheaf toposes over topological spaces, and
furthermore the proof that the Godement construction yields a flabby sheaf is
intuitionistically valid,\footnote{In order for the Godement construction to
work in a intuitionistic metatheory, one has to tweak its definition a little
bit.  Instead of mapping an open subset~$U$ to~$\prod_{x \in U} \F_x$, one has
to map~$U$ to~$\prod_{x \in U} \P_{\leq1}(\F_x)$. This has the added advantage
that it works even if~$\F$ is not inhabited.} it's not unreasonable to believe
that such an intuitionistic proof is possible.

On the other hand, it's certainly not possible to intuitionistically prove that
any module can be embedded into an injective module, since it's consistent with
Zermelo--Fraenkel set theory that no nontrivial injective abelian groups
exist~\cite{blass:inj-proj-axc}.
\end{rem}

\begin{lemma}Let~$X$ be a ringed space (or a ringed locale). Let~$0 \to \E'
\xra{\alpha} \E \xra{\beta} \E'' \to 0$ be a short exact sequence
of~$\O_X$-modules. If~$\E'$ is flabby, then the induced sequence
\[ 0 \longrightarrow \Gamma(X,\E')
  \longrightarrow \Gamma(X,\E)
  \longrightarrow \Gamma(X,\E'') \longrightarrow 0 \]
is exact.
\end{lemma}

\begin{proof}Since taking global sections is left exact (being a right adjoint
functor), it suffices to verify that the map~$\Gamma(X,\E) \to \Gamma(X,\E'')$
is surjective. We'll do this by showing, in the internal language of~$\Sh(X)$,
that the sheaf of preimages of a given global section~$s \in \Gamma(X,\E'')$ is
flabby and therefore has a global section.

In the internal language, this sheaf has the description~$F \defeq \{ u \? \E
\,|\, \beta(u) = s \}$. To verify the internal condition of
Proposition~\ref{prop:internal-char-flabbiness}, let a subsingleton~$K \subseteq
F$ be given. Since~$\beta$ is surjective, there is a preimage~$u_0 \in F$.
The translated set~$K - u_0 \subseteq \E$ is still a subsingleton, and its
preimage under~$\alpha$ is as well. By the assumption on~$\E'$, there is an
element~$v \? \E$ such that~$v \in \alpha^{-1}[K - u_0]$ if~$K - u_0$ is
inhabited. We'll now verify that~$u_0 + \alpha(v) \in K$ if~$K$ is inhabited.

So assume that~$K$ is inhabited. Then~$K - u_0$ is as well. Since the image of
its unique element under~$\beta$ is zero and the given sequence is exact, the
set~$\alpha^{-1}[K - u_0]$ is inhabited as well. Therefore~$v \in \alpha^{-1}[K
- u_0]$. Thus~$u_0 + \alpha(v) \in K$.
\end{proof}

\begin{lemma}Let~$X$ be a ringed space (or a ringed locale). Let~$0 \to \E'
\xra{\alpha} \E \xra{\beta} \E'' \to 0$ be a short exact sequence
of~$\O_X$-modules. If~$\E'$ and~$\E''$ are flabby, then~$\E$ is flabby as well.
\end{lemma}

\begin{proof}We verify the condition of
Proposition~\ref{prop:internal-char-flabbiness} in the internal language
of~$\Sh(X)$.

Let~$K \subseteq \E$ be a subsingleton. Then its image~$\beta[K] \subseteq \E''$
is a subsingleton as well. Since partial elements of~$\E''$ can be refined to
honest elements, there is an element~$s \? \E''$ such that~$\beta[K] \subseteq
\{ s \}$.

Since~$\beta$ is surjective, there is an element~$t_0 \? \E$ such
that~$\beta(t_0) = s$.

The preimage~$\alpha^{-1}[K - t_0] \subseteq \E'$ is a subsingleton. This
partial element can be refined to an honest element, that is there exists an
element~$u \? \E'$ such that~$\alpha^{-1}[K - t_0] \subseteq \{u\}$.

The partial element~$K$ can thereby refined to the honest element~$t \defeq t_0 + \alpha(u)$.
\end{proof}

\XXX{Higher direct images}

\XXX{Ext, Tor}


\chapter{The big Zariski topos}\label{part:big-zariski}

The preceding part demonstrated that working in the internal universe of
the little Zariski topos of a scheme~$S$, the topos of sheaves on~$S$, is
useful for simplifying local work on~$S$. The basic tenet was that sheaves of
modules look just like plain modules and that theorems of intuitionistic
algebra yield theorems about sheaves.

But the little Zariski topos is not particularly suited for dealing with
\emph{schemes} over~$S$. For this, we need a related topos. For the scope of
this introduction only, we employ the following slightly problematic
definition which we'll fix in Section~\ref{sect:proper-choice-of-site}.
We'll keep the base scheme~$S$ fixed throughout this part.


\section{Basics}

\begin{defn}[provisional]The \emph{big Zariski topos}~$\Zar(S)$ of a scheme~$S$ is the
topos of sheaves on the Grothendieck site~$\Sch/S$ of schemes over~$S$.
\end{defn}

Explicitly, an object of~$\Zar(S)$ is a functor~$F : (\Sch/S)^\op \to \Set$
satisfying the gluing condition with respect to ordinary Zariski coverings:
If~$X = \bigcup_i U_i$ is a cover of an~$S$-scheme~$X$ by open subsets, the
canonical diagram
\[ F(X) \longrightarrow \prod_i F(U_i) \xbigtoto{} \prod_{j,k} F(U_j \cap U_k)
\]
should be an equalizer diagram.


{\tocless

\subsection*{Internal language} Just like the topos of sheaves on a topological
space or on a locale admits an internal language, so does the big Zariski topos.
The necessary modifications of the Kripke--Joyal semantics
(Definition~\ref{defn:kripke-joyal}) are straightforward. Instead of
defining recursively the meaning of~``$U \models \varphi$'' for open subsets~$U
\subseteq S$, we define the meaning of~``$T \models \varphi$''
for~$S$-schemes~$T$ and slightly rewrite the rules for implication and universal quantification.
Instead of
\begin{align*}
  U \models \varphi \Rightarrow \psi \quad&\Ll\quad
  \text{for all open~$V \subseteq U$:} \\
  &\qquad\qquad\qquad\qquad\text{$V \models \varphi$ implies $V \models \psi$} \\
  U \models \forall s\?\F\_ \varphi(s) \quad&\Ll\quad
  \text{for all sections~$s \in \Gamma(V, \F)$ on open $V \subseteq U$:} \\
  &\qquad\qquad\qquad\qquad V \models \varphi(s) \\
\intertext{they have to read as follows.}
  T \models \varphi \Rightarrow \psi \quad&\Ll\quad
  \text{for all morphisms~$T' \to T$ in~$\Sch/S$:} \\
  &\qquad\qquad\qquad\qquad\text{$T' \models \varphi$ implies $T' \models \psi$} \\
  T \models \forall s\?F\_ \varphi(s) \quad&\Ll\quad
  \text{for all morphisms~$T' \to T$ in~$\Sch/S$ and} \\
  &\phantom{{}\Ll\quad\text{for }} \text{all sections~$s \in \Gamma(T', F)$:} \\
  &\qquad\qquad\qquad\qquad T' \models \varphi(s)
\end{align*}

The analogs of Proposition~\ref{prop:locality-of-the-internal-language} and
Proposition~\ref{prop:soundness-of-the-internal-language} are true for the internal
language of the big Zariski topos:

\begin{prop}\label{prop:basic-properties-language-big}
Let~$T$ be an~$S$-scheme and~$\varphi$ be a formula over~$T$.
\begin{enumerate}
\item If~$T \models \varphi$ and if there is an intuitionistic proof
that~$\varphi$ implies a further formula~$\psi$, then~$T \models \psi$.
% Adapt text below if numbering changes
\item Let~$T' \to T$ be a morphism of~$S$-schemes. If~$T \models \varphi$,
then~$T' \models \varphi$.
\item If~$T = \bigcup_i T_i$ is an open covering and if~$T_i \models \varphi$
for all~$i$, then~$T \models \varphi$.
\end{enumerate}
\end{prop}

\begin{proof}The proofs of
Proposition~\ref{prop:locality-of-the-internal-language} and
Proposition~\ref{prop:soundness-of-the-internal-language} carry over.
\end{proof}

When working with the internal language of the little Zariski topos, we often
used the fact that if a formula holds on some open subset~$U$, then it also
holds on all open subsets contained in~$U$.
Proposition~\ref{prop:basic-properties-language-big}(2) states a stronger version
of this: All properties which can be expressed using the internal language of
the big Zariski topos are automatically \emph{stable under base change}.

\subsection*{Important objects in the big Zariski topos}\label{page:important-objects} It's convenient to
introduce notation for objects which often appear when working with the big
Zariski topos.

Let~$X$ be an~$S$-scheme. Its functor of points, which maps
an~$S$-scheme~$T$ to~$\Hom_S(T,X)$, is an object of~$\Zar(S)$. We denote it
by~``$\ul{X}$''.

From the internal point of view of~$\Zar(S)$, such a functor~$\ul{X}$ looks
like a single set. It can be pictured as the ``set of points of~$X$'',
where~``point'' doesn't mean ``point of the underlying topological space
of~$X$'', but rather~``$T$-point of~$X$'', where~$T$ varies over
all~$S$-schemes. The internal language of the big Zariski topos hides any
explicit mentions of the stage~$T$; it is therefore a device for reifying the
multitude of points of~$X$, defined on varying stages, as a single entity.

Particularly important is~$\affl$, the functor of points of the affine line
over~$S$. The object~$\ul{S}$ is the terminal object in~$\Zar(S)$. This fits
into the philosophy: From the point of view of the big Zariski topos, the base
scheme should simply look like a point. The functor of points of~$S \amalg S$
looks like a two-element set from the internal point of view.

Let~$\F$ be a sheaf of sets on~$S$. For reasons explained in
Section~\ref{sect:relation-big-little}, we denote by~``$\pi^{-1}(\F)$''
the induced sheaf on~$\Sch/S$ mapping an~$S$-scheme
$(f : T \to S)$ to~$\Gamma(T, f^{-1}(\F))$.

Let~$\F$ be a sheaf of~$\O_S$-modules. We denote by~``$\F^\Zar$''
the induced sheaf on~$\Sch/S$ mapping an~$S$-scheme
$(f : T \to S)$ to~$\Gamma(T, f^*(\F))$.


\subsection*{A first example illustrating the Kripke--Joyal translation rules}
Since all the sets~$\affl(T) \cong \Gamma(T,\O_T)$ carry ring structures, the
object~$\affl$ can be endowed with a canonical structure as a ring object
in~$\Zar(S)$. For a particular~$S$-scheme~$T$, the ring~$\affl(T)$ isn't
necessarily a field, but the system of these rings, conceptualized as a single
entity from the internal point of view, does satisfy a field axiom. In the
case~$S = \Spec\ZZ$, this was first observed by
Kock~\cite{kock:univ-proj-geometry}.

\begin{prop}The ring~$\affl$ is a field from the internal point of view
of~$\Zar(S)$, in the sense that
\[ \Zar(S) \models
  \forall f\?\affl\_
    \neg(f = 0) \Rightarrow \speak{$f$ \inv}. \]
\end{prop}

\begin{proof}According to the Kripke--Joyal semantics of~$\Zar(S)$, we
have to show that for any~$S$-scheme~$T$ and any function~$f \in
\Gamma(T,\O_T)$ the statement~$T \models \neg(f = 0)$ implies~$T \models
\speak{$f$ \inv}$. The antecedent states, for any~$T$-scheme~$T'$, that if the
pullback of~$f$ to~$T'$ vanishes,~$T'$ is then the empty scheme.

As with the analogous statement of the little Zariski topos
(Lemma~\ref{lemma:internal-invertibility}), the consequent means that~$f$ is
invertible in~$\Gamma(T,\O_T)$.

The claim follows by considering the particular~$T$-scheme~$T' \defeq V(f)$.
Since~$f$ vanishes on~$V(f)$, this subscheme is empty and therefore its
complement~$D(f)$ is all of~$T$.
\end{proof}

The field property can be interpreted as follows. A function~$f$ not being the
zero function does not imply that it's invertible. But if~$f$ is
\emph{universally nonzero} in that the only scheme such that pullback of~$f$ to
that scheme vanishes is the empty scheme, then~$f$ is indeed invertible.

}


\section{On the proper choice of a big Zariski site}
\label{sect:proper-choice-of-site}

Unlike with the construction of the little Zariski topos, set-theoretical
issues of size arise when constructing the big Zariski topos. These can be
solved in several different manners, yielding toposes which are not equivalent,
and actually differ in some important aspects, but otherwise enjoy very similar
properties.

{\tocless

\subsection*{Naive approach}
Some authors construct the big Zariski topos of~$S$ as the topos of
sheaves over the site~$\Sch/S$ of all schemes over~$S$. This option is quite
attractive since the Yoneda embedding~$\Sch/S \to \Sh(\Sch/S)$, which sends
an~$S$-scheme to its functor of points, is fully faithful, therefore the
internal language of~$\Sh(\Sch/S)$ can distinguish arbitrary schemes.

However, since~$\Sch/S$ is not essentially small, forming the sheaf topos is
not possible in plain Zermelo--Fraenkel set theory.

Since it's still possible to meaningfully speak of individual
functors~$(\Sch/S)^\op \to S$, we can attach a Kripke--Joyal semantics
to~$\Sh(\Sch/S)$, as long as we keep in mind that~$\Sh(\Sch/S)$ might
not contain a subobject classifier and might not be cartesian closed. From the
internal point of view, powersets and function sets might therefore not exist.

\subsection*{Using Grothendieck universes}
We could also assume the existence of a Gro\-then\-dieck universe~$\U$
containing~$S$ and construct~$\Zar(S)$ as the topos of sheaves over the
small site~$\Sch_\U/S$, the category of~$S$-schemes contained in~$U$.

By the \emph{comparison lemma}~\ref{XXX}, we could also construct~$\Zar(S)$ as
the topos of sheaves over~$\Aff_\U/S$, the category of~$S$-schemes in~$U$ which
are affine (as absolute schemes), and obtain an equivalent topos.

In this case, the Yoneda functor~$\Sch/S \to \Zar(S)$ might not be faithful,
but the restricted Yoneda functor~$\Sch_\U/S \to \Zar(S)$ will.

\subsection*{Approach of the Stacks Project}
The Stacks Project proposes a more nuanced approach, namely expanding a given
set~$M_0$ of schemes containing~$S$ to a superset~$M$ which is closed (up to
isomorphism) under several constructions~\cite[Tag~000H]{stacks-project}: fiber
products, countable coproducts, domains of open and closed immersions and of
morphisms of finite type, spectra of local rings~$\O_{X,x}$, spectra of residue
fields, and others.

The Stacks Project then defines~$\Zar(S)$ as~$\Sh(\Sch_M/S)$, where~$\Sch_M/S$
is the small category of~$S$-schemes in~$M$, or equivalently
as~$\Sh(\Aff_M/S)$. This approach has the advantage that one doesn't have to
assume the existence of a Grothendieck universe; the \emph{partial
universe}~$M$ can be constructed entirely within ZFC set theory using
transfinite recursion.

\subsection*{Employing parsimonious sites}
From a topos-theoretical point of view, it's natural to settle for an even
more parsimonious site: the site~$(\Sch/S)_\lfp$ consisting of the~$S$-schemes
which are locally of finite presentation over~$S$, or equivalently the
essentially small site~$(\Aff/S)_\lfp$ of the~$S$-schemes which are locally of
finite presentation over~$S$ and affine (as absolute schemes).\footnote{It's
not reasonable to restrict to the even smaller site consisting of the finitely
presented~$S$-schemes, since open immersions can fail to be finitely presented.
We want the site used to construct~$\Zar(S)$ to be closed under domains of open immersions,
for instance to facilitate a comparison with the little
Zariski topos~$\Sh(S)$, whose site does contain all open subsets of~$S$.
Furthermore, since a finitely presented~$S$-scheme might not admit an open
covering by finitely presented~$S$-schemes which are affine (as absolute
schemes), the toposes~$\Sh((\Sch/S)_\fp)$ and~$\Sh((\Aff/S)_\fp)$ can differ.}

In the special case that~$S = \Spec(A)$ is affine, this site is the
dual of the category of finitely presented~$A$-algebras; in this case the
topos-theoretic points of the resulting topos are precisely the local~$A$-algebras,
and moreover, the resulting topos is the classifying topos of the theory of
local~$A$-algebras, such that for any Grothendieck topos~$\E$, geometric
morphisms~$\E \to \Sh((\Aff/S)_\lfp)$ correspond to local~$A$-algebras internal
to~$\E$.

In contrast, the toposes arising when using the larger sites have categories of
points which contain further objects in addition to all local~$A$-algebras; and
no simple description of the theory they classify is known.

A further advantage of these parsimonious sites is that they don't require arbitrary
choices of a starting set~$M_0$ or a way of expanding~$M_0$ to a sufficiently
ample set~$M$ of schemes.

However, the parsimonious sites also have a serious disadvantage, namely that
with them, the Yoneda functor is only fully faithful when restricted
to~$(\Sch/S)_\lfp$. For instance, in the case~$S = \Spec(\ZZ)$, the
schemes~$\Spec(\QQ)$ and the empty scheme have isomorphic functors of points,
whereby~$\Spec(\QQ)$ looks like the empty set from the internal point of
view.\footnote{\XXX{explicit proof}}

In the following, we do not commit to a single one of these options for
resolving the set-theoretical size issues, but rather keep any of them
in mind. This approach will sometimes necessitate phrases such as ``for
any~$S$-scheme~$T$ contained in the site used to define~$\Zar(S)$'', which might
seem awkward to a topos-theorist when taken out of context, since
the site used to construct a Grothendieck topos is not at all uniquely
determined by the resulting topos.

We will indicate the few places where the choice of site makes a difference.

}

% XXX: Defining the big Zariski topos without recoursing to ordinary scheme
% theory

\section{Relation between the big and little Zariski toposes}
\label{sect:relation-big-little}

The big Zariski topos~$\Zar(S)$ is a topos over the little Zariski
topos~$\Sh(S)$ in that there is a canonical geometric morphism
\[ \pi : \Zar(S) \longrightarrow \Sh(S) \]
with direct and inverse image parts given by
\[ \pi_*E = E|_{\Sh(S)} \qquad\text{and}\qquad
  \pi^{-1}\F = ((T \xra{f} S) \mapsto \Gamma(T, f^{-1}\F)). \]
Since~$\pi^{-1}$ is fully faithful, this geometric morphism is connected; and
furthermore, it is a local geometric morphism (a further right adjoint~$\pi^!$
which is fully faithful exists).

By general results on local geometric morphisms, the adjoint pair~$(\pi_*
\dashv \pi^!)$ is a geometric morphism which is right inverse to~$\pi$ and
which exhibits~$\Sh(S)$ as a subtopos of~$\Zar(S)$, similarly to how~$\Set$ is
a subtopos of a sheaf topos over a local topological space. In this context,
it's customary to introduce notation for the idempotent monad~$\sharp$ and the
idempotent comonad~$\flat$ arising from the adjoint triple~$\pi^{-1} \dashv
\pi_* \dashv \pi^!$:
\[ \sharp E = \pi^!(E|_{\Sh(S)}) \qquad\text{and}\qquad
  \flat E = \pi^{-1}(E|_{\Sh(S)}). \]

In the case that~$S = \Spec(A)$ is an affine scheme and we employ
one of the parsimonious sites to construct~$\Zar(S)$, it's well-known
that~$\Sh(S)$ classifies local localizations of~$A$ and that~$\Zar(S)$ classifies
arbitrary local~$A$-algebras. On points, the morphism~$\pi$ sends a
local~$A$-algebra~$\varphi : A \to R$ to the local localization~$A \to
A[(\varphi^{-1}[R^\times])^{-1}]$, and its right inverse sends a local
localization~$A \to A[F^{-1}]$ to itself.


\subsection{Recovering the big Zariski topos from the little Zariski topos}

What does~$\Zar(S)$ classify in the case that~$S$ is an arbitrary scheme?
We don't know a nontautologous answer to this question, but we can answer a
related one: What does~$\Zar(S)$ classify as seen from the internal point of
view of~$\Sh(S)$?

To make sense of this question, we employ a slight extension of Shulman's
stacks semantics which allows to refer to locally internal
categories~\cite{penon:locally-internal-categories} over a base topos~$\E$ from
the internal language. Using this extension, a locally internal category
over~$\E$ looks like a locally small category from the internal point of view
of~$\E$. In particular, a geometric morphism~$f : \F \to \E$ gives rise to a
locally internal category (which over an object~$A \in \E$ is given by
the~$\E/A$-enriched category~$\F/f^{-1}A$) which will look like an ordinary
topos from the internal point of view of~$\E$.

For instance, the trivial~$\E$-topos~$\E$ will look like~$\Set$ and the
slice topos~$\E/X$ will look like~$\Set/X$ from the internal point of
view of~$\E$.

\begin{thm}\label{thm:zar-classifies}
In the situation that the site used to construct~$\Zar(S)$ is one of the
parsimonious sites, the big Zariski topos~$\Zar(S)$ is, from the internal point
of view of~$\Sh(S)$, the classifying topos of
the theory of local~$\O_S$-algebras which are local over~$\O_S$.
\end{thm}

For an arbitrary topos~$\F$ over~$\Set$, the concept of an~``$\O_S$-algebra
in~$\F$'' doesn't make any sense -- in contrast to the concept of
an~$A$-algebra in~$\F$, which can either be defined as a ring
homomorphism~$\ul{A} \to R$ in~$\F$ (where~$\ul{A}$ is the pullback of~$A \in
\Set$ to~$\F$) or as a ring object which is equipped with an~$A$-indexed family of
endomorphisms satisfying suitable axioms. However, for a~$\Sh(S)$-topos~$f : \F \to
\Sh(S)$, the concept of an~$\O_S$-algebra in~$\F$ is meaningful: It's a ring
homomorphism~$f^{-1}\O_S \to R$ in~$\F$.

Similarly, there is no absolute ``geometric theory of~$\O_S$-algebras''.
However, there is a geometric theory of~$\O_S$-algebras \emph{internal
to~$\Sh(S)$}. Theorem~\ref{thm:zar-classifies} should be viewed in this light.

The proviso ``local over~$\O_S$'' is as in the discussion of the relative
spectrum from the internal point of view
(Section~\ref{sect:relative-spectrum}).

\begin{proof}[Proof of Theorem~\ref{thm:zar-classifies}]
We have to verify that, from the point of view of~$\Sh(S)$, the topos~$\Zar(S)$
contains a canonical local and local-over-$\O_S$ $\O_S$-algebra and that for
any Grothendieck topos~$\F$, pulling back this canonical algebra yields an
equivalence between the category of geometric morphisms~$\F \to \Zar(S)$ and
the category of local and local-over-$\O_S$ $\O_S$-algebras in~$\F$.

The canonical local and local-over-$\O_S$ $\O_S$-algebra in~$\Zar(S)$ is
the algebra~$\flat\affl \to \affl$. Indeed, the ring~$\affl$ is local and
the homomorphism~$\flat\affl \to \affl$ is local, since its restriction to
any sheaf topos~$\Sh(X)$, where~$f : X \to S$ is an~$S$-scheme contained in the site used
to define~$\Zar(S)$, is local: It's the morphism $f^\sharp : f^{-1}\O_S \to \O_X$.

We now want to verify the universal property, which expressed internally
to~$\Sh(S)$ reads as
\[ \Hom(\E, \Zar(S)) \simeq
  \text{category of local and local-over-$\O_S$ $\O_S$-algebras in~$\E$}. \]
Externally, this means that for any open subset~$U \subseteq S$ and any
topos~$\E$ over~$\Sh(S)/\ul{U}$,
\begin{multline*}
  \qquad\Hom_{\Sh(S)/\ul{U}}(\E, \Zar(S)/\pi^{-1}\ul{U}) \simeq \\
  \text{category of local and local-over-$\pi^{-1}\O_S$ $\pi^{-1}\O_S$-algebras
  in~$\E$}.\qquad
\end{multline*}
We will verify this equivalence in the case that~$S = \Spec(A)$ is affine and
that~$S = U$. This suffices to establish the theorem, since~$\Sh(S)/\ul{U} \simeq
\Sh(U)$, ~$\Zar(S)/\pi^{-1}\ul{U} \simeq \Zar(U)$, and since the internal
language is local.

So let~$f : \E \to \Sh(\Spec(A))$ be a~$\Sh(\Spec(A))$-topos. By the universal
property of~$\Zar(\Spec(A))$ as the classifying topos of local~$A$-algebras, a geometric
morphism~$g : \E \to \Zar(\Spec(A))$ is uniquely determined by a
local~$A$-algebra~$\varphi : \ul{A} \to \B$ in~$\E$. By the universal property
of~$\Sh(\Spec(A))$ as the classifying topos of local localizations of~$A$, the
composition~$\pi \circ g : \E \to \Sh(\Spec(A))$ is uniquely determined by
the local localization~$\ul{A} \to g^{-1} \pi^{-1} \O_{\Spec(A)} = g^{-1}(\flat
\affl)$ in~$\E$.

In the composition
\[ \ul{A} \lra \flat \affl \lra \affl, \]
the first morphism is a local localization and the second morphism is local.
Since these properties can be formulated as geometric
implications,\footnote{A ring homomorphism~$\alpha : R \to T$ is a localization
(that is, isomorphic to the canonical localization morphism~$R \to R[S^{-1}]$
for some multiplicative subset~$S$) if and only if the canonical comparison
morphism~$R[(\alpha^{-1}T^\times)^{-1}] \to T$ is bijective. This is the case
if and only if
\begin{multline*}
  \quad\qquad\forall y\?T\_
  \exists x\?R\_ \exists s\?R\_
  \speak{$\alpha(s)$ \inv} \wedge \alpha(s) y = x
  \qquad\text{and} \\
  \forall x\?R\_ \alpha(x) = 0 \Rightarrow
  \exists s\?R\_ \speak{$\alpha(s)$ \inv} \wedge sx = 0.\qquad\quad
\end{multline*}}
they are preserved by the functor~$g^{-1}$. Since
furthermore such a factorization is unique, the localization~$\ul{A} \to
g^{-1}(\flat \affl)$ which determines~$\pi \circ g$ coincides with the
localization~$\ul{A}[(\varphi^{-1}((\affl)^\times))^{-1}]$. Referring directly
to the involved filters, the filter~$g^{-1}\F$ which determines~$\pi \circ g$
(where~$\F$ is the generic filter of~$\ul{A}$ in~$\Sh(\Spec(A))$) coincides
with the filter~$\varphi^{-1}((\affl)^\times)$. This explains the first
equivalence in the chain
\begin{align*}
  &\mathrel{\phantom{\simeq}} \Hom_{\Sh(\Spec(A))}(\E, \Zar(\Spec(A))) \\
  &\simeq
    \text{category of local algebras~$\varphi : \ul{A} \to \B$ in~$\E$
  such that $\varphi^{-1}\B^\times = f^{-1}\F$} \\
  &\simeq
    \text{category of local algebras~$\psi : f^{-1}\O_{\Spec(A)} \to \B$
    in~$\E$ such that~$\psi$ is local}.
\end{align*}
The second equivalence maps an algebra~$\varphi$
to~$\ul{A}[(\varphi^{-1}\B^\times)^{-1}] \to \B$; conversely, an algebra~$\psi$
is mapped to the composition~$\ul{A} \to f^{-1}\O_{\Spec(A)} \xra{\psi} \B$.
\end{proof}

Similarly to how Theorem~\ref{thm:zar-classifies} shows how the big Zariski
topos of~$S$ looks like from the point of view of~$\Sh(S)$, it's possible to
give an internal description of what the big Zariski topos of an arbitrary
relative spectrum over~$S$ looks like. We state and verify such a description
in Theorem~\ref{thm:big-zariski-topos-of-relative-spectrum}.

It is well-known that the points of~$\Zar(\Spec(R))$, when constructed using
one of the parsimonious sites, are in canonical bijection with the
local~$R$-algebras; for instance, this follows from the description
of~$\Zar(\Spec(R))$ as the classifying topos of the theory of
local~$R$-algebras. For the case of a general base scheme, we introduce the
following definition.

\begin{defn}A \emph{ring over~$S$} is a ring~$A$ together with a
morphism~$\Spec(A) \to S$ of locally ringed locales. A \emph{morphism of rings
over~$S$} is a ring homomorphism which is compatible with the structure
morphisms to~$S$.\end{defn}
% Let A be a local ring. Then LRL morphisms Spec(A) --> S are the same
% as LRL morphisms (pt,A) --> S.

\begin{prop}\label{prop:points-of-big-zariski}
In the situation that one of the parsimonious sites is used to
define~$\Zar(S)$, the category of points of~$\Zar(S)$ is canonically equivalent
to the full subcategory of the rings over~$S$ whose underlying ring is
local.\end{prop}

\begin{proof}By Theorem~\ref{thm:zar-classifies}, a point of~$\Zar(S)$ is given
by a point of~$\Sh(S)$, that is by a point~$s$ of~$S$,\footnote{The
topos-theoretic points of the topos of sheaves over a topological space~$T$ are
in canonical bijection with the locale-theoretic points of~$T$, that is with
locale morphisms~$1 \to T$. If~$T$ is sober, such points are in canonical
bijection with the elements of the underlying set of~$T$. In a classical
metatheory, schemes are sober~\stacksproject{01IS}. If one wants the proof to
work intuitionistically, the base scheme~$S$ has to be defined in a
intuitionistically sensible way, for instance as a locally ringed locale.
Correspondingly, the point~$s$ of~$S$ has to be interpreted in the
locale-theoretic sense.} together with a local~$\O_{S,s}$-algebra~$A$ which is
local over~$\O_{S,s}$. These data define a ring over~$S$, namely the ring~$A$
together with the composite~$\Spec(A) \to \Spec(\O_{S,s}) \to S$.
Since the structure morphism~$\O_{S,s} \to A$ is local, this composite maps the
focal point of~$\Spec(A)$ to the given point~$s \in S$.

Conversely, let a local ring~$A$ together with a morphism~$f : \Spec(A) \to S$ of
locally ringed locales be given. Let~$x \in \Spec(A)$ be the focal point
of~$\Spec(A)$. We set~$s \defeq f(x)$; then~$A$ is an~$\O_{S,s}$-algebra
by~$(f^\sharp)_x$. It is local over~$\O_{S,s}$ since~$f^\sharp$ is a local
homomorphism.

These constructions are mutually inverse since the morphisms $\Spec(\O_{S,s})
\to S$ are monomorphisms in the category of locally ringed locales.
\end{proof}

\begin{rem}\label{rem:zar-classifies-absolute}
In the situation that one of the parsimonious sites is used to
define the big Zariski topos of~$S$, it classifies the theory of local rings
over~$S$. This is a restatement of Theorem~\ref{thm:zar-classifies}.
Explicitly, the theory of local rings over~$R$ is given by:
\begin{enumerate}
\item A theory which~$\Sh(S)$ classifies.
\item Structure and axioms for a ring~$R$.
\item Structure and axioms which guarantee that the interpretation of~$R$ in
any cocomplete topos coincides with the pullback of~$\O_S$.
\item Structure and axioms for a local ring~$A$ and a local homomorphism~$R \to A$.
\end{enumerate}
The fourth item can be substituted by:
\begin{enumerate}
\item[(4')] Structure and axioms for a local ring~$A$ and a morphism~$\Spec(A)
\to (\pt,R)$ of locally ringed locales.
\end{enumerate}
This is because such a morphism is given by a local homomorphism~$\ul{R}
\to \O_{\Spec(A)}$ of sheaves of rings which in turn is given by a local ring
homomorphism~$R \to \Gamma(\Spec(A), \O_{\Spec(A)}) = A$. (Taking global
sections of a local homomorphism of sheaves of rings yields a homomorphism of
rings which will typically fail to be local. However, here taking global
sections coincides with calculating the stalk at the focal point of~$\Spec(A)$,
and pullback preserves locality of ring homomorphisms.)
\end{rem}

Corollary~\ref{cor:pp1-classifies} gives a description of the theory which the
big Zariski topos of~$\PP^1_\ZZ$ classifies, building upon
Remark~\ref{rem:zar-classifies-absolute}.


\subsection{Recovering the little Zariski topos from the big Zariski topos}

Theorem~\ref{thm:zar-classifies} shows that~$\Zar(S)$ can be reconstructed
from~$\Sh(S)$ (and its structure sheaf~$\O_S$). Similarly, it's possible to
reconstruct~$\Sh(S)$ from~$\Zar(S)$ (and the canonical morphism~$\flat \affl
\to \affl$).

\begin{thm}\label{thm:reconstruct-little-topos}
In the situation that the site used to construct~$\Zar(S)$ is one of the
parsimonious sites, the little Zariski topos~$\Sh(S)$ is the largest subtopos
of~$\Zar(S)$ where the canonical morphism~$\flat \affl \to \affl$ is an
isomorphism.
\end{thm}

In other words, the little Zariski topos is the largest subtopos~$\E
\hookrightarrow \Zar(S)$ such that $\Zar(S) \models (\speak{$\flat\affl \to
\affl$ is bijective})^\Box$ (where~$\Box$ is the modal operator corresponding
to the subtopos), that is that the pullback of the canonical
morphism~$\flat\affl \to \affl$ to~$\E$ is an isomorphism.

In the case that~$S
= \Spec(A)$ is affine, we also have the ring~$\ul{A}$ in~$\Zar(S)$ available.
In this case the condition is equivalent to
\[ \Zar(S) \models \speak{$\ul{A} \to \affl$ is a localization}^\Box, \]
since in the composition~$\ul{A} \to \flat\affl \to \affl$ the first morphism
is a localization.

\begin{proof}[Proof of Theorem~\ref{thm:reconstruct-little-topos}]
The little Zariski topos is a subtopos of the big Zariski topos via the right
inverse~$s$ of~$\pi : \Zar(S) \to \Sh(S)$, the geometric morphism~$(\pi_* \dashv
\pi^!)$. The pullback of~$\flat\affl \to \affl$ to~$\Sh(S)$ is therefore the
morphism~$(\flat\affl)|_{\Sh(S)} \to \affl|_{\Sh(S)}$, that is~$\O_S \to \O_S$,
which is an isomorphism.

Let~$f : \E \hookrightarrow \Zar(S)$ be any subtopos such that the pullback
of~$\flat\affl \to \affl$ to~$\E$ is an isomorphism. We want to verify that~$f$
factors over the inclusion~$s : \Sh(S) \hookrightarrow \Zar(S)$.
\[ \xymatrix{
  \E \ar@{^{(}->}[rr]^f \ar@{-->}[rd] && \Zar(S) \\
  & \Sh(S) \ar@{^{(}->}[ru]_s
} \]
A candidate for a morphism~$\E \to \Sh(S)$ witnessing this factorization is the
composite~$\pi \circ f$. It remains to show that~$s \circ (\pi \circ f) = f$.
Both~$s \circ (\pi \circ f)$ and~$f$ are morphisms of~$\Sh(S)$-toposes,
where~$\E$ is regarded as a~$\Sh(S)$-topos by the composition~$\pi \circ f$.
By the universal property of the big Zariski topos given in
Theorem~\ref{thm:zar-classifies}, they are therefore uniquely determined by
the~$\O_S$-algebra they classify.

The morphism~$s \circ (\pi \circ f)$ classifies
the~$\O_S$-algebra~$f^{-1}\pi^{-1}s^{-1}\affl = f^{-1}(\flat\affl)$. The
morphism~$f$ classifies the~$\O_S$-algebra~$f^{-1}\affl$.
Since~$f^{-1}(\flat\affl) \to f^{-1}\affl$ is an isomorphism, these algebras
coincide.
\end{proof}


\subsection{Change of base}
\label{sect:change-of-base}

Let~$f : X \to S$ be a morphism of schemes. In any of the situations that
\begin{enumerate}
\item the parsimonious sites are used to construct the big Zariski
toposes and~$f$ is locally of finite presentation, or
\item the same (Grothendieck or partial) universe is used for constructing both
Zariski toposes and both~$X$ and~$S$ are contained in the universe,
\end{enumerate}
the morphism~$f$ induces an essential geometric morphism~$\Zar(X) \to \Zar(S)$
which we again denote by~``$f$''. Explicitly, the big Zariski toposes are
related by the adjoint triple~$f_! \dashv f^{-1} \dashv f_*$ with
\begin{align*}
  f_* : \Zar(X) \lra \Zar(S),\ F &\longmapsto ((T \xra{g} S) \mapsto F(T \times_S X)), \\
  f^{-1} : \Zar(S) \lra \Zar(X),\ E &\longmapsto ((T \xra{g} X) \mapsto F(T \xra{g} X \xra{f} S)), \\
  f_! : \Zar(X) \lra \Zar(S),\ F &\longmapsto ((T \xra{g} S) \mapsto \coprod_{h : T \to X} F(T \xra{h} X)).
\end{align*}
In situation~(2), the well-definedness of these functors is trivial. In
situation~(1), the well-definedness rests on the lemma that an~$S$-morphism~$h : T \to X$ is locally of
finite presentation if~$T$ and~$X$ are locally of finite presentation over~$S$~\stacksproject{02FV}.

The objects of~$\Zar(S)$ listed on page~\pageref{page:important-objects} pull back
as expected:
\begin{itemize}
\item Let~$Y$ be an~$S$-scheme. Then~$f^{-1} \ul{Y} = \ul{Y \times_S X}$, by
the universal property of the fiber product.
\item In particular, $f^{-1} \affl = \afflx$, since~$\AA^1_S \times_S X =
\AA^1_X$.
\item Let~$\F$ be a sheaf of sets on~$S$. Then~$f^{-1} \pi_S^{-1} \F =
\pi_X^{-1} f^{-1} \F$.
\item Let~$\F$ be a sheaf of~$\O_S$-modules. Then~$f^{-1} \F^\Zar =
(f^* \F)^\Zar$.
\end{itemize}

The functors~$f_! \dashv f^{-1}$ induce an equivalence
\[ \Zar(X) \simeq \Zar(S)/\ul{X}, \]
explicitly described by
\[ \begin{array}{r@{}c@{}l}
  F &{}\longmapsto{}& (f_!F \to f_!1), \\
  ((T \xra{g} X) \mapsto \{ s \in (f^{-1}E)(T) \,|\, \alpha(s) = g \}) &{}\longmapsfrom{}& (E
  \xra{\alpha} \ul{X}).
\end{array} \]
From the internal point of view of~$\Zar(S)$, the big Zariski topos of~$X$ is
therefore simply~$\Set/\ul{X}$, the category of~$\ul{X}$-indexed families of
sets or equivalently the category of sheaves on~$\ul{X}$ considered as a
\emph{discrete} locale. This fits nicely with the philosophy that~``$S$-schemes are
plain unstructured sets from the internal point of view of~$\Zar(S)$''.

In contrast, for the little Zariski toposes, there is no similarly simple
description of the little Zariski topos of~$X$ as a slice of the little Zariski
topos of~$S$. From the internal point of view of~$\Sh(S)$, the topos~$\Sh(X)$
looks like the topos of sheaves over a locale which is not discrete, and the
topos~$\Zar(X)$ doesn't even look like a topos of sheaves over an arbitrary
locale (discrete or not).

The internal language of a slice topos~$\E/I$ admits a
simple description from the point of view of~$\E$. Namely, for
any formula~$\varphi$ over~$\E/I$,
\[ \E/I \models \varphi \qquad\text{iff}\qquad
  \E \models \forall i\?I\_ \varphi(i). \]
For the right hand side to make sense, it has to be interpreted in the
following way. Any object~$(p : M \to I)$ of~$\E/I$ which appears in~$\varphi$,
for instance as a domain of quantification, has to be substituted by the
internal expression~``$p^{-1}[\{i\}]$'' denoting the fiber of~$p$
over~$i\?I$.\footnote{This substitution is less ad~hoc as it might at first
appear. The internal language of a topos~$\E$ is \emph{dependently typed}, meaning
that the types one can quantify over may depend on previously introduced
values. Types in the empty context, depending on no values, correspond to
objects of~$\E$. Types in the context of a variable~$i\?I$ correspond to
objects~$(p : M \to I)$ of~$\E/I$. For instance, in this case one can form
formulas of the form~``$\forall i\?I\_ \forall m\?M(i)\_ \psi(i,m)$''. If in
the translation process using the Kripke--Joyal semantics a formal variable~$i$
was substituted by a generalized element~$i_0 : A \to I$, the
expression~``$M(i_0)$'' has to be interpreted as the pullback~$i_0^* M$.}
For example, if~$(p : M \to I)$ is such an object of~$\E/I$,
\begin{align*}
  &\phantom{\text{iff }} \E/I \models \speak{$M$ is inhabited} \\
  &\text{iff } \phantom{/I}\E \models \forall i\?I\_ \speak{the fiber of $p$ over $i$ is inhabited} \\
  &\text{iff } \phantom{/I}\E \models \forall i\?I\_ \exists m\?M\_ p(m) = i.
\end{align*}

Thanks to this description of the internal language of a slice topos, the
equivalence~$\Zar(X) \simeq \Zar(S)/\ul{X}$ is useful for lifting internal
characterizations concerning properties of~$S$-schemes to properties of
morphisms of~$S$-schemes. For instance, we will see in
Proposition~\ref{prop:char-surjective-morphisms} that the structure morphism of
an~$S$-scheme~$f : Y \to S$ is surjective if and only if
$\Zar(S) \models \neg\neg(\speak{$\ul{Y}$ is inhabited})$.
This automatically implies (Corollary~\ref{cor:char-surjective-morphisms-relative})
that a morphism~$p : Y \to X$ of~$S$-schemes is surjective if and only if
\[ \Zar(S) \models \forall x\?\ul{X}\_ \neg\neg(\speak{the fiber of~$\ul{p}$
over~$x$ is inhabited}). \]

Many properties of morphisms in algebraic geometry, and any properties which
can be characterized using the internal language of the big Zariski topos, are
stable under base change. For those kinds of properties~$P$, if a morphism~$Y
\to X$ is~$P$, then for any point~$x \in X$ the base change~$Y_x \to
\Spec(k(x))$ along~$\Spec(k(x)) \to X$ is~$P$ as well. The converse is usually
false, but the motto~``a morphism is~$P$ if all its fibers are~$P$ in a
continuous fashion'' is still useful for intuition. The equivalence~$\Zar(X)
\simeq \Zar(S)/\ul{X}$ makes this motto precise: For any morphism~$p : Y \to X$
of~$S$-schemes and any formula~$\varphi(M)$ of~$\Zar(S)$ containing a free
variable~$M$,
\[ \Zar(X) \models \varphi(\ul{Y}) \qquad\text{iff}\qquad
  \Zar(S) \models \forall x\?\ul{X}\_ \varphi(\ul{p}^{-1}[\{x\}]), \]
that is $Y$ has property~$\varphi$ when regarded as an~$X$-scheme if and only
if all the fibers of~$Y \to X$ have property~$\varphi$ when regarded
as~$S$-schemes.

% In the situation X --> S,
% * give morphisms Zar(X) --> Zar(S),
% * explain Zar(X) = Zar(S)/X: reason, consequence for the internal language,
%   interpretation of the three functors.
%
% We should maybe also discuss relevance of \flat\affl, for instance for
% constructing Omega? 
%
% In the introduction to Part III, mention that we're first discussing
% generalities and that the development of the synthetic approach appears
% later.

\begin{rem}Some care is needed when dealing with the modalities~$\flat$
and~$\sharp$, since they are not compatible with change of base.
If~$f : X \to S$ is a morphism of schemes, then in general~$f^{-1}(\flat E)
\not\cong \flat(f^{-1}E)$, since
\begin{align*}
  f^{-1}(\flat E) &= ((T \xra{g} X) \mapsto \Gamma(T,
  g^{-1}f^{-1}(E|_{\Sh(S)}))), \quad\text{but}\\
  \flat(f^{-1} E) &= ((T \xra{g} X) \mapsto \Gamma(T, g^{-1}(E|_{\Sh(X)}))).
\end{align*}
A special case in which the canonical morphism~$f^{-1}(\flat E) \to
\flat(f^{-1}E)$ is an isomorphism is when~$f$ is an open immersion.
\end{rem}

% XXX: Mention that therefore it's not possible to characterize \flat
% internally.

% A corollary of the existence of~$\pi^!$ is that~$\pi_*$ is (not only
% continuous, but also) cocontinuous. Geometric.


\subsection{The big Zariski topos of a relative spectrum}

\begin{thm}\label{thm:big-zariski-topos-of-relative-spectrum}
Let~$\A$ be a quasicoherent~$\O_S$-algebra. In the situation that the
parsimonious sites are used for constructing big Zariski toposes, the big
Zariski topos of~$\RelSpec_S(\A)$ is, from the internal point of view
of~$\Sh(S)$, the classifying topos of the theory of
local~$\A$-algebras which are local over~$\O_S$.
\end{thm}

\begin{proof}The proof is similar to the proof of
Theorem~\ref{thm:zar-classifies}. Let~$X = \RelSpec_S(\A)$ and~$f : X \to S$ be the
canonical morphism. The big Zariski topos of~$\RelSpec_S(\A)$ is
a~$\Sh(S)$-topos by the composition~$\Zar(\RelSpec_S(\A)) \to \Zar(S) \to
\Sh(S)$. The pullback of~$\O_S$ along this geometric morphism
is~$f^{-1}(\flat\affl)$. A canonical~$\O_S$-algebra in~$\Zar(\RelSpec_S(\A))$
is therefore
\[ f^{-1}(\flat\affl) \lra \flat\afflx \lra \afflx. \]
This algebra is indeed local and local over~$f^{-1}(\flat\affl)$.

For verifying the universal property, it suffices to restrict to the case
that~$S = \Spec(R)$ is affine, as in the proof of
Theorem~\ref{thm:zar-classifies}, and consider a geometric morphism~$f : \E \to
\Sh(S)$. In this case~$\A = A^\sim$
and~$X = \RelSpec_S(\A) = \Spec(A)$. Let~$\alpha : R \to A$ be the structure
morphism of~$A$. We then have the chain of equivalences
\begin{align*}
  &\mathrel{\phantom{\simeq}} \Hom_{\Sh(S)}(\E, \Zar(X)) \\
  &\simeq \text{cat.\@ of local algebras~$\varphi : \ul{A} \to \B$
  in~$\Zar(X)$ such that~$\ul{\alpha}^{-1} \varphi^{-1} \B^\times = f^{-1}\F$} \\
  &\simeq \text{cat.\@ of local algebras~$\psi : f^{-1}\A \to \B$ such that
  $f^{-1}\O_S \to f^{-1}\A \to \B$ is local}.
\end{align*}
The first equivalence maps a geometric morphism~$g : \E \to \Zar(X)$ to~$\ul{A}
\to g^{-1}\afflx$. The second equivalence acts as follows. Given an
algebra~$\varphi : \ul{A} \to \B$ such that~$\ul{\alpha}^{-1} \varphi^{-1}
\B^\times = f^{-1}\F$, we can factor~$\ul{R} \to \ul{A} \to \B$ uniquely as a
localization~$\ul{R} \to C$ followed by a local homomorphism~$C \to \B$. By the
condition on filters, the localization~$C$ is isomorphic to~$f^{-1}\O_S$. From
the description~$\A = \ul{A}[\F^{-1}]$ it is apparent that~$\ul{A} \to \B$
factors over~$\ul{A} \to f^{-1}\A$. In this way, we obtain
morphisms~$f^{-1}\O_S \to f^{-1}\A \to \B$.
\end{proof}

The only reason why we have supposed that~$\A$ is quasicoherent in the
statement of Theorem~\ref{thm:big-zariski-topos-of-relative-spectrum} is
because else~$\RelSpec_S(\A)$ might fail to be a scheme, whereby the notion
``big Zariski topos of~$\RelSpec_S(\A)$'' is not defined.

In fact, we propose the following definition: If~$(X,\O_X)$ is an arbitrary
locally ringed locale (or even a locally ringed topos), then the big Zariski
topos of~$X$ should be the classifying~$\Sh(X)$-topos of the theory (internal
to~$\Sh(X)$) of local~$\O_X$-algebras which are local over~$\O_X$.
The following proposition shows that this definition is consistent with
Theorem~\ref{thm:zar-classifies} and
with~Theorem~\ref{thm:big-zariski-topos-of-relative-spectrum}.

\begin{prop}Let~$\A$ be an~$\O_S$-algebra. The following constructions,
performed internally to~$\Sh(S)$, yield canonically equivalent toposes:
\begin{enumerate}
\item Constructing first the local spectrum~$X \defeq \Spec(\A|\O_S)$ and then,
internally to~$\Sh_{\Sh(S)}(X)$, the classifying topos of the theory
of~$\O_X$-algebras which are local over~$\O_X$.
\item Constructing the classifying topos of the theory of~$\A$-algebras which
are local over~$\O_S$.
\end{enumerate}
If furthermore~$\A$ is finitely presented as an~$\O_S$-algebra from the
internal point of view of~$\Sh(S)$, then the following construction yields the
same result as well:
\begin{enumerate}
\addtocounter{enumi}{2}
\item Constructing first the big Zariski topos of~$S$ as the classifying topos
of local~$\O_S$-algebras which are local over~$\O_S$ and then constructing,
internally to that topos, the slice topos over~$[\A^\Zar,
\affl]_{\Alg(\affl)}$.
\end{enumerate}
\end{prop}

\begin{proof}
If~$S$ is indeed a scheme, as is supposed throughout this part, and~$\A$ is
quasicoherent, then all three constructions yield the big
Zariski topos of~$\RelSpec_S(\A)$ (defined using one of the parsimonious
sites). For the first construction, this is by
Theorem~\ref{thm:local-spectrum-yields-relative-spectrum} and
Theorem~\ref{thm:zar-classifies}; for the second construction, this is by
Theorem~\ref{thm:big-zariski-topos-of-relative-spectrum}; and for the third
construction, this is by Theorem~\ref{thm:zar-classifies},
Proposition~\ref{prop:relative-spectrum-big-zariski}, and the description of
the slice topos in Section~\ref{sect:change-of-base}. However, the claim also
holds if~$\A$ is not quasicoherent or if~$S$ is an arbitrary locally ringed
locale, and it's instructive to see the proof in this more general situation.

We work in the internal universe of~$\Sh(S)$. Let~$\E$ be an arbitrary
(Gro\-then\-dieck) topos. Then~$\E$-valued points of the three toposes are given
by:
\begin{enumerate}
\item a filter~$F \subseteq \A$ lying over the filter of units of~$\O_S$
together with a local~$\A_F$-algebra~$R$ which is local over~$\A_F$
\item a local~$\A$-algebra which is local over~$\O_S$
\item a local~$\O_S$-algebra~$R$ which is local over~$\O_S$ together with an
element of the stalk of~$[\A^\Zar, \affl]_{\Alg(\affl)}$ at the point
corresponding to~$R$
\end{enumerate}
In the case that~$\A$ is finitely presented, the stalk appearing in
description~(3) is canonically isomorphic to the set of~$R$-algebra
homomorphisms~$\A \otimes_{\O_S} R \to R$, as discussed in
Lemma~\ref{lemma:fp-hom-geometric}.

With these descriptions, the equivalence is immediate. For instance, a
datum~$(F \subseteq \A, \A_F \to R)$ as in description~(1) gives rise to the
datum~$(\O_S \to \A_F \to R)$ as in description~(2). Conversely, the structure
morphism of a datum as in description~(2) can be factored as a localization
followed by a local homomorphism to yield a datum as in~(1).
\end{proof}


\section{The double negation modality}

\begin{prop}\label{prop:notnot-in-big-zariski-topos}
Let~$\varphi$ be a formula over~$S$. Consider the following
statements:
\begin{enumerate}
\item $\Zar(S) \models \neg\neg\varphi$.
\item For all points~$s \in S$, there is a field extension~$K \fieldext k(s)$ such
that~$\Spec(K) \to \Spec(k(s)) \to S$ is contained in the site used to
define~$\Zar(S)$ and such that~$\Spec(K) \models \varphi$.
\item For all closed points~$s \in S$, there is a finite field extension~$K \fieldext
k(s)$ such that~$\Spec(K) \models \varphi$.
\end{enumerate}
Then:
\begin{itemize}
\item Condition~(2) implies condition~(1). The converse holds if the site used
to define~$\Zar(S)$ is closed under taking spectra of residue fields (this
is satisfied for all sites listed in Section~\ref{sect:proper-choice-of-site}
except for the parsimonious sites).
\item If one of the parsimonious sites is used to define~$\Zar(S)$ and~$S$ is
locally Noetherian, condition~(1) implies condition~(3). The converse holds if
additionally~$S$ is locally of finite type over a field.
\end{itemize}
\end{prop}

\begin{proof}We begin with showing that condition~(2) implies condition~(1). By
the Kriple--Joyal translation, we need to verify that
\[ \forall (X \to S)\_
  \Bigl(\forall (T \to X)\_ (T \models \varphi) \Rightarrow T = \emptyset\Bigr)
  \Longrightarrow X = \emptyset, \]
where the universal quantifiers range over all schemes contained in the site
used to define~$\Zar(S)$. So let such an~$S$-scheme~$f : X \to S$ be given. We
show that the fiber over any point~$s \in S$ is empty. By assumption, there is
a field extension~$K \fieldext k(s)$ such that~$\Spec(K) \to \Spec(k(s)) \to S$
is contained in the site used to define~$\Zar(S)$ and such that~$\Spec(K)
\models \varphi$. The base change~$T$ of the fiber~$X_s$ to~$\Spec(K)$ as
indicated in the diagram
\[ \xymatrix{
  T \ar[r]\ar[d] & X_s \ar[r]\ar[d] & X \ar[d] \\
  \Spec(K) \ar[r] & \Spec(k(s)) \ar[r] & S
} \]
is contained in the site used to define~$\Zar(S)$ as well, therefore
saying~``$T \models \varphi$'' is meaningful. And indeed~$T \models \varphi$,
since~$\Spec(K) \models \varphi$. Therefore~$T = \emptyset$. Thus~$X_s =
\emptyset$ as claimed.

For the direction~``(1)~$\Rightarrow$~(2)'', let a point~$s \in S$ be given.
Since we assume that the site used to define~$\Zar(S)$ contains
the~$S$-scheme~$X \defeq \Spec(k(s))$ and since~$X \neq \emptyset$, the
assumption implies that there exists an nonempty~$X$-scheme~$T$ such that~$T
\models \varphi$.  Since~$T$ is nonempty, there exists a point~$t \in T$. By
the morphism~$\Spec(k(t)) \to T \to X$, the field~$K \defeq k(t)$ is an
extension of~$k(s)$, and since~$\Spec(K) \to T \to X \to S$ is contained in the
site, we have~$\Spec(K) \models \varphi$.

The proof that condition~(1) implies condition~(3) in the case that one of the
parsimonious sites is used to define~$\Zar(S)$ and that~$S$ is locally
Noetherian is similar. For a closed point~$s \in S$, the residue field~$k(s)$
can be calculated as~$A/\mmm$, where~$A$ is the ring of functions of an open
affine neighbourhood of~$s$ and~$\mmm$ is a maximal ideal in~$A$. Since~$A$ is
Noetherian, the ideal~$\mmm$ is finitely generated and therefore~$A/\mmm$ is
finitely presented as an~$A$-algebra. Thus the canonical morphism~$\Spec(k(s))
\to \Spec(A) \to S$ is locally of finite presentation and thereby contained in
the parsimonious site. The hypothesis is therefore applicable to~$X \defeq
\Spec(k(s))$ and yields a nonempty~$X$-scheme~$T$ which is locally of finite
presentation over~$X$ such that~$T \models \varphi$.

Since the structure morphism~$T \to X$ is locally of finite presentation, the
scheme~$T$ inherits the property to be locally Noetherian from~$X$. Let~$U
\subseteq T$ be a nonempty open affine subset and let~$t \in U$ be a point
which is closed in~$U$. With the same reasoning as above, the canonical
morphism~$\Spec(k(t)) \to U \to T$ is therefore contained in the parsimonious
site. Thus~$\Spec(k(t)) \models \varphi$. The field~$K \defeq k(t)$ is finitely
presented as an~$k(s)$-algebra. By Noether normalization, it is also of finite
dimension as an~$k(s)$-vector space.

Finally, we verify that condition~(3) implies condition~(1) if one of the
parsimonious sites is used to define~$\Zar(S)$ and if~$S$ is locally
of finite type over a field (and therefore in particular Noetherian). We adopt
the notation of the proof of~``(2)~$\Rightarrow$~(1)''. The argument there
shows that all fibers of~$f$ over closed points are empty. If~$X$ is not empty,
it contains a closed point~$x$ (since~$X$ is locally of finite type over a field,
any point which is closed in an open affine neighbourhood will do). Since~$X$
is locally of finite type over a field, the point~$f(x)$ is closed in~$S$.
Therefore~$x$ is contained in the fiber over a closed point; a contradiction.
\end{proof}

\begin{rem}The proof of Proposition~\ref{prop:notnot-in-big-zariski-topos} uses
classical logic in a substantial way, since repeatedly the lemma that a scheme
is trivial if it doesn't have any points was used. Even if scheme theory is
set up in an intuitionistic sensible way (for instance defining a scheme to be
a locally ringed locale which is locally isomorphic to the locale-theoretic
spectra of rings as discussed in Section~\ref{sect:spectrum-as-a-locale}), one
should therefore not expect the proposition to admit an intuitionistic proof
without additional hypotheses.
\end{rem}

\begin{lemma}\label{lemma:image-coincides}
Let~$f : X \to S$ and~$g : Y \to S$ be~$S$-schemes which are locally contained
in the site used to define~$\Zar(S)$. In the case that the site is one of the
parsimonious sites, further assume that~$f$ and~$g$ are quasicompact and quasiseparated.
\begin{enumerate}
\item The image of~$f$ coincides with the image of~$g$ topologically.
\item $\Zar(S) \models \neg\neg(\speak{$\ul{X}$ inhabited}) \Leftrightarrow
  \neg\neg(\speak{$\ul{Y}$ inhabited})$.
\end{enumerate}
\end{lemma}

\begin{proof}By Proposition~\ref{prop:char-surjective-morphisms}, which we'll
prove below, statement~(2) is equivalent to:
\begin{indentblock}
For any~$S$-scheme~$h : T \to S$ contained in the site used to
define~$\Zar(S)$, the morphism~$X \times_S T \to T$ is surjective if and only
if~$Y \times_T T \to T$ is.
\end{indentblock}
We verify that this statement implies statement~(1). Let~$s \in \im(f)$. Then the canonical
morphism~$X_s \to \Spec(k(s))$ is surjective. Therefore there exists
an~$S$-scheme~$h : T \to S$ which is contained in the site used to
define~$\Zar(S)$ such that~$X \times_S T \to T$ is surjective and such that~$s
\in \im(h)$: If~$\Zar(S)$ is defined using a Grothendieck or partial universe,
this claim is trivial, since we can take~$T \defeq \Spec(k(s))$. If~$\Zar(S)$
is defined using one of the parsimonious sites, we employ the technique of
relative approximation.\footnote{More specifically, we may assume that~$S$ is
affine. Then the lemma on relative approximation~\stacksproject{09MV} can be
applied to write~$\Spec(k(s))$ as a directed limit of an inverse system of
finitely presented~$S$-schemes~$T_i$ with affine transition maps. Let~$U
\subseteq X$ be an open affine subset containing a preimage of~$s$. The
property that~$U \hookrightarrow X \to \Spec(k(s))$ is surjective descends to
one of the morphisms~$U \times_S T_i \to T_i$~\stacksproject{07RR}. In
particular, the morphism~$X \times_S T_i \to T_i$ is surjective. We can
therefore take~$T \defeq T_i$. The image of~$T_i \to S$ contains~$s$
since~$\Spec(k(s)) \to S$ factors over~$T_i \to S$.}

The assumption yields that the induced morphism~$Y \times_T T \to T$ is
surjective. Since~$s \in \im(h)$, also~$s \in \im(g)$.

The proof of the converse containment relation is analogous.

The direction~``(1)~$\Rightarrow$~(2)'' is immediate, since
\[ \im(X \times_S T \to T) = h^{-1} \im(f) = h^{-1} \im(g) = \im(Y \times_S T \to T).
  \qedhere \]
\end{proof}

\begin{rem}Let~$f : X \to S$ be contained in the site used to define~$\Zar(S)$.
In the case that the site is one of the parsimonious sites, further assume
that~$f$ is quasicompact and quasiseparated.
The expression~``$\neg\neg(\speak{$\ul{X}$ is inhabited})$'' of the internal
language of~$\Zar(S)$ denotes the subfunctor of the terminal
functor~$\ul{S} = 1 \in \Zar(S)$ given by
\[ (h : T \to S) \longmapsto \{ \star \,|\, \im(h) \subseteq \im(f) \}. \]
If~$f$ is an open immersion, then this functor coincides with the functor of
points of~$X$, since the set-theoretic image of a morphism of schemes is
contained in an open subset~$U \subseteq S$ if and only if it factors over the
open immersion~$U \hookrightarrow S$.

If~$f$ is a closed immersion, this functor is the functor of points of the
formal completion of~$S$ along~$X$. More generally, for an
arbitrary~$S$-scheme~$X$ and a closed subscheme~$Z \hookrightarrow X$ (such
that both~$X$ and~$S$ are locally contained in the site used to
define~$\Zar(S)$), the internal expression~``$\{ x \? \ul{X} \,|\, \neg\neg(x \in
\ul{Z}) \}$'' denotes the functor of points of the formal completion of~$X$
along~$Z$. For instance, the expression
\[ \{ f \? \affl \,|\, \neg\neg(f = 0) \} =
  \{ f \? \affl \,|\, \speak{$f$ is nilpotent} \} \]
denotes the formal neighbourhood of the origin in the affine line~$\AA^1_S$.
(The equivalence~$\neg\neg(f = 0) \Leftrightarrow \speak{$f$ is nilpotent}$ is
by Proposition~\ref{prop:a1-nilp}.)
\end{rem}


\section{Sheaves of rings, algebras, and modules}

\XXX{locally free, ...}

\subsection{Quasicoherence}

\begin{defn}\label{defn:synth-qcoh}
An~$R$-module~$E$ is \emph{synthetically quasicoherent} if and only if,
for any finitely presented~$R$-algebra~$A$, the canonical~$R$-algebra
homomorphism
\[ E \otimes_R A \longrightarrow [\Spec(A), E] = [[A, R]_{\Alg(R)}, E] \]
which maps a pure tensor~$x \otimes f$ to~$(\varphi \mapsto \varphi(f) x)$ is
bijective. Here and in the following, the set~$[\Spec(A), E]$ is the set of all
maps~$\Spec(A) \to E$, and~$[A,R]_{\Alg(R)}$ is the set of all~$R$-algebra
homomorphisms~$A \to R$.\end{defn}

This definition has the following interpretation. The codomain of the displayed
canonical map is the set of all~$E$-valued functions on~$\Spec(A)$. Elements
of~$E \otimes_R A$ induce such functions; these induced functions can
reasonably be called ``algebraic''. In a synthetic context, there should be no
other~$E$-valued functions as these algebraic ones, and different algebraic
expressions should yield different functions. This is precisely what the
postulated bijectivity expresses.

\begin{thm}\label{thm:qcoh-big-char}
Let~$E \in \Zar(S)$ be an~$\affl$-module.
If~$E$ is quasicoherent, that is of
the form~$(\E_0)^\Zar$ for some quasicoherent~$\O_S$-module~$\E_0$,
then~$E$ is synthetically quasicoherent from the internal point of view of~$\Zar(S)$.
The converse holds in any of the following situations:
\begin{enumerate}
\item The site used to construct~$\Zar(S)$ is one of the parsimonious sites.
\item The base scheme~$S$ is concentrated (quasicompact and quasiseparated) and
the functor~$E$ maps directed limits of inverse systems of~$S$-schemes with
affine transition morphisms to colimits in~$\Set$.
\end{enumerate}
\end{thm}

\begin{proof}To verify that~$E$ is synthetically quasicoherent, we have to
verify a condition for~$\affl$-algebras~$A$ in any
slice~$\Zar(S)/\ul{T}$. If such an algebra is finitely presented from
the internal point of view, then there is a covering~$T = \bigcup_i T_i$ such
that the each of the restrictions of the algebra to the~$T_i$ is of the
form~$(\A_0)^\Zar$ for some finitely presented~$\O_{T_i}$-algebra~$\A_0$.
Without loss of generality, we will just assume that~$A$ itself is of the
form~$(\A_0)^\Zar$ for a finitely presented~$\O_S$-algebra~$\A_0$.

By Proposition~\ref{prop:relative-spectrum-big-zariski}, the internal expression~$\Spec(A)$ is the functor of
points of~$\RelSpec_S \A_0$. For any~$S$-scheme~$f : T \to S$ contained in the site
used to define~$\Zar(S)$, we consider the fiber product
\[ \xymatrix{
\RelSpec_T(f^*\A_0) \ar[r]^{f'} \ar[d]_{p'} & \RelSpec_S\A_0 \ar[d]^p \\
T \ar[r]_f & S.
} \]
Since~$\RelSpec_T(f^*\A_0) \to S$ is contained in the site (for any of our
admissible sites), we may conclude using the following chain of isomorphisms:
\begin{align*}
[\Spec(A), E](T) &\cong
\Hom_{\Zar(S)}(\ul{T}, [\Spec(A), E])
\cong \Hom_{\Zar(S)}(\ul{T} \times \Spec(A), E) \\
&\cong \Hom_{\Zar(S)}(\underline{T \times_S \RelSpec_S\A_0}, E)
\cong E(\RelSpec_T(f^*\A_0)) \\
&\cong \Gamma(\RelSpec_T(f^*\A_0), (p')^* f^* \E_0)
\cong \Gamma(T, (p')_* (p')^* f_* \E_0) \\
&\cong \Gamma(T, f^*\E_0 \otimes_{\O_T} f^*\A_0)
\cong \Gamma(T, (\E_0 \otimes_{\O_S} \A_0)^\Zar) \\
&\cong \Gamma(T, (\E_0)^\Zar \otimes_\affl (\A_0)^\Zar)
\cong \Gamma(T, E \otimes_\affl A).
\end{align*}
The antepenultimate isomorphism is because pullback of modules in~$\Sh(S)$ to
modules in~$\Sh(T)$ commutes with tensor product. The penultimate isomorphism
is because pullback of a sheaf in~$\Sh(S)$ to a sheaf in~$\Zar(S)$ commutes
with tensor product (Lemma~\ref{lemma:zar-tensor-product-commutes}).

For the converse direction, we first verify that the restrictions~$E|_{\Sh(T)}$
to the little Zariski topos of each~$S$-scheme~$T$ contained in the site used
to define~$\Zar(S)$ are quasicoherent~$\O_T$-modules. For this, we employ the
quasicoherence criterion of Theorem~\ref{thm:qcoh-sheafchar}: For any open
affine subset~$T' \subseteq T$ and any function~$h \in \Gamma(T', \O_T)$ we
verify that the canonical morphism
\begin{equation}\label{eqn:want-iso}\tag{$\dagger$}
E|_{\Sh(T)}[h^{-1}] \longrightarrow j_*(E|_{\Sh(D(h))})
\end{equation}
is an isomorphism, where~$j : D(h) \hookrightarrow T'$ denotes the inclusion.
This follows from the assumption of synthetic quasicoherence by considering
the~$\affl$-module~$A \defeq \affl[h^{-1}]$ (in the slice~$\Zar(S)/\ul{T'}$):
This expresses that the canonical morphism
\begin{equation}\label{eqn:have-iso}\tag{$\ddagger$}
E \otimes_\affl \affl[h^{-1}] \longrightarrow [\Spec(A), E]
\end{equation}
is an isomorphism (of~$\affl$-modules in~$\Zar(S)/\ul{T'}$). Restricting the
domain to~$\Sh(T')$ yields the sheaf~$E|_{\Sh(T')} \otimes_{\O_{T'}}
\O_{T'}[h^{-1}]$, since restricting commutes with the geometric constructions
``forming the tensor product'' and ``localizing away from~$h$''.
Since~$\Spec(A)$ is the functor of points of~$D(h)$, restricting
the codomain to~$\Sh(T')$ yields the sheaf~$j_*(E|_{\Sh(D(h))})$.
The canonical morphism~\eqref{eqn:want-iso} which we want to recognize as an
isomorphism is therefore the restriction of the canonical
morphism~\eqref{eqn:have-iso} which we know to be an isomorphism.

A natural candidate for an quasicoherent~$\O_S$-module~$\E_0$ with~$E \cong
(\E_0)^\Zar$ is~$\E_0 \defeq E|_{\Sh(S)}$. We'll show that this is indeed true.
Let~$f : T \to S$ be any~$S$-scheme contained in the site used to
define~$\Zar(S)$. We assume for the time being that~$f$ is of finite
presentation and affine, so~$T \cong \RelSpec_S \A_0$ for some finitely
presented~$\O_S$-algebra~$\A_0$. We want to verify that the canonical morphism
\begin{equation}\label{eqn:want-iso2}\tag{§}
f^*(E|_{\Sh(S)}) \longrightarrow E|_{\Sh(T)}
\end{equation}
is an isomorphism. Since the functor~$f_*$ from quasicoherent~$\O_T$-modules to
quasicoherent~$\O_S$-modules is fully faithful (the morphism~$f$ being affine)
and domain and codomain of that morphism are quasicoherent, it suffices to
verify that its image under~$f_*$ is an isomorphism. This image is the
canonical morphism
\[ E|_{\Sh(S)} \otimes_{\O_S} \A_0 \longrightarrow f_*(E|_{\Sh(T)}). \]
The assumption of synthetic quasicoherence, applied to the~$\affl$-algebra~$A
\defeq (\A_0)^\Zar$, shows that this morphism is an isomorphism.

In situation~(1), the only step left to do is to generalize the argument in the
previous paragraph to morphisms~$f : T \to S$ which are locally of finite
presentation. This works out because there are open covers of~$S$ and~$T$ such
that the appropriate restrictions of~$f$ are of finite presentation and affine.
The assumption of synthetic quasicoherence then needs to be applied to
to~$\affl$-algebras in suitable slices of~$\Zar(S)$, showing that the canonical
morphism~\eqref{eqn:want-iso2} is locally an isomorphism and therefore globally
as well.

In situation~(2), we employ the technique of approximating general~$S$-schemes
by~$S$-schemes of finite presentation. Specifically, let~$f : T \to S$ be an
arbitrary~$S$-scheme contained in the site used to define~$\Zar(S)$. Without
loss of generality, we may assume that~$T$ is an affine scheme. Thus~$T$ is
quasicompact and quasiseparated, and~$S$ is quasiseparated by assumption. We may therefore
apply the lemma of relative approximation~\stacksproject{09MV} to deduce
that~$T$ is a directed limit of an inverse system of~$S$-schemes~$f_i : T_i \to
S$ of finite presentation with affine transition maps. These~$S$-schemes
are contained in the site used to define~$\Zar(S)$. Furthermore, they inherit
quasicompactness and quasiseparatedness from~$S$. Therefore we can apply a
comparison result on the categories of quasicoherent
modules~\stacksproject{01Z0}:
\[
E(T) = E(\lim_i T_i) \cong \colim_i E(T_i) \cong
\colim_i \Gamma(T_i, f_i^* \E_0)
\cong \Gamma(T, f^* \E_0). \qedhere
\]
\end{proof}

\begin{scholium}\label{scholium}
Let~$E \in \Zar(S)$ be a quasicoherent~$\affl$-module.
Let~$A \in \Zar(S)$ be a quasicoherent~$\affl$-algebra such that~$\Spec(A) \in
\Zar(S)$ is representable by an object of the site used to define~$\Zar(S)$.
Then the canonical morphism
\[ E \otimes_\affl A \longrightarrow [\Spec(A), E] \]
is an isomorphism.
\end{scholium}

\begin{proof}The second paragraph of the proof of
Theorem~\ref{thm:qcoh-big-char} applies.\end{proof}

\begin{rem}\label{rem:local-representability}
The condition in Scholium~\ref{scholium} that~$\Spec(A)$ is
representable by an object of the site used to define~$\Zar(S)$ is slightly
unnatural from a topos-theoretic point of view, since the conclusion of the
Scholium depends only on the topos over the site and not the site itself.
In fact, the condition can be weakened and made more natural at the
same time: It suffices to require that~$\Spec(A)$ is \emph{locally}
representable by an object of the site.

However, the condition can't be dropped completely. For instance, if we employ the
parsimonious sites and consider~$S = \Spec \ZZ$,~$E = \affl$, and $A =
\K_S^\Zar$ (where~$\K_S$ is the sheaf of rational functions on~$S$, which in
this case is the constant sheaf~$\ul{\QQ}$), then~$\Spec(A)$ is the functor of
points of the~$\ZZ$-scheme~$\Spec(\QQ)$. This functor coincides with the
functor of points of the empty~$\ZZ$-scheme on the parsimonious sites;
therefore~$\Spec(A) = \emptyset$ from the internal point of view. Thus the
codomain of the canonical morphism is the zero algebra, but the domain is not.
\end{rem}

\begin{rem}\label{rem:radical-not-qcoh}
The quotient~$\affl/\sqrt{(0)}$ in~$\Zar(S)$ is an example
for a sheaf of modules which is not quasicoherent even though all of its
restrictions to the little Zariski toposes~$\Sh(X)$ for morphisms~$f : X \to S$
are:
Since taking the quotient and taking the radical of an ideal are geometric
constructions, we have~$(\affl/\sqrt{(0)})|_{\Sh(X)} \cong \O_X/\sqrt{(0)}$.
These sheaves of modules are quasicoherent (Example~\ref{ex:radical-qcoh}).
However, in general,~$f^*(\O_S/\sqrt{(0)}) \not\cong \O_X/\sqrt{(0)}$.
A concrete counterexample is~$S = \Spec(k)$ and~$X = \Spec(k[T]/(T^2))$.
In this case~$f^*(\O_S/\sqrt{(0)}) \cong f^*(\O_S) \cong \O_X$.
\end{rem}


\subsection{Special properties of the affine line}
The ring object~$\affl$ in the big Zariski topos enjoys several special
properties, some of which are unique in that they're only possible in an
intuitionistic context. We compile here a short list of such
properties. As was already mentioned, at least one of them, the field
property, was already noticed in the 1970s by Kock~\cite{kock:univ-proj-geometry}.

The statements and proofs in this subsection are formulated in the internal
language. The proofs only use the fact that~$\affl$ is a synthetically
quasicoherent local ring. This illustrates the meta-claim that synthetic
quasicoherence is a strong and meaningful condition.

\begin{prop}\label{prop:a1-field}
$\affl$ is a field in the sense that any element which is not zero is
invertible: $\forall x\?\affl\_ \neg(x = 0) \Rightarrow \speak{$x$ \inv}$. More generally,
for any number~$n \geq 0$,
\begin{multline*}
  \qquad\qquad \forall x_1,\ldots,x_n\?\affl\_
  \neg(x_1 = 0 \wedge \cdots \wedge x_n = 0) \Longrightarrow \\
  (\speak{$x_1$ \inv} \vee \cdots \vee \speak{$x_n$ \inv}). \qquad\qquad
\end{multline*}
\end{prop}

\begin{proof}Let $x\?\affl$ be such that~$\neg(x=0)$. We consider the quasicoherence
condition for the finitely presented~$\affl$-algebra~$A \defeq \affl/(x)$.
Since~$\Spec(A) \cong \brak{x=0} = \brak{\bot} = \emptyset$, the condition says
that the canonical homomorphism
\[ \affl/(x) \longrightarrow [\emptyset, \affl] \]
is an isomorphism. Since its codomain is the zero algebra, so is~$\affl/(x)$.
Therefore~$1 \in (x)$, that is,~$x$ is invertible.

The more general statement follows in the same way, by using the quasicoherence
condition for~$A \defeq \affl/(x_1,\ldots,x_n)$. This yields~$1 \in
(x_1,\ldots,x_n)$. Since~$\affl$ is a local ring, one of the~$x_i$ is
invertible.\end{proof}

\begin{prop}\label{prop:a1-not-reduced}
$\affl$ is not a reduced ring: $\neg \Bigl(\forall x\?\affl\_ (\bigvee_{n \geq
0} x^n = 0) \Rightarrow x = 0\Bigr).$
\end{prop}

\begin{proof}Assume that~$\affl$ is reduced. Then the set~$\Delta \defeq \{
\varepsilon \in \affl \,|\, \varepsilon^2 = 0 \}$ is equal to~$\{ 0 \}$.
By the quasicoherence criterion applied to the finitely
presented~$\affl$-algebra~$A \defeq \affl[T]/(T^2)$, the canonical map
\[ \affl[T]/(T^2) \longrightarrow [\Spec(\affl[T]/(T^2)), \affl] \cong
  [\Delta, \affl] \cong \affl \]
is an isomorphism. It maps~$[T]$ to zero (the value of~$T$ at~$0 \in \Delta$).
Thus~$T \in (T^2)$ and therefore~$1 = 0$ in~$\affl$. This is a contradiction.
\end{proof}

In classical logic, Proposition~\ref{prop:a1-field} and
Proposition~\ref{prop:a1-not-reduced} would directly contradict each
other; only an intuitionistic context allows for fields which are not reduced.

That~$\affl$ is not reduced, irrespective of the reducedness of the base
scheme~$S$, should not come as a surprise: Reducedness is not stable under base
change, but all statements of the internal language of~$\Zar(S)$ are.
If~$\affl$ was reduced, then all~$S$-schemes (at least those contained in the
site used to construct~$\Zar(S)$) would be reduced as well. In contrast, the
structure sheaf~$\O_S$ is reduced from the point of view of the little Zariski
topos if and only if~$S$ is reduced (Proposition~\ref{prop:reduced-ring}).

\begin{prop}\label{prop:a1-nilp}
The following statements about an element~$x\?\affl$ are
equivalent:
\begin{enumerate}
\item $x$ is not invertible.
\item $x$ is nilpotent.
\item $x$ is \notnot zero.
\end{enumerate}
\end{prop}

\begin{proof}Let~$x \? \affl$ be not invertible. We consider the quasicoherence
condition for the finitely presented~$\affl$-algebra~$A \defeq \affl[x^{-1}]$.
Since~$\Spec(A) \cong \brak{\speak{$x$ \inv}} = \emptyset$, it follows
that~$\affl[x^{-1}] = 0$, similarly to the proof of
Proposition~\ref{prop:a1-field}. Thus~$x$
is nilpotent.

Let~$x \? \affl$ be a nilpotent element. Thus~$x^n = 0$ for some number~$n \geq
0$. If~$x$ was nonzero, then~$x$ and therefore~$x^n$ would be invertible, in
contradiction to~$0 \neq 1$ since~$\affl$ is a local ring.

Let~$x \? \affl$ be \notnot zero. Then~$x$ is not invertible, since if~$x$ was
invertible, then~$x$ would be nonzero.
\end{proof}

Summarizing, the following facts about nilpotents hold in the internal
universe of the big Zariski topos. Firstly, it's not true that~$\affl$ is
reduced.  But this doesn't mean that there actually exists a nilpotent element
which is not zero. In fact, any nilpotent is \notnot zero.

\begin{prop}Any function~$\affl \to \affl$ is given by a unique polynomial
in~$\affl[T]$.
\end{prop}

\begin{proof}Immediate by considering the quasicoherence condition for the finitely
presented~$\affl$-algebra~$A \defeq \affl[T]$ and noticing that~$\Spec(A) \cong
\affl$.\end{proof}

This statement too cannot be satisfied in classical logic: for infinite
fields the existence part fails and for finite fields the uniqueness part
fails.

\begin{prop}$\affl$ is \emph{weakly algebraically closed}, in the following sense:
Any monic polynomial~$p \? \affl[T]$ of degree at least one does \notnot have
a zero.\end{prop}

\begin{proof}Let~$p \? \affl[T]$ be a monic polynomial of degree at least one. Assume
that~$p$ doesn't have a zero in~$\affl$. Then the spectrum of~$A \defeq
\affl[T]/(p)$ is empty. The quasicoherence condition for~$A$ therefore implies
that~$\affl[T]/(p)$ is zero. This means that~$p$ is invertible in~$\affl[T]$.
A basic lemma in commutative algebra (whose standard proof is constructive)
then implies that with the exception of the constant term in~$p$, all
coefficients are nilpotent. This contradicts the assumption that~$p$ is monic
of degree at least one.\end{proof}

\begin{prop}$\affl$ is infinite in the following sense: For any number~$n \geq 0$
and any given elements~$x_1,\ldots,x_n \? \affl$, there is \notnot an element~$y$
which is distinct from all of the~$x_i$.
\end{prop}

\begin{proof}The polynomial~$f(T) \defeq (T - x_1) \cdots (T - x_n) + 1$
does \notnot have a zero~$y\?\affl$, since~$\affl$ is weakly algebraically
closed. This element cannot equal any~$x_i$, since~$f(x_i) = 1$ is not zero.
\end{proof}

\begin{prop}$\affl$ fulfills a version of the Nullstellensatz:
Let $f_1,\ldots,f_m \in \affl[X_1,\ldots,X_n]$ be polynomials without a common
zero in~$(\affl)^n$. Then there are polynomials~$g_1,\ldots,g_m \in
\affl[X_1,\ldots,X_n]$ such that~$\sum_i g_i f_i = 1$.
\end{prop}

\begin{proof}We consider the quasicoherence condition for the finitely
presented~$\affl$-algebra~$A \defeq \affl[X_1,\ldots,X_n]/(f_1,\ldots,f_m)$.
Since~$\Spec(A) \cong \{ x \in (\affl)^n \,|\, f_1(x) = \ldots = f_m(x) = 0 \}
= \emptyset$, the condition implies that~$A$ is the zero algebra just as in the
verification of Proposition~\ref{prop:a1-field}.
\end{proof}


\section{Basic constructions of relative scheme theory}

With~$\affl$ at hand, we can perform many of the usual constructions of
(relative) scheme theory internally.

\subsubsection*{Group schemes} The functors associated to the standard group schemes~$\GG_\text{a}$, $\GG_\text{m}$,
$\mathrm{GL}_n$, and~$\mu_n$ are given by the internal expressions
\begin{align*}
  \GG_\text{a} &\defeq \affl \text{ (as an additive group)}, \\
  \GG_\text{m} &\defeq \{ x\?\affl \,|\, \speak{$x$ \inv} \}, \\
  \mathrm{GL}_n &\defeq \{ M \? (\affl)^{n \times n} \,|\, \speak{$M$ \inv} \}, \\
  \mu_n &\defeq \{ x \? \affl \,|\, x^n = 1 \}.
\end{align*}

\subsubsection*{Affine and projective space}
Affine~$n$-space over~$S$ is given by~$(\affl)^n$, \ie internally the set
of~$n$-tuples of elements of~$\affl$. The functor of points of
projective~$n$-space over~$X$, with all its nontrivial topological and
ring-theoretical structure, is described by the astoundingly naive expression
\[ \PP^n \defeq \{ (x_0,\ldots,x_n) \? (\affl)^{n+1} \,|\,
  x_0 \neq 0 \vee \cdots \vee x_n \neq 0 \}/{\sim}, \]
where the equivalence relation is the usual rescaling relation from the
internal point of view. This example was suggested by Zhen~Lin Low (private
communication).

More generally, for an~$S$-scheme~$X$, affine and projective~$n$-space
over~$X$ are given by~$\ul{X} \times (\affl)^n$ and~$\ul{X} \times \PP^n$,
respectively.


\subsection{Tangent bundle}

For an~$S$-scheme~$X$, the internal Hom~$[\Delta,\ul{X}] \in \Zar(S)$ describes the
tangent bundle of~$X$, \ie the~$S$-scheme~$\RelSpec_X{\operatorname{Sym}(\Omega^1_{X/S})} \to X \to S$, as can be seen by
chasing the definitions~\cite[Lemma~5.12.1]{brandenburg:tensor-foundations}.
Intuitively, a map~$f : \Delta \to \ul{X}$ from the internal point of view is
given by slightly more data than merely the point~$f(0)$; one also has to
specify first-order information.

This description of the (not necessarily
locally trivial) tangent bundle fits nicely with the intuition of tangent
vectors as infinitesimal curves, and in fact is precisely the definition of the
tangent bundle in synthetic differential geometry~\cite[Def.~7.1]{kock:sdg}.


\subsection{Relative spectrum}

\begin{defn}The \emph{synthetic spectrum} of an~$R$-algebra~$A$ is
\[ \Spec(A) \defeq [A, R]_{\mathrm{Alg}(R)}, \]
the set of~$R$-algebra homomorphisms from~$A$ to~$R$.\end{defn}

\begin{ex}The synthetic spectrum of~$R$ is the one-element set.
More generally, the synthetic spectrum of the algebra~$R[X_1,\ldots,X_n]/(f_1,\ldots,f_m)$
is the solution set~$\{ x \? R^n \,|\, f_1(x) = \cdots = f_n(x) = 0 \}$.
\end{ex}

\begin{ex}The synthetic spectrum of~$R/(f)$ is~$\brak{f = 0}$, the truth value
of the formula~``$f = 0$'', the subsingleton set~$\{ \star \,|\, f = 0 \}$.
If classical logic is available, then this set contains~$\star$ or is empty,
depending on whether~$f$ is zero or not. Similarly, the synthetic spectrum
of~$R[f^{-1}]$ is~$\brak{\speak{$f$ \inv}}$.\end{ex}

\begin{prop}\label{prop:relative-spectrum-big-zariski}
Let~$\A_0$ be an~$\O_S$-algebra (not necessarily quasicoherent).
Then the synthetic spectrum of the~$\affl$-algebra~$(\A_0)^\Zar$, as constructed
in the internal language of~$\Zar(S)$, is the functor of points of~$\RelSpec_S \A_0$.
\end{prop}

\begin{proof}The Hom set occuring in the definition of the synthetic spectrum is
interpreted by the internal Hom when using the internal language. For
any~$S$-scheme~$f : T \to S$ contained in the site used to define~$\Zar(S)$, we
have the following chain of isomorphisms.
\begin{align*}
  (\Spec(A))(T) &= [(\A_0)^\Zar, \affl]_{\mathrm{Alg}(\affl)}(T) \\
  &\cong
  \Hom_{\Zar(S)}(\ul{T}, [(\A_0)^\Zar, \affl]_{\mathrm{Alg}(\affl)}) \\
  &\cong
  \Hom_{\Zar(S)}(\ul{T} \times (\A_0)^\Zar, \affl)_{\ldots} \\
  &\cong
  \Hom_{\Zar(S)/\ul{T}}(\ul{T} \times (\A_0)^\Zar, \ul{T} \times \affl)_{\ldots} \\
  &\cong
  \Hom_{\mathrm{Alg}_{\Zar(T)}(\afflt)}((f^*\A_0)^\Zar, \afflt) \\
  &=
  \Hom_{\mathrm{Alg}_{\Zar(T)}(\afflt)}(\pi^{-1}(f^*\A_0) \otimes_{\pi^{-1}\O_T}
  \afflt, \afflt) \\
  &\cong
  \Hom_{\mathrm{Alg}_{\Zar(T)}(\pi^{-1}\O_T)}(\pi^{-1}(f^*\A_0), \afflt) \\
  &\cong
  \Hom_{\mathrm{Alg}_{\Sh(T)}(\O_T)}(f^*\A_0, \O_T) \\
  &\cong
  \Hom_{\mathrm{Alg}_{\Sh(S)}(\O_S)}(\A_0, f_*\O_T) \\
  &\cong
  \Hom_S(T, \RelSpec_S \A_0).
\end{align*}
The omitted subscripts~``$\ldots$'' should denote that we're only taking the
subset of the Hom set where, for each fixed first argument, the morphisms are
morphisms of~$\affl$-algebras.
\end{proof}

If~$X \in \Zar(S)$ is an arbitrary object, there is a canonical morphism~$X \to
\Spec([X,\affl])$. In the internal language of~$\Zar(X)$ it looks like the
``inclusion into the double dual'':
\[ x \longmapsto \placeholder(x),
  \quad\text{where $\placeholder(x) : [X,\affl] \to \affl,\ \varphi \mapsto
  \varphi(x)$.} \]
The following proposition shows that bijectivity of this map is related to~$X$
being the functor of points of an affine~$S$-scheme (an~$S$-scheme whose
structure morphism to~$S$ is affine).

\begin{prop}A sheaf~$X \in \Zar(S)$ is isomorphic to the functor of points of
an affine~$S$-scheme if, in the internal language of~$\Zar(S)$,
the~$\affl$-algebra~$[X,\affl]$ is quasicoherent and the canonical map~$X \to
\Spec([X,\affl])$ is bijective. The converse holds in any of the following
situations:
\begin{enumerate}
\item The affine~$S$-scheme which~$X$ represents is of finite presentation
over~$S$.
\item The site used to define~$\Zar(S)$ is defined using a partial universe
and the affine~$S$-scheme which~$X$ represents if of finite type over~$S$.
\item The affine~$S$-scheme which~$X$ represents is contained in the site used
to define~$\Zar(S)$. (This situation subsumes the previous ones.)
\end{enumerate}\end{prop}

\begin{proof}The ``if'' direction is straightforward, since the assumption
expresses~$X$ as the functor of points of the relative spectrum of a
quasicoherent~$\O_S$-algebra.

For the ``only if'' direction, we abuse notation and denote the given
affine~$S$-scheme whose functor of points is~$X$ by~``$f : X \to S$''.
Then~$f_*\O_X$ is quasicoherent and the canonical morphism~$X \to
\RelSpec_S{f_*\O_X}$ is an isomorphism. In any of the listed situations, the
internal Hom~$[X,\affl]$ is canonically isomorphic to~$(f_*\O_X)^\Zar$, since
for any object~$T \xra{g} S$ of the site used to define~$\Zar(S)$ we have that
\begin{align*}
  [X,\affl](T) &\cong
  \Hom_{\Zar(S)}(\underline{T}, [X,\affl]) \cong
  \Hom_{\Zar(S)}(\underline{T} \times X,\affl) \\
  &\cong \Hom_{\Zar(S)}(\underline{T} \times \underline{X},\affl) \cong
  \Hom_{\Zar(S)}(\underline{T \times_S X},\affl) \\
  &\cong \affl(T \times_S X) \cong
  (g^* f_* \O_X)(T) =
  (f_* \O_X)^\Zar(T).
\end{align*}
Therefore~$[X,\affl]$ is quasicoherent. The map induced by the isomorphism~$X \to
\RelSpec_S{f_*\O_X}$ on the level of functors of points is precisely the
canonical map~$X \to \Spec([X,\affl])$ as defined in the internal language;
therefore this map is bijective from the internal point of view.
\end{proof}

\begin{rem}As noted in Remark~\ref{rem:local-representability} in a slightly
different context, Condition~(3) in the previous proposition is unnatural from a
topos-theoretic point of view and should be weakened to require only local
representability.
\end{rem}

\begin{rem}Let~$\A_0$ be an~$\O_S$-algebra. Then one can form, internally
to~$\Zar(S)$, two locales related to~$\A_0$: the discrete locale on the synthetic
spectrum of~$(\A_0)^\Zar$, and the local spectrum of~$(\A_0)^\Zar$ over~$\affl$ as described
in~Definition~\ref{defn:local-spectrum}. These locales don't coincide. In fact,
the pullback of a discrete locale is discrete, whereas the pullback of
the local spectrum to any of the little Zariski toposes~$\Sh(X)$, where~$f : X
\to S$ is an~$S$-scheme contained in the site used to define~$\Zar(S)$, is the
relative spectrum~$\RelSpec_X(f^*\A_0)$, which is typically not discrete as
an~$X$-locale. (This is because the locale spectrum construction is geometric,
by Proposition~\ref{prop:local-spectrum-generic}.)

There is, however, a comparison morphism from the discrete locale on the
synthetic spectrum to the local spectrum. On points, it sends
an~$\affl$-algebra homomorphism~$\varphi : (\A_0)^\Zar \to \affl$ to the
filter~$\varphi^{-1}[(\affl)^\times]$.

One can also form, internally to~$\Zar(S)$, the classifying topos
of~$(\A_0)^\Zar$-algebras which are local over~$\affl$. This topos doesn't
coincide with the (toposes of sheaves over) the mentioned two locales, either.
The pullback of that classifying topos to any of the~$\Sh(X)$ is the big
Zariski topos of~$\RelSpec_X(f^*\A_0)$ (built using one of the parsimonious
sites).
\end{rem}


\subsection{Relative Proj construction}

\begin{defn}The \emph{synthetic Proj} of an~$\NN$-graded~$R$-algebra~$A$ is the set
\[ \Proj(A) \defeq
  (\text{set of all surj.\@ graded $R$-algebra homomorphisms~$A \to R[T]$})/R^\times. \]
\end{defn}

\begin{ex}\label{ex:proj-polynomial-ring}
The synthetic Proj of~$R[X_0,\ldots,X_n]$ is canonically isomorphic
to the set points~$[x_0:\cdots:x_n]$ with at
least one coordinate being invertible.
\end{ex}

\begin{prop}\label{prop:relative-proj-big-zariski}
Let~$\A_0$ be a graded~$\O_S$-algebra (not necessarily quasicoherent).
Then the synthetic Proj of the~$\affl$-algebra~$(\A_0)^\Zar$, as constructed
in the internal language of~$\Zar(S)$, is the functor of points of~$\RelProj_S \A_0$.
\end{prop}

\XXX{proof}  % (and check that it's true without the usual hypothesis)}

The following corollary was prompted by a question on
MathOverflow~\cite{mo:pp1}. We are grateful to Yuhao Huang for the impulse.

\begin{cor}\label{cor:pp1-classifies}
The big Zariski topos of the projective line~$\PP^1_\ZZ$ classifies the theory of ``a local
ring together with a point~$[a:b]$'' (that is a pair~$(a,b)$ of ring elements,
where at least one coordinate is invertible, up to multiplication by units).
Explicitly, this theory is given by:
\begin{enumerate}
\item A sort~$A$ together with function symbols, constants, and axioms expressing
that~$A$ is a local ring.
\item A sort~$P$ (to be thought of as the set of $[a:b]$ with $a,b\?A$ where at
least one coordinate is invertible) together with a
relation~$\langle\cdot,\cdot,\cdot\rangle$ on $A \times A \times P$ and the
following axioms:
\begin{itemize}
\item $\speak{$a$ \inv} \vee \speak{$b$ \inv}
\dashv\vdash_{a,b{:}A} \exists p\?P\_ \langle a,b,p \rangle$
\item $\langle a,b,p \rangle \wedge \langle a,b,p' \rangle
\vdash_{a,b{:}A,\,p,p'{:}P} p = p'$
\item $\top \vdash_{p{:}P} \exists a,b\?A\_ \langle a,b,p \rangle$
\item $\langle a,b,p \rangle \wedge \langle a',b',p \rangle
\dashv\vdash_{a,a',b,b'{:}A,\, p{:}P} \exists s\?A\_ \speak{$s$ \inv} \wedge a' = s a \wedge b' = s b$
\end{itemize}
\item A constant of sort~$P$.
\end{enumerate}
\end{cor}

\begin{proof}The big Zariski topos of~$\PP^1_\ZZ$ is a topos over the big
Zariski topos of~$\Spec(\ZZ)$; from the point of view of~$\Zar(\Spec(\ZZ))$, it
is the classifying topos of a point~$[a:b]$ where~$a,b\?\afflz$,
since~$\Zar(\PP^1_\ZZ) \simeq \Zar(\Spec(\ZZ))/\ul{\PP^1_\ZZ}$
as discussed in Section~\ref{sect:change-of-base} and~$\ul{\PP^1_\ZZ} \cong \{ [a:b] \,|\,
a,b\?\afflz \}$ by Proposition~\ref{prop:relative-proj-big-zariski} and
Example~\ref{ex:proj-polynomial-ring}. The big Zariski topos of~$\Spec(\ZZ)$
classifies local rings. Therefore the claim follows by considering the
combined geometric theory.

An alternative proof builds upon Remark~\ref{rem:zar-classifies-absolute} and
the description of the theory which the little Zariski topos of~$\PP^1_\ZZ$
classifies (Proposition~\ref{prop:proj-classifying-locale}). Combining these,
we see that~$\Zar(\PP^1_\ZZ)$ classifies the theory of a homogeneous filter~$F$
of~$\ZZ[X,Y]$ meeting the irrelevant ideal together with a local
homomorphism~$\alpha : \ZZ[X,Y][F^{-1}]_0 \to A$ into a local ring~$A$. Such data
gives rise to a point~$[\alpha(X/u) : \alpha(Y/u)]$, where~$u$ is a
homogeneous element of degree~1 contained in~$F$; and conversely any
point~$[a:b]$ gives rise to a filter
\[ F \defeq \{ f \in \ZZ[X,Y] \,|\, \text{$f_n(a,b)$ is invertible in~$A$
for some~$n \geq 0$} \}, \]
where~$f_n$ is the homogeneous component of degree~$n$ of~$f$, and a local
homomorphism~$\alpha : \ZZ[X,Y][F^{-1}]_0 \to A$ mapping~$f/g$
to~$f(a,b)/g(a,b)$.
\end{proof}


\subsection{Open immersions}

A basic concept in the functor-of-points approach to algebraic geometry is the
concept of an \emph{open subfunctor}. It is used to delimit schemes from more
general kinds of spaces: A functor is deemed to be a scheme if and only if it
admits a covering by open subfunctors which are representable.

The following definition is phrased in such a way as to apply to any of the
several ways to define the big Zariski topos~$\Zar(S)$. In particular, it
applies to the definition using the site consisting of affine schemes which are
locally of finite presentation over~$S$. If~$S$ is affine, the definition only
refers to affine schemes and open subschemes of affine schemes and is therefore
suitable if one wants to found the theory of schemes using the functorial
approach.

\begin{defn}[{\cite[Définition~I.1.3.6 on page~10]{demazure:gabriel},
\cite[Tag~01JI]{stacks-project}}]
A subfunctor~$U \hookrightarrow X$ in~$\Zar(S)$ is an \emph{open subfunctor} if
and only if for any object~$(T \to S)$ of the site used to define~$\Zar(S)$
and any~$x \in X(T)$ there exists an open subscheme~$T_0 \subseteq T$
such that for any object~$(T' \xra{f} T \to S)$ of the site used to
define~$\Zar(S)$ the map~$T' \to T$ factors over~$T_0$ if and only if~$X(f)(x)
\in U(T')$.
\end{defn}

The open subschemes~$T_0 \subseteq T$ appearing in this definition are uniquely
determined by their universal property. The relation of open subfunctors to
open immersions is as follows.

\begin{prop}\label{prop:char-open-immersion}
Let~$X$ be an~$S$-scheme.
\begin{enumerate}
\item Let~$U \subseteq X$ be an open subscheme. Then the subfunctor~$\ul{U}
\hookrightarrow \ul{X}$ is open.
\item If~$\ul{X}$ is locally representable by an object of the site used to
define~$\Zar(S)$, any open subfunctor~$U \hookrightarrow \ul{X}$ is isomorphic
to the open subfunctor associated to an open subscheme of~$X$.
\end{enumerate}
\end{prop}

\begin{proof}For the first claim, let~$(T \to S)$ be an object of the site used
to define~$\Zar(S)$ and let~$x \in \ul{X}(T)$. The open
subscheme~$T_0 \subseteq T$ required by the definition of an open subfunctor can
then be chosen as~$x^{-1}[U]$.

For the second claim, assuming for notational simplicity that~$\ul{X}$ is
directly representable without having to pass to a cover, the desired open
subscheme of~$X$ can be obtained as the witnessing subscheme~``$T_0$'' as it
appears in the definition of an open subfunctor in the special case~$(T \to S)
\defeq (X \to S)$.
\end{proof}

From the point of view of the internal language of~$\Zar(S)$, a subfunctor~$U
\hookrightarrow X$ looks like the inclusion of a subset. The natural question
how one can characterize those inclusions which externally correspond to open
subfunctors is answered as follows.

\begin{defn}In the context of a specified local ring, as for instance~$\affl$
of the big Zariski topos of a scheme, a truth value~$U \subseteq 1$ is
\emph{open} if and only if there exists an ideal~$J \subseteq \affl$ such
that~$\affl/J$ is synthetically quasicoherent
(Definition~\ref{defn:synth-qcoh}) and such that~$U$ holds if and only if~$1 \in
J$. (Section~\ref{sect:basics-on-truth-values} contains generalities on truth
values.)\end{defn}

\begin{ex}Let~$f \? \affl$. Then~``$f \neq 0$'' is an open truth value with
witnessing ideal~$J = (f)$. The quotient~$\affl/J$ is indeed synthetically
quasicoherent, since it is finitely presented. More generally,
let~$f_1,\ldots,f_n \? \affl$. Then~``$f_1 \neq 0 \vee \cdots \vee f_n \neq
0$'' is an open truth value with witnessing ideal~$J =
(f_1,\ldots,f_n)$.\end{ex}

\begin{defn}In the context of a specified local ring, a map~$U \to X$ of sets is a
\emph{synthetic open immersion} if and only if it is injective and for any~$x\?X$
the truth value of~``the fiber of~$x$ is inhabited'' is open.\end{defn}

\begin{ex}The inclusion~$(\affl)^\times \hookrightarrow \affl$ of the
invertible elements is a synthetic open immersion, since for~$x\?\affl$ the
truth value of~``the fiber of~$x$ is inhabited'' equals the truth value of~``$x
\neq 0$''.\end{ex}

\begin{ex}Let~$X \in \Zar(S)$. Let~$f \? [X,\affl]$ from the internal point of
view. The inclusion~$\{ x\?X \,|\, f(x) \neq 0 \} \hookrightarrow X$ is a
synthetic open immersion.\end{ex}

\begin{prop}\label{prop:characterization-open-subfunctor}
Let~$X \in \Zar(S)$ be a Zariski sheaf. A subfunctor~$U
\hookrightarrow X$ is open if and only if, from the internal point of view
of~$\Zar(S)$, the map~$U \hookrightarrow X$ is a synthetic open immersion.
In formal language:
\begin{multline*}
  \qquad\Zar(S) \models
  \forall x\?X\_
  \exists J \subseteq \affl\_ \\
  \speak{$J$ is an ideal} \wedge
  \speak{$\affl/J$ is synth.\@ quasicoherent} \wedge
  (x \in U \Leftrightarrow 1 \in J). \end{multline*}
\end{prop}

We postpone the proof of this proposition in order to give a bit of context
first.

Firstly, the displayed condition is only meaningful in an intuitionistic context as
provided by the big Zariski topos. In classical logic, the condition would be
trivially satisfied for any subfunctor~$U \hookrightarrow X$: Classically, we
have~$(x \in U) \vee (x \not\in U)$. If~$x \in U$, we can pick~$J = (1)$, and
if~$x \not\in U$, we can pick~$J = (0)$ (whereby the quotient~$\affl/J$ is
isomorphic to~$\affl$, thus finitely presented and therefore in particular
synthetically quasicoherent).\footnote{Strictly speaking, incompability with classical
logical surfaces even earlier: in our synthetic quasicoherence condition. The
map~$E \otimes_\affl A \to [\Spec(A), E]$ which the condition demands to be
bijective has hardly any chance to be surjective if the law of excluded middle
is available to define maps~$\Spec(A) \to E$ by case distinction.}

The proposition is often used in the following weakened form.

\begin{cor}\label{cor:sufficient-criterion-open-subfunctor}
Let~$X \in \Zar(S)$ be a Zariski sheaf. Let~$U \hookrightarrow X$ be
a subfunctor. If
\[
  \Zar(S) \models
  \forall x\?X\_
  \bigvee_{n \geq 0} \exists f_1,\ldots,f_n\?\affl\_
  (x \in U \Leftrightarrow \bigvee_i \speak{$f_i$ \inv}),
\]
then the subfunctor is open.
\end{cor}

\begin{proof}
We show that the assumption implies the displayed condition of
Proposition~\ref{prop:characterization-open-subfunctor} in the internal
language. Given elements~$f_1,\ldots,f_n$ as in the assumption, we
construct the ideal~$J \defeq (f_1,\ldots,f_n) \subseteq \affl$. The
quotient~$\affl/J$ is indeed synthetically quasicoherent, since it is finitely presented, and
the statement that~$1 \in J$ is equivalent to one of
the~$f_i$ being invertible by locality of~$\affl$.
\end{proof}

The internal condition appearing in the corollary reflects basic intuition
about openness in algebraic geometry: Intuitively, a subset is open if it is
given by inequations,
so that to decide whether a point belongs to the subset one has to check that
at least one of some numbers is not zero.

Of course, in classical scheme theory, one would put some condition on these
numbers in order not to trivialize the notion. For instance, one would require
that these numbers depend continuously on the point in some sense or, more
specifically, that these numbers are given by evaluating certain locally
defined regular functions at the point.

On first sight, such a condition seems to be lacking in
Corollary~\ref{cor:sufficient-criterion-open-subfunctor}. However, it's
implicitly built into the language, since by the Kripke--Joyal semantics the
external meaning of~``$\exists f\?\affl$'' is that there exist, locally on an
open cover, suitable elements of~$\affl(T)$, that is regular functions on~$T$.

The notion of open truth values is not unique to our account of synthetic
algebraic geometry. Rather, it's a concept in the established and more general
framework of synthetic topology~\cite{escardo,lesnik} which aims to do topology in a
synthetic fashion: Any set should have an intrinsic topology and any map should
be automatically continuous with respect to this intrinsic topology.

This automatic continuity reflects as stability of open subfunctors under
pullbacks:

\begin{lemma}Let~$f : X \to Y$ be a morphism in$~\Zar(S)$. Let~$U
\hookrightarrow Y$ be an open subfunctor. Then its pullback along~$f$,
denoted~``$f^{-1}U \hookrightarrow X$'', is too an open subfunctor.\end{lemma}

\begin{proof}From the internal point of view of~$\Zar(S)$, the
subfunctor~$f^{-1}U \hookrightarrow X$ looks like the inclusion of the
preimage~$f^{-1}[U] \subseteq X$.

So, to verify the claim, let internally an element~$x\?X$ be given. We are to
show that the truth value of~``$x \in f^{-1}[U]$'' is open. This truth value
equals the truth value of~``$f(x) \in U$'' which is open by assumption, and is
therefore open.
\end{proof}

\begin{rem}In the internal language of toposes used to carry out synthetic
differential geometry, there is the concept of an \emph{Penon-open}
subset~\cite[Chapitre~III]{penon}: A subset~$U \subseteq X$ is Penon-open if
and only if
\[ \forall x \in U\_ \forall y\?X\_
  (x \neq y) \vee (y \in U). \]
This notion is not useful in synthetic algebraic geometry, since it is much too
weak: Any subset of the one-element set~$1$ is Penon-open. However, not every
subfunctor of the terminal functor in~$\Zar(S)$ is an open subfunctor.
\end{rem}

In many flavours of synthetic topology, open truth values~$\varphi$
are~$\neg\neg$-stable in that~$\neg\neg\varphi$ implies~$\varphi$. With a
small caveat, this is true for open truth values in the big Zariski topos as
well.

\begin{prop}\label{prop:open-truth-values-stable}
Let~$U \hookrightarrow 1$ be a subfunctor in~$\Zar(S)$ such
that~$\Zar(S) \models \neg\neg U$. Then in any of the following
situations it follows that~$\Zar(S) \models U$:
\begin{enumerate}
\item $U$ is a quasicompact open truth value.
\item $U$ is an arbitrary open truth value and the site used to
define~$\Zar(S)$ is closed under domains of closed immersions. (This is for
instance satisfied for the sites built using a Grothendieck or a partial
universe. It is satisfied for the parsimonious sites if~$S$ is locally
Noetherian.)
\end{enumerate}
\end{prop}

\begin{proof}We give two proofs, an internal one and an external one, since
they employ different ideas.

\emph{Internal proof.} Since~$U$ is an open truth value, there exists an ideal~$J
\subseteq \affl$ such that~$\affl/J$ is synthetically quasicoherent and such that~$U$ holds
if and only if~$1 \in J$. By assumption, the element~1 is \notnot an element
of~$J$; we want to verify that it's actually an element of~$J$.

By Scholium~\ref{scholium}, the canonical homomorphism
\[ \affl/J \longrightarrow [\Spec(\affl/J), \affl] \]
is bijective; the assumptions of that scholium are satisfied in either of
the two situations. The set~$\Spec(\affl/J)$ is isomorphic to~$\brak{J = (0)}$.
Since~$\neg\neg(1 \in J)$, we also have~$\neg(J = (0))$.
Therefore~$\Spec(\affl/J)$ is empty and the codomain of the displayed
isomorphism is the zero algebra. Thus~$\affl/J$ is trivial as well, showing~$1
\in J$.

\emph{External proof.} Since~$U \hookrightarrow 1$ is an open subfunctor, there
is an open subscheme~$S_0 \subseteq S$ such that a morphism~$f : T \to S$
factors over~$S_0$ if and only if~$U(T)$ is inhabited. In both situations it's
possible to endow~$X \defeq S \subseteq S_0$ with the structure of a closed
subscheme such that~$X$ is contained in the site used to define~$\Zar(S)$.
By the universal property of~$S_0$, we have~$X \models \neg U$. Since~$\Zar(S)
\models \neg\neg U$, it follows that~$X$ is empty. Therefore~$S_0 = S$ and~$U$
is globally inhabited.
\end{proof}

\begin{cor}\label{cor:open-subfunctors-d1}
Let~$\gamma : \Delta \to X$ be a morphism in~$\Zar(S)$.
Let~$U \hookrightarrow X$ be an open subfunctor such that~$\Zar(S) \models
\gamma(0) \in U$. Then, in any of the situations in
Proposition~\ref{prop:open-truth-values-stable}, the morphism~$\gamma$ factors
over~$U$.
\end{cor}

\begin{proof}We give an internal proof. Let~$\varepsilon \in \Delta$.
Then~$\neg\neg(\varepsilon = 0)$. Therefore~$\neg\neg(\gamma(\varepsilon) \in
U)$. Since being an element of~$U$ is~$\neg\neg$-stable, it follows
that~$\gamma(\varepsilon) \in U$.
\end{proof}

\begin{rem}Subobjects~$U \hookrightarrow X$ for which any morphism~$\gamma :
\Delta \to X$ with~$\gamma(0) \in U$ factors over~$U$ are called~``$D_1$-open''
in the literature on synthetic differential
geometry~\cite[page~60]{reyes:wraith:note-tangent-bundles}.
Corollary~\ref{cor:open-subfunctors-d1} shows that open subfunctors
are~$D_1$-open.
\end{rem}

In ordinary scheme theory, an inclusion of a standard open subset~$D(f)
\hookrightarrow X$ is isomorphic to the structure morphism of the relative
spectrum~$\RelSpec_X \O_X[f^{-1}]$. Inclusions of more general open subsets
can typically not be described using the relative spectrum construction, the
standard example being the inclusion~$\AA^2_k \setminus \{ 0 \} \hookrightarrow
\AA^2_k$ whose domain is not affine.

An interesting feature of the internal universe of the big Zariski topos is
that it's flexible enough to express \emph{any} open subset as a spectrum.
The contradiction is only apparent since the algebra used for constructing
such a spectrum is not in general quasicoherent.

\begin{prop}Let~$U \hookrightarrow 1$ be an open truth value. In any of the
situations of Proposition~\ref{prop:open-truth-values-stable}, there is a
(not necessarily quasicoherent)~$\affl$-algebra~$A$ such that the inclusion is
isomorphic to the morphism~$\Spec(A) \to 1$.\end{prop}

\begin{proof}The open truth value~$U$ is given by an ideal~$J \subseteq \affl$
such that~$\affl/J$ is synthetically quasicoherent and such that~$U$ holds if
and only if~$1 \in J$. We set~$A \defeq \affl[M^{-1}]$, where~$M$ is the
multiplicatively closed subset
\[ M \defeq \{ f \? \affl \,|\, 1 \in J \Rightarrow \speak{$f$ \inv} \} \subseteq \affl. \]
The spectrum of~$A$ is inhabited if and only if~$M \subseteq (\affl)^\times$,
in which case the unique element of~$\Spec(A)$ is the inverse of the
localization morphism~$\affl \to \affl[M^{-1}]$. Thus~$\Spec(A)$ is isomorphic
to~$\brak{M \subseteq (\affl)^\times}$. Therefore we have to verify that~$U$
holds if and only if~$M \subseteq (\affl)^\times$.

The ``only if'' direction is trivial.

For the ``if'' direction, we exploit the~$\neg\neg$-stability of~$U$. If~$\neg
U$, then~$\neg(1 \in J)$, so~$M = \affl$, and since~$M \subseteq (\affl)^\times$ by
assumption, it follows that zero is invertible. This is a contradiction.
Thus~$\neg\neg U$.
\end{proof}

\begin{proof}[Proof of Proposition~\ref{prop:characterization-open-subfunctor}]
\XXX{fill in proof}
\end{proof}

\begin{rem}The radical~$\sqrt{J}$ of the ideal~$J$ appearing in
Proposition~\ref{prop:characterization-open-subfunctor} is unique:
It is equal to the radical ideal
\[ K \defeq \{ f\?\affl \,|\, \speak{$f$ \inv} \Rightarrow (x \in U) \}
\subseteq \affl. \]
It's obvious that~$J \subseteq K$ and therefore~$\sqrt{J} \subseteq K$.
For the converse direction, let~$f \in K$ be given. Since~$\affl/J$ is
synthetically quasicoherent, the canonical map
\[ (\affl/J)[f^{-1}] \longrightarrow [ \Spec(\affl[f^{-1}]), \affl/J ] \]
is bijective. Since~$\Spec(\affl[f^{-1}]) \cong \brak{\speak{$f$ \inv}}$, the image
of~$1$ is zero: If~$\Spec(\affl[f^{-1}])$ is inhabited, the element~$f$ is invertible and
therefore~$x$ is an element of~$U$. This implies that~$1 \in J$.
Thereby~$\affl/J = 0$. By injectivity of the canonical map, the
algebra~$(\affl/J)[f^{-1}]$ is zero. Therefore~$f^n \in J$ for some natural
number~$n$.\end{rem}

\begin{rem}In view of the previous remark, one might hope to be able to simplify the
condition in Proposition~\ref{prop:characterization-open-subfunctor} as
follows: ``For any~$x\?X$, the quotient~$\affl/K$ modulo the ideal~$K = \{
f\?\affl \,|\, \speak{$f$ \inv} \Rightarrow (x \in U) \}$ is synthetically quasicoherent.''
However, this doesn't work out.
This statement implies the condition in the proposition, but the converse
direction does not hold, since~$\affl/K \cong \affl/\sqrt{J}$ might fail to be
synthetically quasicoherent. For instance that's the case if~$U = \emptyset$; then~$K =
\sqrt{(0)}$ by Proposition~\ref{prop:a1-nilp}. The
quotient~$\affl/\sqrt{(0)}$ is not synthetically quasicoherent by
Remark~\ref{rem:radical-not-qcoh}.\end{rem}

\begin{rem}\label{rem:open-geometric-morphism}
There is the notion of an open geometric morphism of toposes. For the big
Zariski toposes, this notion is not related to open morphisms or open
immersions between schemes: If~$X \to S$ is any morphism of schemes, the
induced geometric morphism~$\Zar(X) \to \Zar(S)$ is isomorphic to the canonical
morphism~$\Zar(S)/\ul{X} \to \Zar(S)$, as detailed in
Section~\ref{sect:change-of-base}). Geometric morphisms of the form~$\E/A \to
\E$ are always open.\end{rem}


\subsection{Closed immersions}

\begin{defn}In the context of a specified local ring, as for instance~$\affl$
of the big Zariski topos of a scheme, a truth value~$Z \subseteq 1$ is
\emph{closed} if and only if there exists an ideal~$J \subseteq \affl$ such
that~$\affl/J$ is synthetically quasicoherent
(Definition~\ref{defn:synth-qcoh}) and such that~$Z$ holds if and only if~$J =
(0)$.\end{defn}

In other words, a truth value~$Z \subseteq 1$ is closed if and only if~$Z$ is
isomorphic to the spectrum of a synthetically quasicoherent quotient algebra
of~$\affl$.

\begin{ex}Let~$f \? \affl$. Then~``$f = 0$'' is a closed truth value with
witnessing ideal~$J = (f)$. More generally, if~$f_1,\ldots,f_n \? \affl$, the
truth value~``$f_1 = \cdots = f_n = 0$'' is closed.\end{ex}

\begin{defn}In the context of a specified local ring, a map~$Z \to X$ of sets is a
\emph{synthetic closed immersion} if and only if it is injective and for
any~$x\?X$ the truth value of~``the fiber of~$x$ is inhabited'' is
closed.\end{defn}

\begin{ex}The inclusion~$\{0\} \hookrightarrow \affl$ is a synthetic closed
immersion. More generally, for any functions~$f_1,\ldots,f_m : (\affl)^n \to
\affl$, the inclusion of the set of their common zeros in~$(\affl)^n$ is a
synthetic closed immersion.\end{ex}

\begin{ex}Let~$X \in \Zar(S)$. Let~$f \? [X,\affl]$ from the internal point of
view. The inclusion~$\{ x\?X \,|\, f(x) = 0 \} \hookrightarrow X$ is a
synthetic closed immersion.\end{ex}

\begin{prop}\label{prop:char-closed-immersion}
Let~$X$ be an~$S$-scheme.
\begin{enumerate}
\item Let~$Z \subseteq X$ be a closed subscheme. Then the subfunctor~$\ul{Z}
\hookrightarrow \ul{X}$ is a synthetic closed immersion from the internal point
of view of~$\Zar(S)$.
\item If~$\ul{X}$ is locally representable by an object of the site used to
define~$\Zar(S)$, any synthetic closed immersion~$Z \hookrightarrow \ul{X}$ is
isomorphic to the subfunctor associated to a closed subscheme of~$X$.
\end{enumerate}
\end{prop}

\begin{proof}To verify the first claim, let a
quasicoherent~$\O_X$-algebra~$\J_0$ be given such that the closed subscheme~$Z
\subseteq X$ is the vanishing scheme of~$\J_0$. Following the translation with
the Kripke--Joyal semantics, let~$f : T \to S$ be an object of the site used to
define~$\Zar(S)$ and let~$x \in \ul{X}(T)$. We define~$J \defeq (f^*
\J_0)^\Zar \in \Zar(S)/\ul{T}$. Then $T \models \speak{$\affl/J$ is
synthetically quasicoherent}$ and $T \models (x \in \ul{Z} \Leftrightarrow J =
(0))$, therefore~``$x \in \ul{Z}$'' is a closed truth value.
\XXX{For the second claim, ...}
% Idea: First show uniqueness of J.
% Let J, K be given.
% Let f in J.
% Then ...
\end{proof}

\begin{rem}\label{rem:closed-geometric-morphism}
There is the notion of a closed geometric morphism of toposes. For
an arbitrary topos~$\E$ and an object~$A \in \E$, the canonical geometric
morphism~$\E/A \to E$ is closed if and only if
\[ \forall U \subseteq A\_
  \forall \varphi \? \Omega\_ \quad
  A \subseteq (U \cup \{ x \in A \,|\, \varphi \}) \quad\Longrightarrow\quad
  (A \subseteq U) \vee \varphi \]
from the internal point of view of~$\E$. If~$X \to S$ is a closed morphism of
schemes, then the induced geometric morphism~$\Sh(X) \to \Sh(S)$ between the
little Zariski toposes is closed in this sense.
% Use criterion given in Wedhorn, then employ classical logic.

However, the induced geometric morphism~$\Zar(X) \to \Zar(S)$ is typically not
closed. For instance, if~$X \to S$ is the embedding of a closed subset~$V(f)$
with~$f \in \Gamma(S,\O_S)$, then the morphism~$\Zar(X) \to \Zar(S)$ is
isomorphic to~$\Zar(S)/\ul{V(f)} \to \Zar(S)$, as discussed in
Section~\ref{sect:change-of-base}. In the special case~$U \defeq \emptyset$ and
$\varphi \defeq \brak{f = 0}$, the displayed closedness condition simplifies
to~$\speak{$f$ \inv} \vee (f = 0)$. This is typically not true in the internal
language of~$\Zar(S)$. A specific counterexample is given in
Example~\ref{ex:translation-equivalence}.
\end{rem}


\subsection{Surjective morphisms}

\begin{prop}\label{prop:char-surjective-morphisms}
Let~$f : X \to S$ be an arbitrary~$S$-scheme. Consider the following
statements:
\begin{enumerate}
\item The morphism~$f$ is surjective.
\item From the internal point of view of~$\Zar(S)$, it's not the case that~$\ul{X}$ is
empty, that is
\[ Zar(S) \models \neg\neg(\speak{$\ul{X}$ is inhabited}). \]
\end{enumerate}
If~$X$ is locally contained in the site used to define~$\Zar(S)$ (for instance,
if~$X$ is contained in the universe used to define~$\Zar(S)$ or if one of the
parsimonious sites is used and~$X$ is locally of finite presentation over~$S$),
then~(1) implies~(2). The converse holds if the site is closed under
taking spectra of residue fields or if one of the parsimonious sites is used
and~$f$ is quasicompact and quasiseparated.
\end{prop}

\begin{proof}The translation of the internal statement using the Kripke--Joyal
semantics is:
\begin{indentblock}For any~$S$-scheme~$T$ of the site used to define~$\Zar(S)$,
if~\mbox{$\ul{X \times_S T} = \ul{\emptyset}$} (as functors of points
of~$T$-schemes), then~$T = \emptyset$.
\end{indentblock}
In the case that the site used to define~$\Zar(S)$ is closed under taking
spectra of residue fields, this implies that~$f$ is surjective as follows.
Let~$s \in S$ be an arbitrary point. The~$S$-scheme~$T \defeq \Spec(k(s))$ is
not empty. Therefore the fiber~$X_s = X \times_S T$ of~$f$ over~$s$ is not empty.

If one of the parsimonious sites is used to define~$\Zar(S)$, we can't apply
the assumption to the~$S$-scheme~$T = \Spec(k(s))$ since it might not be
locally of finite presentation over~$S$. We therefore argue as follows. Without
loss of generality, we may assume that~$S$ is affine. Writing~$k(s)$ as the
canonical filtered colimit of all finitely presented~$\Gamma(S,\O_S)$-algebras
mapping to~$k(s)$ (and then rewriting this filtered colimit as a directed
colimit~\cite[Theorem~1.5]{adamek:rosicky:presentable}), we see
that~$\Spec(k(s))$ is the directed limit of an inverse system of finitely
presented affine~$S$-schemes~$T_i$ with affine transition maps.
In particular, the structure morphisms~$T_i \to S$ are quasicompact and
quasiseparated. By the assumption that the morphism~$X \to S$ is quasicompact and
quasiseparated as well, the schemes~$X \times_S T_i$ are quasicompact and
quasiseparated (as absolute schemes). Therefore, if~$X_s = X \times_S T
= \lim_i (X \times_S T_i)$ is empty, then so is~$X \times_S T_i$ for
some~$i$~\stacksproject{01ZC}. Thus~$T_i = \emptyset$ and
therefore~$\Spec(k(s)) = \emptyset$; this is a contradiction.

For the converse direction, let an~$S$-scheme~$T$ contained in the site used to
define~$\Zar(S)$ be given such that~$\ul{X \times_S T} = \ul{\emptyset}$ as
functors of points of~$T$-schemes. The assumption that~$X$ is locally contained
in the site used to define~$\Zar(S)$ implies that~$X \times_S T = \emptyset$ as
schemes. Since the base change~$X \times_S T \to T$ of~$f$ is surjective, this
implies that~$T$ is empty.
\end{proof}

\begin{cor}\label{cor:char-surjective-morphisms-relative}
Let~$p : X \to Y$ be a morphism of~$S$-schemes. Assume that~$Y$ is is locally
contained in the site used to define~$\Zar(S)$.
Further assume that the site used to define the big Zariski toposes are closed
under taking spectra of residue fields or that the parsimonious sites are used
and that~$p$ is quasicompact and quasiseparated.
Consider the following statements:
\begin{enumerate}
\item The morphism~$p$ is surjective.
\item From the internal point of view of~$\Zar(S)$ all fibers of~$\ul{p}$ are
nonempty, that is
\[ \Zar(S) \models \forall y\?\ul{Y}\_
  \neg\neg \exists x\?\ul{X}\_ \ul{p}(x) = y. \]
\end{enumerate}
If~$X$ is locally contained in the
site used to define~$\Zar(Y)$, then~(1) implies~(2). The converse holds if the
sites used to define the big Zariski toposes are closed under taking spectra of
residue fields or that the parsimonious sites are used and that~$p$ is
quasicompact and quasiseparated.
\end{cor}

\begin{proof}Immediate using Proposition~\ref{prop:char-surjective-morphisms}
and the equivalence~$\Zar(Y) \simeq \Zar(S)/\ul{Y}$,
as explained in Section~\ref{sect:change-of-base}.
\end{proof}

\begin{rem}Combining Proposition~\ref{prop:notnot-in-big-zariski-topos} and
Proposition~\ref{prop:char-surjective-morphisms} yields a proof of the fact
that a quasicompact, quasiseparated, and locally finitely presented morphism~$X \to S$, where~$S$ is locally of
finite type over a field, is surjective if it is surjective on closed points.
\end{rem}

\begin{rem}In the case that the parsimonious sites are used, the assumption in
Proposition~\ref{prop:char-surjective-morphisms} that the morphism~$f$ is
quasicompact can't be dropped. For instance, let~$k$ be an
algebraically closed field. Then the canonical morphism
\[ X \defeq \coprod_{a \in k} \Spec(k[X]/(X-a)) \lra \Spec(k[X]) =\vcentcolon S \]
is surjective on closed points. By Proposition~\ref{prop:notnot-in-big-zariski-topos},
it's not the case that~$\ul{X}$ is empty from the internal point of view
of~$\Zar(S)$. However, the morphism is not surjective.
\end{rem}


\subsection{Universally injective morphisms}

\begin{prop}\label{prop:char-univ-injective-morphisms}
Let~$f : X \to S$ be an~$S$-scheme which is locally contained in the site
used to define~$\Zar(S)$. In the case that the parsimonious sites are used to
define~$\Zar(S)$, further assume that~$f$ is quasicompact and quasiseparated.
Then the following statements are equivalent:
\begin{enumerate}
\item The morphism~$f$ is universally injective.
\item The diagonal morphism~$X \to X \times_S X$ is surjective.
\item From the internal point of view of~$\Zar(S)$, any given elements of~$\ul{X}$
are \notnot equal, that is
\[ \Zar(S) \models \forall x\?\ul{X}\_ \forall x'\?\ul{X}\_ \neg\neg(x = x'). \]
\end{enumerate}
\end{prop}

\begin{proof}The equivalence~``(1)~$\Leftrightarrow$~(2)'' is
well-known~\stacksproject{01S4}. The equivalence~``(2)~$\Leftrightarrow$~(3)''
is by Corollary~\ref{cor:char-surjective-morphisms-relative} and the fact that,
internally, there is \notnot a preimage for any element of~$\ul{X} \times
\ul{X}$ under the diagonal map~$\ul{X} \to \ul{X} \times \ul{X}$ if and only if
any given elements of~$X$ are \notnot equal.
\end{proof}

\begin{cor}Let~$p : X \to Y$ be a morphisms of~$S$-schemes which are locally
contained in the site used to define~$\Zar(S)$. In the case that the
parsimonious sites are used to define~$\Zar(S)$, further assume that~$f$ is
quasicompact and quasiseparated. Then the following statements
are equivalent:
\begin{enumerate}
\item The morphism~$p$ is universally injective.
\item From the internal point of view of~$\Zar(S)$, any given elements of any
fiber of~$p$ are \notnot equal.
\end{enumerate}
\end{cor}

\begin{proof}Immediate using Proposition~\ref{prop:char-univ-injective-morphisms}
and the equivalence~$\Zar(Y) \simeq \Zar(S)/\ul{Y}$,
as explained in Section~\ref{sect:change-of-base}.
\end{proof}


\subsection{Universally closed morphisms}

\begin{defn}In the context of a specified local ring, a set~$X$ is
\emph{synthetically closed} if and only if, for any synthetic closed
immersion~$Z \hookrightarrow X$, there is a closed truth value~$\varphi$ such
that~$Z$ is \notnot inhabited if and only if~$\neg\neg\varphi$.
\end{defn}

\begin{ex}Any singleton set is synthetically closed.\end{ex}

\begin{ex}The specified local ring~$R$ is typically not synthetically closed.
For let~$f \? R$ be an element. Then the inclusion~$Z \defeq \{ g \? R \,|\,
fg - 1 = 0 \}$ is a synthetic closed immersion. The set~$Z$ is \notnot inhabited if and
only if~$f$ is \notnot invertible if and only if~$f$ is invertible (by
Proposition~\ref{prop:a1-field}); \XXX{continue}\end{ex}

\begin{prop}\label{prop:char-closed-image}
Assume that~$S$ is locally Noetherian. Let~$f : X \to S$ be a
finitely presented morphism. In the situation that one the parsimonious
sites is used to define~$\Zar(S)$, the following statements are equivalent:
\begin{enumerate}
\item The morphism~$f$ has closed image.
\item The morphism~$f$ has universally closed image, that is for
any~$S$-scheme~$T$ the image of the induced morphism~$X \times_S T \to T$ is
closed.
\item $\Zar(S) \models \exists \varphi\?\Omega\_
  \speak{$\varphi$ is a closed truth value} \wedge
  (\neg\neg(\speak{$\ul{X}$ inhabited}) \Leftrightarrow \neg\neg\varphi)$.
\end{enumerate}
\end{prop}

\begin{proof}The direction~``(2)~$\Rightarrow$~(1)'' is trivial, and the
direction~``(1)~$\Rightarrow$~(2)'' is immediate, since the image of~$X
\times_T T \to T$ is the preimage of the image of~$f$.

For the~``(1)~$\Rightarrow$~(3)'' direction, we may pick the subfunctor
of~$\ul{S}$ induced by the closed immersion~$\im(f) \hookrightarrow S$ as the
sought truth value. This truth value is closed by
Proposition~\ref{prop:char-closed-immersion} and its double negation is
equivalent to~$\speak{$\ul{X}$ is inhabited}$ by
Lemma~\ref{lemma:image-coincides}.

For the converse direction, we see that, after passing to an open covering
of~$S$ which we won't reflect in the notation, there is a closed subfunctor~$Z
\hookrightarrow 1$ such that~$\Zar(S) \models \neg\neg(\speak{$\ul{X}$
inhabited}) \Leftrightarrow \neg\neg(\speak{$Z$ is inhabited})$. By
Proposition~\ref{prop:char-closed-immersion}, this subfunctor is the functor of
points of a closed subscheme of~$S$. Since~$S$ is locally Noetherian, this
subscheme is locally of finite presentation over~$S$. Therefore
Lemma~\ref{lemma:image-coincides} is applicable. This concludes the proof.
\end{proof}

\begin{cor}Assume that~$S$ is locally Noetherian. Let~$f : X \to S$ be a
finitely presented morphism. In the situation that one the parsimonious
sites is used to define~$\Zar(S)$, the following statements are equivalent:
\begin{enumerate}
\item The morphism~$f$ is universally closed.
\item $\Zar(S) \models \speak{$\ul{X}$ is synthetically closed}$.
\end{enumerate}
\end{cor}

\begin{proof}Immediate using Proposition~\ref{prop:char-closed-immersion} and
Proposition~\ref{prop:char-closed-image}.
\end{proof}


\subsection{Quasicompact and quasiseparated morphisms}

\begin{defn}\label{defn:synthetic-scheme}
In the context of a specified local ring~$R$:
\begin{enumerate}
\item A \emph{synthetic affine scheme} is a set which is isomorphic (as a set)
to the synthetic spectrum of a synthetically
quasicoherent~$R$-algebra.
\item A \emph{quasicompact synthetic scheme} is a set~$X$ which admits a finite
open covering~$X = \bigcup_{i=1}^n U_i$ by synthetic affine schemes~$U_i$.
\item A \emph{locally finitely presented quasicompact synthetic scheme} is a
set~$X$ which admits a finite open covering~$X = \bigcup_{i=1}^n U_i$ such that
the sets~$U_i$ are isomorphic to spectra of finitely presented~$R$-algebras.
\item A \emph{finitely presented synthetic scheme} is a set~$X$ which admits a
finite open covering~$X = \bigcup_{i=1}^n U_i$ such that the sets~$U_i$ are
isomorphic to spectra of finitely presented~$R$-algebras and such that the
intersections~$U_i \cap U_j$ can be covered by finitely many open subsets which
are isomorphic to spectra of finitely presented~$R$-algebras.
\end{enumerate}
\end{defn}

\begin{prop}Let~$X \in \Zar(S)$ be a Zariski sheaf. Consider the following
statements:
\begin{enumerate}
\item $X$ is the functor of points of a quasicompact~$S$-scheme.
\item $X$ is the functor of points of a locally finitely presented quasicompact~$S$-scheme.
\item $X$ is the functor of points of a finitely presented~$S$-scheme (locally
finitely presented, quasicompact, and quasiseparated).
\item[(1')] $\Zar(S) \models \speak{$X$ is a quasicompact synthetic scheme}$.
\item[(2')] $\Zar(S) \models \speak{$X$ is a locally finitely presented quasicompact synthetic scheme}$.
\item[(3')] $\Zar(S) \models \speak{$X$ is a finitely presented synthetic scheme}$.
\end{enumerate}
Then~(1)~$\Rightarrow$~(1'), (2)~$\Leftrightarrow$~(2'),
and~(3)~$\Leftrightarrow$~(3').
\end{prop}

\begin{proof}For proving~(1)~$\Rightarrow$~(1'), (2)~$\Rightarrow$~(2'),
and~(3)~$\Rightarrow$~(3'), we may assume that~$S$ is affine, since the
internal language is local. Let~$X_0$ be an~$S$-scheme representing~$X$. Since
the structure morphism~$X_0 \to S$ is quasicompact and~$S$ is quasicompact,
there exist finitely many open affine subschemes~$U_i \subseteq X_0$ which
cover~$X_0$. By Proposition~\ref{prop:char-open-immersion}, the
subfunctors~$\ul{U_i} \hookrightarrow X$ are synthetic open immersions from the
internal point of view of~$\Zar(S)$. The internal union~$\bigcup_i \ul{U_i}
\hookrightarrow X$ is the functor
\[ T/S \longmapsto \{ f : T \to X_0 \,|\,
  \text{locally, the morphism $f$ factors over one of the opens~$U_i$} \} \]
and therefore coincides with~$X$.

Since each scheme~$U_i$ can be realized as a relative spectrum of a
quasicoherent~$\O_S$-algebra, both~$U_i$ and~$S$ being affine, the sets~$U_i$
are synthetic affine schemes from the internal point of view. This
proves~(1)~$\Rightarrow$~(1').

For the directions (2)~$\Rightarrow$~(2') and~(3)~$\Rightarrow$~(3'), the
covering~$X_0 = \bigcup_i U_i$ can be chosen appropriately: such that the~$U_i$
are spectra of finitely presented~$\O_S$-algebras or that the intersections are
\XXX{continue proof}
\end{proof}

\begin{rem}One can reasonably wonder why we didn't include the following notion
in Definition~\ref{defn:synthetic-scheme}: A \emph{synthetic scheme} is a
set~$X$ which admits an arbitrary open covering by synthetic affine schemes.
The reason is that, with this definition, any subset~$X$ of the singleton
set~$1 = \{\star\}$ is a synthetic scheme, since it admits the open affine
covering~$X = \bigcup \{ 1 \,|\, \star \in X \}$. But not any subfunctor of the
terminal functor is representable by a scheme.

This phenomenon is well-known in synthetic topology; one has to put some
restrictions on the kind of allowed open coverings. Being finite is the
simplest such condition.
\end{rem}


\subsection{Quasiseparated morphisms}

\subsection{Proper morphisms}


\section{Case studies}

\subsection{Punctured plane}

\begin{defn}The \emph{synthetic punctured plane} is the set~$P \defeq (\affl)^2
\setminus \{ 0 \}$.\end{defn}

\begin{prop}The synthetic punctured plane, as constructed by the internal
language of~$\Zar(S)$, is the functor of points of the ordinary punctured plane
over~$S$, that is the open subscheme~$D(X) \cup D(Y) \hookrightarrow
\AA^1_S$.\end{prop}

\begin{prop}The evaluation morphism~$\affl[X,Y] \to [P, \affl]$ is bijective.\end{prop}

%\begin{proof}To verify injectivity, let functions~$f, g : (\affl)^2 \to \affl$
%which agree on~$P$ be given. Since~$\affl$ is synthetically quasicoherent, the
%canonical maps~$\affl[X,Y] \to [(\affl)^2, \affl]$ and~$\affl[X,Y][X^{-1}] \to
%[D(X), \affl]$ are bijective. Therefore~$f$ and~$g$ are given by unique
%polynomials in~$\affl[X,Y]$. Since~$f$ and~$g$ agree on~$D(X)$, these
%polynomials are the same when viewed as elements of~$\affl[X,Y][X^{-1}]$.
%Since~$X \in \affl[X,Y]$ is a regular element, the localization morphism is
%injective. Thus these polynomials agree as elements of~$\affl[X,Y]$, whereby~$f
%= g$ as claimed.
%\end{proof}

\begin{proof}The synthetic punctured plane can be expressed as the pushout
\[ P \cong D(X) \amalg_{D(X) \cap D(Y)} D(Y). \]
Therefore we have the chain of isomorphisms
\begin{align*}
  [P,\affl] &\cong
  [D(X) \amalg_{D(X) \cap D(Y)} D(Y), \affl] \\
  &\cong [D(X), \affl] \times_{[D(X) \cap D(Y), \affl]} [D(Y), \affl] \\
  &\cong \affl[X,X^{-1}] \times_{\affl[XY, (XY)^{-1}]} \affl[Y,Y^{-1}] \\
  &\cong \affl[X,Y].
\end{align*}
The penultimate isomorphism exploits the synthetic quasicoherence of~$\affl$,
which guarantees that the canonical map
\[ \affl[X,X^{-1}] \longrightarrow [\Spec(\affl[X,X^{-1}]), \affl] \cong
  [D(X), \affl] \]
is bijective. The ultimate isomorphism rests on the purely algebraic argument
that elements of~$\affl[X,X^{-1}]$ and~$\affl[Y,Y^{-1}]$ which agree as
elements of~$\affl[(XY),(XY)^{-1}]$ are both given by an element
of~$\affl[X,Y]$ and in fact by the same element.
\end{proof}

\begin{cor}The punctured plane is not affine.
\end{cor}

\begin{proof}The canonical map~$P \to \Spec([P, \affl])$ is isomorphic to the
inclusion~$P \hookrightarrow (\affl)^2$ and therefore not bijective.
\end{proof}


\subsection{Cohomology of projective space}

\subsection{Categorical group quotients}

\subsection{Grassmannian}

Let~$\V$ be a finite locally free~$\O_S$-module.
We want to illustrate the synthetic approach by verifying the basic fact that the Grassmannian~$\Gr(\V,r)$ of
rank-$r$ locally free quotients of~$\V$, defined as a certain functor of points,
is representable by an~$S$-scheme of finite presentation using the internal
language of~$\Zar(S)$.

\begin{defn}The \emph{Grassmannian}~$\Gr(\V,r)$ is the functor which associates
to an~$S$-scheme~$f : T \to S$ the set
\[ \Gr(\V,r)(T) \defeq \{
  \text{$U \subseteq f^*\V$ sub-$\O_T$-module} \,|\,
  \text{$(f^*\V)/U$ is locally free of rank~$r$} \}. \]
\end{defn}

\begin{defn}The \emph{synthetic Grassmannian} of rank-$r$ quotients of a
module~$V$ is the set
\[ \Gr(V,r) \defeq \{ \text{$U \subseteq V$ submodule} \,|\,
  \text{$V/U$ is free of rank~$r$} \}. \]
\end{defn}

We could just as well define the synthetic Grassmannian somewhat more directly
as the set of free rank~$r$-quotients (up to isomorphism). This set is canonically
isomorphic to the Grassmannian as we chose to define it, by mapping a
quotient~$\pi : V \twoheadrightarrow Q$ to the kernel of~$\pi$.

\begin{prop}The synthetic Grassmannian of~$\V$, as constructed by the internal
language of~$\Zar(S)$ where~$\V$ looks like an ordinary free module, coincides
with the functorially defined Grassmannian.\end{prop}

\begin{proof}Immediate from
Definition~\ref{defn:interpretation-internal-constructions} and
Proposition~\ref{prop:locally-free-big-zariski}.
\end{proof}

Having established that the internally constructed synthetic Grassmannian
actually describes the external Grassmannian which we're interested in, we can
switch to a fully internal perspective. We'll reflect this switch notationally
by referring to the~$\affl$-module~$V \defeq \V^\Zar$ instead of~$\V$.

We define for any free submodule~$W \subseteq V$ of rank~$r$ the subset
\[ G_W \defeq \{ U \in \Gr(V,r) \,|\, \text{$W \to V \to V/U$ is bijective} \}. \]
This sets admits a more concrete description, since it is in canonical bijection
to the set
\[ G_W' \defeq \{ \pi : V \to W \,|\, \pi \circ \iota = \id \} \]
of all splittings of the inclusion~$\iota : W \hookrightarrow V$: An element~$U
\in G_W$ corresponds to the splitting~$V \twoheadrightarrow V/U
\xrightarrow{({\cong})^{-1}} W$. Conversely, a splitting~$\pi$ corresponds to~$U
\defeq \ker(\pi) \in G_W$.

\begin{prop}The union of the subsets~$G_W$ is~$\Gr(V,r)$.\end{prop}

\begin{proof}Let~$U \in \Gr(V,r)$. Then there exists a
basis~$([v_1],\ldots,[v_r])$ of~$V/U$. The family~$(v_1,\ldots,v_r)$ is
linearly independent in~$V$, therefore the submodule~$W \defeq
\operatorname{span}(v_1,\ldots,v_r) \subseteq V$ is free of rank~$r$. The
canonical linear map~$W \hookrightarrow V \twoheadrightarrow V/U$ maps
the basis~$(v_i)_i$ to the basis~$([v_i])_i$ and is therefore bijective. Thus~$U
\in G_W$.\end{proof}

\begin{prop}The sets~$G_W$ are (quasicompact-)open subsets of~$\Gr(V,r)$.
\end{prop}

\begin{proof}Let~$U \in \Gr(V,r)$. Then~$U \in G_W$ if and only if
the canonical linear map~$W \hookrightarrow V \twoheadrightarrow V/U$
is bijective. Since~$W$ and~$V/U$ are both free modules of rank~$r$, this map is
given by an~$(r \times r)$-matrix~$M$ over~$\affl$; therefore it's bijective if
and only if the determinant of~$M$ is invertible.

Thus we've found a number which is invertible if and only if~$U \in G_W$. By
Corollary~\ref{cor:sufficient-criterion-open-subfunctor}, the truth value of~``$U
\in G_W$'' is open.\end{proof}

\begin{prop}The sets $G_W$ are affine. Moreover, the algebras which the~$G_W$ are
spectra of are finitely presented.\end{prop}

\begin{proof}The set of all linear maps~$V \to W$ is the spectrum of
the~$\affl$-algebra~$A \defeq \Sym(\Hom_\affl(V,W)^\vee)$, since
\begin{align*}
  \Spec(A) &=
  \Hom_{\mathrm{Alg}(\affl)}(\Sym(\Hom_{\mathrm{Mod}(\affl)}(V,W)^\vee), \affl) \\
  &\cong \Hom_{\mathrm{Mod}(\affl)}(\Hom_{\mathrm{Mod}(\affl)}(V,W)^\vee, \affl) \\
  &=\Hom_{\mathrm{Mod}(\affl)}(V,W)^{\vee\vee} \\
  &\cong \Hom_{\mathrm{Mod}(\affl)}(V,W).
\end{align*}
In the last step the assumption that not only~$W$, but also~$V$ is a free module
of finite rank enters. (This is the first time in this development that we need
this assumption.)

The set~$G_W'$ is a closed subset of this spectrum, namely the locus where the
generic linear map~$V \to W$ is a splitting of the inclusion~$\iota : W
\hookrightarrow V$. If we choose bases of~$V$ and~$W$,
whereby~$\Sym(\Hom_\affl(V,W)^\vee)$ is isomorphic
to~$\affl[M_{11},\ldots,M_{rn}]$, we can be more explicit: The set~$G_W'$ is
isomorphic to
\[ \Spec(k[M_{11},\ldots,M_{rn}]/(MN-I)), \]
where~$I$ is the~$(r \times r)$ identity matrix, $M$ is the generic matrix~$M =
(M_{ij})_{ij}$, and~$N$ is the matrix of~$\iota$ with respect to the chosen
bases. The notation~``$(MN-I)$'' denotes the ideal generated by the entries
of~$MN-I$.
\end{proof}

\begin{cor}The Grassmannian~$\Gr(V,r)$ is a finitely presented scheme.\end{cor}

\begin{proof}We need to verify that~$\Gr(V,r)$ admits a finite covering by
spectra of finitely presented~$\affl$-algebras. We already know that~$\Gr(V,r)$
can be covered by the open subsets~$G_W$ and that these sets are spectra of
finitely presented algebras. Therefore it remains to prove that finitely many of
these subsets suffice to cover~$\Gr(V,r)$.

In fact, if we choose an isomorphism~$V \cong (\affl)^n$, we see
that~$\binom{n}{r}$ of these subsets suffice: namely those where~$W$ is one of
the standard submodules of~$(\affl)^n$ (generated by the standard basis
vectors). For if~$U \in \Gr((\affl)^n,r)$, the surjection~$V \to V/U$ maps
the basis of at least one of these standard submodules to a basis and is
therefore bijective. This is because from any surjective~$(r \times n)$-matrix
over a local ring one can select~$r$ columns which form an linearly independent
family.
\end{proof}

\begin{prop}Let~$U \in \Gr(V,r)$. Then the tangent space at~$U$ is given by
$T_U \Gr(V,r) \cong \Hom_{\affl}(U, V/U)$.\end{prop}

\begin{proof}\XXX{fill in proof}\end{proof}

%\begin{proof}Die Menge der Tangentialvektoren an~$U$ kann kanonisch mit den
%Abbildungen~$\gamma : \Delta \to \Gr(V,r)$ mit~$\gamma(0) = U$ identifiziert
%werden. Dabei ist~$\Delta \defeq \{ \varepsilon \in k \,|\, \varepsilon^2 = 0 \}$.
%Eine solche Abbildung liftet stets zu einer Abbildung von~$\Delta$ in die Menge
%der linear unabhängigen Familien der Länge~$r$ in~$V$. Der Rest sei als
%Übungsaufgabe überlassen. Willkommen in der wunderbaren Welt synthetischer
%Geometrie.
%\end{proof}
% XXX

%\begin{rem}Wiederholt man genau dieselben Argumente in einem anderen Topos --
%einem, der für Differentialgeometrie angepasst ist -- erhält man mehr oder
%weniger die Darstellbarkeit der Grassmannschen als Mannigfaltigkeit. Das
%einzige, was fehlt, ist ein Nachweis der Glattheit.
%\end{rem}
% XXX


\section{Beyond the Zariski topology}

The Zariski topology is of course not the only interesting topology
on~$\Sch/S$. For any finer topology~$\tau$, such as the étale, smooth, or fppf
topology, the big~$\tau$-topos of~$S$, that is the topos of sheaves on~$\Sch/S$ with respect to~$\tau$, is a subtopos
of the big Zariski topos. Therefore there is a modal operator~$\Box_\tau$
in~$\Zar(S)$ reflecting the topology~$\tau$. Explicitly, for an~$S$-scheme~$T$
and a formula~$\varphi$ over~$T$, the meaning of
\[ T \models \Box_\tau \varphi \]
is that there exists a~$\tau$-covering~$(T_i \to T)_i$ of~$T$ such that~$T_i
\models \varphi$ for all~$i$ (where parameters appearing in~$\varphi$ have to
be pulled back along~$T_i \to T$). Succinctly, the formula ``$\Box_\tau
\varphi$'' means that~$\varphi$ holds~$\tau$-locally. Generalizing
Theorem~\ref{thm:box-translation-semantically} from sheaves on locales to
sheaves on arbitrary Grothendieck sites we also have
\[ \Zar(S) \models \varphi^{\Box_\tau} \qquad\text{iff}\qquad
  \Sh((\Sch/S)_\tau) \models \varphi. \]

A basic illustration of these modal operators is provided by the Kummer sequence, that is the short sequence
\[ 1 \lra \mu_n \lra \GG_\text{m} \stackrel{(\underline{\ })^n}{\lra} \GG_\text{m} \lra 1 \]
of multiplicatively-written commutative group objects in~$\Zar(S)$. With the
internal description of~$\mu_n$ and~$\GG_\text{m}$, there is a purely internal
and straightforward proof that this sequence is exact at the first two terms.
But except for trivial cases, the~$n$-th power map~$\GG_\text{m} \to
\GG_\text{m}$ will fail to be an epimorphism;
internally speaking, the statement
\[ \forall f\?(\affl)^\times\_ \phantom{\Box_\text{ét}(}\exists
g\?(\affl)^\times\_ f = g^n\phantom{)} \]
is false in general. However, if~$n$ is invertible in~$\Gamma(S,\O_S)$, the
internal statement
\[ \forall f\?(\affl)^\times\_ \Box_\text{ét}(\exists g\?(\affl)^\times\_ f = g^n) \]
\emph{is} true. In fact, the more general statement
\begin{multline*}
  \forall p\?\affl[X]\_ \speak{$p$ is monic, of positive degree, and separable}
  \Longrightarrow \\
  \Box_\text{ét}(\exists x\?\affl\_ p(x) = 0 \wedge \speak{$p'(x)$ \inv})
\end{multline*}
is true from the internal point of view, where a polynomial~$p$ is called
\emph{separable} if and only if there exists a Bézout representation~$ap + bp'
= 1$. After simplifying, the intepretation of that statement with the
Kripke--Joyal semantics is that for any~$S$-scheme~$T$ and any monic separable
polynomial~$p \in \Gamma(T,\O_T)[X]$ of positive degree there exists an étale
covering~$(T_i \to T)_i$ of~$T$ such that the pullbacks of~$p$ to each of
the~$T_i$ possess a simple zero. The required covering is given
by the canonical surjective étale map~$\RelSpec_T{\O_T[X]/(p)} \to T$.

The following theorem shows that the modal operator~$\Box_\text{ét}$
corresponding to the étale topology admits a purely internal characterization
in~$\Zar(S)$, which furthermore resonates well with the intuition about the
étale topology.

\begin{thm}Let~$S$ be a scheme. The modal operator~$\Box_\text{ét}$
in~$\Zar(S)$ corresponding to the étale topology is the smallest
operator~$\Box$ such that the~$\Box$-translation of the statement~``$\affl$ is
separably closed'' holds.\end{thm}

Here, a ring~$A$ is \emph{separably closed} if and only if
\begin{multline*}
  \forall p\?A[X]\_ \speak{$p$ is monic, of positive degree, and unramifiable}
  \Longrightarrow \\
  \exists x\?A\_ p(x) = 0 \wedge \speak{$p'(x)$ \inv}.
\end{multline*}
We call a polynomial~$p$ over a ring~$A$ \emph{unramifiable} if and
only if it admits at least one simple root in every algebraically closed field
over~$A$. Since quantifying over algebraically closed fields raises red flags
from an intuitionistic point of view, just as quantifying over maximal ideals
does, this condition has to be formulated in a sensible way. One possibility is
to use the \emph{hyperdiscriminants} of~$p$, \ie the elementary symmetric
polynomials in the values of~$p'$ at the roots of~$p$, resulting in a simple
existential statement involving only the coefficients of~$p$; in particular,
the condition for a polynomial to be unramifiable is a geometric formula.
See~\cite[p.~751]{wraith:generic-galois-theory} for the precise formulation.

In more detail, the claim is that firstly~$\Box_\text{ét}$ is a modal operator
such that the displayed formula holds and that secondly, if~$\Box$ is any modal
operator verifying the formula, internally it holds that~$\Box_\text{ét}\varphi
\Rightarrow \Box\varphi$ for any truth value~$\varphi\?\Omega$.

\begin{proof}For the proof we require some familiarity with the concept of
classifying toposes. We are grateful to Felix Geißler for contributing a key step of
the argument.

To verify the first statement, note that the displayed formula is a geometric
implication and that the big étale topos~$\Et(S)$ has \emph{enough points}.
Therefore it suffices to show that for any~$S$-scheme~$T$ and any geometric
point~$\bar t$ of~$T$, the stalk~$\O_{T,\bar t}$ is separably closed. It is
well-known that this is true.

For the second statement we may assume without loss of generality that~$S =
\Spec A$ is affine. It is well-known that, for any cocomplete topos~$\E$,
geometric morphisms~$\E \to \Zar(\Spec A)$ are in canonical one-to-one correspondence
with local algebras over~$\ul{A}$ in~$\E$ (where~$\ul{A}$ denotes the pullback
of~$A$ along the unique geometric morphism~$\E \to \Set$) and that geometric
morphisms~$\E \to \Et(\Spec A)$ are in canonical one-to-one correspondence
with algebras over~$\ul{A}$ which are local and separably closed from the
internal point of view of~$\E$;
see~\cite[Section~VIII.6]{moerdijk-maclane:sheaves-logic}
and~\cite{anel:factorization-systems}.

Therefore a geometric morphism~$\E \to \Zar(\Spec A)$ factors over the
geometric embedding~$\Et(\Spec A) \hookrightarrow \Zar(\Spec A)$ if and only if
the pullback of~$\affla$ along~$\E \to \Zar(\Spec A)$ is separably closed.

Let~$\Box$ be a modal operator in~$\Zar(\Spec A)$ such that the~$\Box$-translation
of~``$\affla$ is separably closed'' holds. Then the pullback of~$\affla$
along~$\Zar(\Spec A)_\Box \hookrightarrow \Zar(\Spec A)$ is separably closed
and therefore this geometric embedding factors over~$\Et(\Spec A)
\hookrightarrow \Zar(\Spec A)$. This shows that any~$\Box$-sheaf is also
a~$\Box_\text{ét}$-sheaf.

The claim that~$\Box_\text{ét}\varphi \Rightarrow \Box\varphi$ for any truth
value~$\varphi\?\Omega$ then follows by combining the following two basic
observations of the theory of modal operators, valid for any modal
operator~$\Box$:
\begin{enumerate}
\item $\Box\varphi \Longleftrightarrow
  \forall \psi\?\Omega\_ ((\Box\psi\Rightarrow\psi) \wedge
  (\varphi\Rightarrow\psi)) \Rightarrow \psi$.
\item $(\Box\psi \Rightarrow \psi) \Longleftrightarrow
  \speak{$\{x \in 1 \,|\, \psi\}$ is a~$\Box$-sheaf}$. \qedhere
\end{enumerate}
\end{proof}

\XXX{other topologies?}

% XXX: A note on tangent bundles in a category with a ring object.
% (http://www.mscand.dk/article/view/11736)

\begin{itemize}
\item synthetic topology?
\end{itemize}


\section{Unsorted}
\begin{itemize}
\item ``functoriality''
\item Kähler differentials
\item closed and open subschemes
\item $j_! \O_U$ flat over~$\O_X$, \ldots
\item Koszul resolution; Beilinson resolution?
\item meta properties: some lemmas about limits of modules
\item locally small categories
\item open/closed immersions
\item morphisms of schemes...
\item proper maps...
\item limits and colimits...
\item Kähler differentials; clear myth that the definition via free modules
``does not glue very well''
(\url{http://www.mathematik.uni-kl.de/~gathmann/class/alggeom-2002/chapter-7.pdf})
\end{itemize}


\chapter*{Appendix}

\newcounter{saved-section-number}
\setcounter{saved-section-number}{\value{section}}

\begin{appendix}

\setcounter{section}{\value{saved-section-number}}

\section{Dictionary between internal and external notions}

{\small\renewcommand{\arraystretch}{1.3}
\begin{longtable}{@{}p{4.4cm}@{\qquad}p{6.7cm}@{\qquad}p{1.5cm}@{}}
  \toprule
  External & Internal & Reference \\ \midrule
  \textbf{Sheaves of sets} \\
  sheaf of sets & set \\
  $\alpha : \F \to \G$ monomorphism & $\alpha$ injective & Ex.\@~\ref{ex:injective-surjective} \\
  $\alpha : \F \to \G$ epimorphism & $\alpha$ surjective & Ex.\@~\ref{ex:injective-surjective} \\
  $\Int(X \setminus \supp\F)$ & truth value of ``$\F$ is a singleton'' & Rem.\@~\ref{rem:support-sheaf-of-sets} \\
  $f : X \to \NN$ upper semicont.\@ & element of~$\widehat\NN$ & Lemma~\ref{lemma:upper-semicontinuous-functions} \\
  $f : X \to \NN$ locally constant & element of~$\NN$ & Lemma~\ref{lemma:upper-semicontinuous-functions} \\\\

  \textbf{Sheaves of rings} \\
  sheaf of rings & ring & Prop.\@~\ref{prop:rings-internally} \\
  local sheaf of rings & local ring & Prop.\@~\ref{prop:local-ring} \\
  $X$ is reduced & $\O_X$ is reduced (and $\neg\text{invertible} \Rightarrow \text{zero}$) & Prop.\@~\ref{prop:reduced-ring} \\
  $\dim X \leq n$ & Krull dimension of~$\O_X$ is~$\leq n$ & Prop.\@~\ref{prop:dimension-scheme-ox} \\
  $X$ is integral at all points & $\O_X$ is a integral domain & Prop.\@~\ref{prop:internal-integrality} \\
  $X$ is locally Noetherian & $\O_X$ is processly Noetherian & Prop.\@~\ref{prop:internal-noetherianity} \\
  $X$ is normal & $\O_X$ is normal (assuming that~$X$ is locally Noetherian) & Prop.\@~\ref{prop:normal-int-ext} \\\\

  \textbf{Sheaves of modules} \\
  sheaf of modules & module \\
  $\F$ is finite locally free & $\F$ is finite free & Prop.\@~\ref{prop:locally-free} \\
  $\F$ is of finite type & $\F$ is finitely generated & Prop.\@~\ref{prop:finite-type-and-co} \\
  $\F$ is of finite presentation & $\F$ is finitely presented & Prop.\@~\ref{prop:finite-type-and-co} \\
  $\F$ is coherent & $\F$ is coherent & Prop.\@~\ref{prop:finite-type-and-co} \\
  $\F$ is quasicoherent & $\F[f^{-1}]$ is a sheaf wrt.\@~$(\speak{$f$ \inv} \Rightarrow \placeholder)$ for~$f\?\O_X$ & Thm.\@~\ref{thm:qcoh-sheafchar} \\
  $\F$ is flat & $\F$ is flat & Prop.\@~\ref{prop:flatness} \\
  $\F$ is torsion & $\F$ is torsion & Prop.\@~\ref{prop:torsion-int-ext} \\
  $M^\sim$ & $\ul{M}[\F^{-1}]$ (localization at generic filter) & Prop.\@~\ref{prop:tilde-construction-internally} \\
  tensor product $\F \otimes \G$ & tensor product $\F \otimes \G$ & Prop.\@~\ref{prop:internal-tensor-product} \\
  dual~$\F^\vee = \HOM_{\O_X}(\F,\O_X)$ & dual $\F^\vee = \Hom_{\O_X}(\F,\O_X)$ \\
  $\Int(X \setminus \supp\F)$ & truth value of~``$\F = 0$'' & Prop.\@~\ref{prop:characterization-support} \\
  quasicoherator of~$\I$ & $\{ s\?\O_X \,|\, \speak{$s$ \inv}
  \Rightarrow s \in \I \}$ ($\I$ a radical ideal) & Prop.\@~\ref{prop:quasicoherator-structure-sheaf} \\
  rank function of~$\F$ & minimal number of generators for~$\F$ & Prop.\@~\ref{prop:rank-function-internally} \\\\

  \multicolumn{3}{@{}l@{}}{\textbf{Subspaces} ($i : A \hookrightarrow X$ closed immersion, $j : U \hookrightarrow X$ open immersion)} \\
  sheaf supported on~$A$ & $\Box$-sheaf, where~$\Box = (\placeholder \vee A^c)$ & Lemma~\ref{lemma:essim-closed-immersion} \\
  sheaf of the form~$j_*(\F)$ & $\Box$-sheaf, where~$\Box = (U \Rightarrow
  \placeholder)$ & \\
  extension of~$\F$ by the empty set & $j_!(\F) = \{ x\?\F \,|\, U \}$ & Lemma~\ref{lemma:extension-by-empty-set} \\
  extension of~$\F$ by zero & $j_!(\F) = \{ x\?\F \,|\, (x = 0) \vee U \}$ & Lemma~\ref{lemma:extension-by-zero} \\
  sheaf with empty/zero stalks on~$U^c$ & sheaf of the form~$j_!(\F)$ \\
  sections of~$\F$ are equal if they agree on dense open & $\F$ is $\neg\neg$-separated & Prop.\@~\ref{prop:negneg-sheaves} \\
  sheaf of sections of~$\F$ defined on dense open subsets & $\F^{++}$ with respect to~$\Box = \neg\neg$ (assuming that~$\F$ is~$\neg\neg$-separated) & Prop.\@~\ref{prop:negneg-sheaves} \\
  $U$ is dense & $\neg\neg U$ & Prop.\@~\ref{prop:modops-kripke} \\
  $U$ is scheme-theoretically dense & $\sdense U$, \ie $\O_X$ is separated wrt.~$(U \Rightarrow \placeholder)$ & Lemma\@~\ref{lemma:scheme-theoretical-density} \\
  $V(\I)$ is reduced & $\I$ is a radical ideal & Lemma~\ref{lemma:closed-subspace-reduced} \\
  $\O_{X_\mathrm{red}}$ & $\O_X/\sqrt{(0)}$ & Lemma~\ref{lemma:reduced-subspace} \\\\

  \multicolumn{3}{@{}l@{}}{\textbf{Rational functions and Cartier divisors}} \\
  $\K_X$ & total quotient ring of~$\O_X$ & Prop.\@~\ref{prop:kx-internally} \\
  Cartier divisor & element of~$\K_X^\times/\O_X^\times$ \\
  effective Cartier divisor & $[s/1]$ with~$s\?\O_X$ regular & Def.\@~\ref{defn:effective-cartier-divisor} \\
  line bundle~$\O_X(D)$ & $D^{-1} \O_X \subseteq \K_X$ & Def.\@~\ref{defn:line-bundle-of-divisor} \\\\

  \multicolumn{3}{@{}l@{}}{\textbf{Topological properties}} \\
  $X$ is quasicompact & ``$\Sh(X) \models$'' commutes with directed disjunctions & Prop.\@~\ref{prop:quasicompact-meta} \\
  $X$ is local & ``$\Sh(X) \models$'' commutes with arbitrary disjunctions & Prop.\@~\ref{prop:local-meta} \\
  $X$ is irreducible & if $\neg(\varphi \wedge \psi)$, then $\neg\varphi$ or~$\neg\psi$ & Prop.\@~\ref{prop:irreducibility-internally} \\
  \bottomrule
\end{longtable}}
% missing: relative spectrum and big Zariski topos


\section{The inference rules of intuitionistic logic}
\label{appendix:inference-rules}

\XXX{cite \cite[Section~D1.3.1]{johnstone:elephant}, talk about~$\in$, and
explain contexts}

\begin{center}
  \textbf{Structural rules} \\
  \vspace{-0.5em}
  \phantom{a}\hfill
  \AxiomC{$\phantom{\seq{\vec x}}$}\UnaryInfC{$\varphi \seq{\vec x} \varphi$}\DisplayProof\hfill
  \AxiomC{$\varphi \seq{\vec x} \psi$}\UnaryInfC{$\varphi[\vec s/\vec x]
  \seq{\vec y} \psi[\vec s/\vec x]$}\DisplayProof\hfill
  \AxiomC{$\varphi \seq{\vec x} \psi$}\AxiomC{$\psi \seq{\vec x}
  \chi$}\BinaryInfC{$\varphi \seq{\vec x} \chi$}\DisplayProof
  \phantom{a}\hfill
  \vspace{2.0em}

  \textbf{Rules for nullary and binary conjunction} \\
  \vspace{-0.5em}
  \phantom{a}\hfill
  \AxiomC{$\phantom{\seq{\vec x}}$}\UnaryInfC{$\varphi \seq{\vec x} \top$}\DisplayProof\hfill
  \AxiomC{$\phantom{\seq{\vec x}}$}\UnaryInfC{$\varphi \wedge \psi \seq{\vec x} \varphi$}\DisplayProof\hfill
  \AxiomC{$\phantom{\seq{\vec x}}$}\UnaryInfC{$\varphi \wedge \psi \seq{\vec x} \psi$}\DisplayProof\hfill
  \AxiomC{$\varphi \seq{\vec x} \psi$}\AxiomC{$\varphi \seq{\vec x} \chi$}\BinaryInfC{$\varphi \seq{\vec x} \psi \wedge \chi$}\DisplayProof
  \phantom{a}\hfill
  \vspace{2em}

  \textbf{Rules for nullary and binary disjunction} \\
  \vspace{-0.5em}
  \phantom{a}\hfill
  \AxiomC{$\phantom{\seq{\vec x}}$}\UnaryInfC{$\bot \seq{\vec x} \varphi$}\DisplayProof\hfill
  \AxiomC{$\phantom{\seq{\vec x}}$}\UnaryInfC{$\varphi \seq{\vec x} \varphi \vee \psi$}\DisplayProof\hfill
  \AxiomC{$\phantom{\seq{\vec x}}$}\UnaryInfC{$\psi \seq{\vec x} \varphi \vee \psi$}\DisplayProof\hfill
  \AxiomC{$\varphi \seq{\vec x} \chi$}\AxiomC{$\psi \seq{\vec x} \chi$}\BinaryInfC{$\varphi \vee \psi \seq{\vec x} \chi$}\DisplayProof
  \phantom{a}\hfill
  \vspace{2em}

  \textbf{Rules for arbitrary set-indexed conjunction and disjunction} \\
  \vspace{-0.5em}
  \phantom{a}\hfill
  \AxiomC{$\phantom{\seq{\vec x}}$}\UnaryInfC{$\bigwedge_{i \in I} \varphi_i \seq{\vec x} \varphi_j$ for all~$j \in I$}\DisplayProof\hfill
  \AxiomC{$\varphi \seq{\vec x} \psi_j$ for all~$j \in I$}\UnaryInfC{$\varphi \seq{\vec x} \bigwedge_{i \in I} \psi_i$}\DisplayProof
  \phantom{a}\hfill
  \vspace{1em}

  \phantom{a}\hfill
  \AxiomC{$\phantom{\seq{\vec x}}$}\UnaryInfC{$\varphi_j \seq{\vec x} \bigvee_{i \in I} \varphi_i$ for all~$j \in I$}\DisplayProof\hfill
  \AxiomC{$\varphi_j \seq{\vec x} \psi$ for all~$j \in I$}\UnaryInfC{$\bigvee_{i \in I} \varphi_i \seq{\vec x} \psi$}\DisplayProof
  \phantom{a}\hfill
  \vspace{2em}

  \textbf{Double rule for implication} \\
  \vspace{-0.5em}
  \phantom{a}\hfill
  \Axiom$\varphi \wedge \psi\ \fCenter\seq{\vec x} \chi$
  \doubleLine
  \UnaryInf$\varphi\ \fCenter\seq{\vec x} \psi \Rightarrow \chi$
  \DisplayProof
  \phantom{a}\hfill
  \vspace{2em}

  \textbf{Double rules for bounded and unbounded quantification} \\
  \vspace{-0.5em}
  \phantom{a}\hfill
  \Axiom$\varphi\ \fCenter\seq{\vec x, y} \psi$
  \doubleLine
  \UnaryInf$\exists y\?Y\_\! \varphi\ \fCenter\seq{\vec x} \psi$
  \DisplayProof
  {\tiny ($y$ not occuring in~$\psi$)}
  \hfill
  \Axiom$\varphi\ \fCenter\seq{\vec x, y} \psi$
  \doubleLine
  \UnaryInf$\varphi\ \fCenter\seq{\vec x\phantom{, y}} \forall y\?Y\_\! \psi$
  \DisplayProof
  {\tiny ($y$ not occuring in~$\varphi$)}
  \hfill\phantom{a}
  \vspace{1em}

  \phantom{a}\hfill
  \Axiom$\varphi\ \fCenter\seq{\vec x, Y} \psi$
  \doubleLine
  \UnaryInf$\exists Y\_\! \varphi\ \fCenter\seq{\vec x} \psi$
  \DisplayProof
  {\tiny ($Y$ not occuring in~$\psi$)}
  \hfill
  \Axiom$\varphi\ \fCenter\seq{\vec x, Y} \psi$
  \doubleLine
  \UnaryInf$\varphi\ \fCenter\seq{\vec x\phantom{, Y}} \forall Y\_\! \psi$
  \DisplayProof
  {\tiny ($Y$ not occuring in~$\varphi$)}
  \hfill\phantom{a}
  \vspace{2em}

  \textbf{Rules for equality} \\
  \vspace{-0.5em}
  \phantom{a}\hfill
  \AxiomC{$\phantom{\seq{\vec x}}$}
  \UnaryInfC{$\top \seq{x} x = x$}
  \DisplayProof
  \hfill
  \AxiomC{$\phantom{\seq{\vec x}}$}
  \UnaryInfC{$(\vec x = \vec y) \wedge \varphi \seq{\vec z} \varphi[\vec y/\vec x]$}
  \DisplayProof
  \hfill\phantom{a} \\[0.5em]
  (``$\vec x = \vec y\,$'' is short for~``$x_1 = y_1 \wedge \cdots \wedge x_n =
  y_n$''.)
\end{center}

\end{appendix}

\nocite{*}
\printbibliography

\XXX{remark on possible pitfalls}

\end{document}

\YYY{typography of $\bigcup$}

Snippet that may be useful later:
By appealing to the axiom of unique choice (see~YYY), we can
define a morphism of sheaves~$\O_X(D) \otimes \O_X(D') \to \O_X(D + D')$ by
internally describing a suitable map using representatives~$D = [f]$,~$D' =
[f']$, as long as the resulting map does not depend on the choice of representatives.

For the third statement, note that it is equivalent to show that
\[ \Gamma(U,\F^+) = \{ (V,s) \,|\,
  \text{$V \subseteq U$ dense open},\
  s \in \Gamma(V,\F),\
  \text{$(V,s)$ maximal} \}, \]
where~``$(V,s)$ maximal'' means that for any other such pair~$(W,t)$ such
that~$V \subseteq W$ and~$t|_V = s$, it holds that~$V = W$. This follows from
the fact that the plus construction can also be defined as
\[ \F^+ \defeq \{ S \subseteq F \,|\,
  \speak{$S$ subsingleton},\
  \neg\neg(\speak{$S$ inhabited}),\
  \speak{$S$ $\neg\neg$-stable} \}. \]

What is a conceptual explanation for O_X(D(f)) = A[f^(-1)] and O_X(U) !=
A[S_U^(-1)] otherwise?

Constructible topology?

Big Zariski topos:
* A^1 local ring, field, but of unbounded Krull dimension

Reassure that constructing morphisms between spectra is *easy* in the localic
setting.

Sch/S or Aff/S

Mention applications to generalized scheme theories, as surveyed by
http://arxiv.org/abs/0909.0069.

Ponder http://hlombardi.free.fr/publis/schemas.pdf.

Normalization from an internal point of view?

Discuss relation to https://golem.ph.utexas.edu/category/2009/06/algebraic_geometry_for_categor.html

Reference http://www.mta.ca/~cat-dist/catlist/1999/finite-topos:
What concept of finiteness is appropriate for those important
mathematical applications in topology for which K/S doesn't
seem right ? (For example the equalizer closure of K/S or...??)
Especially, a suitably "finite" module should be a vector bundle
or a FAC in the sense of Serre so that our simplified topos theory
could apply more directly to those things it should.

Chapter on stacks: A stack on a topological space X should roughly by a
category internal (or locally-internal) to Sh(X) such that, if internally there
exists an object unique up to unique isomorphism with some property, then
externally such an object exists. Note that talking about prestacks should not
be possible from the point of view of Sh(X).

Send Marcelo F. a copy of these notes once they are finished.

Cite: Compactification of Siegel Moduli Schemes by Ching-Li Chai (contains
appendix on Hakim's relative scheme theory specialized to ringed spaces)
Also mention page 151, Remark A.1.2?

In the discussion of quasicoherence in the little Zariski topos, cite
https://projecteuclid.org/download/pdf_1/euclid.rmjm/1250128841 (page 643) to
demonstrate that the quasicoherence condition was already known (but not
recognized as an internal statement).

Cite:
https://case.edu/artsci/phil/In%20press%20There%20is%20no%20ontology%20here.pdf
for background on the functor-of-points approach
also cite, of course, the discussion on the Secret Blogging Seminar

Cite:
http://reh.math.uni-duesseldorf.de/~schroeer/publications_pdf/points_fppf.pdf
for background on sites, and also more specifically

XXX: s/lattice/frame/
XXX: Define morphisms of locales
XXX: axioms on? axioms of?

Cite Zhen Lin's thesis (http://zll22.user.srcf.net/dpmms/thesis.pdf).

No "et al" in the bibliography!

Cite Johnstone's "rings, fields, and spectra".

Mention Steven Gubkin and the nCafé post. Also the Secret Blogging Seminar.

Cite:
http://www.oliviacaramello.com/Unification/ToposTheoreticPreliminariesOliviaCaramello.pdf

Definitely cite:
Shawn J. Henry.
Classifying Topoi and Preservation of Higher Order Logic by Geometric Morphisms.
https://arxiv.org/abs/1305.3254
