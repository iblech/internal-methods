\documentclass{amsart}
\usepackage[utf8]{inputenc}
\usepackage[english]{babel}
\usepackage{amsmath,amsthm,amssymb,stmaryrd,color,graphicx}
\usepackage{setspace}
\usepackage{bussproofs}
\usepackage{array}
\usepackage{comment}
\usepackage[protrusion=true,expansion=true]{microtype}
\usepackage{hyperref}

\title{Using the internal language of toposes in algebraic geometry}
\author{Ingo Blechschmidt}
\email{iblech@web.de}

\begin{document}
\maketitle

\begin{abstract}
  There are several important topoi associated to a scheme, for instance the
  petit and gros Zariski topoi. These come with an internal mathematical language
  which closely resembles the usual formal language of mathematics, but is ``local
  on the base scheme'':

  For example, from the internal perspective, the structure sheaf looks like an
  ordinary local ring (instead of a sheaf of rings with local stalks) and vector
  bundles look like ordinary free modules (instead of sheaves of modules
  satisfying a certain condition). The translation of internal statements and
  proofs is facilitated by an easy mechanical procedure.

  These notes give an introduction to this topic and show how the internal
  point of view can be exploited to give simpler definitions and more conceptual
  proofs of the basic notions and observations in algebraic geometry.
\end{abstract}

\tableofcontents

\section{Introduction}

\section{Kripke--Joyal semantics}
\begin{itemize}
\item definition
\item fundamental properties
\item geometric formulas
\item geometric constructions
\end{itemize}

\section{Sheaves of rings}
\begin{itemize}
\item reducedness
\item field property
\item discreteness
\end{itemize}

\section{Sheaves of modules}
\begin{itemize}
\item of finite type, of finite presentation, coherent
\item basic lemmas
\item flatness
\item important hard exercise
\end{itemize}

\section{Rational functions and Cartier divisors}
\begin{itemize}
\item internal definition of $K_X$
\item internal definition of Cartier divisors
\item correspondence between Cartier divisors and sub-$O_X$-modules of $K_X$
\end{itemize}

\section{Relative spectrum}
\begin{itemize}
\item ...
\end{itemize}

\section{Modalities}
\begin{itemize}
\item negneg
\item spreading of properties from stalk to neighbourhood
\item internal sheafification
\end{itemize}

\section{Unsorted}
\begin{itemize}
\item Kähler differentials
\item completion of the natural numbers, rank function
\item meta properties
\item locally small categories
\item big Zariski topos
\item open/closed immersions
\item morphisms of schemes...
\item proper maps...
\end{itemize}

\end{document}
