\documentclass{amsart}
\usepackage[utf8]{inputenc}
\usepackage[english]{babel}
\usepackage{amsmath,amsthm,amssymb,stmaryrd,color,graphicx}
\usepackage{setspace}
\usepackage{bussproofs}
\usepackage{array}
\usepackage{comment}
\usepackage[protrusion=true,expansion=true]{microtype}
\usepackage{hyperref}

\usepackage{tikz}
\usetikzlibrary{calc,shapes.callouts,shapes.arrows}
\newcommand{\hcancel}[5]{%
    \tikz[baseline=(tocancel.base)]{
        \node[inner sep=0pt,outer sep=0pt] (tocancel) {#1};
        \draw[red, line width=0.5mm] ($(tocancel.south west)+(#2,#3)$) -- ($(tocancel.north east)+(#4,#5)$);
    }%
}

\theoremstyle{definition}
\newtheorem{defn}{Definition}[section]
\newtheorem{ex}[defn]{Example}

\theoremstyle{plain}

\newtheorem{prop}[defn]{Proposition}
\newtheorem{lemma}[defn]{Lemma}
\newtheorem{thm}[defn]{Theorem}

\theoremstyle{remark}
\newtheorem{rem}[defn]{Remark}

\newcommand{\ZZ}{\mathbb{Z}}
\renewcommand{\AA}{\mathbb{A}}
\newcommand{\C}{\mathcal{C}}
\newcommand{\D}{\mathcal{D}}
\newcommand{\E}{\mathcal{E}}
\newcommand{\F}{\mathcal{F}}
\newcommand{\G}{\mathcal{G}}
\renewcommand{\O}{\mathcal{O}}
\renewcommand{\P}{\mathcal{P}}
\newcommand{\NN}{\mathbb{N}}
\newcommand{\RR}{\mathbb{R}}
\newcommand{\GG}{\mathbb{G}}
\newcommand{\Hom}{\mathrm{Hom}}
\newcommand{\id}{\mathrm{id}}
\newcommand{\Aut}[1]{\operatorname{Aut}(#1)}
\newcommand{\GL}{\mathrm{GL}}
\newcommand{\freist}{\_{}\_{}}
\newcommand{\Set}{\mathrm{Set}}
\newcommand{\Grp}{\mathrm{Grp}}
\newcommand{\Vect}{\mathrm{Vect}}
\newcommand{\Sh}{\mathrm{Sh}}
\newcommand{\Zar}{\mathrm{Zar}}
\newcommand{\Sch}{\mathrm{Sch}}
\DeclareMathOperator{\Spec}{Spec}
\DeclareMathOperator{\colim}{colim}
\newcommand{\?}{\,{:}\,}
\renewcommand{\_}{\mathpunct{.}\,}
\newcommand{\speak}[1]{\ulcorner\text{#1}\urcorner}
\newcommand{\Ll}{:\Longleftrightarrow}

\title{Using the internal language of toposes in algebraic geometry}
\author{Ingo Blechschmidt}
\email{iblech@web.de}

\begin{document}
\maketitle

\begin{abstract}
  There are several important topoi associated to a scheme, for instance the
  petit and gros Zariski topoi. These come with an internal mathematical language
  which closely resembles the usual formal language of mathematics, but is ``local
  on the base scheme'':

  For example, from the internal perspective, the structure sheaf looks like an
  ordinary local ring (instead of a sheaf of rings with local stalks) and vector
  bundles look like ordinary free modules (instead of sheaves of modules
  satisfying a certain condition). The translation of internal statements and
  proofs is facilitated by an easy mechanical procedure.

  These expository notes give an introduction to this topic and show how the internal
  point of view can be exploited to give simpler definitions and more conceptual
  proofs of the basic notions and observations in algebraic geometry. No prior
  knowledge about topos theory and formal logic is assumed.
\end{abstract}

\tableofcontents

\section{Introduction}

\section{Kripke--Joyal semantics}

Let~$X$ be a topological space. Later, $X$ will be the underlying space of a
scheme.

\begin{defn}[Kripke--Joyal semantics of a sheaf topos]The meaning of 
\[ U \models \varphi \quad\text{(``$\varphi$ holds on $U$'')} \]
for open subsets~$U \subseteq X$ and formulas~$\varphi$ is given by
the following rules, recursively in the structure of~$\varphi$:
\[ \renewcommand{\arraystretch}{1.3}\begin{array}{@{}lcl@{}}
  U \models f = g \? \F &\Ll& f|_U = g|_U \in \Gamma(U, \F) \\
  U \models \varphi \wedge \psi &\Ll&
    \text{$U \models \varphi$ and $U \models \psi$} \\
  U \models \varphi \vee \psi &\Ll&
    \hcancel{\text{$U \models \varphi$ or $U \models \psi$}}{0pt}{3pt}{0pt}{-2pt} \\
  && \text{there exists a covering $U = \bigcup_i U_i$ s.\,th. for all~$i$:} \\
  && \quad\quad \text{$U_i \models \varphi$ or $U_i \models \psi$} \\
  U \models \varphi \Rightarrow \psi &\Ll&
    \text{for all open~$V \subseteq U$: } 
  \text{$V \models \varphi$ implies $V \models \psi$} \\
  U \models \forall f \? \F\_ \varphi(f) &\Ll&
    \text{for all sections~$f \in \Gamma(V, \F), V \subseteq U$: $V \models
    \varphi(f)$} \\
  U \models \exists f \? \F\_ \varphi(f) &\Ll&
    \hcancel{\text{there exists a section~$f \in \Gamma(U,\F)$ s.\,th. $U
    \models \varphi(f)$}}{0pt}{3pt}{0pt}{-2pt} \\
  &&
    \text{there exists a covering $U = \bigcup_i U_i$ s.\,th. for all~$i$:} \\
  && \quad\quad \text{there exists~$f_i \in \Gamma(U_i, \F)$ s.\,th.
  $U_i \models \varphi(f_i)$} \\
  U \models \forall \F\_ \varphi(\F) &\Ll&
    \text{for all sheaves $\F$ on $V$, $V \subseteq U$: $V \models \varphi(\F)$} \\
  U \models \exists \F\_ \varphi(\F) &\Ll&
    \text{there exists a covering $U = \bigcup_i U_i$ s.\,th. for all~$i$:} \\
  && \quad\quad \text{there exists a sheaf~$\F_i$ on~$U_i$ s.\,th.
  $U_i \models \varphi(\F_i)$}
\end{array} \]
\end{defn}

\begin{rem}The last two rules, concerning \emph{unbounded quantification}, are
not part of the classical Kripke--Joyal semantics, but instead of Mike
Shulman's stack semantics~\cite{shulman:stack}, a slight extension. They are
needed so that we can use formulate universal properties in the internal
language.
\end{rem}

The rules are not all arbitrary. They are finely concerted to make the
following propositions true, which are crucial for a proper appreciation of the
internal language.

\begin{prop}[Locality of the internal language]
Let~$U = \bigcup_i U_i$ be covered by open subsets. Let~$\varphi$
be a formula. Then
\[ U \models \varphi \qquad\text{iff}\qquad
  \text{$U_i \models \varphi$ for each $i$}. \]
\end{prop}
\begin{proof}Induction on the structure of~$\varphi$. Note that the canceled
rules would make this proposition false.\end{proof}

\begin{prop}[Soundness of the internal language]
If a formula~$\varphi$ implies a further formula~$\psi$ in intuitionistic logic, then
\[ U \models \varphi \qquad\text{implies}\qquad
  U \models \psi. \]
\end{prop}
\begin{proof}
Proof by induction on the structure of formal intuitionistic proofs; we are to
show that any inference rule of intuitionistic logic is satisfied by the
Kripke--Joyal semantics. For instance, there is the following rule governing
disjunction:
\begin{quote}
If~$\varphi \vee \psi$ holds, and both $\varphi$ and $\psi$ imply a further
formula~$\chi$, then~$\chi$ holds.
\end{quote}
So we are to prove that if~$U \models \varphi \vee \psi$, $U \models (\varphi
\Rightarrow \chi)$, and $U \models (\psi \Rightarrow \chi)$, then $U \models \chi$.
This is done as follows: By assumption, there exists a covering~$U = \bigcup_i
U_i$ such that on each~$U_i$, $U_i \models \varphi$ or $U_i \models \psi$.
Again by assumption, we may conclude that~$U_i \models \chi$ for each~$i$. The statement
follows because of the locality of the internal language.

A complete list of which rules are to prove is
in~\cite[D1.3.1]{johnstone:elephant}.
\end{proof}

\begin{itemize}
\item geometric formulas
\item geometric constructions
\end{itemize}

\section{Sheaves of rings}
\begin{itemize}
\item reducedness
\item field property
\item discreteness
\end{itemize}

\section{Sheaves of modules}
\begin{itemize}
\item of finite type, of finite presentation, coherent
\item basic lemmas
\item flatness
\item important hard exercise
\end{itemize}

\section{Rational functions and Cartier divisors}
\begin{itemize}
\item internal definition of $K_X$
\item internal definition of Cartier divisors
\item correspondence between Cartier divisors and sub-$O_X$-modules of $K_X$
\end{itemize}

\section{Relative spectrum}
\begin{itemize}
\item ...
\end{itemize}

\section{Modalities}
\begin{itemize}
\item negneg
\item spreading of properties from stalk to neighbourhood
\item internal sheafification
\end{itemize}

\section{Unsorted}
\begin{itemize}
\item Kähler differentials
\item completion of the natural numbers, rank function
\item meta properties
\item locally small categories
\item big Zariski topos
\item open/closed immersions
\item morphisms of schemes...
\item proper maps...
\end{itemize}

\end{document}
