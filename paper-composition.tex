\documentclass[oneside,reqno]{amsart}

\usepackage[utf8]{inputenc}
\usepackage[english]{babel}
\usepackage{amsthm,mathtools,stmaryrd,amssymb,graphicx}
\usepackage{booktabs}
\usepackage[all]{xy}
\usepackage[protrusion=true,expansion=true]{microtype}
\usepackage{xspace}

\usepackage[natbib=true,style=numeric,maxnames=10]{biblatex}
\usepackage[babel]{csquotes}
\bibliography{paper-composition.bib}

\title{Composition of internal theories}

\author{Ingo Blechschmidt}
\email{ingo.blechschmidt@univr.it}

\author{Matthias Hutzler}
\email{matthias.ralph.hutzler@student.uni-augsburg.de}

\author{Alexander Oldenziel}
\email{a.f.d.a.gietelinkoldenziel@gmail.com}

\theoremstyle{definition}
\newtheorem{defn}{Definition}[section]
\newtheorem{ex}[defn]{Example}

\theoremstyle{plain}
\newtheorem{prop}[defn]{Proposition}
\newtheorem{cor}[defn]{Corollary}
\newtheorem{lemma}[defn]{Lemma}
\newtheorem{thm}[defn]{Theorem}
\newtheorem{scholium}[defn]{Scholium}

\theoremstyle{remark}
\newtheorem{rem}[defn]{Remark}
\newtheorem{question}[defn]{Question}
\newtheorem{speculation}[defn]{Speculation}
\newtheorem{caveat}[defn]{Caveat}
\newtheorem{conjecture}[defn]{Conjecture}

\newenvironment{indentblock}{%
  \list{}{\leftmargin\leftmargin}%
  \item\relax
}{%
  \endlist
}

\newcommand{\xra}[1]{\xrightarrow{#1}}
\newcommand{\aaa}{\mathfrak{a}}
\newcommand{\bbb}{\mathfrak{b}}
\newcommand{\mmm}{\mathfrak{m}}
\newcommand{\I}{\mathcal{I}}
\newcommand{\J}{\mathcal{J}}
\newcommand{\E}{\mathcal{E}}
\newcommand{\F}{\mathcal{F}}
\newcommand{\B}{\mathcal{B}}
\newcommand{\C}{\mathcal{C}}
\renewcommand{\AA}{\mathbb{A}}
\newcommand{\EE}{\mathbb{E}}
\newcommand{\NN}{\mathbb{N}}
\newcommand{\RR}{\mathbb{R}}
\newcommand{\TT}{\mathbb{T}}
\newcommand{\ZZ}{\mathbb{Z}}
\renewcommand{\SS}{\mathbb{S}}
\renewcommand{\P}{\mathcal{P}}
\renewcommand{\O}{\mathcal{O}}
\newcommand{\defeq}{\vcentcolon=}
\newcommand{\op}{\mathrm{op}}
\DeclareMathOperator{\Spec}{Spec}
\DeclareMathOperator{\Hom}{Hom}
\DeclareMathOperator{\Sh}{Sh}
\DeclareMathOperator{\PSh}{PSh}
\DeclareMathOperator{\rank}{rank}
\DeclareMathOperator{\length}{length}
\DeclareMathOperator{\List}{List}
\newcommand{\Set}{\mathrm{Set}}
\newcommand{\Eff}{\mathrm{Ef{}f}}
\renewcommand{\_}{\mathpunct{.}\,}
\newcommand{\effective}{ef{}fective\xspace}
\newcommand{\?}{\,{:}\,}
\newcommand{\realizes}{\Vdash}
\newcommand{\notnot}{\emph{not~not}\xspace}
\usepackage{soul}\setul{0.3ex}{}
\let\oldul\ul
\renewcommand{\ul}[1]{\text{\oldul{$#1$}}}
\newcommand{\affl}{\ensuremath{{\ul{\AA}^1}}\xspace}
\newcommand{\speak}[1]{\ulcorner\text{\textnormal{#1}}\urcorner}
\newcommand{\brak}[1]{\llbracket #1 \rrbracket}
\newcommand{\Mod}[1]{{#1}\mathrm{\text{-}mod}}

\newcommand{\stacksproject}[1]{\cite[{\href{https://stacks.math.columbia.edu/tag/#1}{Tag~#1}}]{stacks-project}}

\renewcommand{\paragraph}[1]{\noindent\textbf{#1.}}

\newcommand{\ZFC}{\textsc{zfc}}

\newcommand{\seq}[1]{\mathrel{\vdash\!\!\!_{#1}}}

\begin{document}

\begin{abstract}
  XXX
\end{abstract}

\maketitle
\thispagestyle{empty}

\section{Introduction}

XXX state main lemma (to be proved in Section 3)

\begin{defn}A geometric morphism~$\F \to \E$ is \emph{of presheaf type} if and
only if it is equivalent (XXX be more precise) to the canonical
functor~$\E^{\C^\op} \to \E$ for an internal category~$\C$ in~$\E$.\end{defn}

XXX This notion is the relativization of the notion of a presheaf category; a
geometric theory~$\TT$ is of presheaf type if and only if the canonical
geometric morphism~$\Set[\TT] \to \Set$ is of presheaf type.

\begin{prop}The composition of geometric morphisms of presheaf type is of
presheaf type.\end{prop}

\begin{proof}(This is what Sina confirmed for us.) Spell out construction.
(Maybe in Section 4.)
\end{proof}


\section{Background on internal geometric theories}

XXX: State definition of geometric theory, being careful to only use geometric
constructions

XXX: State definition of model of a geometric theory, being careful to only use
geometric constructions

XXX: Explain internal geometric theories

XXX: Remark subtletly with respect to infinitary disjunctions and infinitary
proofs


\section{Proof of the main result}

XXX: State and prove main lemma about~$\Set[\TT][\TT']$


\section{Criteria for theories to be of presheaf type}

XXX: Give semantic criterion for being of presheaf type: Let~$\TT$ be of
presheaf type and let~$\TT''$ be an extension of~$\TT$ such that there is a
theory~$\TT'$ internal to~$\Set[\TT]$ such that~$\Set[\TT][\TT'] =
\Set[\TT'']$. If~$\TT'$ is of presheaf type from the point of view
of~$\Set[\TT]$, then~$\TT''$ is of presheaf type.

XXX: Explain that in the previous paragraph, the existence of~$\TT'$ is not
actually an assumption, but is in fact always satisfied. (?)

XXX: Draw conclusions, for instance that adding a further constant preserves
being of presheaf type (find lots of further such syntactical examples).


\section{Applications}

XXX: Results from the Master thesis of Matthias

XXX: Show that the naive definition of the big Zariski topos does not work.
(From the point of view of Sh(S), the big Zariski topos of S is \emph{not} the
topos of local~$\O_S$-algebras, but the topos of local~$\O_S$-algebras which
are local over~$\O_S$.)


\section{XXX general todo}

XXX cite Martha Bunge's characterization of functors of presheaf type

XXX cite Shawn Henry's PhD thesis

XXX cite that other guy who mentioned internal geometric theories (forgot whom
right now, might be Freyd or Andreas Blass, either in the result that the
object of Kuratowski-finite subsets is preserved under pullback or in the paper
about existential fixed-point logic)

XXX cite Azriel Lévy
\url{https://www.ams.org/journals/proc/1961-012-02/S0002-9939-1961-0122702-2/S0002-9939-1961-0122702-2.pdf}

XXX Realizability: An introduction to its categorical side, Thm. 2.7.1

\printbibliography

\end{document}
