\documentclass[oneside]{amsart}

\usepackage[utf8]{inputenc}
\usepackage{amsthm,mathtools,stmaryrd,amssymb}
\usepackage{booktabs}
\usepackage[all]{xy}
\usepackage[protrusion=true,expansion=true]{microtype}
\usepackage{xspace}

\usepackage[natbib=true,style=numeric,maxnames=10]{biblatex}
\usepackage[babel]{csquotes}
\bibliography{paper-filmat.bib}

\usepackage{pifont}
\newcommand{\cmark}{\ding{51}}
\newcommand{\xmark}{\ding{55}}

\title[]{Exploring mathematical objects from custom-tailored mathematical universes}
\author{Ingo Blechschmidt}
\address{Università di Verona \\
Department of Computer Science \\
Strada le Grazie 15 \\
37134 Verona, Italy}
\email{iblech@speicherleck.de}

\theoremstyle{definition}
\newtheorem{defn}{Definition}[section]
\newtheorem{ex}[defn]{Example}

\theoremstyle{plain}
\newtheorem{prop}[defn]{Proposition}
\newtheorem{cor}[defn]{Corollary}
\newtheorem{lemma}[defn]{Lemma}
\newtheorem{thm}[defn]{Theorem}
\newtheorem{scholium}[defn]{Scholium}

\theoremstyle{remark}
\newtheorem{rem}[defn]{Remark}
\newtheorem{question}[defn]{Question}
\newtheorem{speculation}[defn]{Speculation}
\newtheorem{caveat}[defn]{Caveat}
\newtheorem{conjecture}[defn]{Conjecture}

\newcommand{\xra}[1]{\xrightarrow{#1}}
\newcommand{\XXX}[1]{\textbf{XXX: #1}}
\newcommand{\aaa}{\mathfrak{a}}
\newcommand{\bbb}{\mathfrak{b}}
\newcommand{\mmm}{\mathfrak{m}}
\newcommand{\I}{\mathcal{I}}
\newcommand{\J}{\mathcal{J}}
\newcommand{\E}{\mathcal{E}}
\newcommand{\F}{\mathcal{F}}
\newcommand{\B}{\mathcal{B}}
\newcommand{\NN}{\mathbb{N}}
\newcommand{\RR}{\mathbb{R}}
\newcommand{\ZZ}{\mathbb{Z}}
\renewcommand{\P}{\mathcal{P}}
\renewcommand{\O}{\mathcal{O}}
\newcommand{\defeq}{\vcentcolon=}
\newcommand{\op}{\mathrm{op}}
\DeclareMathOperator{\Spec}{Spec}
\DeclareMathOperator{\Hom}{Hom}
\DeclareMathOperator{\Mod}{Mod}
\DeclareMathOperator{\Sh}{Sh}
\DeclareMathOperator{\PSh}{PSh}
\newcommand{\Set}{\mathrm{Set}}
\newcommand{\Eff}{\mathrm{Ef{}f}}
\renewcommand{\_}{\mathpunct{.}\,}
\newcommand{\effective}{ef{}fective\xspace}
\newcommand{\?}{\,{:}\,}
\newcommand{\realizes}{\Vdash}

\newcommand{\stacksproject}[1]{\cite[{\href{https://stacks.math.columbia.edu/tag/#1}{Tag~#1}}]{stacks-project}}

\begin{document}

\begin{abstract}
  foo
\end{abstract}

\maketitle
\thispagestyle{empty}

\noindent


\section{Exploring the \effective topos}

% XXX introduction

\begin{tabular}{lll}
  \toprule
  Statement & in $\Set$ & in $\Eff$ \\
  \midrule
  Any natural number is prime or not prime. & \cmark{} (trivially so) & \cmark \\
  There are infinitely many primes. & \cmark & \cmark \\
  Any function $\NN \to \NN$ is the zero function or not. & \cmark{} (trivially so) & \xmark \\
  Any function $\NN \to \NN$ is computable by a Turing machine. & \xmark & \cmark{} (trivially so) \\
  Any function $\RR \to \RR$ is continuous. & \xmark & \cmark \\
  \bottomrule
\end{tabular}

\subsection*{``Any natural number is prime or not.''} Even without knowing what
a prime number is, one can safely judge this statement to be true in
the standard topos, since it is just an instance of the law of excluded middle.

By the Kripke--Joyal semantics, saying that it's true in the \effective topos
amounts to saying that there is a Turing machine which, given a natural
number~$n$ as input, terminates with a correct judgment whether~$n$ is prime or
not. Such a Turing machine indeed exists -- writing such a program is often a
first exercise in programming courses. Hence the statement is also true in the
\effective topos, but for the nontrivial reason that such a machine exists.


\subsection*{``There are infinitely many primes.''} A first-order formalization
of this statement is ``for any natural number~$n$, there is a natural
number~$p$ which is greater than~$n$'', and is known to be true in the standard
topos by any of the many proofs of this fact.

Its external meaning when interpreted in the \effective topos is that there exists
a Turing machine which, given a natural number~$n$ as input, terminates with a
prime number~$p > n$ as output. Such a Turing machine exists, hence the
statement is true in the \effective topos.


\subsection*{``Any function~$\NN \to \NN$ is the zero function or not.''} More
formally, the statement is
\[ \forall f \? \NN^\NN\_
  \bigl((\forall n \? \NN\_ f(n) = 0) \vee
  \neg
  (\forall n \? \NN\_ f(n) = 0)\bigr). \]
By the law of excluded middle, this statement is trivially true in the standard
topos.

Its meaning when interpreted in the \effective topos is that there exists a
Turing machine~$M$ which, given the description of a Turing machine~$F$ which
computes a function~$f : \NN \to \NN$ as input, terminates with a correct
judgment of whether~$f$ is the zero function or not. Such a machine~$M$ does
not exist, hence the statement is false in the \effective topos.

A formal proof that such a machine~$M$ does not exist will reduce its assumed
existence to the undecidability of the halting problem. Intuitively, the issue
is the following. Turing machines are able to simulate other Turing machines.
Hence~$M$ could simulate~$F$ on various inputs to search the list of
function values~$f(0), f(1), \ldots$ for a nonzero number. In case that after
a certain number of steps a nonzero function value is found, the machine~$M$
can correctly output the judgment that~$f$ is not the zero function. But if the
search only turned up zero values, it cannot come to any verdict -- it cannot
rule out that a nonzero function value will show up in the as yet unexplored
part of the function.


\subsection*{``Any function~$\NN \to \NN$ is computable by a Turing machine.''}
The preceding examples could give the impression that what is true in the
\effective topos is simply a subset of what is true in the standard topos. This
statement shows that the relation between these two toposes is more nuanced.

The fundamental observation of computability theory is that, in the standard
topos, there are functions~$\NN \to \NN$ which are not computable by a Turing
machine. Explicit examples include the \emph{halting
function}, which maps a number~$n$ to zero or one depending on whether
the~$n$'th Turing machine (in some fixed enumeration of all Turing machines)
terminates or not, and the \emph{busy beaver function}. Cardinality arguments
even show that most functions~$\NN \to \NN$ are not computable: There
are~$\aleph_0^{\aleph_0} = 2^{\aleph_0}$ functions~$\NN \to \NN$, but
only~$\aleph_0$ Turing machines and hence only~$\aleph_0$ functions which are
computable by a Turing machine.

In contrast, in the \effective topos, any function~$\NN \to \NN$ is computable
by a Turing machine: The external meaning of this internal statement is that
there exists a Turing machine~$M$ which, given a description of a Turing
machine~$F$ computing a function~$f : \NN \to \NN$, outputs a description of a
Turing machine computing~$f$. It is trivial to program such a machine~$M$; the
machine~$M$ simply has to echo its input back to the caller.

To avert a paradox, we should point out where the proof of the fundamental
observation of computability theory employs nonconstructive reasoning, for if
it would admit a constructive proof, it would also hold internally to the
\effective topos, in contradiction to the fact that it does not. The halting
function~$h : \NN \to \NN$, defined using the case distinction
\[ h : n \mapsto \begin{cases}
  1, & \text{if the $n$'th Turing machine terminates}, \\
  0, & \text{if the $n$'th Turing machine does not terminate},
\end{cases} \]
cannot be given as a counterexample in the \effective topos since, in the
\effective topos, it is not actually a total function from~$\NN$ to~$\NN$. It
is only defined on those numbers~$n$ for which the~$n$'th Turing machine
terminates or does not terminate. Assuming the law of excluded middle, this is
a trivial condition; but intuitionistically, it is not. The definition of the
busy beaver function requires a similar case distinction and therefore also
does not give rise to a well-defined counterexample within the \effective
topos.


\subsection*{``Any function~$\RR \to \RR$ is continuous.''} In the standard
topos, this statement is plainly false, with the signum and Heaviside functions
being prominent counterexamples. In the \effective topos, this statement is
true. A formal proof is not entirely straightforward~\cite{XXX}; the intuitive
reason for why this is so is as follows.

What the \effective topos believes to be a real number is, from the external
point of view, a Turing machine~$X$ which outputs, when called with a natural
number~$n$ as input, a rational approximation~$X(n)$. These approximations are
required to be \emph{consistent} in the sense that~$|X(n) - X(m)|
\leq 1/(n+1) + 1/(m+1)$. Intuitively, such a machine~$X$ denotes the real
number~$\lim_{n \to \infty} X(n)$, and the approximations~$X(n)$ must be
within~$1/(n+1)$ of the limit.

A function~$f : \RR \to \RR$ in the \effective topos is therefore given by a
Turing machine~$M$ which, given the description of such a Turing machine~$X$ as
input, outputs the description of a similar such Turing machine~$Y$ as output.
To compute a rational approximation~$Y(n)$, the machine~$Y$ may simulate~$X$
and can therefore determine arbitrarily many rational approximations~$X(m)$.
However, within finite time, the machine~$Y$ can only acquire finitely many
such approximations. Hence a function such as the signum function, for which
even rough rational approximations of~$\operatorname{sgn}(x)$ require infinite
precision in the input~$x$, do not exist in the \effective topos.


\subsection*{The formal translation rules} The previous examples were picked to
convey an intuitive understanding of what statements in the \effective topos
externally mean, and to showcase a variety of different situations. The
formal translation rules are given in Table~\ref{table:eff}.

\begin{table}
  \begin{tabbing}
    $\Eff \models (\forall n\?\NN\_ \varphi(n))\!\!\!$ \= \kill
    $\Eff \models \varphi$ \> iff there is a Turing machine~$M$ such that~$M
    \realizes \varphi$. \\\\
    $M \realizes \top$ \> is true for any machine~$M$. \\
    $M \realizes \bot$ \> is false for any machine~$M$. \\
    $M \realizes (\varphi \wedge \psi)$ \> iff~$M$ terminates on inputs~$0$
    and~$1$ such that $M(0) \realizes \varphi$ and~$M(1) \realizes \psi$. \\
    $M \realizes (\varphi \vee \psi)$ \> iff~$M$ terminates on inputs~$0$ and~$1$
    such that, \\ \> \qquad if~$M(0) = 0$ then~$M(1) \realizes \varphi$, and if~$M(0) \neq
    0$ then~$M(1) \realizes \psi$. \\
    $M \realizes (\varphi \Rightarrow \psi)$ \> iff for any Turing machine~$E$
    such that~$E \realizes \varphi$, \\ \> \qquad $M$ terminates on input~$E$ such
    that~$M(E) \realizes \psi$. \\
    $M \realizes (\forall n\?\NN\_ \varphi(n))$ \> iff for any natural number~$n_0
    \in \NN$, \\ \> \qquad $M$ terminates on input~$n_0$ such that~$M(n_0)
    \realizes \varphi(n_0)$. \\
    $M \realizes (\exists n\?\NN\_ \varphi(n))$ \> iff~$M$ terminates on
    inputs~$0$ and~$1$ such that $M(1) \realizes \varphi(M(0))$.
  \end{tabbing}

  \caption{\label{table:eff} The (first-order fragment of) the translation
  rules defining the meaning of statements internal to the \effective topos.}
\end{table}

\printbibliography

\end{document}


Outline:


Stuff that should be mentioned:

* Andrej's realizability in the real world
* Cauchy vs. Dedekind numbers (physical quantities, ...)
* Syntactical vs. semantical interpretation
* phone call analogy
* detailed explanation of the pretty picture; analogy with "inner models"
  of set theory
* different intention with the alternative universes in comparison to set theory
* quick overview of the several aspects of toposes (maybe at the end,
  as an outlook?)
* enrichment of platonism debate (find better term for this!)
* uncovering further relations between objects
* allowing a switch of perspective
* applications in mathematical practice
* int. logic as common denominator
* arbitrariness of the "standard axioms"
* models of ZF yield toposes (models of ETCS), including a quick discussion of
  equivalence
* what giving up classical logic actually amounts to in practice
* examples in Eff
* examples in Eff(STM)
* examples in Sh(X)
* examples in Spec(A), "reifying all the individual localizations into a
  single coherent entity which can be reasoned about as if it were a single
  ring"
* importance of having an adapted language
* SDG (infinitesimals are well-studied in philosophy of mathematics)
* beautiful and intriguing fact: laws of logic apply to mathematical objects
  not only on the face, but also in different ways
* quick remark on the internal language being based on types instead of sets
* reference to Bohr topos approach
* Source https://plato.stanford.edu/entries/category-theory/ for relevant
  literatur
* the internal language as a generalized modal operator

Send to Georg.
