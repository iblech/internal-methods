\documentclass[oneside]{amsart}

\usepackage[utf8]{inputenc}
\usepackage{amsthm,mathtools,stmaryrd}
\usepackage[all]{xy}
%\usepackage[protrusion=true,expansion=true]{microtype}
\usepackage[hyphens]{url}
\usepackage[breaklinks=true]{hyperref}
\usepackage{xspace}

\usepackage[natbib=true,style=numeric,maxnames=10]{biblatex}
\usepackage[babel]{csquotes}
\bibliography{paper-flabby-objects.bib}

\title{Exploring mathematical objects from custom-tailored mathematical universes}
\author{Ingo Blechschmidt}
\address{Università di Verona \\
Department of Computer Science \\
Strada le Grazie 15 \\
37134 Verona, Italy}
\email{iblech@speicherleck.de}

\theoremstyle{definition}
\newtheorem{defn}{Definition}[section]
\newtheorem{ex}[defn]{Example}

\theoremstyle{plain}
\newtheorem{prop}[defn]{Proposition}
\newtheorem{cor}[defn]{Corollary}
\newtheorem{lemma}[defn]{Lemma}
\newtheorem{thm}[defn]{Theorem}
\newtheorem{scholium}[defn]{Scholium}

\theoremstyle{remark}
\newtheorem{rem}[defn]{Remark}
\newtheorem{question}[defn]{Question}
\newtheorem{speculation}[defn]{Speculation}
\newtheorem{caveat}[defn]{Caveat}
\newtheorem{conjecture}[defn]{Conjecture}

\newcommand{\xra}[1]{\xrightarrow{#1}}
\newcommand{\XXX}[1]{\textbf{XXX: #1}}
\newcommand{\aaa}{\mathfrak{a}}
\newcommand{\bbb}{\mathfrak{b}}
\newcommand{\mmm}{\mathfrak{m}}
\newcommand{\I}{\mathcal{I}}
\newcommand{\J}{\mathcal{J}}
\newcommand{\E}{\mathcal{E}}
\newcommand{\F}{\mathcal{F}}
\newcommand{\B}{\mathcal{B}}
\newcommand{\NN}{\mathbb{N}}
\newcommand{\ZZ}{\mathbb{Z}}
\renewcommand{\P}{\mathcal{P}}
\renewcommand{\O}{\mathcal{O}}
\newcommand{\defeq}{\vcentcolon=}
\newcommand{\op}{\mathrm{op}}
\DeclareMathOperator{\Spec}{Spec}
\DeclareMathOperator{\Hom}{Hom}
\DeclareMathOperator{\Mod}{Mod}
\DeclareMathOperator{\Sh}{Sh}
\DeclareMathOperator{\PSh}{PSh}
\newcommand{\Set}{\mathrm{Set}}
\newcommand{\Eff}{\mathrm{Ef{}f}}
\renewcommand{\_}{\mathpunct{.}\,}
\newcommand{\effective}{ef{}fective\xspace}

\newcommand{\stacksproject}[1]{\cite[{\href{https://stacks.math.columbia.edu/tag/#1}{Tag~#1}}]{stacks-project}}

\begin{document}

\begin{abstract}
  foo
\end{abstract}

\maketitle
\thispagestyle{empty}

\noindent

\printbibliography

\end{document}


Outline:


Stuff that should be mentioned:

* Andrej's realizability in the real world
* Cauchy vs. Dedekind numbers (physical quantities, ...)
* Syntactical vs. semantical interpretation
* phone call analogy
* detailed explanation of the pretty picture; analogy with "inner models"
  of set theory
* different intention with the alternative universes in comparison to set theory
* quick overview of the several aspects of toposes (maybe at the end,
  as an outlook?)
* enrichment of platonism debate (find better term for this!)
* uncovering further relations between objects
* allowing a switch of perspective
* applications in mathematical practice
* int. logic as common denominator
* arbitrariness of the "standard axioms"
* models of ZF yield toposes (models of ETCS), including a quick discussion of
  equivalence
* what giving up classical logic actually amounts to in practice
* examples in Eff
* examples in Sh(X)
* examples in Spec(A), "reifying all the individual localizations into a
  single coherent entity which can be reasoned about as if it were a single
  ring"
* importance of having an adapted language
* SDG (infinitesimals are well-studied in philosophy of mathematics)
* beautiful and intriguing fact: laws of logic apply to mathematical objects
  not only on the face, but also in different ways
* quick remark on the internal language being based on types instead of sets
