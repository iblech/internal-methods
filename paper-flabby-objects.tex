\documentclass[oneside]{amsart}

\usepackage[utf8]{inputenc}
\usepackage{amsthm,mathtools,stmaryrd}
\usepackage[all]{xy}
\usepackage[protrusion=true,expansion=true]{microtype}
\usepackage{hyperref}
\usepackage{xspace}
\usepackage{color}

\usepackage[natbib=true,style=numeric,maxnames=10]{biblatex}
\usepackage[babel]{csquotes}
\bibliography{paper-flabby-objects.bib}

\title{Flabby and injective objects in toposes}
\author{Ingo Blechschmidt}
\address{Università di Verona \\
Department of Computer Science \\
Strada le Grazie 15 \\
37134 Verona, Italy}
\email{iblech@speicherleck.de}

\theoremstyle{definition}
\newtheorem{defn}{Definition}[section]
\newtheorem{ex}[defn]{Example}

\theoremstyle{plain}
\newtheorem{prop}[defn]{Proposition}
\newtheorem{cor}[defn]{Corollary}
\newtheorem{lemma}[defn]{Lemma}
\newtheorem{thm}[defn]{Theorem}
\newtheorem{scholium}[defn]{Scholium}

\theoremstyle{remark}
\newtheorem{rem}[defn]{Remark}
\newtheorem{question}[defn]{Question}
\newtheorem{speculation}[defn]{Speculation}
\newtheorem{caveat}[defn]{Caveat}
\newtheorem{conjecture}[defn]{Conjecture}

\newcommand{\xra}[1]{\xrightarrow{#1}}
\newcommand{\XXX}[1]{\textbf{\textcolor{red}{XXX: #1}}}
\newcommand{\aaa}{\mathfrak{a}}
\newcommand{\bbb}{\mathfrak{b}}
\newcommand{\mmm}{\mathfrak{m}}
\newcommand{\I}{\mathcal{I}}
\newcommand{\J}{\mathcal{J}}
\newcommand{\E}{\mathcal{E}}
\newcommand{\F}{\mathcal{F}}
\newcommand{\B}{\mathcal{B}}
\newcommand{\NN}{\mathbb{N}}
\newcommand{\ZZ}{\mathbb{Z}}
\newcommand{\QQ}{\mathbb{Q}}
\renewcommand{\P}{\mathcal{P}}
\renewcommand{\O}{\mathcal{O}}
\newcommand{\defeq}{\vcentcolon=}
\newcommand{\defeqv}{\vcentcolon\equiv}
\newcommand{\op}{\mathrm{op}}
\DeclareMathOperator{\Spec}{Spec}
\DeclareMathOperator{\Hom}{Hom}
\DeclareMathOperator{\Mod}{Mod}
\DeclareMathOperator{\Sh}{Sh}
\DeclareMathOperator{\PSh}{PSh}
\newcommand{\Set}{\mathrm{Set}}
\newcommand{\Eff}{\mathrm{Ef{}f}}
\renewcommand{\_}{\mathpunct{.}\,}
\newcommand{\effective}{ef{}fective\xspace}

\newcommand{\stacksproject}[1]{\cite[{\href{https://stacks.math.columbia.edu/tag/#1}{Tag~#1}}]{stacks-project}}

\begin{document}

\begin{abstract}
  We introduce a general notion of \emph{flabby objects} in elementary toposes
  and study their basic properties. In the special case of localic toposes, this
  notion reduces to the established notion of flabby sheaves, yielding a
  site-independent characterization of flabby sheaves. Continuing a line of
  research started by Roswitha Harting, we use flabby objects to
  show that an internal notion of injective objects coincides with the
  corresponding external notion, in stark contrast with the situation for
  projective objects. We show as an application that higher direct images can
  be understood as internal cohomology, and we study flabby objects in the
  \effective topos.
\end{abstract}

\maketitle
\thispagestyle{empty}

\noindent
As is nowadays well-established, every topos supports an \emph{internal language}
which can be used to reason about the objects and morphisms of the topos in a
naive element-based language, allowing us to pretend that the objects are plain
sets (or types) and that the morphisms are plain maps between those sets
(\cite[Chapter~6]{borceux:handbook3}, \cite[Section~1.3]{caramello:ttt},
\cite[Chapter~14]{goldblatt:topoi},
\cite[Chapter~VI]{moerdijk-maclane:sheaves-logic}). The internal language is
sound with respect to intuitionistic reasoning, whereby every intuitionistic
theorem holds in every topos.

The internal language of a sheaf topos enables \emph{relativization by
internalization}. For instance, by interpreting the
proposition \begin{quote}``in every short exact sequence of modules, if the two
outer ones are finitely generated then so is the middle one''\end{quote} of
intuitionistic commutative algebra internally to the topos of sheaves over a
space~$X$, we obtain the geometric analogue \begin{quote}``in every short exact
sequence of sheaves of modules over~$X$, if the two outer ones are of finite
type then so is the middle one''.\end{quote}
This way of deducing geometric theorems provides conceptual clarity, reduces
technical overhead and justifies certain kinds of ``fast and loose reasoning''
typical of informal algebraic geometry. As soon as we go beyond the fragment of
geometric sequents and consider more involved first-order or even higher-order
statements, also significant improvements in proof length and proof complexity
can be obtained. For instance, Grothendieck's generic freeness lemma admits a
short and simple proof in this framework, while previously-published proofs
proceed in a somewhat involved series of reduction steps and require a fair
amount of prerequisites in commutative
algebra~\cite{blechschmidt:phd,blechschmidt:generic-freeness}.

The practicality of this approach hinges on the extent to which the dictionary
between internal and external notions has been worked out. For instance, the
simple example displayed above hinges on the dictionary entry stating that a
sheaf of modules is of finite type if and only if it looks like a
finitely generated module from the internal point of view. The motivation for
this note was to find internal characterizations of flabby sheaves and of higher
direct images, and the resulting entries are laid out in
Section~\ref{sect:flabby-objects} and in
Section~\ref{sect:higher-direct-images}: A sheaf is flabby if and only if, from
the internal point of view, it is a flabby set, a notion introduced in
Section~\ref{sect:flabby-sets} below; and higher direct images look like sheaf
cohomology from the internal point of view.

As a byproduct, we demonstrate how the notion of flabby sets is a useful
organizing principle in the study of injective objects. We employ flabby sets
to give a new proof of Roswitha Harting's results that injectivity of sheaves is
a local notion~\cite{harting:remark} and that a sheaf is injective if and only
if it is injective from the internal point of
view~\cite{harting:locally-injective}, which she stated (in slightly different
language) for sheaves of abelian groups. We use the opportunity to correct a
small inaccuracy of hers, namely claiming that the analogous results for sheaves of
modules would be false.

When employing the internal language of a topos, we are always referring to Mike Shulman's
extension of the usual internal language, his \emph{stack
semantics}~\cite{shulman:stack-semantics}. This extension allows to internalize
unbounded quantification, which among other things is required to express the
internal injectivity condition and the internal construction of sheaf
cohomology via injective resolutions.

A further motivation for this note was our desire to seek a constructive
account of sheaf cohomology. Sheaf cohomology is commonly defined using
injective resolutions, which can fail to exist in the absence of the axiom of
choice~\cite{blass:inj-proj-axc}, but flabby resolutions
can also be used in their stead, making them the obvious candidate for a
constructively sensible replacement of the usual definition. However, we show
in Section~\ref{sect:in-eff} and in Section~\ref{sect:conclusion} that flabby
resolutions present their own challenges, and in summary we failed to reach this
goal. The problem of giving a constructive account
of sheaf cohomology is still open~\cite{tenorio-mariano:cohomology}.

In view of almost 80 years of sheaf cohomology, this state of affairs is
slightly embarrassing, challenging the call that ``once [a] subject is better
understood, we can hope to refine its definitions and proofs so as to
avoid [the law of excluded middle]''~\cite[Section~3.4]{hott}.

A constructive account of sheaf cohomology would be highly desirable, not only
out of a philosophical desire to obtain a deeper understanding of the
foundations of sheaf cohomology, but also to: use the tools of sheaf cohomology
in the internal setting of toposes, thereby extending their applicability by
relativization by internalization; and to carry out \emph{integrated
developments} of algorithms for computing sheaf cohomology, where we would
extract algorithms together with termination and correctness proofs from a
hypothetical constructive account.

% XXX eff topos

\textbf{Acknowledgments.} We are grateful to Thorsten Altenkirch, Simon Henry
and Maria Emilia Maietti for insightful discussions and pointers to prior work,
to Jürgen Jost, Marc Nieper-Wißkirchen and Peter Schuster for valuable
guidance, and to Daniel Albert and Giacomo Cozzi for their careful readings of
earlier drafts. The work for this note was carried out at the University of
Augsburg, the Max Planck Institute for Mathematics in the Sciences in
Leipzig and the University of Verona.


\section{Flabby sheaves}

A sheaf~$F$ on a topological space or a locale~$X$ is \emph{flabby} (flasque) if and only
if all restriction maps~$F(X) \to F(U)$ are surjective. The following
properties of flabby sheaves render them fundamental to the theory of sheaf
cohomology:
\begin{enumerate}
\item Let~$(U_i)_i$ be an open covering of~$X$.
A sheaf~$F$ on~$X$ is flabby if and only if all of its restrictions~$F|_{U_i}$
are flabby as sheaves on~$U_i$.
\item Let~$f : X \to Y$ be a continuous map. If~$F$ is a flabby sheaf on~$X$,
then~$f_*(F)$ is a flabby sheaf on~$Y$.
\item[(3)] Let~$0 \to F \to G \to H \to 0$ be a short exact sequence of sheaves of
modules.
\begin{enumerate}
\item If~$F$ is flabby, then this sequence is also exact as a sequence of
presheaves.
\item If~$F$ and~$H$ are flabby, then so is~$G$.
\item If~$F$ and~$G$ are flabby, then so is~$H$.
\end{enumerate}
\item[(4a)] Every sheaf embeds canonically into a flabby sheaf.
\item[(4b)] Every sheaf of modules embeds canonically into a flabby sheaf of modules.
\end{enumerate}

Since we want to develop an analogous theory for flabby objects in elementary
toposes, it is worthwhile to analyze the logical and set-theoretic commitments
which are required to establish these properties. The standard proofs of
properties~(1),~(3a),~(3b) and~(3c) require Zorn's lemma to construct maximal
extensions of given sections.\footnote{We are careful to distinguish between
the axiom of choice and Zorn's lemma. The former implies the latter, but the
converse implication requires the law of excluded middle.}
The standard proof of property~(4b) requires the
law of excluded middle, to ensure that the Godement construction actually
yields a flabby sheaf. Properties~(2) and~(4a) can be verified purely
intuitionistically.

There is an alternative definition of flabbiness, to be introduced below, which
is equivalent to the usual one in presence of Zorn's lemma and which
requires different commitments: For the alternative definition,
properties~(1),~(3b),~(4a) and~(4b) can be verified purely intuitionistically.
There is a substitute for property~(3a) which can be verified purely
intuitionistically.

Both definitions can be generalized to yield notions of flabby objects in
elementary toposes; but for toposes which are not localic, the two resulting
notions will differ, and only the one obtained from the alternative definition
is stable under pullback and
can be characterized in the internal language. We therefore adopt in this paper the
alternative one as the official definition.

\begin{defn}\label{defn:flabby-sheaf}
A sheaf~$F$ on a topological space (or locale)~$X$ is \emph{flabby}
if and only if for all opens~$U$ and all sections~$s \in F(U)$, there is an
open covering~$X = \bigcup_i U_i$ such that, for all~$i$, the section~$s$ can
be extended to a section on~$U \cup U_i$.\footnote{Even in the absence of the
axiom of choice it doesn't make a difference
whether we are stipulating, as in the definition, for each index~$i \in I$ the mere
existence of an extension~$t \in F(U \cup U_i)$, or whether we are stipulating
the existence of a family~$(t_i)_i$ of extensions~$t_i \in F(U \cup U_i)$.
Clearly, the existence of the family implies the existence of individual
extensions. Conversely, if for each~$i \in I$ there exists an extension, there
is a tautologous family of extensions over the enlarged index set~$I' \defeq \{
(i,t) \,|\, i \in I, t \in F(U \cup U_i), t|_U = s \}$, and we still have~$X =
\bigcup_{(i,t) \in I'} U_i$.}
\end{defn}

% Take care: This paragraph is referenced in prop:global-elements.
If~$F$ is a flabby sheaf in the traditional sense, then~$F$ is obviously also
flabby in the sense of Definition~\ref{defn:flabby-sheaf} -- the singleton
covering~$X = X$ will do. Conversely, let~$F$ be a flabby sheaf in the sense of
Definition~\ref{defn:flabby-sheaf}. Let~$s \in F(U)$ be a local section. Zorn's
lemma implies that there is a maximal extension~$s' \in F(U')$. By assumption,
there is an open covering~$X = \bigcup_i U_i$ such that, for all~$i$, the
section~$s'$ can be extended to~$U' \cup U_i$. Since~$s'$ is maximal, $U' \cup
U_i = U'$ for all~$i$. Therefore~$X = \bigcup_i U_i \subseteq U'$; hence~$s'$ is a
global section, as desired.

We remark that unlike the traditional definition of flabbiness,
Definition~\ref{defn:flabby-sheaf} exhibits flabbiness as a manifestly local
notion.


\section{Flabby sets}\label{sect:flabby-sets}

We intend this section to be applied in the internal language of an elementary
topos; we will speak about sets and maps between sets, but intend our arguments
to be applied to objects and morphisms in toposes. We will therefore be careful
to reason purely intuitionistically. We
adopt the terminology of~\cite{kock:partial-maps} regarding subterminals and
subsingletons: A subset~$K \subseteq X$ is \emph{subterminal} if and only if every given
elements are equal ($\forall x,y \in K\_ x = y$), and it is a
\emph{subsingleton} if and only if there is an element~$x \in X$ such that~$K
\subseteq \{ x \}$. Every subsingleton is trivially subterminal, but the converse
might fail.

\begin{defn}\label{defn:flabby-set}A set~$X$ is \emph{flabby} if and only if every subterminal subset
of~$X$ is a subsingleton, that is, if and only if for every subset~$K \subseteq
X$ such that~$\forall x,y \in K\_ x = y$, there exists an element~$x \in X$
such that~$K \subseteq \{ x \}$.
\end{defn}

In the presence of the law of excluded middle, a set is flabby if and only if
it is inhabited. This characterization is a constructive taboo:

\begin{prop}\label{prop:taboo}
If every inhabited set is flabby, then the law of excluded middle
holds.
\end{prop}

\begin{proof}Let~$\varphi$ be a truth value. The set~$X \defeq \{ 0 \}
\cup \{ 1 \,|\, \varphi \} \subseteq \{ 0,1 \}$ is inhabited by~$0$ and
contains~$1$ if and only if~$\varphi$ holds. Let~$K$ be the subterminal~$\{ 1 \,|\, \varphi
\} \subseteq X$. Flabbiness of~$X$ implies that there exists an element~$x \in
X$ such that~$K \subseteq \{x\}$. We have~$x \neq 1$ or~$x = 1$. The first case
entails~$\neg\varphi$. The second case entails~$1 \in X$, so~$\varphi$.
\end{proof}

Let~$\P_{\leq 1}(X)$ be the set of subterminals of~$X$.

\begin{prop}A set~$X$ is flabby if and only if the canonical map~$X \to
\P_{\leq 1}(X)$ which sends an element~$x$ to the singleton set~$\{x\}$ is
final.
\end{prop}

\begin{proof}By definition.\end{proof}

The set~$\P_{\leq 1}(X)$ of subterminals of $X$ can be interpreted as the set
of \emph{partially-defined elements} of $X$. In this view, the empty subset is
the maximally undefined element and a singleton is a maximally defined element.
A set is flabby if and only if every of its partially-defined elements can be
refined to an honest element.

\begin{rem}\label{rem:constant-flabby}
A set~$X$ is \emph{$\neg\neg$-separated} if and only
if~$\neg\neg(x=y)\Rightarrow x=y$ for all elements~$x,y \in X$.
Although Section~\ref{sect:enough-flabby-modules} presents some
relation between flabby sets and~$\neg\neg$-separated sets, neither notion
encompasses the other. The set~$\Omega$ is flabby, but might fail to
be~$\neg\neg$-separated; the set~$\ZZ$ is~$\neg\neg$-separated, even discrete,
but might fail to be flabby. This can abstractly be seen by adapting the proof of
Proposition~\ref{prop:taboo}. An explicit model in which~$\ZZ$ is not flabby
can be obtained by picking any topological space~$T$ such that~$H^1(T,
\underline{\ZZ}) \neq 0$, where~$\underline{\ZZ}$ is the constant sheaf with
stalks~$\ZZ$. For instance, the space~$T$ could be connected (such that every
global section is constant) while having an open which is the disjoint union of
two connected components. Then the sheaf~$\underline{\ZZ}$ is not
flabby and hence, by Proposition~\ref{prop:flabby-sheaves-objects} below, not a
flabby set from the internal point of view of~$\Sh(T)$.
\end{rem}

\begin{defn}\begin{enumerate}
\item A set~$I$ is \emph{injective} if and only if, for every injection~$i
: A \to B$, every map~$f : A \to I$ can be extended to a map~$B \to I$.
\item An~$R$-module~$I$ is \emph{injective} if and only if, for every linear
injection~$i : A \to B$ between~$R$-modules, every linear map~$f : A \to I$ can
be extended to a linear map~$B \to I$, as in the diagram below.
\end{enumerate}
\[ \xymatrix{
  A \ar@{^{(}->}[r]\ar[d] & B \ar@{-->}@/^/[ld] \\
  I
} \]
\end{defn}

In the presence of the law of excluded middle, a set is injective if and only
if it is inhabited. In the presence of the axiom of choice, an abelian group is
injective (as a~$\ZZ$-module) if and only if it is divisible. Injective sets
and modules have been intensively studied in the context of foundations
before~\cite{blass:inj-proj-axc,harting:locally-injective,kenney:injective-choice,aczel-berg-granstroem-schuster:injective};
the following properties are well-known:

\begin{prop}\label{prop:basics-injective}
\begin{enumerate}
% Take care if numbering changes (referenced below)
\item Every set embeds canonically (that is, in a uniform fashion) into an injective set.
\item Every injective module is also injective as a set.
\item Assuming the axiom of choice (so Zorn's lemma in combination with the law of excluded middle), every module embeds into an
injective module.
\end{enumerate}\end{prop}

\begin{proof}\begin{enumerate}
\item One can check that, for instance, the full powerset~$\P(X)$ and the set of
subterminals~$\P_{\leq 1}(X)$ are each injective.
\item This statement follows from general abstract nonsense, since the forgetful functor from modules to sets possesses a monomorphism-preserving left
adjoint. More explicitly, let~$I$ be an injective~$R$-module, let~$i : A \to B$ be an injective map between sets and
let~$f : A \to I$ be an arbitrary map. Then the induced map~$R\langle A \rangle
\to R\langle B \rangle$ between free modules is also injective, the given
map~$f$ lifts to a linear map~$R\langle A \rangle \to I$, and an~$R$-linear
extension~$R\langle B \rangle \to I$ induces an extension~$B \to I$ of~$f$.
\item One verifies that any abelian group embeds into a divisible
abelian group. By Baer's criterion (which in this form requires the axiom of choice), divisible abelian groups are injective.
The result for modules over arbitrary rings then follows purely
formally, since the functor~$A \mapsto \Hom(R,A)$ from abelian groups
to~$R$-modules has a left exact left adjoint with monic unit. \qedhere
\end{enumerate}\end{proof}

We note in passing that the multi-step technique of the proof of
Theorem~\ref{thm:enough-flabby-modules} below can be used to verify
Proposition~\ref{prop:basics-injective}(3) also in the absence of the law of
excluded middle (but still requiring Zorn's lemma). Since this observation is of no
further import for the purposes of this text, details are postponed to
Section~\ref{sect:enough-injective-modules}.

\begin{prop}\label{prop:injective-flabby}
Every injective set is flabby.\end{prop}

\begin{proof}Let~$I$ be an injective set. Let~$K \subseteq I$ be a subterminal.
The inclusion~$f : K \to I$ extends along the injection~$K \to 1 = \{\star\}$
to a map~$1 \to I$. The unique image~$x$ of that map has the property that~$K
\subseteq \{x\}$.\end{proof}

\begin{cor}\label{cor:enough-flabby-sets}
Every set canonically embeds into a flabby set.\end{cor}

\begin{proof}Immediate by Proposition~\ref{prop:basics-injective}(1) and
Proposition~\ref{prop:injective-flabby}.\end{proof}

A further corollary of Proposition~\ref{prop:injective-flabby} is that the
statement ``every inhabited set is injective'' is a constructive taboo: If every
inhabited set is injective, then every inhabited set is flabby, thus the law of
excluded middle follows by Proposition~\ref{prop:taboo}.

\begin{prop}Every singleton set is flabby. The binary cartesian product of flabby sets
is flabby.\end{prop}

\begin{proof}Immediate.\end{proof}

Subsets of flabby sets are in general not flabby, as else every set would be
flabby in view of Corollary~\ref{cor:enough-flabby-sets}.


\subsection{On the existence of enough flabby modules}
\label{sect:enough-flabby-modules}

For the intended application to the theory of sheaf cohomology, the existence
of \emph{enough flabby sheaves of modules} is crucial: Any sheaf of modules
embeds into a flabby sheaf of modules. In this section, we study the existence
of enough flabby modules from the non-sheaf theoretic but constructive point of
view.

We have already established that every set embeds into a flabby set. However,
the supersets suggested by the proof of Corollary~\ref{cor:enough-flabby-sets}
do not carry a module structure even when the base set does. This deficit
raises the following question: Given a module~$M$, does there exist a linear
injection~$M \to M'$ into a set~$M'$ which simultaneously carries a module
structure and is flabby?

An earlier version of this text formulated this question as an open problem.
The positive solution presented below rests on the intermediate notion of
\emph{functionally flabby} sets and makes essential use of sheaves for modal
operators. A short survey for the latter is contained
in~\cite[Section~6]{blechschmidt:phd}; other references
include~\cite[Sections~14.4f.]{goldblatt:topoi},~\cite{goldblatt:modality}
\cite{vries:sheafification} and~\cite{fourman-scott:sheaves-and-logic}.\footnote{On page~5 of the
preprint~\cite{vries:sheafification} there is a slight typing error:
Fact~2.1(i) gives the characterization of~$j$-closedness, not~$j$-denseness.
The correct characterization of~$j$-denseness in that context is~$\forall b \in
B\_ j(b \in A)$.}

\begin{defn}\label{defn:functionally-flabby}
A set~$X$ is \emph{functionally flabby} if and only if there is a
map~$\varepsilon : \P_{\leq1}(X) \to X$ such that for all~$K \in
\P_{\leq1}(X)$, $K \subseteq \{ \varepsilon(K) \}$.\end{defn}

Trivially, every functionally flabby set is flabby in the sense of
Definition~\ref{defn:flabby-set}.
A map~$\varepsilon : \P_{\leq1}(X) \to X$ satisfies the condition in
Definition~\ref{defn:functionally-flabby} if and only if it is a retraction of
the canonical injection~$X \to \P_{\leq1}(X)$.

\begin{prop}For a set~$X$, the following conditions are equivalent:
\begin{enumerate}
\item $X$ is injective.
\item $X$ is functionally flabby.
\end{enumerate}
\end{prop}

\begin{proof}Let~$X$ be an injective set. Then the identity~$X \to X$ can be extended
along the canonical inclusion~$X \to \P_{\leq1}(X)$ to a map~$\varepsilon :
\P_{\leq1}(X) \to X$. This map witnesses that~$X$ is functionally flabby.

Conversely, let~$X$ be a functionally flabby set with witness~$\varepsilon :
\P_{\leq1}(X) \to X$. Let~$i : A \to B$ be an injection and let~$f : A \to X$
be a map. Then~$x \mapsto \varepsilon(f[i^{-1}[\{x\}])$ is an extension of~$f$
along~$i$.
\end{proof}

\begin{thm}\label{thm:enough-flabby-modules}
Every module embeds canonically into a functionally flabby module.\end{thm}

\begin{proof}We explain how to construct the required flabby envelopes
in three steps.
First, let~$M$ be a module which is a sheaf for some modal operator~$\nabla$
with the property that for any
formula~$\varphi$, $\nabla(\varphi \vee (\varphi \Rightarrow \nabla\bot))$. In
this case, the module~$M$ is already functionally flabby: A witnessing choice
function maps a subterminal~$K \subseteq M$ to the unique element~$x \in M$
such that~$\nabla(x \in K')$, where~$K' \defeq K \cup \{ 0 \,|\, \text{$K$
inhabited} \Rightarrow \nabla\bot \}$. Since the set~$K'$ has, unlike~$K$, the property
that~$\nabla(\text{$K'$ is a singleton})$, such an element~$x$ exists and is unique
by the sheaf condition.

Second, let~$M$ be a module which is separated for some modal operator~$\nabla$
as above. Then its sheafification is a functionally flabby module into
which~$M$ embeds since the canonical map from~$M$ to its sheafification is
injective by separatedness of~$M$.

Finally, let~$M$ be an arbitrary module. For any element~$x \in M$,
let~$\nabla_x$ be the modal operator with~$\nabla_x\varphi \defeqv ((\varphi \Rightarrow x
= 0) \Rightarrow x = 0)$. This is a modal operator of the kind as above.
The module~$M$ might not be~$\nabla_x$-separated for
any particular element~$x$, but it is jointly so: For all elements~$a,b\in M$,
if~$\nabla_x(a = b)$ for all~$x \in M$, then~$a = b$ (considering~$x \defeq
a-b$). Hence the canonical map~$M \to \prod_{x \in M} M^{+_x}$ into the product
of the plus constructions with respect to all the modal operators~$\nabla_x$ is
injective, and the desired injection is the composition
\[ M \longrightarrow \prod_{x \in M} M^{+_x} \longrightarrow \prod_{x \in M}
M^{+_x+_x} \]
into the product of the sheafifications.
\end{proof}

\begin{rem}For the purpose of verifying that any module embeds into a flabby
module, it is essential that in the first step of the proof of
Theorem~\ref{thm:enough-flabby-modules} the module~$M$ is not only shown to be flabby,
but even functionally flabby, and moreover in an explicit manner, explicitly
presenting a witnessing map~$\P_{\leq1}(M) \to M$. This is because while an
arbitrary product of flabby sets can fail to be flabby, an arbitrary product of
functionally flabby sets with given witnesses is again so.
\end{rem}

Given the somewhat nontrivial nature of the construction in the proof of
Theorem~\ref{thm:enough-flabby-modules}, a natural question is whether simpler
constructions exist as well.
There are a number of simple constructions which come close to providing
flabby envelopes for arbitrary modules, but all such constructions known to the
author fail in some manner. For instance, given a module~$M$, we could equip the set~$T
\defeq \P_{\leq1}(M)/{\sim}$, where~$K \sim L$ if and only if~$K = L$ or~$K
\cup L \subseteq \{0\}$, with a module structure given by~$0 \defeq [\{0\}]$,
$[K]+[L] \defeq [K+L]$ and~$r [K] \defeq rK$. The resulting module admits a
linear injection from~$M$, sending an element~$x$ to~$[\{x\}]$. However, it
fails to be flabby. Given a subterminal~$E \subseteq \P_{\leq1}(M)/{\sim}$,
there is the well-defined element~$v \defeq [\{x \in M\,|\,\text{$x \in K$ for
some~$[K] \in E$}\}]$, but we cannot verify~$E \subseteq \{v\}$.

That said, there is an alternative construction if sufficiently general
\emph{quotient inductive types}, as suggested by Altenkirch and
Kaposi~\cite{altenkirch-kaposi:qits}, are available. These generalize ordinary
inductive~$W$-types, which exist in any
topos~\cite{moerdijk-palmgren:wellfounded-trees,berg-moerdijk:w-types-in-sheaves,berg-kouwenhoven-gentil:w-types-in-eff}
and whose existence
can indeed be verified in an intuitionistic set theory like~\textsc{izf}~\cite{crosilla:cst-izf}, by allowing to
give constructors and state identifications at the same time. More
specifically, given an~$R$-module~$M$, we can construct a flabby envelope~$T$
of~$M$ as the quotient inductive type generated by the following clauses,
starting out as the construction of the free module over the underlying set
of~$M$:~$0
\in T$ (where~$0$ is a formal symbol); if~$t,s \in T$, then~$t + s \in T$;
if~$t \in T$ and~$r \in R$, then~$rt \in T$; if~$x \in M$, then~$\underline{x}
\in T$; if~$K : I \to T$ is a family of elements of~$T$ indexed by a subterminal, then~$\varepsilon_K \in T$;
if~$t,s,u \in T$ and~$r,r' \in R$, then~$t + (s + u) = (t + s) + u$, $t + s = s + t$, $t + 0 = t
= t + 0$, $0t = 0$, $1t = t$, $r(t+u) = rt + ru$, $(r+r')t = rt + r't$; if~$x,y
\in M$ and~$r \in R$, then~$\underline{0} = 0$, $\underline{x + y} =
\underline{x} + \underline{y}$, $\underline{rx} = r \underline{x}$; and if~$K :
I \to T$ is a family such that~$I$ is inhabited by some element~$i_0$,
then~$\varepsilon_{K} = K(i_0)$.

However, there are two issues with this approach.
Firstly, it is an open question under which circumstances quotient inductive
types can be shown to exist. Zermelo--Fraenkel with choice certainly suffices,
while Zermelo--Fraenkel without choice does not~\cite[Section~9]{shulman-lumsdaine:hits},
hence~\textsc{izf} also does not.\footnote{With quotient inductive types, every infinitary
algebraic theory admits free algebras. However, it is consistent with
Zermelo--Fraenkel set theory that some such theories do not admit free
algebras~\cite{blass:free-algebras}.} The existence of quotient inductive types
seems to be, as the existence of enough injective modules, \emph{constructively
neutral}.
Secondly, by referencing arbitrary families of elements, the construction
transcends the given type-theoretic or set-theoretic universe; the resulting
object~$T$ is not manifestly small even if~$M$ is.


\subsection{Exactness properties}

\begin{prop}\label{prop:hom-flabby}
\begin{enumerate}
\item Let~$I$ be an injective set. Let~$T$ be an arbitrary set. Then the
set~$I^T$ of maps from~$T$ to~$I$ is injective.
\item Let~$I$ be an injective~$R$-module. Let~$T$ be an arbitrary~$R$-module.
If~$T$ is flat, then the module~$\Hom_R(T,I)$ of linear maps from~$T$ to~$I$ is
injective. In the general case, it is at least flabby.
\end{enumerate}\end{prop}

\begin{proof}The first claim follows abstractly from the fact that the Hom
functor~$(\cdot)^T$ has a monomorphism-preserving left adjoint, namely the
product functor~$(\cdot) \times T$. Explicitly, an extension problem as in the
left half of the diagram
\[
  \xymatrix{
    A \ar@{^{(}->}[r]^i\ar[d]_f & B \ar@{-->}@/^/[ld]^{\overline{f}} \\
    I^T
  }
  \qquad
  \xymatrix{
    A \times T \ar@{^{(}->}[r]^{i \times T}\ar[d]_{g \defeq f^t} & B \times T \ar@{-->}@/^/[ld]^{\overline{g}} \\
    I
  }
\]
can be transposed to the extension problem as in the right half. A
solution~$\overline{g}$ gives rise to the solution~$\overline{f}$ of the
original problem by the setting~$\overline{f}(x) = (t \mapsto
\overline{g}(x,t))$.

The injectivity part of the second claim follows entirely analogously,
employing the tensor product instead of the cartesian product.

For the general statement, let~$K$ be a subterminal of~$\Hom_R(T,I)$. Let~$T'$
be the submodule $\{ s \in T \,|\, \text{$s = 0$ or $K$ is inhabited} \}
\subseteq T$ and let~$f : T' \to I$ be the linear map defined as follows:
Let~$s \in T'$. If~$s = 0$, then we set~$f(s) = 0$; if~$K$ is inhabited, then
we set~$f(s) \defeq g(s)$, where~$g$ is an arbitrary element of~$K$. This
association is well-defined. Since~$I$ is injective as a module, there is a
linear extension~$\overline{f} : T \to I$ of~$f$ along the inclusion~$T'
\subseteq T$. If~$K$ is inhabited, this extension is an element of~$K$ as
required.
\end{proof}

%Proposition~\ref{prop:hom-flabby} can be used to give an alternative proof of
%Proposition~\ref{prop:injective-flabby} and to generalize
%Proposition~\ref{prop:injective-flabby} to modules: If~$I$ is an injective set,
%then the set~$I^1 \cong I$ is flabby. If~$I$ is an injective module, then the
%set~$\Hom_R(R,I) \cong I$ is flabby.

\begin{lemma}\label{lemma:set-of-extensions-flabby}
\begin{enumerate}
\item Let~$I$ be an injective set. Let~$i : A \to B$ be an injection.
Let~$f : A \to I$ be an arbitrary map. Then the set of extensions of~$f$ to~$B$
is flabby.
\item Let~$I$ be an injective~$R$-module. Let~$i : A \to B$ be a linear injection.
Let~$f : A \to I$ be an arbitrary linear map. Then the set of linear extensions of~$f$ to~$B$
is flabby.
\end{enumerate}
\end{lemma}

\begin{proof}For the first claim, we set~$X \defeq \{ \bar{f} \in I^B \,|\, \bar{f} \circ i =
f \}$. Let~$K \subseteq X$ be a subterminal. We consider the injectivity diagram
\[ \xymatrix{
  i[A] \cup B' \ar@{^{(}->}[r]\ar[d]_g & B \ar@{-->}@/^/[ld] \\
  I
} \]
where~$B'$ is the set~$\{ s \in B \,|\, \text{$K$ is inhabited} \}$ and the solid
vertical arrow~$g$ is defined in the following way: Let~$s \in i[A] \cup B'$.
If~$s \in i[A]$, we set~$g(s) \defeq f(a)$, where~$a \in A$ is an element such
that~$s = i(a)$. If~$s \in B'$, we set~$g(s) \defeq \bar{f}(s)$,
where~$\bar{f}$ is any element of~$K$. These prescriptions determine a well-defined
map.

Since~$I$ is injective, there exists a dotted map rendering the diagram
commutative. This map is an element of~$X$. If~$K$ is inhabited,
this map is an element of~$K$.

The proof of the second claim is similar. We
set~$X \defeq \{ \bar{f} \in \Hom_R(B,I) \,|\, \bar{f} \circ i =
f \}$. Let~$K \subseteq X$ be a subterminal. We consider the injectivity diagram
\[ \xymatrix{
  i[A] + B' \ar@{^{(}->}[r]\ar[d]_g & B \ar@{-->}@/^/[ld] \\
  I
} \]
where~$B'$ is the submodule~$\{ t \in B \,|\, \text{$t = 0$ or $K$ is
inhabited} \} \subseteq B$ and the solid vertical arrow~$g$ is defined in the following
way: Let~$s \in i[A] + B'$. Then~$s = i(a) + t$ for an element~$a \in A$ and an
element~$t \in B'$. Since~$t \in B'$, $t = 0$ or~$K$ is inhabited. If~$t = 0$,
we set~$g(s) \defeq f(a)$. If~$K$ is inhabited, we set~$g(s) \defeq f(a) +
\bar{f}(s)$, where~$\bar{f}$ is any element of~$K$. These prescriptions
determine a well-defined map.

Since~$I$ is injective, there exists a dotted map rendering the diagram
commutative. This map is an element of~$X$. Furthermore, if~$K$ is inhabited,
then this map is an element of~$K$.
\end{proof}

\begin{prop}\label{prop:set-of-preimages-flabby}
Let~$0 \to M' \xra{i} M \xra{p} M'' \to 0$ be a short exact
sequence of modules. Let~$s \in M''$. If~$M'$ is flabby, then the set of
preimages of~$s$ under~$p$ is flabby.
\end{prop}

\begin{proof}Let~$X \defeq \{ u \in M \,|\, p(u) = s
\}$. Let~$K \subseteq X$ be a subterminal. Since~$p$ is surjective, there is an
element~$u_0 \in X$. The translated set~$K - u_0 \subseteq M$ is still a
subterminal, and its preimage under~$i$ is as well. Since~$M'$ is flabby, there
is an element~$v \in M'$ such that~$i^{-1}[K - u_0] \subseteq \{v\}$. We verify
that~$K \subseteq \{u_0 + i(v)\}$:

Thus let~$u \in K$ be given. Then~$p(u - u_0) = 0$, so by exactness the
set~$i^{-1}[K - u_0]$ is inhabited. It therefore contains~$v$. Thus~$i(v) \in K
- u_0$. Since~$K = \{u\}$, it follows that~$i(v) = u - u_0$, so~$u \in \{u_0 +
i(v)\}$ as claimed.
\end{proof}

Toby Kenney stressed that the notion of an injective set should be regarded as an
interesting strengthening of the constructively rather ill-behaved notion of a
nonempty set~\cite{kenney:injective-choice}. For instance, while the statements
``there is a choice function for every set of nonempty sets'' and even ``there
is a choice function for every set of inhabited sets'' are constructive taboos,
the statement ``there is a choice function for every set of injective sets'' is
constructively neutral. Proposition~\ref{prop:set-of-preimages-flabby} demonstrates
that the notion of a flabby set can be regarded as an interesting intermediate
notion: In the situation of Proposition~\ref{prop:set-of-preimages-flabby}, the
set of preimages is not only not empty or inhabited, but even flabby.

\begin{prop}Let~$0 \to M' \xra{i} M \xra{p} M'' \to 0$ be a short exact
sequence of modules. If~$M'$ and~$M''$ are flabby, so is~$M$.
\end{prop}

\begin{proof}Let~$K \subseteq M$ be a subterminal. Then its image~$p[K] \subseteq M''$
is a subterminal as well. Since~$M''$ is flabby, there is an element~$s \in
M''$ such that~$p[K] \subseteq \{ s \}$.

Since~$p$ is surjective, there is an element~$u_0 \in M$ such that~$p(u_0) =
s$.

The preimage~$i^{-1}[K - u_0] \subseteq M'$ is a subterminal. Since~$M'$ is
flabby, there exists an element~$v \in M'$ such that~$i^{-1}[K - u_0] \subseteq
\{v\}$.

Thus~$K \subseteq \{ u_0 + i(v) \}$.
\end{proof}

Noticeably missing here is a statement as follows: ``Let~$0 \to M' \to M \to
M'' \to 0$ be a short exact sequence of modules. If~$M'$ and~$M$ are flabby, so
is~$M''$.'' Assuming Zorn's lemma in the metatheory, this statement is true in
every topos of sheaves over a locale, but we do not know whether it has an
intuitionistic proof and in fact we surmise that it has not.


\section{Flabby objects}
\label{sect:flabby-objects}

\begin{defn}An object~$X$ of an elementary topos~$\E$ is \emph{flabby} if and
only if the statement~``$X$ is a flabby set'' holds in the stack semantics
of~$\E$.\end{defn}

This definition amounts to the following: An object~$X$ of an elementary
topos~$\E$ is flabby if and only if, for every monomorphism~$K \to A$ and every
morphism~$K \to X$, there exists an epimorphism~$B \to A$ and a morphism~$B
\to X$ such that the following diagram commutes.
\[ \xymatrix{
  K \times_A B \ar@{^{(}->}[r]\ar[d] & B \ar@{-->}@/^/[ld] \\
  X
} \]

Instead of referring to arbitrary stages~$A \in \E$, one can also just refer to
the generic stage: Let~$\P_{\leq1}(X)$ denote the \emph{object of subterminals}
of~$X$; this object is a certain subobject of the powerobject~$\P(X) = [X,\Omega_\E]_\E$.
The subobject~$K_0$ of~$X \times \P_{\leq1}(X)$ classified by the
evaluation morphism~$X \times \P_{\leq1}(X) \to X \times \P(X) \to \Omega_\E$
is the \emph{generic subterminal} of~$X$. The object~$X$ is flabby if and only
if there exists an epimorphism~$B \to \P_{\leq1}(X)$ and a morphism~$B \to X$
such that the following diagram commutes.
\[ \xymatrix{
  K_0 \times_{\P_{\leq1}(X)} B \ar@{^{(}->}[r]\ar[d] & \P_{\leq1}(X) \ar@{-->}@/^/[ld] \\
  X
} \]

\begin{prop}\label{prop:basic-properties-of-flabby-objects}
Let~$X$ and~$T$ be objects of an elementary topos~$\E$.
\begin{enumerate}
\item If~$X$ is flabby, so is~$X \times T$ as an object of~$\E/T$.
\item The converse holds if the unique morphism~$T \to 1$ is an epimorphism.
\end{enumerate}
\end{prop}

\begin{proof}This holds for every property which can be defined
in the stack semantics~\cite[Lemma~7.3]{shulman:stack-semantics}.
\end{proof}

\begin{prop}\label{prop:flabby-sheaves-objects}
Let~$F$ be a sheaf on a topological space~$X$ (or a locale).
Then~$F$ is flabby as a sheaf if and only if~$F$ is flabby as an object of the
sheaf topos~$\Sh(X)$.
\end{prop}

\begin{proof}The proof is routine; we only verify the ``only if'' direction.
Let~$F$ be flabby as a sheaf. It suffices to verify the defining condition for stages
of the form~$A = \Hom(\cdot,U)$, where~$U$ is an open of~$X$. A monomorphism~$K
\to A$ then amounts to an open~$V \subseteq U$ (the union of all opens on
which~$K$ is inhabited). A morphism~$K \to F$ amounts to a section~$s \in
F(V)$. Since~$F$ is flabby as a sheaf, there is an open covering~$X =
\bigcup_{i \in I} V_i$ such that, for all~$i$, the section~$s$ can be extended
to a section~$s_i$ of~$V \cup V_i$. The desired epimorphism is~$B \defeq
\coprod_i \Hom(\cdot,(V \cup V_i) \cap U) \to A$, and the desired morphism~$B
\to X$ is given by the sections~$s_i|_{(V \cup V_i) \cap U}$.
\XXX{choicefree}

As stated, the argument in the previous paragraph requires the axiom of choice
to pick the extensions~$s_i$; this can be avoided by a standard trick of
expanding the index set of the coproduct to include the choices: We redefine $B \defeq
\coprod_{(i,t) \in I'} \Hom(\cdot, (V \cup V_i) \cap U)$, where~$I' = \{ (i \in
I, t \in F(V \cup V_i)) \,|\, t|_V = s \}$ and define the morphism~$B \to X$ on
the~$(i,t)$-summand by~$t|_{(V \cup V_i) \cap U}$.
\end{proof}

\begin{ex}The object of Dedekind reals of a topos is in general not flabby. For
instance, in the case of the topos of sheaves over the real line~$\mathbb{R}^1$, the
object of Dedekind reals is the sheaf~$\mathcal{C}$ of continuous (Dedekind-)real valued
functions~\XXX{cite}. The section~$1/x$ cannot be extended to opens containing
the origin.\end{ex}

\begin{prop}\label{prop:global-elements}
Let~$X$ be a flabby object of a localic topos~$\E$. If
Zorn's lemma is available in the metatheory, then~$X$ possesses a global element (a morphism~$1 \to X$).
\end{prop}

\begin{proof}This is a restatement of the discussion following
Definition~\ref{defn:flabby-sheaf}.
\end{proof}

\begin{rem}\label{rem:flabby-global}
Some condition on the topos is necessary for flabby objects to
possess global elements. An example is given by the~$G$-set~$G$ (with the
translation action), considered as an object of the topos~$BG$ of~$G$-sets,
where~$G$ is a nontrivial group. This object is flabby (because it is inhabited
and~$BG$ is a Boolean topos, assuming the law of excluded middle in the
metatheory), but it does not have any global elements.
\end{rem}

\begin{prop}\label{prop:pushforward-of-flabby-objects}
Let~$f : \F \to \E$ be a geometric morphism. If~$f_*$ preserves epimorphisms,
then~$f_*$ preserves flabby objects.\end{prop}

\begin{proof}Let~$X \in \F$ be a flabby object.
Let~$k : K \to A$ be a monomorphism in~$\E$ and let~$x : K \to f_*(X)$ be an
arbitrary morphism. Without loss of generality, we may assume that~$A$ is the
terminal object~$1$ of~$\E$. Then~$f^*(k) : f^*(K) \to 1$ is a monomorphism
in~$\F$ and the adjoint transpose~$x^t : f^*(K) \to X$
is a morphism in~$\F$. Since~$X$ is flabby, there is an epimorphism~$B \to 1$
in~$\F$ and a morphism~$y : B \to X$ such that the morphism~$f^*(K) \times B
\to X$ factors over~$y$. Hence~$x$ factors over~$f_*(y) : f_*(B) \to f_*(X)$.
We conclude because the morphism~$f_*(B) \to f_*(1)$ is an epimorphism by
assumption.
\end{proof}

The assumption on~$f_*$ of Proposition~\ref{prop:pushforward-of-flabby-objects}
is for instance satisfied if~$f$ is a local geometric morphism.

\begin{rem}Pullbacks of flabby objects along geometric morphisms are usually
not flabby. For instance, constant sheaves can fail to be flabby
(Remark~\ref{rem:constant-flabby}) but arise as pullbacks along geometric
morphisms to the topos~$\Set$, in which most objects are flabby (assuming the
law of excluded middle).\end{rem}

\begin{defn}An object~$I$ of an elementary topos~$\E$ is \emph{externally
injective} if and only if for every monomorphism~$A \to B$ in~$\E$, the canonical
map~$\Hom_\E(B,I) \to \Hom_\E(A,I)$ is surjective. It is \emph{internally
injective} if and only if for every monomorphism~$A \to B$ in~$\E$, the canonical
morphism~$[B,I] \to [A,I]$ between Hom objects is an epimorphism in~$\E$.
\end{defn}

If~$R$ is a ring in an elementary topos~$\E$, a similar definition can be given
for~$R$-modules in~$\E$, referring to the set respectively the object of
linear maps. The condition for an object to be internally injective can be
rephrased in various ways. The following proposition lists five of these
conditions. The equivalence of the first four is due to
Roswitha Harting~\cite{harting:locally-injective}.

\begin{prop}\label{prop:notions-of-internal-injectivity}
Let~$\E$ be an elementary topos. Then the following statements about an
object~$I \in \E$ are equivalent.
% Take care if numbering changes (is referenced below).
\begin{enumerate}
\item[(1)] $I$ is internally injective.
\item[(1')] For every morphism $p : A \to 1$ in $\E$, the object $p^*(I)$ has property~(1)
as an object of $\E/A$.
\item[(2)] The functor~$[\cdot, I] : \E^\op \to \E$ maps monomorphisms in $\E$
to morphisms for which every global element of the target locally (after change of
base along an epimorphism) possesses a preimage.
\item[(2')] For every morphism $p : A \to 1$ in $\E$, the object $p^*(I)$ has property~(2)
as an object of $\E/A$.
\item[(3)] The statement~``$I$ is an injective set'' holds in the stack
semantics of~$\E$.
\end{enumerate}
\end{prop}

\begin{proof}
The implications (1)~$\Rightarrow$~(2), (1')~$\Rightarrow$~(2'),
(1')~$\Rightarrow$~(1) and (2')~$\Rightarrow$~(2) are trivial.

The equivalence (1')~$\Leftrightarrow$~(3) follows directly from the
interpretation rules of the stack semantics.

The implication (2)~$\Rightarrow$~(2') employs the
extra left adjoint $p_! : \E/A \to \E$ of $p^* : \E
\to \E/A$~(which maps an object~$(X \to A)$ to~$X$), as in the usual proof that
injective sheaves remain injective when
restricted to smaller open subsets: We have that $p_* \circ [\cdot, p^*(I)]_{\E/A}
\cong [\cdot, I]_\E \circ p_!$, the functor $p_!$ preserves monomorphisms, and one
can check that $p_*$ reflects the property that global elements locally possess
preimages. Details are in~\cite[Thm.~1.1]{harting:locally-injective}.\footnote{Harting formulates
her theorem for abelian group objects, and has to assume that~$\E$ contains a
natural numbers object to ensure the existence of an abelian version of~$p_!$.}

The implication (2')~$\Rightarrow$~(1') follows by performing an extra change of
base, exploiting that any non-global element becomes a global element after a suitable
change of base.
\end{proof}

Let~$R$ be a ring in~$\E$. Then the analogue of
Proposition~\ref{prop:notions-of-internal-injectivity} holds for~$R$-modules
in~$\E$, if~$\E$ is assumed to have a natural numbers object. The extra
assumption is needed in order to construct the left adjoint~$p_! :
\Mod_{\E/A}(R \times A) \to \Mod_\E(R)$. Phrased in the internal language, this
adjoint maps a family~$(M_a)_{a \in A}$ of~$R$-modules to the direct
sum~$\bigoplus_{a \in A} M_a$. Details on this construction, phrased in the
language of sets but interpretable in the internal language, can for instance
be found in~\cite[page~54]{mines-richman-ruitenburg:constructive-algebra}.

Somewhat surprisingly, and in stark contrast with the situation for internally
projective objects (which are defined dually), internal injectivity coincides
with external injectivity for localic toposes. In the special case of sheaves
of abelian groups, this result is due to Roswitha
Harting~\cite[Proposition~2.1]{harting:locally-injective}.

\begin{thm}\label{thm:injectivity-external-internal}
Let~$I$ be an object of an elementary topos~$\E$. If~$I$ is externally
injective, then~$I$ is also internally injective. The converse holds if~$\E$ is
localic and Zorn's lemma is available in the metatheory.
\end{thm}

\begin{proof}Let~$I$ be an object
which is externally injective. Then~$I$ satisfies Condition~(2) in
Proposition~\ref{prop:notions-of-internal-injectivity}, even without having to
pass to covers.

For the converse direction, let~$I$ be an internally
injective object. Let~$i : A \to B$ be a monomorphism in~$\E$ and let~$f :
A \to I$ be an arbitrary morphism. We want to show that there exists an
extension $B \to I$ of~$f$ along~$i$. To this end, we consider the object of
such extensions, defined by the internal expression
\[ F \defeq \{ \bar{f} \in [B,I] \,|\, \bar{f} \circ i = f \}. \]
Global elements of~$F$ are extensions of the kind we are looking for.
By Lemma~\ref{lemma:set-of-extensions-flabby}(1), interpreted in~$\E$, this object is flabby.
By Proposition~\ref{prop:global-elements}, it has a global element.
\end{proof}

The analogue of Theorem~\ref{thm:injectivity-external-internal} for modules
holds as well, if~$\E$ is assumed to have a natural numbers object. The proof
carries over word for word, only referencing
Lemma~\ref{lemma:set-of-extensions-flabby}(2) instead of
Lemma~\ref{lemma:set-of-extensions-flabby}(1).
It seems that Roswitha Harting was not aware of this generalization, even though she did show that
injectivity of sheaves of modules over topological spaces is a local
notion~\cite[Remark~5]{harting:remark}, as she
(mistakenly) states in~\cite[page~233]{harting:remark} that ``the notions of
injectivity and internal injectivity do not coincide'' for modules.

It is worth noting that, because the internal language machinery was at that
point not as well-developed as it is today, Harting had to go to considerable
length to construct internal direct sums of abelian group
objects~\cite{harting:coproduct}, and in order to verify that taking internal
direct sums is faithful she appealed to Barr's
metatheorem~\cite[Theorem~1.7]{harting:effacements}. Nowadays we can verify
both statements by simply carrying out an intuitionistic proof in the case of
the topos of sets and then trusting the internal language to obtain the
generalization to arbitrary elementary toposes with a natural numbers object.

Since we were careful in Section~\ref{sect:flabby-sets} to use the law of
excluded middle and the axiom of choice only where needed, most results of that
section carry over to flabby and internally injective objects. Specifically, we
have:

\begin{scholium}\label{scholium:properties-of-flabby-objects}
For every elementary topos~$\E$:
\begin{enumerate}
\item Every object embeds canonically into an internally injective object.
\item (If~$\E$ has a natural numbers object.) The underlying unstructured
object of an internally injective module is internally injective.
\item Every internally injective object is flabby.
\item Every object embeds canonically into a flabby object.
\item Every internal module embeds canonically into an internal module which is flabby.
\item The terminal object is flabby. The binary product of flabby objects is flabby.
\item Let~$I$ be an internally injective object. Let~$T$ be an arbitrary
object. Then~$[T,I]$ is a flabby object.
\item (If~$\E$ has a natural numbers object.) Let~$I$ be an internally
injective~$R$-module. Let~$T$ be an arbitrary~$R$-module. Then~$[T,I]_R$, the
subobject of the internal Hom consisting only of the linear maps, is a flabby
object.
\item Let~$0 \to M' \to M \to M'' \to 0$ be a short exact sequence
of~$R$-modules in~$\E$. If~$M'$ and~$M''$ are flabby objects, so is~$M$.
\end{enumerate}
\end{scholium}

\begin{proof}We established the analogous statements for sets and modules purely
intuitionistically in Section~\ref{sect:flabby-sets}, and the stack semantics
is sound with respect to intuitionistic logic.
\end{proof}

\begin{scholium}\label{scholium:exact-as-presheaves}
Let~$0 \to M' \to M \to M'' \to 0$ be a short exact sequence
of~$R$-modules in a localic topos~$\E$. Let~$M'$ be a flabby object.
Assuming Zorn's lemma in the metatheory, the induced sequence~$0 \to
\Gamma(M') \to \Gamma(M) \to \Gamma(M'') \to 0$ of~$\Gamma(R)$-modules is exact,
where~$\Gamma(X) = \Hom_\E(1,X)$.\end{scholium}

\begin{proof}We only have to verify exactness at~$\Gamma(M'')$, so let~$s \in
\Gamma(M'')$. Interpreting Proposition~\ref{prop:set-of-preimages-flabby}
in~$\E$, we see that the object of preimages of~$s$ is flabby. Since~$\E$ is
localic, this object is a flabby sheaf; since Zorn's lemma is available, it
possesses a global element. Such an element is the desired preimage of~$s$
in~$\Gamma(M)$.\end{proof}

If~$\E$ is not necessarily localic or Zorn's lemma is not available, only a
weaker substitute for Scholium~\ref{scholium:exact-as-presheaves} is available:
Given~$s \in \Gamma(M'')$, the object of preimages of~$s$ is flabby. In
particular, given any point of~$\E$, we can extend any local preimage of~$s$ to
a preimage which is defined on an open neighborhood of that point. We believe
that there are situations in which this weaker substitute is good enough,
similar to how in constructive algebra often the existence of a sufficiently
large field extension is good enough where one would classically blithely pass
to an algebraic closure.

\begin{rem}We can dispose the reliance on Zorn's lemma in
Scholium~\ref{scholium:exact-as-presheaves} if instead~$\E$ is compact. This is
because XXX ... finitely many local preimages ... patch together ... by the
following internal observation ...
\end{rem}

\begin{rem}A direct generalization of the traditional notion of a flabby sheaf, as
opposed to our reimagining in Definition~\ref{defn:flabby-sheaf}, to
elementary toposes is the following. An object~$X$ of an elementary topos~$\E$
is \emph{strongly flabby} if and only if, for every monomorphism~$K \to 1$
in~$\E$, every morphism~$K \to X$ lifts to a morphism~$1 \to X$.

One can verify, purely intuitionistically, that a sheaf~$F$ on a space~$T$ is
flabby in the traditional sense if and only if~$F$ is a strongly flabby object
in~$\Sh(T)$, if and only if~$F$ is a flabby object in the presheaf
topos~$\PSh(T)$.

The notion of strongly flabby objects is, however, not local (in the same sense
that the notion of flabby objects is, as stated in
Proposition~\ref{prop:basic-properties-of-flabby-objects})
and therefore cannot be characterized in the internal language. A specific
example is the~$G$-set~$G$ (with the translation action) as in
Remark~\ref{rem:flabby-global}.
This object is not strongly flabby, since the morphism~$\emptyset \to 1$ does
not lift, but its pullback to the slice~$BG/G \simeq \Set$ is (assuming
the law of excluded middle in the metatheory), and the unique morphism~$G \to
1$ is indeed an epimorphism.
\end{rem}


\section{Higher direct images as internal sheaf cohomology}
\label{sect:higher-direct-images}

Let~$X$ be a locale and let~$Y$ be a locale over~$X$, that is, a morphism~$f : Y \to X$ of locales. By the fundamental
relation between locales and topological spaces, this situation arises
for instance, when given a sober topological space and a topological
space over it, as is often the case in algebraic topology or algebraic
geometry. Let a sheaf~$\O_Y$ of rings on~$Y$ be given. Then the
traditional way to define the \emph{higher direct images} of a sheaf~$E$
of~$\O_Y$-modules is to pick an injective resolution~$0 \to E \to I^\bullet$
and set~$R^n f_*(E) \defeq H^n(f_*(I^\bullet))$~\cite[Tag~01DZ]{stacks-project}.

Assuming Zorn's lemma, there are enough injective sheaves of modules so
that this recipe can be carried out. The resulting sheaf
of~$f_*\O_Y$-modules is well-defined in the following sense: Given a further
injective resolution~$0 \to E \to J^\bullet$, there is up to homotopy precisely
one morphism~$I^\bullet \to J^\bullet$ compatible with the identity on~$E$, and
this morphism induces an isomorphism on cohomology.

Higher direct images are pictured as a ``relative'' version of sheaf
cohomology. Due to the result that injectivity of sheaves of modules can be
characterized in the internal language, we can give a precise
rendering of this slogan: \emph{Higher direct images are
internal sheaf cohomology.}

The details are as follows. The over-locale~$Y$ corresponds to a locale~$I(Y)$ internal to~$\Sh(X)$, in
such a way that the category of internal sheaves over this internal locale
coincides with~$\Sh(Y)$~\cite[Scholium~C1.6.4]{johnstone:elephant}; in particular, a given sheaf~$E$ of~$\O_Y$-modules can
be regarded as a sheaf over~$I(Y)$. Under this equivalence, the morphism~$f : Y
\to X$ corresponds to the unique morphism~$I(Y) \to \mathrm{pt}$ to the
internal one-point locale. Hence it makes sense to construct, from the internal
point of view of~$\Sh(X)$, the sheaf cohomology~$H^n(I(Y), E)$ of~$E$.

Usually one would not expect an internal construction which depends on
arbitrary choices to yield a globally-defined sheaf over~$X$ -- following the
definition of the stack semantics we only obtain a family of sheaves defined on
members of some open covering of~$X$; but we verify in
Theorem~\ref{thm:higher-direct-images-as-internal-sheaf-cohomology} below that in our
case, it does, and that the resulting sheaf coincides with~$R^n f_*(E)$.

\begin{lemma}\label{lemma:notions-of-internal-injectivity}
Let~$Y$ be a ringed locale over a locale~$X$. Let~$J$ be a sheaf of modules over~$Y$.
Assuming Zorn's lemma in the metatheory, the following statements are equivalent:
\begin{enumerate}
\item $J$ is an injective sheaf of modules.
\item From the point of view of~$\Sh(Y)$, $J$ is an injective module.
\item From the point of view of~$\Sh(X)$, $J$ is an injective module from the
point of view of~$\Sh(I(Y))$.
\item From the point of view of~$\Sh(X)$, $J$ is an injective sheaf of modules
on~$I(Y)$.
\end{enumerate}
\end{lemma}

\begin{proof}The equivalence of the first two statements is by
Theorem~\ref{thm:injectivity-external-internal}. The equivalence~$\text{(2)}
\Leftrightarrow \text{(3)}$ is by the idempotency of the stack semantics:
$\Sh(Y) \models \varphi$ if and only if~$\Sh(X) \models (\Sh(I(Y)) \models
\varphi)$. (Shulman stated and proved a restricted version of this idempotency property
in his original paper on the stack semantics~\cite[Lemma~7.20]{shulman:stack-semantics}.
A proof of the general case is slightly less
accessible~\cite[Lemma~1.20]{blechschmidt:master}.) The equivalence~$\text{(3)}
\Leftrightarrow \text{(4)}$ is by interpreting
Theorem~\ref{thm:injectivity-external-internal} internally to~$\Sh(X)$. This
requires Zorn's lemma to hold internally to~$\Sh(X)$; this is indeed the case
since we assume Zorn's lemma in the metatheory and the validity of Zorn's
lemma passes from the metatheory to localic
toposes~\cite[Proposition~D4.5.14]{johnstone:elephant}.
\end{proof}

\begin{thm}\label{thm:higher-direct-images-as-internal-sheaf-cohomology}
Let~$f : Y \to X$ be a ringed locale over a locale~$X$. Let~$E$ be a sheaf of modules over~$Y$.
Assuming Zorn's lemma in the metatheory, the expression~``$H^n(I(Y), E)$''
of the internal language of~$\Sh(X)$ denotes a globally-defined sheaf
on~$X$, and this sheaf coincides with~$R^n f_*(E)$.\end{thm}

\begin{proof}By Lemma~\ref{lemma:notions-of-internal-injectivity} and by the
fact that every sheaf of modules over~$Y$ admits an injective resolution, every sheaf
of modules over~$I(Y)$ admits an injective resolution from the point of view
of~$\Sh(X)$. Hence we can, internally to~$\Sh(X)$, carry out the construction of~$H^n(I(Y), E)$.
Externally, this construction yields an open covering of~$X$ such that we have, for each
member~$U$ of that covering
\begin{itemize}
\item a sheaf~$M$ over~$U$,
\item a module structure on~$M$,
\item a resolution~$0 \to E|_{f^{-1}U} \to I^\bullet$ by sheaves of modules which are
internally and hence externally injective and
\item data exhibiting~$M$ as the~$n$-th cohomology
of~$(f|_{f^{-1}U})_*(I^\bullet)$.
\end{itemize}
On intersections of such opens~$U$ and~$U'$, there is exactly one
isomorphism~$M|_{U \cap U'} \to M'|_{U \cap U'}$ of sheaves of modules induced by
a morphism of resolutions which is compatible with the identity on~$E$.
Hence the cocycle condition for these isomorphisms is satisfied, ensuring that
the individual sheaves~$M$ glue to a globally-defined sheaf of modules on~$X$.
(The individual injective resolutions need not glue to a global injective
resolution.)

The claim that this sheaf coincides with~$R^n f_*(E)$ follows from the fact
that we can pick as internal resolution of~$E$ (considered as a sheaf
over~$I(Y)$) the particular injective resolution of~$E$ (considered as a sheaf
over~$Y$) used to define~$R^n f_*(E)$.
\end{proof}

The internal characterization provided by
Theorem~\ref{thm:higher-direct-images-as-internal-sheaf-cohomology} gives, as a
simple application, a logical explanation of the basic fact that higher direct images along
the identity~$\mathrm{id} : X \to X$ vanish: From the internal point of view
of~$\Sh(X)$, the over-locale~$X$ corresponds to the one-point locale, and the
higher cohomology of the one-point locale vanishes.

In algebraic geometry, the internal characterization can be used to immediately
deduce the explicit description of the higher direct images of Serre's twisting
sheaves along the projection~$\mathbb{P}^n_S \to S$, where~$S$ is an arbitrary
base scheme (or even base locally ringed locale), from a computation of the
cohomology of projective~$n$-space. Background on carrying out scheme
theory internally to a topos is given in~\cite[Section~12]{blechschmidt:phd}.


\section{Flabby objects in the \effective topos}
\label{sect:in-eff}

The notion of flabby objects originates from the notion of flabby sheaves and is
therefore closely connected to Grothendieck toposes. Hence it is instructive
to study flabby objects in elementary toposes which are not Grothendieck
toposes, away from their original conceptual home. We begin this study with
establishing the following observation on flabby objects in the \effective
topos. We follow the terminology of Martin Hyland's survey on the \effective
topos~\cite{hyland:effective-topos} and refer to
\cite{oosten:realizability,moerdijk-oosten:topos-theory,phoa:effective,bauer:c2c} for more background.

\begin{prop}\label{prop:flabby-effective-sets}
The only object in the \effective topos which is both flabby and \effective is the
singleton object.
\end{prop}

% XXX In Eff gilt: Jede negneg-separierte abelsche Gruppe bettet in eine injektive
% ein. Nämlich A --> ΔΓA --> ΔI für ΓA --> I Einbettung in eine injektive
% Gruppe.

The intuitive
reason for why Proposition~\ref{prop:flabby-effective-sets} holds is the
following. Let~$X$ be a flabby object in the \effective topos. Then there is a
procedure which computes for any (realizer of a) subterminal~$K \subseteq X$ a
(realizer of an) element~$x_K$
such that~$K \subseteq \{ x_K \}$. However, realizers for subsets are not very
informative; the procedure cannot ask questions such as ``is a
given element of~$X$ contained in~$K$?'' nor query its input in any way.
Hence~$x_K$ will actually be the same element for any subterminal~$K$.
Metaphorically speaking, a procedure witnessing flabbiness has to
conjure elements out of thin air.

This issue does not manifest with objects~$X$ which are not \effective objects
such as double-negation sheaves. Realizers for elements of such objects are
themselves not very informative; for those, a procedure witnessing flabbiness
only has to turn one kind of non-informative realizers into another kind. This
is why, in line with Theorem~\ref{thm:enough-flabby-modules}, it is still true
in the \effective topos that any module of the \effective topos embeds into a
flabby module.

\begin{proof}[Proof of Proposition~\ref{prop:flabby-effective-sets}]
Let~$X$ be an object of the \effective topos. Then the object~$P_{\leq1}(X)$ of
subterminals of~$X$ is a uniform object in the sense
of~\cite[Section~3.4]{moerdijk-oosten:topos-theory}, being a retract of the
uniform object~$\P(X)$ by the surjection
\[ \P(X) \longrightarrow \P_{\leq1}(X),\
  M \longmapsto \{ x \in X \,|\, M = \{ x \} \}. \]
Hence if~$X$ is \effective, the uniformity principle~\cite[Proposition~15.1]{hyland:effective-topos}
\[ \forall K \in \P_{\leq1}(X)\_
  \exists x \in X\_ K \subseteq \{x\} \quad\Longrightarrow\quad
  \exists x_0 \in X\_ \forall K \in \P_{\leq1}(X)\_ K \subseteq \{x_0\} \]
applies. Thus, if~$X$ is \effective and flabby, there is
an element~$x_0 \in X$ such that~$K \subseteq \{x_0\}$ for any subterminal~$K$
of~$X$. The conclusion follows by considering, for any elements~$a,b \in X$,
the subterminals~$\{a\}$ and~$\{b\}$.
\end{proof}

\begin{rem}The analogue of Proposition~\ref{prop:flabby-effective-sets} is true
as well for the realizability topos constructed using infinite time Turing
machines~\cite{bauer:injection,hamkins-lewis:ittm} and indeed for any
realizability topos, with the same proof, as power objects are always
uniform~\cite{johnstone:review-oosten}.\end{rem}


\section{Conclusion}
\label{sect:conclusion}

We originally set out to develop an intuitionistic account of Grothendieck's
sheaf cohomology. Čech methods can be carried out constructively, and
there are constructive accounts of special cases, resulting even in
efficient-in-practice algorithms~\cite{barakat-lh:homalg,barakat-lh:ext},
but there is no established general framework for sheaf
cohomology which would work in an intuitionistic metatheory.

The main obstacle preventing Grothendieck's theory of derived functors to be interpreted
constructively is its reliance on injective resolutions. It is known that in
the absence of Zorn's lemma, much less in a purely intuitionistic
context, there might not be any nontrivial injective abelian
group~\cite{blass:inj-proj-axc}.

Can this issue be remedied by employing flabby resolutions
instead of injective ones? Classically, it is known that they can be used in
their stead, and moreover George Kempf developed the foundations of the
cohomology of quasicoherent sheaves on this premise~\cite{kempf:cohomology}. There are, however, three issues with this
suggestion.

Firstly, all proofs known to us that flabby sheaves are
acyclic for the global sections functor require Zorn's lemma. This problem might
be mitigated by relying on the substitute property discussed
following Scholium~\ref{scholium:exact-as-presheaves}, but still acyclitity in
the usual sense will most likely not be attainable.

Secondly, while we have shown in Theorem~\ref{thm:enough-flabby-modules} that
enough flabby envelopes exist constructively, the analysis in Section~\ref{sect:in-eff}
demonstrates that the presented envelopes are in a sense deeply uncomputable.
Intuitionistic Zermelo--Fraenkel set theory~\cite{crosilla:cst-izf} and the
type theory of toposes~\cite{maietti:modular} do verify that they exist, but
only by virtue of excessive reliance on powersets, precluding concrete
computations; the construction is highly \emph{impredicative}, not meeting the
bar of \emph{predicative mathematics}~\cite{crosilla:predicativity}.

As a result, thirdly, flabby resolutions do not generalize from toposes to the
setting of \emph{arithmetic universes}, the predicative cousins of toposes
introduced by André Joyal which have recently been an important object of
consideration by Milly Maietti and
Steve Vickers~\cite{maietti:au,maietti-vickers:induction,vickers:sketches}.
These are not only interesting on their own and as a convenient foundation
for ensuring predicativity (and hence computability in a strict sense), but also
because they can be used to obtain base-independent proofs for the topos
case~\cite{vickers:classifying,hazratpour:phd}: \emph{Computing cohomology of
concrete spaces should not require fixing a universe of sets first.}

We currently believe that it is not possible to give a constructive account of
a global cohomology functor which would associate to any sheaf of modules its
cohomology. However, it should be possible to do so for a restricted class of
sheaves, while still preserving the good formal properties expected from
derived functors. To this end, a mix of the approaches using
pointwise Kan extensions, as cogently argued for by Emily
Riehl~\cite[Chapter~2]{riehl:cathtpy} (see also~\cite{hinich:what-is,maltsiniotis:adjunction}), and ind-objects as presented in the Stacks
Project~\cite[Tag~05S7]{stacks-project} seems promising.

This framework is sufficiently flexible to not demand derived functors to be
defined everywhere. Rather, they will be defined just for those objects for
which we happen to have a suitable resolution. Only classically, by using
injective or flabby sheaves, can we pretend that every object has such a
resolution. We hope to report on details in future work. Can perhaps even the
rare cases where injective resolutions exceptionally do have some link to
computations, as on projective spaces~\cite{huang:cohomology}, be given a
proper constructive and predicative home?

\printbibliography
\enlargethispage{1em}

\appendix

\section{On the existence of enough injective modules}
\label{sect:enough-injective-modules}

\begin{defn}\begin{enumerate}
\item An abelian group~$I$ is \emph{divisible} if and only if for every
element~$x \in I$ and every natural number~$n \geq 1$, there is an element~$y \in I$
such that~$ny = x$.
\item An abelian group~$I$ \emph{satisfies the Baer condition} if and only if,
for every ideal~$\aaa \subseteq \ZZ$, every ($\ZZ$-)linear map~$\aaa \to I$
admits an extension along the inclusion to a linear map~$\ZZ \to I$.
\end{enumerate}
\end{defn}

Trivially, every group satisfying the Baer condition is also divisible.

\begin{prop}
\label{prop:embed1}
\begin{enumerate}
\item Assuming the law of excluded middle, all divisible abelian groups satisfy
the Baer condition.
\item Assuming Zorn's lemma, all abelian groups which satisfy the Baer condition
are injective.
\end{enumerate}
\end{prop}

\begin{proof}Assuming the law of excluded middle, the only ideals of~$\ZZ$ are
the principal ideals~$(n)$ where~$n \geq 0$. For~$n = 0$ existence of
extensions is trivial and for~$n > 0$ existence of extensions follow from
divisibility.

Assuming Zorn's lemma, let~$I$ be an abelian group satisfying the Baer
condition, let~$i : A \to B$ be a linear injection and let~$f : A \to I$ be a
linear map. The poset of partial extensions of~$f$ contains suprema of chains and
hence a maximal partial extension~$f_0 : A_0 \to I$ with~$A \subseteq A_0
\subseteq B$.

To verify that~$A_0 = B$, let an element~$x \in B$ be given.
By the Baer condition, the linear map~$g : (A_0 : x) \to I$ defined on the
ideal~$(A_0 : x) = \{ n \in \ZZ \,|\, nx \in A_0 \} \subseteq \ZZ$ given by~$g(n) =
f_0(nx)$ can be extended to a linear map~$\overline{g} : \ZZ \to I$.
Hence~$f_0$ can be extended to the map~$A_0 + (x) \to I$ given by~$u + nx
\mapsto f_0(u) + \overline{g}(n)$. By maximality~$A_0 + (x) = A_0$, hence~$x \in A_0$.
\end{proof}

\begin{lemma}\label{lemma:q-mod-z}
Let~$\nabla$ be a modal operator such
that~$\nabla(\varphi \vee (\varphi \Rightarrow \nabla\bot))$ and
let~$(\cdot)^+$ denote the plus construction with respect to~$\nabla$. Then:
\begin{enumerate}
\item The sheafification~$(\QQ/\ZZ)^{++}$ satisfies the Baer condition.
\item Assuming Zorn's lemma, given an element~$x$ of an abelian group~$A$,
there is a linear map~$g : A \to (\QQ/\ZZ)^{++}$ such that~$g(x) = 0
\Rightarrow \nabla(x=0)$.
\end{enumerate}
\end{lemma}

\begin{proof}Let~$i : \aaa \hookrightarrow \ZZ$ be the inclusion of an ideal
and let~$f : \aaa \to (\QQ/\ZZ)^{++}$ be a linear map as in the solid part of the following
commutative diagram:
\[ \xymatrix{
  \aaa \ar@{^{(}->}[r]^i\ar[d]\ar@/_2pc/[dd]_f & \ZZ \ar[d] \\
  \aaa^{++} \ar@{^{(}->}[r]^{i^{++}}\ar@{-->}[d] & \ZZ^{++}\ar@{-->}@/^/[ld]^{f'} \\
  (\QQ/\ZZ)^{++}
} \]
The given map~$f$ factors (uniquely) over the map~$\aaa \to \aaa^{++}$
since~$(\QQ/\ZZ)^{++}$ is a sheaf. The map~$i^{++}$ is still injective as
sheafification is exact. (The two top vertical maps need not be injective,
as~$\ZZ$ need not be separated for~$\nabla$.) Hence we are reduced to an
extension problem in the subtopos of~$\nabla$-sheaves, which is solvable
because this subtopos is Boolean (Proposition~\ref{prop:embed1}(1)). More
explicitly, the extension~$f'$ maps~$1 \in \ZZ$ to the gluing of
\begin{multline*}
  \{ 0 \,|\, \forall x \in \aaa. \nabla(x = 0) \} \cup {}\\
  \{ \tfrac{1}{d} v \,|\, d \geq 1, v \in [0,1),
    f(d) = \underline{v}, \forall x \in \aaa. \nabla(x \in \aaa)
    \Leftrightarrow \nabla(\exists n \in \ZZ. x = nd) \}.
\end{multline*}

For the second part, let~$A$ be an abelian group and let~$x \in A$ be an
element. Let~$i : \ZZ/(0:x) \to A$ be the linear map with~$[1] \mapsto x$,
where~$(0:x)$ is the ideal~$\{ n \in \ZZ \,|\, nx = 0 \}$. In the Boolean
subtopos of~$\nabla$-sheaves, this ideal is either the zero ideal or the
principal ideal~$(d)$ of some positive generator. Let~$h : \ZZ/(0:x) \to
(\QQ/\ZZ)^{++}$ be the linear map sending~$[1]$ to the respective gluing such
that in the first case~$h([1]) = [\tfrac{1}{2}]$ and in the second case~$h([1])
= [\frac{1}{d}]$. Assuming Zorn's lemma such that
Proposition~\ref{prop:embed1}(2) applies, there is an extension~$\overline{h} :
A \to (\QQ/\ZZ)^{++}$. This map is the desired map. If~$\overline{h}(x) =
h([1]) = 0$, then in the subtopos~$d = 1$, hence~$\nabla(x = 0)$.
\end{proof}

\begin{prop}\label{prop:embed2}
\begin{enumerate}
\item Every abelian group embeds canonically into a divisible abelian group.
\item Every abelian group maps canonically to an abelian group functionally satisfying the Baer
condition. Assuming Zorn's lemma, this map is an embedding.
\end{enumerate}
\end{prop}

\begin{proof}For the first part, an abelian group~$A$ embeds into the
divisible group~$\QQ\langle A \rangle / K$, where~$\QQ\langle A \rangle$ is the
underlying abelian group of the free~$\QQ$-vector space on~$A$ and~$K$ is the
kernel of the canonical map~$\ZZ\langle A \rangle \to A$.

The second part is more involved and requires, similar to the proof of
Theorem~\ref{thm:enough-flabby-modules}, a strengthening of the Baer condition
where the required extensions are explicitly given. As a first step, let~$A$ be
an abelian group which is separated for a modal operator~$\nabla$ such
that~$\nabla(\varphi \vee (\varphi \Rightarrow \nabla\bot))$. The codomain of
the canonical map
\[ A \longrightarrow \prod_{g : A \to (\QQ/\ZZ)^{++}} (\QQ/\ZZ)^{++},\
  x \longmapsto (g(x))_g \]
is a product of abelian groups which functionally satisfy the Baer condition by
Lemma~\ref{lemma:q-mod-z}(1) and hence satisfies the Baer condition itself.
Assuming Zorn's lemma, this map is injective by Lemma~\ref{lemma:q-mod-z}(2)
and~$\nabla$-separatedness of~$A$.

For the general case, let an abelian group~$A$ be given. Denoting
by~$(\cdot)^{+_x}$ the plus construction with respect to the modal
operator~$\nabla_x$ with~$\nabla_x\varphi \defeqv ((\varphi \Rightarrow x
= 0) \Rightarrow x = 0)$, the codomain of the canonical map
\[ A \longrightarrow \prod_{x \in A} \prod_{g : A \to (\QQ/\ZZ)^{+_x+_x}}
(\QQ/\ZZ)^{+_x+_x} \]
satisfies the Baer condition and is injective as in the proof of
Theorem~\ref{thm:enough-flabby-modules}.
\end{proof}

\begin{prop}Assuming Zorn's lemma:
\begin{enumerate}
\item Every abelian group embeds canonically into an injective abelian group.
\item Every module embeds canonically into an injective module.
\item Every module in any Grothendieck topos embeds canonically into an injective module.
\end{enumerate}
\end{prop}

\begin{proof}The first claim is by Proposition~\ref{prop:embed2}.

The second claim follows purely formally as in the proof of
Proposition~\ref{prop:basics-injective}(3), using the adjunction between the
category of modules and the category of abelian groups.

For the third claim, let~$\E$ be a Grothendieck topos and let~$\pi : \F \to \E$ be its Barr
covering \XXX{ref}. Zorn's lemma passes from the metatheory to~$\F$ since~$\F$ is a
localic topos. Hence the first part applies in~$\F$. Purely formally, using the
adjunction between~$\mathrm{Ab}(\F)$ and~$\mathrm{Ab}(\E)$, there are enough
injective abelian groups in~$\E$. The existence of enough injective modules
in~$\E$ follows from internalizing the second claim in~$\E$.
\end{proof}

\XXX{cite whom for the general argument?}

\end{document}


TO DO:
* enough injectives employing only Zorn
* first names?
* cite more from Barakat and friends?

FUTURE:
* Does "flabby sheaves are acyclic on all locales" imply Zorn?
* quasicompact/spectral case
