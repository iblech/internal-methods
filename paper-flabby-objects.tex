\documentclass[oneside]{amsart}

\usepackage[utf8]{inputenc}
\usepackage{amsthm,mathtools,tikz,float,caption,stmaryrd}
\usetikzlibrary{patterns,matrix,decorations.pathreplacing}
\usepackage[all]{xy}
\usepackage[protrusion=true,expansion=true]{microtype}
\usepackage{hyperref}

\usepackage[natbib=true,style=numeric,maxnames=10]{biblatex}
\usepackage[babel]{csquotes}
\bibliography{paper-flabby-objects.bib}

\title{Flabby and injective objects in toposes}
\author{Ingo Blechschmidt}
\address{Max Planck Institute for Mathematics in the Sciences \\
Inselstraße 22 \\
04103 Leipzig, Germany}
\email{ingo.blechschmidt@mis.mpg.de}

\theoremstyle{definition}
\newtheorem{defn}{Definition}[section]
\newtheorem{ex}[defn]{Example}

\theoremstyle{plain}
\newtheorem{prop}[defn]{Proposition}
\newtheorem{cor}[defn]{Corollary}
\newtheorem{lemma}[defn]{Lemma}
\newtheorem{thm}[defn]{Theorem}
\newtheorem{scholium}[defn]{Scholium}

\theoremstyle{remark}
\newtheorem{rem}[defn]{Remark}
\newtheorem{question}[defn]{Question}
\newtheorem{speculation}[defn]{Speculation}
\newtheorem{caveat}[defn]{Caveat}
\newtheorem{conjecture}[defn]{Conjecture}

\newcommand{\xra}[1]{\xrightarrow{#1}}
\newcommand{\XXX}[1]{\textbf{XXX: #1}}
\newcommand{\aaa}{\mathfrak{a}}
\newcommand{\bbb}{\mathfrak{b}}
\newcommand{\mmm}{\mathfrak{m}}
\newcommand{\I}{\mathcal{I}}
\newcommand{\J}{\mathcal{J}}
\newcommand{\E}{\mathcal{E}}
\newcommand{\F}{\mathcal{F}}
\newcommand{\B}{\mathcal{B}}
\newcommand{\NN}{\mathbb{N}}
\newcommand{\ZZ}{\mathbb{Z}}
\renewcommand{\P}{\mathcal{P}}
\renewcommand{\O}{\mathcal{O}}
\newcommand{\defeq}{\vcentcolon=}
\newcommand{\op}{\mathrm{op}}
\DeclareMathOperator{\Spec}{Spec}
\DeclareMathOperator{\Hom}{Hom}
\DeclareMathOperator{\Mod}{Mod}
\DeclareMathOperator{\Sh}{Sh}
\DeclareMathOperator{\PSh}{PSh}
\newcommand{\Set}{\mathrm{Set}}
\newcommand{\Eff}{\mathrm{Ef{}f}}
\renewcommand{\_}{\mathpunct{.}\,}

\newcommand{\stacksproject}[1]{\cite[{\href{https://stacks.math.columbia.edu/tag/#1}{Tag~#1}}]{stacks-project}}

\begin{document}

\begin{abstract}
  We introduce a general notion of \emph{flabby objects} in elementary toposes
  and study their basic properties. In the special case of localic toposes, this
  notion reduces to the common notion of flabby sheaves, yielding a
  site-independent characterization of flabby sheaves. Continuing a line of
  research started by Roswitha Harting, we use flabby objects to
  show that an internal notion of injective objects coincides with the
  corresponding external notion, in stark contrast with the situation for
  projective objects. As an application, we give internal characterizations of
  sheaf cohomology and show that higher direct images can be understood as
  internal cohomology.
\end{abstract}

\maketitle
\thispagestyle{empty}

\noindent

% XXX: In Einleitung interne Sprache erinnern.

\section{Flabby sheaves}

A sheaf~$F$ on a topological space (or locale)~$X$ is \emph{flabby} (flasque) if and only
if all restriction maps~$F(X) \to F(U)$ are surjective. The following
properties of flabby sheaves render them fundamental to the theory of sheaf
cohomology:
% XXX: Je nachdem, wie die Einleitung aussieht, hier die Betonung anders
% setzen. Fundamental ist nur, dass man welke Auflösungen zur Berechnung von
% Garbenkohomologie verwenden kann und dass sie von f_* bewahrt werden. Die
% restlichen Eigenschaften sind wichtig, aber nicht fundamental.
\begin{enumerate}
\item Let~$(U_i)_i$ be an open covering of~$X$.
A sheaf~$F$ on~$X$ is flabby if and only if all of its restrictions~$F|_{U_i}$
are flabby as sheaves on~$U_i$.
\item Let~$f : X \to Y$ be a continuous map. If~$F$ is a flabby sheaf on~$X$,
then~$f_*(F)$ is a flabby sheaf on~$Y$.
\item[(3)] Let~$0 \to F \to G \to H \to 0$ be a short exact sequence of sheaves of
modules.
\begin{enumerate}
\item If~$F$ is flabby, then this sequence is also exact as a sequence of
presheaves.
\item If~$F$ and~$H$ are flabby, then so is~$G$.
\item If~$F$ and~$G$ are flabby, then so is~$H$.
\end{enumerate}
\item[(4a)] Any sheaf can be embedded into a flabby sheaf.
\item[(4b)] Any sheaf of modules can be embedded into a flabby sheaf of modules.
\end{enumerate}

Since we want to develop an analogous theory for flabby objects in elementary
toposes, it is worthwhile to analyze the logical and set-theoretic commitments
which are required to establish these properties. The standard proofs of
properties~(1),~(3a),~(3b) and~(3c) require Zorn's lemma to construct maximal
extensions of given sections. The standard proof of property~(4b) requires the
law of excluded middle, to ensure that the Godement construction actually
yields a flabby sheaf. Properties~(2) and~(4a) can be verified purely
intuitionistically.

There is an alternative definition of flabbiness, to be introduced below, which
is equivalent to the usual one in presence of Zorn's lemma and which
requires different commitments: For the alternative definition,
properties~(1),~(3b) and~(4a) can be verified purely intuitionistically, and
property~(2) can be verified purely intuitionistically for open injections~$f$.
There is a substitute for property~(3a) which can be verified purely
intuitionistically. We do not know whether property~(4b) can be established
purely intuitionistically, but we give a rudimentary analysis in
Section~\ref{sect:conclusion}.

Both definitions can be generalized to yield notions of flabby objects in
elementary toposes; but for toposes which are not localic, the two resulting
notions will differ, and only the one obtained from the alternative definition
is stable under pullback and
can be characterized in the internal language. We therefore adopt in this paper the
alternative one as the official definition.

\begin{defn}\label{defn:flabby-sheaf}
A sheaf~$F$ on a topological space (or locale)~$X$ is \emph{flabby}
if and only if for all opens~$U$ and all sections~$s \in F(U)$, there is an
open covering~$X = \bigcup_i U_i$ such that, for all~$i$, the section~$s$ can
be extended to a section on~$U \cup U_i$.\end{defn}

% Take care: This paragraph is referenced in prop:global-elements.
If~$F$ is a flabby sheaf in the traditional sense, then~$F$ is obviously also
flabby in the sense of Definition~\ref{defn:flabby-sheaf} -- singleton
coverings will do. Conversely, let~$F$ be a flabby sheaf in the sense of
Definition~\ref{defn:flabby-sheaf}. Let~$s \in F(U)$ be a local section. Zorn's
lemma implies that there is a maximal extension~$s' \in F(U')$. By assumption,
there is an open covering~$X = \bigcup_i U_i$ such that, for all~$i$, the
section~$s'$ can be extended to~$U' \cup U_i$. Since~$s'$ is maximal, $U' \cup
U_i = U'$ for all~$i$. Therefore~$X = \bigcup_i U_i \subseteq U'$; hence~$s'$ is a
global section, as desired.

We remark that unlike the traditional definition of flabbiness,
Definition~\ref{defn:flabby-sheaf} exhibits flabbiness as a manifestly local
notion.


\section{Flabby sets}\label{sect:flabby-sets}

We intend this section to be applied in the internal language of an elementary
topos; we will speak about sets and maps between sets, but intend our arguments
to be applied to objects and morphisms in toposes. We will therefore be careful
to reason purely intuitionistically. We
adopt the terminology of~\cite{kock:partial-maps} regarding subterminals and
subsingletons: A subset~$K \subseteq X$ is \emph{subterminal} if and only if any given
elements are equal ($\forall x,y \in K\_ x = y$), and it is a
\emph{subsingleton} if and only if there is an element~$x \in X$ such that~$K
\subseteq \{ x \}$. Any subsingleton is trivially subterminal, but the converse
might fail.

\begin{defn}A set~$X$ is \emph{flabby} if and only if any subterminal subset
of~$X$ is a subsingleton, that is, if and only if for any subset~$K \subseteq
X$ such that~$\forall x,y \in K\_ x = y$, there exists an element~$x \in X$
such that~$K \subseteq \{ x \}$.
\end{defn}

In the presence of the law of excluded middle, a set is flabby if and only if
it is inhabited. This characterization is a constructive taboo:

\begin{prop}\label{prop:taboo}
If any inhabited set is flabby, then the law of excluded middle
holds.
\end{prop}

\begin{proof}Let~$\varphi$ be a truth value. The set~$X \defeq \{ 0 \}
\cup \{ 1 \,|\, \varphi \} \subseteq \{ 0,1 \}$ is inhabited by~$0$ and
contains~$1$ if and only if~$\varphi$ holds. Let~$K$ be the subterminal~$\{ 1 \,|\, \varphi
\} \subseteq X$. Flabbiness of~$X$ implies that there exists an element~$x \in
X$ such that~$K \subseteq \{x\}$. We have~$x = 0$ or~$x = 1$. The first case
entails~$\neg\varphi$. The second case entails~$1 \in X$, so~$\varphi$.
\end{proof}

Let~$\P_{\leq 1}(X)$ be the set of subterminals of~$X$.

\begin{prop}A set~$X$ is flabby if and only if the canonical map~$X \to
\P_{\leq 1}(X)$ which sends an element~$x$ to the singleton set~$\{x\}$ is
final.
\end{prop}

\begin{proof}By definition.\end{proof}

The set~$\P_{\leq 1}(X)$ of subterminals of $X$ can be interpreted as the set
of \emph{partially-defined elements} of $X$. In this view, the empty subset is
the maximally undefined element and a singleton is a maximally defined element.
A set is flabby if and only if any of its partially-defined elements can be
refined to an honest element.

\begin{defn}\begin{enumerate}
\item A set~$X$ is \emph{injective} if and only if, for any injection~$i
: A \to B$, any map~$f : A \to X$ can be extended to a map~$B \to X$.
\item An~$R$-module~$M$ is \emph{injective} if and only if, for any linear
injection~$i : A \to B$ between~$R$-modules, any linear map~$f : A \to M$ can
be extended to a linear map~$B \to M$.
\end{enumerate}
\[ \xymatrix{
  A \ar@{^{(}->}[r]\ar[d] & B \ar@{-->}@/^/[ld] &&
  A \ar@{^{(}->}[r]\ar[d] & B \ar@{-->}@/^/[ld] \\
  X &&& M
} \]
\end{defn}

In the presence of the law of excluded middle, a set is injective if and only
if it is inhabited. In the presence of the axiom of choice, an abelian group is
injective (as a~$\ZZ$-module) if and only if it is divisible. Injective sets
and modules have been intensively studied before~\cite{XXX}; the following
properties are well-known:

\begin{prop}\label{prop:basics-injective}
\begin{enumerate}
% Take care if numbering changes (referenced below)
\item Any set can be embedded into an injective set.
\item Any injective module is also injective as a set.
\item Assuming the axiom of choice, any module can be embedded into an
injective module.
\end{enumerate}\end{prop}

\begin{proof}\begin{enumerate}
\item One can check that, for instance, the full powerset~$\P(X)$ and the set of
subterminals~$\P_{\leq 1}(X)$ are each injective.~\cite{XXX}
\item The forgetful functor from modules to sets possesses a left exact left
adjoint. More explicitly, if~$i : A \to B$ is an injective map between sets and
if~$f : A \to M$ is an arbitrary map, then the induced map~$R\langle A \rangle
\to R\langle B \rangle$ between free modules is also injective, the given
map~$f$ lifts to a linear map~$R\langle A \rangle \to M$, and an~$R$-linear
extension~$R\langle B \rangle \to M$ induces an extension~$B \to M$ of~$f$.
\item One verifies that any abelian group can be embedded into a divisible
abelian group. By the axiom of choice, divisible abelian groups are injective.
The result for modules over arbitrary rings then follows purely
formally~\cite{XXX}. \qedhere
\end{enumerate}\end{proof}

\begin{prop}\label{prop:injective-flabby}
Any injective set is flabby.\end{prop}

\begin{proof}Let~$X$ be an injective set. Let~$K \subseteq X$ be a subterminal.
The inclusion~$f : K \to X$ extends along the injection~$K \to 1 = \{\star\}$
to a map~$1 \to X$. The unique image~$x$ of that map has the property that~$K
\subseteq \{x\}$.\end{proof}

\begin{cor}\label{cor:enough-flabby-sets}
Any set can be embedded into a flabby set.\end{cor}

\begin{proof}Immediate by Proposition~\ref{prop:basics-injective}(1) and
Proposition~\ref{prop:injective-flabby}.\end{proof}

A further corollary of Proposition~\ref{prop:injective-flabby} is that the
statement ``any inhabited set is injective'' is a constructive taboo: If any
inhabited set is injective, then any inhabited set is flabby, thus the law of
excluded middle follows by Proposition~\ref{prop:taboo}.

\begin{prop}Any singleton set is flabby. The cartesian product of flabby sets
is flabby.\end{prop}

\begin{proof}Immediate.\end{proof}

Subsets of flabby sets are in general not flabby, as else any set would be
flabby in view of Corollary~\ref{cor:enough-flabby-sets}.

\begin{prop}\label{prop:hom-flabby}
\begin{enumerate}
\item Let~$I$ be an injective set. Let~$T$ be an arbitrary set. Then the
set~$I^T$ of maps from~$T$ to~$I$ is flabby.
\item Let~$I$ be an injective~$R$-module. Let~$T$ be an arbitrary~$R$-module. Then the
set~$\Hom_R(T,I)$ of linear maps from~$T$ to~$I$ is flabby.
\end{enumerate}\end{prop}

\begin{proof}We first cover the case of sets. Let~$K \subseteq I^T$ be a
subterminal. We consider the injectivity diagram
\[ \xymatrix{
  T' \ar@{^{(}->}[r]\ar[d] & T \ar@{-->}@/^/[ld] \\
  I
} \]
where~$T'$ is the subset~$\{ s \in T \,|\, \text{$K$ is inhabited} \} \subseteq T$ and the
solid vertical map sends~$s \in T'$ to~$g(s)$, where~$g$ is an arbitrary element
of~$K$. This association is well-defined. Since~$I$ is injective, a dotted lift
as indicated exists. If~$K$ is inhabited, this lift is an element of~$K$.

The same kind of argument applies to the case of modules. If~$K$ is a
subterminal of~$\Hom_R(T,I)$, we define~$T'$ to be the submodule
$\{ s \in T \,|\, \text{$s = 0$ or $K$ is inhabited} \}$ and consider the
analogous injectivity diagram, where the solid vertical map~$f : T' \to I$ is now
defined by cases: Let~$s \in T'$. If~$s = 0$, then we set~$f(s) = 0$; if~$K$ is
inhabited, then we set~$f(s) \defeq g(s)$, where~$g$ is an arbitrary element
of~$K$. This association is again well-defined, and a dotted lift yields the
desired element of~$\Hom_R(T,I)$.
\end{proof}

Proposition~\ref{prop:hom-flabby} can be used to give an alternative proof of
Proposition~\ref{prop:injective-flabby} and to generalize
Proposition~\ref{prop:injective-flabby} to modules: If~$I$ is an injective set,
then the set~$I^1 \cong I$ is flabby. If~$I$ is an injective module, then the
set~$\Hom_R(R,I) \cong I$ is flabby.

\begin{lemma}\label{lemma:set-of-extensions-flabby}
\begin{enumerate}
\item Let~$I$ be an injective set. Let~$i : A \to B$ be an injection.
Let~$f : A \to I$ be an arbitrary map. Then the set of extensions of~$f$ to~$B$
is flabby.
\item Let~$I$ be an injective~$R$-module. Let~$i : A \to B$ be a linear injection.
Let~$f : A \to I$ be an arbitrary linear map. Then the set of linear extensions of~$f$ to~$B$
is flabby.
\end{enumerate}
\end{lemma}

\begin{proof}For the first claim, we set~$X \defeq \{ \bar{f} \in I^B \,|\, \bar{f} \circ i =
f \}$. Let~$K \subseteq X$ be a subterminal. We consider the injectivity diagram
\[ \xymatrix{
  i[A] \cup B' \ar@{^{(}->}[r]\ar[d]_g & B \ar@{-->}@/^/[ld] \\
  I
} \]
where~$B'$ is the set~$\{ s \in B \,|\, \text{$K$ is inhabited} \}$ and the solid
vertical arrow~$g$ is defined in the following way: Let~$s \in i[A] \cup B'$.
If~$s \in i[A]$, we set~$g(s) \defeq f(a)$, where~$a \in A$ is an element such
that~$s = i(a)$. If~$s \in B'$, we set~$g(s) \defeq \bar{f}(s)$,
where~$\bar{f}$ is any element of~$K$. These prescriptions determine a well-defined
map.

Since~$I$ is injective, there exists a dotted map rendering the diagram
commutative. This map is an element of~$X$. Furthermore, if~$K$ is inhabited,
then this map is an element of~$K$.

The proof of the second claim is similar. We
set~$X \defeq \{ \bar{f} \in \Hom_R(B,I) \,|\, \bar{f} \circ i =
f \}$. Let~$K \subseteq X$ be a subterminal. We consider the injectivity diagram
\[ \xymatrix{
  i[A] + B' \ar@{^{(}->}[r]\ar[d]_g & B \ar@{-->}@/^/[ld] \\
  I
} \]
where~$B'$ is the submodule~$\{ t \in B \,|\, \text{$t = 0$ or $K$ is
inhabited} \} \subseteq B$ and the solid vertical arrow~$g$ is defined in the following
way: Let~$s \in i[A] + B'$. Then~$s = i(a) + t$ for an element~$a \in A$ and an
element~$t \in B'$. Since~$t \in B'$, $t = 0$ or~$K$ is inhabited. If~$t = 0$,
we set~$g(s) \defeq f(a)$. If~$K$ is inhabited, we set~$g(s) \defeq f(a) +
\bar{f}(s)$, where~$\bar{f}$ is any element of~$K$. These prescriptions
determine a well-defined map.

Since~$I$ is injective, there exists a dotted map rendering the diagram
commutative. This map is an element of~$X$. Furthermore, if~$K$ is inhabited,
then this map is an element of~$K$.
\end{proof}

\begin{prop}\label{prop:set-of-preimages-flabby}
Let~$0 \to M' \xra{i} M \xra{p} M'' \to 0$ be a short exact
sequence of modules. Let~$s \in M''$. If~$M'$ is flabby, then the set of
preimages of~$s$ under~$p$ is flabby.
\end{prop}

\begin{proof}Let~$X \defeq \{ u \in M \,|\, p(u) = s
\}$. Let~$K \subseteq X$ be a subterminal. Since~$p$ is surjective, there is an
element~$u_0 \in X$. The translated set~$K - u_0 \subseteq M$ is still a
subterminal, and its preimage under~$i$ is as well. Since~$M'$ is flabby, there
is an element~$v \in M'$ such that~$i^{-1}[K - u_0] \subseteq \{v\}$. We verify
that~$K \subseteq \{u_0 + i(v)\}$.

Thus let~$u \in K$ be given. Then~$p(u - u_0) = 0$, so by exactness the
set~$i^{-1}[K - u_0]$ is inhabited. It therefore contains~$v$. Thus~$i(v) \in K
- u_0$. Since~$K = \{u\}$, it follows that~$i(v) = u - u_0$, so~$u \in \{u_0 +
i(v)\}$ as claimed.
\end{proof}

\begin{prop}Let~$0 \to M' \xra{i} M \xra{p} M'' \to 0$ be a short exact
sequence of modules. If~$M'$ and~$M''$ are flabby, so is~$M$.
\end{prop}

\begin{proof}Let~$K \subseteq M$ be a subterminal. Then its image~$p[K] \subseteq M''$
is a subterminal as well. Since~$M''$ is flabby, there is an element~$s \in
M''$ such that~$p[K] \subseteq \{ s \}$.

Since~$p$ is surjective, there is an element~$u_0 \in M$ such that~$p(u_0) =
s$.

The preimage~$i^{-1}[K - u_0] \subseteq M'$ is a subterminal. Since~$M'$ is
flabby, there exists an element~$v \in M'$ such that~$i^{-1}[K - u_0] \subseteq
\{v\}$.

Thus~$K \subseteq \{ u_0 + i(v) \}$.
\end{proof}

Noticeably missing here is XXX ...


\section{Flabby objects}

\begin{defn}An object~$X$ of an elementary topos~$\E$ is \emph{flabby} if and
only if the statement~``$X$ is a flabby set'' holds in the stack semantics
of~$\E$.\end{defn}

This definition amounts to the following: An object~$X$ of an elementary
topos~$\E$ is flabby if and only if, for any monomorphism~$K \to A$ and any
morphism~$K \to X$, there exists an epimorphism~$B \to A$ and a morphism~$B
\to X$ such that the following diagram commutes.
\[ \xymatrix{
  K \times_A B \ar@{^{(}->}[r]\ar[d] & B \ar@{-->}@/^/[ld] \\
  X
} \]

Instead of referencing arbitrary stages~$A \in \E$, one can also just reference
the generic stage: Let~$\P_{\leq1}(X)$ denote the \emph{object of subterminals}
of~$X$; this object is a certain suboject of~$\P(X) = [X,\Omega_\E]_\E$, the
powerobject of~$X$. The subobject~$K_0$ of~$X \times \P_{\leq1}(X)$ classified by the
evaluation morphism~$X \times \P_{\leq1}(X) \to X \times \P(X) \to \Omega_\E$
is the \emph{generic subterminal} of~$X$. The object~$X$ is flabby if and only
if there exists an epimorphism~$B \to \P_{\leq1}(X)$ and a morphism~$B \to X$
such that the following diagram commutes.
\[ \xymatrix{
  K_0 \times_{\P_{\leq1}(X)} B \ar@{^{(}->}[r]\ar[d] & \P_{\leq1}(X) \ar@{-->}@/^/[ld] \\
  X
} \]

\begin{prop}\label{prop:basic-properties-of-flabby-objects}
Let~$X$ and~$T$ be objects of an elementary topos~$\E$.
\begin{enumerate}
\item If~$X$ is flabby, so is~$X \times T$ as an object of~$\E/T$.
\item The converse holds if the unique morphism~$T \to 1$ is an epimorphism.
\end{enumerate}
\end{prop}

\begin{proof}This holds for any property which can be defined
in the stack semantics~\cite[Lemma~7.3]{shulman:stack-semantics}.
\end{proof}

\begin{prop}\label{prop:flabby-sheaves-objects}
Let~$F$ be a sheaf on a topological space~$X$ (or a locale).
Then~$F$ is flabby as a sheaf if and only if~$F$ is flabby as an object of the
sheaf topos~$\Sh(X)$.
\end{prop}

\begin{proof}The proof is routine; we only verify the ``only if'' direction.
Let~$F$ be flabby as a sheaf. It suffices to verify the definining condition for stages
of the form~$A = \Hom(\cdot,U)$, where~$U$ is an open of~$X$. A monomorphism~$K
\to A$ then amounts to an open~$V \subseteq U$ (the union of all opens on
which~$K$ is inhabited). A morphism~$K \to F$ amounts to a section~$s \in
F(V)$. Since~$F$ is flabby as a sheaf, there is an open covering~$X =
\bigcup_{i \in I} V_i$ such that, for all~$i$, the section~$s$ can be extended
to a section~$s_i$ of~$V \cup V_i$. The desired epimorphism is~$B \defeq
\coprod_i \Hom(\cdot,(V \cup V_i) \cap U) \to A$, and the desired morphism~$B
\to X$ is given by the sections~$s_i|_{(V \cup V_i) \cap U}$.

As stated, the argument in the previous paragraph requires the axiom of choice
to pick the extensions~$s_i$; this can be avoided by a standard trick of
expanding the index set of the coproduct to include the choices: We redefine $B \defeq
\coprod_{(i,t) \in I'} \Hom(\cdot, (V \cup V_i) \cap U)$, where~$I' = \{ (i \in
I, t \in F(V \cup V_i)) \,|\, t|_V = s \}$ and define the morphism~$B \to X$ on
the~$(i,t)$-summand by~$t|_{(V \cup V_i) \cap U}$.
\end{proof}

\begin{prop}\label{prop:global-elements}
Let~$X$ be a flabby object of a localic topos~$\E$. If
Zorn's lemma is available in the metatheory, then~$X$ possesses a global element (a morphism~$1 \to X$).
\end{prop}

\begin{proof}This is a restatement of the discussion following
Definition~\ref{defn:flabby-sheaf}.
\end{proof}

\begin{prop}\label{prop:pushforward-of-flabby-objects}
Let~$f : \F \to \E$ be a geometric morphism. If~$f_*$ preserves epimorphisms,
then~$f_*$ preserves flabby objects.\end{prop}

\begin{proof}Let~$X \in \F$ be a flabby object.
Let~$k : K \to A$ be a monomorphism in~$\E$ and let~$x : K \to f_*(X)$ be an
arbitary morphism. Without loss of generality, we may assume that~$A$ is the
terminal object~$1$ of~$\E$. Then~$f^*(k) : f^*(K) \to 1$ is a monomorphism
in~$\F$ and the adjoint transpose~$x^t : f^*(K) \to X$
is a morphism in~$\F$. Since~$X$ is flabby, there is an epimorphism~$B \to 1$
in~$\F$ and a morphism~$y : B \to X$ such that the morphism~$f^*(K) \times B
\to X$ factors over~$y$. Hence~$x$ factors over~$f_*(y) : f_*(B) \to f_*(X)$.
We conclude because the morphism~$f_*(B) \to f_*(1)$ is an epimorphism by
assumption.
\end{proof}

The assumption on~$f_*$ of Proposition~\ref{prop:pushforward-of-flabby-objects}
is for instance satisfied if~$f$ is a local geometric morphism or if~$f$ is
induced by an open continuous injection between topological spaces. XXX

\begin{defn}An object~$I$ of an elementary topos~$\E$ is \emph{externally
injective} if and only if for any monomorphism~$A \to B$ in~$\E$, the canonical
map~$\Hom_\E(B,I) \to \Hom_\E(A,I)$ is surjective. It is \emph{internally
injective} if and only if for any monomorphism~$A \to B$ in~$\E$, the canonical
morphism~$[B,I] \to [A,I]$ between Hom objects is an epimorphism in~$\E$.
\end{defn}

If~$R$ is a ring in an elementary topos~$\E$, a similar definition can be given
for~$R$-modules in~$\E$, referring only to the set respectively the object of
linear maps. The condition for an object to be internally injective can be
rephrased in various ways. The following proposition lists five of these
conditions. The equivalence of the first four is due to
Harting~\cite{harting:locally-injective}.

\begin{prop}\label{prop:notions-of-internal-injectivity}
Let~$\E$ be an elementary topos. Then the following statements about an
object~$I \in \E$ are equivalent.
% Take care if numbering changes (is referenced below).
\begin{enumerate}
\item[(1)] $I$ is internally injective.
\item[(1')] For any morphism $p : A \to 1$ in $\E$, the object $p^*(I)$ has property~(1)
as an object of $\E/A$.
\item[(2)] The functor~$[\cdot, I] : \E^\op \to \E$ maps monomorphisms in $\E$
to morphisms for which any global element of the target locally (after change of
base along an epimorphism) possesses a preimage.
\item[(2')] For any morphism $p : A \to 1$ in $\E$, the object $p^*(I)$ has property~(2)
as an object of $\E/A$.
\item[(3)] The statement~``$I$ is an injective set'' holds in the stack
semantics of~$\E$.
\end{enumerate}
\end{prop}

\begin{proof}
The implications (1)~$\Rightarrow$~(2), (1')~$\Rightarrow$~(2'),
(1')~$\Rightarrow$~(1) and (2')~$\Rightarrow$~(2) are trivial.

The equivalence (1')~$\Leftrightarrow$~(3) follows directly from the
interpretation rules of the stack semantics.

The implication (2)~$\Rightarrow$~(2') employs the
extra left adjoint $p_! : \E/A \to \E$ of $p^* : \E
\to \E/A$~(which maps an object~$(X \to A)$ to~$X$), as in the usual proof that
injective sheaves remain injective when
restricted to smaller open subsets: We have that $p_* \circ [\cdot, p^*(I)]_{\E/A}
\cong [\cdot, I]_\E \circ p_!$, the functor $p_!$ preserves monomorphisms, and one
can check that $p_*$ reflects the property that global elements locally possess
preimages. Details are in~\cite[Thm.~1.1]{harting}.\footnote{Harting formulates
the statement for abelian group objects, and has to assume that~$\E$ contains a
natural numbers object to ensure the existence of an abelian version of~$p_!$.}

The implication (2')~$\Rightarrow$~(1') follows by performing an extra change of
base, since any non-global element becomes a global element after a suitable
change of base.
\end{proof}

Let~$R$ be a ring in~$\E$. Then the analogue of
Proposition~\ref{prop:notions-of-internal-injectivity} holds for~$R$-modules
in~$\E$, if~$\E$ is assumed to have a natural numbers object. The extra
assumption is needed in order to construct the left adjoint~$p_! :
\Mod_{\E/A}(R \times A) \to \Mod_\E(R)$. Phrased in the internal language, this
adjoint maps a family~$(M_a)_{a \in A}$ of~$R$-modules to the direct
sum~$\bigoplus_{a \in A} M_a$. Details on this construction, phrased in the
language of sets but interpretable in the internal language, can for instance
be found in~\cite[page~54]{mines-richman-ruitenburg:constructive-algebra}.

Somewhat surprisingly, and in stark contrast with the situation for internally
projective objects (which are defined dually), internal injectivity coincides
with external injectivity for localic toposes.

\begin{thm}\label{thm:injectivity-external-internal}
Let~$I$ be an object of an elementary topos~$\E$. If~$I$ is externally
injective, then~$I$ is also internally injective. The converse holds if~$\E$ is
localic and Zorn's lemma is available.
\end{thm}

\begin{proof}For the ``only if'' direction, let~$I$ be an object
which is externally injective. Then~$I$ satisfies Condition~(2) in
Proposition~\ref{prop:notions-of-internal-injectivity}, even without having to
pass to covers.

For the ``if'' direction, let~$I$ be an internally
injective object. Let~$i : A \to B$ be a monomorphism in~$\E$ and let~$f :
A \to I$ be an arbitrary morphism. We want to show that there exists an
extension $B \to I$ of~$f$ along~$i$. To this end, we consider the object of
such extensions, defined by the internal expression
\[ F \defeq \{ \bar{f} \in [B,I] \,|\, \bar{f} \circ i = f \}. \]
Global elements of~$F$ are extensions of the kind we are looking for.
By Lemma~\ref{lemma:set-of-extensions-flabby}, this object is flabby.
By Proposition~\ref{prop:global-elements}, it has a global element.
\end{proof}

The analogue of Theorem~\ref{thm:injectivity-external-internal} for modules
holds as well, if~$\E$ is assumed to have a natural numbers object. The proof
carries over word for word, only referencing
Lemma~\ref{lemma:set-of-extensions-flabby}(2) instead of
Lemma~\ref{lemma:set-of-extensions-flabby}(1).
It seems that Harting was not aware of this, even though she did show that
injectivity of sheaves of modules over topological spaces is a local
notion~\cite{XXX}, as she
(mistakenly) states in~\cite[page~233]{harting:remark} that ``the notions of
injectivity and internal injectivity do not coincide'' for modules.
% XXX comparison with Harting's proof (Barr's lemma...)

Since we were careful in Section~\ref{sect:flabby-sets} to use the law of
excluded middle or the axiom of choice only where needed, most results of that
section carry over to flabby and internally injective objects. Specifically, we
have:

\begin{scholium}\label{scholium:properties-of-flabby-objects}
For any elementary topos~$\E$:
\begin{enumerate}
\item Any object can be embedded into an internally injective object.
\item (If~$\E$ has a natural numbers object.) The underlying unstructured
object of an internally injective module is internally injective.
\item Any internally injective object is flabby.
% Take care if numbering changes (referenced below).
\item Any object can be embedded into a flabby object.
\item The terminal object is flabby. The product of flabby objects is flabby.
\item Let~$I$ be an internally injective object. Let~$T$ be an arbitrary
object. Then~$[T,I]$ is a flabby object.
\item (If~$\E$ has a natural numbers object.) Let~$I$ be an internally
injective~$R$-module. Let~$T$ be an arbitrary~$R$-module. Then~$[T,I]_R$, the
subobject of the internal Hom consisting only of the linear maps, is a flabby
object.
\item Let~$0 \to M' \to M \to M'' \to 0$ be a short exact sequence
of~$R$-modules in~$\E$. If~$M'$ and~$M''$ are flabby objects, so is~$M$.
\end{enumerate}
\end{scholium}

\begin{proof}The analogous statements were established purely
intuitionistically in Section~\ref{sect:flabby-sets}, and the stack semantics
is sound with respect to intuitionistic logic.
\end{proof}

\begin{scholium}\label{scholium:exact-as-presheaves}
Let~$0 \to M' \to M \to M'' \to 0$ be a short exact sequence
of~$R$-modules in an elementary topos~$\E$. Let~$M'$ be a flabby object.
If~$\E$ is localic and Zorn's lemma is available, the induced sequence~$0 \to
\Gamma(M') \to \Gamma(M) \to \Gamma(M') \to 0$ of~$\Gamma(R)$-modules is exact,
where~$\Gamma(X) = \Hom_\E(1,X)$.\end{scholium}

\begin{proof}We only have to verify exactness at~$\Gamma(M'')$, so let~$s \in
\Gamma(M')$. Interpreting Proposition~\ref{prop:set-of-preimages-flabby}
in~$\E$, we see that the object of preimages of~$s$ is flabby. Since~$\E$ is
localic, this object is a flabby sheaf; since Zorn's lemma is available, it
possesses a global element. Such an element is the desired preimage of~$s$
in~$\Gamma(M)$.\end{proof}

If~$\E$ is not necessarily localic or Zorn's lemma is not available, only a
weaker substitute for Scholium~\ref{scholium:exact-as-presheaves} is available:
Given~$s \in \Gamma(M'')$, the object of preimages of~$s$ is flabby. XXX

\begin{rem}A direct generalization of the traditional notion of a flabby sheaf, as
opposed to our reimagining in Definition~\ref{defn:flabby-sheaf}, to
elementary toposes is the following. An object~$X$ of an elementary topos~$\E$
is \emph{strongly flabby} if and only if, for any monomorphism~$K \to 1$
in~$\E$, any morphism~$K \to X$ lifts to a morphism~$1 \to X$.

One can verify, purely intuitionistically, that a sheaf~$F$ on a space~$T$ is
flabby in the traditional sense if and only if~$F$ is a strongly flabby object
of~$\Sh(T)$.

The notion of strongly flabby objects is, however, not stable under base change
and therefore cannot be characterized in the internal language. A specific
example is the~$G$-set~$G$ (with the translation action), considered as an
object of the topos~$BG$ of~$G$-sets, where~$G$ is a nontrivial group.
This object is not strongly flabby, since the morphism~$\emptyset \to G$ does
not lift, but its pullback to the slice~$BG/G \simeq \Set$ is (assuming
the law of excluded middle).
\end{rem}


\section{Characterizing cohomology}

XXX
% EXT, TOR
% Higher direct images


\section{Flabby objects in the effective topos}

The notion of flabby objects originates from the notion of flabby shaves and is
therefore closely connected to Grothendieck toposes. It is therefore instructive
to study flabby objects in elementary toposes which are not Grothendieck
toposes, away from their original conceptual home. We begin this study with
establishing the following observations on flabby objects in the effective
topos. We follow the terminology of~\cite{hyland:effective-topos}.

\begin{prop}\label{prop:flabby-effective-sets}
Let~$X$ be a flabby object in the effective topos. Let~$f : X \to X$
be a morphism. If~$X$ is effective, the statement ``$f$ has a fixed point''
holds in the effective topos.
\end{prop}

\begin{prop}\label{prop:semienough-flabby-modules}
Assuming the law of excluded middle in the metatheory, any~$\neg\neg$-separated
module in the effective topos can be embedded into a flabby module.
\end{prop}

% Δ(X) ist welk, falls X bewohnt.
% Δ(X) ist separiert.
% Ist A separierter Modul, so lässt sich A in einen welken Modul einbetten
% (der seinerseits separiert ist, er ist nämlich Δ(Γ(A))).
% Sind welke Mengen separiert? Nein: Omega.
% Sind welke Moduln separiert? Nein: Ich denke mittlerweile, dass sich jeder
% Modul in einen welken einbetten lässt, über etwas, was wie eine induktive
% Konstruktion aussieht.

% Rem.: diskret =/=> welk, sep =/=> welk (Bsp.: konstante Garbe Z auf Mnf.)
% Umkehrung auch nicht (Bsp.: Godemontkonstruktion, Omega)

% In Eff gilt: Jede negneg-separierte abelsche Gruppe bettet in eine injektive
% ein. Nämlich A --> ΔΓA --> ΔI für ΓA --> I Einbettung in eine injektive
% Gruppe.

The intuitive
reason for why Proposition~\ref{prop:flabby-effective-sets} holds is the
following. Let~$X$ be a flabby object in the effective topos. Then there is a
procedure which computes for any subterminal~$K \subseteq X$ an element~$x_K$
such that~$K \subseteq \{ x_K \}$. This element might not depend extensionally
on~$K$, but this fine point is not important for this discussion. Let
now~$f : X \to X$ be a morphism. We construct the self-referential subset~$K
\defeq \{ f(x_K) \}$; the formal proof below will indicate how this can be
done. Then~$K \subseteq \{ x_K \}$, so~$f(x_K) = x_K$.

A corollary of Proposition~\ref{prop:flabby-effective-sets} is that the trivial
module is the only flabby module in the effective topos whose underlying
unstructured object is an effective set: Given such a flabby module~$M$, let~$v
\in M$ be an arbitrary element. Then the morphism~$x \mapsto v + x$ has a fixed
point; thus~$v + x = x$ for some element~$x$, and hence~$v = 0$.

It is the self-referentiality which makes the proof of
Proposition~\ref{prop:flabby-effective-sets} work, but the blame for paucity
of flabby objects in the effective topos is to put on the realizers for
statements of the form~``$K = K$'', where~$({=})$ is the nonstandard equality
predicate of the powerobject~$\P(X)$. A procedure witnessing flabbiness has to
compute a reflexivity realizer for a suitable element~$x_K$ from a reflexivity
realizer for a given element~$K$. However, such realizers are not very
informative. Metaphorically speaking, a procedure witnessing flabbiness has to
conjure elements out of thin air.

This problem does not manifest with objects~$X$ which are not effective sets.
Reflexivity realizers for these objects are themselves not very informative;
a procedure witnessing flabbiness therefore only has to turn one kind of
non-informative realizers into another kind. The flabby modules featuring in
the proof of Proposition~\ref{prop:semienough-flabby-modules} will accordingly
not be effective sets.

\begin{proof}[Proof of Proposition~\ref{prop:flabby-effective-sets}]
For any Turing machine~$e$, let~$v_e : |X| \to \Sigma$ be the nonstandard
predicate given by
\begin{multline*}
  v_e(x) = \{ m \in \NN \,|\,
  \text{there is an element~$x_0 \in |X|$ such that} \\
  \text{$e$ terminates with an element of~$\llbracket x_0 = x_0 \rrbracket$ and
  $m \in \llbracket x = x_0 \rrbracket$} \}
\end{multline*}
and let~$K_e \in \Sigma^{|X| \times \Sigma}$ be the nonstandard predicate given by
\[ K_e(x,u) = \llbracket (x = x) \wedge (u \leftrightarrow v_e(x)) \rrbracket. \]
One can explicitly construct a realizer~$a_e$ of the statement~``$K_e = K_e$'',
where~$({=})$ is the nonstandard equality predicate of the object~$\P_{\leq1}(X)$
of subterminals of~$X$. This is where the assumption that~$X$ is effective is
important; without it, we could only verify~``$K_e = K_e$'' where~$({=})$ is
the nonstandard equality predicate of the full powerobject~$\P(X)$.

Since~$X$ is flabby, there is a realizer~$r$ for the statement~``$\forall K \in
\P_{\leq1}(X)\_ \exists x \in X\_ \forall y \in X\_ (y \in K \Rightarrow y =
x)$''. Let~$s$ be a realizer for the statement~``$\forall x \in X\_ \exists y
\in Y\_ y = f(x)$''. Let~$e$ be the particular Turing machine which proceeds as
follows:
\begin{enumerate}
\item[1.] Simulate~$r$ on input~$a_e$ in order to obtain a realizer~$b \in
\llbracket x = x \rrbracket$ for some~$x \in |X|$.
\item[2.] Simulate~$s$ on input~$b$ in order to obtain a realizer~$c \in \llbracket
f(x) = f(x) \rrbracket$.
\item[3.] Output~$c$.
\end{enumerate}
The description of the machine~$e$ makes use of the number~$e$ coding it;
the recursion theorem yields a general reason why this self-referentiality is
possible. Here we can even do without this theorem, since a close inspection of
the construction of~$a_e$ shows that~$a_e$ is actually independent of~$e$. This
should not come as a surprise, as reflexivity realizers of~$\P(X)$
and~$\P_{\leq1}(X)$ are known to be not very informative.

Passing~$a_e$ to~$r$ yields a reflexivity realizer of some element~$x_{K_e} \in
|X|$. Therefore the Turing machine~$e$ does terminate, with a reflexivity
realizer for~$f(x_{K_e})$. Thus the statement~``$f(x_{K_e}) \in K$'' is
realized; hence~``$f(x_{K_e}) = x_{K_e}$'' is as well.
\end{proof}

\begin{proof}[Proof of Proposition~\ref{prop:semienough-flabby-modules}]
Let~$(\Gamma \dashv \Delta) : \Set \to \Eff$ be the inclusion of the
double-negation sheaves. For a~$\neg\neg$-separated module~$M$ in the effective
topos, the canonical morphism~$M \to \Delta(\Gamma(M))$ is a monomorphism; the
set~$\Gamma(M)$ is flabby by virtue of being inhabited; and~$\Delta$ preserves
flabby objects by Proposition~\ref{prop:pushforward-of-flabby-objects}.
\end{proof}


\section{Conclusion}
\label{sect:conclusion}

We originally set out to develop an intuitionistic account of Grothendieck's
sheaf cohomology. Čech methods can be carried out constructively, and
XXX(Barakat), but it appears that there is not a general framework for sheaf
cohomology which works in an intuitionistic metatheory.

The main obstacle preventing Grothendieck's theory of derived functors to be interpreted
constructively is its reliance on injective resolutions. It is known that in
the absence of the axiom of choice, much less in a purely intuitionistic
context, there might not be any nontrivial injective abelian
group~\cite{blass:inj-proj-axc}.

In principle, this problem could be remedied by employing flabby resolutions
instead of injective ones. There are, however, two problems with this
suggestion. Firstly, we needed Zorn's lemma to show that flabby sheaves are
acyclic for the global sections functor (Proposition~XXX). This problem might
be mitigated by relying on the substitute property XXX.
The more serious problem is that it is an open question whether
one can show, purely intuitionistically, that any sheaf of modules embeds into
a flabby sheaf of modules. The following is known about this problem:

\begin{enumerate}
\item There is a purely intuitionistic proof that any sheaf of sets embeds into
a flabby sheaf of sets
(Scholium~\ref{scholium:properties-of-flabby-objects}(4)).

\item The existence of enough flabby modules, and even the existence of enough
injective modules, is \emph{not} a constructive taboo, that is, these statements do not
entail a classical principle like the law of excluded middle or the principle
of omniscience. This is because assuming the axiom of choice, any
Grothendieck topos has enough injective (and therefore flabby) modules. 

\item There is a way of embedding any module into a flabby module if one is
prepared to ignore set-theoretical difficulties. Namely, let~$M$ be an~$R$-module.
Inductively construct a collection~$T$ of terms by the following clauses: $0
\in T$ (where~$0$ is a formal symbol);
if~$t,s \in T$, then~$t + s \in T$; if~$t \in T$ and~$r \in R$, then~$rt \in
T$; if~$x \in M$, then~$\underline{x} \in T$; if~$K \subseteq T$ is a
subterminal, then~$\varepsilon_K \in T$. Let~$({\sim})$ be
the finest equivalence relation on~$T$ such that~$t + (s + u) \sim (t + s) +
u$, $t + s \sim s + t$, $t + 0 \sim t \sim 0 + t$, $0t \sim 0$, $1t \sim t$, $r(t+s) \sim rt
+ rs$, $(r+r')t \sim rt + r't$ and such
that~$\underline{0} \sim 0$, $\underline{x+y} \sim \underline{x} + \underline{y}$, $\underline{rx} \sim
r \underline{x}$, and such that~$\varepsilon_{\{t\}} \sim
t$ for all terms~$t$. Then the quotient~$T/{\sim}$ is a flabby~$R$-module
into which~$M$ embeds. However, it is not clear that~$T/{\sim}$ is a set.

\item There appears to be some tension regarding the effective topos:
Proposition~\ref{prop:semienough-flabby-modules} shows that at
least~$\neg\neg$-separated modules in the effective topos always embed into
flabby modules, assuming the law of excluded middle in the metatheory, while
Proposition~\ref{prop:flabby-effective-sets} shows that no nontrivial effective
module is flabby.
\end{enumerate}

We currently believe that it is not possible to give a constructive account of
a global cohomology functor. However, XXX.

\end{document}
