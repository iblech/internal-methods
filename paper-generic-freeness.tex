\documentclass{amsart}

\usepackage[utf8]{inputenc}
\usepackage{amsthm,mathtools}
\usepackage[protrusion=true,expansion=true]{microtype}
\usepackage{hyperref}

\title{An elementary and constructive proof of Grothendieck's generic freeness
lemma}
\author{Ingo Blechschmidt}

\theoremstyle{definition}
\newtheorem{defn}{Definition}[]
\newtheorem{ex}[defn]{Example}

\theoremstyle{plain}
\newtheorem{prop}[defn]{Proposition}
\newtheorem{cor}[defn]{Corollary}
\newtheorem{lemma}[defn]{Lemma}
\newtheorem{thm}[defn]{Theorem}
\newtheorem{scholium}[defn]{Scholium}

\theoremstyle{remark}
\newtheorem{rem}[defn]{Remark}
\newtheorem{question}[defn]{Question}
\newtheorem{speculation}[defn]{Speculation}
\newtheorem{caveat}[defn]{Caveat}
\newtheorem{conjecture}[defn]{Conjecture}

\newcommand{\XXX}[1]{\textbf{XXX: #1}}
\newcommand{\defeq}{\vcentcolon=}
\DeclareMathOperator{\Spec}{Spec}

\begin{document}

\begin{abstract}
  We present a new and direct proof of Grothendieck's generic freeness
  lemma. Unlike the previously published proofs, it doesn't require a series of
  reduction steps and is fully constructive, not involving the axiom of choice
  or even the law of excluded middle. It was found using a general
  topos-theoretic technique. \XXX{self-contained...}
\end{abstract}

\maketitle

We prove Grothendieck's generic freeness lemma for rings and modules in the
following form.

\begin{thm}\label{thm:algebraic}Let~$A$ be a reduced ring. Let~$B$ be
an~$A$-algebra of finite type. Let~$M$ be a finitely generated~$B$-module.
If~$f = 0$ is the only element of~$A$ such that
\begin{enumerate}
\item the~$A[f^{-1}]$-modules $B[f^{-1}]$ and $M[f^{-1}]$ are free,
\item the~$A[f^{-1}]$-algebra~$B[f^{-1}]$ is of finite presentation, and
\item the~$B[f^{-1}]$-module~$M[f^{-1}]$ is finitely presented,
\end{enumerate}
then~$1 = 0$ in~$A$.
\end{thm}

Previously known proofs proceed in a series of intermediate steps, reducing
to the case that~$A$ is a Noetherian integral domain where one can argue by
\emph{dévissage}; but in fact, a direct proof is possible and shorter.

Grothendieck's generic freeness lemma is often presented in contrapositive form
or in the following geometric variant:

\begin{thm}\label{thm:geometric}Let~$A$ be a reduced ring. Let~$B$ be
an~$A$-algebra of finite type. Let~$M$ be a finitely generated~$B$-module. Then
the space~$\Spec(A)$ contains a dense open~$U$ such that over~$U$,
\begin{enumerate}
\item[(a)] $B^\sim$ and~$M^\sim$ are free as sheaves of~$A^\sim$-modules,
\item[(b)] $B^\sim$ is of finite presentation as a sheaf of~$A^\sim$-algebras, and
\item[(c)] $M^\sim$ is finitely presented as a sheaf of~$B^\sim$-modules.
\end{enumerate}
\end{thm}

Theorem~\ref{thm:geometric} immediately follows from
Theorem~\ref{thm:algebraic} by defining~$U$ as the union of all the basic
opens~$D(f)$ such that~(1),~(2), and~(3) hold. It's clear that~(a),~(b),
and~(c) hold over~$U$, and~$U$ is dense for if~$V$ is an arbitrary open
such that~$U \cap V = \emptyset$, the open~$V$ is itself empty: Let~$h \in A$
such that~$D(h) \subseteq V$. The hypothesis implies the assumptions of
Theorem~\ref{thm:algebraic} for the datum~$(A[h^{-1}], B[h^{-1}], M[h^{-1}])$.
Thus~$1 = 0 \in A[h^{-1}]$, so~$h$ is nilpotent and~$D(h) = \emptyset$.

The new proof was found using a general topos-theoretical technique which we
believe to be useful in other situations as well. This technique allows to view
reduced rings and their modules from a different point of view, one from which
reduced rings look like fields. Using this technique, Grothendieck's generic freeness
lemma follows trivially, since the lemma is trivial for fields. The proof
presented here was obtained by unwinding the topos-theoretic proof, yielding a
self-contained argument without any references to topos theory.

\XXX{Acknowledgments, especially Martin Brandenburg}

\XXX{References to prior proofs: EGA, Stacks Project, Charles Staats. Check
\url{https://math.stackexchange.com/questions/1048496/question-on-generic-freeness-ref-matsumura-page-185}}


\section{The proof of the general case}

\begin{lemma}\label{lemma:basis}
Let~$A$ be a ring. Let~$M$ be an~$A$-module with generating
family~$(x_i)_{i \in I}$ where~$I$ is a totally ordered set. Assume that the
only element~$g \in A$ such that one of the~$x_i$ is an~$A[g^{-1}]$-linear
combination in~$M[g^{-1}]$ of other generators with smaller index is~$g = 0$.
Then~$M$ is free with~$(x_i)_{i \in I}$ as a basis.
\end{lemma}

\begin{proof}Let~$\sum_i a_i x_i = 0$. Starting with the greatest
index~$i$ which appears in that sum, we see that in~$M[a_i^{-1}]$, the
element~$x_i$ is an~$A[g^{-1}]$-linear combination of other generators with
smaller index. Thus~$a_i = 0$ by assumption.\end{proof}

\begin{thm}Let~$A$ be a reduced ring. Let~$B$ be an~$A$-algebra of finite type.
Let~$M$ be a finitely generated~$B$-module. If~$f = 0$ is the only element
of~$A$ such that~$M[f^{-1}]$ is free as an~$A[f^{-1}]$-module, then~$1 = 0$
in~$A$.\end{thm}

\begin{proof}Let~$B$ be generated by~$(x_1,\ldots,x_n)$ as an~$A$-algebra and
let~$M$ be generated by~$(v_1,\ldots,v_m)$ as an~$B$-module. We endow the set
\[ I \defeq \{ (\ell, j_1,\ldots,j_n) \,|\,
  \ell \in \{ 1,\ldots,m \},
  j_1,\ldots,j_n \in \{ 0,1,\ldots \} \} \]
with the lexicographic order. The family~$(w_i)_{i \in I} \defeq (x_1^{j_1}
\ldots x_n^{j_n} v_\ell)_{(\ell,j_1,\ldots,j_n) \in I}$ thus generates~$M$ as
an~$A$-module, and we'll call a subfamily~$(w_i)_{i \in I' \subseteq I}$
\emph{good} if and only if for all~$i \in I$, the vector~$w_i$ is a linear
combination of the vectors~$(w_{i'})_{i' \in I, i' \preceq i}$,
and if~$(\ell,j_1,\ldots,j_n) \not\in I'$ implies~$(\ell,k_1,\ldots,k_n) \not\in
I'$ for all~$k_1 \geq j_1, \ldots, k_n \geq j_n$. \XXX{Looks like in Figure XXX}

We induct on the shape of a good generating system, starting with~$(w_i)_{i \in I}$.

We show that~$(w_i)_{i \in I}$ is a basis of~$M$ by verifying the assumptions
of Lemma~\ref{lemma:basis}. Thus let~$g \in A$ be given such that one of
the~$w_i$ is an $A[g^{-1}]$-linear combination of generators with smaller index
in~$M[g^{-1}]$. Removing~$w_i = x_1^{j_1} \ldots x_n^{j_n} v_\ell$ and also all
vectors~$x_1^{k_1} \ldots x_n^{k_n} v_\ell$ where~$k_1 \geq j_1, \ldots, k_n
\geq j_n$, we obtain a subfamily which is still good. By induction, applied
to~$A[g^{-1}]$ and its module~$M[g^{-1}]$, it therefore follows that $A[g^{-1}] = 0$.
This implies that $g = 0$ since~$A$ is reduced.

Thus~$M$ is free. We finish by using the assumption for~$f = 1$.
\end{proof}


\section{The proof of the finitely-generated case}

The following proposition is just an instance of Grothendieck's generic
freeness lemma. We prove it here because it admits an easier proof.

\begin{prop}Let~$A$ be a reduced ring. Let~$M$ be a finitely
generated~$A$-module. If~$f = 0$ is the only element of~$A$ such
that~$M[f^{-1}]$ is a finite free~$A[f^{-1}]$-module, then~$1 = 0$ in~$A$.
\end{prop}

\begin{proof}We proceed by induction on the length of a given generating family
of~$M$. Let~$M$ be generated by~$(v_1,\ldots,v_m)$.

We show that the family~$(v_1,\ldots,v_m)$ is linear independent. Let~$\sum_i
a_i v_i = 0$. Over~$A[a_i^{-1}]$, the vector~$v_i \in M[a_i^{-1}]$ is a linear
combination of the other generators. Thus~$M[a_i^{-1}]$ can be generated as
an~$A[a_i^{-1}]$-module by fewer than~$m$ generators. The induction hypothesis,
applied to this module, yields that~$1 = 0$ in~$A[a_i^{-1}]$. Since~$A$ is
reduced, this amounts to~$a_i = 0$.

We finish by using the assumption for~$f = 1$.
\end{proof}

\end{document}
