\documentclass{ws-rv9x6}
\usepackage[onethmnum]{ws-rv-thm}
\usepackage{ws-rv-van}
\usepackage{mathtools,stmaryrd}
\makeindex
%\newindex{aindx}{adx}{and}{Author Index}       % author index
%\renewindex{default}{idx}{ind}{Subject Index}  % subject index

\newcommand{\A}{\mathcal{A}}
\newcommand{\B}{\mathcal{B}}
\newcommand{\C}{\mathcal{C}}
\newcommand{\M}{\mathcal{M}}
\renewcommand{\AA}{\mathbb{A}}
\newcommand{\E}{\mathcal{E}}
\newcommand{\F}{\mathcal{F}}
\newcommand{\G}{\mathcal{G}}
\newcommand{\J}{\mathcal{J}}
\newcommand{\GG}{\mathbb{G}}
\renewcommand{\O}{\mathcal{O}}
\newcommand{\K}{\mathcal{K}}
\newcommand{\NN}{\mathbb{N}}
\newcommand{\QQ}{\mathbb{Q}}
\newcommand{\RR}{\mathbb{R}}
\newcommand{\TT}{\mathbb{T}}
\newcommand{\PP}{\mathbb{P}}
\newcommand{\ZZ}{\mathbb{Z}}
\newcommand{\CC}{\mathbb{C}}
\renewcommand{\P}{\mathcal{P}}
\newcommand{\aaa}{\mathfrak{a}}
\newcommand{\ppp}{\mathfrak{p}}
\newcommand{\fff}{\mathfrak{f}}
\newcommand{\defeq}{\vcentcolon=}
\newcommand{\defeqv}{\vcentcolon\equiv}
\newcommand{\Sh}{\mathrm{Sh}}
\newcommand{\GL}{\mathrm{GL}}
\newcommand{\Zar}{\mathrm{Zar}}
\newcommand{\op}{\mathrm{op}}
\newcommand{\Set}{\mathrm{Set}}
\newcommand{\Eff}{\mathrm{Ef{}f}}
\newcommand{\Sch}{\mathrm{Sch}}
\newcommand{\Aff}{\mathrm{Aff}}
\newcommand{\Ring}{\mathrm{Ring}}
\newcommand{\LocRing}{\mathrm{LocRing}}
\newcommand{\LRS}{\mathrm{LRS}}
\newcommand{\Hom}{\mathrm{Hom}}
\newcommand{\Spec}{\mathrm{Spec}}
\newcommand{\Gal}{\mathrm{Gal}}
\newcommand{\lra}{\longrightarrow}
\newcommand{\RelSpec}{\operatorname{Spec}}
\renewcommand{\_}{\mathpunct{.}}
\newcommand{\?}{\,{:}\,}
\newcommand{\speak}[1]{\ulcorner\text{\textnormal{#1}}\urcorner}
\newcommand{\ul}[1]{\underline{#1}}
\newcommand{\affl}{\ensuremath{{\ul{\ensuremath{\AA}}^1}}}
\newcommand{\Ll}{\text{iff}}
\newcommand{\inv}{inv.\@}
\newcommand{\seq}[1]{\mathrel{\vdash\!\!\!_{#1}}}
\newcommand{\hg}{\mathbin{:}}  % homogeneous coordinates
\newcommand{\brak}[1]{{\llbracket{#1}\rrbracket}}
\newcommand{\pt}{\mathrm{pt}}
\newcommand{\Loc}{\mathrm{Loc}}
\newcommand{\Top}{\mathrm{Top}}

\begin{document}

\chapter{Generalized spaces for constructive algebra}

\author[I. Blechschmidt]{Ingo Blechschmidt}
% \index[aindx]{Blechschmidt, I.}

\address{Universität Augsburg \\
Institut für Mathematik \\
Universitätsstr. 14 \\
86159 Augsburg, Germany}

\begin{abstract}
XXX
\end{abstract}
\body

%\tableofcontents

\section{Locales}

The notion of \emph{space} is fundamental to large parts of mathematics. It
exists in various flavors, ranging from the basic cartesian spaces~$\RR^n$ to
the more general metric and topological spaces and also including several more
and slightly exotic flavors such as the diffeological spaces of differential topology.

Common to all of the mentioned flavors of space is that \emph{points are their
building blocks}: Cartesian, metric, topological and diffeological spaces are,
first and foremost, sets of points. Their spatial structure -- a metric, a
topology or a diffeology -- is additional data to the underlying set:

\begin{definition}A \emph{metric space} consists of a set~$X$ of points
together with a metric~$d : X \times X \to \RR_{\geq 0}$ satisfying the metric
axioms. \end{definition}

\begin{definition}A \emph{topological space} consists of a set~$X$ of points
together with a set~$\O(X) \subseteq \P(X)$ of point sets which are deemed
\emph{open} such that (arbitrary, set-indexed) unions and finite intersections
of open sets are open.
\end{definition}

Locales are a further and particularly unique flavor of the notion of space
which turn this classical picture upside down: Locales embrace \emph{opens
instead of points as primitive building blocks}. With locales, points are a
derived concept. In particular, the opens of a locale are not sets of points;
in fact, they needn't be sets of anything in peculiar.

The formal definition of a locale will be given below as
Definition~\ref{defn:locale}, but before we will review examples and comment on
the relevance of locales to constructive mathematics.


\subsection{Spaces without points}
\label{sect:examples-no-points}

A metric or topological space without any points is not very interesting: It is
empty, and up to isomorphism there is only one such space. In contrast, a
locale can be nontrivial even if it does not contain any points. This
phenomenon is one instance of the general guiding principle that relinquishing
points increases flexibility.

\paragraph{The locale of surjections~$\NN \twoheadrightarrow \RR$.} As is
well-known, there are no surjections from~$\NN$ to~$\RR$.\footnote{More
precisely, it is a theorem of classical mathematics that the reals are
uncountable in the sense of admitting no surjection from the naturals. The
situation is more subtle in constructive mathematics. Firstly, in the absence
of countable choice, the reals bifurcate into several distinct flavors, hence
one needs to state which flavor of the reals one is referring to. Secondly, all
known proofs of the uncountability of the Cauchy and the Dedekind reals assume
either the law of excluded middle or the axiom of countable choice. In the
absence of either of these axioms, only the MacNeille reals are known to be
uncountable~\cite{blechschmidt-hutzler:macneille}.} Hence there is no
interesting topological space of those surjections. However, there is a
well-defined and nontrivial locale~$X$ of those surjections.

The points of~$X$ are in canonical one-to-one correspondence with the
surjections~$\NN \twoheadrightarrow \RR$, hence~$X$ does not have any points.
But this locale does have uncountably many basic opens~$U_{n,x}$, where~$n$
ranges over the naturals and~$x$ ranges over the reals. We picture~$U_{n,x}$ as
the open of those surjections~$f$ for which~$f(n) = x$, we can compute with
these opens and consider functions on~$X$. For instance:
\begin{enumerate}
\item If~$x \neq y$, then the intersection of~$U_{n,x}$ with~$U_{n,y}$
is truly empty.
\item The union of the~$U_{n,x}$, where~$x$ is a fixed real
number and~$n$ ranges over the naturals, is all of~$X$. No finite number
of these opens covers~$X$, hence~$X$ is not compact.
\item For any real~$x$, there is a well-defined continuous function from~$X$ to
(the localic version of) the naturals which, on the level of points, would map
a surjection~$f : \NN \to \RR$ to the smallest preimage of~$x$.
\item There is a well-defined continuous function from~$X$ to (the localic
version of) the reals which, on the level of points, would map a surjection~$f$
to the number~$\sum_{n=0}^\infty 2^{-n} \arctan(f(n))$. Unlike the previous
example, this function is not locally constant.
\item\label{item:intersection-sublocales} The locale~$X$ is a sublocale of the
locale~$Y$ of arbitrary functions~$\NN \to \RR$ (which can also be realized as
a topological space). It can be obtained as the intersection of the uncountably
many sublocales~$Y_x$, where~$Y_x$ is the sublocale of~$Y$ consisting of those
functions which hit the real~$x$.
\item A certain sublocale of~$X$, the locale~$X'$ of those functions~$\NN \to
\RR$ for which any real has infinitely many preimages, has a fractal nature: It
is covered by the~$U_{0,x}$, where~$x$ ranges over the reals; the pairwise
intersection of these opens is (truly) empty; and they are each isomorphic
to~$X'$ -- on the level of points, by mapping a surjection~$f$ in~$U_{0,x}$ to
the surjection~$(n \mapsto f(n+1))$.
\end{enumerate}

There is nothing special about the real numbers which makes this example work;
in fact, the example works just as well with any set~$M$ in place of the reals.
For any set~$M$, there is a locale of surjections~$\NN \twoheadrightarrow M$,
and this locale is trivial (isomorphic to the empty locale) if and only if~$M$
is empty. Locales of these kind are used as an important reduction step in the
extension of Grothendieck's Galois theory by Joyal and
Tierney~\cite[Section~V.3]{joyal-tierney:galois-theory}.


\paragraph{Intersection of dense sublocales.} In ordinary topology, the
intersection of dense subspaces need not be dense. A simple example is the
intersection of~$\QQ$ with its complement in~$\RR$. In contrast, the
intersection of (even an arbitrary set of) dense sublocales is always again
dense -- even if the intersection might well have no points.

For instance, the locale-theoretic intersection of (the localic version
of)~$\QQ$ with its complement does not have any points and is dense in (the
localic version of) the reals. Intuitively, while these two sublocales do not
have any points in common, there still is nontrivial ``localic glue''.

% This basic fact about locale-theoretic denseness illustrates a more general
% phenomenon of locales: Locales tend to enjoy better formal properties.

Another example is given by item~(\ref{item:intersection-sublocales}) above:
Each of the sublocales~$Y_x$ is dense in~$Y$, hence their intersection~$X$ is
so as well.


\paragraph{The Banach--Tarski paradox.} The Banach--Tarski paradox is the
unintuitive statement that a three-dimensional solid ball in~$\RR^3$ of
radius~$r$ can be partitioned into six disjoint subsets in such a way that
rearranging those subsets using only euclidean motions yields two disjoint
solid balls of radius~$r$ each. The axiom of choice is required to construct
these subsets, and the Banach--Tarski paradox is not in
contradiction with the basic properties of the (Lebesgue) measure in~$\RR^3$
because these intermediate subsets are not measurable.

The traditional way to avoid the Banach--Tarski paradox is to adopt the
\emph{axiom of determinacy} instead of the axiom of choice. Just as the axiom
of choice posits that a certain property of the finite domain also holds for the
infinite, the axiom of determinacy is a certain statement whose finitary
analogue is provable in unadorned Zermelo--Fraenkel set theory. It entails that
all subsets of~$\RR^n$ are measurable.

The Banach--Tarski paradox can also be avoided by adopting a localic point of
view: While the localic counterparts of the six pieces do not have any points
in common, the locale-theoretic pairwise intersections are still nontrivial.
Hence one would not expect the rearrangement of these sublocales to have the
same volume as the original solid ball. [XXX: give reference]


\paragraph{Random sequences.} XXX


\subsection{Constructive concerns}

By relinquishing points, locales provide a more flexible notion of space.
Section~\ref{sect:examples-no-points} illustrates[XXX:word] this observation
with several examples of nontrivial locales without any points. Constructive
mathematics gives a further, orthogonal motivation to study locales: There are
situations in which the relevant spaces do have enough points, but only if one
subscribes to the axiom of choice or similar non-constructive principles. In
these situations, the pointfree approach allowed by locales can help to give
constructive versions of classical results.

XXX ideal objects


\paragraph{Compactness of the unit interval.} The unit interval, when realized
as a topological space, can fail to be be compact in constructive
mathematics. For instance, it fails in the Russian school because the Kleene
tree provides a computable open covering of~$[0,1]$ with no computable finite
subcovering [XXX: reference].

In contrast, the localic version of the unit interval is always compact. The
proof is by an explicit computation with its basic opens. [XXX: reference]


\paragraph{Galois theory.} Let~$L|k$ be a Galois extension. The
fundamental theorem of Galois theory states that there is a bijection between
the intermediate extensions~$L|E|k$ and the closed subgroups of the topological
Galois group~$\Gal(L|k)$. The bijection maps an intermediate extension~$L|E|k$
to the subgroup~$\Gal(L|E)$ and its inverse maps a closed subgroup~$H \subseteq
\Gal(L|k)$ to the intermediate extension~$L^H$.

Much of the proof of the fundamental theorem of Galois theory is constructive,
but some parts use the law of excluded middle and Zorn's lemma in order to
construct certain extensions of given field homomorphisms.\footnote{For
instace, the statement~$E \subseteq L^{\Gal(L|E)}$ is trivial. For the converse
inclusion, let~$x \in L^{\Gal(L|E)}$. Assume for the sake of contradiction
that~$x \not\in E$. Using Zorn's lemma and the law of excluded middle, we find
a homomorphism~$\sigma : L \to L$ with~$\sigma|_E = \operatorname{id}$
and~$\sigma(x) \neq x$. This is a contradiction to~$x \in L^{\Gal(L|E)}$.}
As a consequence, the fundamental theorem of Galois theory is not provable in
constructive mathematics.

However, this failure is not for fundamental Galois-theoretic reasons, but because of an
unfortunate choice in the definitions. There is a notion of a localic group (a
locale~$G$ equipped with continuous maps~$G \times G \xrightarrow{\circ} G$, $G
\xrightarrow{(\cdot)^{-1}} G$, $1 \xrightarrow{e} G$ satisfying the group
axioms), and the topological Galois group has a localic counterpart. The
fundamental theorem can be reformulated to refer to this localic Galois group,
and the proof of this reformulation is entirely constructive.
[XXX: references]

If one is so inclined, then one can obtain the topological version of the
fundamental theorem as a corollary of the localic version; the required
non-constructive principles for this step are neatly packaged up in the study
of the relation of the localic Galois group with the topological one.

\begin{remark}Incidentally, the fundamental theorem of Galois theory also
showcases a related general phenomenon, namely that classical mathematics
allows to push back topological concerns for longer than in constructive
mathematics, whereas in constructive mathematics we have to embrace topology
(in a sufficiently pointfree form such as locales) from the beginning.

To be more specific, classically, the basic version of the fundamental theorem
(intermediate extensions correspond to subgroups) only holds for finite Galois
extensions. For infinite Galois extensions, we have to restrict to
\emph{closed} subgroups. Constructively, the basic version cannot even be shown
for finite extensions; we have to employ spatial language even for those, or
else settle for a discrete version of the fundamental theorem: For finite field
extensions, finite intermediate extensions correspond to finite subgroups.
Classically, intermediate extensions of finite extensions and subgroups of
finite groups are automatically finite, but constructively this can fail.
\end{remark}


\paragraph{Gelfand duality} XXX


\subsection{The basics of the theory of locales}

The starting point to arrive at the definition of a locale is the following
fundamental observation: The partial order~$\O(X)$ of open subsets of a topological
space~$X$ has
\[ \begin{array}{@{}ccc@{}}
\text{arbitrary joins}
&\text{and}&
\text{finite meets}, \\
\bigvee && \wedge
\end{array} \]
and finite meets distribute over arbitrary joins:
\[ U \wedge \bigvee_i V_i = \bigvee_i (U \wedge V_i). \]

This observation motivates the following definition. [XXX: which abstracts...]
\begin{definition}A \emph{frame} is a partial order with (arbitrary,
set-indexed) joins and finite meets such that the distributive law
holds. A \emph{frame homomorphism~$\alpha : A \to A'$} is a monotone map~$A
\to A'$ which preserves arbitrary joins and finite meets.\end{definition}

The notion of a frame is (infinitarily) algebraic. To obtain a geometric
notion, we ``reverse the direction of the arrows'':

\begin{definition}\label{defn:locale}
A \emph{locale}~$X$ is given by a frame~$\O(X)$, the ``frame
of opens of~$X$''. A \emph{morphism~$f : X \to X'$ of locales} (or ``continuous
map of locales'') is a frame homomorphism~$\O(X') \to \O(X)$.\end{definition}

In place of the open sets of points, locales have arbitrary \emph{opens}, the
elements of their underlying frame. The opens of locales behave similar to the
open sets in topology in that arbitrary unions and finite intersections make
sense; but unlike before, they need not be sets of points.

Examples for locales include the following.
\begin{enumerate}
\item Any topological space~$Y$ induces a locale~$L(Y)$ by
setting~$\O(L(Y)) \defeq \O(Y)$. A continuous map~$f : Y \to Y'$ of topological
spaces induces the frame homomorphism~$\O(Y') \to \O(Y),\,U \mapsto f^{-1}[U]$
in the other direction and hence a morphism~$L(Y) \to L(Y')$ of locales in the
same direction.
\item The \emph{one-point locale}~$\pt$ is the locale induced
by the one-point topological space~$\{\star\}$. Its frame of opens is
the powerset of~$\{\star\}$, also known as the set~$\Omega$ of truth
values. Its least element is~$\bot = \emptyset$ and its largest element
is~$\top = \{\star\}$, and potentially not all elements of~$\Omega$ are equal
to of these two.
\item The locale of surjections~$\NN \twoheadrightarrow \RR$ and the localic
version of the reals of Section~\ref{sect:examples-no-points} are best
constructed as \emph{classifying locales}, a notion to be introduced below.
\end{enumerate}

Points do not appear in the definition of a locale, but they can be defined as
a derived concept:
\begin{definition}A \emph{point} of a locale~$X$ is a locale morphism~$\pt \to
X$.\end{definition}

The underlying frame homomorphism~$\alpha : \O(X) \to \Omega$ of a point~$x$
of~$X$ can be pictured as mapping each open of~$X$ to the truth value to which
extent~$x$ belongs to~$U$. Symbolically, we write~$x \inplus u$ iff~$\alpha(u)
= \top$.

The point~$x$ is completely determined by the
set~$\{ u \in \O(X) \,|\, \alpha(u) = \top \}$. This set is a \emph{completely
prime filter}, that is a subset~$\fff \subseteq \O(X)$ which is upward-closed,
closed under finite meets and for which~$\bigvee_i u_i \in \fff$ implies~$u_i
\in \fff$ for some index~$i$; and conversely, any such completely prime filter
gives rise to a point of~$X$.

\begin{definition}A locale~$X$ is \emph{spatial} if and only if its points
suffice to distinguish its opens, that is if for any opens~$u, v \in \O(X)$,
if~$x \inplus u \Leftrightarrow x \inplus v$ for all points~$x$ of~$X$, then~$u =
v$.\end{definition}

For instance, any locale induced by a topological space is spatial. The locale
of surjections~$\NN \to \RR$ is a drastic example of a locale which fails to be
spatial, and the localic real line is spatial in classical mathematics and can
fail to be spatial in constructive mathematics.

\begin{remark}The set of points of a locale~$X$ can be made into a topological space, giving
rise to a functor~$\pt : \Loc \to \Top$. This functor is right adjoint
to the functor~$L : \Top \to \Loc$. A locale~$X$ is spatial iff the canonical
morphism~$L(\pt(X)) \to X$ is an isomorphism, and a topological space~$Y$
is \emph{sober} iff the canonical morphism~$Y \to \pt(L(Y))$ is a
homeomorphism. The space~$\pt(L(Y))$ is the \emph{sobrification} of~$Y$; for
instance, the sobrification of any inhabited indiscrete space is the one-point
space.\end{remark}


\section{Sheaf models}

\section{Applications in constructive algebra}

\end{document} 

XXX cite https://apcz.umk.pl/czasopisma/index.php/LLP/article/viewFile/LLP.2013.009/732
