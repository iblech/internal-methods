\documentclass[oneside]{amsart}

\usepackage[utf8]{inputenc}
\usepackage{amsthm,mathtools,tikz,float,caption}
\usetikzlibrary{patterns,matrix}
\usepackage[protrusion=true]{microtype}
\usepackage{breakurl}
\usepackage{hyperref}

\newcommand{\NN}{\mathbb{N}}
\newcommand{\RR}{\mathbb{R}}

\title{A constructive Knaster--Tarski proof of the uncountability of the reals}
\author{Ingo Blechschmidt and Matthias Hutzler}
%\address{Max Planck Institute for Mathematics in the Sciences \\
%Inselstraße 22 \\
%04103 Leipzig, Germany}
\email{ingo.blechschmidt@mis.mpg.de, matthias.matthias.ralph.hutzler@student.uni-augsburg.de}

\theoremstyle{definition}
\newtheorem{defn}{Definition}[]
\newtheorem{ex}[defn]{Example}

\theoremstyle{plain}
\newtheorem{prop}[defn]{Proposition}
\newtheorem{cor}[defn]{Corollary}
\newtheorem{lemma}[defn]{Lemma}
\newtheorem{thm}[defn]{Theorem}
\newtheorem{scholium}[defn]{Scholium}

\theoremstyle{remark}
\newtheorem{rem}[defn]{Remark}
\newtheorem{question}[defn]{Question}
\newtheorem{speculation}[defn]{Speculation}
\newtheorem{caveat}[defn]{Caveat}
\newtheorem{conjecture}[defn]{Conjecture}

\newcommand{\XXX}[1]{\textbf{XXX: #1}}
\newcommand{\aaa}{\mathfrak{a}}
\newcommand{\bbb}{\mathfrak{b}}
\newcommand{\mmm}{\mathfrak{m}}
\newcommand{\I}{\mathcal{I}}
\newcommand{\J}{\mathcal{J}}
\newcommand{\E}{\mathcal{E}}
\newcommand{\B}{\mathcal{B}}
\renewcommand{\O}{\mathcal{O}}
\newcommand{\defeq}{\vcentcolon=}
\DeclareMathOperator{\Spec}{Spec}

\newcommand{\stacksproject}[1]{\cite[{\href{https://stacks.math.columbia.edu/tag/#1}{Tag~#1}}]{stacks-project}}

\begin{document}

\begin{abstract}
  Leaning heavily on Levy's unusual proof of the uncountability of the reals,
  we give an uncountability proof of the (MacNeille) reals which ... XXX
\end{abstract}

\maketitle
\thispagestyle{empty}

\begin{thm}Let~$f : \NN \to \RR$ be a map. Then there is a number~$x_0 \in \RR$
such that for no~$n \in \NN$, $f(n) = x_0$.\end{thm}

\begin{proof}The map
\[ g : \RR \longrightarrow \RR,\
  x \longmapsto \sup_M \sum_{n \in M} 2^{-n}, \]
where~$M$ ranges over all those finite detachable subsets of~$\NN$ such
that~$f[M] < x$, is well-defined, monotone, has a postfixpoint ($0 \leq g(0)$),
and has an upper bound for all of its postfixpoints (if~$x \leq g(x)$, then~$x
\leq 2$). By the Knaster--Tarski fixed point theorem, it therefore has a
greatest postfixpoint~$x_0$.

If~$x_0 = f(n_0)$ for a number~$n_0 \in \NN$, then for any finite detachable
subset~$M$ of~$\NN$ such that~$f[M] < x_0$,
\[ \sum_{n \in M} 2^{-n} + 2^{-n_0} = \sum_{n \in M \cup \{n_0\}} 2^{-n}
  \leq g(x_0 + 2^{-n_0}), \]
hence~$g(x_0 + 2^{-n_0}) \geq g(x_0) + 2^{-n_0} \geq x_0 + 2^{-n_0}$.
Thus~$x_0 + 2^{-n_0}$ is a greater postfixpoint that~$x_0$, a contradiction.
\end{proof}

\end{document}
