\documentclass[oneside]{amsart}

\usepackage[utf8]{inputenc}
\usepackage{amsthm,mathtools,tikz,float,caption}
\usetikzlibrary{patterns,matrix}
\usepackage[expansion=true,protrusion=true]{microtype}
\usepackage{breakurl}
\usepackage{hyperref}

\newcommand{\NN}{\mathbb{N}}
\newcommand{\RR}{\mathbb{R}}

\title{A constructive Knaster--Tarski proof of the uncountability of the reals}
\author{Ingo Blechschmidt and Matthias Hutzler}
%\address{Max Planck Institute for Mathematics in the Sciences \\
%Inselstraße 22 \\
%04103 Leipzig, Germany}
\email{ingo.blechschmidt@mis.mpg.de, matthias.ralph.hutzler@student.uni-augsburg.de}

\theoremstyle{definition}
\newtheorem{defn}{Definition}[]
\newtheorem{ex}[defn]{Example}

\theoremstyle{plain}
\newtheorem{prop}[defn]{Proposition}
\newtheorem{cor}[defn]{Corollary}
\newtheorem{lemma}[defn]{Lemma}
\newtheorem{thm}[defn]{Theorem}
\newtheorem*{thm*}{Theorem}
\newtheorem{scholium}[defn]{Scholium}

\theoremstyle{remark}
\newtheorem{rem}[defn]{Remark}
\newtheorem{question}[defn]{Question}
\newtheorem{speculation}[defn]{Speculation}
\newtheorem{caveat}[defn]{Caveat}
\newtheorem{conjecture}[defn]{Conjecture}

\newcommand{\XXX}[1]{\textbf{XXX: #1}}
\newcommand{\aaa}{\mathfrak{a}}
\newcommand{\bbb}{\mathfrak{b}}
\newcommand{\mmm}{\mathfrak{m}}
\newcommand{\I}{\mathcal{I}}
\newcommand{\J}{\mathcal{J}}
\newcommand{\E}{\mathcal{E}}
\newcommand{\B}{\mathcal{B}}
\renewcommand{\O}{\mathcal{O}}
\newcommand{\defeq}{\vcentcolon=}
\DeclareMathOperator{\Spec}{Spec}

\newcommand{\stacksproject}[1]{\cite[{\href{https://stacks.math.columbia.edu/tag/#1}{Tag~#1}}]{stacks-project}}

\begin{document}

\begin{abstract}
  We give an uncountability proof of the reals which does not rely on their
  sequential completeness, but on their order completeness. We use neither a
  form of the axiom of choice nor the law of excluded middle, therefore
  the proof applies to the MacNeille reals in any flavor of constructive
  mathematics. The proof leans heavily on Levy's unusual proof of the
  uncountability of the reals.
\end{abstract}

\maketitle
\thispagestyle{empty}

One way to verify the uncountability of the reals is as follows. We first
observe that the usual diagonalization technique shows that the powerset of the
naturals is uncountable. We then show that this powerset is in bijection with
the reals. In constructive mathematics, more precisely the kind of mathematics
which can be carried out in any topos, the first step is still valid, while the
second might fail -- for any of the several possible flavors of the reals
which in constructive mathematics might not coincide. Therefore a different
approach is needed.

This note gives a constructive proof that a specific flavor of the reals,
namely the MacNeille reals, is uncountable. To the best of our knowledge, this
is the first result in that direction. However, it is just a baby step towards
an understanding whether any of the more interesting flavors of the reals --
for instance the Cauchy reals or the Dedekind reals -- can constructively be
shown to be uncountable. The proof presented here is made possible because the
MacNeille reals -- unlike the Cauchy or Dedekind reals -- can constructively be
shown to be order-complete.

Sensibilities of constructive mathematics aside, the proof presented here is
interesting because it uses only the order completeness of the reals, not their
sequential completeness, and because it puts the Knaster--Tarski fixed point
theorem to good use. This fixed point theorem is fundamental to theoretical
computer science, but appears to be seldomly used in classical analysis.
The proof is an adaptation of Levy's unusual proof~\cite{levy}.

\begin{thm*}Let~$f : \NN \to \RR$ be a map. Then there is a number~$x_0 \in \RR$
such that for no~$n \in \NN$, $f(n) = x_0$.\end{thm*}

\begin{proof}The map
\[ g : \RR \longrightarrow \RR,\
  x \longmapsto \sup_M \sum_{n \in M} 2^{-n}, \]
where~$M$ ranges over all those finite detachable subsets of~$\NN$ such
that~$f[M] < x$, is well-defined (because the sets the suprema are taken of are
inhabited by zero and bounded from above by~$2$), monotone, has a postfixpoint
($0 \leq g(0)$), and has an upper bound for all of its postfixpoints (if~$x
\leq g(x)$, then~$x \leq 2$). By the Knaster--Tarski fixed point theorem, it
therefore has a greatest postfixpoint~$x_0$.

If~$x_0 = f(n_0)$ for a number~$n_0 \in \NN$, then for any finite detachable
subset~$M$ of~$\NN$ such that~$f[M] < x_0$,
\[ \sum_{n \in M} 2^{-n} + 2^{-n_0} = \sum_{n \in M \cup \{n_0\}} 2^{-n}
  \leq g(x_0 + 2^{-n_0}), \]
hence~$g(x_0 + 2^{-n_0}) \geq g(x_0) + 2^{-n_0} \geq x_0 + 2^{-n_0}$.
Thus~$x_0 + 2^{-n_0}$ is a greater postfixpoint that~$x_0$, a contradiction.
\end{proof}

It is possible to unwind the application of the Knaster--Tarski fixed point
theorem to obtain an entirely elementary proof of uncountability. This
unwinding makes the impredicative nature of the proof manifest.

\begin{proof}[Second proof]We consider the same map~$g : \RR \to \RR$ as in the
first proof. Let~$x_0$ be the supremum of the set~$A \defeq \{ x \in \RR \,|\,
x \leq g(x) \}$; this supremum exists because~$A$ is inhabited (by zero)
and bounded from above (by~$2$).

If~$x_0 = f(n_0)$ for a number~$n_0 \in \NN$, then~$x_0 - 2^{-n_0}$ is an
upper bound for~$A$, contradicting the fact that~$x_0$ is the least upper bound
of~$A$: Let~$x \in A$. If~$x > x_0 - 2^{-n_0}$, then~$g(x +
2^{-n_0}) \geq g(x) + 2^{-n_0} \geq x + 2^{-n_0}$, hence~$x + 2^{-n_0} \in A$,
thus~$x + 2^{-n_0} \leq x_0$, a contradiction. Hence~$x \leq x_0 - 2^{-n_0}$.
\end{proof}

\end{document}
