\documentclass[oneside,reqno]{amsart}

\usepackage[utf8]{inputenc}
\usepackage{amsthm,mathtools,stmaryrd,amssymb}
\usepackage[all]{xy}
\usepackage[protrusion=true,expansion=true]{microtype}
\usepackage{tikz}
\usepackage{proof}
\usepackage{hyperref}
\usepackage{xspace}
\usepackage{color}

\usepackage[natbib=true,style=numeric,maxnames=10]{biblatex}
\usepackage[babel]{csquotes}
\bibliography{paper-modal-multiverse.bib}

\title{The topos-theoretic multiverse}
\author{Ingo Blechschmidt}
\address{Universität Augsburg \\
Institut für Mathematik \\
Universitätsstr. 14 \\
86159 Augsburg, Germany}
\email{ingo.blechschmidt@math.uni-augsburg.de}

\theoremstyle{definition}
\newtheorem{defn}{Definition}[section]
\newtheorem{ex}[defn]{Example}

\theoremstyle{plain}
\newtheorem{prop}[defn]{Proposition}
\newtheorem{cor}[defn]{Corollary}
\newtheorem{lemma}[defn]{Lemma}
\newtheorem{thm}[defn]{Theorem}
\newtheorem{scholium}[defn]{Scholium}

\theoremstyle{remark}
\newtheorem{rem}[defn]{Remark}
\newtheorem{question}[defn]{Question}
\newtheorem{speculation}[defn]{Speculation}
\newtheorem{caveat}[defn]{Caveat}
\newtheorem{conjecture}[defn]{Conjecture}

\newcommand{\xra}[1]{\xrightarrow{#1}}
\newcommand{\XXX}[1]{\textbf{\textcolor{red}{XXX: #1}}}
\newcommand{\aaa}{\mathfrak{a}}
\newcommand{\bbb}{\mathfrak{b}}
\newcommand{\mmm}{\mathfrak{m}}
\newcommand{\nnn}{\mathfrak{n}}
\newcommand{\I}{\mathcal{I}}
\newcommand{\J}{\mathcal{J}}
\newcommand{\E}{\mathcal{E}}
\newcommand{\U}{\mathcal{U}}
\newcommand{\F}{\mathcal{F}}
\newcommand{\B}{\mathcal{B}}
\newcommand{\NN}{\mathbb{N}}
\newcommand{\TT}{\mathbb{T}}
\newcommand{\ZZ}{\mathbb{Z}}
\newcommand{\QQ}{\mathbb{Q}}
\renewcommand{\P}{\mathcal{P}}
\renewcommand{\O}{\mathcal{O}}
\newcommand{\defeq}{\vcentcolon=}
\newcommand{\defeqv}{\vcentcolon\equiv}
\newcommand{\op}{\mathrm{op}}
\DeclareMathOperator{\Spec}{Spec}
\DeclareMathOperator{\Hom}{Hom}
\DeclareMathOperator{\Mod}{Mod}
\DeclareMathOperator{\Sh}{Sh}
\DeclareMathOperator{\PSh}{PSh}
\newcommand{\Set}{\mathrm{Set}}
\newcommand{\Eff}{\mathrm{Ef{}f}}
\renewcommand{\_}{\mathpunct{.}\,}
\newcommand{\effective}{ef{}fective\xspace}
\newcommand{\notnot}{\emph{not~not}\xspace}

\newcommand{\stacksproject}[1]{\cite[{\href{https://stacks.math.columbia.edu/tag/#1}{Tag~#1}}]{stacks-project}}

% taken from JDH "The modal logic of arithmetic potentialism and the universal algorithm"
\DeclareMathOperator{\possible}{\text{\tikz[scale=.6ex/1cm,baseline=-.6ex,rotate=45,line width=.1ex]{\draw (-1,-1) rectangle (1,1);}}}
\DeclareMathOperator{\necessary}{\text{\tikz[scale=.6ex/1cm,baseline=-.6ex,line width=.1ex]{\draw (-1,-1) rectangle (1,1);}}}
\DeclareMathOperator{\xpossible}{\text{\tikz[scale=.6ex/1cm,baseline=-.6ex,rotate=45,line width=.1ex]{\draw (-1,-1) rectangle (1,1); \draw[very thin] (-.6,-.6) rectangle (.6,.6);}}}
\DeclareMathOperator{\xnecessary}{\text{\tikz[scale=.6ex/1cm,baseline=-.6ex,line width=.1ex]{\draw (-1,-1) rectangle (1,1); \draw[very thin] (-.6,-.6) rectangle (.6,.6);}}}

\newcommand{\?}{\,{:}\,}

\begin{document}

\begin{abstract}
  To help put established results in constructive algebra and constructive
  combinatorics into perspective, give possible background for certain
  inductive definitions and form a unified framework for certain techniques for
  extracting programs from classical proofs, we propose a modal study of the
  topos-theoretic multiverse. Our proposal is inspired by the corresponding study
  of the set-theoretic multiverse, but focuses less on exploring the range of
  set/topos-theoretic possibility and more on concrete applications in
  constructive mathematics.
\end{abstract}

\maketitle
\thispagestyle{empty}

\noindent
Thanks to the finer distinctions constructive mathematics offers, there is a
host of principles which are available in classical mathematics but seem naive
from a constructive point of view. A non-exhaustive list is:
\begin{enumerate}
\renewcommand{\theenumi}{\arabic{enumi}*}
\item A transitive relation is well-founded iff there is no infinite descending
chain.
\item A relation is almost-full iff every infinite sequence is good (meaning
that some term on the sequence is related to some later term -- the notion
of almost-full relations
has been studied in combinatorics~\cite{xxx} and found applications in
termination checking~\cite{xxx}).
\item \emph{Krull's lemma}: A ring element is nilpotent iff every prime
ideal contains it.
\item Every ring has a maximal ideal.\footnote{Here and in the following, by
\emph{ring} we mean commutative ring with unit and by \emph{maximal ideal} we
mean an ideal~$\mmm$ which is \emph{proper} in the sense that~$1 \in \mmm
\Rightarrow 1 = 0$ and such that for every proper ideal~$\nnn$ with~$\mmm
\subseteq \nnn$, $\mmm = \nnn$.}
\item \emph{Markov's principle:} If a function~$\NN \to \NN$ does \notnot have
a zero, then it actually has a zero.
\item \emph{Dependent choice:} If every element of a set is related by some relation to some other
element, every element can be completed to an infinite chain of related
elements.
\item The law of excluded middle holds.
\end{enumerate}
Constructive theorems always carry computational and geometric
content---from every constructive proof, a corresponding algorithm can be
extracted~\cite{bauer:c2c}, and every constructive proof holds also for continuous families of
the objects in question~\cite[Section~4.3]{blechschmidt:filmat}. But the listed classical principles have no
computational witness or fail in continuous families, hence are not
available in constructive mathematics.

In the modal topos-theoretic multiverse, we have the following constructive
replacements to these principles.
\begin{enumerate}
\item A transitive relation is well-founded iff \emph{everywhere} there is no infinite descending
chain.
\item A relation is almost-full iff every infinite sequence \emph{everywhere} is good.
\item A ring element is nilpotent iff all prime
ideals \emph{everywhere} contain it.
\item Every ring \emph{proximally} has a maximal ideal.
\item If a function~$\NN \to \NN$ does \emph{everywhere} \notnot have
a zero, then it actually has a zero.
\item If every element of a set is related by some relation to some other
element, every element can \emph{proximally} be completed to an infinite chain
of related elements.
\item \emph{Barr's theorem, simple version:} \emph{Somewhere,} the law of
excluded middle holds.
\end{enumerate}

Briefly, a statement~$\varphi$ is said to hold \emph{everywhere}
($\necessary\varphi$) iff it holds in every Grothendieck topos over the
current base topos; a statement holds \emph{somewhere} ($\possible\varphi$)
iff it holds in some positive Grothendieck topos over the current base; and a
statement holds \emph{proximally} ($\xpossible\varphi$) iff it holds in
some positive ouvert Grothendieck topos over the current base. More such
\emph{modalities} are also useful and merit study; the precise definitions are
given in Section~\ref{sect:defn-modalities}.

The modal language not only allows us to recover classical principles
as above, but also makes some powerful theorems about the topos-theoretic
landscape smoothly accessible:
\begin{enumerate}
\addtocounter{enumi}{7}
\item \emph{Barr's theorem, full version:} If Zorn's lemma holds, it is
everywhere the case that it (and even the full axiom of choice) hold somewhere.
\item If a geometric implication holds \emph{somewhere}, then it holds already here.
\item \label{item:prox-descends}
If a first-order statement holds \emph{proximally}, then it holds already here.
\item \label{item:prox-countable}
For every (perhaps uncountable) inhabited set~$X$, \emph{proximally} there is a
surjection~$\NN \twoheadrightarrow X$.
\end{enumerate}
An example application of the latter two principles
has recently been studied in constructive
commutative algebra~\cite{blechschmidt-schuster:constructive-maximal-ideals}.
For countable rings, an explicit construction of a maximal ideal is available.
By item~(\ref{item:prox-countable}), this construction can also be carried out
for arbitrary rings, though the result is not a maximal ideal in the narrow
sense; rather, the resulting maximal ideal exists \emph{proximally}, in some
positive ouvert Grothendieck topos. However, the first-order consequences of
its existence, pertaining for instance to concrete statements about polynomials
or matrices, pass down to the base topos by item~(\ref{item:prox-descends}).
In this sense the proximally-existing maximal ideal is like the mathematical
phantoms of Gavin Wraith~\cite{xxx}, encouraging us to broaden our notion of
existence because it promises us to work wonders.

We can also adopt the notions of \emph{switches} and \emph{buttons} from the
modal study of the set-theoretic multiverse~\cite[xxx]{xxx}. Switches are
statements~$\varphi$ such that~$\necessary(\possible\varphi \wedge
\possible\neg\varphi)$ and buttons are statements~$\varphi$ such
that~$\necessary\possible\necessary\varphi$; switches can be toggled on and off
like a light switch, while buttons once pressed cannot be unpressed:
\begin{enumerate}
\addtocounter{enumi}{10}
\item \emph{The law of excluded middle is a switch:} Everywhere it is the case
that somewhere~\textsc{lem} holds and somewhere it does not.
\item \emph{Being countable is a button:} For every set~$X$, everywhere it is
the case that somewhere (even proximally) it is case that everywhere~$X$ is
countable.
\end{enumerate}

We argue that the modal operators~$\necessary,\possible,\xpossible$ and more
suggested in our proposal are natural extensions and refinements of the
familiar double-negation modality~$\neg\neg$ in constructive mathematics.

\textbf{Acknowledgments.} Alexander Oldenziel, \ldots

xxx: idea of multiverse not new, but arguably basic to topos theory; new is systematic
study of its modal nature with the focus on applications

xxx: related work: Hamkins, Victoria Gitman, ... concentrates on exploring
range of set-theoretic possibility, obtaining beautiful results such as ...
Shawn Henry, Andreas Blass ...


\section{Modal operators}
\label{sect:defn-modalities}

\begin{defn}A \emph{Grothendieck topos} is a category which is equivalent to
the category of sheaves on a small site. A \emph{Grothendieck topos over a
given (Grothendieck or elementary) topos}~$\B$ is ``a Grothendieck topos from
the point of view of~$\B$''; using a sufficiently expressive form of the
internal language of~$\B$, which allows to make direct sense of the notion in
scare quotes, this amounts to a bounded geometric morphism~$\E \to
\B$, and this can be taken as the definition.
\end{defn}

Henceworth, by the unqualified word ``topos'' we will mean ``Grothendieck topos
over the current base'', and the current base topos will be denoted~``$\Set$''.
The base topos might be the ``true category of sets'', assuming that this
concept is available in one's ontology, or also some other elementary topos
such as the free topos~\cite{xxx} or the ef{}fective topos~\cite{xxx}.

The xxx (use word of Hamkins) of a base~$\B$ is the collection of Grothendieck
toposes over~$\B$ (which one should formalize in set-theoretic foundations more
precisely as the proper class of small sites in~$\B$).

\begin{rem}Perhaps arbitrary \emph{elementary} toposes over the base, corresponding to
possibly unbounded geometric morphisms~$\E \to \B$ or even arbitrary
fibrations/indexed categories over~$\B$ validating an appropriate form of the
axioms of elementary toposes, should also be
taken as part of the topos-theoretic multiverse. The first foray into the
modal topos-theoretic multiverse outlined in this note
sticks to Grothendieck toposes for ease of
formalizability (``for every small site'' can be expressed also in the more
standard flavors of the internal topos language, while ``for every elementary
topos'' requires more elaborate versions); because the restricted multiverse
is already sufficiently rich for the intended applications; and because we have
a generalization to the predicative setting~\cite{crosilla:xxx} with arithmetic
universes in mind. Predicatively, not even the category of sets might be an
elementary topos, but they do form an arithmetic universe and sheaves over
sites are a meaningful concept.
\end{rem}


\subsection{Positive toposes}

\begin{defn}A topos~$\E$ is \emph{positive} if and only if the geometric
morphism~$f : \E \to \Set$ is surjective, that is iff~$f^*$ reflects
isomorphisms.\end{defn}

\begin{ex}The topos of sheaves over a topological space~$X$ is positive if and
only if~$X$ has a point; more generally, the topos of sheaves over a locale~$X$
is positive if and only if every open covering of the top element of the frame
of~$X$ is inhabited (xxx:verify). As such, positivity is a more informative
version of the weaker property of being nontrivial (the property that the top open
and the bottom open do not coincide).\end{ex}

\begin{ex}The spectrum of a ring~$A$, that is the classifying topos of prime
filters of~$A$ (or equivalently the topos of sheaves over the classifying
locale of prime filters of~$A$) is positive iff~$1 \neq 0$ in~$A$.\end{ex}

\begin{ex}A sufficient criterion for the classifying topos of a geometric
theory~$\TT$ being positive is that xxx.\end{ex}


\subsection{Ouvert toposes}

\begin{defn}A topos~$\E$ is \emph{ouvert} if and only if the geometric
morphism~$f : \E \to \Set$ is open, that is preserves the interpretation of
first-order formulas.\end{defn}

\begin{ex}The topos of sheaves over a topological space is always ouvert. More
generally, the topos of sheaves over a locale~$X$ if ouvert iff there exists a
\emph{positivity predicate} on its frame of opens in the sense of~\cite{xxx}.
With~\textsc{lem}, every locale is ouvert~\cite{xxx}.\end{ex}

\begin{ex}The spectrum of a ring~$A$ is ouvert iff every element of~$A$ is
nilpotent or not~\cite[Proposition~12.51]{blechschmidt:phd}.\end{ex}

\begin{ex}A sufficient criterion for the classifying topos of a geometric
theory~$\TT$ being ouvert is that the indexing sets of all disjunctions
appearing on the right hand side of turnstiles, in a normal form presentation
of~$\TT$, are inhabited~\cite{xxx}.\end{ex}


\subsection{Modal operators}

xxx: properly define language


\section{Generic models and inductive definitions}
\label{sect:generic-models}

In constructive mathematics, the classical definition of well-founded relations
as those transitive relations for which there exist no infinite descending
chains is not particularly useful; while the chain condition is satisfied for
the intended examples, by its negativity the condition is too weak to
facilitate the intended proofs.

The established substitute is to declare that a transitive relation~$({<})$ on a set~$X$
is well-founded if and only if for every subset~$M \subseteq X$,
\begin{equation}\label{eq:ho-wf}\tag{$\star$}
\bigl(\forall x\?X\_ (\forall y\?X\_ y < x \Rightarrow y \in M) \Rightarrow x \in
M\bigr) \Longrightarrow \bigl(\forall x\?X\_ x \in M\bigr).
\end{equation}
More economically, and preferably in predicative contexts where there is no
single set or class of ``all subsets of~$X$'' but for instance a
hierarchy of subsets of increasing universe levels, a transitive relation~$({<})$ is
well-founded if and only if every element of~$X$ is \emph{accessible}, where
the accessibility predicate~$\mathsf{Acc}$ is inductively generated~\cite{xxx}
by the following clause:
\[
  \infer{\mathsf{Acc}(x)}{\forall y<x\_\ \mathsf{Acc}(y)}
\]
In impredicative settings, this inductive definition of well-foundedness
coincides with the higher-order characterization~(\ref{eq:ho-wf}), and for the
purposes of this paper we view the inductive definition as the official one.

Similar inductive notions are used to reformulate other classical definitions
in a constructively more sensible way. For instance, the classical definition
of a binary relation~$R$ on a set~$X$ being \emph{almost-full} is ``every
infinite sequence~$\alpha : \NN \to X$ is \emph{good} in the sense that there
exist numbers~$i < j$ such that~$\alpha(i) \mathrel{R} \alpha(j)$''.
For a constructive reformulation, we shift to finite approximations of
infinite sequences (finite lists of elements of~$X$) and define when such an
approximation is deemed good:\footnote{By~``$\sigma[n]$'', we mean the element
at position~$n$ of the finite list~$\sigma$. This notation is only meaningful
if the length of~$\sigma$ is at least~$n+1$. By~``$\sigma :: x$'' we mean the
enlarged list which has~$x$ as an extra element at its tail end, and by~``$[]$'' we
denote the empty list. In computer science practice, it is often more efficient
to prepend (``$x :: \sigma$'') instead of append, but this detail shall not
concern us here.}
\[ \mathsf{Good}(\sigma) \defeqv (\exists i<j\_ \sigma[i] \mathrel{R} \sigma[j]). \]
We then inductively generate a relation~``$P \,|\, \sigma$'' for monotonous
predicates~$P$ on finite lists expressing that no matter how the given finite
approximation~$\sigma$ evolves over time to a better approximation,
eventually~$P$ will hold, by the clauses
\[
  \infer{P \,|\, \sigma}{P(\sigma)}
  \qquad
  \infer{P \,|\, \sigma}{\forall x \? X\_\ P \,|\, (\sigma :: x)}
\]
and finally define that~$R$ is almost-full iff~``$\mathsf{Good} \,|\, []$''.
With this inductive definition, expected properties of the class of almost-full
relations such as stability under cartesian products (Dickman's lemma),
finite lists (Higman's lemma) or finitely-branching trees (Kruskal's theorem)
can all be constructively verified~\cite{xxx}.

A similar such definition has been proposed by XXX and XXX in commutative
algebra for expressing that a ring is Noetherian~\cite{xxx}; the classical
definition ``every ascending chain of ideals stabilizes'' and also the more
meaningful and classically equivalent characterization as ``every ascending
chain of finitely generated ideals stalls''\footnote{A chain~$\aaa_0 \subseteq
\aaa_1 \subseteq \cdots$ \emph{stalls} iff for some number~$n \in \NN$,
$\aaa_{n+1} = \aaa_n$. We are grateful to Matthias Hutzler for proposing this
terminology.} are constructively too weak; firstly, without the axiom of
dependent choice we can often not construct such chains~\cite{richman:xxx} (but only
``multi-valued chains'' as in~\cite[Section~3.9]{blechschmidt:phd}; but also
see~\cite{richman:xxx}), and secondly, being able to inspect suitable inductive
witnesses helps in proving the Hilbert basis theorem~\cite{xxx}. XXX and XXX
hence propose to call a ring \emph{Noetherian} if and only if~$\mathsf{Stalls}
\,|\, []$, where~$\mathsf{Stalls}$ is the predicate on finite lists of finitely
generated ideals expressing xxx.

Is there a deeper explanation where these inductive definitions come from,
apart from working well and being motivated on general constructive
considerations? Also, constructively the inductive definitions are much
stronger than their classical counterparts, equivalent only in presence
of~\textsc{lem} and~\textsc{dc}. For instance, if a relation is almost-full in
the inductive sense, not only is every infinite sequence good, but also every
infinite ``multi-valued sequence''\footnote{xxx} and every infinite
partially-defined sequence~$\alpha$ for which for every number~$n \in \NN$ it
is \notnot the case that~$\alpha(n)$ exists. Can we pinpoint how much stronger
the inductive definitions are?

The answers to both questions are positive (xxx:language), and the modal
perspective fruitfully clarifies their connection.

Namely, the theories of an infinite sequence and of an infinite descending chain
are geometric. As such, there exist their classifying toposes, containing the
\emph{generic infinite sequence} respectively the \emph{generic infinite
descending chain}, and we may ask: When is this sequence good respectively when
does this chain validate~$\bot$?

\begin{prop}\label{prop:gen-good}Let~$R$ be a relation on a set~$X$. The
generic infinite sequence over~$X$ is good if and only if~$R$ is almost-full in
the inductive sense.\end{prop}

\begin{proof}The classifying topos of the theory of an infinite sequence
over~$X$ can be presented as the topos of sheaves over the site given by the
partially ordered set of finite lists of elements of~$X$ with coverage given
by xxx (see Appendix xxx). The Kripke--Joyal semantics states that the
statement ``$\alpha_0$ is good'', where~$\alpha_0$ is the generic infinite
sequence, holds in the classifying topos if and only if there is a
covering~$\U$ of~$[]$ such that for every open~$U \in \U$, xxx. This precisely
amounts to~$R$ being almost-full in the inductive sense.
\end{proof}

\begin{prop}Let~$({<})$ be a transitive relation on a set~$X$. The generic
infinite descending chain over~$X$ validates~$\bot$ (that is, the classifying
topos of such chains is trivial) if and only if the relation is well-founded in
the inductive sense.\end{prop}

\begin{proof}Similar as the proof of Proposition~\ref{prop:gen-good}. Details
for the variant of ``bad sets'' instead of ``infinite descending chains'' have
been developed (xxx:language) by xxx~\cite{xxx}.\end{proof}


\section{Extracting programs from multiverse proofs}
\label{sect:program-extraction}

\begin{prop}Let~$({\leq})$ be a transitive almost-full relation.
Then~$({<})$, where~$x < y \equiv (x \leq y \wedge \neg(y \leq x))$,
is well-founded.\end{prop}

\begin{proof}Everywhere, there can be no infinite descending chain, as any
such would also be good.\end{proof}

Unrolling this proof gives a program of type~$(\textsf{Good} \,|\, []) \to
\prod_{x:X} \textsf{Acc}(x)$.

\begin{thm}[Dickson's lemma] If~$X$ and~$Y$ are almost-full,
so is~$X \times Y$.\end{thm}

\begin{proof}
  \begin{enumerate}
    \item It suffices to verify that the \emph{generic infinite
    sequence}~$\gamma = (\alpha,\beta) : \NN \to X \times Y$ is good. Since
    being good can be put as a geometric implication (in fact, a geometric
    formula) and since \textsc{lem} holds \emph{somewhere}, we may assume~\textsc{lem}.

    \item By \textsc{lem} and well-foundedness, there is a minimal value~$\alpha(i_0)$
    among all values of~$\alpha$.
    Similarly, there is a minimal value~$\alpha(i_1)$ among~$(\alpha(n))_{n >
    i_0}$, a minimal value~$\alpha(i_2)$ among~$(\alpha(n))_{n > i_1}$, and so
    on. By \emph{proximal dependent choice}, we can proximally collect these indices
    into a function~$i : \NN \to \NN$;\footnote{We could also avoid dependent
    choice by picking not some suitable indices~$i_0, i_1, \ldots$, but the
    least such.} this switches~\textsc{lem} off.

    \item Switching~\textsc{lem} on again, there is a minimal value~$\beta(i(k_0))$
    among all values of~$\beta \circ i$. Hence~$\gamma$ is good in view of
    \begin{align*}
      \alpha(i(k_0)) &\leq \alpha(i(k_0+1)), &
      \beta(i(k_0)) &\leq \beta(i(k_0+1)). \qedhere
    \end{align*}
  \end{enumerate}
\end{proof}


\section{Perspectives}

xxx: determine modal laws

xxx: more case studies

xxx: don't forget right adjoints

xxx: predicative

\end{document}

Cite? https://arxiv.org/abs/2210.04838
