\documentclass[oneside]{amsart}

\usepackage[utf8]{inputenc}
\usepackage{amsthm,mathtools,stmaryrd,amssymb}
\usepackage[all]{xy}
\usepackage[protrusion=true,expansion=true]{microtype}
\usepackage{hyperref}
\usepackage{xspace}
\usepackage{color}

\usepackage[natbib=true,style=numeric,maxnames=10]{biblatex}
\usepackage[babel]{csquotes}
\bibliography{paper-modal-multiverse.bib}

\title{The topos-theoretic multiverse}
\author{Ingo Blechschmidt}
\address{Universität Augsburg \\
Institut für Mathematik \\
Universitätsstr. 14 \\
86159 Augsburg, Germany}
\email{ingo.blechschmidt@math.uni-augsburg.de}

\theoremstyle{definition}
\newtheorem{defn}{Definition}[section]
\newtheorem{ex}[defn]{Example}

\theoremstyle{plain}
\newtheorem{prop}[defn]{Proposition}
\newtheorem{cor}[defn]{Corollary}
\newtheorem{lemma}[defn]{Lemma}
\newtheorem{thm}[defn]{Theorem}
\newtheorem{scholium}[defn]{Scholium}

\theoremstyle{remark}
\newtheorem{rem}[defn]{Remark}
\newtheorem{question}[defn]{Question}
\newtheorem{speculation}[defn]{Speculation}
\newtheorem{caveat}[defn]{Caveat}
\newtheorem{conjecture}[defn]{Conjecture}

\newcommand{\xra}[1]{\xrightarrow{#1}}
\newcommand{\XXX}[1]{\textbf{\textcolor{red}{XXX: #1}}}
\newcommand{\aaa}{\mathfrak{a}}
\newcommand{\bbb}{\mathfrak{b}}
\newcommand{\mmm}{\mathfrak{m}}
\newcommand{\nnn}{\mathfrak{n}}
\newcommand{\I}{\mathcal{I}}
\newcommand{\J}{\mathcal{J}}
\newcommand{\E}{\mathcal{E}}
\newcommand{\F}{\mathcal{F}}
\newcommand{\B}{\mathcal{B}}
\newcommand{\NN}{\mathbb{N}}
\newcommand{\ZZ}{\mathbb{Z}}
\newcommand{\QQ}{\mathbb{Q}}
\renewcommand{\P}{\mathcal{P}}
\renewcommand{\O}{\mathcal{O}}
\newcommand{\defeq}{\vcentcolon=}
\newcommand{\defeqv}{\vcentcolon\equiv}
\newcommand{\op}{\mathrm{op}}
\DeclareMathOperator{\Spec}{Spec}
\DeclareMathOperator{\Hom}{Hom}
\DeclareMathOperator{\Mod}{Mod}
\DeclareMathOperator{\Sh}{Sh}
\DeclareMathOperator{\PSh}{PSh}
\newcommand{\Set}{\mathrm{Set}}
\newcommand{\Eff}{\mathrm{Ef{}f}}
\renewcommand{\_}{\mathpunct{.}\,}
\newcommand{\effective}{ef{}fective\xspace}
\newcommand{\notnot}{\emph{not~not}\xspace}

\newcommand{\stacksproject}[1]{\cite[{\href{https://stacks.math.columbia.edu/tag/#1}{Tag~#1}}]{stacks-project}}

\begin{document}

\begin{abstract}
  To help put earlier results in constructive algebra and constructive
  combinatorics into perspective, explain a possible background for certain
  inductive definitions and give a unified framework for certain techniques for
  program extraction from classical proofs, we propose a modal study of the
  topos-theoretic multiverse.
\end{abstract}

\maketitle
\thispagestyle{empty}

\noindent
Thanks to the finer distinctions constructive mathematics offers, there is a
host of principles which are available in classical mathematics but seem naive
from a constructive point of view. A non-exhaustive list is:
\begin{enumerate}
\item A transitive relation is well-founded iff there is no infinite descending
chain.
\item A relation is almost-full iff every infinite sequence is good (meaning
that some element is related to a later element in the sequence).
\item \emph{Krull's lemma}: A ring element is nilpotent iff every prime
ideal contains it.
\item Every ring has a maximal ideal.\footnote{Here and in the following, by
\emph{ring} we mean commutative ring with unit and by \emph{maximal ideal} we
mean an ideal~$\mmm$ which is \emph{proper} in the sense that~$1 \in \mmm
\Rightarrow 1 = 0$ and such that for every proper ideal~$\nnn$ with~$\mmm
\subseteq \nnn$, $\mmm = \nnn$.}
\item \emph{Markov's principle:} If a function~$\NN \to \NN$ does \notnot have
a zero, then it actually has a zero.
\item \emph{Dependent choice:} If every element of a set is related by some relation to some other
element, every element can be completed to an infinite chain of related
elements.
\item The law of excluded middle holds.
\end{enumerate}
Constructive theorems always have a computational and geometric
content---from every constructive proof, a corresponding algorithm can be
extracted, and every constructive proof holds also for continuous families of
the objects in question. But the listed classical principles have no
computational witness or fail in continuous families, hence are not
available in constructive mathematics.

In the modal topos-theoretic multiverse, we have the following constructive
replacements to these principles.
\begin{enumerate}
\item A transitive relation is well-founded iff \emph{everywhere} there is no infinite descending
chain.
\item A relation is almost-full iff every infinite sequence \emph{everywhere} is good (meaning
that some element is related to a later element in the sequence).
\item \emph{Krull's lemma}: A ring element is nilpotent iff all prime
ideals \emph{everywhere} contain it.
\item Every ring \emph{proximally} has a maximal ideal.
\item If a function~$\NN \to \NN$ does \emph{everywhere} \notnot have
a zero, then it actually has a zero.
\item If every element of a set is related by some relation to some other
element, every element can \emph{proximally} be completed to an infinite chain
of related elements.
\item \emph{Somewhere,} the law of excluded middle holds.
\end{enumerate}

Briefly, a statement is said to hold \emph{everywhere} iff it holds in every
Grothendieck topos over the current base topos; a statement holds
\emph{somewhere} iff it holds in some positive Grothendieck topos over the
current base; and a statement holds \emph{proximally} iff it holds in some
positive ouvert Grothendieck topos over the current base. More such
\emph{modalities} are also useful and merit study.

This modal language not only allows us to recover certain classical principles
as above, but also makes certain powerful theorems about the topos-theoretic
landscape easy to use:
\begin{enumerate}
\addtocounter{enumi}{7}
\item For every (perhaps uncountable) set~$X$, \emph{proximally} there is a
surjection~$\NN \twoheadrightarrow X$.
\end{enumerate}
We can also adopt the notion of \emph{switches} and \emph{buttons} from the
modal study of the set-theoretic multiverse:
\begin{enumerate}
\addtocounter{enumi}{8}
\item \emph{The law of excluded middle is a switch:} Everywhere it is the case
that somewhere~\textsc{lem} holds and somewhere it does not.
\item If Zorn's lemma holds, it is also everythere the case that it holds
somewhere.
\end{enumerate}

\textbf{Acknowledgments.} Alexander Oldenziel, \ldots

\end{document}
