\documentclass[oneside,reqno]{amsart}

\usepackage[utf8]{inputenc}
\usepackage[english]{babel}
\usepackage{amsthm,mathtools,stmaryrd,amssymb,graphicx}
\usepackage{booktabs}
\usepackage[all]{xy}
\usepackage[protrusion=true,expansion=true]{microtype}
\usepackage{xspace}

% lifted from https://arxiv.org/abs/1506.08870
\DeclareFontFamily{U}{min}{}
\DeclareFontShape{U}{min}{m}{n}{<-> udmj30}{}
\newcommand\yon{\!\text{\usefont{U}{min}{m}{n}\symbol{'210}}\!}

\usepackage[natbib=true,style=numeric,maxnames=10]{biblatex}
\usepackage[babel]{csquotes}
\bibliography{paper-qcoh.bib}

\title{A general Nullstellensatz for generalized spaces}
\author{Ingo Blechschmidt}
\email{ingo.blechschmidt@univr.it}

\theoremstyle{definition}
\newtheorem{defn}{Definition}[section]
\newtheorem{ex}[defn]{Example}

\theoremstyle{plain}
\newtheorem{prop}[defn]{Proposition}
\newtheorem{cor}[defn]{Corollary}
\newtheorem{lemma}[defn]{Lemma}
\newtheorem{thm}[defn]{Theorem}
\newtheorem{scholium}[defn]{Scholium}

\theoremstyle{remark}
\newtheorem{rem}[defn]{Remark}
\newtheorem{question}[defn]{Question}
\newtheorem{speculation}[defn]{Speculation}
\newtheorem{caveat}[defn]{Caveat}
\newtheorem{conjecture}[defn]{Conjecture}

\newenvironment{indentblock}{%
  \list{}{\leftmargin\leftmargin}%
  \item\relax
}{%
  \endlist
}

\newcommand{\xra}[1]{\xrightarrow{#1}}
\newcommand{\aaa}{\mathfrak{a}}
\newcommand{\bbb}{\mathfrak{b}}
\newcommand{\mmm}{\mathfrak{m}}
\newcommand{\C}{\mathcal{C}}
\newcommand{\I}{\mathcal{I}}
\newcommand{\J}{\mathcal{J}}
\newcommand{\E}{\mathcal{E}}
\newcommand{\F}{\mathcal{F}}
\newcommand{\B}{\mathcal{B}}
\renewcommand{\AA}{\mathbb{A}}
\newcommand{\EE}{\mathbb{E}}
\newcommand{\NN}{\mathbb{N}}
\newcommand{\RR}{\mathbb{R}}
\newcommand{\TT}{\mathbb{T}}
\newcommand{\ZZ}{\mathbb{Z}}
\renewcommand{\SS}{\mathbb{S}}
\renewcommand{\O}{\mathcal{O}}
\newcommand{\defeq}{\vcentcolon=}
\newcommand{\op}{\mathrm{op}}
\DeclareMathOperator{\Spec}{Spec}
\DeclareMathOperator{\Hom}{Hom}
\DeclareMathOperator{\Sh}{Sh}
\DeclareMathOperator{\PSh}{PSh}
\DeclareMathOperator{\rank}{rank}
\DeclareMathOperator{\length}{length}
\DeclareMathOperator{\List}{List}
\DeclareMathOperator{\id}{id}
\newcommand{\Set}{\mathrm{Set}}
\newcommand{\Eff}{\mathrm{Ef{}f}}
\renewcommand{\_}{\mathpunct{.}\,}
\newcommand{\effective}{ef{}fective\xspace}
\newcommand{\?}{\,{:}\,}
\newcommand{\realizes}{\Vdash}
\newcommand{\notnot}{\emph{not~not}\xspace}
\usepackage{soul}\setul{0.3ex}{}
\let\oldul\ul
\renewcommand{\ul}[1]{\text{\oldul{$#1$}}}
\newcommand{\affl}{\ensuremath{{\ul{\AA}^1}}\xspace}
\newcommand{\speak}[1]{\ulcorner\text{\textnormal{#1}}\urcorner}
\newcommand{\brak}[1]{\llbracket #1 \rrbracket}
\newcommand{\Mod}[1]{{#1}\mathrm{\text{-}mod}}

\newcommand{\stacksproject}[1]{\cite[{\href{https://stacks.math.columbia.edu/tag/#1}{Tag~#1}}]{stacks-project}}

\renewcommand{\paragraph}[1]{\noindent\textbf{#1.}}

\newcommand{\ZFC}{\textsc{zfc}}

\newcommand{\seq}[1]{\mathrel{\vdash\!\!\!_{#1}}}

\begin{document}

\begin{abstract}
  We give a general Nullstellensatz for the generic model of a
  geometric theory, useful as a source of nongeometric sequents validated by
  the generic model, and characterize the first-order and higher-order formulas validated by the
  genetric model.
\end{abstract}

\maketitle
\thispagestyle{empty}

\section{Introduction}

\paragraph{Generic models} Let~$\TT$ be a geometric theory, such as the theory
of rings, of local rings or of intervals. We follow Caramello's
terminology~\cite{caramello:tst} to mean by \emph{geometric theory} a system
given by a set of sorts, a set of finitary function symbols, a set of finitary
relation symbols and a set of axioms, consisting of geometric sequents
(sequents of the form~$\varphi \seq{\vec x} \psi$ where~$\varphi$ and~$\psi$
are geometric formulas, that is formulas built from equality and the relation
symbols by the logical connectives~${\top}\,{\bot}\,{\wedge}\,{\vee}\,{\exists}$
and by arbitrary set-indexed disjunctions~$\bigvee$). By (infinitary) \emph{first-order
formula} we will mean a formula which may contain, additionally to the
connectives allowed for geometric formulas, the connectives~${\Rightarrow}$
and~${\forall}$.

A fundamental result is that there is a \emph{generic model}~$U_\TT$ of~$\TT$,
a model such that for any geometric sequent~$\sigma$, the following notions
coincide:
\begin{enumerate}
\item The sequent~$\sigma$ is provable modulo~$\TT$.
\item The sequent~$\sigma$ holds for any~$\TT$-model in any Grothendieck topos.
\item The sequent~$\sigma$ holds for~$U_\TT$.
\end{enumerate}
One could argue that it is this model which we
implicitly refer to when we utter the phrase ``Let~$M$ be a~$\TT$-model.''. It can
typically not be realized as a set-theoretic model, consisting of a set for
each sort, a function for each function symbol and so on; instead it is a model
in a custom-tailored syntactically constructed Grothendieck topos, the
\emph{classifying topos}~$\Set[\TT]$ of~$\TT$, hence consists of an object
of~$\Set[\TT]$ for each sort, a morphism for each function symbol and so on.

To state what it means for a~$\TT$-structure in a topos~$\E$ to verify the
axioms of~$\TT$, rendering it a model, the \emph{internal language} of~$\E$ is
used, roughly reviewed in Section~\ref{sect:review-language} below. We write~``$\E
\models \alpha$'' to mean that a formula~$\alpha$ holds from the internal
point of view of~$\E$. Since this language is a form of a higher-order
intuitionistic extensional dependent type theory, the classifying topos~$\Set[\TT]$ can
be regarded as a higher-order completion of the geometric theory~$\TT$. The
generic model enjoys the universal property that any~$\TT$-model in any
(Grothendieck) topos~$\E$ is the pullback of~$U_\TT$ along an essentially
unique geometric morphism~$\E \to \Set[\TT]$.
\medskip

\paragraph{Nongeometric sequents} Crucially, the
equivalence~$(1)\Leftrightarrow(2)\Leftrightarrow(3)$ relating
provability and truth in~$\Set[\TT]$ only pertains to geometric sequents. The
generic model may validate additional nongeometric sequents which are not
provable from the axioms of~$\TT$ in first-order or even higher-order logic,
and these nongeometric sequents may be quite surprising and have useful
consequences.

One of the most celebrated such sequents arises in the case that~$\TT$ is the
theory of local rings. In this case, the classifying topos~$\Set[\TT]$ is also
known as the \emph{big Zariski topos} of~$\Spec(\ZZ)$ from algebraic geometry,
the topos of sheaves over the site of schemes locally of finite presentation,
and the generic model is the functor~$\affl$ of points of the affine line, the
functor which maps any (l.o.f.p.) scheme~$X$ to~$\Hom(X,\mathbb{A}^1) =
\O_X(X)$.

From the point of view of the topos, the ring object~$\affl$ is not only a
local ring, but even a field in the sense that
any nonzero element is invertible.
As this condition is of nongeometric form, it is not inherited by arbitrary
local rings, which are indeed typically not fields. However, any intuitionistic
consequence of this condition which is of geometric form is inherited
by any local ring in any topos. Hence we may, when verifying a general fact
about local rings which is expressible as a geometric sequent,
suppose without loss of generality that the given ring is a
field. This observation is due to Kock~\cite{kock:univ-proj-geometry}, who exploited it to
develop projective geometry over local rings, and was further used by Reyes to
prove a Jacobian criterion for étale morphisms~\cite{reyes:cramer}.

A related nongeometric sequent is valid in the little Zariski topos of the
spectrum of a ring~$A$, the classifying topos of local localizations of~$A$.
If~$A$ is reduced, the generic model validates the dual condition that any
noninvertible element is zero. This property has been used to give a short and
even constructive proof of Grothendieck's generic freeness lemma,
substantially improving on previously published
proofs~\cite{blechschmidt:generic-freeness}.

In time, further nongeometric sequents holding in the big Zariski topos of an
arbitrary base scheme have been found~\cite[Section~18.4]{blechschmidt:phd}. These include:
\begin{itemize}
\item $\affl$ is \emph{anonymously algebraically closed} in the sense that
any monic polynomial~$p \? \affl[T]$ of degree at least one does \notnot have a
zero. \smallskip
\item The Nullstellensatz holds:
Let $f_1,\ldots,f_m \in \affl[X_1,\ldots,X_n]$ be polynomials without a common
zero in~$(\affl)^n$. Then there are polynomials~$g_1,\ldots,g_m \in
\affl[X_1,\ldots,X_n]$ such that~$\sum_i g_i f_i = 1$. \smallskip
\item Any function~$\affl \to \affl$ is given by a unique polynomial. \smallskip
\item $\affl$ is microaffine: Let~$\Delta = \{ \varepsilon \?
\affl \,|\, \varepsilon^2 = 0 \}$. Let~$f : \Delta \to \affl$ be an arbitrary
function. Then there are unique elements~$a, b \? \affl$ such
that~$f(\varepsilon) = a + b \varepsilon$ for all~$\varepsilon \? \Delta$. \smallskip
\item $\affl$ is \emph{synthetically quasicoherent}: For any
finitely presentable~$\affl$-algebra~$A$, the canonical homomorphism~$A \to
(\affl)^{\Spec(A)}$, where~$\Spec(A)$ is defined as the set
of~$\affl$-algebra homomorphisms~$A \to \affl$, is bijective.
\end{itemize}
All of these nongeometric sequents are useful for the purposes of synthetic
algebraic geometry, the desire to carry out algebraic geometry in a language
close to the simple language on the 19th and the beginning of the 20th century
while still being fully rigorous and fully general, working over arbitrary base
schemes instead of restricting to the field of complex numbers.
\medskip

\paragraph{Characterizing nongeometric sequents} Referring to one of the previous
examples, Tierney remarked around the time that those sequents were first
studied that ``[it] is surely important, though its precise significance is
still somewhat obscure---as is the case with many such nongeometric
formulas''~\cite[p.~209]{tierney:spectrum}. In view of their importance, is
there a way to discover nongeometric sequents in a systematic fashion? To
characterize the nongeometric sequents holding in classifying toposes? To this
end, Wraith put forward a specific conjecture~\cite[p.~336]{wraith:intuitionistic-algebra}:
\begin{quote}
\emph{The problem of characterising all the non-geometric properties of a generic
model appears to be difficult. If the generic model of a geometric theory~$\TT$
satisfies a sentence~$\alpha$ then any geometric consequence of $\TT + \alpha$ has to be a
consequence of~$\TT$. We might call~$\alpha$ $\TT$-redundant. Does the
generic~$\TT$-model satisfy all~$\TT$-redundant sentences?}
\end{quote}
This question was recently answered in the negative by Bezem, Buchholtz and
Coquand~\cite{bezem-buchholtz-coquand:syntactic-forcing-models}; hence the
characterization we propose is necessarily more nuanced.

Our starting point was the empirical
observation~\cite[p.~164]{blechschmidt:phd} that in the case of the big
Zariski topos, every true known nongeometric sequent followed from just a
single such, namely the synthetic quasicoherence of the generic
model, and in earlier work we surmised that one could formulate an appropriate
metatheorem explaining this observation and generalizing it to arbitrary
classifying toposes~\cite[Speculation~22.1]{blechschmidt:phd}. This hope turned
out to be true, in the sense we will now indicate.
\medskip

\paragraph{A general Nullstellensatz} To explain the relevant background, the
somewhat vague question ``to which extent does the classifying
topos~$\Set[\TT]$ realize that it is the classifying topos for~$\TT$?'' is a
useful guiding principle. This is easiest to visualize with a concrete example
for~$\TT$, such as the theory of rings.

Let~$A$ be a ring. A simple version of the classical Nullstellensatz states:
\emph{For any polynomials~$f$ and~$g$ over~$A$, if any zero of~$f$ is also a
zero of~$g$, then there is a polynomial~$h$ such that~$g = hf$.} The
polynomial~$h$ can be regarded as an ``algebraic certificate'' of the
hypothesis. This principle holds for instance in the case that~$A$ is an
algebraically closed field and~$g$ is the unit polynomial. We will see
below that it is also
true, without any restriction on~$g$, for the generic ring.

We could try to generalize the Nullstellensatz to arbitrary geometric theories~$\TT$ as
follows: \emph{For any geometric sequent~$\sigma$, if~$\sigma$ holds for a
given~$\TT$-model~$M$ then~$\sigma$ is provable modulo~$\TT$.} In place of the
algebraic certificate we now have a logical certificate, a \emph{proof}
of~$\sigma$. However, this generalized statement is typically false, even for
the generic model~$U_\TT$: The statement
\begin{multline*}
  \qquad\qquad\qquad\Set[\TT] \models \ulcorner\text{for any geometric sequent~$\sigma$,} \\
    \text{if~$\sigma$ holds for~$U_\TT$ then~$\ul{\TT}$
    proves~$\sigma$}\urcorner\qquad\qquad\qquad
\end{multline*}
does not hold.\footnote{Here~$\ul{\TT}$ is the internal geometric theory induced
by~$\TT$, obtained by pulling back the set of sorts, the set of function
symbols and so on along the geometric morphism~$\Set[\TT] \to \Set$. For
instance, if~$\TT$ is the theory of rings, then from the internal point of view
of~$\Set[\TT]$ the theory~$\ul{\TT}$ will again be the theory of rings.
More details will be given in Section~\ref{sect:review-theories}. The corner quotes indicate
that for sake of readability, the translation into formal language is to be
carried out by the reader. \par The displayed statement is much stronger than
the statement that for any geometric sequent~$\sigma$, if $\Set[\TT] \models
\speak{$\sigma$ holds for~$U_\TT$}$ then~$\TT$ proves~$\sigma$. This latter statement,
where the universal quantifier and the ``if \ldots then'' have been pulled out,
is true.} In this sense~$\Set[\TT]$ does not believe that~$U_\TT$ is the
generic~$\ul{\TT}$-model.

A concrete counterexample is as follows. Let~$\TT$ be the theory of rings and let~$\sigma$ be
the sequent~$(\top \vdash 1 + 1 = 0)$. Since there is an intuitionistic proof
that~$\TT$ does not prove~$\sigma$ and toposes are sound with respect to
intuitionistic logic, the statement~$\speak{\ul{\TT} proves~$\sigma$}$ is false
from the internal point of view of~$\Set[\TT]$. However, it is not the case
that the statement~$\speak{$1 + 1 = 0$ in~$U_\TT$}$ is false from the internal
point of view. In fact, this statement holds in a nontrivial slice
of~$\Set[\TT]$, the open subtopos coinciding with the classifying topos of
the theory of rings of characteristic two.

Intuitively, the problem is that while the meaning of~$\speak{$\ul{\TT}$
proves~$\sigma$}$ is fixed, the meaning of~$\speak{$\sigma$ holds for~$U_\TT$}$ can vary
with the slice. This problem can be solved by passing from~$\ul{\TT}$ to a
varying theory, the internal theory~$\ul{\TT}/U_\TT$ defined in
Section~\ref{sect:main}. If~$\TT$ is the theory of rings, then~$\ul{\TT}/U_\TT$
is the~$\Set[\TT]$-theory of~$U_\TT$-algebras.
Unlike~$\ul{\TT}$, this theory is not the pullback of an external geometric
theory. We then have, subject to some qualifications made precise in
Section~\ref{sect:main}, the following general Nullstellensatz:

\begin{thm}\label{thm:nullstellensatz0}
Let~$\TT$ be a geometric theory. Then, internally to~$\Set[\TT]$:
\begin{equation}\tag{$\dagger$}\label{eq:nullstellensatz0}
\text{A geometric$^*$ sequent~$\sigma$ holds for~$U_\TT$ if and only
if~$\ul{\TT}/U_\TT$ proves$^*$~$\sigma$.}
\end{equation}
\end{thm}

To illustrate Theorem~\ref{thm:nullstellensatz0}, let~$\TT$ be the theory of
rings and let~$\sigma$ be the sequent~$(f(x) = 0 \seq{x} g(x) = 0)$ for some
polynomials~$f$ and~$g$. To say that~$\sigma$ holds for~$U_\TT$ amounts to
saying that any zero~$x \? U_\TT$ of~$f$ is also a zero of~$g$, and to say
that~$\ul{\TT}/U_\TT$ proves~$\sigma$ amounts to saying that
in~$U_\TT[X]/(f(X))$, the free~$U_\TT$-algebra on one generator~$X$ subject to
the relation~$f(X) = 0$, the relation~$g([X]) = 0$ holds. Hence we obtain
\[ \Set[\TT] \models
  \forall f,g \? U_\TT[X]\_ \bigl(
    (\forall x \? U_\TT\_ f(x) = 0 \Rightarrow g(x) = 0) \Longleftrightarrow
      \exists h \? U_\TT[X]\_ g = hf\bigr). \]

The statement~\eqref{eq:nullstellensatz0} is not a geometric sequent. Therefore
it is not to be expected that it passes from~$\Set[\TT]$ to a
subtopos~$\Set[\TT']$ corresponding to a quotient theory~$\TT'$ of~$\TT$, and
indeed in general it does not. However, there is still a useful substitute,
which we formulate as Theorem~\ref{thm:nullstellensatz-horn}. This substitute
substantially broadens the scope of the Nullstellensatz.

Summarizing, the situation is as follows.
\begin{itemize}
\item The generic model~$U_\TT$ is a
conservative~$\TT$-model. \smallskip
\item The topos~$\Set[\TT]$ does not believe that~$U_\TT$ is a
conservative~$\ul{\TT}$-model. \smallskip
\item The topos~$\Set[\TT]$ does believe that~$U_\TT$
is a conservative$^\star$~$\ul{\TT}/U_\TT$-model.
\end{itemize}

Theorem~\ref{thm:nullstellensatz0} is a source of nongeometric sequents. Indeed,
it is the universal such source in the sense that any first-order formula
which holds for~$U_\TT$ can be deduced from~\eqref{eq:nullstellensatz0}:

\begin{thm}\label{thm:characterization0}
Let~$\TT$ be a geometric theory. Let~$\alpha$ be a first-order
formula over the signature of~$\TT$. Then the following statements are
equivalent.
\begin{enumerate}
\item The formula~$\alpha$ holds for~$U_\TT$. \smallskip
\item The formula~$\alpha$ is provable in first-order intuitionistic logic
modulo the axioms of~$\TT$ and the additional axiom~\eqref{eq:nullstellensatz0}.
\end{enumerate}
\end{thm}

Theorem~\ref{thm:characterization0} characterizes the first-order formulas
which hold for the generic model. We could of course wish for a more explicit
characterization; but since even the characterization of geometric sequents
holding for the generic model (they are precisely those which are provable in
geometric logic modulo~$\TT$) is of a rather implicit nature, this wish appears
unfounded.

We stress that our characterization is more explicit than the tautologous
characterization (``a first-order formula holds for~$U_\TT$ iff it is provable
modulo~$\TT'$, where~$\TT'$ is the first-order theory whose set of axioms is the
set of first-order formulas satisfied by~$U_\TT$'') and the (incorrect)
characterization ``a first-order formula holds for~$U_\TT$ iff it
is~$\TT$-redundant''. Indeed, if~$\TT$ happens to be coherent and recursively
axiomatizable, then in stating Theorem~\ref{thm:characterization0} we may
restrict to coherent existential fixed-point logic, and the resulting theory
will again be recursively axiomatizable.
\medskip


\paragraph{Outline} In Section~\ref{sect:review}, we review background on the
internal language of toposes, classifying toposes and internal geometric theories.
Section~\ref{sect:main} contains proofs of the main theorems in the
full generality of geometric theories. Restricting to Horn theories allows for
a treatment which is more algebraic and less logical in flavor. For the benefit
of readers with a more algebraic background, we include a mostly self-contained
account of the Horn case as Section~\ref{sect:horn}. We generalize our main
theorems to the higher-order case in Section~\ref{sect:higher-order} and
conclude with applications in Section~\ref{sect:applications}.

Throughout we work in a constructive metatheory, to allow our results to be
interpreted internally to toposes.
\medskip


\paragraph{Acknowledgments} XXX


\section{Background}
\label{sect:review}

\subsection{Background on the internal language of Grothendieck toposes}
\label{sect:review-language}

% XXX


\subsection{Background on classifying toposes}
\label{sect:review-classifying-toposes}

We use the usual convention of abbreviating~``$x_1\?X_1,\ldots,x_n\?X_n$''
as~``$\vec x \? \vec X$'' or even just~``$\vec x$''.

\begin{defn}The \emph{syntactic site}~$\C_\TT$ of a geometric theory~$\TT$ has:
\begin{enumerate}
\item as objects ``geometric formulas in contexts''~$\{x_1\?X_1,\ldots,x_n\?X_n\_
\varphi\}$ where~$\varphi$ is a geometric formula over the signature of~$\TT$
in the displayed context; \smallskip
\item as set of morphisms~$\Hom_{\C_\TT}(\{\vec x\_
\varphi\}, \{\vec y\_ \psi\})$ the set of formulas~$\theta$ in the
context~$\vec x, \vec y$ which are~$\TT$-provably functional,
modulo~$\TT$-provable equivalence of such formulas; \smallskip
\item as covering families those families~$(\{\vec x_i\_ \varphi_i\} \xrightarrow{\theta_i}
\{\vec y\_ \psi\})_i$ for which~$\TT$ proves the sequent~$(\psi \seq{\vec y} \bigvee_i \exists \vec x_i\_
\theta_i)$.
\end{enumerate}
\end{defn}

\begin{defn}The \emph{classifying topos}~$\Set[\TT]$ of a geometric
theory~$\TT$ is the topos of set-valued sheaves on~$\C_\TT$.\end{defn}

Writing~$\yon : \C_\TT \to \Set[\TT]$ for the Yoneda embedding, the \emph{generic
model}~$U_\TT$ of~$\TT$ interprets a sort~$X$ of~$\TT$ as the sheaf~$\yon\{x\?X\_ \top\}$,
a function symbol~$f : X_1 \cdots X_n \to Y$ as the morphism given by
the~$\TT$-provably functional formula~$f(x_1,\ldots,x_n) = y$ and a relation
symbol~$R \rightarrowtail X_1 \cdots X_n$ by the subobject~$\yon\{\vec x\_ R(\vec
x)\} \rightarrowtail \yon\{\vec x\_ \top\}$.

\begin{thm}The generic model is universal in the sense that for any
Grothendieck topos~$\E$, the functor
\[ \text{(category of geometric morphisms~$\E \to \Set[\TT]$)} \longrightarrow
\text{(category of~$\TT$-models in~$\E$)} \]
given by~$f \mapsto f^*U_\TT$ is an equivalence of categories.
\end{thm}

\begin{prop}\label{prop:basic-truth}
Let~$\alpha$ and~$\varphi$ be geometric formulas in a context~$\vec
x$ over the signature of a geometric theory~$\TT$. Then the following
statements are equivalent:
\begin{enumerate}
\item $\Set[\TT] \models \forall \vec x\_ (\alpha \Rightarrow \varphi)$.
\smallskip
\item $\{\vec x\_ \alpha\} \models \varphi$, where the free variables
in~$\varphi$ are interpreted as their generic values over~$\{\vec
x\_\alpha\}$, that is the projection maps~$\{ \vec x\_ \alpha \} \to \{
x_i\?X_i\_ \top \})$. \smallskip
\item $\TT$ proves~$(\alpha \seq{\vec x} \varphi)$.
\end{enumerate}
\end{prop}

\begin{proof}The equivalence~$(1) \Leftrightarrow (2)$ follows immediately by
unrolling the Kripke--Joyal semantics. The equivalence~$(2) \Leftrightarrow
(3)$ is by induction on the structure of~$\varphi$.\end{proof}


\subsection{Background on internal geometric theories}
\label{sect:review-theories}

The notions of signatures, geometric theories and classifying toposes can be
relativized to the internal world of arbitrary toposes with natural numbers
objects. Basics on internal signatures, internal geometric
theories and internal classifying toposes are
folklore~\cite[p.~334]{wraith:intuitionistic-algebra}; a careful treatment is
due to Shawn Henry~\cite{henry:phd}.

Briefly, an internal signature~$\Sigma$ internal to a topos~$\E$ consists of an
object of sorts, an object of function symbols, an object of relation symbols,
and various morphisms indicating the sorts involved with the function and
relation symbols.

Given an internal signature~$\Sigma$ internal to a Grothendieck topos~$\E$ (or
elementary topos with a natural numbers object), we can successively build the
object of contexts (the object of lists of sorts); the object of terms
(equipped with a morphism to the object of contexts); the object of atomic
propositions (again equipped with such a morphism); the object of geometric
formulas (again so); the object of geometric sequents (again so). A internal
geometric theory~$\TT$ over~$\Sigma$ is then given by a subobject of the object
of geometric sequents, interpreted as the object of axioms of~$\TT$. Given such
an internal theory~$\TT$, we can build the object of proof trees of~$\TT$.

From the internal point of view of~$\E$, these objects can be obtained by
simply carrying out the usual constructions of the set of contexts, the set of
terms and so on. Having the object of proof trees at hand, we can define
the notion of provability:

\begin{defn}\label{defn:provability}
Let~$\TT$ be a geometric theory internal to a Grothendieck topos~$\E$.
Internally to~$\E$, an element~$\sigma$ of the object of geometric sequents is
\emph{provable modulo~$\TT$} if and only if there is an element of the object
of proof trees of~$\TT$ which has~$\sigma$ as its conclusion.\end{defn}

\begin{ex}An ordinary geometric theory is the same as a geometric theory
internal to the topos~$\Set$. A geometric sequent is provable in the ordinary
sense if and only if, from the internal point of view of~$\Set$, it is provable
in the sense of Definition~\ref{defn:provability}.\end{ex}

\begin{ex}An ordinary geometric theory~$\TT$ over an ordinary
signature~$\Sigma$ can be pulled back along a geometric morphism~$\E \to \Set$
to yield an internal geometric theory~$\ul{\TT}$ over the internal
signature~$\ul{\Sigma}$ in~$\E$. The object of sorts of~$\ul{\Sigma}$ is the pullback of
the set of sorts of~$\Sigma$, and so on.\end{ex}

Disjunctions appearing in internal geometric formulas may be indexed by
arbitrary objects of the topos, just like disjunctions appearing in ordinary
external geometric formulas over an ordinary signature may be indexed by
arbitrary sets. If~$\Sigma$ is an internal signature in a Grothendieck topos~$\E$, the object of geometric
formulas over~$\Sigma$ has an important
subobject, the subobject of those formulas such that locally, any appearing
disjunction is indexed by a constant sheaf. Such internal geometric formulas
will be called \emph{geometric$^\star$ formulas}.

Correspondingly, there is a subobject of the object of proof trees, the object
of~\emph{proof$^\star$ trees} where any occurring geometric formula is a
geometric$^\star$ formula. The shape of such a proof$^\star$ tree is locally
given by the shape of an ordinary external proof tree.

\begin{defn}\label{defn:provability*}
Let~$\TT$ be a geometric theory internal to a Grothendieck topos~$\E$.
Internally to~$\E$, an element~$\sigma$ of the object of geometric$^\star$
sequents is \emph{provable$^\star$ modulo~$\TT$} if and only if there is an
element of the object of proof$^\star$ trees of~$\TT$ which has~$\sigma$ as its
conclusion.\end{defn}

The following lemma shows that for coherent theories, there is no difference
between provability and provability$^*$.

\begin{lemma}Let~$\E$ be a Grothendieck topos. Let~$\TT$ be a coherent theory
internal to~$\E$. Let~$\sigma$ be a coherent sequent over the signature
of~$\TT$. Then the following statements are equivalent.
\begin{enumerate}
\item $\E \models \speak{There is a~$\TT$-derivation of~$\sigma$ of externally
finite shape}.$ \smallskip
\item $\E \models \speak{There is a~$\TT$-derivation of~$\sigma$ of arbitrary
external shape}$, that is~$\E \models \speak{$\sigma$ is provable$^\star$
modulo~$\TT$}$. \smallskip
\item $\E \models \speak{There is a~$\TT$-derivation of~$\sigma$ of arbitrary
internal shape}$, that is~$\E \models \speak{$\sigma$ is provable
modulo~$\TT$}$. \smallskip
\item $\E \models \speak{There is a~$\TT$-derivation of~$\sigma$ of internally
finite shape}.$
\end{enumerate}
\end{lemma}

\begin{proof}It is trivial that~(1) implies~(2) implies~(3).

To verify that~(3) implies~(4), we can mimic the usual proof of this fact in
the internal language of~$\E$: There is a variant of the syntactic
site of~$\TT$ which is built using only coherent sequents and finitary
derivability~\cite[Section~1.4]{caramello:tst}. The sheaf topos over this site
is another model for the classifying topos of~$\TT$, and still validates, like
any Grothendieck topos, full infinitary logic. Hence, if~$\sigma$
is~$\TT$-derivable by a proof tree of arbitrary shape, then~$\sigma$ holds in
this model of the classifying topos. By the analogue of
Proposition~\ref{prop:basic-truth} for this model, we obtain
that~$\sigma$ is~$\TT$-derivable by a proof tree of finite shape.

That~(4) implies~(1) is a routine exercise exploiting that
\[ \E \models \forall X\_ \speak{$X$ is Kuratowski-finite} \Leftrightarrow
  \bigvee_{n \geq 0} \exists x_1,\ldots,x_n \? X\_
  \forall x \? X\_ \bigvee_{i=1}^n x = x_i. \qedhere \]
\end{proof}


\section{Proofs of the main theorems}
\label{sect:main}

Given a geometric theory~$\TT$ with its generic model~$U_\TT$ in~$\Set[\TT]$,
the main theorems reference a certain internal
theory~$\ul{\TT}/U_\TT$ internal to~$\Set[\TT]$. This theory is defined as
follows.

\begin{defn}\label{defn:tu}
Let~$\TT$ be a geometric theory. The theory~$\ul{\TT}/U_\TT$ is the
geometric theory internal to~$\Set[\TT]$ which arises from the pulled-back
theory~$\ul{\TT}$ by adding additional constant symbols~$e_x$ of the
appropriate sorts, one for each element~$x : U_\TT$, axioms~$(\top \vdash
f(e_{x_1},\ldots,e_{x_n}) = e_{f(x_1,\ldots,x_n)})$ for each function
symbol~$f$ and~$n$-tuple of elements of~$U_\TT$ (of the appropriate sorts), and
axioms~$(\top \vdash R(e_{x_1},\ldots,e_{x_n}))$ for each relation symbol~$R$
and~$n$-tuple~$(x_1,\ldots,x_n)$ (of the appropriate sorts) such
that~$R(x_1,\ldots,x_n)$ holds for~$U_\TT$.\end{defn}

From the point of view of~$\Set[\TT]$, a model of~$\ul{\TT}/U_\TT$ is a model
of~$\ul{\TT}$ equipped with a~$\ul{\TT}$-homomorphism from~$U_\TT$. In
particular, the identity~$(U_\TT \to U_\TT)$ is a model of~$\ul{\TT}/U_\TT$.
This is what we mean when we say that~$U_\TT$ is in a canonical way
a~$\ul{\TT}/U_\TT$-model.

\begin{ex}Let~$\TT$ be the theory of rings. Then~$\ul{\TT}/U_\TT$ is, from the
internal point of view of~$\Set[\TT]$, the theory of~$U_\TT$-algebras.\end{ex}

\begin{ex}Let~$\TT$ be a geometric theory. Let~$M$ be a model of~$\TT$ in the
category of sets. Let~$f : \Set \to \Set[\TT]$ be the corresponding geometric
morphism. Then~$f^*(\ul{\TT}/U_\TT)$ is the theory of~$M$-algebras
($\TT$-models equipped with a~$\TT$-homomorphism from~$M$). This is
because~$f^*\ul{\TT} = \TT$, $f^*U_\TT = M$ and because the construction of the
theory~$\ul{\TT}/U_\TT$ is geometric.\end{ex}

\begin{rem}From the internal point of view of~$\Set[\TT]$, we can construct the
classifying topos of~$\ul{\TT}/U_\TT$. Externally, this construction gives rise
to a bounded topos over~$\Set[\TT]$, hence to a Grothendieck topos. Using for
instance the technique described
in~\cite{blechschmidt-hutzler-oldenziel:composition}, one can show that this topos classifies the
theory of homomorphisms between~$\TT$-models. It can also be obtained as the
lax pullback~$(\Set[\TT] \Rightarrow_{\Set[\TT]} \Set[\TT])$.\end{rem}

\begin{lemma}\label{lemma:truth-to-provability}
Let~$\TT$ be a geometric theory. Let~$\alpha$ be a geometric formula over the
signature of~$\TT$ in a context~$x_1\?X_1,\ldots,x_n\?X_n$. Then
\[ \{ \vec x\_ \alpha \} \models \speak{$\ul{\TT}/U_\TT$ proves $(\top
\vdash_{[]} \alpha)$}, \]
where the free variables~$\vec x$ occurring in~$\alpha$ are interpreted as in
Proposition~\ref{prop:basic-truth}.
\end{lemma}

\begin{proof}By induction on the structure of~$\alpha$. The cases
of~``$\top$'' and ``$\wedge$'' are trivial; the cases of~``$\bigvee$''
and~``$\exists$'' follow from passing to suitable coverings; and the case of
atomic propositions is by definition of~$\ul{\TT}/U_\TT$.
\end{proof}

\begin{lemma}\label{lemma:locally-constant}
Let~$\TT$ be a geometric theory.
Let~$\varphi$ be a section of the sheaf of geometric$^\star$ formulas over the signature
of~$\ul{\TT}/U_\TT$ over a stage~$A \in \C_\TT$. Then there is a covering~$(A_i \to A)_i$
of~$A$ such that for each index~$i$, there is a formula~$\varphi_i$ over the signature
of~$\TT/U_\TT(A_i)$ such that~$A_i \models \speak{$\ul{\TT}/U_\TT$ proves$^\star$
$(\varphi \dashv\vdash \varphi_i)$}$.
\end{lemma}

\begin{proof}By passing to a covering, we may suppose that~$\varphi$ is given
by an (external) geometric formula over the signature
of~$\ul{\TT}(A)/U_\TT(A)$.

Any function symbol and relation symbol
of~$\ul{\TT}(A)$ occurring in~$\varphi$ is locally given by a symbol of~$\TT$.
Hence the claim would be trivial if~$\varphi$ were a coherent formula, for in
this case we would just have to pass to further coverings, one for each
occurring symbol, a finite number of times in total.

However, in general, we cannot conclude as easily. Write~$A = \{ \vec x\_
\alpha \}$. Let~$R$ be a relation symbol of~$\ul{\TT}(A)$ occurring
in~$\varphi$. By the explicit description of constant sheaves as sheaves of
locally constant maps, there is a covering~$(\{ \vec y_j\_ \alpha_j \}
\xrightarrow{[\theta_j]} \{ \vec x\_ \alpha \})_j$ such that, restricted to~$\{
\vec y_j\_ \alpha_j \}$, $R$ is given by a relation symbol~$R_j$ of~$\TT$. To construct
the desired formula~$\varphi'$, we replace any such occurrence~$R(\ldots)$
in~$\varphi$ by
\[ \bigvee_j \bigl((\exists \vec y_j\_ \theta_j) \wedge R_j(\ldots)\bigr). \]
In a similar vein we treat any occurrence of function symbols.

The resulting formula~$\varphi'$ is a geometric formula over the signature
of~$\TT/U_\TT(A)$.
The verification of~$A \models \speak{$\ul{\TT}/U_\TT$ proves$^\star$
$(\varphi \dashv\vdash \varphi')$}$ rests on the observation
\[ A \models \speak{$\ul{\TT}/U_\TT$ proves$^\star$ $\bigl((\exists \vec y_k\_
\theta_k) \seq{[]} \bigvee_\ell \{ \top \,|\, \text{$(\exists \vec y_\ell\_ \theta_\ell)$
holds for~$U_\TT$} \}\bigr)$} \]
which in turn can be checked on the covering~$(\{ \vec y_j\_ \alpha_j \}
\xrightarrow{[\theta_j]} \{ \vec x\_ \alpha \})_j$, applying
Lemma~\ref{lemma:truth-to-provability} and using that~$\TT$ (and
hence~$\ul{\TT}$) proves~$(\exists \vec y_j\_ \theta_j) \wedge (\exists \vec
y_k\_ \theta_k) \seq{\vec x} \bigvee \{ \top \,|\, j = k \}$.
\end{proof}

\begin{thm}\label{thm:nullstellensatz}
Let~$\TT$ be a geometric theory. Then, internally to~$\Set[\TT]$, for any
geometric$^\star$ sequent~$\sigma$ over the signature of~$\ul{\TT}/U_\TT$, the
following statements are equivalent:
\begin{enumerate}
\item The sequent~$\sigma$ holds for~$U_\TT$. \smallskip
\item The sequent~$\sigma$ is provable$^\star$ modulo~$\ul{\TT}/U_\TT$.
\end{enumerate}
\end{thm}

\begin{proof}The direction~$(2) \Rightarrow (1)$ is immediate because~$U_\TT$ is, from the
internal point of view of~$\Set[\TT]$, a~$\ul{\TT}/U_\TT$-model. Hence even the
following stronger statement holds internally: \emph{For any geometric
sequent~$\sigma$, if~$\ul{\TT}/U_\TT$ proves~$\sigma$, then~$\sigma$ holds
for~$U_\TT$.}

For the direction~$(1) \Rightarrow (2)$ we have to verify that, given any
stage~$A \in \C_\TT$ and any section~$\sigma$ of the sheaf of
geometric$^\star$ sequents over~$A$, if~$A \models
\speak{$\sigma$ holds for~$U_\TT$}$ then~$A \models
\speak{$\ul{\TT}/U_\TT$ proves$^\star$~$\sigma$}$. By
Lemma~\ref{lemma:locally-constant} we may suppose that~$\sigma$ is an
(external) geometric sequent over the signature of~$\TT/U_\TT(A)$.

Writing~$A = \{ \vec x\_ \alpha \}$ and~$\sigma = (\varphi \seq{\vec y} \psi)$,
we have~$\{ \vec x\_ \alpha \} \models \forall \vec y\_ (\varphi \Rightarrow \psi)$,
hence~$\{ \vec x, \vec y\_ \alpha \wedge \varphi \} \models \psi$. Thus~$\TT$
proves~$(\alpha \wedge \varphi \seq{\vec x, \vec y} \psi)$. This proof can be
pulled back from~$\Set$ to~$\Set[\TT]/\yon A$ to show~$A \models
\speak{$\ul{\TT}/U_\TT$ proves$^\star$~$(\alpha
\wedge \varphi \seq{\vec x, \vec y} \psi)$}$. By
Lemma~\ref{lemma:truth-to-provability}, we also have~$A \models
\speak{$\ul{\TT}/U_\TT$ proves$^\star$~$(\top \seq{[]} \alpha)$}$ (where the free
variables occurring in~$\alpha$ are interpreted as the generic values available
over~$A$), hence~$A \models \speak{$\ul{\TT}/U_\TT$ proves$^\star$~$(\varphi
\seq{\vec y} \psi)$}$.
\end{proof}

\begin{rem}Theorem~\ref{thm:nullstellensatz} cannot be strengthened to
arbitrary geometric sequents. For instance, in the case that~$\TT$ is the
theory of rings, the internal geometric sequent~$(\top \seq{x\?U_\TT}
\bigvee_{a\?U_\TT} (x = e_a))$ trivially holds for~$U_\TT$. However, it is not
provable modulo~$\TT/U_\TT$, as for instance the polynomial algebra~$U_\TT[X]$
does not validate it.\end{rem}

\begin{thm}\label{thm:nullstellensatz-horn}
Let~$\TT$ be a geometric theory. Let~$\TT'$ be a quotient theory
of~$\TT$. Assume that the generic model~$U_\TT$ is a sheaf for the topology
on~$\Set[\TT]$ cutting out the subtopos~$\Set[\TT']$. Then the following
statement holds internally to~$\Set[\TT']$:
\[ \text{A geometric$^\star$ sequent~$\sigma$ with Horn consequent holds for~$U_{\TT'}$
  iff~$\ul{\TT}/U_\TT$ proves$^\star$~$\sigma$.} \]
\end{thm}

\begin{proof}In general, the generic model of~$\TT'$ is the pullback of the
generic model of~$\TT$ to the
subtopos~$\Set[\TT']$~\cite[Lemma~2.3]{caramello:definability}. By the sheaf
assumption, the objects~$U_{\TT'}$ and~$U_\TT$ actually agree, that is~$U_\TT$
is contained in the subtopos and has the universal property of~$U_{\TT'}$.

The ``if'' direction is trivial, as~$U_{\TT'}$ is a~$\ul{\TT}/U_\TT$-model.

For the ``only if'' direction, we use that a statement holds in~$\Set[\TT']$ if
and only if its~$\nabla$-translation holds in~$\Set[\TT]$, where~$\nabla$ is
the modal operator associated to the topology cutting
out~$\Set[\TT']$~\cite[Theorem~6.31]{blechschmidt:phd}. Exploiting some of the
simplification rules of the~$\nabla$-translation~\cite[Section~6.6]{blechschmidt:phd},
it hence suffices to verify, internally to~$\Set[\TT]$, that:
\begin{multline*}
  \text{\emph{For any geometric$^\star$ sequent~$\sigma = (\varphi \seq{\vec x} \psi)$ where~$\psi$ is a Horn formula,}} \\
    \text{\emph{if~$\forall x_1,\ldots,x_n\?U_\TT\_ (\varphi \Rightarrow \nabla
    \psi)$, then~$\ul{\TT}/U_\TT$ proves$^\star$~$\sigma$.}}
\end{multline*}
Since~$\nabla$ commutes with finite conjunctions and since the sheaf assumption
implies that $\nabla(s = t)$ is equivalent
to~$s = t$ and that, for relation symbols~$R$, the
statement~$\nabla(R(s_1,\ldots,s_m))$ is equivalent to~$R(s_1,\ldots,s_m)$,
the statement~$\nabla\psi$ is equivalent to~$\psi$. Hence the claim follows
from Theorem~\ref{thm:nullstellensatz}.
\end{proof}

A situation in which the sheaf assumption of Theorem~\ref{thm:nullstellensatz-horn}
is satisfied is when~$\TT$ is a Horn theory and the topology cutting
out~$\Set[\TT']$ is subcanonical. For instance, this is the case if~$\TT$ is
the theory of rings and~$\Set[\TT']$ is one of several well-known toposes in
algebraic geometry such as the big Zariski topos, the big étale topos or the
big fppf topos.

Theorem~\ref{thm:nullstellensatz} cannot be strengthened to arbitrary
first-order (or first-order$^\star$ or even just coherent) formulas in place of
geometric$^\star$ sequents. For instance, in the case that~$\TT$ is the theory
of local rings, the generic model~$U_\TT$ validates the coherent
formula~$\speak{any element which is not zero is invertible}$, but~$\TT/U_\TT$
does not prove this fact, as it is for instance not validated by the polynomial
algebra~$U_\TT[X]$. However, Theorem~\ref{thm:nullstellensatz} still plays an
important role in understanding first-order formulas:

\begin{thm}\label{thm:characterization}
Let~$\TT$ be a geometric theory. Let~$\alpha$ be a first-order
formula over the signature of~$\TT$. Then the following statements are
equivalent.
\begin{enumerate}
\item The formula~$\alpha$ holds for~$U_\TT$. \smallskip
\item The formula~$\alpha$ is provable in first-order intuitionistic logic
modulo the axioms of~$\TT$ and the additional axioms
\begin{equation}\label{eq:nullstellensatz}\tag{$\ddagger$}
  \speak{$\sigma$ holds for~$U_\TT$} \Longrightarrow \speak{$\ul{\TT}/U_\TT$
  proves$^\star$~$\sigma$}
\end{equation}
where~$\sigma$ ranges over the geometric formulas over the signature of~$\TT$.
\end{enumerate}
\end{thm}

In stating the axiom scheme~\eqref{eq:nullstellensatz}, we are slightly abusing
notation. At face value, this axiom scheme would only make sense in the internal
language of~$\Set[\TT]$, where the generic model~$U_\TT$ actually lives.
However, here we are interpreting it in a purely syntactic context; when
writing~``$U_\TT$'', we actually mean the tautologous ``model'' in which any
sort of~$\TT$ is interpreted by the sort itself. For instance, if~$\sigma =
(\varphi \seq{x_1\!\?\!X_1,\ldots,x_n\!\?\!X_n} \psi)$ is a sequent,
then~$\speak{$\sigma$ holds for~$U_\TT$}$ is to be interpreted as the
formula~``$\forall x_1\?X_1\_ \ldots \forall x_n\?X_n\_ (\varphi \Rightarrow
\psi)$''.

\begin{proof}[Proof of Theorem~\ref{thm:characterization}]
The direction~$(2) \Rightarrow (1)$ is immediate, as the internal
language of~$\Set[\TT]$ validates first-order intuitionistic logic and as it
validates the
axiom scheme~\eqref{eq:nullstellensatz} by Theorem~\ref{thm:nullstellensatz}.

For the converse direction, we note that, given a geometric sequent~$\sigma$
over the signature of~$\TT$, the statement~$\speak{$\ul{\TT}/U_\TT$
proves$^\star$~$\sigma$}$ can be expressed as a geometric formula over the
signature of~$\TT$. Applying this observation successively to subformulas
of the given formula~$\alpha$, Theorem~\ref{thm:nullstellensatz} on the
semantic side and the axiom scheme~\eqref{eq:nullstellensatz} on the syntactic side imply that we may
assume that~$\alpha$ is in fact a geometric formula. Hence we are reduced to
the basic fact that, for geometric formulas~$\varphi$, $\Set[\TT] \models
\varphi$ implies that~$\TT$ proves~$\varphi$.
\end{proof}

% XXX coherent case


\section{The special case of Horn theories}
\label{sect:horn}

Throughout this section, let~$\TT$ be a Horn theory.

\begin{lemma}\label{lemma:free-models}
Let~$X$ be a set equipped with a morphism~$X \to S$ to the set of sorts
of the signature~$\Sigma$ of~$\TT$. Let~$R$ be a set of atomic propositions in which the
elements of~$X$ may appear as new constants of the respective sorts. Then there
is~$\TT\langle X | R \rangle$, the free~$\TT$-model on the generators~$X$ modulo
the relations~$R$.\end{lemma}

\begin{proof}The desired model can be constructed as a term algebra. As a set,
it consists of the terms (in the empty context) of the signature~$\Sigma + X$
modulo the equivalence relation identifying two terms if and only if~$\TT + R$
proves them to be equal. The function symbols~$f$ of~$\Sigma$ are interpreted
by declaring~$\brak{f}([t_1],\ldots,[t_n]) = [f(t_1,\ldots,t_n)]$ and the
relation symbols~$S$ are interpreted by declaring~$([t_1],\ldots,[t_n]) \in
\brak{S} \Leftrightarrow (\TT + R \vdash S(t_1,\ldots,t_n))$.

We omit the required verifications and only remark that while the same
construction could be carried out if~$\TT$ was a general geometric theory, the
resulting object would in general not be a model of~$\TT$.
\end{proof}

\begin{lemma}The category of~$\TT$-models is complete and
cocomplete.\end{lemma}

\begin{proof}Limits are computed as the limits of the underlying sets, colimits
are computed by using the construction of Lemma~\ref{lemma:free-models}. For
instance, the coproduct of~$\TT\langle X | R \rangle$ and~$\TT\langle X' | R'
\rangle$ is~$\TT\langle X \amalg X' \,|\, R, R' \rangle$.\end{proof}

Having the special case of the theory of rings in mind, we write the coproduct
in the category of~$\TT$-models as~``$\otimes$''.

\begin{lemma}Let~$\sigma = (\varphi_1 \wedge \cdots \wedge \varphi_n
\seq{x_1,\ldots,x_k} \psi_1 \wedge \cdots \wedge \psi_m)$ be a Horn sequent
over the signature of~$\TT$. Then the following statements are equivalent.
\begin{enumerate}
\item The theory~$\TT$ proves~$\sigma$. \smallskip
\item In~$\TT\langle x_1,\ldots,x_k \,|\, \varphi_1,\ldots,\varphi_n \rangle$, the
propositions~$\psi_1,\ldots,\psi_m$ hold for
the~$k$-tuple~$([x_1],\ldots,[x_k])$.
\end{enumerate}
\end{lemma}

\begin{proof}By construction of the term algebra.\end{proof}

\begin{lemma}\label{lemma:char-fp-models}
A~$\TT$-model is finitely presentable as an object of the category
of~$\TT$-models if and only if it is isomorphic to a model of the
form~$\TT\langle X | R \rangle$ where~$X$ is Bishop-finite and~$R$ is
Kuratowski-finite.
\end{lemma}

\begin{proof}It is an instructive exercise to verify that models of the stated
form are compact. Conversely, let a~$\TT$-model~$M$ be given. Then~$\TT$ is the
filtered colimit of all models over~$M$ which are of the stated form. If~$M$ is
compact, the identity on~$M$ factors over such a model. Hence~$M$ is a
retract of such a model and hence itself isomorphic to a model of this form.
\end{proof}

% XXX Set[T] einführen

Any~$\TT$-model~$A$ has a mirror image in the topos~$\Set[\TT]$, namely the
functor~$A^\sim : \Mod{\TT}_\mathrm{fp} \to \Set$ given by~$T \mapsto A \otimes T$.
This object is in a canonical way a~$\TT$-model over~$U_\TT$, hence from the
point of view of~$\Set[\TT]$ a~$\ul{\TT}/U_\TT$-model.

\begin{lemma}The functor~$(\cdot)^\sim$ from~$\TT$-models to~$\ul{\TT}/U_\TT$-models
in~$\Set[\TT]$ is left adjoint to the functor~$\Gamma = \Hom(1, \cdot)$ computing
global elements.
\end{lemma}

\begin{proof}A~$U_\TT$-algebra homomorphism~$\alpha : A^\sim \to M$ yields
the~$\TT$-model homo\-mor\-phism~$\alpha_0 : A \to M(0) = \Gamma(M)$, where~$0$ is the
initial~$\TT$-model. Conversely, a~$\TT$-model homomorphism~$\beta : A \to
\Gamma(M)$ yields a~$U_\TT$-algebra homomorphism by summing~$A \to M(0) \to
M(T)$ with the structure morphism~$T = U_\TT(T) \to
M(T)$.\end{proof}

\begin{defn}The \emph{spectrum}~$\Spec(M)$ of a~$U_\TT$-algebra~$M$ in~$\Set[\TT]$
is the result of constructing, internally to~$\Set[\TT]$, the set
of~$U_\TT$-algebra homomorphisms~$M \to U_\TT$.
\end{defn}

% This is wrong:
% Externally, the spectrum of~$M$ is the functor
% mapping a finitely presented~$\TT$-model~$T$ to~$\Hom_T(M(T), T)$, the set
% of~$\TT$-homomorphisms~$M(T) \to T$ compatible with the structure morphisms~$T
% \to M(T)$ and~$T \to T$.

\begin{lemma}\label{lemma:spec-sim-representable}
Let~$A$ be a~$\TT$-model. Then~$\Spec(A^\sim)$ coincides
with~$\yon A$, the functor~$\Hom_{\Mod{\TT}}(A, \cdot)$.
\end{lemma}

\begin{proof}By the Yoneda lemma, the sections of the sheaf~$\Spec(A^\sim) :
\Mod{\TT}_\mathrm{fp} \to \Set$ on an object~$T$ are given by the set
\begin{align*}
  \Spec(A^\sim)(T) &\cong \Hom(\yon T, \Spec(A^\sim)) =
  \Hom(\yon T, [A^\sim,U_\TT]_{U_\TT}) \\
  &\cong \Hom(\yon T \times A^\sim, U_\TT)_{\text{$U_\TT$-algebra homomorphism in second
  argument}} \\
  &\cong \Hom_{U_\TT}(A^\sim, (U_\TT)^{\yon T})
  \cong \Hom_{U_\TT}(A^\sim, U_\TT|T),
\end{align*}
where~$[A^\sim,U_\TT]_{U_\TT}$ is the object of~$U_\TT$-algebra homomorphisms from~$A^\sim$
to~$U_\TT$ (a subobject of the internal Hom~$(U_\TT)^{A^\sim}$); $\Hom_{U_\TT}$
denotes the set of~$U_\TT$-algebra homomorphisms; $(U_\TT)^{\yon T}$ is the object of morphisms
from~$\yon T$ to~$U_\TT$; and~$U_\TT|T$ is the functor~$U_\TT(T \times \cdot)$, that is~$S
\mapsto T \otimes S$.

An arbitrary element~$f \in (\yon A)(T)$, that is an arbitrary~$\TT$-model
homomorphism~$f : A \to T$, induces a~$U_\TT$-algebra homomorphism~$g : A^\sim \to
U_\TT|T$ by setting~$g_S \defeq f \otimes \id_S : A \otimes S \to T \otimes S$. The
given homomorphism~$f$ can be recovered by~$f = g_0$, the component of~$g$ at
the initial model.

Conversely, a~$U_\TT$-algebra homomorphism~$g : A^\sim \to U_\TT|T$ induces
a~$\TT$-model homomorphism~$f : A \to T$ by setting~$f \defeq g_0$. Because~$g$
is a natural transformation and because~$g$ is compatible with the structure
morphisms~$U_\TT \to A^\sim$ and~$U_\TT \to U_\TT|T$, the morphism~$g$ is determined
by~$f$.
\end{proof}

\begin{lemma}\label{lemma:fp-double-dual}
Let~$A$ be a finitely presentable~$\TT$-model. Then the canonical morphism
\[ A^\sim \longrightarrow (U_\TT)^{\Spec(A^\sim)} \]
is an isomorphism of~$U_\TT$-algebras.
\end{lemma}

\begin{proof}By Lemma~\ref{lemma:spec-sim-representable}, the
functor~$\Spec(A^\sim)$ coincides with~$\yon A$. Since by assumption~$A$ is
contained in the site defining~$\Set[\TT]$, the exponential~$(U_\TT)^{\yon A}$
coincides with~$U_\TT|A$ (notation as in the proof in
Lemma~\ref{lemma:spec-sim-representable}), that is, the~$U_\TT$-algebra~$A^\sim$.
\end{proof}

\begin{cor}Let~$A$ and~$B$ be~$\TT$-models. Assume that~$B$ is finitely
presentable. Then the canonical morphism
\[ \Hom_{U_\TT}(A^\sim, B^\sim) \longrightarrow \Spec(A^\sim)^{\Spec(B^\sim)}
\]
is an isomorphism.
\end{cor}

\begin{proof}We have the chain of isomorphisms
\begin{align*}
  \Spec(A^\sim)^{\Spec(B^\sim)} &=
  ([A^\sim,U_\TT]_{U_\TT})^{\Spec(B^\sim)} \cong
  [\Spec(B^\sim) \times A^\sim, U_\TT]_{U_\TT} \\
  &\cong
  [A^\sim, U_\TT^{\Spec(B^\sim)}]_{U_\TT} \cong
  [A^\sim, B^\sim],
\end{align*}
where the final isomorphism is by Lemma~\ref{lemma:fp-double-dual}.
\end{proof}

% XXX terminology "U_T-algebra". location of the tensor explanation

\begin{thm}The generic~$\TT$-model is \emph{quasicoherent} in the following sense:
From the point of view of~$\Set[\TT]$, for any finitely
presentable~$U_\TT$-algebra~$A$ (finitely presented object in the category
of~$U_\TT$-algebras), the canonical~$U_\TT$-algebra homomorphism
\[ A \longrightarrow (U_\TT)^{\Spec(A)} \]
is an isomorphism.
\end{thm}

\begin{proof}The proof of Lemma~\ref{lemma:char-fp-models} is constructive and
thus valid in the internal language of~$\Set[\TT]$. Hence we can apply it,
internally, to the theory~$\ul{\TT}/U_\TT$ to deduce that a~$U_\TT$-algebra~$A$
is finitely presentable if and only if it is isomorphic to a~$U_\TT$-algebra of
the form~$(\ul{\TT}/U_\TT)\langle X | R \rangle$ with~$X$ Bishop-finite and~$R$
Kuratowski-finite.

We therefore have to verify the following internal
statement: \emph{For any number~$n$, for any sorts~$X_1,\ldots,X_n$
of~$\ul{\TT}/U_\TT$, for any number~$m$, for any atomic
propositions~$R_1,\ldots,R_m$ over the signature of~$\ul{\TT}/U_\TT$ extended
by constants~$e_1 \? X_1, \ldots, e_n \? X_n$, the canonical map~$A \to
(U_\TT)^{\Spec(A)}$ where~$A \defeq (\ul{\TT}/U_\TT)\langle
e_1\?X_1,\ldots,e_n\?X_n \,|\, R_1,\ldots,R_m \rangle$ is an isomorphism.}

Following the Kripke--Joyal translation of this statement, let a stage~$T \in
\Mod{\TT}_\mathrm{fp}$,~$T$-elements~$X_1,\ldots,X_n$ of the object of sorts of
the signature of~$\ul{\TT}/U_\TT$ (that is the constant sheaf on the set of
sorts of~$\TT$), and~$T$-elements~$R_1,\ldots,R_m$ of the object of atomic
propositions over the signature of~$\ul{\TT}/U_\TT$ be given. By passing to a
covering of~$T$, we may assume that the~$X_i$ are given by sorts of~$\TT$ and
that the~$R_j$ are given by atomic propositions over the signature
of~$\TT/U_\TT(T)$.

Since the slice~$\Set[\TT]/\yon T$ is equivalent to~$\Set[\TT/T]$,
hence again the classifying topos of a Horn theory, we may without loss of
generality assume that~$T$ is the initial~$\TT$-model.

In this case the claim follows from Lemma~\ref{lemma:fp-double-dual}, since
the result of constructing, internally to~$\Set[\TT]$, the
model~$(\ul{\TT}/U_\TT)\langle
e_1\?X_1,\ldots,e_n\?X_n \,|\, R_1,\ldots,R_m \rangle$ coincides with
the~$U_\TT$-algebra~$(\TT\langle
e_1\?X_1,\ldots,e_n\?X_n \,|\, R_1,\ldots,R_m \rangle)^\sim$.
\end{proof}


\section{The generalization to the higher-order case}
\label{sect:higher-order}

% XXX introduction

By \emph{extended geometric logic} we mean the extension of geometric logic
where we are allowed to form, in addition to the basic sorts supplied by
a given signature, finite limits of sorts and set-indexed colimits of sorts. By
(intuitionistic) \emph{higher-order logic}, we mean the further extension where
we may also form powersorts. These derived sorts come with respective term
constructors (tuple formers, coprojections, set comprehension) and the usual
rules governing these constructors.

An \emph{extended geometric formula} is a formula of extended geometric logic built from equality and
relation symbols by the logical
connectives~${\top}\,{\bot}\,{\wedge}\,{\vee}\,{\exists}$
and by arbitrary set-indexed disjunctions~$\bigvee$. The existential
quantification can be over any sorts of extended geometric logic, including the derived sorts. An
\emph{extended geometric sequent} is a sequent of the form~$(\varphi
\seq{\vec x} \psi)$ where~$\varphi$ and~$\psi$ are extended geometric
formulas and the sorts of the variables~$\vec x$ may be derived sorts.

It is possible to extend the Kripke--Joyal semantics so that higher-order logic
can be interpreted in any Grothendieck topos. The truth of a higher-order
sequent~$(\varphi \seq{\vec x} \psi)$ is in general not preserved under
pullback along geometric morphisms, even if~$\varphi$ and~$\psi$ do not
contain~${\forall}$ and~$\Rightarrow$, since powerobjects are in general not
preserved under pullback. However, as can be deduced from the following lemma,
the truth of extended geometric sequents is preserved; as is folklore, extended
geometric logic is just a thin layer over ordinary geometric logic.

\begin{lemma}\label{lemma:extended-conservative}
Let~$\sigma$ be an extended geometric sequent over the signature
of a geometric theory~$\TT$. Then there is a set-indexed family~$(\sigma_i)_i$ of ordinary
geometric sequents over the same signature, so that~$\sigma$ is provable in
extended geometric logic if and only if all the sequents~$\sigma_i$ are
provable in ordinary geometric logic.\end{lemma}

\begin{proof}Any existential quantification of the form~``$\exists p \? X
\times Y$'' can be replaced by the string~``$\exists x \? X\_ \exists y \?
Y$'', and similarly for free variables of product sorts appearing in the
context of~$\sigma$. In a similar vein more general finite limits are treated.

An existential quantification of the form~``$\exists x \? \coprod_i X_i$'' can
be replaced by the string~``$\bigvee_i \exists x \? X_i$''.

Finally, for any occurrence of a free variable~$x \?
\coprod_i X_i$ in the context of~$\sigma$, we can replace~$\sigma$ by the
family of sequents~$(\sigma_i)_i$, where the sequent~$\sigma_i$ is the same
as~$\sigma$ only that the free variable~$x$ is changed to be of sort~$X_i$ (and
the corresponding change in the consequent and the antecedent is applied,
applying the appropriate coprojection).

After carrying out these steps, the free variables are only of the basic sorts
supplied by the signature of~$\TT$ and existential quantifications only
range over the basic sorts. However, in the consequents and antecedents, still
tuple formers and coprojections may appear. These can be replaced as suggested
by the rules governing these. For instance, an equation~``$\langle x,y \rangle
= \langle x',y' \rangle$'' can be replaced by the conjunction~``$x = x' \wedge
y = y'$'', and an equation~``$\iota_i(x) = \iota_j(y)$'' (where~$\iota_i$
and~$\iota_j$ are coprojections associated with coproduct sorts) can be
replaced by the subsingleton-indexed disjunction~``$\bigvee\{ x = y \,|\, i = j \}$''.
\end{proof}

\begin{thm}\label{thm:definability}
Let~$\TT$ be a geometric theory. Let~$x_1\?X_1,\ldots,x_n\?X_n$ be a context over the
signature of~$\TT$. Then the canonical morphism
\[ \mathrm{Form}^\star_{\vec x}(\ul{\TT}/U_\TT)/(\dashv\vdash_{\vec x}) \longrightarrow P(X_1 \times
\cdots \times X_n) \]
sending, internally speaking, the equivalence class of a geometric$^\star$ formula~$\varphi$ over the
signature of~$\ul{\TT}/U_\TT$ in the context~$\vec x$ to the subset~$\{ (x_1,\ldots,x_n) \,|\, \varphi
\}$ is an isomorphism.
% XXX explain equivalence relation
\end{thm}

\begin{proof}Injectivity is by the Nullstellensatz of
Theorem~\ref{thm:nullstellensatz}. Surjectivity is by the definability
result~\cite[Theorem~2.2]{caramello:definability}, exploiting that the internal
statement localizes well by Lemma~\ref{lemma:truth-to-provability}.
% XXX check reference number
\end{proof}

% An immediate corollary of Theorem~\ref{thm:definability} is the
% Nullstellensatz of Theorem~\ref{thm:nullstellensatz}. Indeed, arguing
% internally to~$\Set[\TT]$, let~$\sigma = (\varphi \seq{\vec x} \psi)$ be a
% geometric$^\star$ sequent over the signature of~$\ul{\TT}/U_\TT$. Assume
% that~$\sigma$ holds for~$U_\TT$. Then the subsets~$\{ (\vec x) \,|\, \varphi
% \}$ and~$\{ (\vec x) \,|\, \varphi \wedge \psi \}$ are equal. Hence, by
% Theorem~\ref{thm:definability}, the formulas~$\varphi$
% and~$\varphi \wedge \psi$ are provably$^\star$ equivalent.
% Thus~$\ul{\TT}/U_\TT$ proves$^\star$~$(\varphi \seq{\vec x} \psi)$.

\begin{cor}Let~$\TT$ be a geometric theory. Let~$\{\vec x\_ \varphi\}$
and~$\{\vec y\_ \psi\}$ be geometric formulas in given contexts. Then,
internally to~$\Set[\TT]$, the canonical map from the set of equivalence classes
of~$\ul{\TT}/U$-provably$^\star$ functional geometric$^\star$ formulas
from~$\{\vec x\_ \varphi\}$ to~$\{\vec y\_ \psi\}$ to the set of maps~$\{(\vec y)
\,|\, \psi \}^{\{(\vec x) \,|\, \varphi\}}$ is a bijection.
\end{cor}

\begin{proof}We argue internally to~$\Set[\TT]$.
The canonical map sends an equivalence class~$[\theta]$ to the unique map~$f :
\{(\vec x) \,|\, \varphi\} \to \{(\vec y) \,|\, \psi \}$ whose graph is given
by the set~$\{ (\vec x, \vec y) \,|\, \theta \}$.

For verifying surjectivity, let a map~$f : \{(\vec x) \,|\, \varphi\} \to
\{(\vec y) \,|\, \psi \}$ be given. Then its graph is a subset of~$\vec X
\times \vec Y$, hence by Theorem~\ref{thm:definability} given by a
geometric$^\star$ formula~$\theta$. Because~$f$ is a map, this formula is
functional; and by the Nullstellensatz, it is~$\ul{\TT}/U_\TT$-provably$^\star$ so.

For verifying injectivity, let~$\theta$ and~$\theta'$
be~$\ul{\TT}/U_\TT$-provably$^\star$ functional formulas which give rise to
identical maps. Then they also give rise to identical graphs, hence
are~$\ul{\TT}/U_\TT$-provably$^\star$ equivalent by
Theorem~\ref{thm:definability}.
\end{proof}

\begin{thm}\label{thm:higher-order-nullstellensatz}
Let~$\TT$ be a geometric theory. Then, internally to~$\Set[\TT]$, for any
extended geometric$^\star$ sequent~$\sigma$ over the signature
of~$\ul{\TT}/U$, the following statements are equivalent:
\begin{enumerate}
\item The sequent~$\sigma$ holds for~$U_\TT$. \smallskip
\item The sequent~$\sigma$ is provable$^\star$ modulo~$\TT/U_\TT$ in extended
geometric logic.
\end{enumerate}
\end{thm}

\begin{proof}The implication~$(2) \Rightarrow (1)$ is
immediate since~$U_\TT$ is a model of~$\TT/U_\TT$. The converse direction is by
Lemma~\ref{lemma:extended-conservative}, which holds internally in~$\Set[\TT]$
as the proof we supplied is constructive, and the Nullstellensatz for ordinary
geometric logic of Theorem~\ref{thm:nullstellensatz}.
\end{proof}

\begin{thm}\label{thm:higher-order-characterization}
Let~$\TT$ be a geometric theory. Let~$\alpha$ be a higher-order formula over
the signature of~$\TT$. Then the following statements are equivalent:
\begin{enumerate}
\item The formula~$\alpha$ holds for~$U_\TT$. \smallskip
\item The formula~$\alpha$ is provable in higher-order intuitionistic logic
modulo the axioms of~$\TT$ and the additional axiom scheme
\begin{equation}\label{eq:higher-order-nullstellensatz}\tag{$\P$}
  \speak{the map~$\mathrm{Form}^\star_{\vec
  x}(\ul{\TT}/U_\TT)/(\dashv\vdash_{\vec x}) \longrightarrow P(X_1 \times
  \cdots \times X_n)$ is bijective},
\end{equation}
where~$X_1,\ldots,X_n$ is any list of sorts.
\end{enumerate}
\end{thm}

\begin{proof}Theorem~\ref{thm:definability} on the semantic side and the axiom
scheme~\eqref{eq:higher-order-nullstellensatz} on the syntactic side allow us to replace any mention of a
powersort~$P(X)$ in~$\alpha$
by~$\mathrm{Form}^\star_{x\!\?\!X}(\ul{\TT}/U_\TT)/({\dashv\vdash_x})$,
similarly to how the proof of Lemma~\ref{lemma:extended-conservative} compiles
extended geometric logic to ordinary geometric. Then we
can argue as in the proof of Theorem~\ref{thm:characterization}, noting that
the axiom scheme indeed entails the axiom scheme~\eqref{eq:nullstellensatz} posited by
Theorem~\ref{thm:characterization}.\end{proof}

% XXX check

%In einem ersten Schritt schreiben wir alle Existenzquantifikationen und freie Variablen
%über Potenzsorten um. Aus "exists A : P(X)" wird "exists φ :
%Form*(T/U_T)_{x:X}/-||-". (Dabei kann X immer noch eine abgeleitete Sorte sein!) Aus
%der freien Variable "A : P(X)" wird analog eine freie Variable.
%
%Entsprechend verändern wir den Rumpf. Aus "x0 ∈ A" wird φ[x0/x].
%
%Dann sind wir fertig mit dem Nullstellensatz für extended geometric logic.
%
%Noch klären:
%* P(X) x Y
%* Form* Teil von extended geometric logic?
%* Beweisbarkeit modulo extended geometric logic ist geometrische Formel?


\section{Applications}
\label{sect:applications}

% XXX

% Zariski

% Kock--Lawvere in SDG

%We do have, for any geometric sequent~$\sigma$, that
%\[ \Set[\TT] \models \speak{$\sigma$ holds for~$U_\TT$} \quad\text{iff}\quad
%  \Set[\TT] \models \speak{$\ul{\TT}$ proves~$\sigma$}, \]
%since~$\Set[\TT] \models \speak{$\sigma$ holds for~$U_\TT$}$ implies that~$\TT$
%proves~$\sigma$ and since~$\TT \vdash \sigma$ implies~$\E \models
%(\ul{\TT} \vdash \sigma)$ for any topos~$\E$ over~$\Set$. However, this
%equivalence is not stable under slicing ...



\printbibliography

\end{document}

* Discuss case of SDG: Why does the Kock--Lawvere there hold only for Weil
  algebras? Answer: It doesn't really! We do have that [U,U] is isomorphic to
  the free C^∞-ring on one generator. But this observation doesn't appear to be
  useful for further developments.

* XXX Cite https://docplayer.net/55796150-Intuitionistic-algebra-some-recent-developments-in-topos-theory.html
  (in general and for internal theories)

* XXX Discuss related work, especially: Olivia 6.1.3, ...


TODO:

* XXX Nichttrivialität des Nullstellensatzes diskutieren (Spezialfall top |- φ
  betrachten, dieser ist nämlich schon mehr oder weniger trivial)
