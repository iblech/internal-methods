\documentclass[oneside,reqno]{amsart}

\usepackage[utf8]{inputenc}
\usepackage{amsthm,mathtools,stmaryrd,amssymb,graphicx}
\usepackage{booktabs}
\usepackage[all]{xy}
\usepackage[protrusion=true,expansion=true]{microtype}
\usepackage{xspace}

\usepackage[natbib=true,style=numeric,maxnames=10]{biblatex}
\usepackage[babel]{csquotes}
\bibliography{paper-filmat.bib}

\graphicspath{{images/}}

\usepackage{pifont}
\newcommand{\cmark}{\ding{51}}
\newcommand{\xmark}{\ding{55}}

\title[]{Reflections on reflection for \\ intuitionistic set theories}
\author{}
\email{iblech@speicherleck.de}

\theoremstyle{definition}
\newtheorem{defn}{Definition}[section]
\newtheorem{ex}[defn]{Example}

\theoremstyle{plain}
\newtheorem{prop}[defn]{Proposition}
\newtheorem{cor}[defn]{Corollary}
\newtheorem{lemma}[defn]{Lemma}
\newtheorem{thm}[defn]{Theorem}
\newtheorem{scholium}[defn]{Scholium}

\theoremstyle{remark}
\newtheorem{rem}[defn]{Remark}
\newtheorem{question}[defn]{Question}
\newtheorem{speculation}[defn]{Speculation}
\newtheorem{caveat}[defn]{Caveat}
\newtheorem{conjecture}[defn]{Conjecture}

\newcommand{\xra}[1]{\xrightarrow{#1}}
\newcommand{\XXX}[1]{\textbf{XXX: #1}}
\newcommand{\aaa}{\mathfrak{a}}
\newcommand{\bbb}{\mathfrak{b}}
\newcommand{\mmm}{\mathfrak{m}}
\newcommand{\I}{\mathcal{I}}
\newcommand{\J}{\mathcal{J}}
\newcommand{\E}{\mathcal{E}}
\newcommand{\F}{\mathcal{F}}
\newcommand{\B}{\mathcal{B}}
\newcommand{\NN}{\mathbb{N}}
\newcommand{\RR}{\mathbb{R}}
\newcommand{\ZZ}{\mathbb{Z}}
\renewcommand{\SS}{\mathbb{S}}
\renewcommand{\P}{\mathcal{P}}
\renewcommand{\O}{\mathcal{O}}
\newcommand{\defeq}{\vcentcolon=}
\newcommand{\op}{\mathrm{op}}
\DeclareMathOperator{\Spec}{Spec}
\DeclareMathOperator{\Hom}{Hom}
\DeclareMathOperator{\Mod}{Mod}
\DeclareMathOperator{\Sh}{Sh}
\DeclareMathOperator{\PSh}{PSh}
\DeclareMathOperator{\rank}{rank}
\newcommand{\Set}{\mathrm{Set}}
\newcommand{\Eff}{\mathrm{Ef{}f}}
\renewcommand{\_}{\mathpunct{.}\,}
\newcommand{\effective}{ef{}fective\xspace}
\newcommand{\?}{\,{:}\,}
\newcommand{\realizes}{\Vdash}
\newcommand{\notnot}{\emph{not~not}\xspace}

\newcommand{\stacksproject}[1]{\cite[{\href{https://stacks.math.columbia.edu/tag/#1}{Tag~#1}}]{stacks-project}}

\renewcommand{\paragraph}[1]{\noindent\textbf{#1.}}

\newcommand{\ZF}{\textsc{zf}}
\newcommand{\IZF}{\textsc{izf}}
\newcommand{\CZF}{\textsc{czf}}
\newcommand{\ZFC}{\textsc{zfc}}
\newcommand{\ZFCS}{\textsc{zfc/s}}
\newcommand{\RRS}{\textsc{rrs}}
\newcommand{\MDC}{\textsc{mdc}}
\newcommand{\ES}{(\IZF+\RRS)/\textsc{s}}

\begin{document}

\begin{abstract}
  We study the well-known reflection principle of Zermelo--Fraenkel set theory
  in the context of intuitionistic Zermelo--Fraenkel set theory~$\IZF$. We show
  that the reflection principle is equivalent to Aczel's relation reflection scheme~$\RRS$. As
  applications, we give a new proof that relativized dependent choice is
  equivalent to the conjunction of the relation reflection scheme and dependent
  choice and we present an intuitionistic version of Feferman's~$\ZFCS$, a
  conservative extension of~$\ZFC$ which is useful as a foundation for category
  theory.
\end{abstract}

\maketitle
\thispagestyle{empty}

\section{Introduction}

The basic form of the reflection principle for Zermelo--Fraenkel set
theory~$\ZF$ is the following.
\begin{thm}Let~$\varphi(x_1,\ldots,x_n)$ be a formula in the language of set
theory with free variables as indicated. Then~$\ZF$ proves
\[ \forall M\_
  \exists S \supseteq M\_
  \forall x_1,\ldots,x_n \in S\_
  \varphi(x_1,\ldots,x_n) \Leftrightarrow \varphi^S(x_1,\ldots,x_n), \]
where~$\varphi^S$ is the~$S$-relativization of~$\varphi$, obtained by
substituting any occurence of~``$\forall x$'' and~``$\exists x$'' by~``$\forall
x \in S$'' and~``$\exists x \in S$''.
Furthermore, the resulting set~$S$ may be supposed to be transitive, to be closed
under subsets or even to be a stage~$V_\alpha$ of the cumulative hierarchy;
% XXX: "hence" instead of "or even"?
and given not a single formula~$\varphi$ but a finite
list~$\varphi_1,\ldots,\varphi_s$ of formulas, we may suppose that~$S$ reflects
all of them.
\end{thm}

The reflection principle is important both for philosophical and for practical
reasons: Philosophically, it tells us that truth of any formula can already be
checked in an initial segment of the universe. Practically, it allows to
% XXX: be more precise about philosophy
transfer results obtained for a restricted class of objects to all such
objects. For instance, if we manage to verify (the $S$-relativization of) some
group-theoretic statement for all groups contained in an arbitrary set~$X$,
then we may deduce by the reflection principle that the statement holds
for all groups in the universe.

This observation has been used by Feferman to construct~$\ZFCS$ (``$\ZFC$ with
smallness''), a conservative extension of~$\ZFC$ which provides a useful
foundation of category theory~\cite{feferman:zfcs}. This system extends~$\ZFC$
by the addition of a new constant symbol~$\SS$ together with axioms stating
that~$\SS$ is transitive, closed under subsets and reflective with respect to
every formula~$\varphi(x_1,\ldots,x_n)$ of the original language:
\[ \label{ax:refl}\tag{$\star$}
\forall x_1,\ldots,x_n \in \SS\_ \varphi(x_1,\ldots,x_n) \Leftrightarrow
\varphi^{\SS}(x_1,\ldots,x_n). \]
The system~$\ZFCS$ is conservative over~$\ZFC$ because any given proof in~$\ZFCS$ uses
only a finite number of instances of the axioms~\eqref{ax:refl}, whereby the reflection
principle can be used to yield an honest set~$S$ which validates just the same equivalences
and can hence be used in place of~$\SS$.

Elements of~$\SS$ are deemed ``small'', so that~$\SS$ is the set of all small
sets. The system~$\ZFCS$ is useful as a foundation for category theory because
it supports, without requiring new set-theoretical commitments such as the
existence of Grothendieck universes, a native treatment of large structures.
for instance, the category of all small sets can be formed entirely
withing~$\ZFCS$, without resorting to classes. We invite the reader wanting
to learn more about the merits of~$\ZFCS$ to a survey by
Shulman~\cite[Section~11]{shulman:sets}.

To our naive eyes, the passage from~$\ZFC$ to~$\ZFCS$ looked sufficiently
innocent so that we set out to create an intuitionistic version of~$\ZFCS$: To
our mind, size issues were entirely separate from issues of constructivity, and
we hence opined that they should be dealt with separately. Such a separation
would not only lead to improved mental hygiene and better understanding, but
would also allow us to use the benefits
of~$\ZFCS$ in situations where the law of excluded middle and the axiom of
choice are not available, such as in realizability semantics, sheaf semantics
or quite generally topos semantics.

We expected this modification to be entirely straightforward. However, the
situation turned out to be more subtle and we failed in our original goal of
verifying the reflection principle for intuitionistic Zermelo--Fraenkel set
theory~$\IZF$. The situation for~$\CZF$, the predicative subtheory of~$\IZF$
commonly heralded as the largest common denominator of all flavors of
constructive set theory, is even more mysterious.

However, we suceeded in verifying the reflection principle for a slight
extension of~$\IZF$:

\begin{thm}The reflection principle is equivalent, over~$\IZF$, to Aczel's relation reflection
scheme~$\RRS$.\end{thm}

Aczel's relation reflection scheme will be reviewed below. It is validated by
all known models of~$\IZF$, hence might be regarded as not entirely
unconstructive, even though it's conjectured to be independent of~$\IZF$. As a
result, the question whether the reflection principle holds for~$\IZF$ remains
open (though conjectured to be false), and for the stronger system~$\IZF+\RRS$
we can give a variant ``with smallness'' which can serve as a set-theoretic
foundation for category theory.

\begin{conjecture}The system $\IZF$ does not prove the reflection
principle.\end{conjecture}

\begin{defn}The system $\ES$ is obtained from~$\IZF+\RRS$ by
adding a constant symbol~$\SS$ together with axioms
stating that~$\SS$ is transitive, closed under subsets and reflective for all
formulas of the original language (that is, adding one instance
of~\eqref{ax:refl} for each formula of~$\IZF$).\end{defn}

\begin{cor}The system~$\ES$ is conservative
over~$\IZF+\RRS$.\end{cor}

This note is organized as follows. Section~\ref{sect:review} reviews the
classical proof of the reflection principle in the context of~$\ZF$ set theory.
Section~\ref{sect:classical-proof} reviews Aczel's relation reflection scheme
and presents our main result. We conclude in Section~\ref{sect:appl-rdc} and
Section~\ref{sect:appl-smallness} with two short applications.

\textbf{Acknowledgments.} XXX


\section{Review of the classical reflection principle}
\label{sect:review}

A basic proof of the reflection principle for~$\ZF$ runs as follows. Our proof
of the reflection principle for~$\IZF+\RRS$ will follow the same outline.

\begin{lemma}\label{lemma:zf-smallstep}
Let~$\varphi(u_1,\ldots,u_n,x)$ be a formula in the language of set
theory with free variables as indicated. Then~$\ZF$ proves
\begin{multline*}
  {\qquad}\forall M\_
  \exists S \supseteq M\_
  \forall u_1,\ldots,u_n \in S\_ \\
  (\exists x\_ \varphi(u_1,\ldots,u_n,x)) \Longrightarrow
  (\exists x \in S\_ \varphi(u_1,\ldots,u_n,x)).{\qquad}
\end{multline*}
Furthermore: (1)~We may suppose that~$S$ is transitive. (2)~We may suppose that~$S$ is
closed under subsets. (3)~Given a finite list~$\varphi_1,\ldots,\varphi_s$
of formulas instead of the single formula~$\varphi$, we may suppose that~$S$
bounds all of the formulas~$\varphi_i$.
% XXX: better explanation
\end{lemma}

\begin{proof}Given a class~$X$ (as commonly understood as the comprehension of
a formula), we denote by~$X^\sim$ its subclass~$\{ x \in X \,|\, \forall y
\in X\_ \rank(x) \leq \rank(y) \}$, where the rank function refers to the stage
in the cumulative hierarchy. The two fundamental properties of this construction are:
This subclass is equal to a set,\footnote{If~$X$ is empty, then
this claim is trivial; if~$X$ is inhabited by some element~$x_0$, then~$X^\sim$ can
be obtained using separation from~$V_{\rank(x_0)+1}$; and in fact, by an argument
using the set of truth values and unbounded separation, the claim can also be
proven in~$\IZF$.} and it is inhabited if and only if~$X$ is.

Starting with~$S_0 \defeq M$, we construct~$S_{k+1}$ from~$S_k$ as the union
\[ S_{k+1} \defeq S_k \cup \bigcup_{u_1,\ldots,u_n \in S_k} \{ x \,|\,
\varphi(u_1,\ldots,u_n,x) \}^\sim. \]
It is then easy to check that~$S \defeq \bigcup_{k \in \NN} S_k$ is a set with the
required property.

For addendum~(1), we redeclare~$S \defeq \bigcup_{k \in \NN} S_k^t$,
where~$S_k^t$ is the transitive closure of~$S_k$.

To ensure both addendum~(1) and addendum~(2), we redeclare~$S \defeq \bigcup_{k
\in \NN} P^\omega(S_k')$, where~$P^\omega(X) \defeq \bigcup_{\ell \in \NN}
P^{(\ell)}(X)$ is the union of iterated powersets.

For addendum~(3), we change the definition of~$S_{k+1}$ to include one summand
for each formula~$\varphi_i$.
\end{proof}

The proof of Lemma~\ref{lemma:zf-smallstep} is mostly constructive; however,
there is one issue with nontrivial ramifications: While~$\IZF$ does show
that the subclass~$X^\sim$ of a given class~$X$ is a set, it does not verify
that~$X^\sim$ is inhabited if~$X$ is. This would amount to the constructive
taboo that any inhabited set contains a rank-minimal element; and in fact, by
a result of Friedman and Scedrov~\cite{XXX}, no definable substitute
for~$X^\sim$ exists. The remedy presented in
Section~\ref{sect:constructive-proof} will construct~$X^\sim$ in a non-unique
fashion and then deal with the resulting fallout that taking the union requires
additional care.

\begin{thm}\label{thm:zf-bigstep}
Let~$\varphi(x_1,\ldots,x_n)$ be a formula in the language of set
theory with free variables as indicated. Then~$\ZF$ proves
\[ \forall M\_
  \exists S \supseteq M\_
  \forall x_1,\ldots,x_n \in S\_
  \varphi(x_1,\ldots,x_n) \Leftrightarrow \varphi^S(x_1,\ldots,x_n), \]
where~$\varphi^S$ is the~$S$-relativization of~$\varphi$. Furthermore, the
resulting set~$S$ may be supposed to be transitive and to be closed under
subsets; and given not a single formula~$\varphi$ but a finite
list~$\varphi_1,\ldots,\varphi_s$ of formulas, we may suppose that~$S$ reflects
all of them.
\end{thm}

\begin{proof}Let a set~$M$ be given. We obtain~$S$ by applying
Lemma~\ref{lemma:zf-smallstep} to the list of all subformulas of~$\varphi$. That
the resulting set~$S$ has the required property can be checked by an induction on the structure
of~$\varphi$. The cases~$\text{``$=$''}, \text{``$\in$''}, \text{``$\top$''},
\text{``$\bot$''}, \text{``$\wedge$''},
\text{``$\vee$''}, \text{``$\Rightarrow$''}$ follow
trivially from the induction hypothesis. The case~``$\forall$'' does
not need to be treated since we may assume without loss of generality that all
universal quantifiers in~$\varphi$ have been rewritten
as~``$\neg\exists\neg$''.

The remaining case is where~$\varphi$ is of the form~$\varphi \equiv (\exists
x\_ \psi(u_1,\ldots,u_m,x))$. In this case, the claim is that
\[ \forall u_1,\ldots,u_m \in S\_
  (\exists x\_ \psi(u_1,\ldots,u_m,x)) \Longleftrightarrow
  (\exists x \in S\_ \psi^S(u_1,\ldots,u_m,x)). \]
This claim follows by the property of~$S$ guaranteed by
Lemma~\ref{lemma:zf-smallstep} and by the induction hypothesis concerning the
subformula~$\psi^S(u_1,\ldots,u_m,x)$.
\end{proof}

The proof of Theorem~\ref{thm:zf-bigstep} is constructive with the sole
exception of treating the case of universal quantifiers by appealing to the law
of excluded middle. For the constructive proof in
Section~\ref{sect:constructive-proof}, we will instead appeal to a strengthened
version of Lemma~\ref{lemma:zf-smallstep}.


\section{Constructivizing the reflection principle}
\label{sect:constructive-proof}

Our constructive rendition of the reflection principle will require Aczel's
relation reflection scheme~$\RRS$ displayed in Definition~\ref{defn:rrs}. This
result is the best possible, as conversely the reflection principle entails the
relation reflection scheme. The relation reflection scheme first surfaced in
the theory of coinductive definitions of classes and enjoys substantial
stability properties, as it passes from the metatheory to XXX[all kinds of]
models. Background on~$\RRS$ can be found in Aczel's original article
introducing it~\cite{aczel:rrs}.

\begin{defn}[$\RRS$, Aczel's relation reflection scheme]\label{defn:rrs}
Let~$X$ and~$R \subseteq X \times X$ be classes. Let~$A \subseteq X$ be a
subset. If~$\forall x \in X\_ \exists y \in X\_ \langle x,y \rangle \in R$,
then there is a set~$B$ such that~$A \subseteq B \subseteq X$ and such
that~$\forall x \in B\_ \exists y \in B\_ \langle x,y \rangle \in R$.
\end{defn}

For our purposes, Palmgren's multivalued dependent choice is slightly more
convenient to work with than~$\RRS$. It is equivalent, over~$\CZF$ and
a~fortiori over~$\IZF$, to~$\RRS$~\cite{palmgren:mdc}.

\begin{defn}[$\MDC$, Palmgren's multivalued dependent choice]
Let~$X$ and~$R \subseteq X \times X$ be classes. Let~$A \subseteq X$ be a
subset. If~$\forall x \in X\_ \exists y \in X\_ \langle x,y \rangle \in R$,
then there is a function~$f : \NN \to P(X)$ (the class of all subsets of~$X$)
such that~$A \subseteq f(0)$ and such that~$\forall x \in f(n)\_ \exists y \in
f(n+1)\_ \langle x,y \rangle \in R$ for every~$n$.\end{defn}

Even though superficially similar, the following lemma is \emph{not} yet a
constructivization of Lemma~\ref{lemma:zf-smallstep}; these two lemmas differ
in the set from which~$u_1,\ldots,u_n$ are drawn.

\begin{lemma}\label{lemma:izf-microstep}
Let~$\varphi(u_1,\ldots,u_n,x)$ be a formula in the language of set
theory with free variables as indicated. Then~$\IZF$ proves
\begin{multline*}
  {\qquad}\forall H\_
  \exists H' \supseteq H\_
  \forall u_1,\ldots,u_n \in H\_ \\
  \begin{array}{@{}rcl@{}}
  (\exists x\_ \varphi(u_1,\ldots,u_n,x)) &\Longrightarrow&
  (\exists x \in H'\_ \varphi(u_1,\ldots,u_n,x)) \quad\mathop{\wedge} \\
  (\forall x\_ \varphi(u_1,\ldots,u_n,x)) &\Longleftarrow&
  (\forall x \in H'\_ \varphi(u_1,\ldots,u_n,x)).\end{array}
  {\qquad}
\end{multline*}
Furthermore: (1)~We may suppose that~$H'$ is transitive. (2)~We may suppose
that~$H'$ is closed under subsets. (3)~Given a finite
list~$\varphi_1,\ldots,\varphi_s$ of formulas instead of the single
formula~$\varphi$, we may suppose that~$H'$ has the displayed property for each
of the formulas~$\varphi_i$.
\end{lemma}

\begin{proof}Let~$\Omega \defeq P(\{0\})$ be the set of truth values.
For given elements~$u_1,\ldots,u_n \in H$, we have
\[ \begin{array}{@{}l@{}l@{}}
  \forall a \in \{ a \in \{0\} \,|\, \exists x\_ \varphi(u_1,\ldots,u_n,x) \}\_ &
  \exists x\_ \varphi(u_1,\ldots,u_n,x) \quad\text{and} \\
  \forall p \in \{ p \in \Omega \,|\, \exists x\_ (0 \in p \Leftrightarrow
  \varphi(u_1,\ldots,u_n,x)) \}\_ &
  \exists x\_ (0 \in p \Leftrightarrow \varphi(u_1,\ldots,u_n,x)).
\end{array} \]
Hence, by collection, there are sets~$C$ and~$D$ such that
\[ \begin{array}{@{}rcl@{}}
  (\exists x\_ \varphi(u_1,\ldots,u_n,x)) &\Longrightarrow&
  (\exists x \in C\_ \varphi(u_1,\ldots,u_n,x)) \quad\text{and} \\
  (\forall x\_ \varphi(u_1,\ldots,u_n,x)) &\Longleftarrow&
  (\forall x \in D\_ \varphi(u_1,\ldots,u_n,x)).
\end{array} \]
The union~$C \cup D$ satisfies both of these conditions at once.

Applying collection again, there is a set~$X$ such that for
any~$u_1,\ldots,u_n \in H$ there exists a set~$E \in X$ such that
\[ \begin{array}{@{}rcl@{}}
  (\exists x\_ \varphi(u_1,\ldots,u_n,x)) &\Longrightarrow&
  (\exists x \in E\_ \varphi(u_1,\ldots,u_n,x)) \quad\text{and} \\
  (\forall x\_ \varphi(u_1,\ldots,u_n,x)) &\Longleftarrow&
  (\forall x \in E\_ \varphi(u_1,\ldots,u_n,x)).
\end{array} \]
Hence the set~$H' \defeq M \cup \bigcup X$ has the required property.

To ensure that~$H'$ is transitive and closed under subsets, we simply pass
first to its transitive closure and then compute the union of all its
finitely-iterated powersets.

In order to accomodate more than a single formula~$\varphi$, we add one summand
in the definition of~$H'$ for each formula~$\varphi_i$.
\end{proof}

\begin{lemma}\label{lemma:izf-smallstep}
Let~$\varphi(u_1,\ldots,u_n,x)$ be a formula in the language of set
theory with free variables as indicated. Then~$\IZF+\RRS$ proves
\begin{multline*}
  {\qquad}\forall M\_
  \exists S \supseteq M\_
  \forall u_1,\ldots,u_n \in S\_ \\
  \begin{array}{@{}rcl@{}}
  (\exists x\_ \varphi(u_1,\ldots,u_n,x)) &\Longrightarrow&
  (\exists x \in S\_ \varphi(u_1,\ldots,u_n,x)) \quad\mathop{\wedge} \\
  (\forall x\_ \varphi(u_1,\ldots,u_n,x)) &\Longleftarrow&
  (\forall x \in S\_ \varphi(u_1,\ldots,u_n,x)).\end{array}
  {\qquad}
\end{multline*}
Furthermore: (1)~We may suppose that~$S$ is transitive. (2)~We may suppose
that~$S$ is closed under subsets. (3)~Given a finite
list~$\varphi_1,\ldots,\varphi_s$ of formulas instead of the single
formula~$\varphi$, we may suppose that~$S$ bounds all of the
formulas~$\varphi_i$.
% XXX: better explanation
\end{lemma}

\begin{proof}By~$\MDC$ and Lemma~\ref{lemma:izf-microstep}, there is a
function~$f : \NN \to P(V)$, where~$V$ is the universe, such that~$M \in
f(0)$ and such that for any number~$k$ and any set~$H \in f(k)$, there is a
set~$H' \in f(k+1)$ such that~$H$ and~$H'$ are related as in the conclusion of
Lemma~\ref{lemma:izf-microstep}.

We set~$S \defeq \bigcup \bigl(\bigcup_{k \in \NN} f(k)\bigr)$. This set~$S$
has the required property; to verify this, it is useful to observe that
given~$u_1,\ldots,u_n \in S$, there exists a common number~$k$ such
that~$u_1,\ldots,u_n \in H$ for some~$H \in f(k)$.

In order to ensure addendum~(1) and~(2), we apply~$\MDC$ in a slightly
different way such that for any number~$k$ and any set~$H \in f(k)$, there is a
set~$H' \in f(k+1)$ such that~$H$ and~$H'$ are related as in the conclusion of
Lemma~\ref{lemma:izf-microstep} and such that~$H'$ is transitive and closed
under subsets. Even though it cannot be expected that for any number~$k$,
any set~$H \in f(k)$ will be transitive and closed under subsets, the union~$S$
will.
\end{proof}

\begin{thm}\label{thm:izf-bigstep}
Let~$\varphi(x_1,\ldots,x_n)$ be a formula in the language of set
theory with free variables as indicated. Then~$\IZF+\RRS$ proves
\[ \forall M\_
  \exists S \supseteq M\_
  \forall x_1,\ldots,x_n \in S\_
  \varphi(x_1,\ldots,x_n) \Leftrightarrow \varphi^S(x_1,\ldots,x_n), \]
where~$\varphi^S$ is the~$S$-relativization of~$\varphi$.
Furthermore, the resulting set~$S$ may be supposed to be transitive and to be closed
under subsets; and given not a single formula~$\varphi$ but a finite
list~$\varphi_1,\ldots,\varphi_s$ of formulas, we may suppose that~$S$ reflects
all of them.
\end{thm}

\begin{proof}The proof of Theorem~\ref{thm:zf-bigstep} carries over. The only
difference is that instead of Lemma~\ref{lemma:zf-smallstep},
Lemma~\ref{lemma:izf-smallstep} has to be used, and that the case for the
universal quantifier has to be treated just as the case for the existential
quantifier is.\end{proof}

\begin{thm}\label{thm:refl-entails-rrs}
Over~$\IZF$, each instance of Aczel's relation reflection scheme~$\RRS$ can be
deduced from suitable instances of the assumption that, given a finite list
of formulas, for every set~$M$ there is a set~$S \supseteq M$ reflecting the
given formulas.
\end{thm}

\begin{proof}We verify the equivalent~$\MDC$ instead of~$\RRS$. Let~$X$ and~$R
\subseteq X \times X$ be classes. Let~$A \subseteq X$ be a subset and
suppose~$\forall x \in X\_ \exists y \in X\_ \langle x,y \rangle \in R$.

By assumption, there is a set~$S \supseteq A$ which reflects the three
formulas~``$x \in X$'', ``$\langle x,y \rangle \in R$'' and~``$\exists y \in X\_ \langle
x,y \rangle \in R$''.

By separation, the class~$S \cap X$ is a set, and hence we may set~$f(k) \defeq S \cap X$
for all numbers~$k$. Then~$A \subseteq f(0)$, any given~$x \in f(k)$, by
assumption and the reflecting property of~$S$ there exists~$y \in S$ such
that~$y \in X$ (hence~$y \in f(k+1)$) and~$\langle x,y \rangle \in R$.
\end{proof}

\end{document}

% XXX: Topos which validates exactly those statements which are true
% externally.

% XXX: Review axioms of IZF.

% XXX: Think about pairs.
