\documentclass[oneside,reqno]{amsart}

\usepackage[utf8]{inputenc}
\usepackage{amsthm,mathtools,stmaryrd,amssymb,graphicx}
\usepackage{booktabs}
\usepackage[all]{xy}
\usepackage[protrusion=true,expansion=true]{microtype}
\usepackage{xspace}

\usepackage[natbib=true,style=numeric,maxnames=10]{biblatex}
\usepackage[babel]{csquotes}
\bibliography{paper-reflection.bib}

\graphicspath{{images/}}

\usepackage{pifont}
\newcommand{\cmark}{\ding{51}}
\newcommand{\xmark}{\ding{55}}

\title[]{Reflections on reflection for \\ intuitionistic set theories}
\author{}
\email{}

\theoremstyle{definition}
\newtheorem{defn}{Definition}[section]
\newtheorem{ex}[defn]{Example}

\theoremstyle{plain}
\newtheorem{prop}[defn]{Proposition}
\newtheorem{cor}[defn]{Corollary}
\newtheorem{lemma}[defn]{Lemma}
\newtheorem{thm}[defn]{Theorem}
\newtheorem{scholium}[defn]{Scholium}

\theoremstyle{remark}
\newtheorem{rem}[defn]{Remark}
\newtheorem{question}[defn]{Question}
\newtheorem{speculation}[defn]{Speculation}
\newtheorem{caveat}[defn]{Caveat}
\newtheorem{conjecture}[defn]{Conjecture}

\newenvironment{indentblock}{%
  \list{}{\leftmargin\leftmargin}%
  \item\relax
}{%
  \endlist
}

\newcommand{\xra}[1]{\xrightarrow{#1}}
\newcommand{\XXX}[1]{\textbf{XXX: #1}}
\newcommand{\aaa}{\mathfrak{a}}
\newcommand{\bbb}{\mathfrak{b}}
\newcommand{\mmm}{\mathfrak{m}}
\newcommand{\I}{\mathcal{I}}
\newcommand{\J}{\mathcal{J}}
\newcommand{\E}{\mathcal{E}}
\newcommand{\F}{\mathcal{F}}
\newcommand{\B}{\mathcal{B}}
\newcommand{\NN}{\mathbb{N}}
\newcommand{\RR}{\mathbb{R}}
\newcommand{\ZZ}{\mathbb{Z}}
\renewcommand{\SS}{\mathbb{S}}
\renewcommand{\P}{\mathcal{P}}
\renewcommand{\O}{\mathcal{O}}
\newcommand{\defeq}{\vcentcolon=}
\newcommand{\op}{\mathrm{op}}
\DeclareMathOperator{\Spec}{Spec}
\DeclareMathOperator{\Hom}{Hom}
\DeclareMathOperator{\Mod}{Mod}
\DeclareMathOperator{\Sh}{Sh}
\DeclareMathOperator{\PSh}{PSh}
\DeclareMathOperator{\rank}{rank}
\DeclareMathOperator{\length}{length}
\DeclareMathOperator{\List}{List}
\newcommand{\Set}{\mathrm{Set}}
\newcommand{\Eff}{\mathrm{Ef{}f}}
\renewcommand{\_}{\mathpunct{.}\,}
\newcommand{\effective}{ef{}fective\xspace}
\newcommand{\?}{\,{:}\,}
\newcommand{\realizes}{\Vdash}
\newcommand{\notnot}{\emph{not~not}\xspace}

\newcommand{\stacksproject}[1]{\cite[{\href{https://stacks.math.columbia.edu/tag/#1}{Tag~#1}}]{stacks-project}}

\renewcommand{\paragraph}[1]{\noindent\textbf{#1.}}

\newcommand{\PRA}{\textsc{pra}}
\newcommand{\ZF}{\textsc{zf}}
\newcommand{\IZF}{\textsc{izf}}
\newcommand{\CZF}{\textsc{czf}}
\newcommand{\ZFC}{\textsc{zfc}}
\newcommand{\ZFCS}{\textsc{zfc/s}}
\newcommand{\ZFCI}{\textsc{zfc+i}}
\newcommand{\RRS}{\textsc{rrs}}
\newcommand{\RDC}{\textsc{rdc}}
\newcommand{\DC}{\textsc{dc}}
\newcommand{\MDC}{\textsc{mdc}}
\newcommand{\ES}{(\IZF+\RRS_2)/\textsc{s}}

\begin{document}

\begin{abstract}
  We study the well-known reflection principle of Zermelo--Fraenkel set theory
  in the context of intuitionistic Zermelo--Fraenkel set theory~$\IZF$. We show
  that the reflection principle is equivalent to~$\RRS_2$, a strengthened
  version of Aczel's relation reflection scheme. As
  applications, we give a new proof that relativized dependent choice is
  equivalent to the conjunction of the relation reflection scheme and dependent
  choice and we present an intuitionistic version of Feferman's~$\ZFCS$, a
  conservative extension of~$\ZFC$ which is useful as a foundation for category
  theory.
\end{abstract}

\maketitle
\thispagestyle{empty}

\section{Introduction}

The basic form of the reflection principle for Zermelo--Fraenkel set
theory~$\ZF$ is the following.
\begin{thm}Let~$\varphi(x_1,\ldots,x_n)$ be a formula in the language of set
theory with free variables as indicated. Then~$\ZF$ proves
\[ \forall M\_
  \exists S \supseteq M\_
  \forall x_1,\ldots,x_n \in S\_
  \varphi(x_1,\ldots,x_n) \Leftrightarrow \varphi^S(x_1,\ldots,x_n), \]
where~$\varphi^S$ is the~$S$-relativization of~$\varphi$, obtained by
substituting any occurrence of~``$\forall x$'' and~``$\exists x$'' by~``$\forall
x \in S$'' and~``$\exists x \in S$''.
Furthermore, the resulting set~$S$ may be supposed to be transitive, to be closed
under subsets or even to be a stage~$V_\alpha$ of the cumulative hierarchy;
% XXX: "hence" instead of "or even"?
and given not a single formula~$\varphi$ but a finite
list~$\varphi_1,\ldots,\varphi_s$ of formulas, we may suppose that~$S$ reflects
all of them.
\end{thm}

The reflection principle is important both for philosophical and for practical
reasons: Philosophically, it tells us that truth of any formula can already be
checked in an initial segment of the universe. Practically, it allows to
% XXX: be more precise about philosophy
transfer results obtained for a restricted class of objects to all such
objects. For instance, if we manage to verify (the $S$-relativization of) some
group-theoretic statement for all groups contained in an arbitrary set~$X$,
then we may deduce by the reflection principle that the statement holds
for all groups in the universe.

This observation has been used by Feferman to construct~$\ZFCS$ (``$\ZFC$ with
smallness''), a conservative extension of~$\ZFC$ which provides a useful
foundation of category theory~\cite{feferman:zfcs}. This system extends~$\ZFC$
by the addition of a new constant symbol~$\SS$ together with axioms stating
that~$\SS$ is transitive, closed under subsets and reflective with respect to
every formula~$\varphi(x_1,\ldots,x_n)$ of the original language:
\[ \label{ax:refl}\tag{$\star$}
\forall x_1,\ldots,x_n \in \SS\_ \varphi(x_1,\ldots,x_n) \Leftrightarrow
\varphi^{\SS}(x_1,\ldots,x_n). \]
The system~$\ZFCS$ is conservative over~$\ZFC$ because any given proof in~$\ZFCS$ uses
only a finite number of instances of the axioms~\eqref{ax:refl}, whereby the reflection
principle can be used to yield an honest set~$S$ which validates just the same equivalences
and can hence be used in place of~$\SS$.

Elements of~$\SS$ are deemed ``small'', so that~$\SS$ is the set of all small
sets. The system~$\ZFCS$ is useful as a foundation for category theory because
it supports, without requiring new set-theoretical commitments such as the
existence of Grothendieck universes, a native treatment of large structures.
For instance, the category of all small sets can be formed entirely
within~$\ZFCS$, without resorting to classes. We invite the reader wanting
to learn more about the merits of~$\ZFCS$ to study the survey by
Shulman~\cite[Section~11]{shulman:sets}.

To our naive eyes, the passage from~$\ZFC$ to~$\ZFCS$ looked sufficiently
innocent so that we set out to create an intuitionistic version of~$\ZFCS$: To
our mind, size issues were entirely separate from issues of constructivity, and
we hence opined that they should be dealt with separately. Such a separation
would not only lead to improved mental hygiene and better understanding, but
would also allow us to use the benefits
of~$\ZFCS$ in situations where the law of excluded middle and the axiom of
choice are not available, such as in realizability semantics, sheaf semantics
or quite generally topos semantics.

We expected this modification to be entirely straightforward. However, the
situation turned out to be more subtle and we failed in our original goal of
verifying the reflection principle for intuitionistic Zermelo--Fraenkel set
theory~$\IZF$. The situation for~$\CZF$, the predicative subsystem of~$\IZF$
commonly heralded as the largest common denominator of all flavors of
constructive set theory, is even more mysterious.

However, we succeeded in verifying the reflection principle for a slight
extension of~$\IZF$:

\begin{thm}The reflection principle is equivalent, over~$\IZF$, to the strong
relation reflection scheme~$\RRS_2$.\end{thm}

We also give a weaker version of the reflection principle which is equivalent
to Aczel's original reflection scheme~$\RRS$.
Both~$\RRS$ and~$\RRS_2$ will be reviewed below. They are validated not
only by~$\ZFC$, but also by~$\ZF$, and furthermore by
all known models of~$\IZF$, hence might be regarded as not entirely
% XXX: be more precise in "all known"
unconstructive, even though they are conjectured to be independent of~$\IZF$. As a
result, the question whether the reflection principle holds for~$\IZF$ remains
open (though conjectured to be false), and for the stronger system~$\IZF+\RRS_2$
we can give a variant ``with smallness'' which can serve as a set-theoretic
foundation for category theory.

\begin{conjecture}The system $\IZF$ does not prove the reflection
principle.\end{conjecture}

\begin{defn}The system $\ES$ is obtained from~$\IZF+\RRS_2$ by
adding a constant symbol~$\SS$ together with axioms
stating that~$\SS$ is transitive, closed under subsets and reflective for all
formulas of the original language (that is, adding one instance
of~\eqref{ax:refl} for each formula of~$\IZF$).\end{defn}

\begin{cor}The system~$\ES$ is conservative
over~$\IZF+\RRS_2$.\end{cor}

This note is organized as follows. Section~\ref{sect:review} reviews the
classical proof of the reflection principle in the context of~$\ZF$ set theory.
Section~\ref{sect:axioms} reviews Aczel's relation reflection scheme and its
variants. Our main result is presented in Section~\ref{sect:constructive-proof}.
We conclude in Section~\ref{sect:appl-rdc} and
Section~\ref{sect:appl-smallness} with two short applications.

\textbf{Acknowledgments.} XXX


\section{Review of the classical reflection principle}
\label{sect:review}

A basic proof of the reflection principle in~$\ZF$ runs as follows. Our proof
of the reflection principle in~$\IZF+\RRS_2$ will follow the same outline.

\begin{lemma}\label{lemma:zf-smallstep}
Let~$\varphi(u_1,\ldots,u_n,x)$ be a formula in the language of set
theory with free variables as indicated. Then~$\ZF$ proves
\begin{multline*}
  {\qquad}\forall M\_
  \exists S \supseteq M\_
  \forall u_1,\ldots,u_n \in S\_ \\
  (\exists x\_ \varphi(u_1,\ldots,u_n,x)) \Longrightarrow
  (\exists x \in S\_ \varphi(u_1,\ldots,u_n,x)).{\qquad}
\end{multline*}
Furthermore: (1)~We may suppose that~$S$ is transitive. (2)~We may suppose that~$S$ is
closed under subsets. (3)~Given a finite list~$\varphi_1,\ldots,\varphi_s$
of formulas instead of the single formula~$\varphi$, we may suppose that~$S$
bounds all of the formulas~$\varphi_i$.
% XXX: better explanation
\end{lemma}

\begin{proof}Given a class~$X$ (as commonly understood as the comprehension of
a formula), we denote by~$X^\sim$ its subclass~$\{ x \in X \,|\, \forall y
\in X\_ \rank(x) \leq \rank(y) \}$, where the rank function refers to the stage
in the cumulative hierarchy. The two fundamental properties of this construction are:
This subclass is equal to a set,\footnote{If~$X$ is empty, then
this claim is trivial; if~$X$ is inhabited by some element~$x_0$, then~$X^\sim$ can
be obtained using separation from~$V_{\rank(x_0)+1}$; and in fact, by an argument
using the set of truth values and unbounded separation, the claim can also be
proven in~$\IZF$.} and it is inhabited if and only if~$X$ is.

Starting with~$S_0 \defeq M$, we construct~$S_{k+1}$ from~$S_k$ as the union
\[ S_{k+1} \defeq S_k \cup \bigcup_{u_1,\ldots,u_n \in S_k} \{ x \,|\,
\varphi(u_1,\ldots,u_n,x) \}^\sim. \]
It is then easy to check that~$S \defeq \bigcup_{k \in \NN} S_k$ is a set with the
required property.

For addendum~(1), we change the definition of~$S_{k+1}$ to be the transitive
closure of what is was before. To further accommodate addendum~(2), we change
this definition again, to be~$P^\omega$ of what is was before,
where~$P^\omega(X) \defeq \bigcup_{\ell \in \NN} P^{(\ell)}(X)$ is the union of
iterated powersets. For addendum~(3), we change the definition of~$S_{k+1}$ to
include one summand for each formula~$\varphi_i$.
\end{proof}

The proof of Lemma~\ref{lemma:zf-smallstep} is mostly constructive; however,
there is one issue with nontrivial ramifications: While~$\IZF$ does show
that the subclass~$X^\sim$ of a given class~$X$ is a set, it does not verify
that~$X^\sim$ is inhabited if~$X$ is. This would amount to the constructive
taboo that any inhabited set contains a rank-minimal element; and in fact, by
a result of Friedman and Scedrov~\cite{XXX}, no definable substitute
for~$X^\sim$ exists. The remedy presented in
Section~\ref{sect:constructive-proof} will construct~$X^\sim$ in a non-unique
fashion and then deal with the resulting fallout that taking the union requires
additional care.

\begin{thm}\label{thm:zf-bigstep}
Let~$\varphi(x_1,\ldots,x_n)$ be a formula in the language of set
theory with free variables as indicated. Then~$\ZF$ proves
\[ \forall M\_
  \exists S \supseteq M\_
  \forall x_1,\ldots,x_n \in S\_
  \varphi(x_1,\ldots,x_n) \Leftrightarrow \varphi^S(x_1,\ldots,x_n), \]
where~$\varphi^S$ is the~$S$-relativization of~$\varphi$. Furthermore, the
resulting set~$S$ may be supposed to be transitive and to be closed under
subsets; and given not a single formula~$\varphi$ but a finite
list~$\varphi_1,\ldots,\varphi_s$ of formulas, we may suppose that~$S$ reflects
all of them.
\end{thm}

\begin{proof}Let a set~$M$ be given. We obtain~$S$ by applying
Lemma~\ref{lemma:zf-smallstep} to the list of all subformulas of~$\varphi$. That
% XXX: "all" subformulas
the resulting set~$S$ has the required property can be checked by an induction on the structure
of~$\varphi$. The cases~$\text{``$=$''}, \text{``$\in$''}, \text{``$\top$''},
\text{``$\bot$''}, \text{``$\wedge$''},
\text{``$\vee$''}, \text{``$\Rightarrow$''}$ follow
trivially from the induction hypothesis. The case~``$\forall$'' does
not need to be treated since we may assume without loss of generality that all
universal quantifiers in~$\varphi$ have been rewritten
as~``$\neg\exists\neg$''.

The remaining case is where~$\varphi$ is of the form~$\varphi \equiv (\exists
x\_ \psi(u_1,\ldots,u_m,x))$. In this case, the claim is that
\[ \forall u_1,\ldots,u_m \in S\_
  (\exists x\_ \psi(u_1,\ldots,u_m,x)) \Longleftrightarrow
  (\exists x \in S\_ \psi^S(u_1,\ldots,u_m,x)). \]
This claim follows by the property of~$S$ guaranteed by
Lemma~\ref{lemma:zf-smallstep} and by the induction hypothesis concerning the
subformula~$\psi^S(u_1,\ldots,u_m,x)$.
\end{proof}

The proof of Theorem~\ref{thm:zf-bigstep} is constructive with the sole
exception of treating the case of universal quantifiers by appealing to the law
of excluded middle. For the constructive proof in
Section~\ref{sect:constructive-proof}, we will instead appeal to a strengthened
version of Lemma~\ref{lemma:zf-smallstep}.


\section{Aczel's relation reflection axiom scheme and its variants}
\label{sect:axioms}

We write~``$R : X \rightrightarrows Y$'' to mean~``$\forall x \in X\_ \exists y
\in Y\_ \langle x,y \rangle \in R$''.

\begin{defn}Let~$X$ and~$R \subseteq X \times X$ be classes. Let~$x_0 \in X$.
\begin{itemize}
\setlength{\itemsep}{0.3em}
\item[$\DC$] \emph{dependent choice} \\ If~$X$ and~$R$ are sets and
if~$R : X \rightrightarrows X$, then there is a function~$f : \NN \to X$ such that~$f(0) =
x_0$ and~$\langle f(k), f(k+1) \rangle \in R$ for all numbers~$k$.
\item[$\RDC$] \emph{relativized dependent choice} \\ If~$R : X
\rightrightarrows X$, then there is a function~$f : \NN \to X$ such that~$f(0)
= x_0$ and~$\langle f(k), f(k+1) \rangle \in R$ for all numbers~$k$.
\item[$\RRS$] \emph{Aczel's relation reflection scheme} \\
If~$R : X \rightrightarrows X$, then there is
a set~$B$ such that~$x_0 \in B \subseteq X$ and~$R : B \rightrightarrows B$.
\item[$\MDC$] \emph{Palmgren's multivalued dependent choice} \\
If~$R : X \rightrightarrows X$, then there is a function~$f : \NN \to P(X)$
(the class of all subsets of~$X$) such that~$x_0 \in f(0)$ and such
that~$\forall x \in f(k)\_ \exists y \in f(k+1)\_ \langle x,y \rangle \in R$
for every number~$k$.
\end{itemize}
\end{defn}

\begin{defn}\label{defn:rrs2}
Let~$X$ and~$R \subseteq X \times X \times X$ be classes. Let~$x_0 \in X$.
\begin{itemize}
\setlength{\itemsep}{0.3em}
\item[$\RRS_2$]
If~$R : X \times X \rightrightarrows X$, then there is a set~$B$ such that~$x_0
\in B \subseteq X$ and~$R : B \times B \rightrightarrows B$.
\item[$\MDC_2$]
If~$R : X \times X \rightrightarrows X$, then there is a function~$f : \NN \to
P(X)$ such that~$x_0 \in f(0)$ and such that~$\forall x \in f(k)\_ \forall x'
\in f(k')\_ \exists y \in f(\max\{k,k'\}+1)\_ \langle x,x',y \rangle \in R$ for
any numbers~$k, k'$.
\end{itemize}
\end{defn}

Aczel's relation reflection scheme~$\RRS$ first surfaced in
the theory of coinductive definitions of classes and enjoys substantial
stability properties, as it passes from the meta theory to XXX[all kinds of]
models. Background on~$\RRS$ can be found in Aczel's original article
introducing it~\cite{aczel:rrs}. $\RRS$ is equivalent, over~$\CZF$ and
a~fortiori over~$\IZF$, to Palmgren's multivalued dependent
choice~\cite{palmgren:mdc}; this observation makes the relationship to
dependent choice more visible.

\begin{rem}Most published renderings of~$\RRS$ and~$\MDC$ do not refer to an initial
element~$x_0 \in X$, but to a set~$A \subseteq X$ and then require instead
of that~$x_0 \in B$, that~$A \subseteq B$. This difference is immaterial, thanks in
one direction to the existence of singleton sets and in the other to strong
collection. The same is true for~$\RRS_2$ and~$\MDC_2$, though the proof is not
as easy.
\end{rem}

% First prove that in RRS_2^∈, we may suppose that not only a single given
% element x0 is contained in B, but also a further given element x1.
% Do this by applying RRS_2^∈ to X^ := { ∅ } U { {x} | x ∈ X } and
% R^(p,p',q) given by:
% * p = ∅   and p' = ∅    and q = {x0}  or
% * p = ∅   and p' = {x0} and q = {x1}  or  the other way round  or
% * p = {x} and p' = {x'} and q = {y} and R(x,x',y).
%
% Now let R : X × X ⇉ X and a set A ⊆ X be given. We want to verify RRS_2^⊆
% for this data.
%
% Given H ∈ P(X), H' ∈ P(X), consider:
%
%    Let x ∈ H, x' ∈ H'.
%    Then by the strengthened RRS_2^∈, there is a set B ⊆ X
%    such that x, x' ∈ B and R : B × B ⇉ B.
%
% We can use collection to obtain a set J of such B's, and consider U J.
%
% We can then apply RRS_2^∈ to this relation to obtain a set T such that
% A ∈ T ⊆ P(X) and ... Set C := U T. Then A ⊆ C ⊆ X and R : C × C ⇉ C.

Aczel's $\RRS$ should not in itself be regarded as a choice principle; however in
its presence, ordinary~$\DC$ entails~$\RDC$. This result is due to
Aczel~\cite[Theorem~2.4]{aczel:rrs}, who proved the equivalence~$\RDC = \RRS +
\DC$ over~$\CZF^-$, and we give a new proof of this equivalence over the
stronger theory~$\IZF$, in Section~\ref{sect:appl-rdc}.

In the presence of~$\DC$, $\RRS$ is equivalent to~$\RRS_2$:

\begin{prop}\label{prop:rdc-rrs2-dc}
Over~$\CZF^-$, $\RDC$ is equivalent to~$\RRS_2 + \DC$.
\end{prop}

\begin{proof}Trivially,~$\RDC$ entails~$\DC$, and~$\RRS_2 + \DC$
entails~$\RDC$ by Aczel's result since~$\RRS_2$ entails~$\RRS$.

To verify that~$\RDC$ entails~$\RRS_2$, let classes~$X$ and~$R \subseteq X
\times X \times X$ be given, let~$x_0 \in X$ be an element and assume~$R : X
\times X \rightrightarrows X$. Let~$\List(X)$ be the class of finite lists with
entries in~$X$. We declare a class~$\widehat X$ by
\[ \widehat X \defeq \{ \langle i,j,v \rangle \,|\, i,j \in
\NN, v \in \List(X), i,j < \length(v) \} \]
and a relation~$\widehat R \subseteq \widehat X \times \widehat X$ by
defining~$\langle \langle i,j,v \rangle, \langle i',j',v' \rangle \rangle \in
\widehat R$ to be equivalent to
\begin{indentblock}
there exists an element~$y \in X$ such that
\begin{enumerate}
\item $\langle v!i, v!j, y \rangle \in R$ (where~$v!k$ is the element of the
list~$v$ at position~$k$),
\item $v'$ is obtained from~$v$ by adding the single element~$y$ at the end and
\item $\langle i',j' \rangle$ is the next point after~$\langle i,j
\rangle$ in some fixed enumeration of~$\NN^2$ which, for each number~$n$, first visits
all points~$\langle k,l \rangle$ with~$k,l < n$ before it visits any of
the other points.
\end{enumerate}
\end{indentblock}
Then~$\widehat R : \widehat X \rightrightarrows \widehat X$. Applying~$\RDC$
with initial value~$\langle 0,0, [x_0] \rangle \in \widehat X$ yields a
function~$f : \NN \to \widehat X$. Let~$B$ be the set of all
entries of the lists contained in the tuples~$f(k)$. Then~$x_0 \in B
\subseteq X$ and~$R : B \times B \rightrightarrows B$.
\end{proof}

A consequence of Proposition~\ref{prop:rdc-rrs2-dc} is that~$\RRS_2$ holds in
Aczel's type-theoretic ``sets as trees'' model
of~$\CZF$~\cite{aczel:sets-as-trees}, since that model validates~$\RDC$.

\begin{prop}Over~$\CZF^-$, $\RRS_2$ is equivalent to~$\MDC_2$.\end{prop}

\begin{proof}The proof in~\cite{palmgren:mdc} carries over.\end{proof}

% XXX: RRS_2 holds in ZF.


\section{Constructivizing the reflection principle}
\label{sect:constructive-proof}

Our constructive rendition of the reflection principle will require the axiom
scheme~$\RRS_2$ displayed in Definition~\ref{defn:rrs2}. This result is the
best possible, as we verify in Theorem~\ref{thm:refl-entails-rrs2} that
conversely the reflection principle entails~$\RRS_2$.

Even though superficially similar, the following lemma is \emph{not} yet a
constructivization of Lemma~\ref{lemma:zf-smallstep}; these two lemmas differ
in the set from which~$u_1,\ldots,u_n$ are drawn.

\begin{lemma}\label{lemma:izf-microstep}
Let~$\varphi(u_1,\ldots,u_n,x)$ be a formula in the language of set
theory with free variables as indicated. Then~$\IZF$ proves
\begin{multline*}
  {\qquad}\forall H\_
  \exists H' \supseteq H\_
  \forall u_1,\ldots,u_n \in H\_ \\
  \begin{array}{@{}rcl@{}}
  (\exists x\_ \varphi(u_1,\ldots,u_n,x)) &\Longrightarrow&
  (\exists x \in H'\_ \varphi(u_1,\ldots,u_n,x)) \quad\mathop{\wedge} \\
  (\forall x\_ \varphi(u_1,\ldots,u_n,x)) &\Longleftarrow&
  (\forall x \in H'\_ \varphi(u_1,\ldots,u_n,x)).\end{array}
  {\qquad}
\end{multline*}
Furthermore: (1)~We may suppose that~$H'$ is transitive. (2)~We may suppose
that~$H'$ is closed under subsets. (3)~Given a finite
list~$\varphi_1,\ldots,\varphi_s$ of formulas instead of the single
formula~$\varphi$, we may suppose that~$H'$ has the displayed property for each
of the formulas~$\varphi_i$.
\end{lemma}

\begin{proof}Let~$\Omega \defeq P(\{0\})$ be the set of truth values.
For given elements~$u_1,\ldots,u_n \in H$, we have
\[ \begin{array}{@{}l@{}l@{}}
  \forall a \in \{ a \in \{0\} \,|\, \exists x\_ \varphi(u_1,\ldots,u_n,x) \}\_ &
  \exists x\_ \varphi(u_1,\ldots,u_n,x) \quad\text{and} \\
  \forall p \in \{ p \in \Omega \,|\, \exists x\_ (0 \in p \Leftrightarrow
  \varphi(u_1,\ldots,u_n,x)) \}\_ &
  \exists x\_ (0 \in p \Leftrightarrow \varphi(u_1,\ldots,u_n,x)).
\end{array} \]
Hence, by collection, there are sets~$C$ and~$D$ such that
\[ \begin{array}{@{}rcl@{}}
  (\exists x\_ \varphi(u_1,\ldots,u_n,x)) &\Longrightarrow&
  (\exists x \in C\_ \varphi(u_1,\ldots,u_n,x)) \quad\text{and} \\
  (\forall x\_ \varphi(u_1,\ldots,u_n,x)) &\Longleftarrow&
  (\forall x \in D\_ \varphi(u_1,\ldots,u_n,x)).
\end{array} \]
The union~$C \cup D$ satisfies both of these conditions at once.

Applying collection again, there is a set~$X$ such that for
any~$u_1,\ldots,u_n \in H$ there exists a set~$E \in X$ such that
\[ \begin{array}{@{}rcl@{}}
  (\exists x\_ \varphi(u_1,\ldots,u_n,x)) &\Longrightarrow&
  (\exists x \in E\_ \varphi(u_1,\ldots,u_n,x)) \quad\text{and} \\
  (\forall x\_ \varphi(u_1,\ldots,u_n,x)) &\Longleftarrow&
  (\forall x \in E\_ \varphi(u_1,\ldots,u_n,x)).
\end{array} \]
Hence the set~$H' \defeq M \cup \bigcup X$ has the required property.

To ensure that~$H'$ is transitive and closed under subsets, we pass
first to its transitive closure and then compute the union of all its
finitely-iterated powersets.

In order to accommodate more than a single formula~$\varphi$, we add one summand
in the definition of~$H'$ for each formula~$\varphi_i$.
\end{proof}

\begin{lemma}\label{lemma:izf-smallstep}
Let~$\varphi(u_1,\ldots,u_n,x)$ be a formula in the language of set
theory with free variables as indicated. Then~$\IZF+\RRS_2$ proves
\begin{multline*}
  {\qquad}\forall M\_
  \exists S \supseteq M\_
  \forall u_1,\ldots,u_n \in S\_ \\
  \begin{array}{@{}rcl@{}}
  (\exists x\_ \varphi(u_1,\ldots,u_n,x)) &\Longrightarrow&
  (\exists x \in S\_ \varphi(u_1,\ldots,u_n,x)) \quad\mathop{\wedge} \\
  (\forall x\_ \varphi(u_1,\ldots,u_n,x)) &\Longleftarrow&
  (\forall x \in S\_ \varphi(u_1,\ldots,u_n,x)).\end{array}
  {\qquad}
\end{multline*}
Furthermore: (1)~We may suppose that~$S$ is transitive. (2)~We may suppose
that~$S$ is closed under subsets. (3)~Given a finite
list~$\varphi_1,\ldots,\varphi_s$ of formulas instead of the single
formula~$\varphi$, we may suppose that~$S$ bounds all of the
formulas~$\varphi_i$.
% XXX: better explanation
\end{lemma}

\begin{proof}By~$\RRS_2$, there is a set~$B$ such that~$M \in B$ and such that
for any~$H_1, H_2 \in B$, there is a set~$H' \in B$ such that~$H \defeq H_1
\cup H_2$ and~$H'$ are related as in the conclusion of Lemma~\ref{lemma:izf-microstep}.

We set~$S \defeq \bigcup B$. This set has the required property. To verify
this, it is useful to observe that given~$u_1,\ldots,u_n \in S$, there is a
common set~$H \in B$ such that~$u_1,\ldots,u_n \in H$.

In order to ensure addendum~(1) and~(2), we apply~$\RRS_2$ in a slightly
different way to guarantee that for any sets~$H_1, H_2 \in B$, there is a
set~$H' \in B$ such that~$H \defeq H_1 \cup H_2$ and~$H'$ are related as in the
conclusion of Lemma~\ref{lemma:izf-microstep} and such that furthermore~$H'$ is
transitive and closed under subsets. Even though it cannot be expected that any
particular set~$H \in B$ will be transitive and closed under subsets, the
union~$S$ will. Addendum~(3) can be ensured because of addendum~(3) of
Lemma~\ref{lemma:izf-microstep}.
\end{proof}

\begin{rem}In the special case~$n = 1$, the proof of Lemma~\ref{lemma:izf-smallstep} can
be simplified to only use~$\RRS$ instead of~$\RRS_2$.\end{rem}

\begin{thm}\label{thm:izf-bigstep}
Let~$\varphi(x_1,\ldots,x_n)$ be a formula in the language of set
theory with free variables as indicated. Then~$\IZF+\RRS_2$ proves
\[ \forall M\_
  \exists S \supseteq M\_
  \forall x_1,\ldots,x_n \in S\_
  \varphi(x_1,\ldots,x_n) \Leftrightarrow \varphi^S(x_1,\ldots,x_n), \]
where~$\varphi^S$ is the~$S$-relativization of~$\varphi$.
Furthermore, the resulting set~$S$ may be supposed to be transitive and to be closed
under subsets; and given not a single formula~$\varphi$ but a finite
list~$\varphi_1,\ldots,\varphi_s$ of formulas, we may suppose that~$S$ reflects
all of them.
\end{thm}

\begin{proof}The proof of Theorem~\ref{thm:zf-bigstep} carries over. The only
difference is that instead of Lemma~\ref{lemma:zf-smallstep},
Lemma~\ref{lemma:izf-smallstep} has to be used, and that the case for the
universal quantifier has to be treated just as the case for the existential
quantifier has to.\end{proof}

% XXX: State mini-reflection theorem.

\begin{thm}\label{thm:refl-entails-rrs2}
Over~$\IZF$, each instance of Aczel's relation reflection scheme~$\RRS_2$ can be
deduced from suitable instances of the assumption that, given a finite list
of formulas, for every set~$M$ there is a set~$S \supseteq M$ reflecting the
given formulas.
\end{thm}

\begin{proof}Let~$X$ and~$R \subseteq X \times X \times X$ be classes. Let~$x_0 \in X$
be an element and suppose~$R : X \times X \rightrightarrows X$.

By assumption, there is a set~$S \ni x_0$ which reflects the three
formulas~``$x \in X$'', ``$\langle x,x',y \rangle \in R$'' and~``$\exists y \in X\_ \langle
x,x',y \rangle \in R$''.

The class~$B \defeq X \cap S$ is a set by separation and contains~$x_0$.
Given~$x,x' \in B$, there is a set~$y$ such that~$y \in X$ and~$\langle x,y,y'
\rangle \in R$. By the reflecting property of~$S$, we can assume that such an
element~$y$ exists in~$S$.

Hence~$R : B \times B \rightrightarrows B$.
\end{proof}


\section{A new proof of~$\RDC = \RRS + \DC$}
\label{sect:appl-rdc}

When he introduced~$\RRS$, Aczel proved that over $\CZF$ without subset
collection, relative dependent choice $\RDC$ is equivalent to the conjunction
of~$\RRS$ and dependent choice~$\DC$~\cite[Theorem~2.4]{aczel:rrs}. Using
reflection, we can provide a new proof of this fact, although over the much
stronger base theory~$\IZF$ instead of~$\CZF^{-}$. The idea is that reflection
allows to reduce~$\RDC$ to~$\DC$.

\begin{prop}Over~$\IZF$, $\RDC$ is equivalent to~$\RRS + \DC$.
\end{prop}

\begin{proof}Trivially,~$\RDC$ implies~$\DC$, and~$\RDC$ implies~$\RRS$ by a
similar, though much simpler, argument as in the proof of
Proposition~\ref{prop:rdc-rrs2-dc}.

Conversely, assume~$\RRS$ and~$\DC$. In order to verify~$\RDC$, let classes~$X$
and~$R \subseteq X \times X$ be given. Let~$x_0 \in X$ and assume~$\forall x
\in X\_ \exists y \in X\_ \langle x,y \rangle \in R$.

\textbf{XXX: Use mini-reflection theorem instead.}
By Theorem~\ref{thm:izf-bigstep}, reflection is available; hence there is a
set~$S \supseteq \{x_0\}$ which reflects the three formulas~``$x \in X$'', ``$\langle x,y
\rangle \in R$'' and~``$\forall x\_ x \in X \Rightarrow \exists y\_ (y \in X \wedge \langle x,y
\rangle \in R)$''.

Hence~$\forall x \in X \cap S\_ \exists y \in X \cap S\_ \langle x,y \rangle
\in R$. By~$\DC$, there is a choice function~$f : \NN \to X \cap S$ such
that~$f(0) = x_0$ and~$\langle f(k), f(k+1) \rangle \in R$ for all numbers~$k$.
This is a function of the kind required by~$\RDC$.
\end{proof}


\section{An intuitionistic version of Feferman's~$\ZFCS$}
\label{sect:appl-smallness}

XXX: repeat definition and conservativity statement?

The system~$\ES$ is also interesting from the point of view of topos theory.
Any topos supports an internal language which can be used to reason about the
objects and morphisms of the topos in a naive element-based language, allowing
us to pretend that the objects are plain sets (or types) and that the morphisms
are plain maps between those sets (\cite[Chapter~6]{borceux:handbook3},
\cite[Section~1.3]{caramello:ttt}, \cite[Chapter~14]{goldblatt:topoi},
\cite[Chapter~VI]{moerdijk-maclane:sheaves-logic}).

Given a topos~$\mathcal{E}$ and a formula~$\varphi$ in its internal language,
we write~``$\mathcal{E} \models \varphi$'' to mean that~$\varphi$ holds
in~$\E$. As a special case, truth in the topos~$\Set$, the category of all sets,
coincides with truth in the background theory; symbolically: $\Set \models
\varphi$ iff~$\varphi$.

However, in the context of~($\IZF$ or)~$\ZFC$, it is difficult to make this
claim precise. Because in~$\ZFC$ the category~$\Set$ of all sets can not be
coded as a set,~$\ZFC$ cannot define truth in~$\Set$. We must therefore resort
to a meta theory in order to express this claim, for instance by stating
that primitive recursive arithmetic~$\PRA$ proves that for any
formula~$\varphi$ of~$\ZFC$, $\ZFC$ proves~``$(\Set \models \varphi)
\Leftrightarrow \varphi$'', where~``$\Set \models \varphi$'' is to be unrolled
by~$\PRA$.

An alternative is offered by~$\ZFCI$, $\ZFC$ plus the existence of a
strongly inaccessible cardinal. In this system, there is a Grothendieck
universe~$U$; we can form the category~$\Set_U$ of all sets of~$U$ as an honest
set; can define truth in~$\Set_U$; and prove, within the system, that for any
formula~$\varphi$, $(\Set_U \models \varphi) \Leftrightarrow (U \models
\varphi)$. This even holds for formulas of the full infinitary language of
toposes, which allows infinite disjunctions and infinite conjunctions; this
extended language could not be treated by resorting to~$\PRA$ as indicated above.

However, truth in a Grothendieck universe~$U$ need not be related to actual
truth. A solution is provided by~$\ZFCS$ and by~$\ES$. In these systems, we can
form the topos~$\Set_\SS$, define truth in it, prove for all formulas~$\varphi$
that~$(\Set_\SS \models \varphi) \Leftrightarrow (\SS \models \varphi)$; and
reflection for~$\SS$ ensures that for each (external, standard)
formula~$\varphi$, the system proves~``$(\Set_\SS \models \varphi)
\Leftrightarrow \varphi$''.

\printbibliography

\end{document}

% XXX: Review axioms of IZF.

% XXX: Think about pairs.

% XXX: Remark that the new proof of reflection also works over ZF, but that the
% usual proof is probably preferable.

% XXX: Über Verallgemeinerung zu CZF spekulieren.

% XXX: Note that the result fails over IZF_R.

% XXX: Reference more choice axioms? e.g. Aczel and Rathjen; and that one PhD
% thesis
