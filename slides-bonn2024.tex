\documentclass[12pt,utf8,notheorems,compress,t,aspectratio=169]{beamer}
\usepackage{etex}

\usepackage{pgfpages}
\usepackage[export]{adjustbox}

% Workaround for the issue described at
% https://tex.stackexchange.com/questions/164406/beamer-using-href-in-notes.
\newcommand{\fixedhref}[2]{\makebox[0pt][l]{\hspace*{\paperwidth}\href{#1}{#2}}\href{#1}{#2}}

\usepackage[english]{babel}

\usepackage[normalem]{ulem}
\usepackage{mathtools}
\usepackage{booktabs}
\usepackage{stmaryrd}
\usepackage{amssymb}
\usepackage{manfnt}
\usepackage{array}
\usepackage{ragged2e}
\usepackage{multicol}
\usepackage{tabto}
\usepackage{xstring}
\usepackage{proof}
\usepackage[all]{xy}
\xyoption{rotate}
\usepackage{tikz}
\usetikzlibrary{calc,shapes,shapes.callouts,shapes.arrows,patterns,fit,backgrounds,decorations.pathmorphing,positioning,svg.path}
\hypersetup{colorlinks=true}

\newcommand*\circled[1]{\tikz[baseline=(char.base)]{%
  \node[shape=circle,draw,inner sep=1pt] (char) {#1};}}

\DeclareFontFamily{U}{bbm}{}
\DeclareFontShape{U}{bbm}{m}{n}
   {  <5> <6> <7> <8> <9> <10> <12> gen * bbm
      <10.95> bbm10%
      <14.4>  bbm12%
      <17.28><20.74><24.88> bbm17}{}
\DeclareFontShape{U}{bbm}{m}{sl}
   {  <5> <6> <7> bbmsl8%
      <8> <9> <10> <12> gen * bbmsl
      <10.95> bbmsl10%
      <14.4> <17.28> <20.74> <24.88> bbmsl12}{}
\DeclareFontShape{U}{bbm}{bx}{n}
   {  <5> <6> <7> <8> <9> <10> <12> gen * bbmbx
      <10.95> bbmbx10%
      <14.4> <17.28> <20.74> <24.88> bbmbx12}{}
\DeclareFontShape{U}{bbm}{bx}{sl}
   {  <5> <6> <7> <8> <9> <10> <10.95> <12> <14.4> <17.28>%
      <20.74> <24.88> bbmbxsl10}{}
\DeclareFontShape{U}{bbm}{b}{n}
   {  <5> <6> <7> <8> <9> <10> <10.95> <12> <14.4> <17.28>%
      <20.74> <24.88> bbmb10}{}
\DeclareMathAlphabet{\mathbbm}{U}{bbm}{m}{n}
\SetMathAlphabet\mathbbm{bold}{U}{bbm}{bx}{n}

\usepackage{pifont}
\newcommand{\cmark}{\ding{51}}
\newcommand{\xmark}{\ding{55}}
\DeclareSymbolFont{extraup}{U}{zavm}{m}{n}
\DeclareMathSymbol{\varheart}{\mathalpha}{extraup}{86}

\graphicspath{{images/}}

\usepackage[protrusion=true,expansion=true]{microtype}

\setlength\parskip{\medskipamount}
\setlength\parindent{0pt}

\title{Constructive forcing}

\author{Ingo Blechschmidt}
\date{September 20th to September 16th, 2023}

%\setbeameroption{show notes on second screen=bottom}
\newcommand{\jnote}[2]{\only<#1>{\note{\setlength\parskip{\medskipamount}\footnotesize\justifying#2\par}}}

%\useinnertheme[shadow=true]
\setbeamerfont{block title}{size={}}

\useinnertheme{rectangles}

\usecolortheme{orchid}
\usecolortheme{seahorse}
\definecolor{mypurple}{RGB}{253,73,34}
\definecolor{mypurpledark}{RGB}{100,0,150}
\setbeamercolor{structure}{fg=mypurple}
\setbeamercolor*{title}{bg=mypurple,fg=white}
\setbeamercolor*{titlelike}{bg=mypurple,fg=white}
\setbeamercolor{frame}{bg=black}

\usefonttheme{serif}
\usepackage[T1]{fontenc}
\usepackage{libertine}

% lifted from https://arxiv.org/abs/1506.08870
\DeclareFontFamily{U}{min}{}
\DeclareFontShape{U}{min}{m}{n}{<-> udmj30}{}
\newcommand\yon{\!\text{\usefont{U}{min}{m}{n}\symbol{'210}}\!}

\newcommand{\A}{\mathcal{A}}
\newcommand{\B}{\mathcal{B}}
\newcommand{\C}{\mathcal{C}}
\newcommand{\M}{\mathcal{M}}
\renewcommand{\AA}{\mathbb{A}}
\newcommand{\BB}{\mathbb{B}}
\newcommand{\pp}{\mathbbm{p}}
\newcommand{\MM}{\mathbb{M}}
\newcommand{\E}{\mathcal{E}}
\newcommand{\F}{\mathcal{F}}
\newcommand{\FF}{\mathbb{F}}
\newcommand{\G}{\mathcal{G}}
\newcommand{\J}{\mathcal{J}}
\newcommand{\GG}{\mathbb{G}}
\renewcommand{\O}{\mathcal{O}}
\newcommand{\K}{\mathcal{K}}
\newcommand{\NN}{\mathbb{N}}
\newcommand{\QQ}{\mathbb{Q}}
\newcommand{\RR}{\mathbb{R}}
\newcommand{\TT}{\mathbb{T}}
\newcommand{\PP}{\mathbb{P}}
\newcommand{\ZZ}{\mathbb{Z}}
\newcommand{\CC}{\mathbb{C}}
\renewcommand{\P}{\mathcal{P}}
\newcommand{\aaa}{\mathfrak{a}}
\newcommand{\bbb}{\mathfrak{b}}
\newcommand{\ccc}{\mathfrak{c}}
\newcommand{\ppp}{\mathfrak{p}}
\newcommand{\fff}{\mathfrak{f}}
\newcommand{\mmm}{\mathfrak{m}}
\newcommand{\defeq}{\vcentcolon=}
\newcommand{\defeqv}{\vcentcolon\equiv}
\newcommand{\Cov}{\mathrm{Cov}}
\renewcommand{\_}{\mathpunct{.}}
\newcommand{\?}{\,{:}\,}
\newcommand{\speak}[1]{\ulcorner\text{\textnormal{#1}}\urcorner}
\newcommand{\inv}{inv.\@}
\newcommand{\forces}{\vDash}

\setbeamertemplate{blocks}[rounded][shadow=false]

\newenvironment{indentblock}{%
  \list{}{\leftmargin\leftmargin}%
  \item\relax
}{%
  \endlist
}

% Adapted from https://latex.org/forum/viewtopic.php?t=2251 (Stefan Kottwitz)
\newenvironment<>{hilblock}{
  \begin{center}
    \begin{minipage}{9.05cm}
      \setlength{\textwidth}{9.05cm}
      \begin{actionenv}#1
        \def\insertblocktitle{}
        \par
        \usebeamertemplate{block begin}}{
        \par
        \usebeamertemplate{block end}
      \end{actionenv}
    \end{minipage}
  \end{center}}

\newenvironment{changemargin}[2]{%
  \begin{list}{}{%
    \setlength{\topsep}{0pt}%
    \setlength{\leftmargin}{#1}%
    \setlength{\rightmargin}{#2}%
    \setlength{\listparindent}{\parindent}%
    \setlength{\itemindent}{\parindent}%
    \setlength{\parsep}{\parskip}%
  }%
  \item[]}{\end{list}}

\tikzset{
  invisible/.style={opacity=0,text opacity=0},
  visible on/.style={alt={#1{}{invisible}}},
  alt/.code args={<#1>#2#3}{%
    \alt<#1>{\pgfkeysalso{#2}}{\pgfkeysalso{#3}}}
}

% https://tex.stackexchange.com/questions/172336/drawing-roman-laurel-leaves-spqr-in-tikz
\tikzset{
  laurel-wreath/.pic = {
    \fill svg{M14.4-24.6c-1.5-1.5-2.6-3.3-3.1-5.3l-.4-1.7c-.2-1.1-.2-4.1 .2-5.7 .2-.9 .3-1.3 .5-1.3l1.4 1.1 2.5 2.4c2.7 2.5 5.2 6 5.8 8 .2 .6-.5 .3-2.2-.9-1.6-1.3-3.3-2.6-5-3.8l.1 1.4c.2 1.4 .5 2.7 1.1 4.6s.8 2.5 .5 2.5l-1.4-1.3zm69.6 1.1 .3-1.2c.8-2.3 1.3-4.8 1.6-7.3l-1.5 1.1c-1.3 .9-2.6 1.9-3.7 3-1.6 1.1-2 1.3-2.1 1 .7-1.8 1.6-3.4 2.8-4.9 1.3-1.7 6.5-6.8 7-6.8 .2 0 .3 .2 .3 .5l.3 1.6c.3 2.2 .2 5.7-.5 7.4-.8 1.9-1.6 3.1-3 4.7-1.1 1.1-1.4 1.3-1.5 .9z};
    \fill svg{M10-29.4c-.8-1.1-1.4-2.2-2-4.1l-.7-3.5c-.2-3 .2-4.4 1.4-8.3l.5-1.4c.2-1.3 .3-1.9 .6-1.9 .3-.2 .6 .3 .7 .8s.9 2.2 1.9 3.6c1.4 2.2 2.7 4.4 3.9 6.6l.9 2.7c0 .6 0 .6-.3 .6-.6 0-4.9-4.4-5.8-6l-.2-.6-.1 1.7-.3 2.8c-.3 2.7-.3 3.8 0 5.5 .6 2 .5 2.4-.5 1.5zm79.2 .3 .4-2.4c.2-1.3 .2-2.7-.1-4.9l-.3-2.8v-1.6l-.7 1c-.8 1.3-5 5.5-5.5 5.5s-.5-.3 .2-1.9c.5-1.7 1.4-3.3 3.3-6.5 2.4-3.6 2.7-3.9 2.8-4.7 .5-1.3 .5-1.4 .8-1.2 .3 0 .6 .8 .6 1.5l.7 2.4c.9 2.7 1.1 3.6 1.2 6 .2 3.1-.5 6-2 8.2-.8 1.3-1.3 1.7-1.4 1.5z};
    \fill svg{M5-40c-.4-3.2-.1-6.5 .9-9.6 .5-1.1 1.6-2.8 2.2-3.4l1.3-1.6 2-2.7 .2 .6c.1 1.3 .4 2.6 .9 3.8l.3 1c.8 1.7 1.1 2.7 1.6 5.3 .6 2.5 .6 4.6 .2 4.6-.3 0-.9-.8-1-1.1l-.5-.8c-1.4-2-3-5.2-2.9-6.5-.9 2.7-2 5.4-3.5 7.9l-.3 .8-.3 .8c0 .5-.6 1.6-.8 1.6l-.3-.7zm89.2 .2-.2-.5-.3-.9-1.1-2.7-1.1-2.4c-.6-1.4-1.2-2.8-1.6-4.2l-.3 .9c-.3 1.3-1.6 3.9-3 6-1.3 2-1.6 2-1.5 0s1.1-6.3 2.2-9c.8-1.7 1.1-3.1 .9-4.1-.2-1.1 .5-.8 2.2 1.8 3.3 4.4 3.8 5.4 4.4 7.8 .6 2.4 .5 7.7-.3 7.8l-.3-.5z};
    \fill svg{M13.9-50.1c-.5-1.9-.8-3.9-.9-5.8-.2-1.6-.1-3.3 .1-4.9-.3 .8-1.7 2.5-4.2 5.1l-3 4.9-.3 .1c-.3 0-.3-2.2 0-3.3 .8-3 1.4-4.6 2.5-6.1 .9-1.3 1.7-1.9 2.5-2.5 1.1-.6 2.7-1.9 3.5-2.7 .9-.9 1.9-1.4 2.2-1.4v1.1l-.3 6.6c0 6.8 .2 6.3-1 8.9-.5 1.1-.8 1.1-1.1 0zm70.8-.4c-.8-2.2-.8-2.5-.7-6.3-.1-2.7-.1-5.5-.2-8.2-.3-1.6-.3-1.9 .5-1.6l.6 .5c1.4 1.4 3 2.5 3.9 3.1 1.3 .9 1.9 1.6 2.7 2.6l.6 .7 .2 .4 .2 .3c.8 .9 2 4.9 2 6.9 .2 1.9-.2 1.9-.9 .5-.7-1.4-1.5-2.7-2.6-4-1.6-1.5-3-3.2-4.2-5 .4 3 .3 6-.5 9 0 .8-.5 2.2-.8 2.3-.2 0-.5-.3-.8-1.2z};
    \fill svg{M16.4-58.5l.2-1.5 .3-3.7c.2-2.8 .3-3.5 1.1-5.4l.7-1.3-.5 .4-1 .7c-.5 .4-1.1 .8-1.5 1.3l-.5 .3-1.9 1.6c-2.2 1.6-2.7 2-3.9 3.6-.5 .8-1.1 1.3-1.3 1.3-.5 0 0-2.4 1.1-4.7 1.5-3.4 4.3-6 7.7-7.4l1.3-.4 1.9-.4 2-.5c1.4 0 1.4 0 1 1.1-.5 .8-.8 2-1.1 4.2-.3 2.3-1.1 4.5-2.2 6.5l-.4 .6c-.6 1.1-1.3 2.1-2 3.2-.5 .6-.8 .8-1 .5zm66.3-.2c-.8-.9-2.8-4.4-3.5-6.1-.6-1.3-.9-2.5-1.1-3.5-.2-2.1-.7-4.1-1.5-6 0-.3 0-.3 1.2-.3l2.1 .5 1.9 .4 1.2 .4 .6 .1 1 .6c3 1.4 5.7 4.6 6.8 8.5l.7 2.6c-.2 .6-.5 .5-1.4-.7-2.2-2.7-4.8-5-7.7-6.9l-1.7-1.3 .6 1.3c.3 .6 .6 1.2 .8 1.9l.3 2.5 .3 3.9c.3 2.4 .2 2.8-.6};
    \fill svg{M21.6-66.1l.4-1.1 .9-3.2c.3-1.9 1.1-3.3 2.4-4.7l.4-.8-1.2 .2-2.2 .3c-2.7 .3-5.3 1.2-7.7 2.5-.6 .5-1.3 .6-1.3 .3 0-.5 .9-1.9 2-2.9 .8-.9 2-1.9 3.2-2.6l.9-.4 2.2-1c.3-.2 1.3-.3 3.2-.1 3 0 4.1 .2 6.3 .7l1.1 .4c.5 .2 .6 .6 .3 .6-.5 0-1.4 .9-1.9 1.7l-1.2 1.8c-1.7 2.8-2.2 3.5-4.6 5.9l-3 2.7-.2-.3zm53.9-2c-2.7-2.8-3.5-3.8-5.4-6.8-.9-1.6-1.4-2.4-1.9-2.5l-.8-.5c-.3 0-.2-.5 .4-.6l1.1-.4c1.9-.6 3-.8 5.6-.9l3.3 .2c2 .6 3.8 1.5 5.4 2.8 .3 0 1.9 1.6 2.5 2.4l.9 1.8c0 .3-.3 .2-1.9-.6-2.8-1.4-4.4-1.9-7.7-2.2l-2.2-.5c-.9-.2-.9-.2-.6 .2 .6 .5 1.7 2 2.1 2.8l.9 2.5c.3 1.5 .6 3 .9 4.6l-2.6-2.3z};
    \fill svg{M34.1-78.7c-3.4-1.3-6.9-2.1-10.6-2.5-.9 0-1.4 0-2.3 .3-2 .5-2 0 0-1.3l2.8-1.2c1.4-.5 1.9-.5 3.8-.6 3.8-.2 6.1 .3 9.3 1.7l3.6 1.1 2.2 .3c1.3 0 1.7 0 2.7-.3 1.1-.3 2.8-1.1 2.8-1.3l-1.3-.9c-1.9-1.4-3.1-2.7-3.1-3.2l.8-.6c.9-.3 1.3-.2 2 .8 .5 .8 1.1 1.4 2.9 2.7 .2 .3 .3 .2 1.1-.3 .9-.8 2.4-2 2.6-2.7 .5-.6 .9-.8 1.8-.5l.8 .6c0 .5-1.4 1.7-3.2 3.2l-1.3 .9c0 .2 1.7 .9 2.9 1.3 .9 .3 1.4 .3 2.7 .3l2.2-.3c1.7-.4 3.4-1 5-1.7 2-.8 4.4-1.3 7.7-1.1 2 .2 2.5 .2 3.8 .6 .9 .3 2.2 .8 2.8 1.2 2 1.1 2 1.6 .2 1.3-1.6-.3-1.9-.3-4.4 0-2.4 .3-4.7 .8-7 1.6l-1.5 .6c-2.9 .3-5.9 .2-8.8-.3-1.7-.3-3.6-.9-6-2.1l-1.1-.4-1.3 .6c-4.5 2.2-9.6 3-14.6 2.2zm-6.3-9.1c};
  }
}

\newcommand{\pointthis}[3]{%
  \tikz[remember picture,baseline]{
    \node[anchor=base,inner sep=0,outer sep=0] (#2) {#2};
    \node[visible on=#1,overlay,rectangle callout,rounded corners,callout relative pointer={(0.3cm,0.5cm)},fill=blue!20] at ($(#2.north)+(-0.1cm,-1.1cm)$) {#3};
  }%
}

\tikzset{
  invisible/.style={opacity=0,text opacity=0},
  visible on/.style={alt={#1{}{invisible}}},
  alt/.code args={<#1>#2#3}{%
    \alt<#1>{\pgfkeysalso{#2}}{\pgfkeysalso{#3}}}
}

\newcommand{\hcancel}[5]{%
  \tikz[baseline=(tocancel.base)]{
    \node[inner sep=0pt,outer sep=0pt] (tocancel) {#1};
    \draw[red!80, line width=0.4mm] ($(tocancel.south west)+(#2,#3)$) -- ($(tocancel.north east)+(#4,#5)$);
  }%
}

\newcommand{\explain}[7]{%
  \tikz[remember picture,baseline]{
    \node[anchor=base,inner sep=2pt,outer sep=0,fill=#3,rounded corners] (label) {#1};
    \node[anchor=north,visible on=<#2>,overlay,rectangle callout,rounded corners,callout
    relative pointer={(0.0cm,0.5cm)+(0.0cm,#6)},fill=#3] at ($(label.south)+(0,-0.3cm)+(#4,#5)$) {#7};
  }%
}

\newcommand{\explainstub}[2]{%
  \tikz[remember picture,baseline]{
    \node[anchor=base,inner sep=2pt,outer sep=0,fill=#2,rounded corners] (label) {#1};
  }%
}

\newcommand{\squiggly}[1]{%
  \tikz[remember picture,baseline]{
    \node[anchor=base,inner sep=0,outer sep=0] (label) {#1};
    \draw[thick,color=red!80,decoration={snake,amplitude=0.5pt,segment
    length=3pt},decorate] ($(label.south west) + (0,-2pt)$) -- ($(label.south east) + (0,-2pt)$);
  }%
}

% Adapted from https://latex.org/forum/viewtopic.php?t=2251 (Stefan Kottwitz)
\newenvironment<>{varblock}[2]{\begin{varblockextra}{#1}{#2}{}}{\end{varblockextra}}
\newenvironment<>{varblockextra}[3]{
  \begin{center}
    \begin{minipage}{#1}
      \begin{actionenv}#4
        {\centering \hil{#2}\par}
	\def\insertblocktitle{}%\centering #2}
        \def\varblockextraend{#3}
	\usebeamertemplate{block begin}}{
        \par
        \usebeamertemplate{block end}
        \varblockextraend
      \end{actionenv}
    \end{minipage}
  \end{center}}

\setbeamertemplate{headline}{}

\setbeamertemplate{frametitle}{%
  \leavevmode%
  \vskip-1.6em%
  \begin{beamercolorbox}[dp=1ex,center,wd=\paperwidth,ht=2.25ex]{title}%
    \vskip0.5em%
    \bf\insertframetitle
  \end{beamercolorbox}%

  \vskip-0.77em\hspace*{-2em}%
  \textcolor{mypurpledark}{\rule[0em]{1.1\paperwidth}{2.4pt}}

  \vskip-0.4em%
}

\setbeamertemplate{navigation symbols}{}

\newcounter{framenumberpreappendix}
\newcommand{\backupstart}{
  \setcounter{framenumberpreappendix}{\value{framenumber}}
}
\newcommand{\backupend}{
  \addtocounter{framenumberpreappendix}{-\value{framenumber}}
  \addtocounter{framenumber}{\value{framenumberpreappendix}}
}

\newcommand{\insertframeextra}{}
\setbeamertemplate{footline}{%
  \begin{beamercolorbox}[wd=\paperwidth,ht=2.25ex,dp=1ex,right,rightskip=1mm,leftskip=1mm]{}%
    % \inserttitle
    \hfill
    \insertframenumber\insertframeextra\,/\,\inserttotalframenumber
  \end{beamercolorbox}%
  \vskip0pt%
}

\newcommand{\hil}[1]{{\usebeamercolor[fg]{item}{\textbf{#1}}}}
\newcommand{\hill}[1]{{\usebeamercolor[fg]{item}{#1}}}
\newcommand{\bad}[1]{\textcolor{red!90}{\textnormal{#1}}}
\newcommand{\good}[1]{\textcolor{mypurple}{\textnormal{#1}}}

\newcommand{\bignumber}[1]{%
  \renewcommand{\insertenumlabel}{#1}\scalebox{1.2}{\!\usebeamertemplate{enumerate item}\!}
}
\newcommand{\normalnumber}[1]{%
  {\renewcommand{\insertenumlabel}{#1}\!\usebeamertemplate{enumerate item}\!}
}
\newcommand{\bigheart}{\includegraphics{heart}}

\newcommand{\subhead}[1]{{\centering\textcolor{gray}{\hrulefill}\quad\textnormal{#1}\quad\textcolor{gray}{\hrulefill}\par}}

\newcommand{\badbox}[1]{\colorbox{red!30}{#1}}
\newcommand{\infobox}[1]{\colorbox{yellow!70}{\color{black}#1}}

% taken from JDH "The modal logic of arithmetic potentialism and the universal algorithm"
\DeclareMathOperator{\possible}{\text{\tikz[scale=.6ex/1cm,baseline=-.6ex,rotate=45,line width=.1ex]{\draw (-1,-1) rectangle (1,1);}}}
\DeclareMathOperator{\necessary}{\text{\tikz[scale=.6ex/1cm,baseline=-.6ex,line width=.1ex]{\draw (-1,-1) rectangle (1,1);}}}
\DeclareMathOperator{\xpossible}{\text{\tikz[scale=.6ex/1cm,baseline=-.6ex,rotate=45,line width=.1ex]{\draw (-1,-1) rectangle (1,1); \draw[very thin] (-.6,-.6) rectangle (.6,.6);}}}
\DeclareMathOperator{\xnecessary}{\text{\tikz[scale=.6ex/1cm,baseline=-.6ex,line width=.1ex]{\draw (-1,-1) rectangle (1,1); \draw[very thin] (-.6,-.6) rectangle (.6,.6);}}}

% Taken from Todd Lehman (CC-BY-SA) at https://tex.stackexchange.com/a/44920/32372

\newcommand{\setisprime}[1]{
  % Sets \isprime based on #1.
  \ifnum#1=1 \gdef\isprime{0} \else \gdef\isprime{1} \fi
  \foreach \sip in {2, 3,5,...,#1} {
    \pgfmathparse{\sip*\sip>#1? 1:0}
    \ifthenelse{\pgfmathresult=1}{
      % Early-out if \sip^2 > #1.
      \breakforeach
    }{
      % Otherwise test if \sip divides #1.
      \pgfmathparse{Mod(#1,\sip)==0? 1:0}
      \ifthenelse{\pgfmathresult=1}{
        \gdef\isprime{0}
        \breakforeach
      }{}
    }
  }
}

\newcommand{\setxy}[1]{
  % Sets \x and \y to loction of cell #1.
  \pgfmathtruncatemacro{\x}{Mod(#1-1,\cols)}
  \pgfmathtruncatemacro{\y}{(#1-1) / \cols}
  \pgfmathtruncatemacro{\y}{\cols - 1 - \y}
  \pgfmathparse{2.5*(\x+.5)}\let\x\pgfmathresult
  \pgfmathparse{2.5*(\y+.5)}\let\y\pgfmathresult
}

\newcommand{\numlabel}[2]{
  % Draws label #2 at cell #1.
  \setxy{\n}
  \node[fill=none, text=black] at (\x,\y) {#2};
}

\newcommand{\drawpolygon}[2]{
  % Draws polygon with #2 vertexes at cell #1.
  \setxy{#1}
  \ifthenelse{#2>1}{ % Polygon must have at least 2 sides.
    \ifthenelse{#2<30}{ % Draw polygon if it has a small number of sides.
      \filldraw (\x,\y) +(90:1)
      \foreach \drawi in {1,...,#2} {-- +(\drawi/#2*360+90:1)} -- cycle;
    }{ % Else approximate with circle.
      \filldraw (\x,\y) circle(1);
    }
  }{}
}

\newcommand{\setpolygoncolor}[1]{
  % Sets color based on #1.
  \gdef\polycolor{black}
  \ifnum#1=2\gdef\polycolor{black!50!white}\fi
  \ifnum#1=3\gdef\polycolor{yellow!95!red}\fi
  \ifnum#1=5\gdef\polycolor{yellow!0!red}\fi
  \ifnum#1=7\gdef\polycolor{blue!75!green}\fi
  \ifnum#1=11\gdef\polycolor{blue!70!red}\fi
  \ifnum#1=13\gdef\polycolor{blue!40!red}\fi
  \ifnum#1=17\gdef\polycolor{green!50!blue}\fi
  \ifnum#1=19\gdef\polycolor{green!80!black}\fi
  \ifnum#1=23\gdef\polycolor{green!50!red}\fi
  \ifnum#1=29\gdef\polycolor{yellow!50!black}\fi
  \ifnum#1=31\gdef\polycolor{orange!50!black}\fi
  \ifnum#1=37\gdef\polycolor{red!50!black}\fi
  \ifnum#1=41\gdef\polycolor{purple!50!black}\fi
  \ifnum#1=43\gdef\polycolor{blue!50!black}\fi
  \ifnum#1=47\gdef\polycolor{green!50!black}\fi
  \ifnum#1=53\gdef\polycolor{white!50!black}\fi
  \ifnum#1=59\gdef\polycolor{white!50!black}\fi
  \ifnum#1=61\gdef\polycolor{white!50!black}\fi
  \ifnum#1=67\gdef\polycolor{white!50!black}\fi
}

\newcommand{\sieve}[2]{
  \def\cols{#1}
  \def\rows{#2}
  \begin{tikzpicture}[scale=.5]
  \pgfmathtruncatemacro{\nmax}{\rows * \cols}

  \foreach \n in {1,...,\nmax} {
    \begin{scope}[fill=gray, fill opacity=.05,
                  draw=gray, draw opacity=.10,
                  line width=4]
      \drawpolygon{\n}{\n}
    \end{scope}
    \setisprime{\n}
    \ifthenelse{\isprime=1}{
      \numlabel{\n}{\bf\n}
    }{
      \def\startintensity{.33}
      \def\incrintensity{.10}
      \def\intensity{\startintensity}

      \def\m{\n}
      \pgfmathtruncatemacro{\i}{\m / 2}

      % Divide \m by \i until \m is extinguished.
      % Increment \i each time it does not divide into \m.
      \whiledo{\m>1}{
        \setisprime{\i}
        \pgfmathparse{Mod(\m,\i)==0? 1:0}
        \ifthenelse{\pgfmathresult=1\and\isprime=1}{
          \setpolygoncolor{\i}
          \begin{scope}[fill=\polycolor, fill opacity=\intensity,
                        draw=\polycolor!85!black, draw opacity=\intensity,
                        line width=\intensity*1.5]
            \drawpolygon{\n}{\i}
          \end{scope}
          \pgfmathtruncatemacro{\m}{\m / \i}
          \pgfmathparse{\intensity + \incrintensity}\let\intensity\pgfmathresult
        }{
          \pgfmathtruncatemacro{\i}{\i - 1}
          \def\intensity{\startintensity}
        }
      }
      \begin{scope}[text=black, text opacity=.5]
        \numlabel{\n}{\scriptsize\n}
      \end{scope}
    }
  }

  \end{tikzpicture}
}

\newcommand{\fakesieve}[2]{
  \def\cols{#1}
  \def\rows{#2}
  \begin{tikzpicture}[scale=.5,opacity=0]
  \pgfmathtruncatemacro{\nmax}{\rows * \cols}

  \foreach \n in {1,...,\nmax} {
    \begin{scope}[fill=gray,
                  draw=gray,
                  line width=4]
      \drawpolygon{\n}{\n}
    \end{scope}
    \setisprime{\n}
    \ifthenelse{\isprime=1}{
      \numlabel{\n}{\bf\n}
    }{
      \def\startintensity{.33}
      \def\incrintensity{.10}
      \def\intensity{\startintensity}

      \def\m{\n}
      \pgfmathtruncatemacro{\i}{\m / 2}

      % Divide \m by \i until \m is extinguished.
      % Increment \i each time it does not divide into \m.
      \whiledo{\m>1}{
        \setisprime{\i}
        \pgfmathparse{Mod(\m,\i)==0? 1:0}
        \ifthenelse{\pgfmathresult=1\and\isprime=1}{
          \setpolygoncolor{\i}
          \begin{scope}[fill=\polycolor,
                        draw=\polycolor!85!black,
                        line width=\intensity*1.5]
            \drawpolygon{\n}{\i}
          \end{scope}
          \pgfmathtruncatemacro{\m}{\m / \i}
          \pgfmathparse{\intensity + \incrintensity}\let\intensity\pgfmathresult
        }{
          \pgfmathtruncatemacro{\i}{\i - 1}
          \def\intensity{\startintensity}
        }
      }
      \begin{scope}[text=black]
        \numlabel{\n}{\scriptsize\n}
      \end{scope}
    }
  }

  \end{tikzpicture}
}


\newcommand{\triang}{\hil{$\blacktriangleright$}}
\newcommand{\concat}{\mathbin{{+}\mspace{-8mu}{+}}}

\newcommand{\astikznode}[2]{\tikz[baseline,remember picture]{\node[anchor=base,inner sep=0,outer sep=0.1em] (#1) {#2};}}
\newcommand{\astikznodecircled}[3]{\tikz[baseline,remember picture]{\node[anchor=base,circle,draw=#2,thick,inner sep=0,outer sep=0.05em] (#1) {#3};}}
\newcommand{\astikznodetransparentlycircled}[2]{\tikz[baseline,remember picture]{\node[anchor=base,circle,opacity=0,draw=white,text opacity=1,thick,inner sep=0,outer sep=0.05em] (#1) {#2};}}

\setbeamersize{text margin left=1.60em,text margin right=1.60em}

\newlength\stextwidth
\newcommand\makesamewidth[3][c]{%
  \settowidth{\stextwidth}{#2}%
  \makebox[\stextwidth][#1]{#3}%
}

\begin{document}

\addtocounter{framenumber}{-1}

{\usebackgroundtemplate{\begin{minipage}{\paperwidth}\includegraphics[width=\paperwidth]{swansea-bay}\end{minipage}}
\begin{frame}[c]
  \centering

  \bigskip
  \includegraphics[height=0.32\textwidth]{phantoms}
  \bigskip
  \bigskip
  \bigskip
  \color{white}

  \begin{tikzpicture}
    \def\R{8pt}
    \node (title) {\phantom{qquad}\textcolor{white}{Maximal ideals in commutative algebra as convenient fictions}\phantom{qquad}};
    \begin{pgfonlayer}{background}
      \draw[decorate, very thick, draw=white]
        ($(title.south west) + (\R, 0)$) arc(270:180:\R) --
        ($(title.north west) + (0, -\R)$) arc(180:90:\R) --
        ($(title.north east) + (-\R, 0)$) arc(90:0:\R) --
        ($(title.south east) + (0, \R)$) arc(0:-90:\R) --
        cycle;
    \end{pgfonlayer}
  \end{tikzpicture}

  \scriptsize
  \textit{-- an invitation --}

  \bigskip
  %(Agda formalization available)
  %\bigskip

  Bonn Constructive Algebra Seminar \\
  July 22th, 2024
  \ \\
  \bigskip
  \bigskip
  \bigskip

  \begin{columns}
    \begin{column}{0.4\textwidth}
      \centering
      Ingo Blechschmidt \\
      University of Augsburg
    \end{column}
    \begin{column}{0.4\textwidth}
      \centering
      Peter Schuster \\
      University of Verona
    \end{column}
  \end{columns}
  \par
\end{frame}}

\definecolor{mypurple}{RGB}{150,0,255}
\setbeamercolor{structure}{fg=mypurple}

\begin{frame}
  \begin{center}\includegraphics[width=0.2\textwidth]{eigenvector}\end{center}

  Let a continuous family of symmetric matrices be given:
  \[
  \begin{pmatrix}a_{11}(t)&\cdots&a_{1n}(t)\\\vdots&&\vdots\\a_{n1}(t)&\cdots&a_{nn}(t)\end{pmatrix}
  \]

  Then for every parameter value~$t \in \Omega$, classically there is

  \hil{$\blacktriangleright$} a full list of eigenvalues~$\lambda_1(t),\ldots,\lambda_n(t)$ and \\
  \hil{$\blacktriangleright$} an eigenvector basis~$(v_1(t),\ldots,v_n(t))$.
  \bigskip

  \begin{columns}[c]
    \begin{column}{0.01\textwidth}
      \includegraphics[height=2.4em]{question-mark}
    \end{column}
    \begin{column}{0.9\textwidth}
      \mbox{Can locally the functions~$\lambda_i$ be chosen to be continuous?
      \only<2->{\hil{Yes.}}} \\
      How about the~$v_i$? \only<2->{\hil{No}\only<3->{, but \hil{yes} on a dense
      open subset of~$\Omega$.}}
    \end{column}
  \end{columns}
\end{frame}

{\usebackgroundtemplate{\begin{minipage}{\paperwidth}\vspace*{5.95cm}\includegraphics[width=\paperwidth]{fr1}\end{minipage}}
\begin{frame}{Maximal ideals}
  \only<1-7>{\textbf{Thm.}
  Let~$M$ be a surjective matrix with more rows than columns over a
  ring~$A$. Then~$1 = 0$ in~$A$.

  \visible<2->{\emph{Proof.} \bad{Assume not.}}
  \visible<3->{Then there is~a \bad{maximal ideal} $\mmm$.}
  \visible<5->{The matrix is surjective over~$A/\mmm$.}
  \visible<6->{Since~$A/\mmm$ is a field, this is a contradiction to basic linear algebra.\qed}}

  \only<4-7>{\bigskip\par\centering\scalebox{0.9}{\centering\begin{tikzpicture}
    \node (0) at (0,1) {$(0) = \{0\}$};
    \node (1) at (0,5) {$(1) = \ZZ$};
    \node (2) at (-2,4) {$(2)$};
    \node [right of=2] (3) {$(3)$};
    \node [below of=2] (4) {$(4)$};
    \node [below of=2, xshift=0.7cm] (6) {$(6)$};
    \node [right of=3] (5) {$(5)$};
    \node [right of=5] (7) {$(7)$};
    \node [right of=7] (7d) {$\ldots$\phantom{(}};
    \node [right of=7d, xshift=3cm, yshift=-2cm] (max)
    {\vbox{\small{\it maximal among the proper ideals} \\ \medskip \hspace*{-6.75em}\textbullet \quad $\neg(1 \in
    \mmm)$ \\ \medskip \textbullet \quad $\neg\bigl(1 \in \mmm + (x)\bigr) \Rightarrow x \in \mmm$}};
    \node [below of=4] (8) {$(8)$};
    \node [right of=8, xshift=3cm] (8d) {$\ldots$};
    \draw (0) -- (8);
    \draw (0) -- (8d);
    \draw (0) -- (6);
    \draw (2) -- (1);
    \draw (3) -- (1);
    \draw (5) -- (1);
    \draw (7) -- (1);
    \draw (7d) -- (1);
    \draw (4) -- (2);
    \draw (8) -- (4);
    \draw (6) -- (2);
    \draw (6) -- (3);
    \draw [mypurple!30, thick, shorten <=-2pt, shorten >=-2pt, ->] (max) to [out=120, in=-30] (7d);
    \begin{pgfonlayer}{background}
      \draw[decorate, very thick, draw=mypurple!30]
        ($(2.south west) + (8pt, 0)$) arc(270:180:8pt) --
        ($(2.north west) + (0, -8pt)$) arc(180:90:8pt) --
        ($(7d.north east) + (-8pt, 0)$) arc(90:0:8pt) --
        ($(7d.south east) + (0, 8pt)$) arc(0:-90:8pt) --
        cycle;
    \end{pgfonlayer}
  \end{tikzpicture}\par}\par}

  \pause
  \pause
  \pause
  \pause
  \pause
  \pause
  \medskip
  \raggedright

  Let~$A$ be a ring. \emph{Does there exist a maximal ideal~$\mmm \subseteq A$?}
  \pause
  \begin{enumerate}
    \item \good{Yes}, if \bad{Zorn's lemma} is available.
    \bigskip
    \pause

    \item \good{Yes}, if~$A$ is countable and membership of finitely generated ideals is decidable:
    {\footnotesize
    Let~$A = \{ x_0, x_1, \ldots \}$. Then set:
    \begin{align*}
      \mmm_0 &\defeq \{ 0 \}, &
      \mmm_{n+1} &\defeq \begin{cases}
        \mmm_n + (x_n), & \text{if $1 \not\in \mmm_n + (x_n)$}, {\qquad\quad\quad\!\!}\\
        \mmm_n, & \text{else.}
      \end{cases}
    \end{align*}}
    \pause

    \item \good{Yes}, if~$A$ is countable (irrespective of membership decidability):
    {\footnotesize\begin{align*}
      \mmm_0 &\defeq \{ 0 \}, &
      \mmm_{n+1} &\defeq \mmm_n + (\underbrace{\{ x \in A \,|\, x = x_n \wedge
        1 \not\in \mmm_n + (x_n) \}}_{\text{a certain subsingleton set}})
    \end{align*}}
    \vspace*{3.5em}
    \visible<11>{{
      \centering
      \raisebox{0pt}[0pt][0pt]{
        \hspace*{2.5em}
        \scalebox{0.9}{\begin{tikzpicture}
          \node (inner) at (17.3mm,-20mm) {\textit{``a bad joke''}};
          \path (0,0) pic{laurel-wreath};
        \end{tikzpicture}}
        \hspace*{-3em}
        \scalebox{0.9}{\begin{tikzpicture}
          \node (inner) at (17.5mm,-18mm) {\vbox{\small\centering\textit{``non- \\informative''}}};
          \path (0,0) pic{laurel-wreath};
        \end{tikzpicture}}
      }
      \par
    }}
    \pause
    \pause
    \vspace*{-3.5em}

    \item In the general case: \bad{No}\pause, but \good{yes} in a \emph{suitable forcing extension}\pause, and \\
    \emph{bounded first-order consequences} of its existence there \good{do hold} in
    the base universe.
  \end{enumerate}
\end{frame}

\begin{frame}{Maximal ideals}
  \textbf{Thm.}
  Let~$M$ be a surjective matrix with more rows than columns over a
  ring~$A$. Then~$1 = 0$ in~$A$.

  \emph{Proof (classical).} \bad{Assume not.}
  Then there is~a \bad{maximal ideal} $\mmm$.
  The matrix is surjective over~$A/\mmm$.
  Since~$A/\mmm$ is a field, this is a contradiction to basic linear algebra.\qed
  \pause

  \emph{Proof (constructive, special case).} Write~$M =
  \left(\begin{smallmatrix}x\\y\end{smallmatrix}\right)$. By surjectivity,
  we have~$u, v \in A$ with
  \[
    u \left(\begin{smallmatrix}x\\y\end{smallmatrix}\right) = \left(\begin{smallmatrix}1\\0\end{smallmatrix}\right)
    \quad\text{and}\quad
    v \left(\begin{smallmatrix}x\\y\end{smallmatrix}\right) = \left(\begin{smallmatrix}0\\1\end{smallmatrix}\right).
  \]
  Hence
  $
    1 = (vy) (ux) = (uy) (vx) = 0
  $. \qed

  \bigskip
  \centering
  \colorbox{white!20}{\emph{Abstract proofs should be blueprints for concrete ones.}}
\end{frame}}

{\usebackgroundtemplate{\begin{minipage}{\paperwidth}\includegraphics[height=\paperheight]{sea-of-clouds-2}\end{minipage}}
\begin{frame}{Noetherian conditions}
  \begin{itemize}
    \item ``Every ideal is finitely generated.''
    \medskip\pause

    ---\emph{What about $\{ x \in \ZZ \,|\, x = 0 \vee \varphi \} \subseteq \ZZ$?}
    \pause
    \bigskip

    \item ``Every ascending chain of finitely generated ideals stabilizes,
    i.\@e. given~$\aaa_0 \subseteq \aaa_1 \subseteq \ldots$, there is a
    number~$n$ such that~$\aaa_n = \aaa_{n+1} = \aaa_{n+2} = \ldots$.''
    \medskip\pause

    ---\emph{Even descending sequences of natural numbers might fail to
    stabilize.}
    \pause\bigskip

    \item ``Every ascending chain of finitely generated ideals stalls,
    i.\@e. given~$\aaa_0 \subseteq \aaa_1 \subseteq \ldots$, there is a
    number~$n$ such that~$\aaa_n = \aaa_{n+1}$.''
    \medskip\pause

    ---\emph{There might not be enough such chains.}
    \pause\bigskip

    \item ``Every infinite sequence of ring elements is good,
    i.\@e. given~$x_0, x_1, \ldots$, there is a number~$n$ such that~$x_n \in
    (x_0,\ldots,x_{n-1})$.''
    \medskip\pause

    ---\emph{There might not be enough such sequences.}
  \end{itemize}
\end{frame}}

{\usebackgroundtemplate{\begin{minipage}{\paperwidth}\includegraphics[height=\paperheight]{sea-of-clouds-2}\end{minipage}}
\begin{frame}{Infinite data}
  \vspace*{-1em}
  \[ \astikznodetransparentlycircled{xm}{7}\!,
    \quad \astikznodetransparentlycircled{x0}{4}\!,
    \quad \only<1-2>{\astikznodetransparentlycircled{t1}{3}}\only<3->{\astikznodecircled{t1}{mypurple}{3}}\!,
    \quad \only<1>{\ldots}\pause \astikznodetransparentlycircled{x1}{1}\!,
    \quad \only<2>{\ldots}\pause \astikznodecircled{t2}{mypurple}{8}\!,
    \quad \only<3>{\ldots} \visible<4->{\astikznodetransparentlycircled{x2}{2}\!,}
    \quad \only<4->{\ldots} \]
  {\centering\begin{tikzpicture}[remember picture,overlay]
    \node[draw=mypurple, circle, thick, inner sep=0.1em] (t3) {\scriptsize$\leq$};
    \path[draw=mypurple,thick]
      (t1)
      to [out=-90, in=180] (t3)
      to [out=0, in=-90] (t2);
  \end{tikzpicture}\par}
  \medskip
  \pause

  \textbf{Thm.} Every sequence~$\alpha : \NN \to \NN$ is \hil{good} in that
  there exist~$i < j$ with~$\alpha(i) \leq \alpha(j)$.
  \pause

  \emph{Proof.} \emph{(offensive?)} By~\badbox{\textsc{lem}}, there is a
  minimum~$\alpha(i)$.
  Set~$j \defeq i + 1$. \qed\par
  \pause
  \medskip

  \textbf{Def.} A preorder~$X$ is \hil{well\only<9->{$^\star$}} iff every sequence~$\NN \to X$ is good.

  \textbf{Examples.} $(\NN,{\leq}),\ \
  \color{white}\only<7->{\color{red!90}}\astikznode{onlyclass}{$\underbrace{\color{black}X \times Y,\ \ X^*,\ \ \mathrm{Tree}(X)}_{\text{\visible<7->{\bad{only classically}}}}$}$.
  \pause
  \pause
  \medskip

  \begin{tikzpicture}[remember picture,overlay]
    \node[thick, fill=black, rectangle, inner sep=0.3em, right=2em of onlyclass] (moral) {
      \begin{minipage}{6cm}
        \begin{columns}
          \begin{column}{0.15\textwidth}
            \hspace*{1.0em}\color{white}\dbend
          \end{column}
          \begin{column}{0.95\textwidth}
            \color{white}\footnotesize
            \it Don't quantify over points of spaces which might not have enough.
          \end{column}
        \end{columns}
      \end{minipage}
    };
    \path[draw=red!90,thick,-stealth]
      (moral) to
      [out=180, in=-00] (onlyclass.east);
  \end{tikzpicture}
  \pause

  \textbf{Def.} A preorder is \hil{well} iff any of the following equivalent
  conditions hold:
  \begin{enumerate}
    \item The \hil{generic sequence} $\NN \to X$ is good.
    \pause
    \item Every sequence $\NN \to X$ \hil{in every forcing extension} is good.
    \pause
    \item There is a \hil{well-founded tree} witnessing universal goodness.
  \end{enumerate}

%  $\mathsf{Good}\,[\sigma_1,\ldots,\sigma_n] \defeqv (\exists(i < j)\_ \sigma_i \leq \sigma_j)$.
%  \textbf{Def.} For a predicate~$P$ on finite lists over a set~$X$, inductively
%  define:
%  \[
%    \infer{P \,|\, \sigma}{P\sigma}
%    \qquad
%    \infer{P \,|\, \sigma}{\forall(x \in X)\_\ P \,|\, \sigma x}
%  \]
%
%  \textbf{Def.} A preorder is \hil{well} iff~$\mathsf{Good} \,|\, [\,]$, where
%  $\mathsf{Good}\,[\sigma_1,\ldots,\sigma_n] \defeqv (\exists(i < j)\_ \sigma_i \leq \sigma_j)$.
\end{frame}}


\section{Basics of forcing}

\begin{frame}{Ingredients for forcing}
  To construct a forcing extension, we require:
  \begin{enumerate}
    \item a base universe~$V$
    \item a preorder~$L$ of \hil{forcing conditions} in~$V$\!,
    pictured as \hil{finite approximations}

    (\emph{convention:} $\tau \preccurlyeq \sigma$ means that~$\tau$ is a
    better finite approximation than~$\sigma$)
    \item a \hil{covering system} governing how finite approximations evolve to
    better ones

    (for each~$\sigma \in L$, a set~$\Cov(\sigma) \subseteq
    P({\downarrow}\sigma)$, with a simulation condition)
  \end{enumerate}
  In the forcing extension~$V^\nabla$, there will then be a \hil{generic filter} (ideal
  object).
  \pause

  \vspace*{-1em}
  \begin{columns}
    \begin{column}{0.49\textwidth}
      \small
      \begin{block}{For the generic surjection~$\NN \twoheadrightarrow X$}
        \justifying
        Use \hil{finite lists}~$\sigma \in X^*$ as forcing conditions,
        where $\tau \preccurlyeq \sigma$ iff~$\sigma$ is an initial segment of~$\tau$,
        and be prepared to grow~$\sigma$ to \ldots
        \footnotesize
        \begin{enumerate}
          \item[(a)] one of~$\{ \sigma x \,|\, x \in X \}$, to make~$\sigma$ more defined
          \item[(b)] one of~$\{ \sigma \tau \,|\, \tau \in X^*, a \in
          \sigma\tau \}$, for any~$a \in X$, to make~$\sigma$ more surjective
        \end{enumerate}
      \end{block}
    \end{column}
    \pause

    \begin{column}{0.45\textwidth}
      \small
      \begin{block}{For the generic prime ideal of a ring~$A$}
        \justifying
        Use \hil{f.g.\@ ideals} as forcing conditions, where $\bbb \preccurlyeq
        \aaa$ iff~$\bbb \supseteq \aaa$, and be prepared to grow~$\aaa$ to \ldots
        \footnotesize
        \begin{enumerate}
          \item[(a)] one of~$\emptyset$, if~$1 \in \aaa$, to make~$\aaa$ more proper
          \item[(b)] one of~$\{ \aaa+(x), \aaa+(y) \}$, if~$xy \in \aaa$, to
          make~$\aaa$ more prime
        \end{enumerate}
      \end{block}
    \end{column}
  \end{columns}
\end{frame}

{\usebackgroundtemplate{\begin{minipage}{\paperwidth}\vspace*{-1cm}\includegraphics[width=\paperwidth]{forest-light-colored}\end{minipage}}
\begin{frame}{The eventually monad}
  Let~$L$ be a forcing notion.

  Let~$P$ be a monotone predicate on~$L$
  (if $\tau \preccurlyeq \sigma$, then $P\sigma \Rightarrow P\tau$). \\
  For instance, in the case~$L = X^*$:
  \begin{itemize}
    \item $\mathsf{Repeats}\, x_0\ldots x_{n-1} \defeqv \exists i\_ \exists j\_ i < j \wedge x_i = x_j$
    \item $\mathsf{Good}\, \,\,\,\,\,\,x_0\ldots x_{n-1} \defeqv \exists i\_ \exists j\_ i < j \wedge x_i \leq x_j$
    \quad (for some preorder~$\leq$ on~$X$)
  \end{itemize}
  \pause

  We then define~``\hil{$P \mid \sigma$}'' (``$P$ bars~$\sigma$'') inductively by the following clauses:
  \begin{enumerate}
    \item If~$P\sigma$, then~$P \mid \sigma$.
    \item If~$P \mid \tau$ for all~$\tau \in R$, where~$R$ is some covering
    of~$\sigma$, then~$P \mid \sigma$.
  \end{enumerate}
  So~$P \mid \sigma$ expresses in a \hil{direct inductive fashion}:
  \[ \text{``No matter
  how~$\sigma$ evolves to a better approximation~$\tau$, eventually~$P\tau$
  will hold.''} \]

  \pause
  We use quantifier-like notation: ``$\nabla(\tau \preccurlyeq \sigma)\_
  P\tau$'' means ``$P \mid \sigma$''.
\end{frame}}
% BOARD:
% - examples for P | σ
% - abuse of notation

\begin{frame}{Proof translations}
  \textbf{Thm.} Every~\textsc{iqc}-proof remains correct, with at most a
  polynomial increase in length, if throughout we
  replace
  \[\begin{array}{rcl@{\quad\text{where}\quad}rcl}
    \exists & \leadsto & \exists^\mathrm{cl},
    & \exists^\mathrm{cl} &\defeqv& \neg\neg\exists, \\
    \vee & \leadsto & \vee^\mathrm{cl},
    & \alpha \vee^\mathrm{cl} \beta &\defeqv& \neg\neg(\alpha \vee \beta), \\
    = & \leadsto & =^\mathrm{cl},
    & s =^\mathrm{cl} t &\defeqv& \neg\neg(s = t).
  \end{array} \]
  \pause

  \begin{columns}[c]
    \begin{column}{0.01\textwidth}
      \includegraphics[height=2.4em]{sheafification-man-2}
    \end{column}
    \quad
    \begin{column}{0.9\textwidth}
      \hil{When we say:}\ \ some statement ``holds in~$V^{\neg\neg}$'', \\
      \makesamewidth[l]{\hil{When we say:}}{\hil{we mean:}}\ \ its translation holds in~$V$.
    \end{column}
  \end{columns}
  \bigskip

  Similarly for arbitrary forcing extensions~$V^\nabla$, ``just with~$\nabla$
  instead of~$\neg\neg$''.
  \bigskip

  \pause
  \textbf{Ex.} As~$\neg\neg(\varphi \vee \neg\varphi)$ is a theorem
  of~\textsc{iqc}, the law of excluded middle holds in~$V^{\neg\neg}$.
\end{frame}

\newcommand{\defeqvi}{\quad iff\quad}
\begin{frame}{The $\nabla$-translation}
  \small
  \only<1>{For bounded first-order formulas over the (large) first-order signature which has
  \begin{enumerate}
    \scriptsize
    \item one sort~$\underline{X}$ for each set~$X$ in the base universe,
    \\[-1.2em]
    \item one~$n$-ary function symbol~$\underline{f} : \underline{X_1} \times
    \cdots \times \underline{X_n} \to \underline{Y}$ for each map~$f : X_1 \times
    \cdots \times X_n \to Y$,
    \\[-1.2em]
    \item one~$n$-ary relation symbol~$\underline{R} \hookrightarrow
    \underline{X_1} \times \cdots \times \underline{X_n}$ for each relation~$R
    \subseteq X_1 \times \cdots \times X_n$, and
    \\[-1.2em]
    \item an additional unary relation symbol~$G \hookrightarrow \underline{L}$
    (for the \emph{generic filter} of~$L$),
  \end{enumerate}
  we recursively define:}
  \scriptsize
  \only<2->{\vspace*{-0.4em}}
  \begin{tabbing}
    \quad \= $\sigma \forces \forces \forall(x\?\underline{X})\_ \varphi$ \=
    \defeqvi $\textcolor{gray}{\forall(\tau \preccurlyeq \sigma)\_}\
    \forall(x_0 \in X)\_ \tau \forces \varphi[\underline{x_0}/x]$.\qquad\quad \=
    $\sigma \forces \exists(x\?\underline{X})\_ \varphi$
    \= $\sigma \forces \underline{R}(\underline{s_1},\ldots,\underline{s_n})$ \= \defeqvi $s = t$. \= \kill

    \> $\sigma \forces s = t$
    \> \defeqvi $\nabla \sigma\_ \llbracket s \rrbracket = \llbracket t \rrbracket$.
    \> $\sigma \forces \underline{R}(s_1,\ldots,s_n)$
    \> \defeqvi $\nabla\sigma\_ R(\llbracket s_1 \rrbracket,\ldots,\llbracket s_n \rrbracket)$. \\[0.3em]

    \> $\sigma \forces \varphi \Rightarrow \psi$
    \> \defeqvi $\textcolor{gray}{\forall(\tau \preccurlyeq \sigma)\_}\ (\tau \forces \varphi) \Rightarrow
    (\tau \forces \psi)$.
    \> $\sigma \forces G\tau$
    \> \defeqvi $\nabla\sigma\_ \sigma \preccurlyeq \llbracket\tau\rrbracket$. \\[0.3em]

    \> $\sigma \forces \top$ \> \defeqvi $\top$.
    \> $\sigma \forces \bot$ \> \defeqvi $\hil{$\nabla\sigma\_$}\ \bot$ \\[0.3em]

    \> $\sigma \forces \varphi \wedge \psi$
    \> \defeqvi $(\sigma \forces \varphi) \wedge (\sigma \forces \psi)$.
    \> $\sigma \forces \varphi \vee \psi$
    \> \defeqvi $\hil{$\nabla\sigma\_$}\ (\sigma \forces \varphi) \vee (\sigma \forces \psi)$. \\[0.3em]

    \> $\sigma \forces \forall(x\?\underline{X})\_ \varphi$
    \> \defeqvi $\textcolor{gray}{\forall(\tau \preccurlyeq \sigma)\_}\ \forall(x_0 \in X)\_ \tau \forces
    \varphi[\underline{x_0}/x]$.
    \> $\sigma \forces \exists(x\?\underline{X})\_ \varphi$
    \> \defeqvi $\hil{$\nabla\sigma\_$}\ \exists(x_0 \in X)\_ \sigma \forces \varphi[\underline{x_0}/x]$.
  \end{tabbing}
  \small
  \only<1>{Finally, we say that~$\varphi$ ``holds in~$V^\nabla$'' iff for all~$\sigma
  \in L$, $\sigma \forces \varphi$.}

  \footnotesize
  \begin{tabular}{@{}lp{0.27\textwidth}p{0.48\textwidth}@{}}
    \toprule
    forcing notion & statement about~$V^\nabla$ & external meaning \\
    \midrule
    surjection $\NN \twoheadrightarrow X$ &
    ``the gen.\@ surj.\@ is surjective'' &
    $\forall(\sigma{\in}X^*)\_ \forall(a{\in}X)\_ \nabla(\tau{\preccurlyeq}\sigma)\_ \exists(n{\in}\NN)\_ \tau[n] = a$. \\[0.4em]
    \pause

    map $\NN \to X$ &
    ``the gen.\@ sequence is good'' &
    $\mathsf{Good} \mid [\,]$. \\[0.4em]

    frame of opens &
    ``every complex number has a square root'' &
    For every open~$U \subseteq X$ and every cont.\@
    function $f : U \to \CC$, there is an open covering $U = \bigcup_i U_i$ such
    that for each index~$i$, there is a cont.\@ function $g : U_i \to \CC$
    such that~$g^2 = f$. \\[4.3em]

    big Zariski &
    ``$x \neq 0 \Rightarrow \text{$x$ inv.}$'' &
    If the only f.p.\@ $k$-algebra in which~$x = 0$ is the zero algebra,
    then~$x$ is invertible in~$k$.\\[1.5em]
    \pause

    little Zariski &
    ``every f.g. vector space does \emph{not not} have a basis'' &
    \makesamewidth[l]{}{\phantom{x}}\hil{Grothendieck's generic freeness lemma}
  \end{tabular}
\end{frame}

\begin{frame}{Outlook}
  \begin{block}{Passing to and from extensions}
    \justifying\small
    \textbf{Thm.} Let~$\varphi$ be a \hil{bounded first-order formula} not
    mentioning~$G$. In each of the following situations, we have that
    $\varphi$ holds in~$V^\nabla$ iff~$\varphi$
    holds in $V$:

    \vspace*{-0.5em}
    \begin{enumerate}
      \item $L$ and all coverings are inhabited (proximality). \\[-1em]
      \item $L$ contains a top element, every covering of the
      top element is inhabited, and~$\varphi$ is a coherent implication
      (positivity).
    \end{enumerate}
  \end{block}

  \vspace*{-1em}
  \begin{columns}
    \begin{column}{0.46\textwidth}
      \begin{block}{The mystery of nongeometric sequents}
        \justifying
        The \hil{generic ideal} of a ring is maximal:
        \vspace*{-1em}
        \[ (x \in \aaa \Rightarrow 1 \in \aaa) \Longrightarrow 1 \in \aaa + (x). \]

        The \hil{generic ring} is a field:
        \vspace*{-0.7em}
        \[ (x = 0 \Rightarrow 1 = 0) \Longrightarrow (\exists y\_ xy = 1). \]
      \end{block}

    \end{column}

    \begin{column}{0.50\textwidth}
      \begin{block}{Traveling the multiverse \ldots}
        \textsc{lem} is a \hil{switch} and \hil{holds positively};
        being countable is a \hil{button}.
        \medskip

        Every instance of \textsc{dc} \hil{holds proximally}.
        \medskip

        A geometric implication is provable iff it holds \hil{everywhere}.
      \end{block}
      \vspace*{-0.3em}
      \hfill\footnotesize\ldots{} upwards, but always keeping ties to the
      base.{\ }
    \end{column}
  \end{columns}
\end{frame}

\begin{frame}{Formalities}
  \small
  \textbf{Def.} A \hil{forcing notion} consists of
  a preorder~$L$ of \hil{forcing conditions}, and
  for every~$\sigma \in L$, a set~$\Cov(\sigma) \subseteq
  P({\downarrow}\sigma)$ of \hil{coverings} of~$\sigma$
  such that: If~$\tau \preccurlyeq \sigma$ and~$R \in \Cov(\sigma)$, there
  should be a covering~$S \in \Cov(\tau)$ such that~$S \subseteq
  {\downarrow}R$.
  \bigskip

  {\centering\footnotesize\begin{tabular}{llll}
    \toprule
    & preorder~$L$ & coverings of an element~$\sigma \in L$ & filters of~$L$ \\
    \midrule
    \normalnumber{1} & $X^*$ & $\{ \sigma x \,|\, x \in X \}$ & maps~$\NN \to X$ \\
    \normalnumber{2} & $X^*$ & $\{ \sigma x \,|\, x \in X \}$,\ \ $\{ \sigma\tau \,|\, \tau \in X^*, a \in \sigma\tau \}$ for each~$a \in X$ & surjections~$\NN \twoheadrightarrow X$ \\
    \normalnumber{3} & f.g. ideals & --- & ideals \\
    \normalnumber{4} & f.g. ideals & $\{ \sigma+(a), \sigma+(b) \}$ for each~$ab \in \sigma$,\ \ $\{\}$ if~$1 \in \sigma$ & prime ideals \\
    \normalnumber{5} & opens & $\mathcal{U}$ such that~$\sigma = \bigcup \mathcal{U}$ & points \\
    \normalnumber{6} & $\{\star\}$ & $\{ \star \,|\, \varphi \} \cup \{ \star \,|\, \neg\varphi \}$ &
    witnesses of~\textsc{lem}
    \\
    \bottomrule
  \end{tabular}\par}
  \bigskip

  \textbf{Def.} A \emph{filter} of a forcing notion~$(L,\mathrm{Cov})$
  is a subset~$F \subseteq L$ such that
  \vspace*{-0.4em}
  \begin{enumerate}
    \scriptsize
    \item $F$ is upward-closed: if~$\tau \preccurlyeq \sigma$ and if~$\tau \in F$, then~$\sigma \in F$; \\[-3.0em]
    \item $F$ is downward-directed: $F$ is inhabited, and if~$\alpha,\beta \in F$,
    then there is a common refinement~$\sigma \preccurlyeq \alpha,\beta$ such
    that~$\sigma \in F$; and \\[-2.0em]
    \item $F$ splits the covering system: if~$\sigma \in F$ and~$R \in
    \Cov(\sigma)$, then~$\tau \in F$ for some~$\tau \in R$.
  \end{enumerate}
\end{frame}

\addtocounter{framenumber}{-1}

\end{document}
