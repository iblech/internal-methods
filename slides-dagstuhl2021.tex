\documentclass[12pt,utf8,notheorems,compress,t]{beamer}
\usepackage{etex}

\usepackage{pgfpages}
\usepackage[export]{adjustbox}
%\setbeameroption{show notes on second screen}
%\setbeamertemplate{note page}[plain]
%\newcommand{\jnote}[2]{\only<#1>{\note{\setlength\parskip{\medskipamount}\justifying\footnotesize#2\par}}}
\newcommand{\jnote}[2]{}

% Workaround for the issue described at
% https://tex.stackexchange.com/questions/164406/beamer-using-href-in-notes.
\newcommand{\fixedhref}[2]{\makebox[0pt][l]{\hspace*{\paperwidth}\href{#1}{#2}}\href{#1}{#2}}

\usepackage[english]{babel}

\usepackage{graphbox}
\usepackage{mathtools}
\usepackage{booktabs}
\usepackage{stmaryrd}
\usepackage{amssymb}
\usepackage{array}
\usepackage{ragged2e}
\usepackage{multicol}
\usepackage{tabto}
\usepackage{xstring}
\usepackage{ifthen}
\usepackage[normalem]{ulem}
\usepackage[all]{xy}
\xyoption{rotate}
\usepackage{tikz}
\usetikzlibrary{calc,shapes,shapes.callouts,shapes.arrows,patterns,fit,backgrounds,decorations.pathmorphing,positioning}
\hypersetup{colorlinks=true}

\newcommand*\circled[1]{\tikz[baseline=(char.base)]{%
  \node[shape=circle,draw,inner sep=1pt] (char) {#1};}}

\DeclareFontFamily{U}{bbm}{}
\DeclareFontShape{U}{bbm}{m}{n}
   {  <5> <6> <7> <8> <9> <10> <12> gen * bbm
      <10.95> bbm10%
      <14.4>  bbm12%
      <17.28><20.74><24.88> bbm17}{}
\DeclareFontShape{U}{bbm}{m}{sl}
   {  <5> <6> <7> bbmsl8%
      <8> <9> <10> <12> gen * bbmsl
      <10.95> bbmsl10%
      <14.4> <17.28> <20.74> <24.88> bbmsl12}{}
\DeclareFontShape{U}{bbm}{bx}{n}
   {  <5> <6> <7> <8> <9> <10> <12> gen * bbmbx
      <10.95> bbmbx10%
      <14.4> <17.28> <20.74> <24.88> bbmbx12}{}
\DeclareFontShape{U}{bbm}{bx}{sl}
   {  <5> <6> <7> <8> <9> <10> <10.95> <12> <14.4> <17.28>%
      <20.74> <24.88> bbmbxsl10}{}
\DeclareFontShape{U}{bbm}{b}{n}
   {  <5> <6> <7> <8> <9> <10> <10.95> <12> <14.4> <17.28>%
      <20.74> <24.88> bbmb10}{}
\DeclareMathAlphabet{\mathbbm}{U}{bbm}{m}{n}
\SetMathAlphabet\mathbbm{bold}{U}{bbm}{bx}{n}


\usepackage{pifont}
\newcommand{\cmark}{\ding{51}}
\newcommand{\xmark}{\ding{55}}
\DeclareSymbolFont{extraup}{U}{zavm}{m}{n}
\DeclareMathSymbol{\varheart}{\mathalpha}{extraup}{86}

\graphicspath{{images/}}

\usepackage[protrusion=true,expansion=true]{microtype}

\setlength\parskip{\medskipamount}
\setlength\parindent{0pt}

\title{Bridging the foundational gap: Updating algebraic geometry in
face of current challenges regarding formalizability, constructivity and
predicativity}
\author{Ingo Blechschmidt}
\date{November 23th, 2021}

\useinnertheme[shadow=true]{rounded}
\setbeamerfont{block title}{size={}}

\useinnertheme{rectangles}

\usecolortheme{orchid}
\usecolortheme{seahorse}
\definecolor{mypurple}{RGB}{150,0,255}
\setbeamercolor{structure}{fg=mypurple}
\definecolor{myred}{RGB}{150,0,0}
\setbeamercolor*{title}{bg=myred,fg=white}
\setbeamercolor*{titlelike}{bg=myred,fg=white}
\setbeamercolor{frame}{bg=black}

\usefonttheme{serif}
\usepackage[T1]{fontenc}
\usepackage{libertine}

% lifted from https://arxiv.org/abs/1506.08870
\DeclareFontFamily{U}{min}{}
\DeclareFontShape{U}{min}{m}{n}{<-> udmj30}{}
\newcommand\yon{\!\text{\usefont{U}{min}{m}{n}\symbol{'210}}\!}

\newcommand{\A}{\mathcal{A}}
\newcommand{\B}{\mathcal{B}}
\newcommand{\C}{\mathcal{C}}
\newcommand{\M}{\mathcal{M}}
\renewcommand{\AA}{\mathbb{A}}
\newcommand{\BB}{\mathbb{B}}
\newcommand{\pp}{\mathbbm{p}}
\newcommand{\MM}{\mathbb{M}}
\newcommand{\E}{\mathcal{E}}
\newcommand{\F}{\mathcal{F}}
\newcommand{\FF}{\mathbb{F}}
\newcommand{\G}{\mathcal{G}}
\newcommand{\J}{\mathcal{J}}
\newcommand{\GG}{\mathbb{G}}
\renewcommand{\O}{\mathcal{O}}
\newcommand{\K}{\mathcal{K}}
\newcommand{\NN}{\mathbb{N}}
\newcommand{\QQ}{\mathbb{Q}}
\newcommand{\RR}{\mathbb{R}}
\newcommand{\TT}{\mathbb{T}}
\newcommand{\PP}{\mathbb{P}}
\newcommand{\ZZ}{\mathbb{Z}}
\newcommand{\CC}{\mathbb{C}}
\renewcommand{\P}{\mathcal{P}}
\newcommand{\aaa}{\mathfrak{a}}
\newcommand{\ppp}{\mathfrak{p}}
\newcommand{\fff}{\mathfrak{f}}
\newcommand{\defeq}{\vcentcolon=}
\newcommand{\defeqv}{\vcentcolon\equiv}
\newcommand{\Sh}{\mathrm{Sh}}
\newcommand{\GL}{\mathrm{GL}}
\newcommand{\Zar}{\mathrm{Zar}}
\newcommand{\op}{\mathrm{op}}
\newcommand{\Set}{\mathrm{Set}}
\newcommand{\Eff}{\mathrm{Ef{}f}}
\newcommand{\Sch}{\mathrm{Sch}}
\newcommand{\Aff}{\mathrm{Aff}}
\newcommand{\Ring}{\mathrm{Ring}}
\newcommand{\LocRing}{\mathrm{LocRing}}
\newcommand{\LRS}{\mathrm{LRS}}
\newcommand{\Hom}{\mathrm{Hom}}
\newcommand{\Spec}{\mathrm{Spec}}
\newcommand{\lra}{\longrightarrow}
\newcommand{\RelSpec}{\operatorname{Spec}}
\renewcommand{\_}{\mathpunct{.}}
\newcommand{\?}{\,{:}\,}
\newcommand{\speak}[1]{\ulcorner\text{\textnormal{#1}}\urcorner}
\newcommand{\ul}[1]{\underline{#1}}
\newcommand{\affl}{\ensuremath{{\ul{\ensuremath{\AA}}^1}}}
\newcommand{\Ll}{\text{iff}}
\newcommand{\inv}{inv.\@}
\newcommand{\seq}[1]{\mathrel{\vdash\!\!\!_{#1}}}
\newcommand{\hg}{\mathbin{:}}  % homogeneous coordinates

\setbeamertemplate{blocks}[rounded][shadow=false]

\newenvironment{indentblock}{%
  \list{}{\leftmargin\leftmargin}%
  \item\relax
}{%
  \endlist
}

% Adapted from https://latex.org/forum/viewtopic.php?t=2251 (Stefan Kottwitz)
\newenvironment<>{hilblock}{
  \begin{center}
    \begin{minipage}{9.05cm}
      \setlength{\textwidth}{9.05cm}
      \begin{actionenv}#1
        \def\insertblocktitle{}
        \par
        \usebeamertemplate{block begin}}{
        \par
        \usebeamertemplate{block end}
      \end{actionenv}
    \end{minipage}
  \end{center}}

\newenvironment{changemargin}[2]{%
  \begin{list}{}{%
    \setlength{\topsep}{0pt}%
    \setlength{\leftmargin}{#1}%
    \setlength{\rightmargin}{#2}%
    \setlength{\listparindent}{\parindent}%
    \setlength{\itemindent}{\parindent}%
    \setlength{\parsep}{\parskip}%
  }%
  \item[]}{\end{list}}

\tikzset{
  invisible/.style={opacity=0,text opacity=0},
  visible on/.style={alt={#1{}{invisible}}},
  alt/.code args={<#1>#2#3}{%
    \alt<#1>{\pgfkeysalso{#2}}{\pgfkeysalso{#3}}}
}

\newcommand{\pointthis}[3]{%
  \tikz[remember picture,baseline]{
    \node[anchor=base,inner sep=0,outer sep=0] (#2) {#2};
    \node[visible on=#1,overlay,rectangle callout,rounded corners,callout relative pointer={(0.3cm,0.5cm)},fill=blue!20] at ($(#2.north)+(-0.1cm,-1.1cm)$) {#3};
  }%
}

\tikzset{
  invisible/.style={opacity=0,text opacity=0},
  visible on/.style={alt={#1{}{invisible}}},
  alt/.code args={<#1>#2#3}{%
    \alt<#1>{\pgfkeysalso{#2}}{\pgfkeysalso{#3}}}
}

\newcommand{\hcancel}[5]{%
  \tikz[baseline=(tocancel.base)]{
    \node[inner sep=0pt,outer sep=0pt] (tocancel) {#1};
    \draw[red!80, line width=0.4mm] ($(tocancel.south west)+(#2,#3)$) -- ($(tocancel.north east)+(#4,#5)$);
  }%
}

\newcommand{\explain}[7]{%
  \tikz[remember picture,baseline]{
    \node[anchor=base,inner sep=2pt,outer sep=0,fill=#3,rounded corners] (label) {#1};
    \node[anchor=north,visible on=<#2>,overlay,rectangle callout,rounded corners,callout
    relative pointer={(0.0cm,0.5cm)+(0.0cm,#6)},fill=#3] at ($(label.south)+(0,-0.3cm)+(#4,#5)$) {#7};
  }%
}

\newcommand{\explainstub}[2]{%
  \tikz[remember picture,baseline]{
    \node[anchor=base,inner sep=2pt,outer sep=0,fill=#2,rounded corners] (label) {#1};
  }%
}

\newcommand{\squiggly}[1]{%
  \tikz[remember picture,baseline]{
    \node[anchor=base,inner sep=0,outer sep=0] (label) {#1};
    \draw[thick,color=red!80,decoration={snake,amplitude=0.5pt,segment
    length=3pt},decorate] ($(label.south west) + (0,-2pt)$) -- ($(label.south east) + (0,-2pt)$);
  }%
}

% Adapted from https://latex.org/forum/viewtopic.php?t=2251 (Stefan Kottwitz)
\newenvironment<>{varblock}[2]{\begin{varblockextra}{#1}{#2}{}}{\end{varblockextra}}
\newenvironment<>{varblockextra}[3]{
  \begin{center}
    \begin{minipage}{#1}
      \begin{actionenv}#4
        {\centering \hil{#2}\par}
	\def\insertblocktitle{}%\centering #2}
        \def\varblockextraend{#3}
	\usebeamertemplate{block begin}}{
        \par
        \usebeamertemplate{block end}
        \varblockextraend
      \end{actionenv}
    \end{minipage}
  \end{center}}

\setbeamertemplate{headline}{}

\setbeamertemplate{frametitle}{%
  \vskip0.5em%
  \leavevmode%
  \begin{beamercolorbox}[dp=1ex,center]{}%
    \usebeamercolor[fg]{item}{\textbf{{\Large\insertframetitle}}}
  \end{beamercolorbox}%
  \vskip-0.2em%
}

\setbeamertemplate{navigation symbols}{}

\newcounter{framenumberpreappendix}
\newcommand{\backupstart}{
  \setcounter{framenumberpreappendix}{\value{framenumber}}
}
\newcommand{\backupend}{
  \addtocounter{framenumberpreappendix}{-\value{framenumber}}
  \addtocounter{framenumber}{\value{framenumberpreappendix}}
}

\newcommand{\insertframeextra}{}
\setbeamertemplate{footline}{%
  \begin{beamercolorbox}[wd=\paperwidth,ht=2.25ex,dp=1ex,right,rightskip=1mm,leftskip=1mm]{}%
    % \inserttitle
    \hfill
    \insertframenumber\insertframeextra\,/\,\inserttotalframenumber
  \end{beamercolorbox}%
  \vskip0pt%
}


\newcommand{\hil}[1]{{\usebeamercolor[fg]{item}{\textbf{#1}}}}
\newcommand{\bad}[1]{\textcolor{red!90}{\textnormal{#1}}}

\newcommand{\bignumber}[1]{%
  \renewcommand{\insertenumlabel}{#1}\scalebox{1.2}{\!\usebeamertemplate{enumerate item}\!}
}
\newcommand{\bigheart}{\includegraphics{heart}}

\newcommand{\subhead}[1]{{\centering\textcolor{gray}{\hrulefill}\quad\textnormal{#1}\quad\textcolor{gray}{\hrulefill}\par}}

\begin{document}

\addtocounter{framenumber}{-1}

%\setbeamertemplate{headline}{\mynav{gray}{gray}{gray}}

{\usebackgroundtemplate{\begin{minipage}{\paperwidth}\vspace*{4.95cm}\includegraphics[width=\paperwidth]{topos-horses}\end{minipage}}
\begin{frame}[c]
  \centering

  \bigskip
  %\includegraphics[height=0.32\textwidth]{olivia-lattices}
  \bigskip
  \bigskip
  \bigskip
  \bigskip

  \begin{tikzpicture}
    \def\R{8pt}
    \node (pretitle) {Bridging the foundational gap:};
    \node (title) [below of=pretitle] {\vbox{\vspace*{-0.5em}Updating \hil{algebraic
    geometry} in face of \hil{current challenges} regarding
    \hil{formalizability}, \hil{constructivity} and \hil{predicativity}}};
    \begin{pgfonlayer}{background}
      \draw[decorate, very thick, draw=mypurple]
        ($(title.south west) + (\R, 0)$) arc(270:180:\R) --
        ($(title.north west) + (0, -\R)$) arc(180:90:\R) --
        ($(title.north east) + (-\R, 0)$) arc(90:0:\R) --
        ($(title.south east) + (0, \R)$) arc(0:-90:\R) --
        cycle;
    \end{pgfonlayer}
  \end{tikzpicture}

  \scriptsize
  \textit{-- an invitation --}
  \bigskip
  \bigskip
  \bigskip

  Dagstuhl Seminar 20202: \\
  Geometric Logic, Constructivisation, and Automated Theorem Proving \\
  \ \\
  November 23th, 2021
  \bigskip

  Ingo Blechschmidt \\
  University of Augsburg
  \par
\end{frame}}

{\usebackgroundtemplate{\begin{minipage}{\paperwidth}\vspace*{5.95cm}\includegraphics[width=\paperwidth]{fr1}\end{minipage}}
\begin{frame}{Algebraic geometry in a nutshell}
  Turn \hil{commutative rings} into \hil{spaces}, and \hil{glue} those
  spaces.
  \bigskip

  \subhead{Examples}
  \begin{enumerate}
    \item $\Spec(k[X_1,\ldots,X_n]) = \AA^n$.
    \item $\Spec(k[X,Y]/(Y-X^2)) = \text{standard parabola}$.
    \item Gluing
    \begin{tikzpicture}[scale=0.2,baseline={([yshift=-.6ex]current bounding box.center)}]
      \draw (0,0) circle (1);
      \draw [fill=white] (90:1) circle (0.3);
    \end{tikzpicture}
    with
    \begin{tikzpicture}[scale=0.2,baseline={([yshift=-.6ex]current bounding box.center)}]
      \draw (0,0) circle (1);
      \draw [fill=white] (270:1) circle (0.3);
    \end{tikzpicture}
    along
    \begin{tikzpicture}[scale=0.2,baseline={([yshift=-.6ex]current bounding box.center)}]
      \draw (0,0) circle (1);
      \draw [fill=white] (90:1) circle (0.3);
      \draw [fill=white] (270:1) circle (0.3);
    \end{tikzpicture}
    yields
    $\PP^1$.
  \end{enumerate}
  \bigskip

  \subhead{Concrete results}
  Fermat's Last Theorem: For~$n \geq 3$, no positive integers satisfy
  \[ a^n + b^n = c^n. \]
\end{frame}}

{\usebackgroundtemplate{\begin{minipage}{\paperwidth}\vspace*{3.95cm}\includegraphics[width=\paperwidth]{staircase}\end{minipage}}
\begin{frame}{Transfinite methods}
  The standard presentation of algebraic geometry hinges on:
  \begin{itemize}
    \item large structures: classes, large categories, universes, \ldots
    \item powersets
    \item law of excluded middle
    \item axiom of choice
  \end{itemize}
  \bigskip

  \subhead{despite:}
  \begin{enumerate}
    \item subject matter (in part) very concrete
    \item computer algebra systems for computations practical
    \item constructive algebra well-established
    \item high-level proofs often constructive
  \end{enumerate}
\end{frame}}

\begin{frame}{Formalizing algebraic geometry}
  \bigskip
  \bigskip
  \bigskip
  \includegraphics[width=\textwidth]{schemes-in-lean2}

  \vspace*{-17em}
  \includegraphics[frame=0.1em,bb=0 170 610 625,width=0.3\textwidth]{schemes-in-coq}\hfill
  \includegraphics[frame=0.1em,bb=0 170 610 625,width=0.3\textwidth]{schemes-in-lean}
\end{frame}

\begin{frame}{A trinitarian challenge}
  \begin{center}\tiny $1^3 + \cdots + n^3 = (1 + \cdots + n)^2$\end{center}
  \vspace*{-0.6em}

  \begin{columns}
    \begin{column}{0.33\textwidth}
      \centering
      \includegraphics[height=7em]{lem}

      \hil{constructivity}
      \par
    \end{column}
    \begin{column}{0.33\textwidth}
      \centering
      \includegraphics[height=7em]{sum-of-cubics}

      \hil{elegance}
      \par
    \end{column}

    \begin{column}{0.33\textwidth}
      \centering
      \includegraphics[height=7em]{proof-assistants}

      \par
      \hil{formalizability}
    \end{column}
  \end{columns}
  \bigskip
  \bigskip

  \subhead{Thesis}
  \centering
  Elegant mathematics lends itself to rewarding formalization.
\end{frame}

\begin{frame}{Foundational possibilities}
  \small

  \subhead{Schemes and scheme morphisms}
  \only<1-2>{
  \begin{itemize}
    \setlength\itemsep{-0.2em}
    %\item locally ringed spaces (or locales, distributive lattices, sites,
    %toposes, arithmetic universes), formal geometries
    \item locally ringed \ldots \\
    \ldots{} topological spaces \tabto{4cm}-- \textit{unconstructive, hard to formalize} \\
    \ldots{} locales \tabto{4cm}-- \textit{impredicative, superfluous opens} \\
    \ldots{} distributive lattices \tabto{4cm}-- \textit{need extension from basis} \\
    \ldots{} sites \tabto{4cm}-- \textit{current favorite! morphisms intricate} \\
    \ldots{} toposes \tabto{4cm}-- \textit{large structure} \\
    \ldots{} arithmetic universes \tabto{4cm}\mbox{-- \textit{a bit better;
    issue with relative spectrum}}
    \item formal geometries \tabto{4cm}-- \textit{can be regarded as sites}
    \item functor of points \tabto{4cm}-- \textit{large structure, issue with
    sizes or \\ \qquad
    with schemes not of finite presentation over the base ring}
    \item formal gluing data \tabto{4cm}-- \textit{morphisms intricate}
  \end{itemize}
  \textbf{A surprise of uncertain import:}
  \pause
  Internally to the big Zariski topos of a base scheme, the Zariski spectrum of
  a finitely presented algebra does have enough points! [Cherubini--Coquand]}
  \bigskip
  \pause

  \subhead{Zariski cohomology}
  \begin{itemize}
    \setlength\itemsep{-0.2em}
    \small
    \item Čech cohomology \tabto{4.5cm}-- \textit{ad hoc, but fine for}
    \tabto{4.5cm}\phantom{--} \textit{quasicompact separated schemes}
    \item injective resolutions \tabto{4.5cm}-- \textit{hopelessly unconstructive}
    \item dynamical injectives \tabto{4.5cm}-- \textit{??}
    \item flabby resolutions \tabto{4.5cm}-- \textit{probably unconstructive}
    \item pointwise Kan extensions \tabto{4.5cm}-- \textit{fine! partially
    defined $\mathbb{R}\Gamma$;} \tabto{4.5cm}-- \textit{existence
    verified for} \tabto{4.5cm}\phantom{--} \textit{quasicompact separated schemes;}
    \tabto{4.5cm}\phantom{--} \textit{hyper coverings?}
  \end{itemize}

  \subhead{Étale cohomology}
  \textcolor{red!90}{??}
\end{frame}

\end{document}

Consider: Thierry, Space of Valuations!
