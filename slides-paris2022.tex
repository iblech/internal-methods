\documentclass[12pt,utf8,notheorems,compress,t]{beamer}
\usepackage{etex}

\usepackage{pgfpages}
\usepackage[export]{adjustbox}

% Workaround for the issue described at
% https://tex.stackexchange.com/questions/164406/beamer-using-href-in-notes.
\newcommand{\fixedhref}[2]{\makebox[0pt][l]{\hspace*{\paperwidth}\href{#1}{#2}}\href{#1}{#2}}

\usepackage[english]{babel}

\usepackage[normalem]{ulem}
\usepackage{graphbox}
\usepackage{mathtools}
\usepackage{booktabs}
\usepackage{stmaryrd}
\usepackage{amssymb}
\usepackage{array}
\usepackage{ragged2e}
\usepackage{multicol}
\usepackage{tabto}
\usepackage{xstring}
\usepackage{proof}
\usepackage{ifthen}
\usepackage[normalem]{ulem}
\usepackage[all]{xy}
\xyoption{rotate}
\usepackage{tikz}
\usetikzlibrary{calc,shapes,shapes.callouts,shapes.arrows,patterns,fit,backgrounds,decorations.pathmorphing,positioning}
\hypersetup{colorlinks=true}

\newcommand*\circled[1]{\tikz[baseline=(char.base)]{%
  \node[shape=circle,draw,inner sep=1pt] (char) {#1};}}

\DeclareFontFamily{U}{bbm}{}
\DeclareFontShape{U}{bbm}{m}{n}
   {  <5> <6> <7> <8> <9> <10> <12> gen * bbm
      <10.95> bbm10%
      <14.4>  bbm12%
      <17.28><20.74><24.88> bbm17}{}
\DeclareFontShape{U}{bbm}{m}{sl}
   {  <5> <6> <7> bbmsl8%
      <8> <9> <10> <12> gen * bbmsl
      <10.95> bbmsl10%
      <14.4> <17.28> <20.74> <24.88> bbmsl12}{}
\DeclareFontShape{U}{bbm}{bx}{n}
   {  <5> <6> <7> <8> <9> <10> <12> gen * bbmbx
      <10.95> bbmbx10%
      <14.4> <17.28> <20.74> <24.88> bbmbx12}{}
\DeclareFontShape{U}{bbm}{bx}{sl}
   {  <5> <6> <7> <8> <9> <10> <10.95> <12> <14.4> <17.28>%
      <20.74> <24.88> bbmbxsl10}{}
\DeclareFontShape{U}{bbm}{b}{n}
   {  <5> <6> <7> <8> <9> <10> <10.95> <12> <14.4> <17.28>%
      <20.74> <24.88> bbmb10}{}
\DeclareMathAlphabet{\mathbbm}{U}{bbm}{m}{n}
\SetMathAlphabet\mathbbm{bold}{U}{bbm}{bx}{n}

\usepackage{pifont}
\newcommand{\cmark}{\ding{51}}
\newcommand{\xmark}{\ding{55}}
\DeclareSymbolFont{extraup}{U}{zavm}{m}{n}
\DeclareMathSymbol{\varheart}{\mathalpha}{extraup}{86}

\graphicspath{{images/}}

\usepackage[protrusion=true,expansion=true]{microtype}

\setlength\parskip{\medskipamount}
\setlength\parindent{0pt}

\title{A modal logical multiverse for commutative algebra and combinatorics}

\author{Ingo Blechschmidt}
\date{November 7th, 2022}

%\useinnertheme[shadow=true]
\setbeamerfont{block title}{size={}}

\useinnertheme{rectangles}

\usecolortheme{orchid}
\usecolortheme{seahorse}
\definecolor{mypurple}{RGB}{253,73,34}
\setbeamercolor{structure}{fg=mypurple}
\definecolor{myred}{RGB}{150,0,0}
%\setbeamercolor*{title}{bg=myred,fg=white}
%\setbeamercolor*{titlelike}{bg=myred,fg=white}
\setbeamercolor*{title}{bg=mypurple,fg=white}
\setbeamercolor*{titlelike}{bg=mypurple,fg=white}
\setbeamercolor{frame}{bg=black}

\usefonttheme{serif}
\usepackage[T1]{fontenc}
\usepackage{libertine}

% lifted from https://arxiv.org/abs/1506.08870
\DeclareFontFamily{U}{min}{}
\DeclareFontShape{U}{min}{m}{n}{<-> udmj30}{}
\newcommand\yon{\!\text{\usefont{U}{min}{m}{n}\symbol{'210}}\!}

\newcommand{\A}{\mathcal{A}}
\newcommand{\B}{\mathcal{B}}
\newcommand{\C}{\mathcal{C}}
\newcommand{\M}{\mathcal{M}}
\renewcommand{\AA}{\mathbb{A}}
\newcommand{\BB}{\mathbb{B}}
\newcommand{\pp}{\mathbbm{p}}
\newcommand{\MM}{\mathbb{M}}
\newcommand{\E}{\mathcal{E}}
\newcommand{\F}{\mathcal{F}}
\newcommand{\FF}{\mathbb{F}}
\newcommand{\G}{\mathcal{G}}
\newcommand{\J}{\mathcal{J}}
\newcommand{\GG}{\mathbb{G}}
\renewcommand{\O}{\mathcal{O}}
\newcommand{\K}{\mathcal{K}}
\newcommand{\NN}{\mathbb{N}}
\newcommand{\QQ}{\mathbb{Q}}
\newcommand{\RR}{\mathbb{R}}
\newcommand{\TT}{\mathbb{T}}
\newcommand{\PP}{\mathbb{P}}
\newcommand{\ZZ}{\mathbb{Z}}
\newcommand{\CC}{\mathbb{C}}
\renewcommand{\P}{\mathcal{P}}
\newcommand{\aaa}{\mathfrak{a}}
\newcommand{\ccc}{\mathfrak{c}}
\newcommand{\ppp}{\mathfrak{p}}
\newcommand{\fff}{\mathfrak{f}}
\newcommand{\mmm}{\mathfrak{m}}
\newcommand{\defeq}{\vcentcolon=}
\newcommand{\defeqv}{\vcentcolon\equiv}
\newcommand{\Sh}{\mathrm{Sh}}
\newcommand{\GL}{\mathrm{GL}}
\newcommand{\Zar}{\mathrm{Zar}}
\newcommand{\op}{\mathrm{op}}
\newcommand{\Set}{\mathrm{Set}}
\newcommand{\Eff}{\mathrm{Ef{}f}}
\newcommand{\Sch}{\mathrm{Sch}}
\newcommand{\Aff}{\mathrm{Aff}}
\newcommand{\Ring}{\mathrm{Ring}}
\newcommand{\LocRing}{\mathrm{LocRing}}
\newcommand{\LRS}{\mathrm{LRS}}
\newcommand{\Hom}{\mathrm{Hom}}
\newcommand{\Spec}{\mathrm{Spec}}
\newcommand{\lra}{\longrightarrow}
\newcommand{\RelSpec}{\operatorname{Spec}}
\renewcommand{\_}{\mathpunct{.}}
\newcommand{\?}{\,{:}\,}
\newcommand{\speak}[1]{\ulcorner\text{\textnormal{#1}}\urcorner}
\newcommand{\ul}[1]{\underline{#1}}
\newcommand{\affl}{\ensuremath{{\ul{\ensuremath{\AA}}^1}}}
\newcommand{\Ll}{\text{iff}}
\newcommand{\inv}{inv.\@}
\newcommand{\seq}[1]{\mathrel{\vdash\!\!\!_{#1}}}
\newcommand{\hg}{\mathbin{:}}  % homogeneous coordinates

\setbeamertemplate{blocks}[rounded][shadow=false]

\newenvironment{indentblock}{%
  \list{}{\leftmargin\leftmargin}%
  \item\relax
}{%
  \endlist
}

% Adapted from https://latex.org/forum/viewtopic.php?t=2251 (Stefan Kottwitz)
\newenvironment<>{hilblock}{
  \begin{center}
    \begin{minipage}{9.05cm}
      \setlength{\textwidth}{9.05cm}
      \begin{actionenv}#1
        \def\insertblocktitle{}
        \par
        \usebeamertemplate{block begin}}{
        \par
        \usebeamertemplate{block end}
      \end{actionenv}
    \end{minipage}
  \end{center}}

\newenvironment{changemargin}[2]{%
  \begin{list}{}{%
    \setlength{\topsep}{0pt}%
    \setlength{\leftmargin}{#1}%
    \setlength{\rightmargin}{#2}%
    \setlength{\listparindent}{\parindent}%
    \setlength{\itemindent}{\parindent}%
    \setlength{\parsep}{\parskip}%
  }%
  \item[]}{\end{list}}

\tikzset{
  invisible/.style={opacity=0,text opacity=0},
  visible on/.style={alt={#1{}{invisible}}},
  alt/.code args={<#1>#2#3}{%
    \alt<#1>{\pgfkeysalso{#2}}{\pgfkeysalso{#3}}}
}

\newcommand{\pointthis}[3]{%
  \tikz[remember picture,baseline]{
    \node[anchor=base,inner sep=0,outer sep=0] (#2) {#2};
    \node[visible on=#1,overlay,rectangle callout,rounded corners,callout relative pointer={(0.3cm,0.5cm)},fill=blue!20] at ($(#2.north)+(-0.1cm,-1.1cm)$) {#3};
  }%
}

\tikzset{
  invisible/.style={opacity=0,text opacity=0},
  visible on/.style={alt={#1{}{invisible}}},
  alt/.code args={<#1>#2#3}{%
    \alt<#1>{\pgfkeysalso{#2}}{\pgfkeysalso{#3}}}
}

\newcommand{\hcancel}[5]{%
  \tikz[baseline=(tocancel.base)]{
    \node[inner sep=0pt,outer sep=0pt] (tocancel) {#1};
    \draw[red!80, line width=0.4mm] ($(tocancel.south west)+(#2,#3)$) -- ($(tocancel.north east)+(#4,#5)$);
  }%
}

\newcommand{\explain}[7]{%
  \tikz[remember picture,baseline]{
    \node[anchor=base,inner sep=2pt,outer sep=0,fill=#3,rounded corners] (label) {#1};
    \node[anchor=north,visible on=<#2>,overlay,rectangle callout,rounded corners,callout
    relative pointer={(0.0cm,0.5cm)+(0.0cm,#6)},fill=#3] at ($(label.south)+(0,-0.3cm)+(#4,#5)$) {#7};
  }%
}

\newcommand{\explainstub}[2]{%
  \tikz[remember picture,baseline]{
    \node[anchor=base,inner sep=2pt,outer sep=0,fill=#2,rounded corners] (label) {#1};
  }%
}

\newcommand{\squiggly}[1]{%
  \tikz[remember picture,baseline]{
    \node[anchor=base,inner sep=0,outer sep=0] (label) {#1};
    \draw[thick,color=red!80,decoration={snake,amplitude=0.5pt,segment
    length=3pt},decorate] ($(label.south west) + (0,-2pt)$) -- ($(label.south east) + (0,-2pt)$);
  }%
}

% Adapted from https://latex.org/forum/viewtopic.php?t=2251 (Stefan Kottwitz)
\newenvironment<>{varblock}[2]{\begin{varblockextra}{#1}{#2}{}}{\end{varblockextra}}
\newenvironment<>{varblockextra}[3]{
  \begin{center}
    \begin{minipage}{#1}
      \begin{actionenv}#4
        {\centering \hil{#2}\par}
	\def\insertblocktitle{}%\centering #2}
        \def\varblockextraend{#3}
	\usebeamertemplate{block begin}}{
        \par
        \usebeamertemplate{block end}
        \varblockextraend
      \end{actionenv}
    \end{minipage}
  \end{center}}

\setbeamertemplate{headline}{}

\setbeamertemplate{frametitle}{%
  \leavevmode%
  \vskip-1.6em%
  \begin{beamercolorbox}[dp=1ex,center,wd=\paperwidth,ht=2.25ex]{title}%
    \bf\insertframetitle
  \end{beamercolorbox}%
  \vskip-0.2em%
}

\setbeamertemplate{navigation symbols}{}

\newcounter{framenumberpreappendix}
\newcommand{\backupstart}{
  \setcounter{framenumberpreappendix}{\value{framenumber}}
}
\newcommand{\backupend}{
  \addtocounter{framenumberpreappendix}{-\value{framenumber}}
  \addtocounter{framenumber}{\value{framenumberpreappendix}}
}

\newcommand{\insertframeextra}{}
\setbeamertemplate{footline}{%
  \begin{beamercolorbox}[wd=\paperwidth,ht=2.25ex,dp=1ex,right,rightskip=1mm,leftskip=1mm]{}%
    % \inserttitle
    \hfill
    \insertframenumber\insertframeextra\,/\,\inserttotalframenumber
  \end{beamercolorbox}%
  \vskip0pt%
}


\newcommand{\hil}[1]{{\usebeamercolor[fg]{item}{\textbf{#1}}}}
\newcommand{\bad}[1]{\textcolor{red!90}{\textnormal{#1}}}
\newcommand{\good}[1]{\textcolor{mypurple}{\textnormal{#1}}}

\newcommand{\bignumber}[1]{%
  \renewcommand{\insertenumlabel}{#1}\scalebox{1.2}{\!\usebeamertemplate{enumerate item}\!}
}
\newcommand{\normalnumber}[1]{%
  {\renewcommand{\insertenumlabel}{#1}\!\usebeamertemplate{enumerate item}\!}
}
\newcommand{\bigheart}{\includegraphics{heart}}

\newcommand{\subhead}[1]{{\centering\textcolor{gray}{\hrulefill}\quad\textnormal{#1}\quad\textcolor{gray}{\hrulefill}\par}}

\newcommand{\badbox}[1]{\colorbox{red!30}{#1}}
\newcommand{\infobox}[1]{\colorbox{yellow!70}{\color{black}#1}}

% taken from JDH "The modal logic of arithmetic potentialism and the universal algorithm"
\DeclareMathOperator{\possible}{\text{\tikz[scale=.6ex/1cm,baseline=-.6ex,rotate=45,line width=.1ex]{\draw (-1,-1) rectangle (1,1);}}}
\DeclareMathOperator{\necessary}{\text{\tikz[scale=.6ex/1cm,baseline=-.6ex,line width=.1ex]{\draw (-1,-1) rectangle (1,1);}}}
\DeclareMathOperator{\xpossible}{\text{\tikz[scale=.6ex/1cm,baseline=-.6ex,rotate=45,line width=.1ex]{\draw (-1,-1) rectangle (1,1); \draw[very thin] (-.6,-.6) rectangle (.6,.6);}}}
\DeclareMathOperator{\xnecessary}{\text{\tikz[scale=.6ex/1cm,baseline=-.6ex,line width=.1ex]{\draw (-1,-1) rectangle (1,1); \draw[very thin] (-.6,-.6) rectangle (.6,.6);}}}

% Taken from Todd Lehman (CC-BY-SA) at https://tex.stackexchange.com/a/44920/32372

\newcommand{\setisprime}[1]{
  % Sets \isprime based on #1.
  \ifnum#1=1 \gdef\isprime{0} \else \gdef\isprime{1} \fi
  \foreach \sip in {2, 3,5,...,#1} {
    \pgfmathparse{\sip*\sip>#1? 1:0}
    \ifthenelse{\pgfmathresult=1}{
      % Early-out if \sip^2 > #1.
      \breakforeach
    }{
      % Otherwise test if \sip divides #1.
      \pgfmathparse{Mod(#1,\sip)==0? 1:0}
      \ifthenelse{\pgfmathresult=1}{
        \gdef\isprime{0}
        \breakforeach
      }{}
    }
  }
}

\newcommand{\setxy}[1]{
  % Sets \x and \y to loction of cell #1.
  \pgfmathtruncatemacro{\x}{Mod(#1-1,\cols)}
  \pgfmathtruncatemacro{\y}{(#1-1) / \cols}
  \pgfmathtruncatemacro{\y}{\cols - 1 - \y}
  \pgfmathparse{2.5*(\x+.5)}\let\x\pgfmathresult
  \pgfmathparse{2.5*(\y+.5)}\let\y\pgfmathresult
}

\newcommand{\numlabel}[2]{
  % Draws label #2 at cell #1.
  \setxy{\n}
  \node[fill=none, text=black] at (\x,\y) {#2};
}

\newcommand{\drawpolygon}[2]{
  % Draws polygon with #2 vertexes at cell #1.
  \setxy{#1}
  \ifthenelse{#2>1}{ % Polygon must have at least 2 sides.
    \ifthenelse{#2<30}{ % Draw polygon if it has a small number of sides.
      \filldraw (\x,\y) +(90:1)
      \foreach \drawi in {1,...,#2} {-- +(\drawi/#2*360+90:1)} -- cycle;
    }{ % Else approximate with circle.
      \filldraw (\x,\y) circle(1);
    }
  }{}
}

\newcommand{\setpolygoncolor}[1]{
  % Sets color based on #1.
  \gdef\polycolor{black}
  \ifnum#1=2\gdef\polycolor{black!50!white}\fi
  \ifnum#1=3\gdef\polycolor{yellow!95!red}\fi
  \ifnum#1=5\gdef\polycolor{yellow!0!red}\fi
  \ifnum#1=7\gdef\polycolor{blue!75!green}\fi
  \ifnum#1=11\gdef\polycolor{blue!70!red}\fi
  \ifnum#1=13\gdef\polycolor{blue!40!red}\fi
  \ifnum#1=17\gdef\polycolor{green!50!blue}\fi
  \ifnum#1=19\gdef\polycolor{green!80!black}\fi
  \ifnum#1=23\gdef\polycolor{green!50!red}\fi
  \ifnum#1=29\gdef\polycolor{yellow!50!black}\fi
  \ifnum#1=31\gdef\polycolor{orange!50!black}\fi
  \ifnum#1=37\gdef\polycolor{red!50!black}\fi
  \ifnum#1=41\gdef\polycolor{purple!50!black}\fi
  \ifnum#1=43\gdef\polycolor{blue!50!black}\fi
  \ifnum#1=47\gdef\polycolor{green!50!black}\fi
  \ifnum#1=53\gdef\polycolor{white!50!black}\fi
  \ifnum#1=59\gdef\polycolor{white!50!black}\fi
  \ifnum#1=61\gdef\polycolor{white!50!black}\fi
  \ifnum#1=67\gdef\polycolor{white!50!black}\fi
}

\newcommand{\sieve}[2]{
  \def\cols{#1}
  \def\rows{#2}
  \begin{tikzpicture}[scale=.5]
  \pgfmathtruncatemacro{\nmax}{\rows * \cols}

  \foreach \n in {1,...,\nmax} {
    \begin{scope}[fill=gray, fill opacity=.05,
                  draw=gray, draw opacity=.10,
                  line width=4]
      \drawpolygon{\n}{\n}
    \end{scope}
    \setisprime{\n}
    \ifthenelse{\isprime=1}{
      \numlabel{\n}{\bf\n}
    }{
      \def\startintensity{.33}
      \def\incrintensity{.10}
      \def\intensity{\startintensity}

      \def\m{\n}
      \pgfmathtruncatemacro{\i}{\m / 2}

      % Divide \m by \i until \m is extinguished.
      % Increment \i each time it does not divide into \m.
      \whiledo{\m>1}{
        \setisprime{\i}
        \pgfmathparse{Mod(\m,\i)==0? 1:0}
        \ifthenelse{\pgfmathresult=1\and\isprime=1}{
          \setpolygoncolor{\i}
          \begin{scope}[fill=\polycolor, fill opacity=\intensity,
                        draw=\polycolor!85!black, draw opacity=\intensity,
                        line width=\intensity*1.5]
            \drawpolygon{\n}{\i}
          \end{scope}
          \pgfmathtruncatemacro{\m}{\m / \i}
          \pgfmathparse{\intensity + \incrintensity}\let\intensity\pgfmathresult
        }{
          \pgfmathtruncatemacro{\i}{\i - 1}
          \def\intensity{\startintensity}
        }
      }
      \begin{scope}[text=black, text opacity=.5]
        \numlabel{\n}{\scriptsize\n}
      \end{scope}
    }
  }

  \end{tikzpicture}
}

\newcommand{\fakesieve}[2]{
  \def\cols{#1}
  \def\rows{#2}
  \begin{tikzpicture}[scale=.5,opacity=0]
  \pgfmathtruncatemacro{\nmax}{\rows * \cols}

  \foreach \n in {1,...,\nmax} {
    \begin{scope}[fill=gray,
                  draw=gray,
                  line width=4]
      \drawpolygon{\n}{\n}
    \end{scope}
    \setisprime{\n}
    \ifthenelse{\isprime=1}{
      \numlabel{\n}{\bf\n}
    }{
      \def\startintensity{.33}
      \def\incrintensity{.10}
      \def\intensity{\startintensity}

      \def\m{\n}
      \pgfmathtruncatemacro{\i}{\m / 2}

      % Divide \m by \i until \m is extinguished.
      % Increment \i each time it does not divide into \m.
      \whiledo{\m>1}{
        \setisprime{\i}
        \pgfmathparse{Mod(\m,\i)==0? 1:0}
        \ifthenelse{\pgfmathresult=1\and\isprime=1}{
          \setpolygoncolor{\i}
          \begin{scope}[fill=\polycolor,
                        draw=\polycolor!85!black,
                        line width=\intensity*1.5]
            \drawpolygon{\n}{\i}
          \end{scope}
          \pgfmathtruncatemacro{\m}{\m / \i}
          \pgfmathparse{\intensity + \incrintensity}\let\intensity\pgfmathresult
        }{
          \pgfmathtruncatemacro{\i}{\i - 1}
          \def\intensity{\startintensity}
        }
      }
      \begin{scope}[text=black]
        \numlabel{\n}{\scriptsize\n}
      \end{scope}
    }
  }

  \end{tikzpicture}
}


\begin{document}

\addtocounter{framenumber}{-1}

%\setbeamertemplate{headline}{\mynav{gray}{gray}{gray}}

{\usebackgroundtemplate{\begin{minipage}{\paperwidth}\vspace*{1.59cm}\includegraphics[width=\paperwidth]{forest-light}\end{minipage}}
\begin{frame}[c]
  \centering

  \bigskip
  %\includegraphics[height=0.32\textwidth]{olivia-lattices}
  \bigskip
  \bigskip
  \bigskip

  \scriptsize
  \textit{-- an invitation --}

  \setbeamercolor{block body}{bg=black!100}
  \begin{block}{}
  \centering\normalsize\color{white}
  A \hil{modal logical multiverse} for \\ commutative algebra and combinatorics
  \end{block}

  \bigskip
  \bigskip
  \bigskip
  \bigskip
  \bigskip
  \bigskip
  \bigskip

  \emph{Séminaire Général de Logique} \\
  IMJ-PRG
  \bigskip

  Paris \\
  November 7th, 2022
  \bigskip

  Ingo Blechschmidt \\
  University of Augsburg
  \par
\end{frame}}

\definecolor{mypurple}{RGB}{150,0,255}
\setbeamercolor{structure}{fg=mypurple}


\section{Motivation}

{\usebackgroundtemplate{\begin{minipage}{\paperwidth}\vspace*{5.95cm}\includegraphics[width=\paperwidth]{fr1-lighter}\end{minipage}}
\begin{frame}{Three proofs}
  \begin{columns}[c]
    \begin{column}{0.15\textwidth}
      \scalebox{0.23}{\sieve{6}{6}}
    \end{column}
    \begin{column}{0.85\textwidth}
      \begin{varblock}{\textwidth}{}
        \justifying
        \textbf{Thm.}
        For every~$n$, there is a prime larger than~$n$.
      \end{varblock}
      \vspace*{-0.4em}
      \emph{Proof.} Any prime factor of~$n! + 1$ will do. \qed
    \end{column}
  \end{columns}
  \pause

  \begin{columns}[c]
    \begin{column}{0.15\textwidth}
      \includegraphics[width=\textwidth,valign=t]{monad}
    \end{column}
    \begin{column}{0.85\textwidth}
      \begin{varblock}{\textwidth}{}
        \justifying
        \textbf{Thm.}
        Every infinite sequence~$\alpha : \NN \to \NN$ is \emph{good}
        in that there are numbers~$i < j$ such that~$\alpha(i) \leq \alpha(j)$.
      \end{varblock}
      \vspace*{-0.4em}
      \mbox{\emph{Proof.} There is a \bad{minimal value}~$\alpha(i)$.
      Set~$j \defeq i{+}1$.\qed}
    \end{column}
  \end{columns}
  \pause

  \begin{columns}[c]
    \begin{column}{0.15\textwidth}
      \includegraphics[width=\textwidth,valign=t]{calabi-yau}
    \end{column}
    \begin{column}{0.85\textwidth}
      \begin{varblock}{\textwidth}{}
        \justifying
        \textbf{Thm.}
        Let~$M$ be a surjective matrix with more rows than columns over a
        ring~$A$. Then~$1 = 0$ in~$A$.
      \end{varblock}
      \vspace*{-0.4em}
      \justifying\emph{Proof.}
      \only<3>{\bad{Assume not.} Then there is~a \bad{maximal ideal} $\mmm$.
      The matrix is surjective over~$A/\mmm$. Since~$A/\mmm$ is a field, this
      is a contradiction to basic linear algebra. \\[1.3em]\ }\only<4>{Write~$M =
      \left(\begin{smallmatrix}x\\y\end{smallmatrix}\right)$. By surjectivity,
      have~$u, v$ with
      \vspace*{-0.6em}
      \[
        u \left(\begin{smallmatrix}x\\y\end{smallmatrix}\right) = \left(\begin{smallmatrix}1\\0\end{smallmatrix}\right)
        \quad\text{and}\quad
        v \left(\begin{smallmatrix}x\\y\end{smallmatrix}\right) = \left(\begin{smallmatrix}0\\1\end{smallmatrix}\right).
      \]

      \vspace*{-0.6em}
      Hence
      $
        1 = (vy) (ux) = (uy) (vx) = 0
      $.}
      \qed
    \end{column}
  \end{columns}
\end{frame}}
% XXX: chart of ideals of ℤ

% \begin{document}

\begin{frame}{Three take-aways}
  \bigskip
  \begin{columns}[c]
    \begin{column}{0.25\textwidth}
      \includegraphics[width=\textwidth,valign=c]{turing-machine}
    \end{column}
    \begin{column}{0.75\textwidth}
      Abstract proofs should be blueprints for concrete ones.
    \end{column}
  \end{columns}
  \pause
  \bigskip
  \bigskip

  \begin{columns}[c]
    \begin{column}{0.25\textwidth}
      \includegraphics[width=\textwidth,valign=c]{3-adic-numbers}
    \end{column}
    \begin{column}{0.75\textwidth}
      For every inhabited set~$X$,
      there is a \hil{flavor~of~parametrized~mathematics} \\
      which is home
      to a gadget called \\
      \hil{generic surjection}~$\NN \twoheadrightarrow X$.
    \end{column}
  \end{columns}
  \pause
  \bigskip
  \bigskip

  \begin{columns}[c]
    \begin{column}{0.25\textwidth}
      \includegraphics[width=\textwidth,valign=c]{multiverse}
    \end{column}
    \begin{column}{0.75\textwidth}
      There is a \hil{rich modal multiverse} of \\
      flavors of parametrized mathematics.
    \end{column}
  \end{columns}

  \visible<3->{\begin{tikzpicture}[overlay]
    \node[anchor=south east,inner sep=0] (image) at (10.8,0.3) {
      \reflectbox{\includegraphics[width=1em]{butterfly}}
    };
  \end{tikzpicture}}
\end{frame}

% \begin{document}


\section{The set-theoretic multiverse}

{\usebackgroundtemplate{\begin{minipage}{\paperwidth}\vspace*{3.59cm}\includegraphics[width=\paperwidth]{staircase}\end{minipage}}
\begin{frame}{A brief timeline}
  \begin{itemize}
    \item[18xx] Cantor studies the \hil{continuum hypothesis}, the claim
    that~$2^{\aleph_0} = \aleph_1$.
    \item[19xx] Zermelo and Fraenkel offer the axioms of~\textsc{zfc}.
    \item[1938] Gödel proves: If~\textsc{zfc} is consistent, then so
    is~\textsc{zfc}+\textsc{ch}.
    \item[1963] Cohen proves: If~\textsc{zfc} is consistent, then so
    is~\textsc{zfc}+$\neg$\textsc{ch}.
    \item[since 1963 and ongoing] Set theorists pursue additional axioms to
    settle~\textsc{ch} (one way or another).  % XXX persue?
    \item[201x] Hamkins offers his paper on the \hil{multiverse position} in
    the philosophy of set theory, arguing that the program of pursing
    additional axioms (while successful in many ways) is doomed to fail in
    settling~\textsc{ch}.
  \end{itemize}
\end{frame}
\begin{frame}{The set-theoretic multiverse}
  \begin{tikzpicture}[remember picture,overlay]
    \node[xshift=-1.5cm,yshift=-3.5cm] at (current page.north east)
    {\includegraphics[width=2cm]{branching}};
  \end{tikzpicture}
  \vspace*{-1.5em}

  \begin{block}{}
    \justifying
    \textbf{Def.} A \hil{model of set theory} is a (perhaps class-sized) structure $(M,{\in})$
    satisfying axioms such as those of~\textsc{zfc}.
  \end{block}
  \vspace*{-0.6em}
  {\small\emph{Examples.}\vspace*{-0.9em}
  \begin{itemize}
    \item $V$, the class of all sets \vspace*{-0.6em}
    \item $L$, Gödel's constructible universe \vspace*{-0.6em}
    \item $V[G]$, a forcing extension containing a generic filter~$G$ of \\ some
    poset of forcing conditions \vspace*{-0.6em}
    \item Henkin/term models from consistency of (extensions of)~\textsc{zfc}
  \end{itemize}}
  \pause

  %{\centering\emph{Embracing all models of set theory:}\par}
  \begin{block}{}
    \justifying
    \textbf{Def.} $\possible\varphi$ iff~$\varphi$ holds in \hil{some extension} of
    the current universe. \\
    \phantom{\textbf{Def.}} $\necessary\varphi$ iff~$\varphi$ holds in \hil{all extensions}
    of the current universe.
  \end{block}
  \vspace*{-0.6em}
  \begin{itemize}
    \item $\necessary(\possible\textsc{CH} \wedge \possible\neg\text{CH})$,
    the continuum hypothesis is a \hil{switch}
    \item \ \\[-1.2em]\mbox{$\necessary\possible\necessary(\text{$X$ is countable})$,
    existence of an enumeration is a \hil{button}}
  \end{itemize}
  \pause
  \medskip

  \textbf{NB.} In the multiverse of extensions of a given field~$K$,

  \vspace*{-0.5em}
  \begin{itemize}
    \item ``there is a square root of~$-1$'' is a button and
    \item ``every polynomial splits into linear factors'' is a switch.
  \end{itemize}
\end{frame}}


% \begin{document}

\section{Parametrized mathematics}

\renewcommand{\insertframeextra}{a}
\begin{frame}{Parametrized mathematics}
  \medskip
  {\centering\includegraphics[width=0.7\textwidth]{eigenvector}\par}
  \small

  \begin{changemargin}{-0.5em}{-1.5em}
  \begin{enumerate}
    \item[\infobox{local}]
    \ \\[-1.2em]\mbox{``Every real symmetric matrix
    does \only<4>{\hil{not not} }have an eigenvector.''
    \good{\cmark}}
    \medskip

    \item[\infobox{global}] ``For every continuous family of symmetric
    matrices, \\\only<4>{\hil{on a dense open} }eigenvectors can locally
    be picked continuously.''
    \only<1-2>{\includegraphics[height=0.7em]{question-mark}}%
    \only<3>{\bad{\xmark}}%
    \only<4>{\good{\cmark}}

    \smallskip
    \pause
    \justifying
    \scriptsize
    ``Let~$X$ be a topological space and let~$A : X \to
    M^\text{sym}_n(\RR)$ be a continuous map to the space of symmetric~$(n \times
    n)$-matrices. Then there is an open covering~$\bigcup_{i \in I} U_i$
    \only<1-3>{of~$X$}\only<4>{\hil{of a dense open subset}~$U \subseteq X$} such
    that or all indices~$i \in I$, there is a continuous map~$v : U_i \to \RR^n$
    such that for all~$x \in U_i$, the vector~$v(x)$ is an eigenvector
    of~$A(x)$.''
  \end{enumerate}
  \end{changemargin}
\end{frame}

\addtocounter{framenumber}{-1}
\renewcommand{\insertframeextra}{b}
\begin{frame}{Parametrized mathematics}
  \bigskip
  \bigskip
  {\centering
    \includegraphics[height=9em]{damian-gvirtz-moduli-1}
    \quad
    \includegraphics[height=9em]{damian-gvirtz-moduli-2}
    \par}
  \small
  \bigskip

  \begin{changemargin}{-0.5em}{-1.5em}
  \begin{enumerate}
    \item[\infobox{local}]
    \ \\[-1.2em]``Let~$M$ be a finitely generated module over a
    field~$k$. \\ Then~$M$ is \only<4>{\hil{not not} }finite free.''
    \good{\cmark}
    \medskip

    \item[\infobox{global}] ``Let~$M$ be a finitely generated module over a
    ring~$A$. \\ Then~$M^\sim$ is finite locally
    free\only<4>{ \hil{on a dense open}}.''
    \only<1-2>{\includegraphics[height=0.7em]{question-mark}}%
    \only<3>{\bad{\xmark}}%
    \only<4>{\good{\cmark}}

    \smallskip
    \pause
    \justifying
    \scriptsize
    ``Let~$M$ be a finitely generated module over an
    arbitrary commutative ring~$A$. \only<1-3>{Then there is a partition~$1 = f_1 + \cdots +
    f_n \in A$ of unity such that, for each index~$i$, the localized
    module~$M[f_i^{-1}]$ is finite free over~$A[f_i^{-1}]$.}\only<4>{\hil{If~$f = 0$
    is the only element of~$A$} such that~$M[f^{-1}]$ is finite free
    over~$A[f^{-1}]$, \hil{then~$1 = 0$ in~$A$}.}''
  \end{enumerate}
  \end{changemargin}
\end{frame}

\addtocounter{framenumber}{-1}
\renewcommand{\insertframeextra}{c}
{\usebackgroundtemplate{\begin{minipage}{\paperwidth}\vspace*{-1em}\includegraphics[width=\paperwidth]{train-tracks-cropped}\end{minipage}}
\begin{frame}{Parametrized mathematics}
  \bigskip
  \bigskip
  \bigskip
  \bigskip
  \bigskip
  \bigskip
  \bigskip
  \bigskip
  \bigskip
  \bigskip
  \bigskip
  \bigskip
  \bigskip
  \bigskip
  \small

  \begin{changemargin}{-0.5em}{-1.5em}
  \begin{enumerate}
    \item[\infobox{local}]
    \ \\[-1.2em]``Let $R$ be a ring.
    Let $n \geq 0$ be an integer.
    We have
    $$
    H^q(\mathbf{P}^n, \mathcal{O}_{\mathbf{P}^n_R}(d)) = \textit{(omitted)}
    $$
    as $R$-modules.''
    \good{\cmark}
    \medskip

    \item[\infobox{global}] ``Let $S$ be a scheme. Let $n \geq 1$.
    Let $\mathcal{E}$ be a finite locally
    free $\mathcal{O}_S$-module of constant rank $n + 1$.
    For the structure morphism
    $
    \pi : \mathbf{P}(\mathcal{E}) \longrightarrow S,
    $
    we have
    $$
    R^q\pi_*(\mathcal{O}_{\mathbf{P}(\mathcal{E})}(d)) =
    \textit{(omitted)}
    $$
    as sheaves of~$\mathcal{O}_S$-modules.'' \good{\cmark}
  \end{enumerate}
  \end{changemargin}
\end{frame}}

% \begin{document}

\renewcommand{\insertframeextra}{}
{\usebackgroundtemplate{\begin{minipage}{\paperwidth}\vspace*{4.95cm}\includegraphics[width=\paperwidth]{topos-horses-lighter}\end{minipage}}
\begin{frame}{Sites and toposes}
  For every \hil{site}~$\C$, the
  \hil{sheaves} on~$\C$ form a \hil{topos}. \emph{Notable sites:}
  \begin{enumerate}
    \item The \hil{site of opens} of a topological space
    \item The \hil{Zariski site} of a ring
    \item The \hil{site of finite approximations} of surjections~$\NN
    \twoheadrightarrow X$
    \item The \hil{classifying site} of a geometric theory
  \end{enumerate}
  \pause

  Every topos supports an \hil{internal language},
  and this language is sound with respect to \hil{intuitionistic
  logic}.
  \pause

  \begin{block}{}
    \justifying
    \textbf{Def.} A statement~$\varphi$ holds \ldots
    \begin{itemize}
      \item\small \hil{everywhere} ($\necessary\varphi$) iff
      it holds in every (Grothendieck) topos (over the current base topos).
      \item \hil{somewhere} ($\possible\varphi$) iff
      it holds in some positive topos.
      \item \hil{proximally} ($\xpossible\varphi$) iff
      it holds in some positive ouvert topos.
    \end{itemize}
  \end{block}
\end{frame}
\begin{frame}{The multiverse of parametrized mathematics}
  \begin{block}{}
    \justifying
    \textbf{Def.} A statement~$\varphi$ holds \ldots
    \begin{itemize}
      \item\small \hil{everywhere} ($\necessary\varphi$) iff
      it holds in every (Grothendieck) topos (over the current base topos).
      \item \hil{somewhere} ($\possible\varphi$) iff
      it holds in some positive topos.
      \item \hil{proximally} ($\xpossible\varphi$) iff
      it holds in some positive ouvert topos.
    \end{itemize}
  \end{block}

  \begin{changemargin}{-1.2em}{-1.2em}
    \small
    \begin{enumerate}
      %\item If a function~$\NN \to \NN$ does \emph{everywhere} \emph{not~not} have
      %a zero, then it actually has a zero.
      \item \emph{Somewhere,} the law of
      excluded middle holds. \emph{In fact, we even have:} \\
      It is \emph{everywhere} the case that the law of
      excluded middle holds \emph{somewhere}.
      \smallskip
      \item Assuming Zorn's lemma, it is
      everywhere the case that it the axiom of choice holds
      somewhere. [Barr]
      \smallskip
      \item If a geometric implication holds \emph{somewhere}, then it holds already here.
      \smallskip
      \item If a first-order statement holds \emph{proximally}, then it holds already here.
      \smallskip
      \item For every set~$X$, \emph{proximally} there is a
      surjection~$\NN \twoheadrightarrow X$.
      \smallskip
      \item Every ring \emph{proximally} has a maximal ideal.
    \end{enumerate}
  \end{changemargin}
\end{frame}}


\end{document}

A modal logical multiverse for commutative algebra and combinatorics

In the spirit of the set-theoretic multiverse philosophy put forward by
Joel David Hamkins, we explore a related modal multiverse populated by
Kripke models and more general worlds. In this
multiverse—as the talk will explain—the law of excluded middle can be
switched on and off like a light bulb and countability is a button (for
every set X of every world, there is a larger world containing a
surjection ℕ → X). Our interest in this multiverse is because of
concrete applications in commutative algebra and combinatorics,
including the endeavor of extracting algorithms from proofs utilizing
transfinite techniques. The talk will be framed by several examples of
this kind.


=== Motivation

Infinitude of primes
- algorithm visible from the proof

Two motivating questions:
- Dickson's lemma
- Surjectivity of matrices
How to extract algorithms here?
Spoil elementary proof of surjective case


=== Set-theoretic multiverse

Quick sketch of history including dream solution to CH

Multiverse position: acknowledge that notion of set is underdetermined by ZFC
axioms, embrace multiverse of models

Switches and buttons

Analogy with field extensions

Menagerie of properties of the multiverse


=== Relaxation to topos-theoretic case

"Parametric mathematics" (esp. visible with toposes over spaces)

Modal operators

Examples of modal statements
