\documentclass[12pt,utf8,notheorems,compress,t]{beamer}
\usepackage{etex}

\usepackage{pgfpages}
\usepackage[export]{adjustbox}

% Workaround for the issue described at
% https://tex.stackexchange.com/questions/164406/beamer-using-href-in-notes.
\newcommand{\fixedhref}[2]{\makebox[0pt][l]{\hspace*{\paperwidth}\href{#1}{#2}}\href{#1}{#2}}

\usepackage[english]{babel}

\usepackage[normalem]{ulem}
\usepackage{graphbox}
\usepackage{mathtools}
\usepackage{booktabs}
\usepackage{stmaryrd}
\usepackage{amssymb}
\usepackage{array}
\usepackage{ragged2e}
\usepackage{multicol}
\usepackage{tabto}
\usepackage{xstring}
\usepackage{proof}
\usepackage{ifthen}
\usepackage[normalem]{ulem}
\usepackage[all]{xy}
\xyoption{rotate}
\usepackage{tikz}
\usetikzlibrary{calc,shapes,shapes.callouts,shapes.arrows,patterns,fit,backgrounds,decorations.pathmorphing,positioning}
\hypersetup{colorlinks=true}

\newcommand*\circled[1]{\tikz[baseline=(char.base)]{%
  \node[shape=circle,draw,inner sep=1pt] (char) {#1};}}

\DeclareFontFamily{U}{bbm}{}
\DeclareFontShape{U}{bbm}{m}{n}
   {  <5> <6> <7> <8> <9> <10> <12> gen * bbm
      <10.95> bbm10%
      <14.4>  bbm12%
      <17.28><20.74><24.88> bbm17}{}
\DeclareFontShape{U}{bbm}{m}{sl}
   {  <5> <6> <7> bbmsl8%
      <8> <9> <10> <12> gen * bbmsl
      <10.95> bbmsl10%
      <14.4> <17.28> <20.74> <24.88> bbmsl12}{}
\DeclareFontShape{U}{bbm}{bx}{n}
   {  <5> <6> <7> <8> <9> <10> <12> gen * bbmbx
      <10.95> bbmbx10%
      <14.4> <17.28> <20.74> <24.88> bbmbx12}{}
\DeclareFontShape{U}{bbm}{bx}{sl}
   {  <5> <6> <7> <8> <9> <10> <10.95> <12> <14.4> <17.28>%
      <20.74> <24.88> bbmbxsl10}{}
\DeclareFontShape{U}{bbm}{b}{n}
   {  <5> <6> <7> <8> <9> <10> <10.95> <12> <14.4> <17.28>%
      <20.74> <24.88> bbmb10}{}
\DeclareMathAlphabet{\mathbbm}{U}{bbm}{m}{n}
\SetMathAlphabet\mathbbm{bold}{U}{bbm}{bx}{n}

\usepackage{pifont}
\newcommand{\cmark}{\ding{51}}
\newcommand{\xmark}{\ding{55}}
\DeclareSymbolFont{extraup}{U}{zavm}{m}{n}
\DeclareMathSymbol{\varheart}{\mathalpha}{extraup}{86}

\graphicspath{{images/}}

\usepackage[protrusion=true,expansion=true]{microtype}

\setlength\parskip{\medskipamount}
\setlength\parindent{0pt}

\title{A modal logical multiverse for commutative algebra and combinatorics}

\author{Ingo Blechschmidt}
\date{November 7th, 2022}

%\useinnertheme[shadow=true]
\setbeamerfont{block title}{size={}}

\useinnertheme{rectangles}

\usecolortheme{orchid}
\usecolortheme{seahorse}
\definecolor{mypurple}{RGB}{253,73,34}
\setbeamercolor{structure}{fg=mypurple}
\definecolor{myred}{RGB}{150,0,0}
%\setbeamercolor*{title}{bg=myred,fg=white}
%\setbeamercolor*{titlelike}{bg=myred,fg=white}
\setbeamercolor*{title}{bg=mypurple,fg=white}
\setbeamercolor*{titlelike}{bg=mypurple,fg=white}
\setbeamercolor{frame}{bg=black}

\usefonttheme{serif}
\usepackage[T1]{fontenc}
\usepackage{libertine}

% lifted from https://arxiv.org/abs/1506.08870
\DeclareFontFamily{U}{min}{}
\DeclareFontShape{U}{min}{m}{n}{<-> udmj30}{}
\newcommand\yon{\!\text{\usefont{U}{min}{m}{n}\symbol{'210}}\!}

\newcommand{\A}{\mathcal{A}}
\newcommand{\B}{\mathcal{B}}
\newcommand{\C}{\mathcal{C}}
\newcommand{\M}{\mathcal{M}}
\renewcommand{\AA}{\mathbb{A}}
\newcommand{\BB}{\mathbb{B}}
\newcommand{\pp}{\mathbbm{p}}
\newcommand{\MM}{\mathbb{M}}
\newcommand{\E}{\mathcal{E}}
\newcommand{\F}{\mathcal{F}}
\newcommand{\FF}{\mathbb{F}}
\newcommand{\G}{\mathcal{G}}
\newcommand{\J}{\mathcal{J}}
\newcommand{\GG}{\mathbb{G}}
\renewcommand{\O}{\mathcal{O}}
\newcommand{\K}{\mathcal{K}}
\newcommand{\NN}{\mathbb{N}}
\newcommand{\QQ}{\mathbb{Q}}
\newcommand{\RR}{\mathbb{R}}
\newcommand{\TT}{\mathbb{T}}
\newcommand{\PP}{\mathbb{P}}
\newcommand{\ZZ}{\mathbb{Z}}
\newcommand{\CC}{\mathbb{C}}
\renewcommand{\P}{\mathcal{P}}
\newcommand{\aaa}{\mathfrak{a}}
\newcommand{\ppp}{\mathfrak{p}}
\newcommand{\fff}{\mathfrak{f}}
\newcommand{\mmm}{\mathfrak{m}}
\newcommand{\defeq}{\vcentcolon=}
\newcommand{\defeqv}{\vcentcolon\equiv}
\newcommand{\Sh}{\mathrm{Sh}}
\newcommand{\GL}{\mathrm{GL}}
\newcommand{\Zar}{\mathrm{Zar}}
\newcommand{\op}{\mathrm{op}}
\newcommand{\Set}{\mathrm{Set}}
\newcommand{\Eff}{\mathrm{Ef{}f}}
\newcommand{\Sch}{\mathrm{Sch}}
\newcommand{\Aff}{\mathrm{Aff}}
\newcommand{\Ring}{\mathrm{Ring}}
\newcommand{\LocRing}{\mathrm{LocRing}}
\newcommand{\LRS}{\mathrm{LRS}}
\newcommand{\Hom}{\mathrm{Hom}}
\newcommand{\Spec}{\mathrm{Spec}}
\newcommand{\lra}{\longrightarrow}
\newcommand{\RelSpec}{\operatorname{Spec}}
\renewcommand{\_}{\mathpunct{.}}
\newcommand{\?}{\,{:}\,}
\newcommand{\speak}[1]{\ulcorner\text{\textnormal{#1}}\urcorner}
\newcommand{\ul}[1]{\underline{#1}}
\newcommand{\affl}{\ensuremath{{\ul{\ensuremath{\AA}}^1}}}
\newcommand{\Ll}{\text{iff}}
\newcommand{\inv}{inv.\@}
\newcommand{\seq}[1]{\mathrel{\vdash\!\!\!_{#1}}}
\newcommand{\hg}{\mathbin{:}}  % homogeneous coordinates

\setbeamertemplate{blocks}[rounded][shadow=false]

\newenvironment{indentblock}{%
  \list{}{\leftmargin\leftmargin}%
  \item\relax
}{%
  \endlist
}

% Adapted from https://latex.org/forum/viewtopic.php?t=2251 (Stefan Kottwitz)
\newenvironment<>{hilblock}{
  \begin{center}
    \begin{minipage}{9.05cm}
      \setlength{\textwidth}{9.05cm}
      \begin{actionenv}#1
        \def\insertblocktitle{}
        \par
        \usebeamertemplate{block begin}}{
        \par
        \usebeamertemplate{block end}
      \end{actionenv}
    \end{minipage}
  \end{center}}

\newenvironment{changemargin}[2]{%
  \begin{list}{}{%
    \setlength{\topsep}{0pt}%
    \setlength{\leftmargin}{#1}%
    \setlength{\rightmargin}{#2}%
    \setlength{\listparindent}{\parindent}%
    \setlength{\itemindent}{\parindent}%
    \setlength{\parsep}{\parskip}%
  }%
  \item[]}{\end{list}}

\tikzset{
  invisible/.style={opacity=0,text opacity=0},
  visible on/.style={alt={#1{}{invisible}}},
  alt/.code args={<#1>#2#3}{%
    \alt<#1>{\pgfkeysalso{#2}}{\pgfkeysalso{#3}}}
}

\newcommand{\pointthis}[3]{%
  \tikz[remember picture,baseline]{
    \node[anchor=base,inner sep=0,outer sep=0] (#2) {#2};
    \node[visible on=#1,overlay,rectangle callout,rounded corners,callout relative pointer={(0.3cm,0.5cm)},fill=blue!20] at ($(#2.north)+(-0.1cm,-1.1cm)$) {#3};
  }%
}

\tikzset{
  invisible/.style={opacity=0,text opacity=0},
  visible on/.style={alt={#1{}{invisible}}},
  alt/.code args={<#1>#2#3}{%
    \alt<#1>{\pgfkeysalso{#2}}{\pgfkeysalso{#3}}}
}

\newcommand{\hcancel}[5]{%
  \tikz[baseline=(tocancel.base)]{
    \node[inner sep=0pt,outer sep=0pt] (tocancel) {#1};
    \draw[red!80, line width=0.4mm] ($(tocancel.south west)+(#2,#3)$) -- ($(tocancel.north east)+(#4,#5)$);
  }%
}

\newcommand{\explain}[7]{%
  \tikz[remember picture,baseline]{
    \node[anchor=base,inner sep=2pt,outer sep=0,fill=#3,rounded corners] (label) {#1};
    \node[anchor=north,visible on=<#2>,overlay,rectangle callout,rounded corners,callout
    relative pointer={(0.0cm,0.5cm)+(0.0cm,#6)},fill=#3] at ($(label.south)+(0,-0.3cm)+(#4,#5)$) {#7};
  }%
}

\newcommand{\explainstub}[2]{%
  \tikz[remember picture,baseline]{
    \node[anchor=base,inner sep=2pt,outer sep=0,fill=#2,rounded corners] (label) {#1};
  }%
}

\newcommand{\squiggly}[1]{%
  \tikz[remember picture,baseline]{
    \node[anchor=base,inner sep=0,outer sep=0] (label) {#1};
    \draw[thick,color=red!80,decoration={snake,amplitude=0.5pt,segment
    length=3pt},decorate] ($(label.south west) + (0,-2pt)$) -- ($(label.south east) + (0,-2pt)$);
  }%
}

% Adapted from https://latex.org/forum/viewtopic.php?t=2251 (Stefan Kottwitz)
\newenvironment<>{varblock}[2]{\begin{varblockextra}{#1}{#2}{}}{\end{varblockextra}}
\newenvironment<>{varblockextra}[3]{
  \begin{center}
    \begin{minipage}{#1}
      \begin{actionenv}#4
        {\centering \hil{#2}\par}
	\def\insertblocktitle{}%\centering #2}
        \def\varblockextraend{#3}
	\usebeamertemplate{block begin}}{
        \par
        \usebeamertemplate{block end}
        \varblockextraend
      \end{actionenv}
    \end{minipage}
  \end{center}}

\setbeamertemplate{headline}{}

\setbeamertemplate{frametitle}{%
  \leavevmode%
  \vskip-1.6em%
  \begin{beamercolorbox}[dp=1ex,center,wd=\paperwidth,ht=2.25ex]{title}%
    \bf\insertframetitle
  \end{beamercolorbox}%
  \vskip-0.2em%
}

\setbeamertemplate{navigation symbols}{}

\newcounter{framenumberpreappendix}
\newcommand{\backupstart}{
  \setcounter{framenumberpreappendix}{\value{framenumber}}
}
\newcommand{\backupend}{
  \addtocounter{framenumberpreappendix}{-\value{framenumber}}
  \addtocounter{framenumber}{\value{framenumberpreappendix}}
}

\newcommand{\insertframeextra}{}
\setbeamertemplate{footline}{%
  \begin{beamercolorbox}[wd=\paperwidth,ht=2.25ex,dp=1ex,right,rightskip=1mm,leftskip=1mm]{}%
    % \inserttitle
    \hfill
    \insertframenumber\insertframeextra\,/\,\inserttotalframenumber
  \end{beamercolorbox}%
  \vskip0pt%
}


\newcommand{\hil}[1]{{\usebeamercolor[fg]{item}{\textbf{#1}}}}
\newcommand{\bad}[1]{\textcolor{red!90}{\textnormal{#1}}}

\newcommand{\bignumber}[1]{%
  \renewcommand{\insertenumlabel}{#1}\scalebox{1.2}{\!\usebeamertemplate{enumerate item}\!}
}
\newcommand{\normalnumber}[1]{%
  {\renewcommand{\insertenumlabel}{#1}\!\usebeamertemplate{enumerate item}\!}
}
\newcommand{\bigheart}{\includegraphics{heart}}

\newcommand{\subhead}[1]{{\centering\textcolor{gray}{\hrulefill}\quad\textnormal{#1}\quad\textcolor{gray}{\hrulefill}\par}}

\newcommand{\badbox}[1]{\colorbox{red!30}{#1}}

% taken from JDH "The modal logic of arithmetic potentialism and the universal algorithm"
\DeclareMathOperator{\possible}{\text{\tikz[scale=.6ex/1cm,baseline=-.6ex,rotate=45,line width=.1ex]{\draw (-1,-1) rectangle (1,1);}}}
\DeclareMathOperator{\necessary}{\text{\tikz[scale=.6ex/1cm,baseline=-.6ex,line width=.1ex]{\draw (-1,-1) rectangle (1,1);}}}
\DeclareMathOperator{\xpossible}{\text{\tikz[scale=.6ex/1cm,baseline=-.6ex,rotate=45,line width=.1ex]{\draw (-1,-1) rectangle (1,1); \draw[very thin] (-.6,-.6) rectangle (.6,.6);}}}
\DeclareMathOperator{\xnecessary}{\text{\tikz[scale=.6ex/1cm,baseline=-.6ex,line width=.1ex]{\draw (-1,-1) rectangle (1,1); \draw[very thin] (-.6,-.6) rectangle (.6,.6);}}}

\begin{document}

\addtocounter{framenumber}{-1}

%\setbeamertemplate{headline}{\mynav{gray}{gray}{gray}}

{\usebackgroundtemplate{\begin{minipage}{\paperwidth}\vspace*{1.59cm}\includegraphics[width=\paperwidth]{forest-light}\end{minipage}}
\begin{frame}[c]
  \centering

  \bigskip
  %\includegraphics[height=0.32\textwidth]{olivia-lattices}
  \bigskip
  \bigskip
  \bigskip

  \scriptsize
  \textit{-- an invitation --}

  \setbeamercolor{block body}{bg=black!100}
  \begin{block}{}
  \centering\normalsize\color{white}
  A \hil{modal logical multiverse} for \\ commutative algebra and combinatorics
  \end{block}

  \bigskip
  \bigskip
  \bigskip
  \bigskip
  \bigskip
  \bigskip
  \bigskip

  \emph{Séminaire Général de Logique} \\
  IMJ-PRG
  \bigskip

  Paris \\
  November 7th, 2022
  \bigskip

  Ingo Blechschmidt \\
  University of Augsburg
  \par
\end{frame}}

\definecolor{mypurple}{RGB}{150,0,255}
\setbeamercolor{structure}{fg=mypurple}


\section{Motivation}

{\usebackgroundtemplate{\begin{minipage}{\paperwidth}\vspace*{5.95cm}\includegraphics[width=\paperwidth]{fr1-lighter}\end{minipage}}
\begin{frame}{Three proofs}
  \begin{center}\bignumber{1}\end{center}
  \vspace*{-1.3em}
  \begin{varblock}{\textwidth}{}
    \justifying
    \textbf{Theorem.}
    For every number~$n$, there is a prime larger than~$n$.
  \end{varblock}
  \vspace*{-0.4em}
  {\emph{Proof.} Any prime factor of~$n! + 1$ will do. \qed}
  \pause
  \medskip

  \begin{center}\bignumber{2}\end{center}
  \vspace*{-1.3em}
  \begin{varblock}{\textwidth}{}
    \justifying
    \textbf{Theorem.}
    Every infinite sequence~$\alpha : \NN \to \NN$ is \emph{good}
    in that there are numbers~$i < j$ such that~$\alpha(i) \leq \alpha(j)$.
  \end{varblock}
  \vspace*{-0.4em}
  {\emph{Proof.} By~\badbox{\textsc{lem}}, there is a minimal value~$\alpha(i)$.
  Set~$j \defeq i + 1$.\qed}
  \pause
  \medskip

  \begin{center}\bignumber{3}\end{center}
  \vspace*{-1.3em}
  \begin{varblock}{\textwidth}{}
    \justifying
    \textbf{Theorem.}
    Let~$M$ be a surjective matrix with more rows than columns over a
    commutative ring~$A$.
    Then~$1 = 0$ in~$A$.
  \end{varblock}
  \vspace*{-0.4em}
  {\justifying\emph{Proof.}
  \only<3>{\bad{Assume not.} Then there is~a \bad{maximal ideal} $\mmm$.
  The matrix is surjective over~$A/\mmm$, contradicting
  basic linear algebra.}\only<4>{Write~$M = \left(\begin{smallmatrix}x\\y\end{smallmatrix}\right)$. By surjectivity,
  have~$u, v$ with
  $
    u \left(\begin{smallmatrix}x\\y\end{smallmatrix}\right) = \left(\begin{smallmatrix}1\\0\end{smallmatrix}\right)
  $ and $
    v \left(\begin{smallmatrix}x\\y\end{smallmatrix}\right) = \left(\begin{smallmatrix}0\\1\end{smallmatrix}\right).
  $
  Hence
  $
    1 = (vy) (ux) = (uy) (vx) = 0
  $.}
  \qed\par}
\end{frame}}
% XXX: chart of ideals of ℤ

\begin{frame}{Three take-aways}
  \bigskip
  \begin{columns}[c]
    \begin{column}{0.25\textwidth}
      \includegraphics[width=\textwidth,valign=c]{turing-machine}
    \end{column}
    \begin{column}{0.75\textwidth}
      Abstract proofs should be blueprints for concrete ones.
    \end{column}
  \end{columns}
  \pause
  \bigskip
  \bigskip

  \begin{columns}[c]
    \begin{column}{0.25\textwidth}
      \includegraphics[width=\textwidth,valign=c]{3-adic-numbers}
    \end{column}
    \begin{column}{0.75\textwidth}
      For every inhabited set~$X$,
      there is a \hil{flavor~of~parametrized~mathematics} \\
      which is home
      to a gadget called \\
      \hil{generic surjection}~$\NN \twoheadrightarrow X$.
    \end{column}
  \end{columns}
  \pause
  \bigskip
  \bigskip

  \begin{columns}[c]
    \begin{column}{0.25\textwidth}
      \includegraphics[width=\textwidth,valign=c]{multiverse}
    \end{column}
    \begin{column}{0.75\textwidth}
      There is a \hil{rich multiverse} of \\
      flavors of parametrized mathematics.
    \end{column}
  \end{columns}
\end{frame}

\end{document}

A modal logical multiverse for commutative algebra and combinatorics

In the spirit of the set-theoretic multiverse philosophy put forward by
Joel David Hamkins, we explore a related modal multiverse populated by
Kripke models and more general worlds. In this
multiverse—as the talk will explain—the law of excluded middle can be
switched on and off like a light bulb and countability is a button (for
every set X of every world, there is a larger world containing a
surjection ℕ → X). Our interest in this multiverse is because of
concrete applications in commutative algebra and combinatorics,
including the endeavor of extracting algorithms from proofs utilizing
transfinite techniques. The talk will be framed by several examples of
this kind.


=== Motivation

Infinitude of primes
- algorithm visible from the proof

Two motivating questions:
- Dickson's lemma
- Surjectivity of matrices
How to extract algorithms here?
Spoil elementary proof of surjective case


=== Set-theoretic multiverse

Quick sketch of history including dream solution to CH

Multiverse position: acknowledge that notion of set is underdetermined by ZFC
axioms, embrace multiverse of models

Switches and buttons

Analogy with field extensions


=== Relaxation to topos-theoretic case

"Parametric mathematics" (esp. visible with toposes over spaces)

Modal operators

Examples of modal statements
