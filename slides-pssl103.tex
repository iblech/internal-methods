"Any map from the reals to the reals is smooth." This statement holds in the
context of synthetic differential geometry, a well-developed account of smooth
manifolds and related notions which allows us to bring formal reasoning much
closer to geometric intuition.

We employ the internal language of the big Zariski topos of a base scheme to
give a similar account of algebraic geometry, reminiscent of the language of
the classical Italian school and incorporating Grothendieck's functor-of-points
philosophy. From the point of view of the Zariski topos, a scheme over the base
will look like a plain old set, and the affine line will look like a certain
field. Central to the synthetic account is the notion of "synthetic
quasicoherence", which doesn't have an analog in synthetic differential
geometry and which gives the account a distinct algebraic flavor.

The higher-order axiom of synthetic quasicoherence implies all known internal
properties of the affine line, for instance that it is a field and that it is
algebraically closed in a weak sense. We surmise that this is for a deeper
reason, related to the age-old question "which nongeometric sequents hold in
the classifying topos of a geometric theory?". The second part of the talk
reports on work in progress about this topic.
