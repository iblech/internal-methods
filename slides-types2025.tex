- General structure like in HoTTest / ABMV?

- Proof of Berardi–Buriola–Schuster conjecture
- Higher-order functions defined everywhere (in every topos): dialogue trees
- Generic sequence is good = inductively good
- Fun properties of generic surjection

- Let's play Agda

---

1. Hilbert's program
   - basics about wqo's
   - subsequence lemma
   - Dickson
   - inductive reimagination
   - questions: where from? why work? how much stronger?

2. Where do our most cherished definitions come from?
   - a familiar business: enlarging mathematical structures
   - adjoining generic function, surjection, prime ideal, ...
   - observations:
     - well iff generic sequence is good iff everywhere, every sequence is good
       (in particular, every multivalued and every up-to-¬¬-defined sequence is good)
     - wellfounded iff for the generic decreasing sequence, ⊥ iff ...
   - fun facts about generic gadgets

3. Modal language for harnessing the multiverse
   - definition
   - examples
   - implicational wellness
   - dialogue

4. Implementation
   - idea: realize generic gadget as some kind of limit of approximations from
     the base
   - reinterpret, in a mechanic fashion, assertions about generic objects as
     assertions about approximations
     - ex.: "the generic function is defined on input n" as "no matter how the
       empty list evolves over a time to a better approximation xs, eventually
       xs will have length at least succ n"
   - crucially, this interpretation is sound with respect to intuitionistic
     reasoning

---

Towards topological type theory for decrypting transfinite methods in classical mathematics

In constructive mathematics, an ongoing challenge is to extracting
computational content from infinitary arguments in classical
mathematics—especially those that yield concrete results via abstract tools
such as minimal bad sequences or maximal ideals. Today, a wide range of such
techniques exists, constituting a late partial fulfillment of Hilbert's
program.

This talk presents work in progress on porting the modal operators "everywhere"
and "somewhere", as developed in set theory by Joel David Hamkins, Victoria
Gitman and their collaborators, into type theory. This modal enrichment
provides a uniform language for expressing several established
constructivization techniques, and is at the core of a relatively recent new
technique that offers insights on the origins of some of our most cherished
inductive definitions. These developments are closely connected to recent
advances in type-theoretic presheaf and sheaf models, dialogue
continuity, synthetic algebraic geometry and well-quasi-order theory.

---

Which functions are we excluding when we form the type of all functions between
two types? Does every field have an algebraic closure?

Inspired by the modal approach to the set-theoretic multiverse of Joel David
Hamkins, Victoria Gitman and their collaborators, we aim to introduce the modal
operators "everywhere" and "somewhere" to homotopy type theory, proposing a
multiversal perspective on these motivating questions.

---

Well quasi-orders are a combinatorial notion with featureful applications in
graph theory, termination checking, commutative algebra and other subjects.
However, their classical definition, though concise and elegant, poses
significant challenges in constructive mathematics and frameworks that lack
function sets. In response, several constructive substitutes have been
developed, most recently an implicational definition by Stefano Berardi,
Gabriele Buriola and Peter Schuster. In this talk, we investigate how a modal
approach can reinterpret classical proofs involving transfinite methods as
blueprints for constructive proofs grounded in these alternatives, thereby
combining the best of both worlds: Short and abstract proofs, but with
constructive content. We also indicate how the modal approach allows us to
compare the implicational definition with an earlier inductive one.

---

Combinatorics and commutative algebra abound with situations where we prove
quite concrete results by quite abstract transfinite methods, such as minimal
bad sequences or maximal ideals. Amazingly, such infinitary arguments can often
be understood as blueprints for quite explicit computations—as called for by
Hilbert’s programme. In this talk, we will travel the modal toposophic
multiverse to facilitate this kind of mining abstract proofs for refined
quantitative results. We will use a celebrated theorem from order theory about
sequences N→N that everybody can relate with as a running example.

---

Where do some of our most cherished inductive definitions come from? Which
functions are we excluding when we form the type of all functions between two
types? Does every field have an algebraic closure?

Inspired by the modal approach to the set-theoretic multiverse of Joel David
Hamkins, Victoria Gitman and their collaborators, we aim to introduce the modal
operators "everywhere" and "somewhere" to homotopy type theory, proposing a
multiversal perspective on these motivating questions.

Unlike the set-theoretic role model, we focus less on exploring the range of
foundational possibility and more on concrete applications in constructive
mathematics, with the goal of porting results of classical mathematics to
homotopy type theory. It will turn out that every field has an algebraic
closure somewhere; that a transitive relation is well-founded iff nowhere there
is an infinite descending chain; and that somewhere, the law of excluded middle
holds.

This is ongoing joint work with Alexander Oldenziel and connected to the recent
advances with type-theoretic presheaf models and sheaf models.
