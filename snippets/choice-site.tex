\documentclass[10pt]{amsart}
\usepackage[utf8]{inputenc}
\usepackage[english]{babel}
\usepackage{etex}
\usepackage{amsmath,amsthm,amssymb,stmaryrd,color,graphicx,multirow}
\usepackage{mathtools}
\usepackage{setspace}
\usepackage{bussproofs}
\usepackage{xspace}
\usepackage{longtable}
\usepackage{booktabs}
\usepackage{array}
\usepackage[protrusion=true,expansion=true]{microtype}
\usepackage[bookmarksdepth=2,pdfencoding=auto]{hyperref}
\usepackage[all]{xy}

\usepackage{tikz}
\usetikzlibrary{calc,shapes.callouts,shapes.arrows}
\newcommand{\hcancel}[5]{%
    \tikz[baseline=(tocancel.base)]{
        \node[inner sep=0pt,outer sep=0pt] (tocancel) {#1};
        \draw[red, line width=0.3mm] ($(tocancel.south west)+(#2,#3)$) -- ($(tocancel.north east)+(#4,#5)$);
    }%
}

\usepackage[natbib=true,style=numeric]{biblatex}
\usepackage[babel]{csquotes}
\bibliography{bibliography}

\theoremstyle{definition}
\newtheorem{defn}{Definition}[section]
\newtheorem{ex}[defn]{Example}

\theoremstyle{plain}

\newtheorem{prop}[defn]{Proposition}
\newtheorem{cor}[defn]{Corollary}
\newtheorem{lemma}[defn]{Lemma}
\newtheorem{thm}[defn]{Theorem}

\theoremstyle{remark}
\newtheorem{rem}[defn]{Remark}
\newtheorem{question}[defn]{Question}

\newcommand{\ZZ}{\mathbb{Z}}
\newcommand{\FF}{\mathbb{F}}
\renewcommand{\AA}{\mathbb{A}}
\newcommand{\A}{\mathcal{A}}
\renewcommand{\C}{\mathcal{C}}
\newcommand{\D}{\mathcal{D}}
\newcommand{\E}{\mathcal{E}}
\newcommand{\F}{\mathcal{F}}
\renewcommand{\G}{\mathcal{G}}
\renewcommand{\H}{\mathcal{H}}
\renewcommand{\O}{\mathcal{O}}
\newcommand{\K}{\mathcal{K}}
\newcommand{\N}{\mathcal{N}}
\newcommand{\M}{\mathcal{M}}
\renewcommand{\L}{\mathcal{L}}
\renewcommand{\P}{\mathcal{P}}
\newcommand{\R}{\mathcal{R}}
\newcommand{\T}{\mathcal{T}}
\newcommand{\I}{\mathcal{I}}
\newcommand{\J}{\mathcal{J}}
\renewcommand{\S}{\mathcal{S}}
\renewcommand{\U}{\mathcal{U}}
\newcommand{\NN}{\mathbb{N}}
\newcommand{\PP}{\mathbb{P}}
\newcommand{\RR}{\mathbb{R}}
\newcommand{\QQ}{\mathbb{Q}}
\newcommand{\GG}{\mathbb{G}}
\newcommand{\aaa}{\mathfrak{a}}
\newcommand{\ppp}{\mathfrak{p}}
\newcommand{\qqq}{\mathfrak{q}}
\newcommand{\mmm}{\mathfrak{m}}
\newcommand{\nnn}{\mathfrak{n}}
\newcommand{\Hom}{\mathrm{Hom}}
\newcommand{\HOM}{\mathcal{H}\mathrm{om}}
\newcommand{\id}{\mathrm{id}}
\newcommand{\GL}{\mathrm{GL}}
\newcommand{\placeholder}{\underline{\quad}}
\newcommand{\ul}[1]{\underline{#1}}
\newcommand{\Set}{\mathrm{Set}}
\newcommand{\Grp}{\mathrm{Grp}}
\newcommand{\Vect}{\mathrm{Vect}}
\newcommand{\Sh}{\mathrm{Sh}}
\newcommand{\PSh}{\mathrm{PSh}}
\newcommand{\Zar}{\mathrm{Zar}}
\newcommand{\Et}{\mathrm{\acute{E}t}}
\newcommand{\Sch}{\mathrm{Sch}}
\newcommand{\Aff}{\mathrm{Aff}}
\newcommand{\Mod}{\mathrm{Mod}}
\newcommand{\Alg}{\mathrm{Alg}}
\newcommand{\Ring}{\mathrm{Ring}}
\newcommand{\LRL}{\mathrm{LRL}}
\newcommand{\pt}{\mathrm{pt}}
\newcommand{\tors}{\mathrm{tors}}
\newcommand{\lfp}{\mathrm{lfp}}
\newcommand{\fp}{\mathrm{fp}}
\DeclareMathOperator{\Spec}{Spec}
\newcommand{\QcohSpec}[2]{\mathrm{Spec}^{\mathrm{qcoh}}_{#1}{#2}}
\newcommand{\RelSpec}[2]{\mathrm{RelSpec}_{#1}{#2}}
\newcommand{\op}{\mathrm{op}}
\DeclareMathOperator{\colim}{colim}
\DeclareMathOperator{\rank}{rank}
\DeclareMathOperator{\Ann}{Ann}
\DeclareMathOperator{\Int}{int}
\DeclareMathOperator{\Clos}{cl}
\DeclareMathOperator{\Kernel}{ker}
\DeclareMathOperator{\supp}{supp}
\newcommand{\Ass}{\mathrm{Ass}}
\newcommand{\Open}{\T}
\newcommand{\?}{\,{:}\,}
\renewcommand{\_}{\mathpunct{.}\,}
\newcommand{\speak}[1]{\ulcorner\text{\textnormal{#1}}\urcorner}
\newcommand{\Ll}{\vcentcolon\Longleftrightarrow}
\newcommand{\notat}[1]{{!#1}}
\newcommand{\lra}{\longrightarrow}
\newcommand{\lhra}{\ensuremath{\lhook\joinrel\relbar\joinrel\rightarrow}}
\newcommand{\hra}{\hookrightarrow}
\newcommand{\brak}[1]{{\llbracket{#1}\rrbracket}}
\newcommand{\sdense}{{\widehat\Box}}
\newcommand{\sdenseother}{\Box}
\newcommand{\ie}{i.\,e.\@\xspace}
\newcommand{\eg}{e.\,g.\@\xspace}
\newcommand{\vs}{vs.\@\xspace}
\newcommand{\resp}{resp.\@\xspace}
\newcommand{\inv}{inv.\@}
\newcommand{\notnot}{\emph{not not}\xspace}
\newcommand{\affl}{\ensuremath{{\ul{\AA}^1_S}}\xspace}
\newcommand{\afflx}{\ensuremath{{\ul{\AA}^1_X}}\xspace}
\newcommand{\affla}{\ensuremath{{\ul{\AA}^1_{\Spec A}}}\xspace}
\newcommand{\xra}{\xrightarrow}

\newcommand{\XXX}[1]{\textbf{XXX: #1}}

\newcommand{\defeq}{\vcentcolon=}
\newcommand{\defequiv}{\vcentcolon\equiv}
\newcommand{\seq}[1]{\mathrel{\vdash\!\!\!_{#1}}}

\definecolor{gray}{rgb}{0.7,0.7,0.7}

\begin{document}

Unlike with the construction of the little Zariski topos, set-theoretical
issues of size arise when constructing the big Zariski topos. These can be
solved in several different manners, yielding toposes which are not equivalent,
and actually differ in some important aspects, but otherwise enjoy very similar
properties.

\subsubsection*{Naive approach}
Some authors construct the big Zariski topos of~$S$ as the topos of
sheaves over the site~$\Sch/S$ of all schemes over~$S$. This option is quite
attractive since the Yoneda functor~$\Sch/S \to \Sh(\Sch/S)$, which sends
an~$S$-scheme to its functor of points, is fully faithful, therefore the
internal language of~$\Sh(\Sch/S)$ can distinguish arbitrary schemes.

However, since~$\Sch/S$ is not essentially small, forming the sheaf topos is
not possible in plain Zermelo--Fraenkel set theory.

Since it's still possible to meaningfully speak of individual
functors~$(\Sch/S)^\op \to S$, we can attach a Kripke--Joyal semantics
to~$\Sh(\Sch/S)$, as long as we keep in mind that~$\Sh(\Sch/S)$ might
not contain a subobject classifier and might not be cartesian closed. From the
internal point of view, powersets and function sets might therefore not exist.

\subsubsection*{Using Grothendieck universes}
We could also assume the existence of a Grothendieck universe~$\U$
containing~$S$ and construct~$\Zar(S)$ as the topos of sheaves over the
small site~$\Sch_\U/S$, the category of~$S$-schemes contained in~$U$.

By the \emph{comparison lemma}~\ref{XXX}, we could also construct~$\Zar(S)$ as
the topos of sheaves over~$\Aff_\U/S$, the category of~$S$-schemes in~$U$ which
are affine (as absolute schemes), and obtain an equivalent topos.

In this case, the Yoneda functor~$\Sch/S \to \Zar(S)$ might not be faithful,
but the restricted Yoneda functor~$\Sch_\U/S \to \Zar(S)$ will.

\subsubsection*{Approach of the Stacks Project}
The Stacks Project proposes a more nuanced approach, namely expanding a given
set~$M_0$ of schemes containing~$S$ to a superset~$M$ which is closed (up to
isomorphism) under several constructions~\cite[Tag~000H]{stacks-project}: fiber
products, countable coproducts, domains of open and closed immersions and of
morphisms of finite type, spectra of local rings~$\O_{X,x}$, spectra of residue
fields, and others.

The Stacks Project then defines~$\Zar(S)$ as~$\Sh(\Sch_M/S)$, where~$\Sch_M/S$
is the small category of~$S$-schemes in~$M$, or equivalently
as~$\Sh(\Aff_M/S)$. This approach has the advantage that one doesn't have to
assume the existence of a Grothendieck universe; the \emph{partial
universe}~$M$ can be constructed entirely within ZFC set theory using
transfinite recursion.

\subsubsection*{Employing parsimonious sites}
From a topos-theoretical point of view, it's natural to settle for an even
more parsimonious site: the site~$(\Sch/S)_\lfp$ consisting of the~$S$-schemes
which are locally of finite presentation over~$S$, or equivalently the
essentially small site~$(\Aff/S)_\lfp$ of the~$S$-schemes which are locally of
finite presentation over~$S$ and affine (as absolute schemes).\footnote{It's
not reasonable to restrict to the even smaller site consisting of the finitely
presented~$S$-schemes, since open immersions can fail to be finitely presented.
We want the site used to construct~$\Zar(S)$ to be closed under open immersions
for various reasons, for instance to facilitate a comparison with the little
Zariski topos~$\Sh(S)$, whose site does contain all open subsets of~$S$.
Furthermore, since a finitely presented~$S$-scheme might not admit an open
covering by finitely presented~$S$-schemes which are affine (as absolute
schemes), the toposes~$\Sh((\Sch/S)_\fp)$ and~$\Sh((\Aff/S)_\fp)$ can differ.}

In the special case that~$S = \Spec(A)$ is affine, this site is the
dual of the category of finitely presented~$A$-algebras; in this case the
topos-theoretic points of the resulting topos are precisely the local~$A$-algebras,
and moreover, the resulting topos is the classifying topos of the theory of
local~$A$-algebras, such that for any Grothendieck topos~$\E$, geometric
morphisms~$\E \to \Sh((\Aff/S)_\lfp)$ correspond to local~$A$-algebras internal
to~$\E$.

In contrast, the toposes arising when using the larger sites have categories of
points which contain further objects in addition to all local~$A$-algebras; and
no simple description of the theory they classify is known.

A further advantage of these parsimonious sites is that they don't require arbitrary
choices of a starting set~$M_0$ or a way of expanding~$M_0$ to a sufficiently
ample set~$M$ of schemes.

However, the parsimonious sites also have a serious disadvantage, namely that
with them, the Yoneda functor is only fully faithful when restricted
to~$(\Sch/S)_\lfp$. For instance, in the case~$S = \Spec(\ZZ)$, the
schemes~$\Spec(\QQ)$ and the empty scheme have isomorphic functors of points,
whereby~$\Spec(\QQ)$ looks like the empty set from the internal point of
view.\footnote{XXX explicit proof}

In the following, we do not commit to a single one of these options for
resolving the set-theoretical size issues, but rather keep any of these options
in mind. This approach will sometimes necessitate phrases such as ``for
any~$S$-scheme~$T$ contained in the site used to define~$\Zar(S)$'', which might
seem awkward for a seasoned topos-theorist when taken out of context, since
the site used to construct a Grothendieck topos is not at all uniquely
determined by the resulting topos.

We will indicate the few places where the choice of site makes a difference.

\end{document}
