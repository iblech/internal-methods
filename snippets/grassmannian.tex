\documentclass[10pt]{amsart}
\usepackage[utf8]{inputenc}
\usepackage[english]{babel}
\usepackage{etex}
\usepackage{amsmath,amsthm,amssymb,stmaryrd,color,graphicx,multirow}
\usepackage{mathtools}
\usepackage{setspace}
\usepackage{bussproofs}
\usepackage{xspace}
\usepackage{longtable}
\usepackage{booktabs}
\usepackage{array}
\usepackage[protrusion=true,expansion=true]{microtype}
\usepackage[bookmarksdepth=2,pdfencoding=auto]{hyperref}
\usepackage[all]{xy}

\usepackage{tikz}
\usetikzlibrary{calc,shapes.callouts,shapes.arrows}
\newcommand{\hcancel}[5]{%
    \tikz[baseline=(tocancel.base)]{
        \node[inner sep=0pt,outer sep=0pt] (tocancel) {#1};
        \draw[red, line width=0.3mm] ($(tocancel.south west)+(#2,#3)$) -- ($(tocancel.north east)+(#4,#5)$);
    }%
}

\usepackage[natbib=true,style=numeric]{biblatex}
\usepackage[babel]{csquotes}
\bibliography{bibliography}

\theoremstyle{definition}
\newtheorem{defn}{Definition}[section]
\newtheorem{ex}[defn]{Example}

\theoremstyle{plain}

\newtheorem{prop}[defn]{Proposition}
\newtheorem{cor}[defn]{Corollary}
\newtheorem{lemma}[defn]{Lemma}
\newtheorem{thm}[defn]{Theorem}

\theoremstyle{remark}
\newtheorem{rem}[defn]{Remark}
\newtheorem{question}[defn]{Question}

\newcommand{\ZZ}{\mathbb{Z}}
\newcommand{\FF}{\mathbb{F}}
\renewcommand{\AA}{\mathbb{A}}
\newcommand{\A}{\mathcal{A}}
\renewcommand{\C}{\mathcal{C}}
\newcommand{\D}{\mathcal{D}}
\newcommand{\E}{\mathcal{E}}
\newcommand{\F}{\mathcal{F}}
\renewcommand{\G}{\mathcal{G}}
\renewcommand{\H}{\mathcal{H}}
\renewcommand{\O}{\mathcal{O}}
\newcommand{\K}{\mathcal{K}}
\newcommand{\N}{\mathcal{N}}
\newcommand{\M}{\mathcal{M}}
\renewcommand{\L}{\mathcal{L}}
\renewcommand{\P}{\mathcal{P}}
\newcommand{\R}{\mathcal{R}}
\newcommand{\T}{\mathcal{T}}
\newcommand{\I}{\mathcal{I}}
\newcommand{\J}{\mathcal{J}}
\renewcommand{\S}{\mathcal{S}}
\newcommand{\V}{\mathcal{V}}
\newcommand{\NN}{\mathbb{N}}
\newcommand{\PP}{\mathbb{P}}
\newcommand{\RR}{\mathbb{R}}
\newcommand{\QQ}{\mathbb{Q}}
\newcommand{\GG}{\mathbb{G}}
\newcommand{\aaa}{\mathfrak{a}}
\newcommand{\ppp}{\mathfrak{p}}
\newcommand{\qqq}{\mathfrak{q}}
\newcommand{\mmm}{\mathfrak{m}}
\newcommand{\nnn}{\mathfrak{n}}
\newcommand{\Hom}{\mathrm{Hom}}
\newcommand{\HOM}{\mathcal{H}\mathrm{om}}
\newcommand{\id}{\mathrm{id}}
\newcommand{\GL}{\mathrm{GL}}
\newcommand{\placeholder}{\underline{\quad}}
\newcommand{\ul}[1]{\underline{#1}}
\newcommand{\Set}{\mathrm{Set}}
\newcommand{\Grp}{\mathrm{Grp}}
\newcommand{\Vect}{\mathrm{Vect}}
\newcommand{\Sh}{\mathrm{Sh}}
\newcommand{\PSh}{\mathrm{PSh}}
\newcommand{\Zar}{\mathrm{Zar}}
\newcommand{\Et}{\mathrm{\acute{E}t}}
\newcommand{\Sch}{\mathrm{Sch}}
\newcommand{\Mod}{\mathrm{Mod}}
\newcommand{\Alg}{\mathrm{Alg}}
\newcommand{\Ring}{\mathrm{Ring}}
\newcommand{\LRL}{\mathrm{LRL}}
\newcommand{\pt}{\mathrm{pt}}
\newcommand{\tors}{\mathrm{tors}}
\DeclareMathOperator{\Spec}{Spec}
\newcommand{\QcohSpec}[2]{\mathrm{Spec}^{\mathrm{qcoh}}_{#1}{#2}}
\newcommand{\RelSpec}[2]{\mathrm{RelSpec}_{#1}{#2}}
\newcommand{\op}{\mathrm{op}}
\DeclareMathOperator{\colim}{colim}
\DeclareMathOperator{\rank}{rank}
\DeclareMathOperator{\Ann}{Ann}
\DeclareMathOperator{\Int}{int}
\DeclareMathOperator{\Clos}{cl}
\DeclareMathOperator{\Kernel}{ker}
\DeclareMathOperator{\supp}{supp}
\newcommand{\Ass}{\mathrm{Ass}}
\newcommand{\Sym}{\mathrm{Sym}}
\newcommand{\Open}{\T}
\newcommand{\?}{\,{:}\,}
\renewcommand{\_}{\mathpunct{.}\,}
\newcommand{\speak}[1]{\ulcorner\text{\textnormal{#1}}\urcorner}
\newcommand{\Ll}{\vcentcolon\Longleftrightarrow}
\newcommand{\notat}[1]{{!#1}}
\newcommand{\lra}{\longrightarrow}
\newcommand{\lhra}{\ensuremath{\lhook\joinrel\relbar\joinrel\rightarrow}}
\newcommand{\hra}{\hookrightarrow}
\newcommand{\brak}[1]{{\llbracket{#1}\rrbracket}}
\newcommand{\sdense}{{\widehat\Box}}
\newcommand{\sdenseother}{\Box}
\newcommand{\ie}{i.\,e.\@\xspace}
\newcommand{\eg}{e.\,g.\@\xspace}
\newcommand{\vs}{vs.\@\xspace}
\newcommand{\resp}{resp.\@\xspace}
\newcommand{\inv}{inv.\@}
\newcommand{\notnot}{\emph{not not}\xspace}
\newcommand{\affl}{\ensuremath{{\ul{\AA}^1_S}}\xspace}
\newcommand{\afflx}{\ensuremath{{\ul{\AA}^1_X}}\xspace}
\newcommand{\affla}{\ensuremath{{\ul{\AA}^1_{\Spec A}}}\xspace}
\newcommand{\xra}{\xrightarrow}

\newcommand{\XXX}[1]{\textbf{XXX: #1}}

\newcommand{\defeq}{\vcentcolon=}
\newcommand{\defequiv}{\vcentcolon\equiv}
\newcommand{\seq}[1]{\mathrel{\vdash\!\!\!_{#1}}}

\newcommand{\Gr}{\mathrm{Gr}}

\definecolor{gray}{rgb}{0.7,0.7,0.7}

\begin{document}

Let~$S$ be a base scheme and~$\V$ a finite locally free~$\O_S$-module.
We want to illustrate the synthetic approach by verifying using the internal
language of~$\Zar(S)$ the basic fact that the Grassmannian~$\Gr(\V,r)$ of
rank-$r$ locally free quotients of~$\V$, defined as a certain functor of points,
is representable by an~$S$-scheme of finite presentation.

\begin{defn}The \emph{Grassmannian}~$\Gr(\V,r)$ is the functor which associates
to an~$S$-scheme~$f : T \to S$ the set
\[ \Gr(\V,r)(T) \defeq \{
  \text{$U \subseteq f^*\V$ sub-$\O_T$-module} \,|\,
  \text{$(f^*\V)/U$ is locally free of rank~$r$} \}. \]
\end{defn}

\begin{defn}The \emph{synthetic Grassmannian} of rank-$r$ quotients of a
module~$V$ is the set
\[ \Gr(V,r) \defeq \{ \text{$U \subseteq V$ submodule} \,|\,
  \text{$V/U$ is free of rank~$r$} \}. \]
\end{defn}

\begin{prop}The synthetic Grassmannian of~$\V$, as constructed by the internal
language of~$\Zar(S)$ where~$\V$ looks like an ordinary free module, coincides
with the functorially defined Grassmannian.\end{prop}

\begin{proof}Immediate from
Definition~\ref{defn:interpretation-internal-constructions} and
the proof of Proposition~\ref{prop:locally-free-big-zariski}.
\end{proof}

Having established that the internally constructed synthetic Grassmannian
actually describes the external Grassmannian which we're interested in, we can
switch to a fully internal perspective. We'll reflect this switch notationally
by referring to the~$\affl$-module~$V \defeq \V^\Zar$ instead of~$\V$.

We define for any free submodule~$W \subseteq V$ of rank~$r$ the subset
\[ G_W \defeq \{ U \in \Gr(V,r) \,|\, \text{$W \to V \to V/U$ is bijective} \}. \]
This sets admits a more concrete description, since it is in canonical bijection
to the set
\[ G_W' \defeq \{ \pi : V \to W \,|\, \pi \circ \iota = \id \} \]
of all splittings of the inclusion~$\iota : W \hookrightarrow V$: An element~$U
\in G_W$ corresponds to the splitting~$V \twoheadrightarrow V/U
\xrightarrow{({\cong})^{-1}} W$. Conversely, a splitting~$\pi$ corresponds to~$U
\defeq \ker(\pi) \in G_W$.

\begin{prop}The union of the subsets~$G_W$ is~$\Gr(V,r)$.\end{prop}

\begin{proof}Let~$U \in \Gr(V,r)$. Then there exists a
basis~$([v_1],\ldots,[v_r])$ of~$V/U$. The family~$(v_1,\ldots,v_r)$ is
linearly independent in~$V$, therefore the submodule~$W \defeq
\operatorname{span}(v_1,\ldots,v_r) \subseteq V$ is free of rank~$r$. The
canonical linear map~$W \hookrightarrow V \twoheadrightarrow V/U$ maps
the basis~$(v_i)_i$ to the basis~$([v_i])_i$ and is therefore bijective. Thus~$U
\in G_W$.\end{proof}

\begin{prop}The sets~$G_W$ are (quasicompact-)open subsets of~$\Gr(V,r)$.
\end{prop}

\begin{proof}Let~$U \in \Gr(V,r)$. Then~$U \in G_W$ if and only if
the canonical linear map~$W \hookrightarrow V \twoheadrightarrow V/U$
is bijective. Since~$W$ and~$V/U$ are both free modules of rank~$r$, this map is
given by an~$(r \times r)$-matrix~$M$ over~$\affl$; therefore it's bijective if
and only if the determinant of~$M$ is invertible.

Thus we've found a number which is invertible if and only if~$U \in G_W$. By
Corollary~\ref{cor:sufficient-criterion-open-subfunctor}, the truth value of~``$U
\in G_W$'' is open.\end{proof}

\begin{prop}The sets $G_W$ are affine. Moreover, the algebras which the~$G_W$ are
spectra of are finitely presented.\end{prop}

\begin{proof}The set of all linear maps~$V \to W$ is the spectrum of
the~$\affl$-algebra~$A \defeq \Sym(\Hom_\affl(V,W)^\vee)$, since
\begin{align*}
  \Spec(A) &=
  \Hom_{\mathrm{Alg}(\affl)}(\Sym(\Hom_{\mathrm{Mod}(\affl)}(V,W)^\vee), \affl) \\
  &\cong \Hom_{\mathrm{Mod}(\affl)}(\Hom_{\mathrm{Mod}(\affl)}(V,W)^\vee, \affl) \\
  &=\Hom_{\mathrm{Mod}(\affl)}(V,W)^{\vee\vee} \\
  &\cong \Hom_{\mathrm{Mod}(\affl)}(V,W).
\end{align*}
In the last step the assumption that not only~$W$, but also~$V$ is a free module
of finite rank enters. (This is the first time in this development that we need
this assumption.)

The set~$G_W'$ is a closed subset of this spectrum, namely the locus where the
generic linear map~$V \to W$ is a splitting of the inclusion~$\iota : W
\hookrightarrow V$. If we choose bases of~$V$ and~$W$,
whereby~$\Sym(\Hom_\affl(V,W)^\vee)$ is isomorphic
to~$\affl[M_{11},\ldots,M_{rn}]$, we can be more explicit: The set~$G_W'$ is
isomorphic to
\[ \Spec(k[M_{11},\ldots,M_{rn}]/(MN-I)), \]
where~$I$ is the~$(r \times r)$ identity matrix, $M$ is the generic matrix~$M =
(M_{ij})_{ij}$, and~$N$ is the matrix of~$\iota$ with respect to the chosen
bases. The notation~``$(MN-I)$'' denotes the ideal generated by the entries
of~$MN-I$.
\end{proof}

\end{document}

\begin{cor}Die Grassmannsche~$\Gr(V,r)$ ist ein Schema von endlichem
Typ.\end{cor}

\begin{proof}Die Grassmannsche~$\Gr(V,r)$ besitzt eine offene
Überdeckung durch die affinen Schemata~$G_W$ und ist daher ein Schema.

Wenn wir einen Isomorphismus~$V \cong k^n$ wählen, sehen wir, dass
schon~$\binom{n}{r}$ viele dieser affinen Schemata genügen: nämlich diese,
wo~$W$ einer der Standarduntermoduln von~$k^n$ ist (erzeugt durch
Einheitsvektoren).

Denn ist~$U \in \Gr(k^n,r)$, so bildet die Surjektion~$V \to V/U$ die Basis von
mindestens einem dieser Standarduntermoduln auf eine Basis ab und ist daher
bijektiv. (Aus einer surjektiven~$(r \times n)$-Matrix über einem lokalen Ring
kann man stets~$r$ Spalten auswählen, die eine linear unabhängige Familie
bilden.)
\end{proof}

\begin{prop}Für den Tangentialraum an~$U \in \Gr(V,r)$ gilt:
$T_U \Gr(V,r) \cong \Hom(U, V/U)$.\end{prop}

\begin{proof}Die Menge der Tangentialvektoren an~$U$ kann kanonisch mit den
Abbildungen~$\gamma : \Delta \to \Gr(V,r)$ mit~$\gamma(0) = U$ identifiziert
werden. Dabei ist~$\Delta \defeq \{ \varepsilon \in k \,|\, \varepsilon^2 = 0 \}$.
Eine solche Abbildung liftet stets zu einer Abbildung von~$\Delta$ in die Menge
der linear unabhängigen Familien der Länge~$r$ in~$V$. Der Rest sei als
Übungsaufgabe überlassen. Willkommen in der wunderbaren Welt synthetischer
Geometrie.
\end{proof}

\enlargethispage{2em}

\begin{rem}Wiederholt man genau dieselben Argumente in einem anderen Topos --
einem, der für Differentialgeometrie angepasst ist -- erhält man mehr oder
weniger die Darstellbarkeit der Grassmannschen als Mannigfaltigkeit. Das
einzige, was fehlt, ist ein Nachweis der Glattheit.
\end{rem}

\end{document}
