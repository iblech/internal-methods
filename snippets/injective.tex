\documentclass[10pt,reqno]{amsart}
\usepackage[utf8]{inputenc}
\usepackage[english]{babel}
\usepackage{etex}
\usepackage{amsmath,amsthm,amssymb,stmaryrd,color,graphicx,multirow}
\usepackage{mathtools}
\usepackage{setspace}
\usepackage{bussproofs}
\usepackage{xspace}
\usepackage{longtable}
\usepackage{booktabs}
\usepackage{array}
\usepackage[protrusion=true,expansion=true]{microtype}
\usepackage[bookmarksdepth=2,pdfencoding=auto]{hyperref}
\usepackage[all]{xy}
\usepackage{soul}\setul{0.3ex}{}

\usepackage{tikz}
\usetikzlibrary{calc,shapes.callouts,shapes.arrows}
\newcommand{\hcancel}[5]{%
    \tikz[baseline=(tocancel.base)]{
        \node[inner sep=0pt,outer sep=0pt] (tocancel) {#1};
        \draw[red, line width=0.3mm] ($(tocancel.south west)+(#2,#3)$) -- ($(tocancel.north east)+(#4,#5)$);
    }%
}

\usepackage[natbib=true,style=numeric]{biblatex}
\usepackage[babel]{csquotes}
\bibliography{bibliography}

\theoremstyle{definition}
\newtheorem{defn}{Definition}[section]
\newtheorem{ex}[defn]{Example}

\theoremstyle{plain}

\newtheorem{prop}[defn]{Proposition}
\newtheorem{cor}[defn]{Corollary}
\newtheorem{lemma}[defn]{Lemma}
\newtheorem{thm}[defn]{Theorem}
\newtheorem{scholium}[defn]{Scholium}

\theoremstyle{remark}
\newtheorem{rem}[defn]{Remark}
\newtheorem{question}[defn]{Question}

\newcommand{\ZZ}{\mathbb{Z}}
\newcommand{\FF}{\mathbb{F}}
\renewcommand{\AA}{\mathbb{A}}
\newcommand{\A}{\mathcal{A}}
\renewcommand{\C}{\mathcal{C}}
\newcommand{\D}{\mathcal{D}}
\newcommand{\E}{\mathcal{E}}
\newcommand{\F}{\mathcal{F}}
\renewcommand{\G}{\mathcal{G}}
\renewcommand{\H}{\mathcal{H}}
\renewcommand{\O}{\mathcal{O}}
\newcommand{\K}{\mathcal{K}}
\newcommand{\N}{\mathcal{N}}
\newcommand{\M}{\mathcal{M}}
\renewcommand{\L}{\mathcal{L}}
\renewcommand{\P}{\mathcal{P}}
\newcommand{\R}{\mathcal{R}}
\newcommand{\T}{\mathcal{T}}
\newcommand{\I}{\mathcal{I}}
\newcommand{\J}{\mathcal{J}}
\renewcommand{\S}{\mathcal{S}}
\newcommand{\NN}{\mathbb{N}}
\newcommand{\PP}{\mathbb{P}}
\newcommand{\RR}{\mathbb{R}}
\newcommand{\QQ}{\mathbb{Q}}
\newcommand{\GG}{\mathbb{G}}
\newcommand{\aaa}{\mathfrak{a}}
\newcommand{\ppp}{\mathfrak{p}}
\newcommand{\qqq}{\mathfrak{q}}
\newcommand{\mmm}{\mathfrak{m}}
\newcommand{\nnn}{\mathfrak{n}}
\newcommand{\Hom}{\mathrm{Hom}}
\newcommand{\HOM}{\mathcal{H}\mathrm{om}}
\newcommand{\id}{\mathrm{id}}
\newcommand{\GL}{\mathrm{GL}}
\newcommand{\placeholder}{\underline{\quad}}
\let\oldul\ul
\renewcommand{\ul}[1]{\text{\oldul{$#1$}}}
\newcommand{\Set}{\mathrm{Set}}
\newcommand{\Grp}{\mathrm{Grp}}
\newcommand{\Vect}{\mathrm{Vect}}
\newcommand{\Sh}{\mathrm{Sh}}
\newcommand{\PSh}{\mathrm{PSh}}
\newcommand{\Zar}{\mathrm{Zar}}
\newcommand{\Et}{\mathrm{\acute{E}t}}
\newcommand{\Sch}{\mathrm{Sch}}
\newcommand{\Mod}{\mathrm{Mod}}
\newcommand{\Alg}{\mathrm{Alg}}
\newcommand{\Ring}{\mathrm{Ring}}
\newcommand{\LRL}{\mathrm{LRL}}
\newcommand{\pt}{\mathrm{pt}}
\newcommand{\tors}{\mathrm{tors}}
\DeclareMathOperator{\Spec}{Spec}
\newcommand{\RelSpek}{\operatorname{\ul{\mathrm{Spec}}}}
\newcommand{\QcohSpec}[2]{\mathrm{Spec}^{\mathrm{qcoh}}_{#1}{#2}}
\newcommand{\op}{\mathrm{op}}
\DeclareMathOperator{\colim}{colim}
\DeclareMathOperator{\rank}{rank}
\DeclareMathOperator{\Ann}{Ann}
\DeclareMathOperator{\Int}{int}
\DeclareMathOperator{\Clos}{cl}
\DeclareMathOperator{\Kernel}{ker}
\DeclareMathOperator{\supp}{supp}
\newcommand{\Ass}{\mathrm{Ass}}
\newcommand{\Open}{\T}
\newcommand{\?}{\,{:}\,}
\renewcommand{\_}{\mathpunct{.}\,}
\newcommand{\speak}[1]{\ulcorner\text{\textnormal{#1}}\urcorner}
\newcommand{\Ll}{\vcentcolon\Longleftrightarrow}
\newcommand{\notat}[1]{{!#1}}
\newcommand{\lra}{\longrightarrow}
\newcommand{\lhra}{\ensuremath{\lhook\joinrel\relbar\joinrel\rightarrow}}
\newcommand{\hra}{\hookrightarrow}
\newcommand{\brak}[1]{{\llbracket{#1}\rrbracket}}
\newcommand{\sdense}{{\widehat\Box}}
\newcommand{\sdenseother}{\Box}
\newcommand{\ie}{i.\,e.\@\xspace}
\newcommand{\eg}{e.\,g.\@\xspace}
\newcommand{\vs}{vs.\@\xspace}
\newcommand{\resp}{resp.\@\xspace}
\newcommand{\inv}{inv.\@}
\newcommand{\notnot}{\emph{not not}\xspace}
\newcommand{\affl}{\ensuremath{{\ul{\AA}^1_S}}\xspace}
\newcommand{\afflx}{\ensuremath{{\ul{\AA}^1_X}}\xspace}
\newcommand{\affla}{\ensuremath{{\ul{\AA}^1_{\Spec A}}}\xspace}
\newcommand{\xra}{\xrightarrow}

\newcommand{\XXX}[1]{\textbf{XXX: #1}}

\newcommand{\defeq}{\vcentcolon=}
\newcommand{\defequiv}{\vcentcolon\equiv}
\newcommand{\seq}[1]{\mathrel{\vdash\!\!\!_{#1}}}

\newcommand{\stacksproject}[1]{\cite[{\href{http://stacks.math.columbia.edu/tag/#1}{Tag~#1}}]{stacks-project}}

\definecolor{gray}{rgb}{0.7,0.7,0.7}

\begin{document}

\subsection{Flabby sheaves}

Recall that a sheaf~$\F$ of sets on a topological space (or locale)~$X$ is
\emph{flabby} if and only if, for any open subset~$U \subseteq X$ the
restriction map~$\F(X) \to \F(U)$ is surjective.

Flabbiness of a sheaf is a local property, even though it doesn't seem like that
at first sight: If the restrictions~$\F|_{U_i}$ of~$\F$ to the members of an
open covering~$X = \bigcup_i U_i$ are flabby, then the verification that~$\F$ is
flabby can't proceed as follows. ``Let~$s \in \F(U)$ be an arbitrary section.
Since each~$\F|_{U_i}$ is flabby, the section~$s|_{U \cap U_i}$ extends to a
section on~$U_i$.'' The reason is that the individual extensions obtained in
this way might not glue.

A correct proof employs Zorn's lemma in a typical way, considering a maximal
extension and then verifying that the subset this maximal extension is defined
on is all of~$X$.

Since flabbiness is a local property, it's not unreasonable to expect that
flabbiness can be characterized in the internal language. The following
proposition shows that this is indeed the case.

\begin{prop}\label{prop:internal-char-flabbiness}
Let~$\F$ be a sheaf of sets on a topological space~$X$ (or a locale).
Then the following statements are equivalent:
\begin{enumerate}
\item $\F$ is flabby.

\item ``Any section of~$\F$ can be locally extended'':
For any open subset $U \subseteq X$ and any section $s \in \F(U)$ there is
an open covering $X = \bigcup_i V_i$ such that, for each $i$, there is an
extension of $s$ to $U \cup V_i$ (that is, a section $s' \in \F(U \cup V_i)$
such that $s'|_U = s$).

(If $X$ is a space instead of a locale, this can be equivalently formulated as
follows: For any open subset $U \subseteq X$, any section $s \in \F(U)$, and any
point $x \in X$, there is an open neighbourhood $V$ of $x$ and an extension of
$s$ to $U \cup V$.)

% Proof of "internally injective ==> externally injective" references the
% following condition by number.
\item From the point of view of the internal language of~$\Sh(X)$, for any
subsingleton $K \subseteq \F$ there exists an element $s \? \F$ such that~$s \in
K$ if~$K$ is inhabited. More precisely,
\begin{multline*}
  \qquad\qquad\Sh(X) \models
  \forall K \subseteq \F\_
  (\forall s,s'\?K\_ s = s') \Longrightarrow \\
  \exists s\?\F\_ (\text{$K$ is inhabited} \Rightarrow s \in K).\qquad\qquad
\end{multline*}

\item The canonical map $\F \to \P_{\leq 1}(\F), s \mapsto \{s\}$ is
final from the internal point of view, that is
\[ \Sh(X) \models
  \forall K \? \P_{\leq 1}(\F)\_
  \exists s \? \F\_
  K \subseteq \{s\}, \]
where $\P_{\leq 1}(\F)$ is the object of subsingletons of $\F$.
\end{enumerate}
\end{prop}

\begin{proof}
The implication~``(1)~$\Rightarrow$~(2)'' is trivial. The converse direction uses a
typical argument with Zorn's lemma, considering a maximal extension. The
equivalence~``(2)~$\Leftrightarrow$~(3)'' is routine, using the Kripke--Joyal
semantics to interpret the internal statement. Condition~4 is a straightforward
reformulation of Condition~3.
\end{proof}

Condition~2 of the proposition is, unlike the standard definition of flabbiness,
manifestly local. Also its equivalence with Condition~3 and Condition~4 is
intuitionistically valid; therefore one might consider to adopt Condition~2 as
the definition of flabbiness.

The object~$\P_{\leq 1}(\F)$ of subsingletons of $\F$ can be
interpreted as the object of \emph{partially-defined
elements} of $\F$. In this view, the empty subset is the maximally undefined
element and a singleton is a maximally defined element. The proposition shows
that~$\F$ is flabby if and only if any such partially-defined element can be
refined to an honest element of~$\F$.


\subsection{Injective sheaves}

Recall that an object~$I$ of a category~$\C$ is \emph{injective} if and only if,
for any monomorphism~$X \to Y$ in~$\C$ and any morphism~$X \to I$, there
is a lifting such that the diagram
\[ \xymatrix{
  X \ar@{^{(}->}[r]\ar[d] & Y \ar@{-->}[dl] \\
  I
} \]
commutes. Equivalently, an object~$I$ is injective if and only if the Hom
functor~$\Hom_\C(\placeholder, I) : \C^\op \to \Set$ maps monomorphisms in~$\C$ to
surjective maps. This general definition is often specialized to one of these cases:
to the category of modules over a ring, to the category of set-valued sheaves
on a topological space, and to the category of sheaves of~$\O_X$-modules on a
ringed space~$(X,\O_X)$.

The definition is seldomly applied in the category of sets, since in a classical
context it's easy to show that a set is injective if and only if it's
inhabited, thereby completely settling the question which objects are
injective in a trivial manner.

The question is more interesting in an intuitionistic setting, since
intuitionistically one cannot prove that inhabited sets are
injective~\cite{aczel-et-al:injective}; but one can still verify that any set embeds
into an injective set: The powerset~$\P(X)$ and even the smaller
set~$\P_{\leq1}(X)$ of subsingletons of a given set~$X$ are injective. We verify
this in the proof of Lemma~\ref{lemma:enough-flabby}.

For a cartesian closed category~$\C$, there is also the notion of an
\emph{internally injective} object. This is an object~$I$ such that the
internal Hom functor~$[\placeholder, I] : \C^\op \to \C$ maps monomorphisms
in~$\C$ to epimorphisms. In the specical case that~$\C$ is a elementary topos
with a natural numbers object, such as the topos of set-valued sheaves on a
space, this condition can be rephrased in several ways. The following
proposition lists five of these conditions. The equivalence of the first four is
due to Harting~\cite{harting}.

\begin{prop}\label{prop:notions-of-internal-injectivity}
Let~$\E$ be an elementary topos with a natural numbers object. Then
the following statements about an object~$I \in \E$ are equivalent.
\begin{enumerate}
\item $I$ is internally injective.
\item The functor~$[\placeholder, I] : \E^\op \to \E$ maps monomorphisms in $\E$
to morphisms for which any global element of the target locally (after change of
base along an epimorphism) possesses a preimage.
\item For any morphism $p : A \to 1$ in $\E$, the object $p^*I$ has property~(1)
as an object of $\E/A$.
\item For any morphism $p : A \to 1$ in $\E$, the object $p^*I$ has property~(2)
as an object of $\E/A$.
\item From the point of view of the internal language of~$\E$, the object~$I$
is injective.\footnote{In Section~\ref{sect:internal-language}, we have only
introduced the internal language for sheaf toposes. The general definition is
in~\cite[Section~7]{shulman:stack}.}
\end{enumerate}
\end{prop}

\begin{proof}
The implications ``(1)~$\Rightarrow$~(2)'', ``3 $\Rightarrow$~(4)'', ``(3)
$\Rightarrow$~(1)'', and ``(4)~$\Rightarrow$~(2)'' are trivial.

The equivalence ``(3)~$\Leftrightarrow$~(5)'' follows directly from the
interpretation rules of the stack semantics.

The implication ``(2)~$\Rightarrow$~(4)'' employs the
extra left adjoint $p_! : \E/A \to \E$ of $p^* : \E
\to \E/A$~(which maps an object~$(X \to A)$ to~$X$), as in the usual proof that
injective sheaves remain injective when
restricted to smaller open subsets: We have that $p_* \circ [\placeholder, p^*I]_{\E/A}
\cong [\placeholder, I]_\E \circ p_!$, the functor $p_!$ preserves monomorphisms, and one
can check that $p_*$ reflects the property that global elements locally possess
preimages. Details are in~\cite[Thm.~(1)1]{harting}.

The implication ``(4)~$\Rightarrow$~(3)'' follows by performing an extra change of
base, since any non-global element becomes a global element after a suitable
change of base.
\end{proof}

Somewhat surprisingly, and in stark contrast with the situation for internally
projective objects (which are defined dually), internal injectivity coincides
with external injecticity for sheaf toposes over spaces.

\begin{thm}\label{thm:char-injectivity}
Let~$X$ be a topological space (or a locale). An object~$\I \in \Sh(X)$ is
injective if and only if it is internally injective.
\end{thm}

\begin{proof}For the ``only if'' direction, let~$\I$ be an injective sheaf of
sets. Then~$\I$ satisfies Condition~(2) in
Proposition~\ref{prop:notions-of-internal-injectivity}, even without having to
pass to covers.

For the ``if'' direction, let~$\I$ be an internally injective object. Let~$m :
\E \to \F$ be a monomorphism in~$\Sh(X)$ and let~$k : \E \to \I$ be an arbitrary
morphism. We want to show that there exists an extension $\F \to \I$ of~$k$
along~$m$. To this end, we consider the sheaf defined by the internal expression
\[ \G \defeq \brak{\{ k' \? [\F,\I] | k' \circ m = k \}}. \]
Global sections of~$\G$ are extensions of the kind we're looking for.
Therefore it suffices to show that~$\G$ is flabby. We do this by verifying
Condition~(3) of Proposition~\ref{prop:internal-char-flabbiness} in the internal
language of~$\Sh(X)$.

Let~$K \subseteq \G$ be a subsingleton. We consider the injectivity diagram
\[ \xymatrix{
  m[\E] \cup \F' \ar@{^{(}->}[r]\ar[d] & \F \ar@{-->}[ld] \\
  \I,
} \]
where~$\F' \defeq \{ s \? \F \,|\, \text{$K$ is inhabited} \}$ and the solid
vertical arrow is defined in the following way: It should map an element~$s \in
\F'$ to~$k'(s)$, where~$k'$ is any element of~$K$; and it should map an
element~$m[u] \in m[\E]$ to~$k(u)$. These prescriptions determine a well-defined
map.

Since~$\I$ is injective from the internal point of view we're taking up here,
there exists a dotted map rendering the diagram commutative. This map is an
element of~$\G$. Furthermore, this map is an element of~$K$, if~$K$ is
inhabited.
\end{proof}

\begin{thm}\label{thm:char-injectivity-modules}
Let~$(X,\O_X)$ be a ringed topological space (or a ringed locale).
An~$\O_X$-module~$\I$ is injective if and only if it is internally injective.
\end{thm}

\begin{proof}Proposition~\ref{prop:notions-of-internal-injectivity} can be
adapted from sheaves to sets to sheaves of modules, with the same proof.
% XXX: check that.

The proof of Theorem~\ref{thm:char-injectivity} can be adopted as well.
It suffices to change~``$m[\E] \cup \F'$'' to~``$m[\E] + \F''$'', where~$\F''
\defeq \{ s \? \F \,|\, \text{$s = 0$ or $K$ is inhabited} \}$.
\end{proof}

% XXX: Remark that proof requires AxC and that it's a unique feature of sheaf toposes.
% Counterexample?


\subsection{Internal proofs of common lemmas}

\begin{lemma}A sheaf of sets or a sheaf of modules is injective if and only if it
is locally injective.\end{lemma}

\begin{proof}By Theorem~\ref{thm:char-injectivity} respectively
Theorem~\ref{thm:char-injectivity-modules}, injectivity can be characterized in the
internal language. Any such property is local.\end{proof}

\begin{lemma}Let~$X$ be a topological space (or a locale).
\begin{enumerate}
\item Let~$\I$ be an injective sheaf of sets over~$X$. Let~$\F$ be an arbitrary
sheaf of sets. Then~$\HOM(\F,\I)$ is flabby.
\item Let~$\I$ be an injective sheaf of modules over some sheaf~$\O_X$ of rings
over~$X$. Let~$\F$ be an arbitrary sheaf of modules.
Then~$\HOM_{\O_X}(\F,\I)$ is flabby.
\end{enumerate}
\end{lemma}

\begin{proof}
We cover the case of sheaf of sets first. By Theorem~\ref{thm:char-injectivity}
and Proposition~\ref{prop:internal-char-flabbiness}, it suffices to give an
intuitionistic proof of the following statement: If~$I$ is an injective set
and~$F$ is an arbitrary set, then partially defined elements of the set~$[F,I]$
of all maps~$F \to I$ can be refined to honest elements.

Thus let a subsingleton~$K \subseteq [F,I]$ be given. We consider the
injectivity diagram
\[ \xymatrix{
  F' \ar[r]\ar[d] & F \ar@{-->}[ld] \\
  I
} \]
where~$F'$ is the subset~$\{ s \? F \,|\, \text{$K$ is inhabited} \} \subseteq F$ and the
solid vertical map sends~$s \in F'$ to~$f(s)$, where~$f$ is an arbitrary element
of~$K$. This association is well-defined. Since~$I$ is injective, a dotted lift
as indicated exists. If~$K$ is inhabited, this lift is an element of~$K$.

The same kind of argument applies to the case of sheaves of modules, relying on
Theorem~\ref{thm:char-injectivity-modules} and defining~$F'$ as the
submodule~``$\{ s \? F \,|\, \text{$s = 0$ or $K$ is inhabited} \}$''.
\end{proof}

\begin{cor}Injective sheaves of sets and injective sheaves of modules are
flabby.\end{cor}

\begin{proof}Follows from the previous lemma by considering the special cases~$\F
\defeq 1$ respectively~$\F \defeq \O_X$.\end{proof}

\begin{lemma}\label{lemma:enough-flabby}
Let~$X$ be a topological space (or a locale). Any sheaf of sets
over~$X$ can be embedded into an injective (therefore flabby) sheaf of sets.
\end{lemma}

\begin{proof}By Proposition~\ref{prop:internal-char-flabbiness}, it suffices to
give an intuitionistic proof of the following statement: Any set~$F$ can be
embedded into an injective set.

As already indicated there at least two simple ways that~$F$ can be embedded
into an injective set: by embedding~$F$ in its powerset~$\P(F)$ or by
embedding~$F$ in~$\P_{\leq1}(F)$, the set of subsingletons of~$F$. For
conciseness, we only verify that~$\P_{\leq1}(F)$ is injective.

So let~$m : A \hookrightarrow B$ be an injective map and let~$k : A \to
\P_{\leq1}(F)$ be an arbitrary map. Then we can extend~$k$ to a map~$k' : B \to
\P_{\leq1}(F)$ by defining for~$y \? B$
\begin{align*}
  k'(y) &\defeq \bigcup k[m^{-1}[\{y\}]] \\
  &\phantom{\vcentcolon}= \{ s\?F \,|\, \text{$s \in k(x)$ for some~$x \in A$ such that~$m(x) = y$} \}.
  \qedhere
\end{align*}
\end{proof}

\begin{rem}The \emph{Godement construction} is a well-known way of embedding an
inhabited sheaf of sets~$\F$ into an injective sheaf, namely embedding it into
the sheaf of not necessarily continuous sections of the étale space of~$\F$:
\[ U \subseteq X \quad\longmapsto\quad
  \prod_{x \in U} \F_x. \]
The sheaf~$\P_{\leq1}(\F)$ does not coincide with this construction.
Instead by Definition~\ref{defn:interpretation-internal-constructions}, it is the sheaf with
\[ U \subseteq X \quad\longmapsto\quad
  \{ \langle V, s \rangle \,|\,
    \text{$V \subseteq U$ open, $s \in \F(V)$} \}. \]
It's not possible to describe the Godement construction in the internal language
of~$\Sh(X)$, since in the Godement construction the underlying set of~$X$
enters. But the sheaf topos of~$X$ doesn't remember this set. For instance, if~$X$
is an inhabited indiscrete topological spac, then~$\Sh(X)$ is equivalent
to~$\Set$.
\end{rem}

\begin{rem}It's not known to me whether it's possible to intuitionistically
prove that any module can be embedded into a module which satisfies the internal
flabbiness criterion of Proposition~\ref{prop:internal-char-flabbiness}. This
would give an internal proof that any sheaf of modules can be embedded into a
flabby sheaf of modules. The naive candidates don't work: The
set~$\P_{\leq1}(F)$ doesn't admit a canonical module structure, and the free
module over that set is not flabby in general.

Since by the Godement construction the statement is true in many models of
intuitionistic logic, the sheaf toposes over topological spaces, and
furthermore the proof that the Godement construction yields a flabby sheaf is
intuitionistically valid,\footnote{In order for the Godement construction to
work in a intuitionistic metatheory, one has to tweak its definition a little
bit.  Instead of mapping an open subset~$U$ to~$\prod_{x \in U} \F_x$, one has
to map~$U$ to~$\prod_{x \in U} \P_{\leq1}(\F_x)$. This has the added advantage
that it works even if~$\F$ is not inhabited.} it's not unreasonable to believe
that such an intuitionistic proof is possible.

On the other hand, it's certainly not possible to intuitionistically prove that
any module can be embedded into an injective module, since it's consistent with
Zermelo--Fraenkel set theory that no nontrivial injective abelian groups
exist~\cite{blass:inj-proj-axc}.
\end{rem}

\begin{lemma}Let~$X$ be a ringed space (or ringed locale). Let~$0 \to \E'
\xra{\alpha} \E \xra{\beta} \E'' \to 0$ be a short exact sequence
of~$\O_X$-modules. If~$\E'$ is flabby, then the induced sequence
\[ 0 \longrightarrow \Gamma(X,\E')
  \longrightarrow \Gamma(X,\E)
  \longrightarrow \Gamma(X,\E'') \longrightarrow 0 \]
is exact.
\end{lemma}

\begin{proof}Since taking global sections is left exact (being a right adjoint
functor), it suffices to verify that the map~$\Gamma(X,\E) \to \Gamma(X,\E'')$
is surjective. We'll do this by showing, in the internal language of~$\Sh(X)$,
that the sheaf of preimages of a given global section~$s \in \Gamma(X,\E'')$ is
flabby and therefore has a global section.

In the internal language, this sheaf has the description~$F \defeq \{ u \? \E
\,|\, \beta(u) = s \}$. To verify the internal condition of
Proposition~\ref{prop:internal-char-flabbiness}, let a subsingleton~$K \subseteq
F$ be given. Since~$\beta$ is surjective, there is a preimage~$u_0 \in F$.
The translated set~$K - u_0 \subseteq \E$ is still a subsingleton, and its
preimage under~$\alpha$ is as well. By the assumption on~$\E'$, there is an
element~$v \? \E$ such that~$v \in \alpha^{-1}[K - u_0]$ if~$K - u_0$ is
inhabited. We'll now verify that~$u_0 + \alpha(v) \in K$ if~$K$ is inhabited.

So assume that~$K$ is inhabited. Then~$K - u_0$ is as well. Since the image of
its unique element under~$\beta$ is zero and the given sequence is exact, the
set~$\alpha^{-1}[K - u_0]$ is inhabited as well. Therefore~$v \in \alpha^{-1}[K
- u_0]$. Thus~$u_0 + \alpha(v) \in K$.
\end{proof}

\begin{lemma}Let~$X$ be a ringed space (or ringed locale). Let~$0 \to \E'
\xra{\alpha} \E \xra{\beta} \E'' \to 0$ be a short exact sequence
of~$\O_X$-modules. If~$\E'$ and~$\E''$ are flabby, then~$\E$ is flabby as well.
\end{lemma}

\begin{proof}We verify the condition of
Proposition~\ref{prop:internal-char-flabbiness} in the internal language
of~$\Sh(X)$.

Let~$K \subseteq \E$ be a subsingleton. Then its image~$\beta[K] \subseteq \E''$
is a subsingleton as well. Since partial elements of~$\E''$ can be refined to
honest elements, there is an element~$s \? \E''$ such that~$\beta[K] \subseteq
\{ s \}$.

Since~$\beta$ is surjective, there is an element~$t_0 \? \E$ such
that~$\beta(t_0) = s$.

The preimage~$\alpha^{-1}[K - t_0] \subseteq \E'$ is a subsingleton. This
partial element can be refined to an honest element, that is there exists an
element~$u \? \E'$ such that~$\alpha^{-1}[K - t_0] \subseteq \{u\}$.

The partial element~$K$ can thereby refined to the honest element~$t \defeq t_0 + \alpha(u)$.
\end{proof}

\end{document}
