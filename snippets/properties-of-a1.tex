\documentclass[10pt]{amsart}
\usepackage[utf8]{inputenc}
\usepackage[english]{babel}
\usepackage{etex}
\usepackage{amsmath,amsthm,amssymb,stmaryrd,color,graphicx,multirow}
\usepackage{mathtools}
\usepackage{setspace}
\usepackage{bussproofs}
\usepackage{xspace}
\usepackage{longtable}
\usepackage{booktabs}
\usepackage{array}
\usepackage[protrusion=true,expansion=true]{microtype}
\usepackage[bookmarksdepth=2,pdfencoding=auto]{hyperref}
\usepackage[all]{xy}

\usepackage{tikz}
\usetikzlibrary{calc,shapes.callouts,shapes.arrows}
\newcommand{\hcancel}[5]{%
    \tikz[baseline=(tocancel.base)]{
        \node[inner sep=0pt,outer sep=0pt] (tocancel) {#1};
        \draw[red, line width=0.3mm] ($(tocancel.south west)+(#2,#3)$) -- ($(tocancel.north east)+(#4,#5)$);
    }%
}

\usepackage[natbib=true,style=numeric]{biblatex}
\usepackage[babel]{csquotes}
\bibliography{bibliography}

\theoremstyle{definition}
\newtheorem{defn}{Definition}[section]
\newtheorem{ex}[defn]{Example}

\theoremstyle{plain}

\newtheorem{prop}[defn]{Proposition}
\newtheorem{cor}[defn]{Corollary}
\newtheorem{lemma}[defn]{Lemma}
\newtheorem{thm}[defn]{Theorem}

\theoremstyle{remark}
\newtheorem{rem}[defn]{Remark}
\newtheorem{question}[defn]{Question}

\newcommand{\ZZ}{\mathbb{Z}}
\newcommand{\FF}{\mathbb{F}}
\renewcommand{\AA}{\mathbb{A}}
\newcommand{\A}{\mathcal{A}}
\renewcommand{\C}{\mathcal{C}}
\newcommand{\D}{\mathcal{D}}
\newcommand{\E}{\mathcal{E}}
\newcommand{\F}{\mathcal{F}}
\renewcommand{\G}{\mathcal{G}}
\renewcommand{\H}{\mathcal{H}}
\renewcommand{\O}{\mathcal{O}}
\newcommand{\K}{\mathcal{K}}
\newcommand{\N}{\mathcal{N}}
\newcommand{\M}{\mathcal{M}}
\renewcommand{\L}{\mathcal{L}}
\renewcommand{\P}{\mathcal{P}}
\newcommand{\R}{\mathcal{R}}
\newcommand{\T}{\mathcal{T}}
\newcommand{\I}{\mathcal{I}}
\newcommand{\J}{\mathcal{J}}
\renewcommand{\S}{\mathcal{S}}
\newcommand{\NN}{\mathbb{N}}
\newcommand{\PP}{\mathbb{P}}
\newcommand{\RR}{\mathbb{R}}
\newcommand{\QQ}{\mathbb{Q}}
\newcommand{\GG}{\mathbb{G}}
\newcommand{\aaa}{\mathfrak{a}}
\newcommand{\ppp}{\mathfrak{p}}
\newcommand{\qqq}{\mathfrak{q}}
\newcommand{\mmm}{\mathfrak{m}}
\newcommand{\nnn}{\mathfrak{n}}
\newcommand{\Hom}{\mathrm{Hom}}
\newcommand{\HOM}{\mathcal{H}\mathrm{om}}
\newcommand{\id}{\mathrm{id}}
\newcommand{\GL}{\mathrm{GL}}
\newcommand{\placeholder}{\underline{\quad}}
\newcommand{\ul}[1]{\underline{#1}}
\newcommand{\Set}{\mathrm{Set}}
\newcommand{\Grp}{\mathrm{Grp}}
\newcommand{\Vect}{\mathrm{Vect}}
\newcommand{\Sh}{\mathrm{Sh}}
\newcommand{\PSh}{\mathrm{PSh}}
\newcommand{\Zar}{\mathrm{Zar}}
\newcommand{\Et}{\mathrm{\acute{E}t}}
\newcommand{\Sch}{\mathrm{Sch}}
\newcommand{\Mod}{\mathrm{Mod}}
\newcommand{\Alg}{\mathrm{Alg}}
\newcommand{\Ring}{\mathrm{Ring}}
\newcommand{\LRL}{\mathrm{LRL}}
\newcommand{\pt}{\mathrm{pt}}
\newcommand{\tors}{\mathrm{tors}}
\DeclareMathOperator{\Spec}{Spec}
\newcommand{\QcohSpec}[2]{\mathrm{Spec}^{\mathrm{qcoh}}_{#1}{#2}}
\newcommand{\RelSpec}[2]{\mathrm{RelSpec}_{#1}{#2}}
\newcommand{\op}{\mathrm{op}}
\DeclareMathOperator{\colim}{colim}
\DeclareMathOperator{\rank}{rank}
\DeclareMathOperator{\Ann}{Ann}
\DeclareMathOperator{\Int}{int}
\DeclareMathOperator{\Clos}{cl}
\DeclareMathOperator{\Kernel}{ker}
\DeclareMathOperator{\supp}{supp}
\newcommand{\Ass}{\mathrm{Ass}}
\newcommand{\Open}{\T}
\newcommand{\?}{\,{:}\,}
\renewcommand{\_}{\mathpunct{.}\,}
\newcommand{\speak}[1]{\ulcorner\text{\textnormal{#1}}\urcorner}
\newcommand{\Ll}{\vcentcolon\Longleftrightarrow}
\newcommand{\notat}[1]{{!#1}}
\newcommand{\lra}{\longrightarrow}
\newcommand{\lhra}{\ensuremath{\lhook\joinrel\relbar\joinrel\rightarrow}}
\newcommand{\hra}{\hookrightarrow}
\newcommand{\brak}[1]{{\llbracket{#1}\rrbracket}}
\newcommand{\sdense}{{\widehat\Box}}
\newcommand{\sdenseother}{\Box}
\newcommand{\ie}{i.\,e.\@\xspace}
\newcommand{\eg}{e.\,g.\@\xspace}
\newcommand{\vs}{vs.\@\xspace}
\newcommand{\resp}{resp.\@\xspace}
\newcommand{\inv}{inv.\@}
\newcommand{\notnot}{\emph{not not}\xspace}
\newcommand{\affl}{\ensuremath{{\ul{\AA}^1_S}}\xspace}
\newcommand{\afflx}{\ensuremath{{\ul{\AA}^1_X}}\xspace}
\newcommand{\affla}{\ensuremath{{\ul{\AA}^1_{\Spec A}}}\xspace}
\newcommand{\xra}{\xrightarrow}

\newcommand{\XXX}[1]{\textbf{XXX: #1}}

\newcommand{\defeq}{\vcentcolon=}
\newcommand{\defequiv}{\vcentcolon\equiv}
\newcommand{\seq}[1]{\mathrel{\vdash\!\!\!_{#1}}}

\definecolor{gray}{rgb}{0.7,0.7,0.7}

\begin{document}

\begin{prop}The following properties of~$\affl$, as seen from the internal
point of view of~$\Zar(S)$, are consequences of~$\affl$ being (a local ring and) synthetically
quasicoherent:
% Adjust paragraph below the proof if numbering changes.
% Proof of third and seventh property references first property by number.
\begin{enumerate}
\item $\affl$ is a field in the sense that any element which is not zero is
invertible: $\forall x\?\affl\_ \neg(x = 0) \Rightarrow \speak{$x$ \inv}$. More generally,
for any number~$n \geq 0$,
\begin{multline*}
  \qquad\qquad \forall x_1,\ldots,x_n\?\affl\_
  \neg(x_1 = 0 \wedge \cdots \wedge x_n = 0) \Longrightarrow \\
  (\speak{$x_1$ \inv} \vee \cdots \vee \speak{$x_n$ \inv}). \qquad\qquad
\end{multline*}
\item $\affl$ is not a reduced ring: $\neg \forall x\?\affl\_ (\bigvee_{n \geq
0} x^n = 0) \Rightarrow x = 0.$
\item Any element of $\affl$ which is not invertible is nilpotent.
\item Any function~$\affl \to \affl$ is given by a unique polynomial
in~$\affl[T]$.
\item $\affl$ is \emph{weakly algebraically closed}, in the following sense:
Any monic polynomial~$p \in \affl[T]$ of degree at least one does \notnot have
a zero.
\item $\affl$ is infinite in the following sense: For any number~$n \geq 0$
and any given elements~$x_1,\ldots,x_n \? \affl$, there is \notnot an element~$y$
which is distinct from all of the~$x_i$.
\item $\affl$ fulfills the conclusion of the Nullstellensatz:
Let~$f_1,\ldots,f_m \in \affl[X_1,\ldots,X_n]$ be polynomials without a common
zero in~$(\affl)^n$. Then there are polynomials~$g_1,\ldots,g_m \in
\affl[X_1,\ldots,X_n]$ such that~$\sum_i g_i f_i = 1$.
\end{enumerate}\end{prop}

\begin{proof}\begin{enumerate}
\item Let $x\?\affl$ be such that~$\neg(x=0)$. We consider the quasicoherence
condition for the finitely presented~$\affl$-algebra~$A \defeq \affl/(x)$.
Since~$\Spec(A) \cong \brak{x=0} = \brak{\bot} = \emptyset$, the condition says
that the canonical homomorphism
\[ \affl/(x) \longrightarrow [\emptyset, \affl] \]
is an isomorphism. Since its codomain is the zero algebra, so is~$\affl/(x)$.
Therefore~$1 \in (x)$, that is,~$x$ is invertible.

The more general statement follows in the same way, by using the quasicoherence
condition for~$A \defeq \affl/(x_1,\ldots,x_n)$. This yields~$1 \in
(x_1,\ldots,x_n)$. Since~$\affl$ is a local ring, one of the~$x_i$ is
invertible.

\item Assume that~$\affl$ is reduced. Then the set~$\Delta \defeq \{
\varepsilon \in \affl \,|\, \varepsilon^2 = 0 \}$ is equal to~$\{ 0 \}$.
By the quasicoherence criterion applied to the finitely
presented~$\affl$-algebra~$A \defeq \affl[T]/(T^2)$, the canonical map
\[ \affl[T]/(T^2) \longrightarrow [\Spec(\affl[T]/(T^2)), \affl] \cong
  [\Delta, \affl] \cong \affl \]
is an isomorphism. It maps~$[T]$ to zero (the value of~$T$ at~$0 \in \Delta$).
Thus~$T \in (T^2)$ and therefore~$1 = 0$ in~$\affl$. This is a contradiction.

\item Let~$x \? \affl$ be not invertible. We consider the quasicoherence
condition for the finitely presented~$\affl$-algebra~$A \defeq \affl[x^{-1}]$.
Since~$\Spec(A) \cong \brak{\speak{$x$ \inv}} = \emptyset$, it follows
that~$\affl[x^{-1}] = 0$, similarly to the proof of the first claim. Thus~$x$
is nilpotent.

\item Immediate by considering the quasicoherence condition for the finitely
presented~$\affl$-algebra~$A \defeq \affl[T]$ and noticing that~$\Spec(A) \cong
\affl$.

\item Let~$p \in \affl[T]$ be a monic polynomial of degree at least one. Assume
that~$p$ doesn't have a zero in~$\affl$. Then the spectrum of~$A \defeq
\affl[T]/(p)$ is empty. The quasicoherence condition for~$A$ therefore implies
that~$\affl[T]/(p)$ is zero. This means that~$p$ is invertible in~$\affl[T]$.
A basic lemma in commutative algebra (whose standard proof is constructive)
then implies that with the exception of the constant term in~$p$, all
coefficients are nilpotent. This contradicts the assumption that~$p$ is monic
of degree at least one.

\item The polynomial~$f(T) \defeq (T - x_1) \cdots (T - x_n) + 1$
does \notnot have a zero~$y\?\affl$, since~$\affl$ is weakly algebraically
closed. This element cannot equal any~$x_i$, since~$f(x_i) = 1$ is not zero.

\item We consider the quasicoherence condition for the finitely
presented~$\affl$-algebra~$A \defeq \affl[X_1,\ldots,X_n]/(f_1,\ldots,f_m)$.
Since~$\Spec(A) \cong \{ x \in (\affl)^n \,|\, f_1(x) = \ldots = f_m(x) = 0 \}
= \emptyset$, the condition implies that~$A$ is the zero algebra just as in the
verification of the first property.\qedhere
\end{enumerate}\end{proof}

In classical logic, the first two properties would directly contradict each
other; only an intuitionistic context allows for fields which are not reduced.
Also the third statement cannot be satisfied in classical logic: for infinite
fields the existence part fails and for finite fields the uniqueness part
fails.

That~$\affl$ is not reduced, irrespective of the reducedness of the base
scheme~$S$, should not come as a surprise: Reducedness is not stable under base
change, but all statements of the internal language of~$\Zar(S)$ are.
If~$\affl$ was reduced, then all~$S$-schemes (at least those contained in the
site used to construct~$\Zar(S)$) would be reduced as well. In contrast, the
structure sheaf~$\O_S$ is reduced from the point of view of the little Zariski
topos if and only if~$S$ is reduced (Proposition~\ref{prop:reduced-ring}).

\end{document}
