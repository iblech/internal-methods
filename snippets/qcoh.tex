\documentclass[10pt,reqno]{amsart}
\usepackage[utf8]{inputenc}
\usepackage[english]{babel}
\usepackage{etex}
\usepackage{amsmath,amsthm,amssymb,stmaryrd,color,graphicx,multirow}
\usepackage{mathtools}
\usepackage{setspace}
\usepackage{bussproofs}
\usepackage{xspace}
\usepackage{longtable}
\usepackage{booktabs}
\usepackage{array}
\usepackage[protrusion=true,expansion=true]{microtype}
\usepackage[bookmarksdepth=2,pdfencoding=auto]{hyperref}
\usepackage[all]{xy}
\usepackage{soul}\setul{0.3ex}{}

\usepackage{tikz}
\usetikzlibrary{calc,shapes.callouts,shapes.arrows}
\newcommand{\hcancel}[5]{%
    \tikz[baseline=(tocancel.base)]{
        \node[inner sep=0pt,outer sep=0pt] (tocancel) {#1};
        \draw[red, line width=0.3mm] ($(tocancel.south west)+(#2,#3)$) -- ($(tocancel.north east)+(#4,#5)$);
    }%
}

\usepackage[natbib=true,style=numeric]{biblatex}
\usepackage[babel]{csquotes}
\bibliography{bibliography}

\theoremstyle{definition}
\newtheorem{defn}{Definition}[section]
\newtheorem{ex}[defn]{Example}

\theoremstyle{plain}

\newtheorem{prop}[defn]{Proposition}
\newtheorem{cor}[defn]{Corollary}
\newtheorem{lemma}[defn]{Lemma}
\newtheorem{thm}[defn]{Theorem}
\newtheorem{scholium}[defn]{Scholium}

\theoremstyle{remark}
\newtheorem{rem}[defn]{Remark}
\newtheorem{question}[defn]{Question}

\newcommand{\ZZ}{\mathbb{Z}}
\newcommand{\FF}{\mathbb{F}}
\renewcommand{\AA}{\mathbb{A}}
\newcommand{\A}{\mathcal{A}}
\renewcommand{\C}{\mathcal{C}}
\newcommand{\D}{\mathcal{D}}
\newcommand{\E}{\mathcal{E}}
\newcommand{\F}{\mathcal{F}}
\renewcommand{\G}{\mathcal{G}}
\renewcommand{\H}{\mathcal{H}}
\renewcommand{\O}{\mathcal{O}}
\newcommand{\K}{\mathcal{K}}
\newcommand{\N}{\mathcal{N}}
\newcommand{\M}{\mathcal{M}}
\renewcommand{\L}{\mathcal{L}}
\renewcommand{\P}{\mathcal{P}}
\newcommand{\R}{\mathcal{R}}
\newcommand{\T}{\mathcal{T}}
\newcommand{\I}{\mathcal{I}}
\newcommand{\J}{\mathcal{J}}
\renewcommand{\S}{\mathcal{S}}
\newcommand{\NN}{\mathbb{N}}
\newcommand{\PP}{\mathbb{P}}
\newcommand{\RR}{\mathbb{R}}
\newcommand{\QQ}{\mathbb{Q}}
\newcommand{\GG}{\mathbb{G}}
\newcommand{\aaa}{\mathfrak{a}}
\newcommand{\ppp}{\mathfrak{p}}
\newcommand{\qqq}{\mathfrak{q}}
\newcommand{\mmm}{\mathfrak{m}}
\newcommand{\nnn}{\mathfrak{n}}
\newcommand{\Hom}{\mathrm{Hom}}
\newcommand{\HOM}{\mathcal{H}\mathrm{om}}
\newcommand{\id}{\mathrm{id}}
\newcommand{\GL}{\mathrm{GL}}
\newcommand{\placeholder}{\underline{\quad}}
\let\oldul\ul
\renewcommand{\ul}[1]{\text{\oldul{$#1$}}}
\newcommand{\Set}{\mathrm{Set}}
\newcommand{\Grp}{\mathrm{Grp}}
\newcommand{\Vect}{\mathrm{Vect}}
\newcommand{\Sh}{\mathrm{Sh}}
\newcommand{\PSh}{\mathrm{PSh}}
\newcommand{\Zar}{\mathrm{Zar}}
\newcommand{\Et}{\mathrm{\acute{E}t}}
\newcommand{\Sch}{\mathrm{Sch}}
\newcommand{\Mod}{\mathrm{Mod}}
\newcommand{\Alg}{\mathrm{Alg}}
\newcommand{\Ring}{\mathrm{Ring}}
\newcommand{\LRL}{\mathrm{LRL}}
\newcommand{\pt}{\mathrm{pt}}
\newcommand{\tors}{\mathrm{tors}}
\DeclareMathOperator{\Spec}{Spec}
\newcommand{\RelSpek}{\operatorname{\ul{\mathrm{Spec}}}}
\newcommand{\QcohSpec}[2]{\mathrm{Spec}^{\mathrm{qcoh}}_{#1}{#2}}
\newcommand{\op}{\mathrm{op}}
\DeclareMathOperator{\colim}{colim}
\DeclareMathOperator{\rank}{rank}
\DeclareMathOperator{\Ann}{Ann}
\DeclareMathOperator{\Int}{int}
\DeclareMathOperator{\Clos}{cl}
\DeclareMathOperator{\Kernel}{ker}
\DeclareMathOperator{\supp}{supp}
\newcommand{\Ass}{\mathrm{Ass}}
\newcommand{\Open}{\T}
\newcommand{\?}{\,{:}\,}
\renewcommand{\_}{\mathpunct{.}\,}
\newcommand{\speak}[1]{\ulcorner\text{\textnormal{#1}}\urcorner}
\newcommand{\Ll}{\vcentcolon\Longleftrightarrow}
\newcommand{\notat}[1]{{!#1}}
\newcommand{\lra}{\longrightarrow}
\newcommand{\lhra}{\ensuremath{\lhook\joinrel\relbar\joinrel\rightarrow}}
\newcommand{\hra}{\hookrightarrow}
\newcommand{\brak}[1]{{\llbracket{#1}\rrbracket}}
\newcommand{\sdense}{{\widehat\Box}}
\newcommand{\sdenseother}{\Box}
\newcommand{\ie}{i.\,e.\@\xspace}
\newcommand{\eg}{e.\,g.\@\xspace}
\newcommand{\vs}{vs.\@\xspace}
\newcommand{\resp}{resp.\@\xspace}
\newcommand{\inv}{inv.\@}
\newcommand{\notnot}{\emph{not not}\xspace}
\newcommand{\affl}{\ensuremath{{\ul{\AA}^1_S}}\xspace}
\newcommand{\afflx}{\ensuremath{{\ul{\AA}^1_X}}\xspace}
\newcommand{\affla}{\ensuremath{{\ul{\AA}^1_{\Spec A}}}\xspace}
\newcommand{\xra}{\xrightarrow}

\newcommand{\XXX}[1]{\textbf{XXX: #1}}

\newcommand{\defeq}{\vcentcolon=}
\newcommand{\defequiv}{\vcentcolon\equiv}
\newcommand{\seq}[1]{\mathrel{\vdash\!\!\!_{#1}}}

\newcommand{\stacksproject}[1]{\cite[{\href{http://stacks.math.columbia.edu/tag/#1}{Tag~#1}}]{stacks-project}}

\definecolor{gray}{rgb}{0.7,0.7,0.7}

\begin{document}

\begin{defn}An~$R$-module~$E$ is \emph{synthetically quasicoherent} if and only if,
for any finitely presented~$R$-algebra~$A$, the canonical~$R$-algebra
homomorphism
\[ E \otimes_R A \longrightarrow [\Spec(A), E] = [[A, R]_R, E] \]
which maps a pure tensor~$x \otimes f$ to~$(\varphi \mapsto \varphi(f) x)$ is
bijective.\end{defn}

This definition has the following interpretation. The codomain of the displayed
canonical map is the set of all~$E$-valued functions on~$\Spec(A)$. Elements
of~$E \otimes_R A$ induce such functions; these induced functions can
reasonably be called ``algebraic''. In a synthetic context, there should be no
other~$E$-valued functions as these algebraic ones, and different algebraic
expressions should yield different functions. This is precisely what the
postulated bijectivity expresses.

\begin{thm}\label{thm:qcoh-big-char}
Let~$E \in \Zar(S)$ be an~$\affl$-module.
If~$E$ is quasicoherent, that is of
the form~$(\E_0)^\Zar$ for some quasicoherent~$\O_S$-module~$\E_0$,
then~$E$ is synthetically quasicoherent from the internal point of view of~$\Zar(S)$.
The converse holds in any of the following situations:
\begin{enumerate}
\item The site used to construct~$\Zar(S)$ is one of the parsimonious sites.
\item The base scheme~$S$ is concentrated (quasicompact and quasiseparated) and
the functor~$E$ maps directed limits of inverse systems of~$S$-schemes with
affine transition morphisms to colimits in~$\Set$.
\end{enumerate}
\end{thm}

\begin{proof}To verify that~$E$ is synthetically quasicoherent, we have to
verify a condition for~$\affl$-algebras~$A$ in any
slice~$\Zar(S)/\ul{T}$. If such an algebra is finitely presented from
the internal point of view, then there is a covering~$T = \bigcup_i T_i$ such
that the each of the restrictions of the algebra to the~$T_i$ is of the
form~$(\A_0)^\Zar$ for some finitely presented~$\O_{T_i}$-algebra~$\A_0$.
Without loss of generality, we will just assume that~$A$ itself is of the
form~$(\A_0)^\Zar$ for a finitely presented~$\O_S$-algebra~$\A_0$.

By Proposition~\ref{prop:relative-spectrum-big-zariski}, the internal expression~$\Spec(A)$ is the functor of
points of~$\RelSpek_S \A_0$. For any~$S$-scheme~$f : T \to S$ contained in the site
used to define~$\Zar(S)$, we consider the fiber product
\[ \xymatrix{
  \RelSpek_T(f^*\A_0) \ar[r]^{f'} \ar[d]_{p'} & \RelSpek_S\A_0 \ar[d]^p \\
  T \ar[r]_f & S.
} \]
Since~$\RelSpek_T(f^*\A_0) \to S$ is contained in the site (for any of our
admissible sites), we may conclude using the following chain of isomorphisms:
\begin{align*}
  [\Spec(A), E](T) &\cong
  \Hom_{\Zar(S)}(\ul{T}, [\Spec(A), E])
  \cong \Hom_{\Zar(S)}(\ul{T} \times \Spec(A), E) \\
  &\cong \Hom_{\Zar(S)}(\underline{T \times_S \RelSpek_S\A_0}, E)
  \cong E(\RelSpek_T(f^*\A_0)) \\
  &\cong \Gamma(\RelSpek_T(f^*\A_0), (p')^* f^* \E_0)
  \cong \Gamma(T, (p')_* (p')^* f_* \E_0) \\
  &\cong \Gamma(T, f^*\E_0 \otimes_{\O_T} f^*\A_0)
  \cong \Gamma(T, (\E_0 \otimes_{\O_S} \A_0)^\Zar) \\
  &\cong \Gamma(T, (\E_0)^\Zar \otimes_\affl (\A_0)^\Zar)
  \cong \Gamma(T, E \otimes_\affl A).
\end{align*}
The antepenultimate isomorphism is because pullback of modules in~$\Sh(S)$ to
modules in~$\Sh(T)$ commutes with tensor product. The penultimate isomorphism
is because pullback of a sheaf in~$\Sh(S)$ to a sheaf in~$\Zar(S)$ commutes
with tensor product (Lemma~\ref{lemma:zar-tensor-product-commutes}).

For the converse direction, we first verify that the restrictions~$E|_{\Sh(T)}$
to the little Zariski topos of each~$S$-scheme~$T$ contained in the site used
to define~$\Zar(S)$ are quasicoherent~$\O_T$-modules. For this, we employ the
quasicoherence criterion of Theorem~\ref{thm:qcoh-sheafchar}: For any open
affine subset~$T' \subseteq T$ and any function~$h \in \Gamma(T', \O_T)$ we
verify that the canonical morphism
\begin{equation}\label{eqn:want-iso}\tag{$\dagger$}
  E|_{\Sh(T)}[h^{-1}] \longrightarrow j_*(E|_{\Sh(D(h))})
\end{equation}
is an isomorphism, where~$j : D(h) \hookrightarrow T'$ denotes the inclusion.
This follows from the assumption of synthetic quasicoherence by considering
the~$\affl$-module~$A \defeq \affl[h^{-1}]$ (in the slice~$\Zar(S)/\ul{T'}$):
This expresses that the canonical morphism
\begin{equation}\label{eqn:have-iso}\tag{$\ddagger$}
  E \otimes_\affl \affl[h^{-1}] \longrightarrow [\Spec(A), E]
\end{equation}
is an isomorphism (of~$\affl$-modules in~$\Zar(S)/\ul{T'}$). Restricting the
domain to~$\Sh(T')$ yields the sheaf~$E|_{\Sh(T')} \otimes_{\O_{T'}}
\O_{T'}[h^{-1}]$, since restricting commutes with the geometric constructions
``forming the tensor product'' and ``localizing away from~$h$''.
Since~$\Spec(A)$ is the functor of points of~$D(h)$, restricting
the codomain to~$\Sh(T')$ yields the sheaf~$j_*(E|_{\Sh(D(h))})$.
The canonical morphism~\eqref{eqn:want-iso} which we want to recognize as an
isomorphism is therefore the restriction of the canonical
morphism~\eqref{eqn:have-iso} which we know to be an isomorphism.

A natural candidate for an quasicoherent~$\O_S$-module~$\E_0$ with~$E \cong
(\E_0)^\Zar$ is~$\E_0 \defeq E|_{\Sh(S)}$. We'll show that this is indeed true.
Let~$f : T \to S$ be any~$S$-scheme contained in the site used to
define~$\Zar(S)$. We assume for the time being that~$f$ is of finite
presentation and affine, so~$T \cong \RelSpek_S \A_0$ for some finitely
presented~$\O_S$-algebra~$\A_0$. We want to verify that the canonical morphism
\begin{equation}\label{eqn:want-iso2}\tag{§}
  f^*(E|_{\Sh(S)}) \longrightarrow E|_{\Sh(T)}
\end{equation}
is an isomorphism. Since the functor~$f_*$ from quasicoherent~$\O_T$-modules to
quasicoherent~$\O_S$-modules is fully faithful (the morphism~$f$ being affine)
and domain and codomain of that morphism are quasicoherent, it suffices to
verify that its image under~$f_*$ is an isomorphism. This image is the
canonical morphism
\[ E|_{\Sh(S)} \otimes_{\O_S} \A_0 \longrightarrow f_*(E|_{\Sh(T)}). \]
The assumption of synthetic quasicoherence, applied to the~$\affl$-algebra~$A
\defeq (\A_0)^\Zar$, shows that this morphism is an isomorphism.

In situation~(1), the only step left to do is to generalize the argument is the
previous paragraph to morphisms~$f : T \to S$ which are locally of finite
presentation. This works out because there are open covers of~$S$ and~$T$ such
that the appropriate restrictions of~$f$ are of finite presentation and affine.
The assumption of synthetic quasicoherence then needs to be applied to
to~$\affl$-algebras in suitable slices of~$\Zar(S)$, showing that the canonical
morphism~\eqref{eqn:want-iso2} is locally an isomorphism and therefore globally
as well.

In situation~(2), we employ the technique of approximating general~$S$-schemes
by~$S$-schemes of finite presentation. Specifically, let~$f : T \to S$ be an
arbitrary~$S$-scheme contained in the site used to define~$\Zar(S)$. Without
loss of generality, we may assume that~$T$ is an affine scheme. Thus~$T$ is
quasicompact and quasiseparated, and~$S$ is quasiseparated by assumption. We may therefore
apply the lemma of relative approximation~\stacksproject{09MV} to deduce
that~$T$ is a directed limit of an inverse system of~$S$-schemes~$f_i : T_i \to
S$ of finite presentation with affine transition maps. These~$S$-schemes
are contained in the site used to define~$\Zar(S)$. Furthermore, they inherit
quasicompactness and quasiseparatedness from~$S$. Therefore we can apply a
comparison result on the categories of quasicoherent
modules~\stacksproject{01Z0}:
\[
  E(T) = E(\lim_i T_i) \cong \colim_i E(T_i) \cong
  \colim_i \Gamma(T_i, f_i^* \E_0)
  \cong \Gamma(T, f^* \E_0). \qedhere
\]
\end{proof}

\begin{scholium}\label{scholium}
Let~$E \in \Zar(S)$ be a quasicoherent~$\affl$-module.
Let~$A \in \Zar(S)$ be a quasicoherent~$\affl$-algebra such that~$\Spec(A) \in
\Zar(S)$ is representable by an object of the site used to define~$\Zar(S)$.
Then the canonical morphism
\[ E \otimes_\affl A \longrightarrow [\Spec(A), E] \]
is an isomorphism.
\end{scholium}

\begin{proof}The second paragraph of the proof of
Theorem~\ref{thm:qcoh-big-char} applies.\end{proof}

\begin{rem}\label{rem:local-representability}
The condition in Scholium~\ref{scholium} that~$\Spec(A)$ is
representable by an object of the site used to define~$\Zar(S)$ is slightly
unnatural from a topos-theoretic point of view, since the conclusion of the
Scholium depends only on the topos over the site and not the site itself.
In fact, the condition can be weakened and made more natural at the
same time: It suffices to require that~$\Spec(A)$ is \emph{locally}
representable by an object of the site.

However, the condition can't be dropped completely. For instance, if we employ the
parsimonious sites and consider~$S = \Spec \ZZ$,~$E = \affl$, and $A =
\K_S^\Zar$ (where~$\K_S$ is the sheaf of rational functions on~$S$, which in
this case is the constant sheaf~$\ul{\QQ}$), then~$\Spec(A)$ is the functor of
points of the~$\ZZ$-scheme~$\Spec(\QQ)$. This functor coincides with the
functor of points of the empty~$\ZZ$-scheme on the parsimonious sites;
therefore~$\Spec(A) = \emptyset$ from the internal point of view. Thus the
codomain of the canonical morphism is the zero algebra, but the domain is not.
\end{rem}

\end{document}
