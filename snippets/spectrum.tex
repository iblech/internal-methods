\documentclass[10pt,reqno]{amsart}
\usepackage[utf8]{inputenc}
\usepackage[english]{babel}
\usepackage{etex}
\usepackage{amsmath,amsthm,amssymb,stmaryrd,color,graphicx,multirow}
\usepackage{mathtools}
\usepackage{setspace}
\usepackage{bussproofs}
\usepackage{xspace}
\usepackage{longtable}
\usepackage{booktabs}
\usepackage{array}
\usepackage[protrusion=true,expansion=true]{microtype}
\usepackage[bookmarksdepth=2,pdfencoding=auto]{hyperref}
\usepackage[all]{xy}
\usepackage{soul}\setul{0.3ex}{}

\usepackage{tikz}
\usetikzlibrary{calc,shapes.callouts,shapes.arrows}
\newcommand{\hcancel}[5]{%
    \tikz[baseline=(tocancel.base)]{
        \node[inner sep=0pt,outer sep=0pt] (tocancel) {#1};
        \draw[red, line width=0.3mm] ($(tocancel.south west)+(#2,#3)$) -- ($(tocancel.north east)+(#4,#5)$);
    }%
}

\usepackage[natbib=true,style=numeric]{biblatex}
\usepackage[babel]{csquotes}
\bibliography{bibliography}

\theoremstyle{definition}
\newtheorem{defn}{Definition}[section]
\newtheorem{ex}[defn]{Example}

\theoremstyle{plain}

\newtheorem{prop}[defn]{Proposition}
\newtheorem{cor}[defn]{Corollary}
\newtheorem{lemma}[defn]{Lemma}
\newtheorem{thm}[defn]{Theorem}
\newtheorem{scholium}[defn]{Scholium}

\theoremstyle{remark}
\newtheorem{rem}[defn]{Remark}
\newtheorem{question}[defn]{Question}

\newcommand{\ZZ}{\mathbb{Z}}
\newcommand{\FF}{\mathbb{F}}
\renewcommand{\AA}{\mathbb{A}}
\newcommand{\A}{\mathcal{A}}
\renewcommand{\C}{\mathcal{C}}
\newcommand{\D}{\mathcal{D}}
\newcommand{\E}{\mathcal{E}}
\newcommand{\F}{\mathcal{F}}
\renewcommand{\G}{\mathcal{G}}
\renewcommand{\H}{\mathcal{H}}
\renewcommand{\O}{\mathcal{O}}
\newcommand{\K}{\mathcal{K}}
\newcommand{\N}{\mathcal{N}}
\newcommand{\M}{\mathcal{M}}
\renewcommand{\L}{\mathcal{L}}
\renewcommand{\P}{\mathcal{P}}
\newcommand{\R}{\mathcal{R}}
\newcommand{\T}{\mathcal{T}}
\newcommand{\I}{\mathcal{I}}
\newcommand{\J}{\mathcal{J}}
\renewcommand{\S}{\mathcal{S}}
\newcommand{\NN}{\mathbb{N}}
\newcommand{\PP}{\mathbb{P}}
\newcommand{\RR}{\mathbb{R}}
\newcommand{\QQ}{\mathbb{Q}}
\newcommand{\GG}{\mathbb{G}}
\newcommand{\aaa}{\mathfrak{a}}
\newcommand{\ppp}{\mathfrak{p}}
\newcommand{\qqq}{\mathfrak{q}}
\newcommand{\mmm}{\mathfrak{m}}
\newcommand{\nnn}{\mathfrak{n}}
\newcommand{\Hom}{\mathrm{Hom}}
\newcommand{\HOM}{\mathcal{H}\mathrm{om}}
\newcommand{\id}{\mathrm{id}}
\newcommand{\GL}{\mathrm{GL}}
\newcommand{\placeholder}{\underline{\quad}}
\let\oldul\ul
\renewcommand{\ul}[1]{\text{\oldul{$#1$}}}
\newcommand{\Set}{\mathrm{Set}}
\newcommand{\Grp}{\mathrm{Grp}}
\newcommand{\Vect}{\mathrm{Vect}}
\newcommand{\Sh}{\mathrm{Sh}}
\newcommand{\PSh}{\mathrm{PSh}}
\newcommand{\Zar}{\mathrm{Zar}}
\newcommand{\Et}{\mathrm{\acute{E}t}}
\newcommand{\Sch}{\mathrm{Sch}}
\newcommand{\Mod}{\mathrm{Mod}}
\newcommand{\Alg}{\mathrm{Alg}}
\newcommand{\Ring}{\mathrm{Ring}}
\newcommand{\LRL}{\mathrm{LRL}}
\newcommand{\pt}{\mathrm{pt}}
\newcommand{\tors}{\mathrm{tors}}
\DeclareMathOperator{\Spec}{Spec}
\newcommand{\RelSpek}{\operatorname{\ul{\mathrm{Spec}}}}
\newcommand{\QcohSpec}[2]{\mathrm{Spec}^{\mathrm{qcoh}}_{#1}{#2}}
\newcommand{\op}{\mathrm{op}}
\DeclareMathOperator{\colim}{colim}
\DeclareMathOperator{\rank}{rank}
\DeclareMathOperator{\Ann}{Ann}
\DeclareMathOperator{\Int}{int}
\DeclareMathOperator{\Clos}{cl}
\DeclareMathOperator{\Kernel}{ker}
\DeclareMathOperator{\supp}{supp}
\newcommand{\Ass}{\mathrm{Ass}}
\newcommand{\Open}{\T}
\newcommand{\?}{\,{:}\,}
\renewcommand{\_}{\mathpunct{.}\,}
\newcommand{\speak}[1]{\ulcorner\text{\textnormal{#1}}\urcorner}
\newcommand{\Ll}{\vcentcolon\Longleftrightarrow}
\newcommand{\notat}[1]{{!#1}}
\newcommand{\lra}{\longrightarrow}
\newcommand{\lhra}{\ensuremath{\lhook\joinrel\relbar\joinrel\rightarrow}}
\newcommand{\hra}{\hookrightarrow}
\newcommand{\brak}[1]{{\llbracket{#1}\rrbracket}}
\newcommand{\sdense}{{\widehat\Box}}
\newcommand{\sdenseother}{\Box}
\newcommand{\ie}{i.\,e.\@\xspace}
\newcommand{\eg}{e.\,g.\@\xspace}
\newcommand{\vs}{vs.\@\xspace}
\newcommand{\resp}{resp.\@\xspace}
\newcommand{\inv}{inv.\@}
\newcommand{\notnot}{\emph{not not}\xspace}
\newcommand{\affl}{\ensuremath{{\ul{\AA}^1_S}}\xspace}
\newcommand{\afflx}{\ensuremath{{\ul{\AA}^1_X}}\xspace}
\newcommand{\afflt}{\ensuremath{{\ul{\AA}^1_T}}\xspace}
\newcommand{\affla}{\ensuremath{{\ul{\AA}^1_{\Spec A}}}\xspace}
\newcommand{\xra}{\xrightarrow}

\newcommand{\XXX}[1]{\textbf{XXX: #1}}

\newcommand{\defeq}{\vcentcolon=}
\newcommand{\defequiv}{\vcentcolon\equiv}
\newcommand{\seq}[1]{\mathrel{\vdash\!\!\!_{#1}}}

\newcommand{\stacksproject}[1]{\cite[{\href{http://stacks.math.columbia.edu/tag/#1}{Tag~#1}}]{stacks-project}}

\definecolor{gray}{rgb}{0.7,0.7,0.7}

\begin{document}

\begin{defn}The \emph{synthetic spectrum} of an~$R$-algebra~$A$ is
\[ \Spec(A) \defeq [A, R]_{\mathrm{Alg}(R)}, \]
the set of~$R$-algebra homomorphisms from~$A$ to~$R$.\end{defn}

\begin{ex}The synthetic spectrum of~$R$ is the one-element set.
More generally, the synthetic spectrum of the algebra~$R[X_1,\ldots,X_n]/(f_1,\ldots,f_m)$
is the solution set~$\{ x \? R^n \,|\, f_1(x) = \cdots = f_n(x) = 0 \}$.
\end{ex}

\begin{ex}The synthetic spectrum of~$R/(f)$ is~$\brak{f = 0}$, the truth value
of the formula~``$f = 0$'', the subsingleton set~$\{ \star \,|\, f = 0 \}$.
If classical logic is available, then this set contains~$\star$ or is empty,
depending on whether~$f$ is zero or not. Similarly, the synthetic spectrum
of~$R[f^{-1}]$ is~$\brak{\speak{$f$ \inv}}$.\end{ex}

\begin{prop}\label{prop:relative-spectrum-big-zariski}
Let~$\A_0$ be an~$\O_S$-algebra (not necessarily quasicoherent).
Then the synthetic spectrum of the~$\affl$-algebra~$(\A_0)^\Zar$, as constructed
in the internal language of~$\Zar(S)$, is the functor of points of~$\RelSpek_S \A_0$.
\end{prop}

\begin{proof}The Hom set occuring in the definition of the synthetic spectrum is
interpreted by the internal Hom when using the internal language. For
any~$S$-scheme~$f : T \to S$ contained in the site used to define~$\Zar(S)$, we
have the following chain of isomorphisms.
\begin{align*}
  (\Spec(A))(T) &= [(\A_0)^\Zar, \affl]_{\mathrm{Alg}(\affl)}(T) \\
  &\cong
  \Hom_{\Zar(S)}(\ul{T}, [(\A_0)^\Zar, \affl]_{\mathrm{Alg}(\affl)}) \\
  &\cong
  \Hom_{\Zar(S)}(\ul{T} \times (\A_0)^\Zar, \affl)_{\ldots} \\
  &\cong
  \Hom_{\Zar(S)/\ul{T}}(\ul{T} \times (\A_0)^\Zar, \ul{T} \times \affl)_{\ldots} \\
  &\cong
  \Hom_{\mathrm{Alg}_{\Zar(T)}(\afflt)}((f^*\A_0)^\Zar, \afflt) \\
  &=
  \Hom_{\mathrm{Alg}_{\Zar(T)}(\afflt)}(\pi^{-1}(f^*\A_0) \otimes_{\pi^{-1}\O_T}
  \afflt, \afflt) \\
  &\cong
  \Hom_{\mathrm{Alg}_{\Zar(T)}(\pi^{-1}\O_T)}(\pi^{-1}(f^*\A_0), \afflt) \\
  &\cong
  \Hom_{\mathrm{Alg}_{\Sh(T)}(\O_T)}(f^*\A_0, \O_T) \\
  &\cong
  \Hom_{\mathrm{Alg}_{\Sh(S)}(\O_S)}(\A_0, f_*\O_T) \\
  &\cong
  \Hom_S(T, \RelSpek_S \A_0).
\end{align*}
The omitted subscripts~``$\ldots$'' should denote that we're only taking the
subset of the Hom set where, for each fixed first argument, the morphisms are
morphisms of~$\affl$-algebras.
\end{proof}

\end{document}
